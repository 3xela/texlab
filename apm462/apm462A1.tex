\documentclass[12pt, a4paper]{article}
\usepackage[lmargin =0.5 in, 
rmargin=0.5in, 
tmargin=1in,
bmargin=0.5in]{geometry}
\geometry{letterpaper}
\usepackage{tikz-cd}
\usepackage{amsmath}
\usepackage{amssymb}
\usepackage{blindtext}
\usepackage{titlesec}
\usepackage{enumitem}
\usepackage{fancyhdr}
\usepackage{amsthm}
\usepackage{graphicx}
\usepackage{cool}
\usepackage{thmtools}
\usepackage{hyperref}
\graphicspath{ }					%path to an image

%-------- sexy font ------------%
%\usepackage{libertine}
%\usepackage{libertinust1math}

%\usepackage{mlmodern}				% very nice and classic
%\usepackage[utopia]{mathdesign}
%\usepackage[T1]{fontenc}


\usepackage{mlmodern}
\usepackage{eulervm}
%\usepackage{tgtermes} 				%times new roman
%-------- sexy font ------------%


% Problem Styles
%====================================================================%


\newtheorem{problem}{Problem}


\theoremstyle{definition}
\newtheorem{thm}{Theorem}
\newtheorem{lemma}{Lemma}
\newtheorem{prop}{Proposition}
\newtheorem{cor}{Corollary}
\newtheorem{fact}{Fact}
\newtheorem{defn}{Definition}
\newtheorem{example}{Example}
\newtheorem{question}{Question}

\newtheorem{manualprobleminner}{Problem}

\newenvironment{manualproblem}[1]{%
	\renewcommand\themanualprobleminner{#1}%
	\manualprobleminner
}{\endmanualprobleminner}

\newcommand{\penum}{ \begin{enumerate}[label=\bf(\alph*), leftmargin=0pt]}
	\newcommand{\epenum}{ \end{enumerate} }

% Math fonts shortcuts
%====================================================================%

\newcommand{\ring}{\mathcal{R}}
\newcommand{\N}{\mathbb{N}}                           % Natural numbers
\newcommand{\Z}{\mathbb{Z}}                           % Integers
\newcommand{\R}{\mathbb{R}}                           % Real numbers
\newcommand{\C}{\mathbb{C}}                           % Complex numbers
\newcommand{\F}{\mathbb{F}}                           % Arbitrary field
\newcommand{\Q}{\mathbb{Q}}                           % Arbitrary field
\newcommand{\PP}{\mathcal{P}}                         % Partition
\newcommand{\M}{\mathcal{M}}                         % Mathcal M
\newcommand{\eL}{\mathcal{L}}                         % Mathcal L
\newcommand{\T}{\mathbb{T}}                         % Mathcal T
\newcommand{\U}{\mathcal{U}}                         % Mathcal U\\
\newcommand{\V}{\mathcal{V}}                         % Mathcal V

% symbol shortcuts
%====================================================================%

\newcommand{\bd}{\partial}
\newcommand{\grad}{\nabla}
\newcommand{\lam}{\lambda}
\newcommand{\imp}{\implies}
\newcommand{\all}{\forall}
\newcommand{\exs}{\exists}
\newcommand{\delt}{\delta}
\newcommand{\ep}{\varepsilon}
\newcommand{\ra}{\rightarrow}
\newcommand{\vph}{\varphi}

\newcommand{\ol}{\overline}
\newcommand{\f}{\frac}
\newcommand{\lf}{\lfrac}
\newcommand{\df}{\dfrac}

% bracketting shortcuts
%====================================================================%
\newcommand{\abs}[1]{\left| #1 \right|}
\newcommand{\babs}[1]{\Big|#1\Big|}
\newcommand{\bound}{\Big|}
\newcommand{\BB}[1]{\left(#1\right)}
\newcommand{\dd}{\mathrm{d}}
\newcommand{\artanh}{\mathrm{artanh}}
\newcommand{\Med}{\mathrm{Med}}
\newcommand{\Cov}{\mathrm{Cov}}
\newcommand{\Corr}{\mathrm{Corr}}
\newcommand{\tr}{\mathrm{tr}}
\newcommand{\Range}[1]{\mathrm{range}(#1)}
\newcommand{\Null}[1]{\mathrm{null}(#1)}
\newcommand{\lan}{\langle}
\newcommand{\ran}{\rangle}
\newcommand{\norm}[1]{\left\lVert#1\right\rVert}
\newcommand{\inn}[1]{\lan#1\ran}
\newcommand{\op}[1]{\operatorname{#1}}
\newcommand{\bmat}[1]{\begin{bmatrix}#1\end{bmatrix}}
\newcommand{\pmat}[1]{\begin{pmatrix}#1\end{pmatrix}}
\newcommand{\vmat}[1]{\begin{vmatrix}#1\end{vmatrix}}

\newcommand{\amogus}{{\bigcap}\kern-0.8em\raisebox{0.3ex}{$\subset$}}
\newcommand{\Note}{\textbf{Note: }}
\newcommand{\Aside}{{\bf Aside: }}
%restriction
%\newcommand{\op}[1]{\operatorname{#1}}
%\newcommand{\done}{$$\mathcal{QED}$$}

%====================================================================%


\setlength{\parindent}{0pt}      	% No paragraph indentations
\pagestyle{fancy}
\fancyhf{}							% fancy header

\setcounter{secnumdepth}{0}			% sections are numbered but numbers do not appear
\setcounter{tocdepth}{2} 			% no subsubsections in toc

%template
%====================================================================%
%\begin{manualproblem}{1}
%Spivak.
%\end{manualproblem}

%\begin{proof}[Solution]
%\end{proof}

%----------- or -----------%

%\begin{problem} 		
%\end{problem}	

%\penum
%	\item
%\epenum
%====================================================================%


\newcommand{\Course}{APM462}
\newcommand{\hwNumber}{1}

%preamble

\title{}
\author{A.N.}
\date{\today}
\lhead{\Course A\hwNumber}
\rhead{\thepage}
%\cfoot{\thepage}


%====================================================================%
\begin{document}



\begin{problem}
\end{problem}
Suppose that $f$ continuous at $a$ and $g$ continuous at $f(a)$. Recall that for every $\ep_f$ there is a $\delta_f$ so that 
$\norm{x-a}< \delta_f$ implies that $\norm{f(a) - f(x)}<\epsilon_f$. Similarly for any $\ep_g>0$ we have a $\delta_g>0$ so that $\norm{f(a) - y} <\delta_g$ implies that $\norm{g(f(a)) - g(y)} < \ep_g$. We wish to show the same is true for $g\circ f$. Given $\ep >0$, we choose $\delta$ in the following way. Select $\delta_g $ for $y = f(x)$ from continuity of $g$. Then take $\ep_f = \delta_g$ and choose $\delta = \delta_f$ satisfying continuity of $f$ at $a$. This indeed works since: 
$$\norm{a - x} < \delta = \delta_f \implies \norm{f(a) - f(x)}< \ep_f = \delta_g \implies  \norm{g(f(a)) - g(f(x))} < \ep.$$\
Thus the composition of two continuous functions is continuous.   
\newpage
\begin{problem}
\end{problem}
Since $K_1,K_2$ are compact they are both closed and bounded. There exists constants $M_1,M_2$ so that $|x|^2\leq M_1$ for $x\in K_1$, and $|y|^2\leq M_2$. For any $(x,y)\in K_1\times K_2$ we have that:
$$|(x,y)| = \sqrt{|x|^2 + |y|^2} \leq \sqrt{M_1 + M_2}.$$
Therefore $K_1 \times K_2$ is bounded. It is closed since $K_1,K_2$ are both closed. Therefore $K_1\times K_2$ is closed and bounded and hence compact. 
\newpage
\begin{problem}
\end{problem}
Let $g: \R \to \R^n$ and $f: \R^n \to \R$ be $C^1$. For simplicity, take $n=2$. We write $g(t) = (x(t), y(t))$.  We wish to compute the derivative of $\phi = f \circ g$. 
First note that $$\phi(t+h) - \phi(t) = \left[f(x(t+h) , y(t+h)) - f(x(t+h) , y(t)) \right] + \left[f(x(t+h) , y(t)) - f(x(t) , y(t))	\right].$$
We define $g_1(k) = f(x(t+h) , y(t+k))$ and $g_2 (h) = f(x(t+h) , y(t))$. By the mean value theorem on $\R$, there exists a $\theta_1, \theta_2$ so that:
$$f(x(t+h) , y(t+h)) - f(x(t+h) , y(t)) = h \frac{\partial f}{\partial y} \cdot \frac{\partial y}{\partial t}(t+\theta_1 h),$$
and 
$$f(x(t+h) , y(t)) - f(x(t) , y(t)) = h \frac{\partial f}{\partial x}\cdot \frac{\partial x}{\partial t}(t + \theta_2 h).$$
We now compute the derivative of $\phi$ as:
\begin{align*}
	\phi^\prime(t)= \lim_{h \to 0}\frac{\phi(t+h) - \phi(t)}{h} & = \lim_{h \to 0} \frac{h \frac{\partial f}{\partial y} \cdot \frac{\partial y}{\partial t}(t+\theta_1 h)+h \frac{\partial f}{\partial x}\cdot \frac{\partial x}{\partial t}(t + \theta_2 h)}{h} \tag{using above}
	\\ & = \lim_{h \to 0} \frac{\partial f}{\partial y} \cdot \frac{\partial y}{\partial t}(t+\theta_1 h) + \frac{\partial f}{\partial x}\cdot \frac{\partial x}{\partial t}(t + \theta_2 h)
	\\ & = \frac{\partial f}{\partial y} \cdot \frac{\partial y}{\partial t}(t) + \frac{\partial f}{\partial x}\cdot \frac{\partial x}{\partial t}(t ) \tag{by Continuity of the partials}
\end{align*}
\newpage
\begin{problem}
\end{problem}
\penum
\item We show that $\grad f(x) = A^T b$ satisfies the limit definition of the gradient. 
\begin{align*}
	\lim_{h \to 0} \frac{f(x+h) - f(x)- A^Tb \cdot h}{h} & = \lim_{h \to 0} \frac{b \cdot A(x+h) - b \cdot Ax - A^Tb \cdot h}{h}
	\\ & = \lim_{h \to 0} \frac{b \cdot Ax + b \cdot Ah - b \cdot Ax - b \cdot Ah}{h} \tag{since $A^Tb = b \cdot A$}
	\\ & = 0
\end{align*}
By uniqueness of the gradient we have that $\grad f(x) = b \cdot A^T$. 
\item We show that $\grad f (x) = (A+A^T)x$ satisfies the limit definition of the gradient. 
\begin{align*}
	\lim_{h \to 0} \frac{f(x+h) - f(x)- (A+ A^T)x\cdot h}{h}
	  &= \lim_{h \to 0}\frac{x \cdot A x + h \cdot Ax + x \cdot Ah + h \cdot A h - x \cdot Ax -(A+ A^T)x\cdot h }{h}
	  \\ & = \lim_{h \to 0} \frac{h \cdot A x + x \cdot Ah - (A+A^T)x \cdot h}{h}
	  \\ & = \lim_{h \to 0 } \frac{h \cdot A^T x + h \cdot Ax +  (A+A^T)x\cdot h}{h}
	  \\ & = \lim_{h\to 0} \frac{h \cdot(A+A^T)x - h \cdot (A+A^T)x}{h}
	  \\ & = 0
\end{align*}
\epenum 
\newpage
\begin{problem}
\end{problem}
We parametrize the right half of the cone as $(t,t), t\geq 0$. The distance from the right half of the cone, to the point $p = (3,0)$ is thus
$$d(t) = (t-3)^2 + t^2.$$
Necessary first order conditions give us that $$d^\prime(t) v = (4t - 6)v \geq 0$$ for $v$ being a feasible direction of $\R_{t\geq 0}$. 
If $t>0$ then we imply have that $4t-6 = 0$ so $t =\frac{3}{2}$. At the boundary i.e. when $t=0$ we have that $(4-6)v \geq 0$. Since $v$ is a feasible direction of $0$ in the half line we must have that $v>0$. 
Therefore this equality cannot be satisfied. Thus our candidate point for the right half is $t = \frac{3}{2}$. On the left half, we parametrize with $(-t,t)$ for $t\geq 0$.  Our distance function is given as 
$$d(t) = (3+t)^2 + t^2.$$
First order conditions give us that 
$$d^\prime(t)v = (4t+6)v \geq 0.$$
On the interior we have equality with $0$ so $4t+6 = 0 $ so $t = -\frac{3}{2}$. At 0 we have the inequality $6v \geq 0$ for all $v<0$. Therefore $t=0$ is a candidate point. Our 2 candidate points are $(\frac{3}{2}, \frac{3}{2}) , (-\frac{3}{2} , \frac{3}{2})$. Notice that $d^{\prime \prime}(t) =4$ in both cases, so both the necessary and sufficient conditions are satisfied for a local minimum.
Therefore one of them must be at the minimum distance. We can see that it must be $(\frac{3}{2}, \frac{3}{2})$ since $\norm{(\frac{3}{2}, \frac{3}{2} )- (3,0))} = \frac{3\sqrt{2}}{2}$, and $\norm{(-\frac{3}{2}, \frac{3}{2}) - (3,0)} = \frac{3\sqrt{10}}{2}$. 
\newpage
\begin{problem}
\end{problem}
\penum 
\item Since we are optimizing on an open set, we have that the necessary condition must be $\grad f_\alpha(x,y) = 0$. We compute the gradient as: 
$$0 = \grad f_\alpha (x,y) = \bmat{2\alpha^2 x + 5y \\ 4(\alpha+1) y +5x -3}.$$
This will be solved for $x= \frac{-15}{8(\alpha+1)\alpha^2 -  25}, y= \frac{6\alpha^2}{8(\alpha+1)\alpha^2 - 25}$
\item We compute the hessian of $f$ as 
$$H(f) = \bmat{2\alpha^2 & 5 \\ 5 & 4(\alpha +1)}.$$
The necessary condition for local minima is $H(f_\alpha)\geq 0$. Sylvestors criterion tells us this happens exactly when $4(\alpha+1)\geq 0, 8\alpha^2(\alpha+1) - 25 \geq0$. This is satisfied when $\alpha \geq 1.194$. 
\item The matrix $H(f)$ will be positive definite when $\alpha> 1.194$. 
\item We require that $\alpha > 1.194$. As above set $Q = Hess(f_\alpha)$. We have that $\grad f_\alpha(x)  = Qx - b$, where $ b = (0,3)$. First note that we can write $$f(x) = \frac{1}{2} (x- x^\ast) \cdot Q (x-x^\ast) - \frac{1}{2} x^\ast \cdot Q x^\ast$$
for $x^\ast = Q^{-1}b$. Note that $x^\ast =( \frac{-15}{8(\alpha+1)\alpha^2 -  25},  \frac{6\alpha^2}{8(\alpha+1)\alpha^2 - 25} )$ which agrees with the computation done in $a$. We claim that the global minimum is attained at $x^\ast$. Since $Q$ is positive definite, we have that $\frac{1}{2} (x- x^\ast) \cdot Q (x-x^\ast) \geq 0$ for all $x$ with equality being attained exactly when $x = x^\ast$. Thus $$f(x) \geq - \frac{1}{2} x^\ast Q x^\ast = f(x_0). $$ So $x_0$ is the global minimum.
\epenum
\newpage
\begin{problem}
\end{problem}
We compute the first order necessary condition: 
$$0= \grad f(x) = \bmat{4(x-3) \\ 2(y-5)}.$$
This is satisfies at $x = 3,y=5$. This is not in the domain however. Hence the necessary and sufficient conditions for a minimum are not satisfied on the interior. We check the boundary for a minimum. Let $\gamma(t) = (\cos t, \sin t)$ on $[0,2\pi]$. We wish to minimize $f\circ \gamma$. Note that this will attain a minimum, since it is a continuous function on a compact set. It is also periodic with period $2\pi$ so once we determine the global minimum, it will be the minimum of the entire function. We have: 
$$0= (f\circ \gamma)^\prime (t)= 2\cos t(\sin t -5) - 4\sin t(\cos t - 3) =-2 \sin t \cos t - 10 \cos t + 12 \sin t.$$
This will be solved by $t = 0.759, t = 3.775$. Note that $f^{\prime \prime}(0.759) >0, f^{\prime \prime}(0.759)<0$. Therefore on the boundary, the minimum is attained at $t = 0.759$. This will be the only minimum on the boundary, by periodicity. Therefore the function attains its minimum at $(x,y) = (0.725, 0.688)$ at a value of $28.96$. 
\end{document}
