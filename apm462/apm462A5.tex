\documentclass[12pt, a4paper]{article}
\usepackage[lmargin =1 in, 
rmargin=1 in, 
tmargin=1in,
bmargin=0.5in]{geometry}
\geometry{letterpaper}
\usepackage{tikz-cd}
\usepackage{amsmath}
\usepackage{amssymb}
\usepackage{blindtext}
\usepackage{titlesec}
\usepackage{enumitem}
\usepackage{fancyhdr}
\usepackage{amsthm}
\usepackage{graphicx}
\usepackage{cool}
\usepackage{thmtools}
\usepackage{hyperref}
\graphicspath{ }					%path to an image

%-------- sexy font ------------%
%\usepackage{libertine}
%\usepackage{libertinust1math}

%\usepackage{mlmodern}				% very nice and classic
%\usepackage[utopia]{mathdesign}
%\usepackage[T1]{fontenc}

\usepackage{mlmodern}
\usepackage{eulervm}
%\usepackage{tgtermes} 				%times new roman
%-------- sexy font ------------%


% Problem Styles
%====================================================================%


\newtheorem{problem}{Problem}


\theoremstyle{definition}
\newtheorem{thm}{Theorem}
\newtheorem{lemma}{Lemma}
\newtheorem{prop}{Proposition}
\newtheorem{cor}{Corollary}
\newtheorem{fact}{Fact}
\newtheorem{defn}{Definition}
\newtheorem{example}{Example}
\newtheorem{question}{Question}

\newtheorem{manualprobleminner}{Problem}

\newenvironment{manualproblem}[1]{%
	\renewcommand\themanualprobleminner{#1}%
	\manualprobleminner
}{\endmanualprobleminner}

\newcommand{\penum}{ \begin{enumerate}[label=\bf(\alph*), leftmargin=0pt]}
	\newcommand{\epenum}{ \end{enumerate} }

% Math fonts shortcuts
%====================================================================%

\newcommand{\ring}{\mathcal{R}}
\newcommand{\N}{\mathbb{N}}                           % Natural numbers
\newcommand{\Z}{\mathbb{Z}}                           % Integers
\newcommand{\R}{\mathbb{R}}                           % Real numbers
\newcommand{\C}{\mathbb{C}}                           % Complex numbers
\newcommand{\F}{\mathbb{F}}                           % Arbitrary field
\newcommand{\Q}{\mathbb{Q}}                           % Arbitrary field
\newcommand{\PP}{\mathcal{P}}                         % Partition
\newcommand{\M}{\mathcal{M}}                         % Mathcal M
\newcommand{\eL}{\mathcal{L}}                         % Mathcal L
\newcommand{\T}{\mathbb{T}}                         % Mathcal T
\newcommand{\U}{\mathcal{U}}                         % Mathcal U\\
\newcommand{\V}{\mathcal{V}}                         % Mathcal V

% symbol shortcuts
%====================================================================%

\newcommand{\bd}{\partial}
\newcommand{\grad}{\nabla}
\newcommand{\lam}{\lambda}
\newcommand{\imp}{\implies}
\newcommand{\all}{\forall}
\newcommand{\exs}{\exists}
\newcommand{\delt}{\delta}
\newcommand{\ep}{\varepsilon}
\newcommand{\ra}{\rightarrow}
\newcommand{\vph}{\varphi}

\newcommand{\ol}{\overline}
\newcommand{\f}{\frac}
\newcommand{\lf}{\lfrac}
\newcommand{\df}{\dfrac}

% bracketting shortcuts
%====================================================================%
\newcommand{\abs}[1]{\left| #1 \right|}
\newcommand{\babs}[1]{\Big|#1\Big|}
\newcommand{\bound}{\Big|}
\newcommand{\BB}[1]{\left(#1\right)}
\newcommand{\dd}{\mathrm{d}}
\newcommand{\artanh}{\mathrm{artanh}}
\newcommand{\Med}{\mathrm{Med}}
\newcommand{\Cov}{\mathrm{Cov}}
\newcommand{\Corr}{\mathrm{Corr}}
\newcommand{\tr}{\mathrm{tr}}
\newcommand{\Range}[1]{\mathrm{range}(#1)}
\newcommand{\Null}[1]{\mathrm{null}(#1)}
\newcommand{\lan}{\left\langle}
\newcommand{\ran}{\right\rangle}
\newcommand{\norm}[1]{\left\lVert#1\right\rVert}
\newcommand{\inn}[1]{\lan#1\ran}
\newcommand{\op}[1]{\operatorname{#1}}
\newcommand{\bmat}[1]{\begin{bmatrix}#1\end{bmatrix}}
\newcommand{\pmat}[1]{\begin{pmatrix}#1\end{pmatrix}}
\newcommand{\vmat}[1]{\begin{vmatrix}#1\end{vmatrix}}

\newcommand{\amogus}{{\bigcap}\kern-0.8em\raisebox{0.3ex}{$\subset$}}
\newcommand{\Note}{\textbf{Note: }}
\newcommand{\Aside}{{\bf Aside: }}
%restriction
%\newcommand{\op}[1]{\operatorname{#1}}
%\newcommand{\done}{$$\mathcal{QED}$$}

%====================================================================%


\setlength{\parindent}{0pt}      	% No paragraph indentations
\pagestyle{fancy}
\fancyhf{}							% fancy header

\setcounter{secnumdepth}{0}			% sections are numbered but numbers do not appear
\setcounter{tocdepth}{2} 			% no subsubsections in toc

%template
%====================================================================%
%\begin{manualproblem}{1}
%Spivak.
%\end{manualproblem}

%\begin{proof}[Solution]
%\end{proof}

%----------- or -----------%

%\begin{problem} 		
%\end{problem}	

%\penum
%	\item
%\epenum
%====================================================================%


\newcommand{\Course}{APM462}
\newcommand{\hwNumber}{5}

%preamble

\title{}
\author{A.N.}
\date{\today}
\lhead{\Course A\hwNumber}
\rhead{\thepage}
%\cfoot{\thepage}


%====================================================================%
\begin{document}



\begin{problem}
% problem number 1
\end{problem}
\penum 
\item The Euler-Lagrange Equations are:
	$$ \frac{ d }{ dt } \left( \frac{ \partial \eL }{ \partial \dot{\mathbf{x}} } \right) = \frac{ \partial \eL  }{ \partial \mathbf{x} }. $$
Our Lagrangian is taken to be:
$$ \eL(x,y,\dot{x}, \dot{y}) = \frac{ 1 }{ 2 }  \left( \dot{x}^2 + \dot{y}^2 - x^2 - y^2\right). $$ 
Therefore our Euler-Lagrange equations are given as:
$$ \frac{ d }{ dt }  \bmat{ \dot{x} \\ \dot {y} } = \bmat{\ddot{x} \\  \ddot{y}} =  \bmat{-x \\ -y}  .$$ 
Therefore the Euler-Lagrange equations are:
$$ \bmat{\ddot{x} \\ \ddot{y} } = \bmat{-x \\- y} .$$ 
\item It is an elementary result from ODE's that the solutions for $x,y$ are of the following forms: 
$$ \bmat{x(t) \\ y(t)}  = \bmat{C_1 \cos t + C_2 \sin t \\ D_1 \cos t + D_2 \sin t} .$$
We now solve them subject to the initian conditions $(x(0), y(0)) = (A_1,A_2)$, $(x(1), y(1)) = (B_1,B_2)$.
First we see that:
$$ \bmat{A_1 \\ A_2}  = \bmat{x(0) \\ y(0)} = \bmat{C_1 \\ D_1}.$$ 
The other condition gives us:
$$ \bmat{B_1 \\ B_2} = \bmat{x(1) \\ y(1)}  = \bmat{A_1 \cos 1 + C_2 \sin 1\\ A_2 \cos 1 + D_2 \sin 1} \implies \bmat{C_2 \\ D_2} = \bmat{ \frac{ B_1 - A_1\cos 1 }{ \sin 1 } \\ \frac{B_2 - A_2 \cos 1  }{ \sin 1 }} .$$
It follows that our path $(x,y)$ takes the following form :
$$ \bmat{x(t) \\ y(t) } = \bmat{A_1 \cos t + \frac{ B_1 - A_1 \cos 1 }{ \sin 1 }\sin t \\ A_2 \cos t + \frac{ B_2 - A_2\cos 1 }{ \sin 1 }\sin t} $$ 
\epenum
\newpage
\begin{problem}
% problem number 2
\end{problem}
\penum
\item Maximimzing $F$ is equivalent to minimizing $-F$, subject to  $H(x,y,z) = 2xy + 2xz + 2yz-A = 0$. So we take our lagrangian to be:
	$$ \eL(x,y,z, \dot{x}, \dot{y}, \dot{z}) = -xyz +\lambda H(x,y,z). $$ 
Our Euler-Lagrange equations are: 
$$ \frac{ d }{ dt } \left( \frac{ \partial \eL }{ \partial \dot{\mathbf{x}} } \right) = \frac{ \partial \eL  }{ \partial \mathbf{x} }. $$ 
Observe that our Lagrangian has no dependance on $\dot{x},\dot{y}, \dot{z}$, so the lefthand side vanishes. Our Lagrangian reduces to :
$$ \bmat{0\\0\\0}= \frac{ \partial \eL }{ \partial x } = \bmat{-yz + \lambda (2y + 2z)\\ -xz+ \lambda(2x + 2z) \\ -xy + \lambda (2x + 2y) } .$$
Therefore the Euler-Lagrange Equations are:
$$ \bmat{yz \\ xz \\ xy} = \lambda \bmat{2y+2z \\ 2x+2z  \\ 2x+2y} $$ 
subject to $$ A = 2xy + 2xz + 2yz. $$ 
Where it is understood that $x,y,z$ and possibly $\lambda$ are functions of $t$. 
\item First note that $\lambda \neq 0$, since if it was we would have that $x=y=z=0$, which cannot happen by the constraint. We multiply the Euler-Lagrange equations by $x,y,z$ respectively to get:
	$$ \bmat{xyz \\ xyz \\ xyz} = \lambda \bmat{2xy+ 2xz \\ 2xy + 2yz \\ 2xz + 2yz} .$$ 
Setting these equal to eachother, we get:
$$ \lambda \left( 2xy + 2xz  \right) = \lambda \left( 2xy + 2 yz \right), \lambda \left( 2xy + 2xz \right) = \lambda \left( 2xz + 2yz \right), \lambda \left( 2xy + 2yz  \right) = \lambda \left( 2xz + 2yz \right). $$
Since $\lambda \neq 0$ we get that $xz = yz, xy = xz, xy = yz$. Further more since $x,y,z$ are positive so we can conclude that $x=y=z$. Using the constraint we see that:
$$A =  2xy + 2yz + 2xz   = 6x^2 \implies x = \sqrt{\frac{ A }{6 }}.$$ 
Thus $x=y=z = \sqrt{ \frac{ A }{ 6 }}$. 
\epenum
\newpage
\begin{problem}
% problem number 3
\end{problem}
\penum 
\item We define our Lagrangian:
	$$ \eL = \frac{ 1 }{ 2 } \left[ \dot{x}^2 + \dot{y}^2 + \dot{z}^2 \right] - gz + \lambda \left( z - x^2 - y^2 \right). $$ 
The E-L equations are: 
$$ \frac{ d }{ dt } \left( \frac{ \partial \eL  }{ \partial \dot{\mathbf{x}} } \right) = \frac{ \partial \eL  }{ \partial \mathbf{x} }. $$ 
We compute the E-L equations as: 
$$ \frac{ d }{ dt  } \bmat{\dot{x} \\ \dot{y} \\ \dot{z} } = \bmat{\ddot{x} \\ \ddot{y} \\ \ddot{z}}  = \bmat{-2\lambda x \\ -2\lambda y \\ -g + \lambda}$$ 
Subject to $z = x^2 + y^2$. 
\item Note that for $(\cos t , \sin t , at^2 + bt +c)$ to satisfy the (unconstrained) Euler Lagrange equations we can take $\lambda = \frac{ 1 }{ 2 }$, to get that $x,y$ satisfy $x^{\prime \prime } = -x$ and $y^{\prime \prime } = -y$. Since $\sin, \cos$ are solutions to this. We are left with 
$$ \ddot{z} = -g + \frac{ 1 }{ 2 }. $$
Integrating this twice with respect to $t$ we get 
$$ z(t) = \frac{ 1 }{ 2 } \left( -g + \frac{ 1 }{ 2 } \right)t^2 +bt +c. $$ 
Therefore for $a = \frac{ -g }{ 2 } + \frac{ 1 }{ 4 }$, and any $b,c$ we have that the curve 
$$ \left( x(t), y(t), z(t) \right) = \left( \cos t , \sin t , \frac{ 1 }{ 2 } \left( -g+ \frac{ 1 }{ 2 } \right)t^2 + bt + c \right) $$ 
solves the Euler Lagrange equations. 
\item We substitute in $z = x^2 + y^2$ into our lagrangian, and using the chain rule to compute $\dot{z} = 2x \dot{x} + 2y\dot{y}$, we get: 
	$$ \eL = \frac{ 1 }{ 2 } \left( \dot{x}^2 + \dot{y}^2 + 4 x^2 \dot{x}^2 + 8xy\dot{x} \dot{y} + 4 y^2 \dot{y}^2 \right) - g \left( x^2 + y^2 \right). $$ 
We compute the Euler-Lagrange equations: 
$$ \frac{ d }{ dt } \bmat{ \dot{x} + 4x^2 \dot{x} + 4xy\dot{y} \\ \dot{y}  + 4y^2 \dot{y} + 4x \dot{x} y} = \bmat{\ddot{x} + 8 x \dot{x}^2 + 4x^2 \ddot{x} + 4 \left( \dot{x} y\dot{y} + x \dot{y}^2 + xy\ddot{y} \right) \\ \ddot{y}+ 8y\dot{y}^2 + 4y^2 \ddot{y} + 4 \left( \dot{x}^2y + x \ddot{x} y + x \dot{x} \dot{y} \right)}  = \bmat{4 x \dot{x}^2 + 4y \dot{x} \dot{y} - 2gx	\\ 4y\dot{y}^2 + 4x\dot{x}\dot{y} -2gy}.$$
Simplifying this becomes: 
$$ \bmat{\ddot{x} \left( 1 + 4x^2 \right) + 4 \left( x \dot{y}^2 + x y \ddot{y} \right) + 2gx \\ \ddot{y} \left( 1+ 4y^2 \right) + 4 \left( \dot{x}^2 y+ x \ddot{x}y \right) + 2gy} = \bmat{0 \\ 0}. $$ 
\epenum
\newpage
\begin{problem}
% problem number 4
\end{problem}
\penum 
\item We wish to find a $u$ satisfying 
$$ \frac{ \delta F }{ \delta u }+ \lambda \frac{ \delta G }{ \delta u } = 0 .$$
Since $\eL_f =  \frac{ 1 }{ 2 }\dot{u}^2$, we compute that 
$$ \frac{ \delta F }{ \delta u } =- \frac{ d }{ dt } \frac{ \partial \eL_f }{\partial \dot{u} } + \frac{ \partial \eL_f }{ \partial u } =- \ddot{u}. $$ 
Similarly for $G$, $\eL_g = u^2$. We compute:
$$ \frac{ \delta G }{ \delta u } = - \frac{ d }{ dt } \frac{ \partial \eL_g }{ \partial \dot{u} } + \frac{ \partial \eL_g }{ \partial u } = 2u.$$
Our first order condition becomes:
$$ - \ddot{u} + 2\lambda u = 0. $$ 
\item First observe that $u(t) = Ae^{t} + Be^{-t}$ satisfies $-u'' + u =0$. So it will solve the Euler-Lagrange equation for $\lambda = \frac{ 1 }{ 2 }$. We require that $u(0) = 0, u(\pi) = 1$. This tells us that 
\begin{align*}
	A+B & = 0
	\\ Ae^{\pi} + Be^{-\pi} & = 1
\end{align*}
Solving, we get:
$$ \bmat{A \\ B } = \bmat{ \frac{ 1 }{ e^\pi - e^{-\pi} } \\  \frac{ -1 }{ e^\pi - e^{-\pi} }} $$ 
\epenum
\newpage
\begin{problem}
% problem number 5
\end{problem}
\penum 
\item We write $\eL_f = \sqrt{\dot{x}^2 + \dot{y}^2} , \eL_H = x^2 + y^2 + z^2 -1$ as the Lagrangians for $F,H$ respectively. The Euler-Lagrange equations are:
	$$ \frac{ \delta F }{ \delta u } + \lambda \frac{ \delta H }{ \delta u } = 0.$$
This becomes:
$$ - \frac{ d }{ dt  } \frac{ \partial \eL_f }{ \partial \dot{\mathbf{x}} }  + \frac{ \partial \eL_f }{ \partial \mathbf{x} } + \lambda \left( - \frac{ d }{ dt } \frac{ \partial \eL_H }{ \partial \dot{\mathbf{x}} } + \frac{ \partial \eL_H }{ \partial \mathbf{x} } \right) = 0.$$ 
Explicitly computing this, coordinate wise we get: 
$$ \frac{ d }{ dt } \bmat{ \frac{ \dot{x} }{ \sqrt{\dot{x}^2 + \dot{y}^2} } \\  \frac{ \dot{y} }{ \sqrt{\dot{x}^2 + \dot{y}^2 }} \\ 0}  = \bmat{  \frac{ \ddot{x} (\dot{x}^2 + \dot{y}^2)^{1/2} - \dot{x} \left( \dot{x} \ddot{x} + \dot{y} \ddot{y} \right) \left( \dot{x}^2 + \dot{y}^2 \right)^{-1/2}}{ \left( \dot{x}^2 + \dot{y}^2 \right)}  \\     \frac{ \ddot{y} (\dot{x}^2 + \dot{y}^2)^{1/2} - \dot{y} \left( \dot{x} \ddot{x} + \dot{y} \ddot{y} \right) \left( \dot{x}^2 + \dot{y}^2 \right)^{-1/2}}{ \left( \dot{x}^2 + \dot{y}^2 \right)}     \\  0}= -\lambda  \bmat{2x \\ 2y \\ 2z} $$ 
\item We now find a $\lambda$ so that $(\sin t, \cos t , 0)$ satisfy these equations.
	Observe that the Euler-Lagrange equations are: 
	$$ \frac{ d }{ dt } \bmat{ \cos t, \\ -\sin t \\ 0 } = -\lambda \bmat{2 \sin t, \\ 2 \cos t \\ 0} \implies \bmat{-\sin t \\ -\cos t \\ 0 } = -\lambda \bmat{  2 \sin t \\ 2\cos t \\ 0}.$$ 
	Taking $\lambda = - \frac{ 1 }{ 2 }$ will suffice. 
\epenum
\newpage
\begin{problem}
% problem number 6
\end{problem}
\penum 
\item To maximize $F$ we minimize $-F$. Let $\eL_F = -x \cdot y, \eL_G = \sqrt{\dot{x}^2 + \dot{y}^2}$. The Euler-Lagrange equations can be written in terms of function derivatives as follows: 
$$ \frac{ \delta \eL_f }{ \delta u } + \lambda \frac{ \delta \eL_G }{ \delta u } = 0. $$ 
We compute: 
$$ \frac{ \delta \eL_F }{ \delta u } = - \frac{ d }{ dt  } \left( \frac{ \partial \eL_F }{ \partial \dot{\mathbf{x}}  } \right) + \frac{ \partial \eL_F }{ \partial \mathbf{x} } = \frac{ d }{ dt } \bmat{0 \\ -x} + \bmat{-\dot{y} \\ 0} = \bmat{-\dot{y} \\ \dot{x}}.  $$ 
Similarly for $\eL_G$ , we compute the functional derivative as: 
$$ \frac{ \delta \eL_G }{ \delta u } = - \frac{ d }{ dt  } \left( \frac{ \partial \eL_G }{ \partial \dot{\mathbf{x}} } \right) + \frac{ \partial \eL_G }{ \partial \mathbf{x} } = - \frac{ d }{ dt } \bmat{ \frac{ \dot{x} }{ \sqrt{\dot{x}^2 + \dot{y}^2 } } \\ \frac{ \dot{y} }{ \sqrt{\dot{x}^2 + \dot{y}^2}} }. $$
Therefore the Euler-Lagrange equations can be written as:
$$ \bmat{-\dot{y} \\ \dot{x}} = \lambda \frac{ d }{ dt } \bmat{ \frac{ \dot{x} }{ \sqrt{\dot{x}^2 + \dot{y}^2} } \\ \frac{ \dot{y} }{ \sqrt{\dot{x}^2 + \dot{y}^2} } } $$ 
\item We verify that $(x(t), y(t)) = (\cos t, \sin t)$ solves. Taking $\lambda =1$, we can rewrite this as 
	$$ \bmat{-\dot{y} \\ \dot{x}} = \frac{ d }{ dt } \bmat{\dot{x} \\ \dot{y}} = \bmat{\ddot{x} \\ \ddot{y}}.$$
We get: 
$$-\dot{y} =- \cos t  = \frac{ d^2 }{ dt^2 }\cos t = \ddot{x} $$ 
and, 
$$ \dot{x} = - \sin t = \frac{ dt}{ dt^2 } \sin t = \ddot{y}.$$
This satisfies the ODE and the boundary values. We conclude that a circle maximizes enclosed area for given perimeter $2\pi$. 
\epenum
\newpage


\end{document}
