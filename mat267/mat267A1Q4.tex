\documentclass[letterpaper]{article}
\usepackage[letterpaper,margin=1in,footskip=0.25in]{geometry}
\usepackage[utf8]{inputenc}
\usepackage{amsmath}
\usepackage{amsthm}
\usepackage{amssymb, pifont}
\usepackage{mathrsfs}
\usepackage{enumitem}
\usepackage{fancyhdr}
\usepackage{hyperref}

\pagestyle{fancy}
\fancyhf{}
\rhead{MAT 267}
\lhead{Assignment 1}
\rfoot{Page \thepage}

\setlength\parindent{24pt}
\renewcommand\qedsymbol{$\blacksquare$}

\DeclareMathOperator{\U}{\mathcal{U}}
\DeclareMathOperator{\Prt}{\mathbb{P}}
\DeclareMathOperator{\R}{\mathbb{R}}
\DeclareMathOperator{\N}{\mathbb{N}}
\DeclareMathOperator{\Z}{\mathbb{Z}}
\DeclareMathOperator{\Q}{\mathbb{Q}}
\DeclareMathOperator{\C}{\mathbb{C}}
\DeclareMathOperator{\ep}{\varepsilon}
\DeclareMathOperator{\identity}{\mathbf{0}}
\DeclareMathOperator{\card}{card}
\newcommand{\suchthat}{;\ifnum\currentgrouptype=16 \middle\fi|;}

\newtheorem{lemma}{Lemma}

\newcommand{\tr}{\mathrm{tr}}
\newcommand{\ra}{\rightarrow}
\newcommand{\lan}{\langle}
\newcommand{\ran}{\rangle}
\newcommand{\norm}[1]{\left\lVert#1\right\rVert}
\newcommand{\inn}[1]{\lan#1\ran}
\newcommand{\ol}{\overline}
\begin{document}
\noindent Q4a: We claim that $x^\prime(t)>0$ for all $t\in \R$. First, notice that if $x^\prime(t)=0$ for some $t\in\R$, we have that by uniqueness $x\equiv a$ on all of $\R$. This can not be the case though, since $f(x(0))=f(a)>0$. Suppose now that $f(t_0)<0$ at some point $t_0$. Then by taking a sufficiently large interval with endpoints $t_0$ and $0$, the intermediate value theorem implies that at some point $x_0$, $f(x_0)=0$. Once again, this implies that we have a constant solution which can not be the case
\newline \\Q4b:  We first take note that the function $x(t)$ can never  be equal to $0$ or $1$, since this would imply it is a constant solution. Thus we have that $0<x(t)<1$. Since $x(t)$ is strictly increasing and continuous and bounded aboce at each point, by the monotone convergence theorem $\lim_{t \to \infty}x(t)$ converges to the supremum of $x(t)$. This will be 1. Simliarly, $\lim_{t\to -\infty}x(t)$ will converge to the infimum, which will be 0. 
We can evaluate that $\lim_{|t| \to \infty} x^\prime(t)=0$ in the following way. 
\begin{align*}
    \lim_{t\to \infty} x^\prime & = \lim_{|t|\to \infty} \lim_{h\to 0} \frac{x(t+h)-x(t)}{h}
    \\ & = \lim_{h\to 0} \lim_{|t|\to \infty} \frac{x(t+h)-x(t)}{h}
    \\ & = \lim_{h\to 0} \frac{0}{h}
    \\ & =0
\end{align*}
\newline \\ Q4c: Suppose that $y(t)$ is a solution of $x^\prime = f(x)$, with initial value $y(0)>=b>a$. We know that $y$ will share properties with $x$, namely those shown in $a$ and $b$.
Suppose that in some neighborhood of a point $t_0$, we have that $y(t_0)\leq x(t_0)$. The intermediate value theorem tells us that the functions $x$ and $y$ must be equal at some point. However, from our discussion in class we know that integreal curves can not cross eachother. Hence $y(t)>x(t) \forall t$. 
\end{document}