\documentclass[letterpaper]{article}
\usepackage[letterpaper,margin=1in,footskip=0.25in]{geometry}
\usepackage[utf8]{inputenc}
\usepackage{amsmath}
\usepackage{amsthm}
\usepackage{amssymb, pifont}
\usepackage{mathrsfs}
\usepackage{enumitem}
\usepackage{fancyhdr}
\usepackage{hyperref}

\pagestyle{fancy}
\fancyhf{}
\rhead{MAT 267}
\lhead{Assignment 3}
\rfoot{Page \thepage}

\setlength\parindent{24pt}
\renewcommand\qedsymbol{$\blacksquare$}

\DeclareMathOperator{\U}{\mathcal{U}}
\DeclareMathOperator{\Prt}{\mathbb{P}}
\DeclareMathOperator{\R}{\mathbb{R}}
\DeclareMathOperator{\N}{\mathbb{N}}
\DeclareMathOperator{\Z}{\mathbb{Z}}
\DeclareMathOperator{\Q}{\mathbb{Q}}
\DeclareMathOperator{\C}{\mathbb{C}}
\DeclareMathOperator{\ep}{\varepsilon}
\DeclareMathOperator{\identity}{\mathbf{0}}
\DeclareMathOperator{\card}{card}
\newcommand{\suchthat}{;\ifnum\currentgrouptype=16 \middle\fi|;}

\newtheorem{lemma}{Lemma}

\newcommand{\tr}{\mathrm{tr}}
\newcommand{\ra}{\rightarrow}
\newcommand{\lan}{\langle}
\newcommand{\ran}{\rangle}
\newcommand{\norm}[1]{\left\lVert#1\right\rVert}
\newcommand{\inn}[1]{\lan#1\ran}
\newcommand{\ol}{\overline}
\begin{document}
\noindent Q1a: We see that the solution is periodic with respect to $x_1,x_2$ with a period of $\sqrt{2}\pi$, and is periodic with a period of $\frac{2\pi}{\sqrt3}$ 
We see however, that the system will not oscillate, since the periods are both irrational. Since the eigenvalues are pure complex, the system will have a center and hence will neither grow nor decay. 
The stable solution will be $0$, and there will be no sources or sinks. We now consider the second ODE. With respect to the first two coordinates, this ODE will exhibit periodic behaviour, while in the 3rd and 4th coordinates, it will be a spiral sink towards 0. We have that 0 is a stable point, and that every initial condition will eventually decay to 0. These systems differ, since one will have a circular motion around 0, while another will decay towards 0. 
\end{document}