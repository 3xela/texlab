\documentclass[letterpaper]{article}
\usepackage[letterpaper,margin=1in,footskip=0.25in]{geometry}
\usepackage[utf8]{inputenc}
\usepackage{amsmath}
\usepackage{amsthm}
\usepackage{amssymb, pifont}
\usepackage{mathrsfs}
\usepackage{enumitem}
\usepackage{fancyhdr}
\usepackage{hyperref}

\pagestyle{fancy}
\fancyhf{}
\rhead{MAT 267}
\lhead{Assignment 3}
\rfoot{Page \thepage}

\setlength\parindent{24pt}
\renewcommand\qedsymbol{$\blacksquare$}

\DeclareMathOperator{\U}{\mathcal{U}}
\DeclareMathOperator{\Prt}{\mathbb{P}}
\DeclareMathOperator{\R}{\mathbb{R}}
\DeclareMathOperator{\N}{\mathbb{N}}
\DeclareMathOperator{\Z}{\mathbb{Z}}
\DeclareMathOperator{\Q}{\mathbb{Q}}
\DeclareMathOperator{\C}{\mathbb{C}}
\DeclareMathOperator{\ep}{\varepsilon}
\DeclareMathOperator{\identity}{\mathbf{0}}
\DeclareMathOperator{\card}{card}
\newcommand{\suchthat}{;\ifnum\currentgrouptype=16 \middle\fi|;}

\newtheorem{lemma}{Lemma}

\newcommand{\tr}{\mathrm{tr}}
\newcommand{\ra}{\rightarrow}
\newcommand{\lan}{\langle}
\newcommand{\ran}{\rangle}
\newcommand{\norm}[1]{\left\lVert#1\right\rVert}
\newcommand{\inn}[1]{\lan#1\ran}
\newcommand{\ol}{\overline}
\begin{document}
\noindent Q4a: We claim that $U$ is a contraction with constant $\frac{L}{M}$. We first observe the following: 
    $$\frac{d}{dt} e^{-Mt}\int_0^t|\gamma_1(s)-\gamma_2(s) |ds  = -M e^{-Mt} \int_{0}^t |\gamma_1(s) -\gamma_2(s) | + e^{-Mt} |\gamma_1(t) -\gamma_2(t)| $$
Hence at some $t_0$, we have a maximum and so $$\frac{1}{M} |\gamma_1(t_0) -\gamma_2(t_0)| = \int_{0}^{t_0} |\gamma_1(s) - \gamma_2(s)| ds$$
\newline We now will show that $U$ is indeed a contraction. Observe: 
\begin{align*} 
    \norm{U(\gamma_1)-U(\gamma_2)}_M & = \norm{\int_{0}^t f(\gamma_1(s)) - f(\gamma_2(s))ds}_M
    \\ & = \sup_{t\geq 0} e^{-Mt} \Big|\int_{0}^t f(\gamma_1(s) -f(\gamma_2(s)) ds \Big|
    \\ & \leq \sup_{t\geq 0} e^{-Mt} \int_{0}^t \Big| f(\gamma_1(s)) -f(\gamma_2(s)) ds\Big|
    \\ & \leq \sup_{t\geq 0} L e^{-Mt} \int_{0}^t |\gamma_1(s)-\gamma_2(s) |ds \tag{by Lipschitz}
    \\ & \leq \frac{L}{M} e^{-Mt_0} |\gamma_1(t_0) - \gamma_2(t_0)|
    \\ & \leq \frac{L}{M} \sup_{t\geq 0} e^{-Mt} |\gamma_1(t) -\gamma_2(t)|
    \\ & = \frac{L}{M} \norm{\gamma_1-\gamma_2}_M
\end{align*}
If it is the case that $M>L$, then this map will be a contraction
\newline \\ Q4b: Since $U$ is a contraction if the condition from 4a is met, we have that there exists a unique fixed point $y$ where $y(t) = U(y(t))$ or equivalently $y^\prime = f(y) $, with $y(0)=v$
\newline \\ Q4c: By 4b, $y(t)$ exists on all $t \geq 0$. Since $y\in C_M$ we have that
$$\sup_{t\geq 0}e^{-Mt} |y(t)| \leq K \implies \exists t_0: |y(t_0)| \leq K e^{Mt} \implies |y(t)| \leq Le^{-Mt}$$
\end{document}