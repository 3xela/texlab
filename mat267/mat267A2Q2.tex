\documentclass[letterpaper]{article}
\usepackage[letterpaper,margin=1in,footskip=0.25in]{geometry}
\usepackage[utf8]{inputenc}
\usepackage{amsmath}
\usepackage{amsthm}
\usepackage{amssymb, pifont}
\usepackage{mathrsfs}
\usepackage{enumitem}
\usepackage{fancyhdr}
\usepackage{hyperref}

\pagestyle{fancy}
\fancyhf{}
\rhead{MAT 267}
\lhead{Assignment 2}
\rfoot{Page \thepage}

\setlength\parindent{24pt}
\renewcommand\qedsymbol{$\blacksquare$}

\DeclareMathOperator{\U}{\mathcal{U}}
\DeclareMathOperator{\Prt}{\mathbb{P}}
\DeclareMathOperator{\R}{\mathbb{R}}
\DeclareMathOperator{\N}{\mathbb{N}}
\DeclareMathOperator{\Z}{\mathbb{Z}}
\DeclareMathOperator{\Q}{\mathbb{Q}}
\DeclareMathOperator{\C}{\mathbb{C}}
\DeclareMathOperator{\ep}{\varepsilon}
\DeclareMathOperator{\identity}{\mathbf{0}}
\DeclareMathOperator{\card}{card}
\newcommand{\suchthat}{;\ifnum\currentgrouptype=16 \middle\fi|;}

\newtheorem{lemma}{Lemma}

\newcommand{\tr}{\mathrm{tr}}
\newcommand{\ra}{\rightarrow}
\newcommand{\lan}{\langle}
\newcommand{\ran}{\rangle}
\newcommand{\norm}[1]{\left\lVert#1\right\rVert}
\newcommand{\inn}[1]{\lan#1\ran}
\newcommand{\ol}{\overline}
\begin{document}
\noindent Q1a: The matrix $A = \begin{pmatrix} 0 & 0 & a \\ 0 & b & 0 \\ a & 0 & 0 \end{pmatrix}$ will have eigenvalues of $\lambda_1 = -a$ corresonding to $v_1 = \begin{pmatrix} -1 \\ 0 \\ 1 \end{pmatrix}$, $\lambda_2 = b$ corresponding to $v_2 = \begin{pmatrix} 0 \\ 1 \\ 0 \end{pmatrix}$ and $\lambda_3 = a$ corresponding to $v_3 = \begin{pmatrix} 1\\ 0 \\ 1 \end{pmatrix}$. Therefore this will have a general solution of $x(t) = \alpha_1 e^{-at}\cdot v_1 + \alpha_2 e^{bt} \cdot v_2 +\alpha_3 e^{at}\cdot v_3$ for some constants $\alpha_1,\alpha_2,\alpha_3\in \R$ which will depend on the inital conditions of this system.  
\end{document}