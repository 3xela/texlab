\documentclass[letterpaper]{article}
\usepackage[letterpaper,margin=1in,footskip=0.25in]{geometry}
\usepackage[utf8]{inputenc}
\usepackage{amsmath}
\usepackage{amsthm}
\usepackage{amssymb, pifont}
\usepackage{mathrsfs}
\usepackage{enumitem}
\usepackage{fancyhdr}
\usepackage{hyperref}

\pagestyle{fancy}
\fancyhf{}
\rhead{MAT 267}
\lhead{Assignment 1}
\rfoot{Page \thepage}

\setlength\parindent{24pt}
\renewcommand\qedsymbol{$\blacksquare$}

\DeclareMathOperator{\U}{\mathcal{U}}
\DeclareMathOperator{\Prt}{\mathbb{P}}
\DeclareMathOperator{\R}{\mathbb{R}}
\DeclareMathOperator{\N}{\mathbb{N}}
\DeclareMathOperator{\Z}{\mathbb{Z}}
\DeclareMathOperator{\Q}{\mathbb{Q}}
\DeclareMathOperator{\C}{\mathbb{C}}
\DeclareMathOperator{\ep}{\varepsilon}
\DeclareMathOperator{\identity}{\mathbf{0}}
\DeclareMathOperator{\card}{card}
\newcommand{\suchthat}{;\ifnum\currentgrouptype=16 \middle\fi|;}

\newtheorem{lemma}{Lemma}

\newcommand{\tr}{\mathrm{tr}}
\newcommand{\ra}{\rightarrow}
\newcommand{\lan}{\langle}
\newcommand{\ran}{\rangle}
\newcommand{\norm}[1]{\left\lVert#1\right\rVert}
\newcommand{\inn}[1]{\lan#1\ran}
\newcommand{\ol}{\overline}
\begin{document}
\noindent Q2a: We wish to solve the differential system of equations $X^\prime = \begin{pmatrix}
    1 & 2 \\ 0 & 3
\end{pmatrix}X$. We first notice that the matrix $\begin{pmatrix}
    1 & 2 \\ 0 & 3
\end{pmatrix}$ has eigenvalues $\lambda_1 = 1$ and $\lambda_2=3$, with corresponding eigenvectors $v_1=\begin{pmatrix}
    1 \\ 0
\end{pmatrix}$ and $v_2 = \begin{pmatrix}
    1 \\ 1
\end{pmatrix}$. Therefore the general solution takes the form of $$X(t) = \alpha e^{t}\cdot \begin{pmatrix}
    1 \\ 0
\end{pmatrix} + \beta e^{3t} \cdot \begin{pmatrix}
    1\\ 1
\end{pmatrix}$$ for $\alpha,\beta \in \R$. This will correspond to the slope field 4. 
\newline \\ Q2b: We wish to solve $X^\prime = \begin{pmatrix}
    1 & 2 \\ 3 & 6
\end{pmatrix}X$. We see that the matrix $\begin{pmatrix}
    1 & 2 \\ 3 & 6
\end{pmatrix}$ has eigenvalues $\lambda_1=0$ corresponding to $v_1= \begin{pmatrix}
    -2 \\ 1
\end{pmatrix}$ and $\lambda_2=7$ corresponding to $v_2 = \begin{pmatrix}
    1\\3
\end{pmatrix}$ Therefore, the general solution will be of the form $$X(t) = \alpha \cdot \begin{pmatrix}
    -2\\1
\end{pmatrix} + \beta e^{7t}\cdot \begin{pmatrix}
    1\\3
\end{pmatrix}$$ for $\alpha,\beta\in \R$. This will correspond to slope field 2. 
\newline \\ Q2c: To solve the system $X^\prime = \begin{pmatrix}
    1 & 2 \\ 1 & 0
\end{pmatrix}X$ we will compute the eigenvalues and the corresponding eigenvectors. We see that $\lambda_1=2$ corresponding to $v_2=\begin{pmatrix}
    2\\1
\end{pmatrix}$ and $\lambda_2= -1$ corresponding to $v_2 = \begin{pmatrix}
    -1\\1
\end{pmatrix}$. Therefore the solution to this ODE is $$X(t) = \alpha e^{2t}\cdot \begin{pmatrix}
    2\\1
\end{pmatrix} + \beta e^{-1} \cdot \begin{pmatrix}
    -1 \\ 1
\end{pmatrix}$$ for $\alpha,\beta\in \R$. This will correspond to slope field 1. 
\newline \\ Q2d: To solve the system $X^\prime = \begin{pmatrix}
    1 & 2 \\ 3& -3
\end{pmatrix}X$ we will compute the eigenvectors and eigenvalues of $\begin{pmatrix}
    1 & 2 \\ 3& -3
\end{pmatrix}$. We see that $\lambda_1 = -1 + \sqrt{10}$ coresponding to $v_1=\begin{pmatrix}
    2+ \sqrt{10}\\3
\end{pmatrix}$, and $\lambda_2=-1\sqrt{10}$ with $v_2=\begin{pmatrix}
    2-\sqrt{10}\\3
\end{pmatrix}$ Therefore the solution to this ODE will be $$X(t)= \alpha e^{(-1+\sqrt{10})t}\cdot \begin{pmatrix}
    2+\sqrt{10}\\ 3
\end{pmatrix} + \beta e^{(-1-\sqrt{10})t}\cdot \begin{pmatrix}
    2-\sqrt{10}\\ 3
\end{pmatrix}$$ for $\alpha,\beta\in \R$. This will correspond to slope field 3. 
\end{document}