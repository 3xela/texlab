\documentclass[letterpaper]{article}
\usepackage[letterpaper,margin=1in,footskip=0.25in]{geometry}
\usepackage[utf8]{inputenc}
\usepackage{amsmath}
\usepackage{amsthm}
\usepackage{amssymb, pifont}
\usepackage{mathrsfs}
\usepackage{enumitem}
\usepackage{fancyhdr}
\usepackage{hyperref}

\pagestyle{fancy}
\fancyhf{}
\rhead{MAT 267}
\lhead{Assignment 1}
\rfoot{Page \thepage}

\setlength\parindent{24pt}
\renewcommand\qedsymbol{$\blacksquare$}

\DeclareMathOperator{\U}{\mathcal{U}}
\DeclareMathOperator{\Prt}{\mathbb{P}}
\DeclareMathOperator{\R}{\mathbb{R}}
\DeclareMathOperator{\N}{\mathbb{N}}
\DeclareMathOperator{\Z}{\mathbb{Z}}
\DeclareMathOperator{\Q}{\mathbb{Q}}
\DeclareMathOperator{\C}{\mathbb{C}}
\DeclareMathOperator{\ep}{\varepsilon}
\DeclareMathOperator{\identity}{\mathbf{0}}
\DeclareMathOperator{\card}{card}
\newcommand{\suchthat}{;\ifnum\currentgrouptype=16 \middle\fi|;}

\newtheorem{lemma}{Lemma}

\newcommand{\tr}{\mathrm{tr}}
\newcommand{\ra}{\rightarrow}
\newcommand{\lan}{\langle}
\newcommand{\ran}{\rangle}
\newcommand{\norm}[1]{\left\lVert#1\right\rVert}
\newcommand{\inn}[1]{\lan#1\ran}
\newcommand{\ol}{\overline}
\begin{document}
\noindent Q2a: Solve $x^\prime = 2x+t^2$. We guess that $x$ takes the form of a quadratic, namely $x(t)=at^2+bt+c$. We know that $x^\prime(t)=2at+b$. For equality to hold, we can choose constants $a=b=-\frac{1}{2}$, and $c=\frac{b}{2}=-\frac{1}{4}$. Therefore we have that 
$$x^\prime = -t+\frac{-1}{2} = 2(-\frac{1}{2}t^2-\frac{-1}{2}t-\frac{1}{4})+t^2=2x+t^2$$
\newline \\ Q2b: Solve $x^\prime = 2x + e^{3t}$. We will guess that $x(t)=ce^(3t)$ for some constant $c$. We note that $x^\prime=3ce^{3t}$. Thus for equality to hold, we can take $c=1$ and observe that 
$$x^\prime = 3e^{3t} = 2e^{3t}+e^{3t} = 2x+e^{3t}$$
\newline \\ Q2c: To solve $x^\prime = 2x + te^{3t}$ we will proceed by guessing something of the form $x(t) = cte^{3t}-de^3t$. By simple computation we see that $x^\prime(t)=3ce^{3t}+3cte^{3t}-3de^{3t}$, and $2x+te^{3t} = 2cte^{3t}-2de^{3t}+te^{3t}$. We have equality holding if we choose constants $c=d=1$. Thus $x(t)=te^{3t}-e^{3t}$ solves. 
\newline \\ Q2d: We guess a solution of the form $x(t)= c \sin(t) + 6d \cos(t) + e^{kt}$. We have that $x^\prime = c \cdot cos(t)-d\sin(t)+ke^{kt}$ and $2x-cos(t)=2c \sin(t) + (2d-1)cos(t) + 2e^{kt}$. We see that when $k=2$, $c=-\frac{2}{5}$, $d=-\frac{2}{5}$. Therefore, $x(t)=-\frac{cos(t)}{5}-\frac{2\sin(t)}{5}+e^{2t}$ solves. 
\newline \\ Q2e:  To solve $x^\prime = 2x+f(t)$, we guess a solution of the form  $x(t) = te^{2t}+ e^{2t}$. We see that $x^\prime = e^{2t} + 2te^{3t} + 2e^{2t}$, and $2x+e^{2t}=2te^{2t}+2e^{2t}+e^{2t}$. These functions are equal, so our guess is correct. 
\end{document}