\documentclass[letterpaper]{article}
\usepackage[letterpaper,margin=1in,footskip=0.25in]{geometry}
\usepackage[utf8]{inputenc}
\usepackage{amsmath}
\usepackage{amsthm}
\usepackage{amssymb, pifont}
\usepackage{mathrsfs}
\usepackage{enumitem}
\usepackage{fancyhdr}
\usepackage{hyperref}

\pagestyle{fancy}
\fancyhf{}
\rhead{MAT 454}
\lhead{Assignment 1}
\rfoot{Page \thepage}

\setlength\parindent{24pt}
\renewcommand\qedsymbol{$\blacksquare$}

\DeclareMathOperator{\Qu}{\mathcal{Q}_8}
\DeclareMathOperator{\F}{\mathbb{F}}
\DeclareMathOperator{\T}{\mathcal{T}}
\DeclareMathOperator{\V}{\mathcal{V}}
\DeclareMathOperator{\U}{\mathcal{U}}
\DeclareMathOperator{\Prt}{\mathbb{P}}
\DeclareMathOperator{\R}{\mathbb{R}}
\DeclareMathOperator{\N}{\mathbb{N}}
\DeclareMathOperator{\Z}{\mathbb{Z}}
\DeclareMathOperator{\Q}{\mathbb{Q}}
\DeclareMathOperator{\C}{\mathbb{C}}
\DeclareMathOperator{\ep}{\varepsilon}
\DeclareMathOperator{\identity}{\mathbf{0}}
\DeclareMathOperator{\card}{card}
\newcommand{\suchthat}{;\ifnum\currentgrouptype=16 \middle\fi|;}

\newtheorem{lemma}{Lemma}

\newcommand{\tr}{\mathrm{tr}}
\newcommand{\ra}{\rightarrow}
\newcommand{\lan}{\langle}
\newcommand{\ran}{\rangle}
\newcommand{\norm}[1]{\left\lVert#1\right\rVert}
\newcommand{\inn}[1]{\lan#1\ran}
\newcommand{\ol}{\overline}
\newcommand{\ci}{i}
\begin{document}
\noindent Q2a: Let $f(z)$ be a holomorphic funtion. Since every open set is the union of open balls, it is enough to show that $f$ maps an open ball to an open set. 
Note that $f$ must be analytic since it is holomorphic. Therefore the zeros of $f^\prime$ are isolated. Every open set either contains a zero of $f^\prime$, or it does not. 
If we take an open set $U$ so that $f^\prime(z) \neq 0 $ on $U$, then the inverse function theorem tells us that $f|U$ is a bijection and hence is open when restricted to this set. 
Now take some open $V$ such that it contains a zero of $f^\prime$, $z_0$. We have that $V \setminus \{z_0\}$ is open, and so $f(V\setminus\{z_0\})$ is open by the previous case.
Continuity of $f$ tells us that $f(z_0) \in int(f(V)).$ Therefore $f(V)$ is open. Any open set can be written as the union of open sets either containing a zero of $f^\prime$ or not.
Therefore $f$ is an open mapping. 
\newline \\ Q2b: Clearly $f(U)\subset V$. We show that $V \subset f(V)$. Suppose $x\in V$ such that $f^{-1}(\{x\})= \emptyset$. Define $X = \ol{B_\varepsilon(x)} $.
We have that $f^{-1}(X)$ is compact and nonempty. Take $\{K_n\}$ to be a family of nested decreasing compact sets, whose intersection is $x$. 
We have that for each $n$, $f^{-1}(K_n)$ is compact and the sequence $\{f^{-1}(K_n)\}$ is also a decreasing family of nested compact sets.
From basic topology, the intersection $\bigcap_{n}f^{-1}(K_n)$ must be nonempty. It follows that $f^{-1}(\{x\})$ is nonempty. A contradiction. Therefore $f(U)=V$. 
\newline \\ Q2c: Consider the mapping $f(z): \C \to \C$ defined as $f(z) = |z|$. This will be a continuous but not holomorphic mapping. Note that $f(\C) = \R_{x\geq 0}$, which is not an open set.  


\end{document}