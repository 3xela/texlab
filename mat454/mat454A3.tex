\documentclass[12pt, a4paper]{article}
\usepackage[lmargin =0.5 in, 
rmargin=0.5in, 
tmargin=1in,
bmargin=0.5in]{geometry}
\geometry{letterpaper}
\usepackage{amsmath}
\usepackage{amssymb}
\usepackage{blindtext}
\usepackage{titlesec}
\usepackage{enumitem}
\usepackage{fancyhdr}
\usepackage{amsthm}
\usepackage{graphicx}
\usepackage{cool}
\usepackage{thmtools}
\usepackage{hyperref}
\graphicspath{ }					%path to an image

%-------- sexy font ------------%
%\usepackage{libertine}
%\usepackage{libertinust1math}

%\usepackage{mlmodern}				% very nice and classic
%\usepackage[utopia]{mathdesign}
%\usepackage[T1]{fontenc}


\usepackage{mlmodern}
\usepackage{eulervm}
%\usepackage{tgtermes} 				%times new roman
%-------- sexy font ------------%


% Problem Styles
%====================================================================%


\newtheorem{problem}{Problem}


\theoremstyle{definition}
\newtheorem{thm}{Theorem}
\newtheorem{lemma}{Lemma}
\newtheorem{prop}{Proposition}
\newtheorem{cor}{Corollary}
\newtheorem{fact}{Fact}
\newtheorem{defn}{Definition}
\newtheorem{example}{Example}
\newtheorem{question}{Question}

\newtheorem{manualprobleminner}{Problem}

\newenvironment{manualproblem}[1]{%
	\renewcommand\themanualprobleminner{#1}%
	\manualprobleminner
}{\endmanualprobleminner}

\newcommand{\penum}{ \begin{enumerate}[label=\bf(\alph*), leftmargin=0pt]}
	\newcommand{\epenum}{ \end{enumerate} }

% Math fonts shortcuts
%====================================================================%

\newcommand{\ring}{\mathcal{R}}
\newcommand{\N}{\mathbb{N}}                           % Natural numbers
\newcommand{\Z}{\mathbb{Z}}                           % Integers
\newcommand{\R}{\mathbb{R}}                           % Real numbers
\newcommand{\C}{\mathbb{C}}                           % Complex numbers
\newcommand{\F}{\mathbb{F}}                           % Arbitrary field
\newcommand{\Q}{\mathbb{Q}}                           % Arbitrary field
\newcommand{\PP}{\mathcal{P}}                         % Partition
\newcommand{\M}{\mathcal{M}}                         % Mathcal M
\newcommand{\eL}{\mathcal{L}}                         % Mathcal L
\newcommand{\T}{\mathcal{T}}                         % Mathcal T
\newcommand{\U}{\mathcal{U}}                         % Mathcal U\\
\newcommand{\V}{\mathcal{V}}                         % Mathcal V

% symbol shortcuts
%====================================================================%

\newcommand{\lam}{\lambda}
\newcommand{\imp}{\implies}
\newcommand{\all}{\forall}
\newcommand{\exs}{\exists}
\newcommand{\delt}{\delta}
\newcommand{\eps}{\varepsilon}
\newcommand{\ra}{\rightarrow}

\newcommand{\ol}{\overline}
\newcommand{\f}{\frac}
\newcommand{\lf}{\lfrac}
\newcommand{\df}{\dfrac}

% bracketting shortcuts
%====================================================================%
\newcommand{\abs}[1]{\left| #1 \right|}
\newcommand{\babs}[1]{\Big|#1\Big|}
\newcommand{\bound}{\Big|}
\newcommand{\BB}[1]{\left(#1\right)}
\newcommand{\dd}{\mathrm{d}}
\newcommand{\artanh}{\mathrm{artanh}}
\newcommand{\Med}{\mathrm{Med}}
\newcommand{\Cov}{\mathrm{Cov}}
\newcommand{\Corr}{\mathrm{Corr}}
\newcommand{\tr}{\mathrm{tr}}
\newcommand{\Range}[1]{\mathrm{range}(#1)}
\newcommand{\Null}[1]{\mathrm{null}(#1)}
\newcommand{\lan}{\langle}
\newcommand{\ran}{\rangle}
\newcommand{\norm}[1]{\left\lVert#1\right\rVert}
\newcommand{\inn}[1]{\lan#1\ran}
\newcommand{\op}[1]{\operatorname{#1}}
\newcommand{\bmat}[1]{\begin{bmatrix}#1\end{bmatrix}}
\newcommand{\pmat}[1]{\begin{pmatrix}#1\end{pmatrix}}
\newcommand{\vmat}[1]{\begin{vmatrix}#1\end{vmatrix}}

\newcommand{\amogus}{{\bigcap}\kern-0.8em\raisebox{0.3ex}{$\subset$}}
\newcommand{\Note}{\textbf{Note: }}
\newcommand{\Aside}{{\bf Aside: }}
%restriction
%\newcommand{\op}[1]{\operatorname{#1}}
%\newcommand{\done}{$$\mathcal{QED}$$}

%====================================================================%


\setlength{\parindent}{0pt}      	% No paragraph indentations
\pagestyle{fancy}
\fancyhf{}							% fancy header

\setcounter{secnumdepth}{0}			% sections are numbered but numbers do not appear
\setcounter{tocdepth}{2} 			% no subsubsections in toc

%template
%====================================================================%
%\begin{manualproblem}{1}
%Spivak.
%\end{manualproblem}

%\begin{proof}[Solution]
%\end{proof}

%----------- or -----------%

%\begin{problem} 		
%\end{problem}	

%\penum
%	\item
%\epenum
%====================================================================%


\newcommand{\Course}{MAT454 }
\newcommand{\hwNumber}{3}

%preamble

\title{a}
\author{A.N.}
\date{\today}

\lhead{\Course A\hwNumber}
\rhead{\thepage}
%\cfoot{\thepage}


%====================================================================%
\begin{document}
	\begin{problem}
	\end{problem}
We have previously shown that $$\frac{1}{z^2} + \sum_{n= -\infty, n\neq 0 }^\infty \frac{1}{(z-n)^2} = \Big(\frac{\pi}{\sin(\pi z)} \Big)^2.$$ Sufficiently near $0$, the righthand side admits the following laurent series: 
$$\Big(\frac{\pi}{\sin(\pi z)} \Big)^2 = \Big( \frac{\pi}{\pi z - \frac{1}{6}\pi^3 z^3 + \dots  } \Big)^2 = \frac{1}{z^2} + \frac{\pi^2}{3} + z^2(\dots).$$ Therefore at $z=0$
we have that $$\sum_{n= -\infty}^\infty \frac{1}{n^2} =\frac{\pi^2}{3},$$ and by symmetry 
$$\sum_{n= 1}^\infty \frac{1}{n^2} = \frac{\pi^2}{6}. $$
Our intial expression is holomorphic so we apply the derivative twice to yield that 
$$\frac{6}{z^4} + \sum_{n = -\infty}^{\infty} \frac{6}{(z-n)^4} = \frac{2\pi^4 (1+ 2\cos^2(\pi z))}{\sin^4(\pi z)}.$$ Near $z=0$ the righthand side has the following laurent expansion: 
$$\frac{2\pi^4 (1+ 2\cos^2(\pi z))}{\sin^4(\pi z)} = \frac{6}{z^4} + \frac{2\pi^4}{15} + z( \dots). $$
Therefore at $z=0$ we have that $$\sum_{n = -\infty }^\infty \frac{6}{(n)^4} = \frac{2\pi^4}{15}.$$ By symmetry we have $$\sum_{n = 1}^\infty \frac{1}{n^4} = \frac{\pi^4}{90}. $$
\newpage
\begin{problem}
\end{problem}
\begin{enumerate}[label=\alph*)]
	\item Let $\{a_k\}$ and $\{b_k\}$ be the zeros and poles of $f$ respectively. They must all be of order 2 since $f$ is an even elliptic function. Then the function $$ f(z) \cdot \prod_{k=1}^n \frac{\wp(z) - \wp(b_k)}{\wp(z) - \wp(a_k)}$$ is elliptic, has no zeros or poles since the zeros sum to $0$ mod $\Gamma$. Hence it is constant by Liouvilles Theorem. Therefore we can write $$f(z) = c\cdot \prod_{k=1}^n \frac{\wp(z) - \wp(a_k)}{\wp(z) - \wp(b_k)}. $$
	If $0 $ is a pole of order $2l$, then the function $$f(z) \cdot \prod_{k=1}^n \frac{\wp(z) - \wp(a_k)}{\wp(z) - \wp(b_k)} \cdot \frac{1}{\wp(z)^l}$$ has no zeros nor poles and is hence constant. Similarly, if $0$ is a zero of degree $2l$ we have that $$f(z) \cdot \prod_{k=1}^n \frac{\wp(z) - \wp(a_k)}{\wp(z) - \wp(b_k)} \cdot \wp(z)^l $$ is constant. Therefore if $f$ is an even elliptic function we can write $$f(z) = R(\wp),$$ where $R$ is a rational function. 
	\item If $f$ is an odd elliptic function, then $$\frac{f(z)}{\wp^\prime(z)}$$ must be even and elliptic and so can be written as some rational function of $\wp$. So $$f(z) = \wp^\prime R(\wp). $$
	\item We write $$f(z) = \frac{1}{2} \Big(f(z) + f(-z) \Big)  + \frac{1}{2} \Big(f(z) - f(-z) \Big). $$
	The first summand on the right hand side is even and elliptic and so can be written as a rational function of $\wp$. The second summand is odd and so can be written as a rational function of $\wp, \wp^\prime$. Therefore $f(z)$ can be written as a rational function of $\wp, \wp^\prime$.
\end{enumerate} 
	\newpage
	\begin{problem}
	\end{problem}
 By results from class it is enough to check that \begin{enumerate}[label = \arabic*:]
 	\item $1+z^{2^n}$ converges to $1$ uniformly as $n\to \infty$ 
 	\item  $\sum_{n=0}^\infty log(1+z^{2^n})$ is uniformly and absolutely convergent on compact subsets of $|z|<1$.  
 	 \end{enumerate}
  For 1, we have that $|z^{2^n}| \to 0$ uniformly as $n\to \infty$ for $|z|<1$. It remains to show $2$ holds. From first year calculus we have that $$|\log(1+z^{2^n})| \leq |z|^{2^n}$$ for all $z$. We have that after finitely many $n$, $$|z|^{2^n} < \frac{1}{n^2}.$$
  Thus by the Weierstrass M test the series $\sum_{n=0}^\infty log(1+z^{2^n})$ will converge absolutely and uniformly, and thus so will $\prod_{n=0}^\infty (1+z^{2^n})$. 
  \newpage
  \begin{problem}
  \end{problem}
\begin{enumerate}[label = \alph*)]
	\item
To show that $$ f(z) = \prod_{n=1}^\infty \Big( 1+ \frac{z}{n}\Big)e^{-\frac{z}{n}}$$ is an 
entire function we will show that \begin{enumerate}[label = \roman*)]
	\item $\Big( 1+ \frac{z}{n}\Big)e^{-\frac{z}{n}} \to 1$ as $n\to \infty$
	\item $\sum_{n=1}^\infty \log \Big(\Big( 1+ \frac{z}{n}\Big)e^{-\frac{z}{n}} \Big)$ converges uniformly and absolutely. 
 \end{enumerate}
First $i$ follows from using the taylor expansion of $e$, yielding that $$\Big( 1+ \frac{z}{n}\Big)e^{-\frac{z}{n}} = 1 - \frac{z}{2n^2} + \frac{z^3}{3n^3} + \dots. $$
This clearly converges to $1$ uniformly as $n\to \infty$. We now check $ii$. We have that for each $z$, $$\Big| \log \Big(\Big( 1+ \frac{z}{n}\Big)e^{-\frac{z}{n}} \Big) \Big| \leq \frac{|z|^2}{n^2},$$ so by the Weierstrass M test $ii$ converges uniformly and absolutely. 
Therefore $f(z)$ represents an entire function. The zeros of $f$ occur exactly when $\Big( 1+ \frac{z}{n}\Big)e^{-\frac{z}{n}}$ is zero, which happens only on the negative integers. 
\item By part $a)$, $$\frac{1}{H(z)}=ze^z \prod_{n=1}^\infty (1+ \frac{z}{n})e^{-\frac{z}{n}}$$ is holomorphic. By logarithmic differentiation, we compute that $$ - \frac{H^\prime(z)}{H(z)} = \frac{d}{dz} \Big( log \Big(\frac{1}{H(z) } \Big) \Big) = \frac{1}{z} + 1 + \sum_{n=1}^\infty \frac{-z}{(nz+n^2)}.$$
This function is holomorphic since it is the derivative of a holomorphic function. Taking the derivative once more we get that $$\frac{d}{dz} \Big( - \frac{H^\prime(z)}{H(z)} \Big) = \frac{d}{dz} \Big( \frac{1}{z} + 1 +  \sum_{n=1}^\infty \frac{-z}{(nz+n^2)} \Big) = \sum_{n=0}^\infty \frac{-1}{(z+n)^2}. $$ As desired. 
 \end{enumerate}
\newpage
\begin{problem}
\end{problem}
\penum
\item
We define the following functions: $$g(z) = z\prod_{i=1}^\infty \left(1 - \frac{z}{n}\right)e^{-\frac{z}{n}}, \tilde{g}(z)  = \prod_{i=1}^\infty \left(1 - \frac{z+1}{n} \right)e^{-\frac{z+1}{n}}.$$ Both of these functions are holomorphic, so we can apply logarithmic differentiation to see that $$\log(g(z))^\prime = \frac{1}{z} + \sum \frac{1}{1 - \frac{z}{n}}- \frac{1}{n} = \sum \frac{1}{1 - \frac{z+1}{n}}  - \frac{1}{n} = \log(\tilde{g}(z))^\prime.$$ Therefore for some $c$ we have that $g(z) = e^c \tilde{g}(z).$ Now define $$f(z) = e^{cz} \prod_{i=1}^\infty \left( 1 - \frac{z}{n}\right) e^{-\frac{z}{n}}.$$ It follows that $$f(z+1) = e^c e^{cz} \prod_{i=1}^n \left(1- \frac{z+1}{n}	\right) e^{- \frac{z+1}{n}} = e^{c}e^{cz}\tilde{g}(z) = e^{cz}g(z) =zf(z).$$
Our choice of $f$ is holomorphic since constructed from holomorphic functions. It also satisfies the given properties. 
\item Let $p(z) = a_nz^n + \dots a_1z+a_0 = a_n(z-c_1)\dots (z-c_n).$ We define $$F(z) = a_n \prod_{i=1}^n f(z-c_i),$$ where $f$ is defined as above. Then, $$F(z+1) = a_n \prod_{i=1}^n f(z+1-c_i) = a_n \prod_{i=1}^\infty f((z-c_i) +1) =a_n \prod_{i=1}^\infty(z-c_i)f(z-c_i) = p(z)F(z)$$
\epenum
\newpage
\begin{problem}
\end{problem}
Let $n\in \N, 0\neq f\in H(\C)$ be given. We first suppose that there is some entire $g$ so that $f = g^n$. If the zero set of $f$ is empty the result is clear by A1Q4. Now suppose that $f(a)=0$ with order of $k$. Then sufficiently close to $a$, we can write $$f(z) = (z-a)^k \cdot \tilde{f}(z), g(z) = (z-a)^m\cdot \tilde{g}(z)$$
for nonzero $\tilde{f}, \tilde{g}.$ By assumption we have $$(z-a)^k \tilde{f}(z) = (z-a)^{nl}\tilde{g}^n(z).$$ Since $\tilde{f},\tilde{g}^n$ nonzero, we have that $k|nl$ i.e. $k$ is a multiple of $n$. Conversely suppose that every zero of $f$ has order divisible by $n$. Let $\{a_i\}$ be the zero set with corresponding orders $\{nk_i\}$. We define $$\tilde{g}(z) = z^{k_0} \prod_{i}^\infty \left[ \left(1- \frac{z}{a_i} \right) e^{p_i(z)}\right]$$ in such a way so that $a_i$ is a root of $\tilde{g}(z)$ with order $k_i$ for certain polynomials $p_i(z)$. It follows that the quotient $f/\tilde{g}^n$ is holomorphic and nonzero, so we can write $$\frac{f(z)}{\tilde{g}^n(z)} = e^{h(z)}$$ for some entire $h(z)$. Thus we have that $$f(z) = e^{h(z)}\tilde{g}^n(z) = \left(e^{\frac{h(z)}{n}}\tilde{g}(z) \right)^n = g^n(z).$$ Where we take $g(z) = e^{\frac{h(z)}{n}} \tilde{g}(z). $ This is exactly what we wanted to show. 
\newpage
\begin{problem}
\end{problem}
Let $\{a_i\}$ be the zero set of $f_1$. Let $\{b_i\}$ be the zero set of $f_2$. We can define a holomorphic function $h$ so that $h(a_i)=0, h(b_i)=1$ and $a_i$ is a root of $h(a_i)=0$ with the same order as $f_1(a_i)=0$, and $b_i$ is a root of $h(b_i)-1=0$ with the same order as $f_2(b_i)=0$. Then we have that $\frac{h(z) -1}{f_2(z)}=g_2(z)$ is holomorphic. We also have that the quotient $\frac{h(z)}{f_1(z)}$ is holomorphic, since it has no poles. Letting $g_1(z ) = \frac{h(z)}{f_1(z)}$ we see that $$\frac{h(z) - 1}{f_2(z)} = h_2(z) \implies f_1(z)g_1(z) = 1-f_2(z)g_2(z) \implies f_1(z)g_1(z) + f_2(z)g_2(z) = 1. $$
\end{document}