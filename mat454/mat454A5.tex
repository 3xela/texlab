\documentclass[12pt, a4paper]{article}
\usepackage[lmargin =0.5 in, 
rmargin=0.5in, 
tmargin=1in,
bmargin=0.5in]{geometry}
\geometry{letterpaper}
\usepackage{amsmath}
\usepackage{amssymb}
\usepackage{blindtext}
\usepackage{titlesec}
\usepackage{enumitem}
\usepackage{fancyhdr}
\usepackage{amsthm}
\usepackage{graphicx}
\usepackage{cool}
\usepackage{thmtools}
\usepackage{hyperref}
\graphicspath{ }					%path to an image

%-------- sexy font ------------%
%\usepackage{libertine}
%\usepackage{libertinust1math}

%\usepackage{mlmodern}				% very nice and classic
%\usepackage[utopia]{mathdesign}
%\usepackage[T1]{fontenc}


\usepackage{mlmodern}
\usepackage{eulervm}
%\usepackage{tgtermes} 				%times new roman
%-------- sexy font ------------%


% Problem Styles
%====================================================================%


\newtheorem{problem}{Problem}


\theoremstyle{definition}
\newtheorem{thm}{Theorem}
\newtheorem{lemma}{Lemma}
\newtheorem{prop}{Proposition}
\newtheorem{cor}{Corollary}
\newtheorem{fact}{Fact}
\newtheorem{defn}{Definition}
\newtheorem{example}{Example}
\newtheorem{question}{Question}

\newtheorem{manualprobleminner}{Problem}

\newenvironment{manualproblem}[1]{%
	\renewcommand\themanualprobleminner{#1}%
	\manualprobleminner
}{\endmanualprobleminner}

\newcommand{\penum}{ \begin{enumerate}[label=\bf(\alph*), leftmargin=0pt]}
	\newcommand{\epenum}{ \end{enumerate} }

% Math fonts shortcuts
%====================================================================%

\newcommand{\ring}{\mathcal{R}}
\newcommand{\N}{\mathbb{N}}                           % Natural numbers
\newcommand{\Z}{\mathbb{Z}}                           % Integers
\newcommand{\R}{\mathbb{R}}                           % Real numbers
\newcommand{\C}{\mathbb{C}}                           % Complex numbers
\newcommand{\F}{\mathbb{F}}                           % Arbitrary field
\newcommand{\Q}{\mathbb{Q}}                           % Arbitrary field
\newcommand{\PP}{\mathcal{P}}                         % Partition
\newcommand{\M}{\mathcal{M}}                         % Mathcal M
\newcommand{\eL}{\mathcal{L}}                         % Mathcal L
\newcommand{\T}{\mathcal{T}}                         % Mathcal T
\newcommand{\U}{\mathcal{U}}                         % Mathcal U\\
\newcommand{\V}{\mathcal{V}}                         % Mathcal V

% symbol shortcuts
%====================================================================%

\newcommand{\lam}{\lambda}
\newcommand{\imp}{\implies}
\newcommand{\all}{\forall}
\newcommand{\exs}{\exists}
\newcommand{\delt}{\delta}
\newcommand{\eps}{\varepsilon}
\newcommand{\ra}{\rightarrow}

\newcommand{\ol}{\overline}
\newcommand{\f}{\frac}
\newcommand{\lf}{\lfrac}
\newcommand{\df}{\dfrac}

% bracketting shortcuts
%====================================================================%
\newcommand{\abs}[1]{\left| #1 \right|}
\newcommand{\babs}[1]{\Big|#1\Big|}
\newcommand{\bound}{\Big|}
\newcommand{\BB}[1]{\left(#1\right)}
\newcommand{\dd}{\mathrm{d}}
\newcommand{\artanh}{\mathrm{artanh}}
\newcommand{\Med}{\mathrm{Med}}
\newcommand{\Cov}{\mathrm{Cov}}
\newcommand{\Corr}{\mathrm{Corr}}
\newcommand{\tr}{\mathrm{tr}}
\newcommand{\Range}[1]{\mathrm{range}(#1)}
\newcommand{\Null}[1]{\mathrm{null}(#1)}
\newcommand{\lan}{\langle}
\newcommand{\ran}{\rangle}
\newcommand{\norm}[1]{\left\lVert#1\right\rVert}
\newcommand{\inn}[1]{\lan#1\ran}
\newcommand{\op}[1]{\operatorname{#1}}
\newcommand{\bmat}[1]{\begin{bmatrix}#1\end{bmatrix}}
\newcommand{\pmat}[1]{\begin{pmatrix}#1\end{pmatrix}}
\newcommand{\vmat}[1]{\begin{vmatrix}#1\end{vmatrix}}

\newcommand{\amogus}{{\bigcap}\kern-0.8em\raisebox{0.3ex}{$\subset$}}
\newcommand{\Note}{\textbf{Note: }}
\newcommand{\Aside}{{\bf Aside: }}
%restriction
%\newcommand{\op}[1]{\operatorname{#1}}
%\newcommand{\done}{$$\mathcal{QED}$$}

%====================================================================%


\setlength{\parindent}{0pt}      	% No paragraph indentations
\pagestyle{fancy}
\fancyhf{}							% fancy header

\setcounter{secnumdepth}{0}			% sections are numbered but numbers do not appear
\setcounter{tocdepth}{2} 			% no subsubsections in toc

%template
%====================================================================%
%\begin{manualproblem}{1}
%Spivak.
%\end{manualproblem}

%\begin{proof}[Solution]
%\end{proof}

%----------- or -----------%

%\begin{problem} 		
%\end{problem}	

%\penum
%	\item
%\epenum
%====================================================================%


\newcommand{\Course}{MAT454 }
\newcommand{\hwNumber}{5}

%preamble

\title{a}
\author{A.N.}
\date{\today}

\lhead{\Course A\hwNumber}
\rhead{\thepage}
%\cfoot{\thepage}


%====================================================================%
\begin{document}
	\begin{problem}
	\end{problem}
We compute that $$P \circ h = \left(z+\frac{1}{z} \right)^4 - 4\left(z+\frac{1}{z}\right)^2 + 2 =z^4+4z^2+6+\frac{4}{z^2}+ \frac{1}{z^4}-4z^2-8 - \frac{4}{z^2}+2 = z^4+\frac{1}{z^4} = h\circ Q. $$ Now, notice that $$P^n \circ h = \overbrace{P \circ \dots \circ P}^{\text{n times}} \circ h = h \circ Q^n = z^{4n} + \frac{1}{z^{4n}}. $$
Furthermore, from MAT354 we know that $h(z)$ is a conformal mapping of $D$ to $\C \setminus{[-2,2]}$. Thus $\{P^n|_U\}$ is a normal family on some $U$, a neighbourhood of some $a\in [-2,2]$ if and only if $ \left\{(P^n\circ h)|_{h(U)} \right\}$ is normal. Our previous computation verifies that $\left\{(P^n\circ h)|_{h(U)} \right\} =  \left\{(h\circ Q^n)|_{h(U)} \right\}$. For any choice of $U$,the family  $ \left\{(h\circ Q^n)|_{h(U)} \right\}$ becomes pointwise unbounded as $n\to \infty$. Therefore not a normal family. 
\newpage
\begin{problem}
\end{problem}
\penum
\item If $f_n\to \infty$ we are done and $d(f_n,\infty) = \frac{2}{\sqrt{1+ |f_n|^2}} \rightrightarrows 0 $. Suppose that $f_n\to f$ uniformly for $f$ not identically $\infty$ on compact subsets. Let $z$ be a point so that $f_n(z)\to f(z)$ with $f(z)\neq \infty$. By marty's theorem, it is sufficient to show that $\{f_n^\sharp\}$ is locally bounded in a neighbourhood of $z$, then the limit of this sequence will be holomorphic. Notice that $f_n^\sharp(z) = \frac{2|f_n^\prime(z)|}{1+ |f_n(z)|^2}$. If we had that $f_n^\sharp(z)\to \infty$ then it must be that $f_n^\prime(z)\to \infty$. However, this would contradict Montel's Little theorem, since $f_n(z)$ is bounded by $|f(z)|+c$ for some constant $c$. Therefore $f$ must have no poles i.e. it is holomorphic. 
\item If $f_n \to f$ with $f$ holomorphic, then on any compact subset, $f_n$ and $f$ are both bounded uniformly, say by $M$. So convergence in the chordal metric implies that $$d(f_n,f) = \frac{2|f_n-f|}{\sqrt{(1+|f|^2)(1+|f_n|^2)} } \leq \frac{2|f_n-f|}{1+M} \rightrightarrows 0 \implies |f-f_n| \rightrightarrows 0. $$ 
\epenum
\newpage 
\begin{problem}
\end{problem}
Let $f$ be a periodic holomorphic function with period $c$. Suppose $f$ has no fixed points, that is $f(z)-z \neq 0$. Define the holomorphic function $g(z) = f(z) -z$. We have that $g(z)\neq 0$. Furthermore, $g(z-c) = f(z-c)- z+c = g(z)+c. $ Since $g(z)$ omits $0$, it must also omit $c$. This contradicts Picards little theorem. 
\newpage 
\begin{problem}
\end{problem} 
\penum 
\item Consider the function $f(z) = e^{z}+z$. This has a fixed point if there is some $z$ so that $z= f(z)= e^z+z$. Note that no such $z$ can exist however, since such $z$ must also satisfy $e^z=0$.
\item First suppose that $f\circ f$ has no fixed points. Then it follows that $f$ has no fixed points, since any fixed point of $f$ will be a fixed point of $f \circ f$. Consider the function $$g(z) = \frac{f(f(z)) - z}{f(z) - z}.$$ Note that $g$ is entire since $f$ is, and $g$ omits $0$. By Picards theorem, we have that at some $w$, $g(w) =1$. Therefore $f(f(w)) = f(w)$. Which contradicts the assumption that $f\circ f$ has no fixed points. 
\epenum
\newpage
\begin{problem}
\end{problem}
We write $$w = z-\sqrt{z^2-1} \implies \sqrt{z^2-1} = w-z \implies w^2-2zw+1 = 0.$$
This defines a Riemann surface with branch points at $z= \pm1$. So it looks like two copies of $\C$ identified along the interval $[-1,1]$. If we have the holomorphic differential form $$\omega = \frac{dz}{\sqrt{z^2-1}},$$ 
we can write it as $$\omega = \frac{dz}{z-w}$$  in a neighbourhood of $(z_0,w_0)\in X$ away from $w=z$. In a neighbourhood of $z=w$, we have $$d(w^2-2zw +1) = 0 \implies wdz = (w-z)dw.$$ So $\omega = \frac{w-z}{w \sqrt{z^2-1}} dw.$
Homogenizing, our equation becomes $$\left(	\frac{w}{t}\right)^2 -2 \left(\frac{w}{t}\cdot  \frac{z}{t}\right) + 1 = 0 \implies w^2 - 2zw + t^2 = 0.$$
Setting $t=0$, we determine that the coordinates at $\infty$ are $[0,1,0]$ and $[\frac{1}{2}, 1, 0]$ By finding the values of $z,w$ when $w,z$ are local coordinates respectively at $t=0$. We compute the residues of $\omega$ at $\infty$ as: $$res(\omega, [0,1,0]) = 1, res(\omega, [\frac{1}{2} ,1, 0]) = -1.$$

\end{document}
