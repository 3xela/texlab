\documentclass[letterpaper]{article}
\usepackage[letterpaper,margin=1in,footskip=0.25in]{geometry}
\usepackage[utf8]{inputenc}
\usepackage{amsmath}
\usepackage{amsthm}
\usepackage{amssymb, pifont}
\usepackage{mathrsfs}
\usepackage{enumitem}
\usepackage{fancyhdr}
\usepackage{hyperref}

\pagestyle{fancy}
\fancyhf{}
\rhead{MAT 454}
\lhead{Assignment 1}
\rfoot{Page \thepage}

\setlength\parindent{24pt}
\renewcommand\qedsymbol{$\blacksquare$}

\DeclareMathOperator{\Qu}{\mathcal{Q}_8}
\DeclareMathOperator{\F}{\mathbb{F}}
\DeclareMathOperator{\T}{\mathcal{T}}
\DeclareMathOperator{\V}{\mathcal{V}}
\DeclareMathOperator{\U}{\mathcal{U}}
\DeclareMathOperator{\Prt}{\mathbb{P}}
\DeclareMathOperator{\R}{\mathbb{R}}
\DeclareMathOperator{\N}{\mathbb{N}}
\DeclareMathOperator{\Z}{\mathbb{Z}}
\DeclareMathOperator{\Q}{\mathbb{Q}}
\DeclareMathOperator{\C}{\mathbb{C}}
\DeclareMathOperator{\ep}{\varepsilon}
\DeclareMathOperator{\identity}{\mathbf{0}}
\DeclareMathOperator{\card}{card}
\newcommand{\suchthat}{;\ifnum\currentgrouptype=16 \middle\fi|;}

\newtheorem{lemma}{Lemma}

\newcommand{\tr}{\mathrm{tr}}
\newcommand{\ra}{\rightarrow}
\newcommand{\lan}{\langle}
\newcommand{\ran}{\rangle}
\newcommand{\norm}[1]{\left\lVert#1\right\rVert}
\newcommand{\inn}[1]{\lan#1\ran}
\newcommand{\ol}{\overline}
\newcommand{\ci}{i}
\begin{document} \noindent 
Q4a: Fix $z_0\in \Omega$. Since $\Omega$ simply connected, for any $z\in \Omega$ one can define the curve $\gamma(t):[0,1] \to \Omega$ so that $\gamma(0)=z_0$ and $\gamma(1)=z$.
We define the function $$g(z) = \int_{\gamma} \frac{f^\prime(z)}{f(z)}dz -f(z_0).$$ This makes sense since $f(z)\neq 0$ for all $z$, and is well defined since on any other path $\delta$ satisfying the same endpoint conditions as $\gamma$,
 the holomorphicity of $\frac{f^\prime(z)}{f(z)}$ implies that $$\int_{\gamma - \delta} \frac{f^\prime(z)}{f(z)}dz =0 \implies \int_\gamma \frac{f^\prime(z)}{f(z)}dz = \int_\delta \frac{f^\prime(z)}{f(z)}dz.$$
We compute that $$g(z)= \int_{\gamma} \frac{f^\prime(z)}{f(z)}dz -f(z_0) = \int_{[0,1]}\frac{f^\prime(\gamma(t))}{f(\gamma(t))}\cdot \gamma^\prime(t) dt - f(z_0) = \log(f(\gamma(t)))\Big|_{0}^1 - f(z_0) = \log(f(z)).$$ This shows that $g(z)$ is our desired construction, 
and is independant of choice of base point. Finally, we have that $g(z)$ is holomorphic, since it is equal to a composition of holomorphic functions. 
\newline \\ Q4b: We define the function $g(z) = \exp(\frac{1}{n}\log(f(z)))$. We have that $g(z)$ is a composition of holomorphic functions. We see that $$\log(g(z)) = \frac{1}{n} \log(f(z))\implies log(g(z)^n) = \log(f(z)) \implies g(z)^n = f(z),$$ As desired.
\end{document}