\documentclass[letterpaper]{article}
\usepackage[letterpaper,margin=1in,footskip=0.25in]{geometry}
\usepackage[utf8]{inputenc}
\usepackage{amsmath}
\usepackage{amsthm}
\usepackage{amssymb, pifont}
\usepackage{mathrsfs}
\usepackage{enumitem}
\usepackage{fancyhdr}
\usepackage{hyperref}

\pagestyle{fancy}
\fancyhf{}
\rhead{MAT 454}
\lhead{Assignment 2}
\rfoot{Page \thepage}

\setlength\parindent{24pt}
\renewcommand\qedsymbol{$\blacksquare$}

\DeclareMathOperator{\Qu}{\mathcal{Q}_8}
\DeclareMathOperator{\F}{\mathbb{F}}
\DeclareMathOperator{\T}{\mathcal{T}}
\DeclareMathOperator{\V}{\mathcal{V}}
\DeclareMathOperator{\U}{\mathcal{U}}
\DeclareMathOperator{\Prt}{\mathbb{P}}
\DeclareMathOperator{\R}{\mathbb{R}}
\DeclareMathOperator{\N}{\mathbb{N}}
\DeclareMathOperator{\Z}{\mathbb{Z}}
\DeclareMathOperator{\Q}{\mathbb{Q}}
\DeclareMathOperator{\C}{\mathbb{C}}
\DeclareMathOperator{\ep}{\varepsilon}
\DeclareMathOperator{\identity}{\mathbf{0}}
\DeclareMathOperator{\card}{card}
\newcommand{\suchthat}{;\ifnum\currentgrouptype=16 \middle\fi|;}

\newtheorem{lemma}{Lemma}

\newcommand{\tr}{\mathrm{tr}}
\newcommand{\ra}{\rightarrow}
\newcommand{\lan}{\langle}
\newcommand{\ran}{\rangle}
\newcommand{\norm}[1]{\left\lVert#1\right\rVert}
\newcommand{\inn}[1]{\lan#1\ran}
\newcommand{\ol}{\overline}
\newcommand{\ci}{i}
\begin{document} \noindent  Q6:  Note that for any $a,b\in \C$ the function 
$$g(z) = \wp^\prime(z) - a \wp(z) -b$$ is periodic, with period $\Gamma$. By Cartan $iii \S5$ 
prop 5.1, as proven in lecture the number of zeros and periods is equal in any given parellelogram. 
Thus it is enough to show that there are $3$ poles. Note that from the definition of $\wp^\prime$, $0$ will be a triple pole.
Since the pole is 
of order $3$, for any $a,b$, $g(z)$ will also have $3$ poles and hence $3$ zeros. We now claim
that the sum of the zeros is equal to a period. Note that by prop 5.2 and class discussion, 
the sum of the roots for $\wp^\prime(z)-a\wp(z)=b$, $\alpha_i$ and the poles $\beta_i$ satisfy $$\sum \alpha_i \equiv \sum \beta_i mod(\Gamma).$$
Since $\beta_i=0$, the sum of the $\alpha_i's$ will be congruent to $0$ mod $\Gamma$ and hence be a period. 
Now given $u,v$ we wish to find $a,b$ so that $$g(u) = g(v)=0.$$
Since the zeros must sum to $0$ the last zero in the period parellelogram must be $-u-v$. So any $a,b$ that make $g(u)= g(v)=0$ will also make $g(-u-v)=0$. 
We see that by solving the system of equations given by $g(u) = g(v) =0$, we can take $$a = \frac{\wp^\prime(v) - \wp^\prime(u)}{\wp(v) - \wp(u)}, b = \frac{\wp^\prime(u)\wp(v) - \wp^\prime(v)\wp(u)}{\wp(v) - \wp(u)}.$$
Therefore we get that $$ 0 = \wp^\prime(-u-v) - \frac{\wp^\prime(v) - \wp^\prime(u)}{\wp(v) - \wp(u)} \cdot \wp(-u-v) -  \frac{\wp^\prime(u)\wp(v) - \wp^\prime(v)\wp(u)}{\wp(v) - \wp(u)}.$$
Multiplying across by $\wp(v) - \wp(u)$ gives us $$\det \begin{pmatrix}
    \wp(u) & \wp^\prime(u) & 1 \\
    \wp(v) & \wp^\prime(v) & 1 \\ 
    \wp(-u-v) & \wp^\prime(-u-v) & 1
\end{pmatrix}=0$$

\end{document}