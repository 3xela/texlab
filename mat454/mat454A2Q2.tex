\documentclass[letterpaper]{article}
\usepackage[letterpaper,margin=1in,footskip=0.25in]{geometry}
\usepackage[utf8]{inputenc}
\usepackage{amsmath}
\usepackage{amsthm}
\usepackage{amssymb, pifont}
\usepackage{mathrsfs}
\usepackage{enumitem}
\usepackage{fancyhdr}
\usepackage{hyperref}

\pagestyle{fancy}
\fancyhf{}
\rhead{MAT 454}
\lhead{Assignment 2}
\rfoot{Page \thepage}

\setlength\parindent{24pt}
\renewcommand\qedsymbol{$\blacksquare$}

\DeclareMathOperator{\Qu}{\mathcal{Q}_8}
\DeclareMathOperator{\F}{\mathbb{F}}
\DeclareMathOperator{\T}{\mathcal{T}}
\DeclareMathOperator{\V}{\mathcal{V}}
\DeclareMathOperator{\U}{\mathcal{U}}
\DeclareMathOperator{\Prt}{\mathbb{P}}
\DeclareMathOperator{\R}{\mathbb{R}}
\DeclareMathOperator{\N}{\mathbb{N}}
\DeclareMathOperator{\Z}{\mathbb{Z}}
\DeclareMathOperator{\Q}{\mathbb{Q}}
\DeclareMathOperator{\C}{\mathbb{C}}
\DeclareMathOperator{\ep}{\varepsilon}
\DeclareMathOperator{\identity}{\mathbf{0}}
\DeclareMathOperator{\card}{card}
\newcommand{\suchthat}{;\ifnum\currentgrouptype=16 \middle\fi|;}

\newtheorem{lemma}{Lemma}

\newcommand{\tr}{\mathrm{tr}}
\newcommand{\ra}{\rightarrow}
\newcommand{\lan}{\langle}
\newcommand{\ran}{\rangle}
\newcommand{\norm}[1]{\left\lVert#1\right\rVert}
\newcommand{\inn}[1]{\lan#1\ran}
\newcommand{\ol}{\overline}
\newcommand{\ci}{i}
\begin{document} \noindent  Q2: If our differential form is given by $$f(z)dz = \frac{1}{(z-3)(z^5-1)}dz.$$
To compute the integral along $|z|=4$, we can instead change to coordinates at $\infty$ and compute the residue. 
Changing coordiantes to $\infty$ by substituting $z\mapsto \frac{1}{z}$ gives us $$f(\frac{1}{z}) d\frac{1}{z} = \frac{-z^4}{(1-3z)(1-z^5)}dz.$$
Thus we compute that $$\oint_{|z| = 4} f(z)dz = \oint_{|z| = \frac{1}{4}} \frac{-z^4}{(1-3z)(1-z^5)}dz = 2\pi i Res\Big(- \frac{1}{z^2} f(\frac{1}{z}), 0 \Big) = 0,$$
since our function is holomorphic at $\infty$.  To compute the same integral along $|z|=2$, we know that the sum of the residues including at infinty is $0$. Therefore we just compute the residue at $z=3$ since this is not
enclosed by $|z|=2$, and our final answer will be $-(2\pi i \cdot res(f,3))$. We compute the residue at $3$ as $$res(f,3) = \frac{1}{ [ (z-3)(z^5-1) ]^\prime } \Big|_{z=3} = \frac{1}{(6z^5-1-15z^3)}\Big|_{z=3} = \frac{1}{242}.$$
Therefore $$\oint_{|z|=2} \frac{1}{(z-3)(z^5-1)}dz = -(2\pi i \frac{1}{242}) = - \frac{\pi i}{141}$$
\end{document}