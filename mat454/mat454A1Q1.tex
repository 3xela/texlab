\documentclass[letterpaper]{article}
\usepackage[letterpaper,margin=1in,footskip=0.25in]{geometry}
\usepackage[utf8]{inputenc}
\usepackage{amsmath}
\usepackage{amsthm}
\usepackage{amssymb, pifont}
\usepackage{mathrsfs}
\usepackage{enumitem}
\usepackage{fancyhdr}
\usepackage{hyperref}

\pagestyle{fancy}
\fancyhf{}
\rhead{MAT 454}
\lhead{Assignment 1}
\rfoot{Page \thepage}

\setlength\parindent{24pt}
\renewcommand\qedsymbol{$\blacksquare$}

\DeclareMathOperator{\Qu}{\mathcal{Q}_8}
\DeclareMathOperator{\F}{\mathbb{F}}
\DeclareMathOperator{\T}{\mathcal{T}}
\DeclareMathOperator{\V}{\mathcal{V}}
\DeclareMathOperator{\U}{\mathcal{U}}
\DeclareMathOperator{\Prt}{\mathbb{P}}
\DeclareMathOperator{\R}{\mathbb{R}}
\DeclareMathOperator{\N}{\mathbb{N}}
\DeclareMathOperator{\Z}{\mathbb{Z}}
\DeclareMathOperator{\Q}{\mathbb{Q}}
\DeclareMathOperator{\C}{\mathbb{C}}
\DeclareMathOperator{\ep}{\varepsilon}
\DeclareMathOperator{\identity}{\mathbf{0}}
\DeclareMathOperator{\card}{card}
\newcommand{\suchthat}{;\ifnum\currentgrouptype=16 \middle\fi|;}

\newtheorem{lemma}{Lemma}

\newcommand{\tr}{\mathrm{tr}}
\newcommand{\ra}{\rightarrow}
\newcommand{\lan}{\langle}
\newcommand{\ran}{\rangle}
\newcommand{\norm}[1]{\left\lVert#1\right\rVert}
\newcommand{\inn}[1]{\lan#1\ran}
\newcommand{\ol}{\overline}
\newcommand{\ci}{i}
\begin{document}
\noindent Q1i: We write $f(z) = c(z-b_1)\dots(z-b_k).$ It follows that $f^\prime(z) = c \sum_{i=1}^k \prod_{j\neq i}^k (z-b_j)$. When $f(z)$ iz nonzero the quotient becomes $$\frac{f^\prime(z) }{f(z)} = \frac{\sum_{i=1}^k \prod_{j\neq i}^k (z-b_j)}{(z-b_1)\dots(z-b_k)} = \sum_{i=1}^k \frac{1}{(z-b_i)}. $$
If $f(z)=0$, and the zero is of degree 1, then the quotient will just be infinty. If $z_0$ is a zero of degree $k>1$,  then in the summation, the $z_0$ term will appear $k-1$ times. 
\newline \\ Q1ii: We have that $$0 = \sum_{i=1}^k \frac{1}{(z-b_k)} = 0 \implies 0 = \sum_{i=1}^k \frac{1}{(z-b_i)} \cdot \ol{ \Big( \frac{z-b_i}{z-b_i}\Big) } \implies 0 = \sum_{i=1}^k \frac{\ol{z} - \ol{b_i}}{|z-b_i|^2}$$
Seperating the summands we get that $$\ol{z} \cdot \sum_{i=1}^k \frac{1}{|z-b_i|^2} = \sum_{i=1}^k \frac{\ol{b_i}}{|z-b_i|^2},$$ as desired. If we have that $z = b_l$ for some $i$, then as we take $z \to b_l$ then the terms in the summation with index $l$ will both approach the same value and hence they will not be counted. 
\newline \\ Suppose that $z$ satisfies $P^\prime(z) = 0$. Then by conjugating the result from $Q1ii$, we have that $z$ satisfies the expression $$z \cdot \sum_{i=1}^k\frac{1}{|z-b_i|^2} = \sum_{i=1}^k \frac{b_i}{|z-b_i|^2}.$$
Thus we write $$z = \frac{\sum_{i=1}^k \frac{b_i}{|z-b_i|^2}}{\sum_{i=1}^k\frac{1}{|z-b_i|^2} }.$$ The coefficients of the $b_i$ are all positive since it is a sum of norms. Their sum will be $$\frac{\sum_{i=1}^k \frac{1}{|z-b_i|^2}}{\sum_{i=1}^k\frac{1}{|z-b_i|^2} }= 1.$$ Therefore $z$ is a convex linear combination of the $b_i's$ and so it belongs to the convex hull they generate.  
\end{document}