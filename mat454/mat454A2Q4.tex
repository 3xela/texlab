\documentclass[letterpaper]{article}
\usepackage[letterpaper,margin=1in,footskip=0.25in]{geometry}
\usepackage[utf8]{inputenc}
\usepackage{amsmath}
\usepackage{amsthm}
\usepackage{amssymb, pifont}
\usepackage{mathrsfs}
\usepackage{enumitem}
\usepackage{fancyhdr}
\usepackage{hyperref}

\pagestyle{fancy}
\fancyhf{}
\rhead{MAT 454}
\lhead{Assignment 2}
\rfoot{Page \thepage}

\setlength\parindent{24pt}
\renewcommand\qedsymbol{$\blacksquare$}

\DeclareMathOperator{\Qu}{\mathcal{Q}_8}
\DeclareMathOperator{\F}{\mathbb{F}}
\DeclareMathOperator{\T}{\mathcal{T}}
\DeclareMathOperator{\V}{\mathcal{V}}
\DeclareMathOperator{\U}{\mathcal{U}}
\DeclareMathOperator{\Prt}{\mathbb{P}}
\DeclareMathOperator{\R}{\mathbb{R}}
\DeclareMathOperator{\N}{\mathbb{N}}
\DeclareMathOperator{\Z}{\mathbb{Z}}
\DeclareMathOperator{\Q}{\mathbb{Q}}
\DeclareMathOperator{\C}{\mathbb{C}}
\DeclareMathOperator{\ep}{\varepsilon}
\DeclareMathOperator{\identity}{\mathbf{0}}
\DeclareMathOperator{\card}{card}
\newcommand{\suchthat}{;\ifnum\currentgrouptype=16 \middle\fi|;}

\newtheorem{lemma}{Lemma}

\newcommand{\tr}{\mathrm{tr}}
\newcommand{\ra}{\rightarrow}
\newcommand{\lan}{\langle}
\newcommand{\ran}{\rangle}
\newcommand{\norm}[1]{\left\lVert#1\right\rVert}
\newcommand{\inn}[1]{\lan#1\ran}
\newcommand{\ol}{\overline}
\newcommand{\ci}{i}
\begin{document} \noindent  Q4: Letting $$f(z) = \sum_{n= -\infty}^{\infty} \frac{1}{(z+n)^2+a^2},$$ and 
$$g(z) = \frac{\pi}{a} \cdot \frac{\sinh(2\pi a)}{\cosh(2\pi a) - \cos(2\pi z)} = \frac{\pi}{a} \cdot \frac{-i\sin(2\pi i a)}{\cos(2\pi i a) - \cos(2\pi z)}.$$
Notice that $f(z)$ has simple poles of degree two at $z = -n \pm ia$. Note $g(z)$ does as well, following from the $2\pi$ periodicity of $\cos$. We also note that
$f(z+1)=f(z)$ and $g(z+1)=g(z)$. Finally the last property we will discuss is the boundary condition. Once can see that since $$\cos(2\pi z) = \frac{1}{2}(e^{2\pi i z} + e^{-2\pi i z} ),$$ 
if we write $z = x+iy$, we have that $g(z)\to 0$ uniformly as $y\to \infty$. Similarly, $f(z)\to 0$ as  We now prove that $f=g$. In a neighbourhood of every pole $z = -n\pm ia$, 
we can write $$f = \frac{1}{(z - n \pm ia)^2 +a^2} + \tilde{f}(z), g = \frac{1}{(z - n \pm ia)^2 +a^2} + \tilde{g}(z).$$ 
For some holomorphic functions $\tilde{f},\tilde{g}$. Hence $f-g$ willl be holomorphic. On any strip given by $a_1 \leq z \leq a_2$ intersected with $|y|\leq b$, we have that $f-g$ is 
holomorphic and hence bounded. For $|y|>b$, the boundary condition tells us that $f-g\to 0$ and hence is bounded. Extending to $\C$ with periodicity yields 
that $f-g$ is bounded and hence constant by Liouvilles Theorem. Finally, the boundary condition tells us that the constant is $0$ so in fact $f=g$. 

\end{document}