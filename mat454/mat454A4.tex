\documentclass[12pt, a4paper]{article}
\usepackage[lmargin =0.5 in, 
rmargin=0.5in, 
tmargin=1in,
bmargin=0.5in]{geometry}
\geometry{letterpaper}
\usepackage{amsmath}
\usepackage{amssymb}
\usepackage{blindtext}
\usepackage{titlesec}
\usepackage{enumitem}
\usepackage{fancyhdr}
\usepackage{amsthm}
\usepackage{graphicx}
\usepackage{cool}
\usepackage{thmtools}
\usepackage{hyperref}
\graphicspath{ }					%path to an image

%-------- sexy font ------------%
%\usepackage{libertine}
%\usepackage{libertinust1math}

%\usepackage{mlmodern}				% very nice and classic
%\usepackage[utopia]{mathdesign}
%\usepackage[T1]{fontenc}


\usepackage{mlmodern}
\usepackage{eulervm}
%\usepackage{tgtermes} 				%times new roman
%-------- sexy font ------------%


% Problem Styles
%====================================================================%


\newtheorem{problem}{Problem}


\theoremstyle{definition}
\newtheorem{thm}{Theorem}
\newtheorem{lemma}{Lemma}
\newtheorem{prop}{Proposition}
\newtheorem{cor}{Corollary}
\newtheorem{fact}{Fact}
\newtheorem{defn}{Definition}
\newtheorem{example}{Example}
\newtheorem{question}{Question}

\newtheorem{manualprobleminner}{Problem}

\newenvironment{manualproblem}[1]{%
	\renewcommand\themanualprobleminner{#1}%
	\manualprobleminner
}{\endmanualprobleminner}

\newcommand{\penum}{ \begin{enumerate}[label=\bf(\alph*), leftmargin=0pt]}
	\newcommand{\epenum}{ \end{enumerate} }

% Math fonts shortcuts
%====================================================================%

\newcommand{\ring}{\mathcal{R}}
\newcommand{\N}{\mathbb{N}}                           % Natural numbers
\newcommand{\Z}{\mathbb{Z}}                           % Integers
\newcommand{\R}{\mathbb{R}}                           % Real numbers
\newcommand{\C}{\mathbb{C}}                           % Complex numbers
\newcommand{\F}{\mathbb{F}}                           % Arbitrary field
\newcommand{\Q}{\mathbb{Q}}                           % Arbitrary field
\newcommand{\PP}{\mathcal{P}}                         % Partition
\newcommand{\M}{\mathcal{M}}                         % Mathcal M
\newcommand{\eL}{\mathcal{L}}                         % Mathcal L
\newcommand{\T}{\mathcal{T}}                         % Mathcal T
\newcommand{\U}{\mathcal{U}}                         % Mathcal U\\
\newcommand{\V}{\mathcal{V}}                         % Mathcal V

% symbol shortcuts
%====================================================================%

\newcommand{\lam}{\lambda}
\newcommand{\imp}{\implies}
\newcommand{\all}{\forall}
\newcommand{\exs}{\exists}
\newcommand{\delt}{\delta}
\newcommand{\eps}{\varepsilon}
\newcommand{\ra}{\rightarrow}

\newcommand{\ol}{\overline}
\newcommand{\f}{\frac}
\newcommand{\lf}{\lfrac}
\newcommand{\df}{\dfrac}

% bracketting shortcuts
%====================================================================%
\newcommand{\abs}[1]{\left| #1 \right|}
\newcommand{\babs}[1]{\Big|#1\Big|}
\newcommand{\bound}{\Big|}
\newcommand{\BB}[1]{\left(#1\right)}
\newcommand{\dd}{\mathrm{d}}
\newcommand{\artanh}{\mathrm{artanh}}
\newcommand{\Med}{\mathrm{Med}}
\newcommand{\Cov}{\mathrm{Cov}}
\newcommand{\Corr}{\mathrm{Corr}}
\newcommand{\tr}{\mathrm{tr}}
\newcommand{\Range}[1]{\mathrm{range}(#1)}
\newcommand{\Null}[1]{\mathrm{null}(#1)}
\newcommand{\lan}{\langle}
\newcommand{\ran}{\rangle}
\newcommand{\norm}[1]{\left\lVert#1\right\rVert}
\newcommand{\inn}[1]{\lan#1\ran}
\newcommand{\op}[1]{\operatorname{#1}}
\newcommand{\bmat}[1]{\begin{bmatrix}#1\end{bmatrix}}
\newcommand{\pmat}[1]{\begin{pmatrix}#1\end{pmatrix}}
\newcommand{\vmat}[1]{\begin{vmatrix}#1\end{vmatrix}}

\newcommand{\amogus}{{\bigcap}\kern-0.8em\raisebox{0.3ex}{$\subset$}}
\newcommand{\Note}{\textbf{Note: }}
\newcommand{\Aside}{{\bf Aside: }}
%restriction
%\newcommand{\op}[1]{\operatorname{#1}}
%\newcommand{\done}{$$\mathcal{QED}$$}

%====================================================================%


\setlength{\parindent}{0pt}      	% No paragraph indentations
\pagestyle{fancy}
\fancyhf{}							% fancy header

\setcounter{secnumdepth}{0}			% sections are numbered but numbers do not appear
\setcounter{tocdepth}{2} 			% no subsubsections in toc

%template
%====================================================================%
%\begin{manualproblem}{1}
%Spivak.
%\end{manualproblem}

%\begin{proof}[Solution]
%\end{proof}

%----------- or -----------%

%\begin{problem} 		
%\end{problem}	

%\penum
%	\item
%\epenum
%====================================================================%


\newcommand{\Course}{MAT454 }
\newcommand{\hwNumber}{3}

%preamble

\title{a}
\author{A.N.}
\date{\today}

\lhead{\Course A\hwNumber}
\rhead{\thepage}
%\cfoot{\thepage}


%====================================================================%
\begin{document}
	\begin{problem}
	\end{problem}
By Montels little theorem, $\{f_k\}$ is normal if and only if $\{f^\prime_k\}$ is locally bounded and at some $z_0$,  $f(z_0)$ is uniformly bounded. We see that $$f_k^\prime (z) = cos(kz).$$ Therefore $\{f_k^\prime\}$ is bounded since for $|z|<1$, $$|cos(kz) | \leq cos(k|z|)\leq 1. $$ Furthermore we have at $z_0 =0,$ $f_k(0) = \frac{\sin(kz)}{k} =0$. Therefore $\{f_k\}$ is a normal family. 
\newpage
\begin{problem}
\end{problem}
\penum
\item First note that by Harnacks Inequality, this is true for the real part of $f$, $i.e.$ we have that $$\frac{1-|z|}{1+|z|} \leq Re(f) \leq \frac{1+|z|}{1-|z|}.$$ A similar argument can be made for the imaginary part of $f$ by adjusting constant so $im(f)>0$. It follows that the inequality holds for any $f\in \mathcal{A}$. 
\item We claim that $\mathcal{A}$ is locally bounded. Take any $z\in D$. Then on any sufficiently small neighbourhood of $z$ containing $z$ we have that $|f(z)| \leq \frac{1+|z_0|}{1-|z_0|}$ at some $z_0$ in the disk, for all $f\in \mathcal{A}$. Thus $\mathcal{A}$ is locally bounded and hence normal. 
\item By Cauchys inequality, we have that for each $f_k$,  $$a^k_1 \leq r^{-1} \sup_{|z| = r} |f_k(z)|\leq r^{-1} \sup_{|z| = r} \frac{1}{2\pi} \int_{|z| = r} |f(z)|dz \leq 1$$
Since this is true for all $k$, we have that $|f_k^\prime(0)| \leq 1$. 
\epenum
\newpage
\begin{problem}
\end{problem}
\penum
\item By the Riemann Mapping Theorem, there exists a conformal $g: \Omega \to D$. For $a\in \Omega$, we define $h: D\to D$ by $h(z) = e^{i \theta}\frac{z-g(a)}{1 - \ol{g(a)} g(z)}$ for some $\theta$. We define $f = h \circ g,$ so $f =e^{i \theta} \frac{g(z) - g(a)}{1 - \ol{g(a)}g(z) } $. Notice that $f$ is a conformal mapping of $\Omega$ to $D$, with $f(a) = 0$. It remains to show that $f^\prime(a) >0$. We compute that $$f^\prime(z) = e^{i\theta}\frac{g^\prime(z) (1-\ol{g(a)} g(z) ) + \ol{g(a)} (g(z) -g(a) ) }{(1-\ol{g(a)} g(z))^2}.$$ Evaluating at $z= a$ we get $$f^\prime(z) = e^{i\theta} \frac{g^\prime(a) (1- |g(a)|^2)}{(1- |g(a)|^2)^2}.$$
This will be positive for some choice of $\theta$, so that $e^{i\theta}g^\prime(a) >0$. 
We now claim that such $f$ is unique. Suppose $f_1,f_2$ satisfy our desired properties. Then $f_2 \circ f_1^{-1} \in Aut(D)$. Furthermore, $f_2\circ f_1^{-1}(0) = f_2(a) = 0$. So by Schwartz' Lemma $f_1 = \lambda f_2$ for some $\lambda \in U(1)$. Since $f_1^\prime(a), f_2^\prime(a) >0$ we have that $\lambda = 1$. 
\item 
\begin{enumerate}[label = \roman*)]
	\item Let $\gamma \subset \Omega$ be a closed curve. There must be some minimal $N$ so that $\gamma \subset \Omega_N$. Since $\Omega_N$ is simply connected $\gamma$ can be deformed to a point in $\Omega_N$ and hence in $\Omega$. 
	\item Note that $\{f_n\}$ is a normal family, since $\{f^\prime_n\}$ is locally bounded, and $f_n(0) = 0$ for all $n$. Therefore there is a uniformly convergent subseqence $\{f_{n_k}\}$ that converges to some $f: D\to \Omega$. Note that by uniform convergence, we have that $f(0)=0$, and $f^\prime(0)>0$. By a previous result, we have that $f$ is $1-1$ as well. It follows that $f$ is a conformal mapping of $D \to \Omega$. Furthermore, it is unique by $3a$. This is true for every subsequence of $\{f_n\}$, since $\lim_{n\to \infty} \Omega_n = \Omega$. It follows that every subsequence converges uniformly to $f$ so $f_n$ converges uniformly to $f$. 
\end{enumerate}
\epenum
\newpage
\begin{problem}
\end{problem}
We identify $\C \setminus{ \{0\}}$ with $S^2 \setminus{ \{S,N\} }$, (Riemann Sphere without the poles). Therefore an automorphism of $\C \setminus{ \{0\} }$ is an automorphism of $S^2$ which either fixes the poles or reverses them i.e. $f(0) = 0, f(\infty) = \infty$ or $f(0) = \infty , f(\infty)  = 0$. We have that $f$ must be a fractional linear transformation. If $f$ fixes the poles, it must be of the form $f(z) = az$ for nonzero $a\in C$. If $f$ swaps the poles it must be of the form $f(z) = \frac{c}{z}$ for nonzero $c\in \C$. This gives a complete description of $Aut(\C \setminus{\{0\}})$.
\newpage
\begin{problem}
\end{problem}
\penum
\item By the discussion from class, a rectangle with corners at $k,-k, -k+ik^\prime, k+ik^\prime$ is the image of the Riemann mapping given by $$F(w) = \int_0^w \frac{dt}{\sqrt{(1-t^2)(1-k^2t^2)}},$$ with $F(1) = k$. Therefore to have $F(\infty)$ be a corner, it is enough to find $g\in Aut(\mathbb{H}^+)$ so $g(\infty) = 1$. Taking $$g(z) = \frac{z+1}{z+2},$$ will suffice. Then $F\circ g$ will be a conformal mapping of $\mathbb{H}^+$ onto the rectangle, with $F\circ g(\infty) = k$. 
\item By tiling with the rectangles given by $a)$, we obtain $\wp(z)$ generated by $\Gamma = \inn{4k, 2ik^\prime}$, corresponding to the elliptic curve $(\wp^\prime(z), \wp(z)) \subset \C^2$. 
\epenum
\newpage
\begin{problem}
\end{problem}
We claim the image of the map will be the attatched image. Note that each $a_i$ gets sent to $0$. We claim that on each arc between $a_i, a_{i+1}$ the argument of $f$ is constant. We have that $$\log^\prime(f(z)) = -\frac{1}{z} + \sum_{k=1}^n \frac{\lambda_k}{(z-a_k )}.$$ We claim that the imaginary part of this function is constant for $|z|=1, z = e^{i\theta}, \theta \in (arg(a_i), arg(a_{i+1}))$. Then, $$\log^\prime(f(e^{i\theta})) = - e^{-i \theta} + \sum_{k=1}^n \frac{\lambda_k}{(e^{i\theta} - a_k)} = -e^{-i \theta} + \sum_{k=1}^n \frac{\lambda_k(e^{-i\theta} - \ol{a_k})}{2- Re(e^{-i\theta } a_k)}.$$
Since the imaginary part of this is constant, we have that this will map the arcs to 0. 
\newpage
\begin{problem}
\end{problem}
Let $f$ be meromorphic, defined on $S^2$. Then the spherical derivative at $z$ is given by: 
$$f^\#(z) = \lim_{w\to z} \frac{d(f(z), f(w))}{d(z,w)},$$ where $d$ is the chordal metric on $S^2$. First assume that $z$ is not a pole of $f$. Then we have that $$f^\#(z) = \lim_{w\to z} \frac{d(f(z), f(w))}{d(z,w)} = \lim_{w\to z} \frac{\rho(f(z), f(w)) +|f(z)-f(w)|^2}{d(z,w)} = \lim_{w\to z} \frac{\rho(f(z), f(w))}{|z-w|}.$$ If $z$ is a pole, since $f^\# = \frac{1}{f^\#}$, we apply the previous computation and conclude that the desired equality holds. 
\end{document}