\documentclass[letterpaper]{article}
\usepackage[letterpaper,margin=1in,footskip=0.25in]{geometry}
\usepackage[utf8]{inputenc}
\usepackage{amsmath}
\usepackage{amsthm}
\usepackage{amssymb, pifont}
\usepackage{mathrsfs}
\usepackage{enumitem}
\usepackage{fancyhdr}
\usepackage{hyperref}

\pagestyle{fancy}
\fancyhf{}
\rhead{MAT 454}
\lhead{Assignment 2}
\rfoot{Page \thepage}

\setlength\parindent{24pt}
\renewcommand\qedsymbol{$\blacksquare$}

\DeclareMathOperator{\Qu}{\mathcal{Q}_8}
\DeclareMathOperator{\F}{\mathbb{F}}
\DeclareMathOperator{\T}{\mathcal{T}}
\DeclareMathOperator{\V}{\mathcal{V}}
\DeclareMathOperator{\U}{\mathcal{U}}
\DeclareMathOperator{\Prt}{\mathbb{P}}
\DeclareMathOperator{\R}{\mathbb{R}}
\DeclareMathOperator{\N}{\mathbb{N}}
\DeclareMathOperator{\Z}{\mathbb{Z}}
\DeclareMathOperator{\Q}{\mathbb{Q}}
\DeclareMathOperator{\C}{\mathbb{C}}
\DeclareMathOperator{\ep}{\varepsilon}
\DeclareMathOperator{\identity}{\mathbf{0}}
\DeclareMathOperator{\card}{card}
\newcommand{\suchthat}{;\ifnum\currentgrouptype=16 \middle\fi|;}

\newtheorem{lemma}{Lemma}

\newcommand{\tr}{\mathrm{tr}}
\newcommand{\ra}{\rightarrow}
\newcommand{\lan}{\langle}
\newcommand{\ran}{\rangle}
\newcommand{\norm}[1]{\left\lVert#1\right\rVert}
\newcommand{\inn}[1]{\lan#1\ran}
\newcommand{\ol}{\overline}
\newcommand{\ci}{i}
\begin{document} \noindent  Q5a: 
We define $$f(z) = \sum_{n= -\infty}^\infty \frac{(-1)^n}{(z-n)^2}$$ and 
$$g(z) = \frac{\pi^2}{(\sin\pi z)(\tan \pi z)}.$$
Observe that both $f$ and $g$ have simple removable poles of order 2 on the integers, which we know from the power series expansion of 
$\sin$ and $\tan$. We also know that both $f$ and $g$ are periodic, with a period of $1$. For all poles 
$n$, we can write $$f(z) = \frac{(-1)^n}{(z-n)^2} + \tilde{f}(z)$$ for a holomorphic $\tilde{f}$ in some neighbourhood of $n$. 
Similarly, using the laurent series expansion of $g$, we can write $$g(z) = \frac{(-1)^n}{(z-n)^2} + \tilde{g}(z)$$ for some holomorphic $\tilde{g}(z)$. 
Similarly to $Q4$, if we write $z=x+iy$ we have that $f(z),g(z) \to 0 $ as $|y| \to \infty$. 
Thus on any strip $a_1 \leq x \leq a_2$, for $|y|\leq b$, the holomorphic function $f-g$ will attain a maximum. The limit behaviour tells us 
that for $|y|>b$, $f-g$ is bounded as well. Therefore for $a_1 \leq x \leq a_2$, $f-g$ is bounded. 
Extending by periodicity tells us that $f-g$ is bounded and hence constant by Liouvilles theorem. The limit behaviour tells us that $f-g=0$. As desired. 
\newline \\ Q5b: Let $$f(z) = \frac{1}{z} + \sum_{n\geq 1}^\infty (-1)^n \frac{2z}{(z^2-n^2)} = \frac{1}{z} + \sum_{n\geq 1}^\infty (-1)^n \Big[ \frac{1}{(z-n)} + \frac{1}{(z+n)} \Big].$$ 
This is a series of meromorphic functions, which is normally convergent so we can take the derivative term by term to get 
$$f^\prime(z) = -\frac{1}{z^2} + \sum_{n\geq 1} (-1)^n \Big[ \frac{1}{(z-n)^2}  + \frac{1}{(z+n)^2} \Big] =  \sum_{n= -\infty}^\infty \frac{(-1)^n}{(z-n)^2} = \frac{\pi^2}{(\sin\pi z)(\tan\pi z)} = \frac{d}{dz}\Big[ \frac{\pi}{\sin \pi z} \Big].$$
Therefore $f(z)$ and $\frac{\pi}{\sin \pi z}$ differ by a constant. Since they are both odd they must be equal. 

\end{document}