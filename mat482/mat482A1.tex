\documentclass[12pt, a4paper]{article}
\usepackage[lmargin =0.5 in, 
rmargin=0.5in, 
tmargin=1in,
bmargin=0.5in]{geometry}
\geometry{letterpaper}
\usepackage{tikz-cd}
\usepackage{amsmath}
\usepackage{amssymb}
\usepackage{blindtext}
\usepackage{titlesec}
\usepackage{enumitem}
\usepackage{fancyhdr}
\usepackage{amsthm}
\usepackage{graphicx}
\usepackage{cool}
\usepackage{thmtools}
\usepackage{hyperref}
\graphicspath{ }					%path to an image

%-------- sexy font ------------%
%\usepackage{libertine}
%\usepackage{libertinust1math}

%\usepackage{mlmodern}				% very nice and classic
%\usepackage[utopia]{mathdesign}
%\usepackage[T1]{fontenc}


\usepackage{mlmodern}
\usepackage{eulervm}
%\usepackage{tgtermes} 				%times new roman
%-------- sexy font ------------%


% Problem Styles
%====================================================================%


\newtheorem{problem}{Problem}


\theoremstyle{definition}
\newtheorem{thm}{Theorem}
\newtheorem{lemma}{Lemma}
\newtheorem{prop}{Proposition}
\newtheorem{cor}{Corollary}
\newtheorem{fact}{Fact}
\newtheorem{defn}{Definition}
\newtheorem{example}{Example}
\newtheorem{question}{Question}

\newtheorem{manualprobleminner}{Problem}

\newenvironment{manualproblem}[1]{%
	\renewcommand\themanualprobleminner{#1}%
	\manualprobleminner
}{\endmanualprobleminner}

\newcommand{\penum}{ \begin{enumerate}[label=\bf(\alph*), leftmargin=0pt]}
	\newcommand{\epenum}{ \end{enumerate} }

% Math fonts shortcuts
%====================================================================%

\newcommand{\ring}{\mathcal{R}}
\newcommand{\N}{\mathbb{N}}                           % Natural numbers
\newcommand{\Z}{\mathbb{Z}}                           % Integers
\newcommand{\R}{\mathbb{R}}                           % Real numbers
\newcommand{\C}{\mathbb{C}}                           % Complex numbers
\newcommand{\F}{\mathbb{F}}                           % Arbitrary field
\newcommand{\Q}{\mathbb{Q}}                           % Arbitrary field
\newcommand{\PP}{\mathcal{P}}                         % Partition
\newcommand{\M}{\mathcal{M}}                         % Mathcal M
\newcommand{\eL}{\mathcal{L}}                         % Mathcal L
\newcommand{\T}{\mathbb{T}}                         % Mathcal T
\newcommand{\U}{\mathcal{U}}                         % Mathcal U\\
\newcommand{\V}{\mathcal{V}}                         % Mathcal V

% symbol shortcuts
%====================================================================%

\newcommand{\bd}{\partial}
\newcommand{\grad}{\nabla}
\newcommand{\lam}{\lambda}
\newcommand{\imp}{\implies}
\newcommand{\all}{\forall}
\newcommand{\exs}{\exists}
\newcommand{\delt}{\delta}
\newcommand{\ep}{\varepsilon}
\newcommand{\ra}{\rightarrow}
\newcommand{\vph}{\varphi}

\newcommand{\ol}{\overline}
\newcommand{\f}{\frac}
\newcommand{\lf}{\lfrac}
\newcommand{\df}{\dfrac}

% bracketting shortcuts
%====================================================================%
\newcommand{\abs}[1]{\left| #1 \right|}
\newcommand{\babs}[1]{\Big|#1\Big|}
\newcommand{\bound}{\Big|}
\newcommand{\BB}[1]{\left(#1\right)}
\newcommand{\dd}{\mathrm{d}}
\newcommand{\artanh}{\mathrm{artanh}}
\newcommand{\Med}{\mathrm{Med}}
\newcommand{\Cov}{\mathrm{Cov}}
\newcommand{\Corr}{\mathrm{Corr}}
\newcommand{\tr}{\mathrm{tr}}
\newcommand{\Range}[1]{\mathrm{range}(#1)}
\newcommand{\Null}[1]{\mathrm{null}(#1)}
\newcommand{\lan}{\langle}
\newcommand{\ran}{\rangle}
\newcommand{\norm}[1]{\left\lVert#1\right\rVert}
\newcommand{\inn}[1]{\lan#1\ran}
\newcommand{\op}[1]{\operatorname{#1}}
\newcommand{\bmat}[1]{\begin{bmatrix}#1\end{bmatrix}}
\newcommand{\pmat}[1]{\begin{pmatrix}#1\end{pmatrix}}
\newcommand{\vmat}[1]{\begin{vmatrix}#1\end{vmatrix}}

\newcommand{\amogus}{{\bigcap}\kern-0.8em\raisebox{0.3ex}{$\subset$}}
\newcommand{\Note}{\textbf{Note: }}
\newcommand{\Aside}{{\bf Aside: }}
%restriction
%\newcommand{\op}[1]{\operatorname{#1}}
%\newcommand{\done}{$$\mathcal{QED}$$}

%====================================================================%


\setlength{\parindent}{0pt}      	% No paragraph indentations
\pagestyle{fancy}
\fancyhf{}							% fancy header

\setcounter{secnumdepth}{0}			% sections are numbered but numbers do not appear
\setcounter{tocdepth}{2} 			% no subsubsections in toc

%template
%====================================================================%
%\begin{manualproblem}{1}
%Spivak.
%\end{manualproblem}

%\begin{proof}[Solution]
%\end{proof}

%----------- or -----------%

%\begin{problem} 		
%\end{problem}	

%\penum
%	\item
%\epenum
%====================================================================%


\newcommand{\Course}{MAT482}
\newcommand{\hwNumber}{1}

%preamble

\title{Elliptic PDE's HW1}
\author{Alexander Neagoe }
\date{\today}
\lhead{\Course A\hwNumber}
\rhead{\thepage}
%\cfoot{\thepage}


%====================================================================%
\begin{document}

\maketitle

\begin{problem}
\end{problem}
If $u$ is constant then the result is true for $C=1$. Suppose now that $u$ is not constant. Consider the case where $\Omega  = B_1(0)$ and $\Omega^\prime = B_r(0)$, where $r<1$.  
\newpage
\begin{problem}
\end{problem}
\newpage
\begin{problem}
\end{problem}
Recall the the product rule for the laplacian: 
$$\Delta(uv) = v \Delta u + 2\grad u \cdot \grad v + u \Delta v,$$
And Green's first identity:
$$\int_{B_r(0)} \grad \psi \cdot \grad \varphi = \oint_{\bd B_r(0)} \psi \grad \varphi - \int_{B_r(0)} \psi \Delta \varphi.$$
We integrate the product rule for $v= \frac{1}{|x|^{n-2}}$, and apply Green's first identity to get:
$$\int_{B_r(0)}\Delta \left( \frac{u}{|x|^{n-2}} \right) = \int_{B_r(0)} \frac{\Delta u}{|x|^{n-2}}  + 2 \left( \oint_{\bd B_r(0)} u \cdot \grad \frac{1}{|x|^{n-2}} - \int_{B_r(0)}u \cdot \Delta \frac{1}{|x|^{n-2}}\right) + \int_{B_r(0)} u \cdot \Delta \frac{1}{|x|^{n-2}}.$$
We apply the Divergence theorem to the lefthand side: 
$$\int_{B_r(0)}\Delta \left( \frac{u}{|x|^{n-2}} \right) = \oint_{\bd B_r(0)} \grad\left(\frac{u}{|x|^{n-2}}\right) = \oint_{\bd B_r(0)} \frac{1}{|x|^{n-2}} \grad u + \oint_{\bd B_r(0)}u \grad \frac{1}{|x|^{n-2}}.$$
We have that:
$$\grad \frac{1}{|x|^{n-2}} = \frac{(2-n)}{|x|^{n}} x.$$
Therefore we can write the lefthand side as: 
$$\oint_{\bd B_r(0)} \frac{1}{|x|^{n-2}} \grad u  + \oint_{\bd B_r(0)} g(x) \frac{(2-n)}{|x|^n}x \cdot x d\sigma(x) =\oint_{\bd B_r(0)}\frac{1}{|x|^{n-2}} \grad u + \frac{(2-n)}{|r|^{n-1}} \oint_{\bd B_r(0)} g(x) d\sigma(x).$$
Now on the righthand side, we have:
$$\int_{B_r(0)} \frac{f}{|x|^{n-2}} + 2 \frac{(2-n)}{|r|^{n-1}} \oint_{\bd B_r(0)} g(x) d\sigma(x) - \int_{B_r(0)} u \cdot \Delta \frac{1}{|x|^{n-2}} .$$
Equality of both of these gives us that:
$$\int_{B_r(0)} u \cdot \Delta \frac{1}{|x|^{n-2}} = \frac{(2-n)}{|r|^{n-1}} \oint_{\bd B_r(0)} g(x) d\sigma(x) + \int_{B_r(0)} \frac{f}{|x|^{n-2}} - \oint_{\bd B_r(0)} \grad u \cdot \frac{1}{|x|^{n-2}}.$$
By the Divergence theorem we can simplify the last integral as:
$$\oint_{\bd B_r(0)} \grad u \cdot \frac{1}{|x|^{n-2}} = \oint_{\bd B_r(0)} \frac{\grad u}{r^{n-2}} = \int_{B_r(0)} \frac{f}{|r|^{n-2}} $$
Since $\Delta \frac{1}{|x|^{n-2}}= \omega_n n(2-n)\delta_0$ we get that:
$$\omega_n n(2-n) u(0) = \frac{(2-n)}{r^{n-1}} \oint_{\bd B_r(0)}g(x)d\sigma(x) + \int_{B_r(0)} \left( \frac{1}{|x|^{n-2}} - \frac{1}{r^{n-2}}\right) f dx  $$
\newpage
\begin{problem}
\end{problem}
\newpage
\begin{problem}
\end{problem}
We first claim that $D(\eta u)$ exists. For any $\varphi \in C_c^\infty(\Omega)$, the following is true: 
$$\int u \cdot D(\varphi \cdot \eta) = - \int \varphi \cdot \eta \cdot Du.$$
Since on smooth functions, the weak derivative agrees with the usual derivative, we have that $$\int u \cdot D(\varphi \cdot \eta) = \int u \left(\eta D\varphi + \varphi D \eta\right).$$
Rearranging the first equation we get that 
$$\int \eta u \cdot D\varphi = - \int\varphi \left( \eta Du + u D \eta \right).$$
This is exactly the definition of $D(\eta u)$. We now claim that $\eta u \in W^{1,p}(\Omega)$.
The weak derivative exists by the above computation. It remains to show that it belongs to $L^p(\Omega)$. First we have that $\eta u$ belongs to $L^p(\Omega)$ for all $\eta \in C_c^\infty(\Omega)$, since $\eta$ is bounded above by some constant and $u\in L^p(\Omega)$. We have that $\eta Du, u D\eta\in L^p(\Omega)$ for the same reason. Their sum also belongs to $L^p(\Omega)$, so $D(\eta u)\in L^p(\Omega)$. 
\newpage
\begin{problem}
\end{problem}
\newpage
\begin{problem}
\end{problem}
\newpage
\begin{problem}
\end{problem}
\newpage
\begin{problem}
\end{problem}
\newpage
\begin{problem}
\end{problem}
\penum
\item 
Suppose that $\{x_n\}$ converges to some $v \in \mathcal{H}$. Then we must have that $\norm{x_n - v}^2\to 0$ as $n \to \infty$. This is the same as saying that $\inn{x_n - v, x_n-v} \to 0$. Using bilinearity, we get:
$$\inn{x_n - v, x_n-v}  = \norm{x_n}^2 - 2 Re(\inn{x_n,v}) + \norm{v}^2. $$ Since $\norm{v} <\infty$ and $\norm{v}^2 = \sum_{n=1}^\infty |\inn{x_n,v}|^2$ we must have that $Re(\inn{x_n,v}) \to 0$. Since $\{x_n\}$ is orthonormal, we have that 
$$\norm{x_n}^2 - 2Re(\inn{x_n,v}) + \norm{v}^2 = 1 - 2 Re(\inn{x_n,v}) + \norm{v}^2 \to 1 + \norm{v}^2 \geq 1.$$
Thus this sequence cannot converge. 
\item For any vector $v\in \mathcal{H}$ we have that $v = \sum_{i=1}^\infty \inn{x_n,v}x_n$ and $\norm{v}^2 = \sum_{i=1}^\infty |\inn{x_n,v}|^2$. Therefore we must have that $\inn{x_n,v} \to 0$ or $\inn{x_n,v} \to \inn{0,v}$. Thus $\{x_n\}$ converges weakly to $0$. 

\epenum

\end{document}