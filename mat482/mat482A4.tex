\documentclass[12pt, a4paper]{article}
\usepackage[lmargin =0.5 in, 
rmargin=0.5in, 
tmargin=1in,
bmargin=0.5in]{geometry}
\geometry{letterpaper}
\usepackage{tikz-cd}
\usepackage{amsmath}
\usepackage{amssymb}
\usepackage{blindtext}
\usepackage{titlesec}
\usepackage{enumitem}
\usepackage{fancyhdr}
\usepackage{amsthm}
\usepackage{graphicx}
\usepackage{cool}
\usepackage{thmtools}
\usepackage{hyperref}
\graphicspath{ }					%path to an image

%-------- sexy font ------------%
%\usepackage{libertine}
%\usepackage{libertinust1math}

%\usepackage{mlmodern}				% very nice and classic
%\usepackage[utopia]{mathdesign}
%\usepackage[T1]{fontenc}


\usepackage{mlmodern}
\usepackage{eulervm}
%\usepackage{tgtermes} 				%times new roman
%-------- sexy font ------------%


% Problem Styles
%====================================================================%


\newtheorem{problem}{Problem}


\theoremstyle{definition}
\newtheorem{thm}{Theorem}
\newtheorem{lemma}{Lemma}
\newtheorem{prop}{Proposition}
\newtheorem{cor}{Corollary}
\newtheorem{fact}{Fact}
\newtheorem{defn}{Definition}
\newtheorem{example}{Example}
\newtheorem{question}{Question}

\newtheorem{manualprobleminner}{Problem}

\newenvironment{manualproblem}[1]{%
	\renewcommand\themanualprobleminner{#1}%
	\manualprobleminner
}{\endmanualprobleminner}

\newcommand{\penum}{ \begin{enumerate}[label=\bf(\alph*), leftmargin=0pt]}
	\newcommand{\epenum}{ \end{enumerate} }

% Math fonts shortcuts
%====================================================================%

\newcommand{\ring}{\mathcal{R}}
\newcommand{\N}{\mathbb{N}}                           % Natural numbers
\newcommand{\Z}{\mathbb{Z}}                           % Integers
\newcommand{\R}{\mathbb{R}}                           % Real numbers
\newcommand{\C}{\mathbb{C}}                           % Complex numbers
\newcommand{\F}{\mathbb{F}}                           % Arbitrary field
\newcommand{\Q}{\mathbb{Q}}                           % Arbitrary field
\newcommand{\PP}{\mathcal{P}}                         % Partition
\newcommand{\M}{\mathcal{M}}                         % Mathcal M
\newcommand{\eL}{\mathcal{L}}                         % Mathcal L
\newcommand{\T}{\mathbb{T}}                         % Mathcal T
\newcommand{\U}{\mathcal{U}}                         % Mathcal U\\
\newcommand{\V}{\mathcal{V}}                         % Mathcal V

% symbol shortcuts
%====================================================================%

\newcommand{\bd}{\partial}
\newcommand{\grad}{\nabla}
\newcommand{\lam}{\lambda}
\newcommand{\imp}{\implies}
\newcommand{\all}{\forall}
\newcommand{\exs}{\exists}
\newcommand{\delt}{\delta}
\newcommand{\ep}{\varepsilon}
\newcommand{\ra}{\rightarrow}
\newcommand{\vph}{\varphi}

\newcommand{\ol}{\overline}
\newcommand{\f}{\frac}
\newcommand{\lf}{\lfrac}
\newcommand{\df}{\dfrac}

% bracketting shortcuts
%====================================================================%
\newcommand{\abs}[1]{\left| #1 \right|}
\newcommand{\babs}[1]{\Big|#1\Big|}
\newcommand{\bound}{\Big|}
\newcommand{\BB}[1]{\left(#1\right)}
\newcommand{\dd}{\mathrm{d}}
\newcommand{\artanh}{\mathrm{artanh}}
\newcommand{\Med}{\mathrm{Med}}
\newcommand{\Cov}{\mathrm{Cov}}
\newcommand{\Corr}{\mathrm{Corr}}
\newcommand{\tr}{\mathrm{tr}}
\newcommand{\Range}[1]{\mathrm{range}(#1)}
\newcommand{\Null}[1]{\mathrm{null}(#1)}
\newcommand{\lan}{\left\langle}
\newcommand{\ran}{\right\rangle}
\newcommand{\norm}[1]{\left\lVert#1\right\rVert}
\newcommand{\inn}[1]{\lan#1\ran}
\newcommand{\op}[1]{\operatorname{#1}}
\newcommand{\bmat}[1]{\begin{bmatrix}#1\end{bmatrix}}
\newcommand{\pmat}[1]{\begin{pmatrix}#1\end{pmatrix}}
\newcommand{\vmat}[1]{\begin{vmatrix}#1\end{vmatrix}}

\newcommand{\amogus}{{\bigcap}\kern-0.8em\raisebox{0.3ex}{$\subset$}}
\newcommand{\Note}{\textbf{Note: }}
\newcommand{\Aside}{{\bf Aside: }}
%restriction
%\newcommand{\op}[1]{\operatorname{#1}}
%\newcommand{\done}{$$\mathcal{QED}$$}

%====================================================================%


\setlength{\parindent}{0pt}      	% No paragraph indentations
\pagestyle{fancy}
\fancyhf{}							% fancy header

\setcounter{secnumdepth}{0}			% sections are numbered but numbers do not appear
\setcounter{tocdepth}{2} 			% no subsubsections in toc

%template
%====================================================================%
%\begin{manualproblem}{1}
%Spivak.
%\end{manualproblem}

%\begin{proof}[Solution]
%\end{proof}

%----------- or -----------%

%\begin{problem} 		
%\end{problem}	

%\penum
%	\item
%\epenum
%====================================================================%


\newcommand{\Course}{MAT482}
\newcommand{\hwNumber}{4}

%preamble

\title{MAT482 HW4}
\author{A.N.}
\date{\today}
\lhead{\Course A\hwNumber}
\rhead{\thepage}
%\cfoot{\thepage}


%====================================================================%
\begin{document}

\maketitle

\begin{problem}
% problem number 1
\end{problem}
We first claim that $B_{ \frac{\ep}{2} } (0) \subset Du \left(\Gamma^+ \cap \left\{ u < u(0)+ \frac{\ep}{2}  \right\} \right)$.
Take $|\xi| < \frac{\ep}{2}$. We define the affine linear mapping $l(x) = \xi \cdot x  + u(0)$.
Note that on $\bd B_1$, we have that
$$l(x) \leq |\xi| \cdot |x| + u(0) < u(x).$$
Therefore we have that $ \left\{ l>u \right\} \subset \subset B_1$. Similarly to $ABP$, we have the existence of an $a>0$ so that $l - a$ touches $u$ at some $x_0$, and $l-a \leq u$ elsewhere, so that $x\in \Gamma^+$. Furthermore, 
$$ u(x_0) = \xi \cdot x_0 + u(0) - a \leq |\xi | \cdot |x_0| + u(0) < \frac{\ep}{2} + u(0). $$ 
Therefore $x_0 \in \Gamma^+ \cap \left\{ u< u(0) + \frac{\ep}{2} \right\}$ and $l-a$ is tangent to $u$ at $x_0$, so $\xi = Du(x_0)$ i.e. $\xi \in D \left( \Gamma^+ \cap \left\{ u< u(0) + \frac{\ep}{2} \right\} \right)$. 
We apply the change of variables theorem for $g = 1$, and get that 
$$ \omega_n \left( \frac{\ep}{2} \right)^n = \int_{B_\frac{\ep}{2}} 1 \leq \int_{ Du\left(\Gamma^+ \cap \left\{ u< u(0) + \frac{\ep}{2} \right\} \right)} 1 dy \leq \int_{\Gamma^+ \cap \left\{ u< u(0) + \frac{\ep}{2} \right\}} \det D^2 u. $$ 
Dividing and taking the $n-$th root, we see
$$\ep \leq \frac{2}{\sqrt[n]{\omega_n}}\int_{\Gamma^+ \cap \left\{ u< u(0) + \frac{\ep}{2} \right\}} \det D^2 u. $$
\newpage
\begin{problem}
% problem number 2
\end{problem}
\penum
\item Suppose that $| \left\{ |u-u_Q|>kC(n) \right\}\cap Q| \leq 2^{-k} |Q| $. Then for some choice of $\delta(n)$, we have that 
\begin{align*}
	\int_Q e^{\delta(n) |u-u_Q|} & = \delta(n) \int_0^\infty e^{\delta t} \left| \left\{ \left| u-u_Q \right|>t \right\}\cap Q \right|dt 
	\\ & = \delta(n) \sum_{k=0}^\infty \int_{kC}^{(k+1)C} e^{\delta t} \left| \left\{ \left| u-u_Q \right|>t \right\}\cap Q \right|dt 
	\\ & \leq \delta(n) \sum_{k=0}^\infty  \int_{kc}^{k(c+1)} 2^{-k} |Q| e^{\delta t} dt
	\\ & = \delta |Q| \sum_{k=0}^\infty 2^{-k} \int_{kC}^{C(k+1)} e^{\delta t} dt
	\\ & = \delta |Q| (e^{\delta C} - 1) \sum_{k=0}^\infty 2^{-k} e^{\delta kc}
	\\ & = |Q| \frac{2(e^{\delta C} - 1)}{2- e^{\delta C}}
\end{align*}
Dividing by $|Q|$, and setting $\frac{2(e^{\delta C} - 1)}{2- e^{\delta C}}= C$, we see that choosing $\delta = \frac{1}{c}\log \left( \frac{2(1+C)}{2+C} \right)$ works. Since $C$ depends on $n$ so will $\delta$. 
\item It is sufficient to show that $ \left\{ \left| u-u_Q \right|>2 \right\}\cap Q \subset \bigcup \tilde{Q}$, up to a set of measure 0. Note that by applying the search algorithm we get that if $x_i \in  \left\{ \left| u-u_Q \right|>2 \right\}\cap Q $ then $x_0$ must belong to some $\tilde{Q}$. It remains to show that the set of $x$ in $ \left\{ \left| u-u_Q \right|>2 \right\}\cap Q $ but not in $\bigcup \tilde{Q}$ is measure $0$. 
Call this set $N$. If $x_0 \in N$ then we can produce a sequence of cubes shrinking down to it, say $ \left\{ Q_i \right\}$. By the lebesque differentiation theorem, we have that 
$$ \lim_{n\to \infty} \frac{1}{|Q_n|}\int_{Q_i} |u- u_Q| dx = u(y). $$
This must be greater than $2$ except on a null set. This contradicts the assumption that $x_0\not \in \tilde{Q}$ for any $\tilde{Q}$. 
\item We have that 
$$ 2 \leq \frac{1}{|\tilde{Q}|}\int_{\tilde{Q}} |u-u_Q|. $$ 
Therefore 
$$ |\tilde{Q}| \leq \frac{1}{2}\int_{\tilde{Q}} |u-u_Q| $$ 
Summing over each $\tilde{Q}$, we get that 
$$\sum_{\tilde{Q}} |\tilde{Q}| \leq \frac{1}{2}\sum\int_{\tilde{Q}} |u-u_Q| \leq \frac{1}{2}\int_{Q} |u-u_Q| \leq \frac{1}{2}|Q|.$$
Applying this to part $b)$ we see that 
$$ \left| \left\{ \left| u-u_Q \right|>2 \right\}\cap Q \right|\leq \frac{1}{2}|Q|.$$
\item Let $\tilde{Q}^\prime$ be the predecessor of $\tilde{Q}$. Then,
	\begin{align*}
		|u_Q - u_{\tilde{Q}}| & = \left| \frac{1}{|\tilde{Q}|}\int_{\tilde{Q}} u-u_Q \right|
		\\ & \leq \frac{1}{|\tilde{Q}|} \int_{|\tilde{Q}^\prime|} |u-u_Q|dy
		\\ & \leq \frac{|\tilde{Q}^\prime|}{|\tilde{Q}|}\cdot \frac{1}{|\tilde{Q}^\prime|} \int_{|\tilde{Q}^\prime|} |u-u_Q|dy
		\\ & \leq 2\cdot 2^{n+1}.
	\end{align*}
\item 
\item 
\epenum
\newpage
\begin{problem}
% problem number 3
\end{problem}
\newpage


\end{document}
