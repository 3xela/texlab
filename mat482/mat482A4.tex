\documentclass[12pt, a4paper]{article}
\usepackage[lmargin =0.5 in, 
rmargin=0.5in, 
tmargin=1in,
bmargin=0.5in]{geometry}
\geometry{letterpaper}
\usepackage{tikz-cd}
\usepackage{amsmath}
\usepackage{amssymb}
\usepackage{blindtext}
\usepackage{titlesec}
\usepackage{enumitem}
\usepackage{fancyhdr}
\usepackage{amsthm}
\usepackage{graphicx}
\usepackage{cool}
\usepackage{thmtools}
\usepackage{hyperref}
\graphicspath{ }					%path to an image

%-------- sexy font ------------%
%\usepackage{libertine}
%\usepackage{libertinust1math}

%\usepackage{mlmodern}				% very nice and classic
%\usepackage[utopia]{mathdesign}
%\usepackage[T1]{fontenc}


\usepackage{mlmodern}
\usepackage{eulervm}
%\usepackage{tgtermes} 				%times new roman
%-------- sexy font ------------%


% Problem Styles
%====================================================================%


\newtheorem{problem}{Problem}


\theoremstyle{definition}
\newtheorem{thm}{Theorem}
\newtheorem{lemma}{Lemma}
\newtheorem{prop}{Proposition}
\newtheorem{cor}{Corollary}
\newtheorem{fact}{Fact}
\newtheorem{defn}{Definition}
\newtheorem{example}{Example}
\newtheorem{question}{Question}

\newtheorem{manualprobleminner}{Problem}

\newenvironment{manualproblem}[1]{%
	\renewcommand\themanualprobleminner{#1}%
	\manualprobleminner
}{\endmanualprobleminner}

\newcommand{\penum}{ \begin{enumerate}[label=\bf(\alph*), leftmargin=0pt]}
	\newcommand{\epenum}{ \end{enumerate} }

% Math fonts shortcuts
%====================================================================%

\newcommand{\ring}{\mathcal{R}}
\newcommand{\N}{\mathbb{N}}                           % Natural numbers
\newcommand{\Z}{\mathbb{Z}}                           % Integers
\newcommand{\R}{\mathbb{R}}                           % Real numbers
\newcommand{\C}{\mathbb{C}}                           % Complex numbers
\newcommand{\F}{\mathbb{F}}                           % Arbitrary field
\newcommand{\Q}{\mathbb{Q}}                           % Arbitrary field
\newcommand{\PP}{\mathcal{P}}                         % Partition
\newcommand{\M}{\mathcal{M}}                         % Mathcal M
\newcommand{\eL}{\mathcal{L}}                         % Mathcal L
\newcommand{\T}{\mathbb{T}}                         % Mathcal T
\newcommand{\U}{\mathcal{U}}                         % Mathcal U\\
\newcommand{\V}{\mathcal{V}}                         % Mathcal V

% symbol shortcuts
%====================================================================%

\newcommand{\bd}{\partial}
\newcommand{\grad}{\nabla}
\newcommand{\lam}{\lambda}
\newcommand{\imp}{\implies}
\newcommand{\all}{\forall}
\newcommand{\exs}{\exists}
\newcommand{\delt}{\delta}
\newcommand{\ep}{\varepsilon}
\newcommand{\ra}{\rightarrow}
\newcommand{\vph}{\varphi}

\newcommand{\ol}{\overline}
\newcommand{\f}{\frac}
\newcommand{\lf}{\lfrac}
\newcommand{\df}{\dfrac}

% bracketting shortcuts
%====================================================================%
\newcommand{\abs}[1]{\left| #1 \right|}
\newcommand{\babs}[1]{\Big|#1\Big|}
\newcommand{\bound}{\Big|}
\newcommand{\BB}[1]{\left(#1\right)}
\newcommand{\dd}{\mathrm{d}}
\newcommand{\artanh}{\mathrm{artanh}}
\newcommand{\Med}{\mathrm{Med}}
\newcommand{\Cov}{\mathrm{Cov}}
\newcommand{\Corr}{\mathrm{Corr}}
\newcommand{\tr}{\mathrm{tr}}
\newcommand{\Range}[1]{\mathrm{range}(#1)}
\newcommand{\Null}[1]{\mathrm{null}(#1)}
\newcommand{\lan}{\left\langle}
\newcommand{\ran}{\right\rangle}
\newcommand{\norm}[1]{\left\lVert#1\right\rVert}
\newcommand{\inn}[1]{\lan#1\ran}
\newcommand{\op}[1]{\operatorname{#1}}
\newcommand{\bmat}[1]{\begin{bmatrix}#1\end{bmatrix}}
\newcommand{\pmat}[1]{\begin{pmatrix}#1\end{pmatrix}}
\newcommand{\vmat}[1]{\begin{vmatrix}#1\end{vmatrix}}

\newcommand{\amogus}{{\bigcap}\kern-0.8em\raisebox{0.3ex}{$\subset$}}
\newcommand{\Note}{\textbf{Note: }}
\newcommand{\Aside}{{\bf Aside: }}
%restriction
%\newcommand{\op}[1]{\operatorname{#1}}
%\newcommand{\done}{$$\mathcal{QED}$$}

%====================================================================%


\setlength{\parindent}{0pt}      	% No paragraph indentations
\pagestyle{fancy}
\fancyhf{}							% fancy header

\setcounter{secnumdepth}{0}			% sections are numbered but numbers do not appear
\setcounter{tocdepth}{2} 			% no subsubsections in toc

%template
%====================================================================%
%\begin{manualproblem}{1}
%Spivak.
%\end{manualproblem}

%\begin{proof}[Solution]
%\end{proof}

%----------- or -----------%

%\begin{problem} 		
%\end{problem}	

%\penum
%	\item
%\epenum
%====================================================================%


\newcommand{\Course}{MAT482}
\newcommand{\hwNumber}{4}

%preamble

\title{MAT482 HW4}
\author{A.N.}
\date{\today}
\lhead{\Course A\hwNumber}
\rhead{\thepage}
%\cfoot{\thepage}


%====================================================================%
\begin{document}

\maketitle

\begin{problem}
% problem number 1
\end{problem}
We first claim that $B_{ \frac{\ep}{2} } (0) \subset Du \left(\Gamma^+ \cap \left\{ u < u(0)+ \frac{\ep}{2}  \right\} \right)$.
Take $|\xi| < \frac{\ep}{2}$. We define the affine linear mapping $l(x) = \xi \cdot x  + u(0)$.
Note that on $\bd B_1$, we have that
$$l(x) \leq |\xi| \cdot |x| + u(0) < u(x).$$
Therefore we have that $ \left\{ l>u \right\} \subset \subset B_1$. Similarly to $ABP$, we have the existence of an $a>0$ so that $l - a$ touches $u$ at some $x_0$, and $l-a \leq u$ elsewhere, so that $x\in \Gamma^+$. Furthermore, 
$$ u(x_0) = \xi \cdot x_0 + u(0) - a \leq |\xi | \cdot |x_0| + u(0) < \frac{\ep}{2} + u(0). $$ 
Therefore $x_0 \in \Gamma^+ \cap \left\{ u< u(0) + \frac{\ep}{2} \right\}$ and $l-a$ is tangent to $u$ at $x_0$, so $\xi = Du(x_0)$ i.e. $\xi \in D \left( \Gamma^+ \cap \left\{ u< u(0) + \frac{\ep}{2} \right\} \right)$. 
We apply the change of variables theorem for $g = 1$, and get that 
$$ \omega_n \left( \frac{\ep}{2} \right)^n = \int_{B_\frac{\ep}{2}} 1 \leq \int_{ Du\left(\Gamma^+ \cap \left\{ u< u(0) + \frac{\ep}{2} \right\} \right)} 1 dy \leq \int_{\Gamma^+ \cap \left\{ u< u(0) + \frac{\ep}{2} \right\}} \det D^2 u. $$ 
Dividing and taking the $n-$th root, we see
$$\ep \leq \frac{2}{\sqrt[n]{\omega_n}}\int_{\Gamma^+ \cap \left\{ u< u(0) + \frac{\ep}{2} \right\}} \det D^2 u. $$
\newpage
\begin{problem}
% problem number 2
\end{problem}
\penum
\item Suppose that $| \left\{ |u-u_Q|>kC(n) \right\}\cap Q| \leq 2^{-k} |Q| $. Then for some choice of $\delta(n)$, we have that 
\begin{align*}
	\int_Q e^{\delta(n) |u-u_Q|} & = \delta(n) \int_0^\infty e^{\delta t} \left| \left\{ \left| u-u_Q \right|>t \right\}\cap Q \right|dt 
	\\ & = \delta(n) \sum_{k=0}^\infty \int_{kC}^{(k+1)C} e^{\delta t} \left| \left\{ \left| u-u_Q \right|>t \right\}\cap Q \right|dt 
	\\ & \leq \delta(n) \sum_{k=0}^\infty  \int_{kc}^{k(c+1)} 2^{-k} |Q| e^{\delta t} dt
	\\ & = \delta |Q| \sum_{k=0}^\infty 2^{-k} \int_{kC}^{C(k+1)} e^{\delta t} dt
	\\ & = \delta |Q| (e^{\delta C} - 1) \sum_{k=0}^\infty 2^{-k} e^{\delta kc}
	\\ & = |Q| \frac{2(e^{\delta C} - 1)}{2- e^{\delta C}}
\end{align*}
Dividing by $|Q|$, and setting $\frac{2(e^{\delta C} - 1)}{2- e^{\delta C}}= C$, we see that choosing $\delta = \frac{1}{c}\log \left( \frac{2(1+C)}{2+C} \right)$ works. Since $C$ depends on $n$ so will $\delta$. 
\item It is sufficient to show that $ \left\{ \left| u-u_Q \right|>2 \right\}\cap Q \subset \bigcup \tilde{Q}$, up to a set of measure 0. Note that by applying the search algorithm we get that if $x_i \in  \left\{ \left| u-u_Q \right|>2 \right\}\cap Q $ then $x_0$ must belong to some $\tilde{Q}$. It remains to show that the set of $x$ in $ \left\{ \left| u-u_Q \right|>2 \right\}\cap Q $ but not in $\bigcup \tilde{Q}$ is measure $0$. 
Call this set $N$. If $x_0 \in N$ then we can produce a sequence of cubes shrinking down to it, say $ \left\{ Q_i \right\}$. By the lebesque differentiation theorem, we have that 
$$ \lim_{n\to \infty} \frac{1}{|Q_n|}\int_{Q_i} |u- u_Q| dx = u(y). $$
This must be greater than $2$ except on a null set. This contradicts the assumption that $x_0\not \in \tilde{Q}$ for any $\tilde{Q}$. 
\item We have that 
$$ 2 \leq \frac{1}{|\tilde{Q}|}\int_{\tilde{Q}} |u-u_Q|. $$ 
Therefore 
$$ |\tilde{Q}| \leq \frac{1}{2}\int_{\tilde{Q}} |u-u_Q| $$ 
Summing over each $\tilde{Q}$, we get that 
$$\sum_{\tilde{Q}} |\tilde{Q}| \leq \frac{1}{2}\sum\int_{\tilde{Q}} |u-u_Q| \leq \frac{1}{2}\int_{Q} |u-u_Q| \leq \frac{1}{2}|Q|.$$
Applying this to part $b)$ we see that 
$$ \left| \left\{ \left| u-u_Q \right|>2 \right\}\cap Q \right|\leq \frac{1}{2}|Q|.$$
\item Let $\tilde{Q}^\prime$ be the predecessor of $\tilde{Q}$. Then,
	\begin{align*}
		|u_Q - u_{\tilde{Q}}| & = \left| \frac{1}{|\tilde{Q}|}\int_{\tilde{Q}} u-u_Q \right|
		\\ & \leq \frac{1}{|\tilde{Q}|} \int_{|\tilde{Q}^\prime|} |u-u_Q|dy
		\\ & \leq \frac{|\tilde{Q}^\prime|}{|\tilde{Q}|}\cdot \frac{1}{|\tilde{Q}^\prime|} \int_{|\tilde{Q}^\prime|} |u-u_Q|dy
		\\ & \leq 2\cdot 2^{n+1}.
	\end{align*}
\item Fix a $\tilde{Q}$. The following inequality holds: 
	$$ |u-u_Q| \leq |u-u_{\tilde{Q}}| + |u_Q - u_{\tilde{Q}}|. $$
	We have that $|u-u_Q|> 2 \cdot 2^{n+1}$ only if $|u-u_{\tilde{Q}}|$ or $|u_Q - u_{\tilde{Q}}|$ are greater than $2 \cdot 2^{n+1}$. 
	We have shown in $d)$ that this cannot occur for $|u_Q - u_{\tilde{Q}}|$, therefore if $ \left| u-u_Q \right|>2\cdot 2^{n+1}$, we must have that $ \left| u-u_{\tilde{Q}} \right|>2 \cdot 2^{n+1}$. 
$$ \left\{ \left| u-u_Q \right|> 2 \cdot 2^{n+1} \right\} \subset \left\{ \left| |u-u_{\tilde{Q}} \right|>2 \cdot 2^{n+1} \right\}. $$ 
It follows that 
\begin{align*}
	\left| \left\{ \left| u-u_Q \right|>2\cdot 2^{n+1} \right\}\cap Q  \right| & = 	\sum_{\tilde{Q}}\left| \left\{ \left| u-u_Q \right|>2\cdot 2^{n+1} \right\}\cap \tilde{Q}  \right| 
	\\ & \leq \sum_{\tilde{Q}}\left| \left\{ \left| u-u_{\tilde{Q}} \right|>2\cdot 2^{n+1} \right\}\cap \tilde{Q}  \right|
	\\ & \leq \sum_{\tilde{Q}}\left| \left\{ \left| u-u_{\tilde{Q}} \right|>2 \right\}\cap \tilde{Q}  \right|
\end{align*}
We now perform the same C-Z decomposition on each $\tilde{Q}$ labeling the children cubes as $R$. where we keep $R$ if $ \frac{1}{|R|}\int_R |u-u_Q| \geq 2 $. We can reapply $b),c),d)$ to these cubes to get that 
$$ \left| \left\{ \left| u-u_Q \right|>2 \right\}\cap \tilde{Q} \right|\leq \frac{1}{2}|\tilde{Q}| $$ 
and by summing over over all $ \tilde{Q}$, we get that 
$$\sum_{\tilde{Q}} \left| \left\{ \left| u-u_{\tilde{Q}} \right|>2 \right\}\cap \tilde{Q}  \right| \leq \sum_{\tilde{Q}} \frac{1}{2}|\tilde{Q}| \leq \frac{1}{4}|Q|. $$ 
As desired. 
\item Taking the same $R$ as before, we have that 
$$ \left| u-u_Q \right|  \leq |u-u_R| + |u_R - u_{\tilde{Q}}| + |u_Q - u_{\tilde{Q}}|.$$ 
Similarly as above, we have that $|u-u_Q|> 3\cdot 2^{n+1}$ implies that $|u-u_R|>2^{n+1}$ for the $R$ kept from the C-Z decomposition. Therefore we compute that 
\begin{align*}
	\left| \left\{ \left| u-u_Q \right|>3\cdot 2^{n+1} \right\}\cap Q \right| & = \sum_{\tilde{Q}}	\left| \left\{ \left| u-u_Q \right|>3\cdot 2^{n+1} \right\}\cap \tilde{Q} \right| 
	\\ & = \sum_{\tilde{Q}} \sum_R 	\left| \left\{ \left| u-u_Q \right|>3\cdot 2^{n+1} \right\}\cap R \right| 
	\\ & \leq \sum_{\tilde{Q}} \sum_R \left| \left\{ \left| u-u_Q \right|> 2^{n+1} \right\}\cap R \right| 
	\\ & \leq \sum_{\tilde{Q}} \sum_R \left| \left\{ \left| u-u_Q \right|> 2 \right\}\cap R \right| 
\end{align*}
Once again as in $e)$, this willl be bounded above by 
$$ \sum_{\tilde{Q}} \sum_R \frac{1}{2}|R| \leq \sum_{\tilde{Q}} 2^{-2} |\tilde{Q}| \leq2^{-3} |Q|. $$ 
We can repeat this again for each $R$ and producing $\tilde{R}$ using C-Z decomposition. We get that
$$ |u-u_Q|\leq |u-u_{\tilde{R}}| + |u_R - u_{\tilde{R}}| + |u_R - u_{\tilde{Q}}| + |u_{\tilde{Q}} - u_Q|. $$ 
We will see that 
$$ \left| \left\{ \left| u-u_Q \right|>4 \cdot 2^{n+1} \right\}\cap Q \right|\leq 2^{-4} |Q|. $$
We can iterate this $k$ times and conclude that 
$$ \left| \left\{ \left| u-u_Q \right|> k \cdot 2^{n+1} \right\}\cap Q \right| \leq 2^{-k}|Q|. $$
This establishes $a)$ and we conclude the result. 
\epenum
\newpage
\begin{problem}
% problem number 3
\end{problem}
\penum 
\item Take $R = \tau^{k-1}$. Then applying the given bound to this we see that:
	$$ \omega( \tau^k) \leq \gamma \omega(\tau^{k-1})  + \gamma \tau^{k1} F(\tau^{k-1}) \leq \gamma \left( \gamma \omega(\tau^{k-2}) + \gamma \tau^{k-2} F(\tau^{k-2}) \right)+ \gamma \tau^{k-1} F(\tau^{k-1})$$ 
Iterating $k-1$ more times, we get that the above is less than or equal to by the nondecreasingness of $F$, 
$$ \gamma^k \omega(1) + \sum_{i=1}^k \gamma^i \tau^{k-i} F(\tau^{k-i}) \leq \gamma^k \left( \omega(1) + F(1) \sum_{i=1}^\infty \left(\frac{\tau}{\gamma}\right)^{k-i} \right) = \gamma^k \left( \omega(1) + C(\gamma, \tau) F(1) \right).$$ 
Letting $C(\tau, \gamma) = C$. We take $\alpha = \frac{\log \gamma}{\log \tau}$, so that $\gamma = \tau^\alpha$. Then for any $R\in (0,1]$, take $k$ so that $\tau^{k+1} \leq R \leq \tau^k$. 
Since $\omega$ nondecreasing we have that 
$$ \omega(R) \leq \omega(\tau^k) \leq (\tau^k)^\alpha (\omega(1) + C F(1))  = \tau^{-\alpha} (\tau^{k+1})^\alpha \left( \omega(1) + C \ F(1) \right)\leq \tau^{-\alpha} R^\alpha (\omega(1) + C F(1)).$$
Take $ C = \max \left\{ \tau^{-\alpha} , C \tau^{-\alpha} \right\}$ to get 
$$\omega(R) \leq C R^\alpha \left( \omega(1) + F(1) \right).$$
\item We first claim that under the assumptions of theorem 0.4, 
	$$  \sup_{B_{R/2}}  \leq C \left( \inf_{B_{R/2}} + R \norm{ f }_{L^n(B_{R/2})} \right) $$ 
We first define $\tilde{u} (x) = u(Rx)$. Notice that this solves a rescaled PDE, $\tilde{a_{ij}}\tilde{u_{ij}} = \tilde{f}$, where $\tilde{a_{ij}} = a_{ij}(Rx)$ and $\tilde{f} = f(Rx)$. 
Applying theorem 0.4 we get that 
$$ \sup_{B_{R/2}} u = \sup_{B_{1/2}} \tilde{u} \leq C \left( \inf_{B_{1/2}} \tilde{u} + \norm{\tilde{f}}_{L^n(B_1)}\right)  = C \left( \inf_{B_{r/2}} u + \norm{\tilde{f}}_{L^n(B_1)} \right)$$ 
Finally 
$$ \norm{\tilde{f}}_{L^n(B_1)} = \left( \int_{B_1} R^2n \left| f(Rx) \right|^n dx \right)^{ \frac{ 1 }{ n }}  = R \left( \int_{B_R} \left| f(y) \right|^n dy \right)^{ \frac{ 1 }{ n }} = R \norm{f}_{L^n(B_R)} $$
We conclude the claim. 
Define the following functions: 
$$ v(x) = \sup_{B_R} u - u(x), \quad w(x) = u(x) - \inf_{B_R} u(x).$$ 
These will solve the same PDE as $u$, since they differ by a constant and sign, and they are both nonegative.  
Therefore by the claim, we have that 
$$ \sup_{B_{R/2}} v = \sup_{B_{R}} u - \inf_{B_{R/2}} u \leq C \left( \sup_{B_r} u - \sup_{B_{R/2}} u + R \norm{f}_{L^n(B_R)} \right),$$ 
and
$$ \sup_{B_{R/2}} w = \sup_{B_{R/2}} u - \inf_{B_R} u \leq \left( \inf_{B_{R/2}} u - \inf_{B_R} u + R \norm{f}_{L^n(B_R)} \right) $$ 
We add these two equations together to get that
$$ osc_{B_{R/2}} u \leq \frac{ C-1 }{ C+1 } osc_{B_R} u + \frac{ 2R }{ C+1 } \norm{f}_{L^n(B_R)}\leq \frac{ C-1 }{ C+1 }osc_{B_R} u + \frac{ C-1 }{ C+1 }R \norm{f}_{L^n(B_R)}. $$
Where we take $C$ sufficiently large so that $C\geq 3$. We now define $\omega(R) = osc_{B_R} u$, $F(R) = \norm{f}_{L^n(B_R)}$. We have that $\omega, F$ are non decreasing and that 
$$\omega(\tau R) \leq \gamma \omega (R) + \gamma RF(R) $$
for $\tau = \frac{ 1 }{ 2 }$, $\gamma = \frac{ C-1 }{ C+1 }$. 
By the $a)$, we have that 
$$ osc_{B_R} \leq CR^\alpha \left( \omega(1) + F(1) \right) = C R^\alpha \left( osc_1 u + \norm{f}_{L^n(B_1)} \right). $$
Which implies that 
$$ |u(x) - u(0)| \leq C |x|^\alpha \left( \norm{u}_{L^\infty(B_R)} + \norm{f}_{L^n(B_1)} \right)$$ 
It remains to show that $u\in C^\alpha(B_{1/2})$. 
For $x\in B_{1/2}$ define $u_x(a) = u( x + \frac{ 1 }{ 4 }a)$. Since $u_x$ has the same bounds as $u$, we can apply the above result to $u_x$:
$$ |u_x(a) - u_x(0) | = \left| u(x+ \frac{ 1 }{ 4 }a) - u(x) \right| \leq 4^\alpha C | \frac{ a }{ 4 }|^\alpha \left( \norm{u}_{L^\infty(B_1)} + \norm{f}_{L^n(B_1)} \right) .$$
This is the same as saying that for $x,y\in B_{1/2}$, so that $|x-y|\leq \frac{ 1 }{ 4 }$, 
$$ |u(x) - u(y)| \leq C \left( \norm{u}_{L^\infty(B_1)} + \norm{f}_{L^n(B_1)} \right)|x-y|^\alpha. $$ 
If we have that $x,y\in B_{1/2}$ with $|x-y|> \frac{ 1 }{ 4 }$, then 
$$|u(x) - u(y) | \leq 2 \norm{u}_{L^\infty(B_1)} \frac{ |x-y|^\alpha }{ |x-y|^\alpha } \leq 2\cdot 4^\alpha \norm{u}_{L^\infty(B_1)} |x-y|^\alpha.$$
We can take $C$ sufficiently large so that the righthand side of the above inequality is less than \\ $C \left( \norm{u}_{L^\infty(B_1)} + \norm{f}_{L^n(B_1)} \right)$. Thus we are done. 
\epenum
\end{document}
