\documentclass[12pt, a4paper]{article}
\usepackage[lmargin =0.5 in, 
rmargin=0.5in, 
tmargin=1in,
bmargin=0.5in]{geometry}
\geometry{letterpaper}
\usepackage{tikz-cd}
\usepackage{amsmath}
\usepackage{amssymb}
\usepackage{blindtext}
\usepackage{titlesec}
\usepackage{enumitem}
\usepackage{fancyhdr}
\usepackage{amsthm}
\usepackage{graphicx}
\usepackage{cool}
\usepackage{thmtools}
\usepackage{hyperref}
\graphicspath{ }					%path to an image

%-------- sexy font ------------%
%\usepackage{libertine}
%\usepackage{libertinust1math}

%\usepackage{mlmodern}				% very nice and classic
%\usepackage[utopia]{mathdesign}
%\usepackage[T1]{fontenc}


\usepackage{mlmodern}
\usepackage{eulervm}
%\usepackage{tgtermes} 				%times new roman
%-------- sexy font ------------%


% Problem Styles
%====================================================================%


\newtheorem{problem}{Problem}


\theoremstyle{definition}
\newtheorem{thm}{Theorem}
\newtheorem{lemma}{Lemma}
\newtheorem{prop}{Proposition}
\newtheorem{cor}{Corollary}
\newtheorem{fact}{Fact}
\newtheorem{defn}{Definition}
\newtheorem{example}{Example}
\newtheorem{question}{Question}

\newtheorem{manualprobleminner}{Problem}

\newenvironment{manualproblem}[1]{%
	\renewcommand\themanualprobleminner{#1}%
	\manualprobleminner
}{\endmanualprobleminner}

\newcommand{\penum}{ \begin{enumerate}[label=\bf(\alph*), leftmargin=0pt]}
	\newcommand{\epenum}{ \end{enumerate} }

% Math fonts shortcuts
%====================================================================%

\newcommand{\ring}{\mathcal{R}}
\newcommand{\N}{\mathbb{N}}                           % Natural numbers
\newcommand{\Z}{\mathbb{Z}}                           % Integers
\newcommand{\R}{\mathbb{R}}                           % Real numbers
\newcommand{\C}{\mathbb{C}}                           % Complex numbers
\newcommand{\F}{\mathbb{F}}                           % Arbitrary field
\newcommand{\Q}{\mathbb{Q}}                           % Arbitrary field
\newcommand{\PP}{\mathcal{P}}                         % Partition
\newcommand{\M}{\mathcal{M}}                         % Mathcal M
\newcommand{\eL}{\mathcal{L}}                         % Mathcal L
\newcommand{\T}{\mathbb{T}}                         % Mathcal T
\newcommand{\U}{\mathcal{U}}                         % Mathcal U\\
\newcommand{\V}{\mathcal{V}}                         % Mathcal V

% symbol shortcuts
%====================================================================%

\newcommand{\bd}{\partial}
\newcommand{\grad}{\nabla}
\newcommand{\lam}{\lambda}
\newcommand{\imp}{\implies}
\newcommand{\all}{\forall}
\newcommand{\exs}{\exists}
\newcommand{\delt}{\delta}
\newcommand{\ep}{\varepsilon}
\newcommand{\ra}{\rightarrow}
\newcommand{\vph}{\varphi}

\newcommand{\ol}{\overline}
\newcommand{\f}{\frac}
\newcommand{\lf}{\lfrac}
\newcommand{\df}{\dfrac}

% bracketting shortcuts
%====================================================================%
\newcommand{\abs}[1]{\left| #1 \right|}
\newcommand{\babs}[1]{\Big|#1\Big|}
\newcommand{\bound}{\Big|}
\newcommand{\BB}[1]{\left(#1\right)}
\newcommand{\dd}{\mathrm{d}}
\newcommand{\artanh}{\mathrm{artanh}}
\newcommand{\Med}{\mathrm{Med}}
\newcommand{\Cov}{\mathrm{Cov}}
\newcommand{\Corr}{\mathrm{Corr}}
\newcommand{\tr}{\mathrm{tr}}
\newcommand{\Range}[1]{\mathrm{range}(#1)}
\newcommand{\Null}[1]{\mathrm{null}(#1)}
\newcommand{\lan}{\langle}
\newcommand{\ran}{\rangle}
\newcommand{\norm}[1]{\left\lVert#1\right\rVert}
\newcommand{\inn}[1]{\lan#1\ran}
\newcommand{\op}[1]{\operatorname{#1}}
\newcommand{\bmat}[1]{\begin{bmatrix}#1\end{bmatrix}}
\newcommand{\pmat}[1]{\begin{pmatrix}#1\end{pmatrix}}
\newcommand{\vmat}[1]{\begin{vmatrix}#1\end{vmatrix}}

\newcommand{\amogus}{{\bigcap}\kern-0.8em\raisebox{0.3ex}{$\subset$}}
\newcommand{\Note}{\textbf{Note: }}
\newcommand{\Aside}{{\bf Aside: }}
%restriction
%\newcommand{\op}[1]{\operatorname{#1}}
%\newcommand{\done}{$$\mathcal{QED}$$}

%====================================================================%


\setlength{\parindent}{0pt}      	% No paragraph indentations
\pagestyle{fancy}
\fancyhf{}							% fancy header

\setcounter{secnumdepth}{0}			% sections are numbered but numbers do not appear
\setcounter{tocdepth}{2} 			% no subsubsections in toc

%template
%====================================================================%
%\begin{manualproblem}{1}
%Spivak.
%\end{manualproblem}

%\begin{proof}[Solution]
%\end{proof}

%----------- or -----------%

%\begin{problem} 		
%\end{problem}	

%\penum
%	\item
%\epenum
%====================================================================%


\newcommand{\Course}{MAT482}
\newcommand{\hwNumber}{2}

%preamble

\title{MAT482 A2}
\author{Alexander Neagoe}
\date{\today}
\lhead{\Course A\hwNumber}
\rhead{\thepage}
%\cfoot{\thepage}


%====================================================================%
\begin{document}

\maketitle

\begin{problem}
\end{problem}
\penum
\item Define the map $\varphi : \Omega \to \Gamma_u$ as $\varphi(x) = (x,u(x))$. Consider $\Gamma_u$ as a submanifold of $\R^{n+1}$. Since $\Gamma_u$ has codimension 1, Given any top form on $\R^{n+1}$, $\omega$ we can induce a top form on $\Gamma$ as follows: 
$$\omega(x)_{\Gamma_u} (v_1, \dots , v_n) = \omega(x)(N(x) , v_1 , \dots , v_n)$$
where $N(x)$ is a unit normal to $\Gamma_u$. If $\{e_i\}$ is the standard basis for $\R^n \subset\R^{n+1}$, we need $ \inn{\varphi^\ast e_i. N(x) } = 0$. Since $\varphi^\ast e_i = \left(e_i, \frac{\partial u}{\partial x_i} \right)$, take $N(x) = (\grad u(x), -1)$.
If we take $\omega$ to be $dx_1 \wedge \dots \wedge dx_{n}$, then the volume of $\Gamma_u$ will be given as: 
\begin{align*}
Vol(\Gamma_u) &= \int_{\Gamma_u} \omega_{\Gamma_u}(x) dx_{I} 
\\ &= \int_{\Gamma_u} \omega(x)dx_I 
\\ &= \int_\Omega \omega(\varphi(x))d \varphi^\ast x_I  
\\& = \int_\Omega \frac{1}{\norm{(\grad u(x) , -1)}} \det ( \varphi_i \cdot \varphi_j ) dy_1 \wedge \dots  \wedge dy_n 
\\ & = \int_\Omega \norm{(\grad u(x), -1)} dy
\\ & = \int_\Omega \sqrt{|\grad u (x) |^2 + 1} dy
\end{align*} 
Where $\det ( \varphi_i \cdot \varphi_j ) = \norm{\grad u(x), 1}^2$ since 
$$\det ( \varphi_i \cdot \varphi_j ) = 1 + \sum_i^n u_i^2 + \sum_{\sigma \neq id} (-1)^\sigma \prod_{i}^{n} u_{\sigma(i)} - \sum_{\sigma \neq id} (-1)^\sigma \prod_{i}^{n} u_{\sigma(i)} = \norm{1+ |\grad u(x)|}^2 $$
\item Let $v \in C^1(\Omega) \cap C^0 (\ol{\Omega})$ function so that $v|_{\bd \Omega} = 0$. We compute the variation of $A$ as 
\begin{align*}
	\frac{d}{d\ep} A(u + \ep v)\Big|_{\ep = 0} & = \frac{d}{d \ep} \Big|_{\ep = 0} \int_{\Omega} \sqrt{1 + |\grad  u + \ep \grad v|^2} dx
	\\ & = \int_{\Omega} \frac{\grad v \cdot (\ep \grad v + \grad u)}{\sqrt{1 + |\grad u + \ep \grad v|^2}}dx \Big|_{\ep = 0}
	\\ & = \int_\Omega \frac{\grad v \cdot \grad u}{\sqrt{1 + |\grad u + \ep \grad v|^2}}dx 
	\\ & = \sum_{i=1}^n \int_{\Omega} \frac{u_i \cdot v_i}{\sqrt{1 + |\grad u + \ep \grad v|^2}}dx
\end{align*}
\item If $u$ is a critical point of $A$ for all $v$, we must have that 
$$0 = \sum_{i=1}^n \int_{\Omega} \frac{u_i \cdot v_i}{\sqrt{1 + |\grad u + \ep \grad v|^2}}dx =\sum_{i=1}^n \int_{\Omega} \frac{\partial}{\partial x_i} \left(\frac{u_i}{\sqrt{1+|\grad u}|^2} \right)v dx.$$
Therefore $u$ satisfies the following "divergence type" equation weakly:
$$0 = \sum_{i=1}^n \frac{\partial}{\partial x_i} \frac{u_i}{\sqrt{1+ |\grad u|^2}}.$$
Assuming that $u$ has higher order regularity, we can write the PDE in the form $a^{ij}u_{ij}$ as follows: 
\begin{align*}0 &= \sum_{i=1}^n \frac{\partial}{\partial x_i} \frac{u_i}{\sqrt{1+ |\grad u|^2}}  
	\\ & = \sum_{i=1}^n \frac{u_{ii} \sqrt{1+ |\grad u|^2} - u_i \left(\sqrt{1+ |\grad u|^2} \right)_i }{ 1+ |\grad u|^2 } 
	\\ & = \sum_{i=1}^n \frac{u_{ii} \sqrt{1+ |\grad u|^2}- u_i \cdot (\sqrt{1+ |\grad u|^2})^{-1} \sum_{j=1}^n u_{ji}u_j} {1+|\grad u|^2}
	\\ & = \sum_{i} \left( u_{ii} -  u_i \frac{1}{1+ |\grad u|^2 } \sum_{j=1}^n u_{ji} u_j  \right)
\end{align*}
From this we can read off the $\tilde{a}^{ij}$ as $\tilde{a}^{ij} = \delta_{ij}  - \frac{u_i u_j}{1+|\grad u|^2 }$. We claim that this PDE is elliptic. We can write $\tilde{a}^{ij} =I - B$, where $B_{ij} = \frac{u_i u_j}{1 + |\grad u|^2}$. To show that $\tilde{a}^{ij}$ is positive definite, we show that it has a maxmimum eigenvalue $0<\lambda <1$. It will then immediately follow that $I -B$ has a maximum eigenvalue of $1>1 - \lambda$. We have that 
$$B = \frac{1}{1+|\grad u|^2} (\grad u) (\grad u)^T.$$
Since $\grad u ^T$ has codomain $\R$, it must have kernel of dimension $n-1$. Therefore $B$ has kernel of dim $n-1$. Note though that $\grad u$ is an eigenvalue of $B$, since 
$$B (\grad u) = \frac{|\grad u|^2}{1+|\grad u|^2} \grad u.$$ 
The remaining eigenvalue is therefore $\frac{|\grad u|^2}{1+|\grad u|^2}$ which is less than $1$.
\item Say $u$ weakly satisfies $\tilde{a}^{ij}u_{ij}$ and $\sup_{\Omega}|\grad u| <C$. The bound on $\grad u$ tells us that in fact the $\tilde{a}^{ij}$ are bounded, say by $D$ and uniform ellitpicity holds by applying the argument for $c)$. Let $V \subset \subset \Omega$, and $W$ so that $V \subset \subset W \subset \subset \Omega$. Let $\phi$ be a smooth cut off function that is $1$ on $V$ and $0$ outside of $W$. Since $u$ is a weak solution, we have that 
$$\int_U \tilde{a}^{ij}u_i v_j dx = 0. \quad (*)$$
Take $h$ sufficiently small, and take $$v = - D^{-h}_k \left( \phi^2 D^h_k u \right).$$
Substituting $v$, we can rewrite $(*)$ as: 
\begin{align*}
	\int_U \tilde{a}^{ij}u_i v_j dx & =- \int_U \tilde{a}^{ij} u_i \cdot \left[ D^{-h}_k \left( \phi^2 D^h_k u \right)\right]_j dx 
	\\ & = \int_{U} D^h_k \left(\tilde{a}^{ij} u_i \right) (\phi^2 D^h_k u)_j dx \tag{integration by parts for difference quotients}
	\\ & = \int_{U} \tilde{a}^{ij,h} (D_k^h u_i) (\phi^2 D^h_h u)_j + (D^h_k \tilde{a}^{ij})u_i (\phi^2 D_k^h u)_j dx\tag{product rule for difference quotients}
\end{align*}
Expanding this out using the product rule, we get:
\begin{align*}
	0 & = \overbrace{\int_U \tilde{a}^{ij}D_k^hu_i D_k^h u_j \phi^2 dx}^{A_1}+\overbrace{ \int_U \tilde{a}^{ij} D_k^h u_iD_k^h u2\phi\phi_j +(D_k^h \tilde{a}^{ij})u_iD^h_ku_j\phi^2 + (D_k^h \tilde{a}^{ij})u_i D_k^h u2\phi \phi_j dx}^{A_2}.
\end{align*}
By uniform ellipticity, we have the following lower bound on $A_1$: 
$$A_1 \geq \lambda \int_u \phi^2 |D^h_k Du|^2 dx.$$
The bound on $ |\grad u|$ tells us: 
$$|A_2| \leq  D \int_U \phi |D_k^h Du|\cdot |D_k^h u| + \phi |D_k^h Du|\cdot |D_u| + \phi |D_k^h u|\cdot |Du| dx.$$
By Cauchy Schwartz,
$$|A_2| \leq \ep \int_U \phi^2 |D_k^hDu|^2 dx + \frac{D}{\ep} \int_W |D_k^h u|^2 + |Du|^2 dx.$$
Take $\ep = \lambda/2$, and using the fact (proved in lecture) that 
$$\int_W |D_k^h u|^2 dx \leq D \int_{U} |Du|^2 dx$$
we see that $$|A_2| \leq \frac{\lambda}{2} \int_U \phi^2 |D_k^hDu|^2 dx + D \int_U |Du|^2 dx.$$
Putting these estimates together we have that 
$$0 \geq \frac{\lambda}{2} \int_U \phi^2 |D^h_k Du|^2 dx - D\int_U |Du|^2 dx.$$
Therefore $u\in W^{2,2}_{loc}$. 
\item We cannot apply Sobolev bootstrap because this PDE is nonlinear and so we do not have bounds on higher order derivatives of $u$. In the proof above it is crucial that we bound the coefficients $\tilde{a}^{ij}$, which depend on $u$. If higher order derivatives become unbounded we cannot make this estimate. 
\epenum

\newpage
\begin{problem}
\end{problem}
\penum 
\item First note that we can always recenter our polynomial $P_x(y)$ as 
 $$P^\prime_x(y) = P_x(x+y) = \frac{1}{2}a_x^{ij}(x_i+y+_i)(x_j+y_j) + b_x^{i}(x_i+y_i) + c_x.$$
The coefficients $a^{ij}_x, b^i_x,c_x$ are all continuous as functions of $x$, since if we take $|x-y|<r$, $z = \frac{x+y}{2}$ for $w\in B_\frac{r}{2}(z)$
$$|P_x(w)- P_x(w)| \leq |u(w) - P_x(w)| + |u(w) - P_y(w)| \leq 2 A_1r^{\alpha + 2}.$$
Therefore the coefficients of $P_x$ are continuous, and so the coefficients of $P^\prime_x$ are continuous as well. We now claim that $P$ is the taylor series of $u$. We translate $P$ to $P^\prime$, and get that for any $r>0$ that
$$|u(x) - P^\prime_x(0)| = |u(x) - c_x| \leq A_2r^{2+\alpha}.$$
Taking $r \to 0$ gives us that $u(x) = c_x$. For a basis vector $e_i$ and sufficiently small $h$, we have
$$|u(x + he_i) - P_x(x+he_i)| = |u(x+he_i)  - u(x) - \frac{1}{2}a^{ii}_x h^2 - b_{x,i}h| \leq A_2 h^{2+\alpha}.$$ 
Since the coefficients of $P$ are bounded, by the triangle inequality we have that
$$|u(x+he_i) - u(x) - b^i_{x}h| \leq A_2h^{2+\alpha} + \frac{1}{2}|a^{ii}|h^2.$$
We get that $u_i(x) = b^i_x$. It remains to show that $u_{ij}(x) = a^{ij}_x $.
Recall the formula
$$u_{ii} = \lim_{h\to 0} \frac{u(x+he_i) - 2 u(x) + u(h-he_i)}{h^2}.$$
We can write that
$$ |(u(x+he_i ) - P^\prime_x(he_i)) + (u(x-he_i) - P^\prime_x(-he_i))| =|u(x+he_i) - 2u(x) + u(x-h_i) - a^{ii}_x h^2|  \leq 2A_2h^{2+\alpha}.$$
Once again by taking $h\to 0$ we get that $u_{ii}(x) = a^{ii}_x$. 
We denote the second derivative of $u$ in the $e_i+e_j$ direction as $u_{i+j,i+j}$. By multilinearity, we have
$$u_{i+j,i+j}(x) = a^{ii}_x + 2 a^{ij}_x + a^{jj}_x.$$
Therefore by
$$u_{ij}(x) = \frac{1}{2} \left[u_{i+j,i+j} (x) - u_{ii}(x) - u_{jj}(x)\right] = a^{ij}_x.$$
Therefore $u$ is $C^2$. It remains to show that $u\in C^{2,\alpha}$. We need to find a $C$ so that
$$|u_{ij}(x) - u_{ij}(y)| \leq C |x-y|^\alpha.$$
Take $x,y$ so that $|x-y|=r$ and $B_r(x), B_r(y) \subset B_1(0)$. 
We define $P(z) = P_x(y+w) - P_x(z+w)$. Recall from above that $\sup |P|\leq 2A_2r^{2+\alpha}$.
At $w = 0$ we get that $|c_x| \leq 2A_2r^{2+\alpha}$. Taking $w = re_i/4$ into $|P(w) + P(-w)|$, we get that 
$$\frac{r^2}{32} |a_x^{ii} - a_y^{ii}| \leq 8 A_2 r^{2+\alpha}$$
and so $|a_x^{ii} - a_y^{ii}| \leq Cr^{\alpha}$. We can do a similar argument with $w = \frac{r(e_i+e_j)}{2}$ to get a similar bound on $|a^{ij}_x - a^{ij}_y|$. Finally taking $s\in (0,1)$ by above we have that $[u_\beta]_\alpha <C A_2$ for $C$ depending on $s$. Since $|u_\alpha|_{L^\infty}<A_1$ for $\alpha = 2$. Taking the maximum between these constants, 
$$\norm{u}_{C^{2 , \alpha}} = \sum_{|\alpha| \leq 2}\norm{u_{\alpha}} + \sum_{|\beta| = 2} C(n,s)(A_1+A_2).$$ 
\item Take $P_x(y)$ to be the 2nd order taylor polynomial of $f$ at $x$. The coefficients of $P$ will be bounded by the $C^2$ norm of $f$. Taking $r>0$ and $y\in B_r(x) \cap B_1(0)$, 
by Taylor's Theorem, 
$$|f(x) - P_x(y)  | \leq \Big| |f(y)  - \sum_{\alpha \leq 1} \frac{f_\alpha(x)}{\alpha!}(y-x)^\alpha  - \sum_{|\alpha| = 2} \frac{f_\alpha(x)}{\alpha!}(y-x)^\alpha\Big|.$$
By the Lagrange form of the remainder, we have some $z$ so that $|z - x| \leq |y-x|$ so that
$$\Big| |f(y)  - \sum_{\alpha \leq 1} \frac{f_\alpha(x)}{\alpha!}(y-x)^\alpha  - \sum_{|\alpha| = 2} \frac{f_\alpha(x)}{\alpha!}(y-x)^\alpha\Big| = \Big| \sum_{|\alpha | = 2} \frac{f_{\alpha}(z)}{\alpha!}(y-x)^\alpha - \sum_{|\alpha | = 2} \frac{f_{\alpha}(z)}{\alpha!}(y-x)^\alpha \Big|.$$
This will be controlled by $$\sum_{|\alpha| = 2} \frac{|f_{\beta}(\xi)  - f_\beta(x)|}{\alpha!}|x-y|^{2} \leq \sum_{|\alpha| = 2} C|x-y|^{2+\alpha} =C_n |y-x|^{2+\alpha}.$$
For $C = \sup \frac{|f_{\alpha}(\xi)  - f_\alpha(x)|}{|x-y|^\alpha}$. 
We have that $C_n$ is controlled by $C^{2, \alpha}$ norm of $f$ since it will be a part of the sum that defines it. 
\epenum

\newpage
\begin{problem}
\end{problem}
\penum
\item 
Since $$|a_\xi e^{i \xi \cdot x}| = |a_\xi|,$$
by the Weierstrass M-test, we have that $\sum_{\xi \in \Z^n } a_\xi e^{i \xi \cdot x}$ converges absolutely and uniformly. Defining $f= \sum_{\xi \in \Z^n} a_\xi e^{i \xi \cdot x}$, $f$ will be continuous since each $a_\xi e^{i\xi \cdot x}$ is. 
We have that $$\sup_{ \T^n } |f| = \sup_{\T^n}  \left| \sum_{\xi \in \Z^n} a_\xi e^{i \xi \cdot x} \right| \leq \sup_{\T^n } \sum_{\xi \in \Z^n } \left|a_\xi e^{i \xi \cdot x} \right| = \sum_{\xi \in \Z^n} |a_\xi|.$$
\item We first claim the integration by parts holds for $C^\infty$ functions. If $u,v \in C^\infty$, then 
$$\int_{[-\pi, \pi]^n } (D_i u )v = \int_{[-\pi , \pi]^{n-1}} \int_{[-\pi , \pi]}(D_i u)v = \int_{[-\pi, \pi]^{n-1}}  \left((uv)\Big|_{-\pi}^\pi - \int_{[-\pi , \pi]} u (D_i v) \right)  = - \int_{[-\pi, \pi]^n } u (D_iv).$$ 
Now suppose that $u \in W^{1,2} (\T^n)$. Take a sequence $\{u_n\} \subset C^\infty(\T^n)$ converging to $u$. We must have that 
$$\int_{[-\pi, \pi]^n } (D_i u_n)v  = - \int_{[-\pi, \pi]^n } u_n (D_i v).$$
The righthand side converges to $\int_{[-\pi, \pi]^n } u (D_i v)$. By definition this is equal to $\int_{[-\pi, \pi]^n } (D_i u)v$. Therefore $(D_i u_n) \to (D_i u)$ by uniqueness of the derivative. We apply the same reasoning for $v\in W^{1,2}(\T^n)$ to get that $$\int_{[-\pi, \pi]^n } (D_iv)u = \int_{[-\pi, \pi]^n }v(D_i u)$$
\item Since $u$ and its $k$'th derivatives are in $L^2$ we have that the Fourier series is defined. We compute the fourier coefficients using $b)$ as: 
$$\widehat{D^\alpha u}(\xi) = \int_{\T^n} D^\alpha u e^{-i x\cdot \xi } dx = (-1)^{|\alpha|}\int_{\T^n} u D^\alpha ( e^{-ix \cdot \xi}) dx = (i \xi)^\alpha \hat{u}(\xi).$$
Therefore the Fourier series for $D^\alpha u$ is
$$D^\alpha u(x)=\sum_{\xi \in \Z^n} (i\xi)^\alpha \hat{u}(\xi)e^{i \xi \cdot x}.$$
We next claim that the following holds for some constants $C_1,C_2$:
$$C_1 (1+ |\xi|^2)^k \leq \sum_{|\alpha| \leq k} |\xi^\alpha|^2 \leq C_2(1+ |\xi|^2)^k. \quad (*)$$
We first prove that $C_1 (1+ |\xi|^2)^k \leq \sum_{|\alpha| \leq k} |\xi^\alpha|^2$. Consider the function $\sum_{i=1}^n |x_i^2|^k$. This is strictly positive on $S^n$, and so has a minimum $m$. By homogeneity, $\sum_{i=1}^n |x_i^2|^k \geq m |x^2|^k$. Therefore
$$(1+|x|^2)^k \leq 2^n \left[ 1+ m^{-1} \sum_{i=1}^n  |x_i^2|^k \right] \leq 2^k m^{-1} \sum_{|\alpha| \leq k} |x^\alpha|^2.$$ Dividing by $\leq 2^k m^{-1}$ gives us the desired bound. A similar argument with the maximum gives us the bound from above. 
Therefore $(*)$ holds. We can adjust our constants so that $C_1 = C^{-1}, C_2 = C$ for some $C$. 
Thus we have that 
$$C^{-1} (1+ |\xi|^2)^k \leq \sum_{|\alpha| \leq k} |\xi^\alpha|^2 \leq C(1+ |\xi|^2)^k.$$
Multiply both sides by $\hat{u}(\xi)$, and sum over all $\xi$ to get that 
$$C^{-1} \sum_{\xi \in \Z^n}(1+ |\xi|^2)^k \hat{u}(\xi) \leq \sum_{|\alpha| \leq k} \left( \sum_{\xi \in \Z^n} |\xi^\alpha|^2 |\hat{u} (\xi)|^2\right) \leq C \sum_{\xi \in \Z^n}(1+ |\xi|^2)^k \hat{u}(\xi).$$
We use Parsevall's identity to make sense of the middle term: 
$$\sum_{|\alpha| \leq k} \left( \sum_{\xi \in \Z^n} |\xi^\alpha|^2 |\hat{u} (\xi)|^2\right) = \sum_{|\alpha| \leq k} \left( \sum_{\xi \in \Z^n} \left|\widehat{D^\alpha u} (\xi)\right|^2 \right) =\sum_{|\alpha| \leq k} \norm{D^\alpha u}^2_{L^2} = \norm{u}_{W^{k,2}(\T^n)}^2.$$
We conclude that 
$$C^{-1} \sum_{\xi \in \Z^n}(1+ |\xi|^2)^k \hat{u}(\xi) \leq \norm{u}_{W^{k,2}(\T^n)}^2 \leq C \sum_{\xi \in \Z^n}(1+ |\xi|^2)^k \hat{u}(\xi).$$ Therefore $\norm{\cdot}_{W^{k,2}(\T^n)}$ is equivalent to the norm given by $\norm{u}_\ast = \sum_{\xi \in \Z^n} (1+|\xi|^2)^k |a_\xi|^2$. This makes sense to define by $a)$. 
\item Consider the series $$\sum_{\xi \in \Z^n} \frac{1}{(1+ |\xi|^2)^\alpha }.$$
By the integral test, this will converge if and only if the integral $\int_{\R^n}\frac{1}{(1+|x|^2)^\alpha}$ converges. Changing to spherical coordinates, we have that 
$$\int_{\R^n}\frac{1}{(1+|x|^2)} dx = \omega_n \int_0^\infty \frac{r^{n-1}}{(1+r^2)^\alpha} dr <\infty \iff 2\alpha - n+1>1 \iff \alpha > \frac{n}{2}. $$
We now claim that $u \in C^p(\T^n)$. In order for this to be true, we must have that 
$$D^\alpha u = \sum_{\xi \in \Z^n} \widehat{D^\alpha u} e^{i \xi \cdot x} = \sum_{\xi \in \Z^n} (i\xi)^\alpha \hat{u}(\xi) e^{- \xi \cdot x}$$ converges for $|\alpha| \leq p$, since if the derivatives of up to order $p$ did exist, they must be of this form by $c)$. By part $a)$ it is enough to show that $$\sum_{\xi \in \Z^n} |\xi|^\alpha |\hat{u}(\xi)| < \infty.$$
Observe: 
\begin{align*}
	\sum_{\xi \in \Z^n} |\xi|^\alpha |\hat{u}(\xi)| & \leq \sum_{\xi \in \Z^n } (1+|\xi|^2)^\frac{p}{2} |\hat{u}(\xi)|  \tag{since $|\alpha| \leq p$}
	\\ & \leq \left(\sum_{\xi \in \Z^n} (1+|\xi|^2)^k| \hat{u}(\xi)|^2 \right)^{1/2} \left( \sum_{\xi \in \Z^n} \frac{1}{(1+|\xi|^2)^{k-p}} \right)^\frac{1}{2} \tag{Cauchy-Schwarz on $\mathfrak{l}^2$}
\end{align*}
The first term of the product is finite by $c)$, the second term is finite because $k-p>\frac{n}{2}$. Thus it makes sense to define $D^\alpha u$ as we did. We now determine a bound on $\norm{u}_{C^p}$. 
\begin{align*}
	\norm{u}_{C^p} &= \sum_{|\alpha |\leq p} \sup_{\T^n} |D^\alpha u|
	\\ & \leq \sum_{|\alpha |\leq p} \sum_{\xi \in \Z^n}|a^\alpha_\xi| \tag{part $a)$}
	\\ & = \sum_{|\alpha |\leq p} \sum_{\xi \in \Z^n} |\xi^\alpha| |\hat{u}(\xi)|
	\\ & \leq \sum_{|\alpha |\leq p}\left(\sum_{\xi \in \Z^n} (1+|\xi|^2)^k| \hat{u}(\xi)|^2 \right)^{1/2} \left( \sum_{\xi \in \Z^n} \frac{1}{(1+|\xi|^2)^{k-p}} \right)^\frac{1}{2} \tag{by above}
	\\ & \leq C \norm{u}_{W^{k,2}(\T^n)} \sum_{|\alpha |\leq p}\left(\sum_{\xi \in \Z^n} (1+|\xi|^2)^k| \hat{u}(\xi)|^2 \right)^{1/2} \tag{by $c)$}
	\\ & \leq C_{n,p,k} \norm{u}_{W^{k,2}(\T^n)}
\end{align*} 
\item For any $u$ in the kernel of $L$ we have that 
$$0 = Lu = a^{ij}u_{ij} + b^i u_i + cu.$$
Fourier transforming both sides, we get
$$0 = \hat{u}(\xi) (-a^{ij}\xi_i \xi_j + ib^i \xi_i + c).$$
Thus the kernel of $L$ is given exactly by the $u$ satisfying the above. Similarly the kernel of $L^\dagger$ is given by 
$$0 = \hat{u}(\xi) (-a^{ij}\xi_i \xi_j - ib^i \xi_i + c).$$
We claim that the kernel will be finite dimensional. Applying $L$ to $e^{i \xi \cdot x}$, we have that $\xi$ must satisfy 
$$-a^{ij}\xi_i \xi_j + ib^i \xi_i +c = 0.$$
Since $a,b,c$ real this will only be satisfied if 
$$a^{ij}\xi_i\xi_j  = c , \quad ib^i\xi_i = 0.$$
By ellipticity, we have that $$c = a^{ij}\xi_i \xi_j > \lambda |\xi|^2.$$
Only finitely many $\xi$ can satisfy this. Similarly for $L^\dagger$. 
They will be empty if $b^i \neq 0$ and if $c < 0$, since 
$$-a^{ij} \xi_i \xi_j + c < -\lambda |\xi|^2 + c <0.$$
The kernels will always be equal, since we can always complex conjugate a root of $L$ to get a root of $L^\dagger$. Since the coefficients are real $\xi$ will satisfy
$$a^{ij}\xi_i \xi_j + ib^i \xi_i +c = 0 \iff a^{ij}\xi_i \xi_j - ib^i \xi_i +c = 0$$
\item First if $\phi \in \ker (L^\dagger)$, we have that 
$$0 = \inn{u, L^\dagger \phi} = \inn{f , \phi}. $$
So $f$ must be orthogonal to the kernel of $L^\dagger$. We have shown that $u\in W^{2,2}$ if $\sum_{\xi \in \Z^n} (1+|\xi|^2)^2 |\hat{u}(\xi)|^2 < \infty$. We apply the very weak solution condition to $\phi = e^{i \xi \cdot x}$. We compute 
$$\inn{u, L^\dagger e^{i \xi \cdot x}} = \inn{u, (-a^{ij} \xi_i \xi_j - i b^i \xi_i +c) e^{i \xi \cdot x}} = \inn{f, e^{i \xi \cdot x}}. $$
We see that $$\hat{u} (\xi ) (-a^{ij} \xi_i \xi_j - i b^i \xi_i +c) = \hat{f}(\xi).$$
Since $f$ is orthogonal to the kernel of $L^\dagger$, we can divide by the polynomial and get that 
$$\hat{u} (\xi) = \frac{\hat{f}(\xi)}{-a^{ij} \xi_i \xi_j - i b^i \xi_i +c}.$$ Note that for some sufficiently large $M$, for all $|\xi|\geq M$ we have that $|-a^{ij} \xi_i \xi_j - i b^i \xi_i +c| \geq \frac{\lambda}{2} |\xi |^2$
Therefore 
$$\sum_{\xi \in \Z^n} (1+|\xi|^2)^2 |\hat{u}(\xi)|^2  \leq  \sum_{|\xi |< M} (1+|\xi|^2)^2 |\hat{u}(\xi)|^2 + \frac{2}{\lambda }\sum_{|\xi| \geq M} (1+|\xi|^2)^2 \frac{|\hat{f}(\xi)|^2}{|\xi|^4}.$$
The first summand is controlled by $\norm{u}_{L^2(\T^n)}$ since it is finite, and the second summand converges and is controlled by $\norm{f}_{L^2(\T^n)}$ by Plancharel's theorem. Thus we have that 
$$\sum_{\xi \in \Z^n} (1+|\xi|^2)^2 |\hat{u}(\xi)|^2 \leq C \left( \norm{u}_{L^2(\T^n)} + \norm{f}_{L^2(\T^n)}\right) < \infty. $$ Since $\sum_{\xi \in \Z^n} (1+|\xi|^2)^2 |\hat{u}(\xi)|^2$ is controlled by the $W^{2,2}$ norm, we have that 

$$\norm{u}_{W^{2,2}(\T^n)} \leq C \left( \norm{u}_{L^2(\T^n)} + \norm{f}_{L^2(\T^n)}\right).$$
If $u$ is additionally orthogonal to the kernel of $L$, we must have that $\inn{u , e^{i \xi \cdot x}} = 0$ whenever $Le^{i \xi \cdot x} = 0$. Recall from part $e)$ that the kernel of $L$ is generated by $\xi \in \Z^n $ solving $-a^{ij} \xi_i \xi_j + i b^i \xi_i + c = 0$. By complex conjugating, this will be the exact same set of $\xi$ that satisfy $-a^{ij} \xi_i \xi_j - i b^i \xi_i + c = 0$ since all the coefficients are real. Therefore we have that $\ker L = \ker L^\dagger$. 
Similarly as before, we have an $M$ as above. 
\begin{align*}
\sum_{\xi \in \Z^n} (1+|\xi|^2)^2 |\hat{u}(\xi)|^2 
&= \sum_{|\xi | < M} (1+|\xi|^2)^2\frac{|\hat{f}(\xi)|^2}{|(a^{ij} \xi_i\xi_j - i b_i\xi_i + c )|^2} +\sum_{|\xi| \geq M} (1+|\xi|^2)^2 \frac{|\hat{f}(\xi)|^2}{|(a^{ij} \xi_i\xi_j - i b_i\xi_i + c )|^2}
\\ & = \sum_{|\xi | < M} (1+|\xi|^2)^2\frac{|\hat{f}(\xi)|^2}{|(a^{ij} \xi_i\xi_j - i b_i\xi_i + c )|^2} + \frac{\lambda}{2} \sum_{|\xi| \geq M} (1+|\xi|^2)^2\frac{|\hat{f}(\xi)|^2}{|\xi|^4}
\\ & \leq C\norm{f}_{L^2(\T^n)}.  
\end{align*}
Similarly we can control $\sum_{\xi \in \Z^n} (1+|\xi|^2)^2 |\hat{u}(\xi)|^2$ by $C \norm{u}_{W^{2,2}(\T^n)}$ so we conclude that 
$$\norm{u}_{W^{2,2}(\T^n)} \leq C \norm{f}_{L^2(\T^n)}. $$
\item Define $G$ as $$Gv(x) = \sum_{\xi \in \Z^n} \frac{\hat{v}(\xi)}{(-a^{ij} \xi_i \xi_j - i b^{i} \xi_i + c)} e^{i \xi \cdot x}.$$
This will be well defined since the coefficients $\hat{v}(\xi)$ for $\xi$ satisfying $(-a^{ij} \xi_i \xi_j - i b^{i} \xi_i + c) =0$ are 0, since $v\in (\ker L^\dagger)^\perp$. 
Since $\widehat{\alpha f + g}(\xi) = \alpha \hat{f} (\xi)+ \hat{g}(\xi)$ we have that $G$ as defined is linear. We claim $G$ as defined has the desired codomain when defined on $(\ker L^\dagger)^\perp$. Suppose that $\eta \in (\ker L^\dagger)^\perp$. We must have that for whenever $(-a^{ij} \xi^\prime_i \xi^\prime_j - ib^{i}\xi^\prime_i +c ) = 0$, $\inn{\eta, e^{i \xi^\prime \cdot x}} =0$. It follows that for an arbitrary $\xi^\prime$, 
$$\inn{G\eta, e^{i \xi^\prime \cdot x}} = \inn{\sum_{\xi \in \Z^n} \frac{\hat{\eta}(\xi)}{(-a^{ij}\xi_i \xi_j - i b^{i}\xi_j + c)}e^{i \xi \cdot x},e^{i \xi^\prime \cdot x}}= \inn{\frac{{\eta}(\xi^\prime)}{(-a^{ij}\xi_i \xi_j - i b^{i}\xi_j + c)} e^{i \xi^\prime \cdot x}, e^{i \xi^\prime \cdot x}}.$$
By $f)$ we have that $\inn{G\eta, e^{i \xi^\prime \cdot x} } = 0$ if and only if $(-a^{ij}\xi_i \xi_j + i b^{i}\xi_j + c) = 0$. Since the kernels are equal by part $f$ we have that $(-a^{ij}\xi_i \xi_j + i b^{i}\xi_j + c)$ is in fact 0. So if $\eta$ is in $(\ker L^\dagger)^\perp$ then $G\eta$ is in $(\ker L)^\perp$. Furthermore we must have that $G\eta \in W^{2,2}(\T^n)$, since the fourier coefficients of $G\eta$ take the form of $\frac{\hat{\eta}(\xi)}{(-a^{ij}\xi_i \xi_j - i b^{i}\xi_j + c)}$. This will satisfy our equivalent formulation of $W^{2,2}(\T^n)$ by running a similar argument as $f)$, by bounding finitely many terms and using ellipticity to bound the remainder. Furthermore, we have that $Gf = u$ since
$$Gf = \sum_{\xi \in \Z^n} \frac{\hat{f}(\xi)}{(-a^{ij}\xi_i \xi_j - i b^{i}\xi_j + c)} e^{i \xi \cdot x} = \sum_{\xi \in \Z^n} \hat{u}(\xi) e^{i \xi \cdot x} = u.$$
From this it immediately follows that $$LGf = L(u) = f, \quad GLu = G(f) = u$$
and $$\norm{Gf}_{W^{2,2}(\T^n)} \leq C \norm{f}_{L^2}$$
for the same reason as $f)$. Equality of $GLu =u$ is in the weak sense. We can say that $u =  Gf$ is in $C^p$ when $f\in W^{k,2}(\T^n)$ , for $k>p + \frac{n}{2}$ by part $d)$.  
\epenum

\end{document}