\documentclass[12pt, a4paper]{article}
\usepackage[lmargin =1 in, 
rmargin=1in, 
tmargin=1in,
bmargin=0.5in]{geometry}
\geometry{letterpaper}
\usepackage{tikz-cd}
\usepackage{amsmath}
\usepackage{amssymb}
\usepackage{blindtext}
\usepackage{titlesec}
\usepackage{enumitem}
\usepackage{fancyhdr}
\usepackage{amsthm}
\usepackage{graphicx}
\usepackage{cool}
\usepackage{thmtools}
\usepackage{hyperref}
\usepackage[most]{tcolorbox}
\graphicspath{ }					%path to an image

%-------- sexy font ------------%
%\usepackage{libertine}
%\usepackage{libertinust1math}

%\usepackage{mlmodern}				% very nice and classic
%\usepackage[utopia]{mathdesign}
%\usepackage[T1]{fontenc}


\usepackage{mlmodern}
\usepackage{eulervm}
%\usepackage{tgtermes} 				%times new roman
%-------- sexy font ------------%


% Problem Styles
%====================================================================%


\newtheorem{problem}{Problem}


\theoremstyle{definition}
\newtheorem{thm}{Theorem}
\newtheorem{lemma}{Lemma}
\newtheorem{prop}{Proposition}
\newtheorem{cor}{Corollary}
\newtheorem{fact}{Fact}
\newtheorem{defn}{Definition}
\newtheorem{example}{Example}
\newtheorem{question}{Question}

\newtheorem{manualprobleminner}{Problem}

\newenvironment{manualproblem}[1]{%
	\renewcommand\themanualprobleminner{#1}%
	\manualprobleminner
}{\endmanualprobleminner}

\newcommand{\penum}{ \begin{enumerate}[label=\bf(\alph*), leftmargin=0pt]}
	\newcommand{\epenum}{ \end{enumerate} }

% Math fonts shortcuts
%====================================================================%
\newcommand{\X}{\mathfrak{X}}
\newcommand{\ring}{\mathcal{R}}
\newcommand{\N}{\mathbb{N}}                           % Natural numbers
\newcommand{\Z}{\mathbb{Z}}                           % Integers
\newcommand{\R}{\mathbb{R}}                           % Real numbers
\newcommand{\C}{\mathbb{C}}                           % Complex numbers
\newcommand{\F}{\mathbb{F}}                           % Arbitrary field
\newcommand{\Q}{\mathbb{Q}}                           % Arbitrary field
\newcommand{\PP}{\mathcal{P}}                         % Partition
\newcommand{\M}{\mathcal{M}}                         % Mathcal M
\newcommand{\eL}{\mathcal{L}}                         % Mathcal L
\newcommand{\T}{\mathbb{T}}                         % Mathcal T
\newcommand{\U}{\mathcal{U}}                         % Mathcal U\\
\newcommand{\V}{\mathcal{V}}                         % Mathcal V

% symbol shortcuts
%====================================================================%
\newcommand{\ssubset}{\subset\joinrel\subset}
\newcommand{\bd}{\partial}
\newcommand{\grad}{\nabla}
\newcommand{\lam}{\lambda}
\newcommand{\imp}{\implies}
\newcommand{\all}{\forall}
\newcommand{\exs}{\exists}
\newcommand{\delt}{\delta}
\newcommand{\ep}{\varepsilon}
\newcommand{\ra}{\rightarrow}
\newcommand{\vph}{\varphi}

\newcommand{\ol}{\overline}
\newcommand{\f}{\frac}
\newcommand{\lf}{\lfrac}
\newcommand{\df}{\dfrac}
\newcommand{\st}{\text{ s.t. }}
% bracketting shortcuts
%====================================================================%
\newcommand{\abs}[1]{\left| #1 \right|}
\newcommand{\babs}[1]{\Big|#1\Big|}
\newcommand{\bound}{\Big|}
\newcommand{\BB}[1]{\left(#1\right)}
\newcommand{\dd}{\mathrm{d}}
\newcommand{\artanh}{\mathrm{artanh}}
\newcommand{\Med}{\mathrm{Med}}
\newcommand{\Cov}{\mathrm{Cov}}
\newcommand{\Corr}{\mathrm{Corr}}
\newcommand{\tr}{\mathrm{tr}}
\newcommand{\Range}[1]{\mathrm{range}(#1)}
\newcommand{\Null}[1]{\mathrm{null}(#1)}
\newcommand{\lan}{\left\langle}
\newcommand{\ran}{\right\rangle}
\newcommand{\norm}[1]{\left\lVert#1\right\rVert}
\newcommand{\inn}[1]{\lan#1\ran}
\newcommand{\op}[1]{\operatorname{#1}}
\newcommand{\bmat}[1]{\begin{bmatrix}#1\end{bmatrix}}
\newcommand{\pmat}[1]{\begin{pmatrix}#1\end{pmatrix}}
\newcommand{\vmat}[1]{\begin{vmatrix}#1\end{vmatrix}}

\newcommand{\amogus}{{\bigcap}\kern-0.8em\raisebox{0.3ex}{$\subset$}}
\newcommand{\Note}{\textbf{Note: }}
\newcommand{\Aside}{{\bf Aside: }}
%restriction
%\newcommand{\op}[1]{\operatorname{#1}}
%\newcommand{\done}{$$\mathcal{QED}$$}

%====================================================================%


\setlength{\parindent}{0pt}      	% No paragraph indentations
\pagestyle{fancy}
\fancyhf{}							% fancy header

\setcounter{secnumdepth}{0}			% sections are numbered but numbers do not appear
\setcounter{tocdepth}{2} 			% no subsubsections in toc

%template
%====================================================================%
%\begin{manualproblem}{1}
%Spivak.
%\end{manualproblem}

%\begin{proof}[Solution]
%\end{proof}

%----------- or -----------%

%\begin{problem} 		
%\end{problem}	

%\penum
%	\item
%\epenum
%====================================================================%


\newcommand{\Course}{MAT482}
\newcommand{\hwNumber}{0}

%preamble

\title{Elliptic PDEs}
\author{A.N.}
\date{\today}
\lhead{\Course A\hwNumber}
\rhead{\thepage}
%\cfoot{\thepage}


%====================================================================%
\begin{document}

\maketitle
\begin{tcolorbox}[colback = white]
\begin{defn}
	If $u$ is $C^2$ in $\Omega\subset \R^n$, we define the \textbf{Laplacian} of $u$ to be:
	$$ \Delta u = \sum_{i=1}^n \frac{ \partial^2 }{ \partial x_i^2 } u = \grad \cdot (\grad u) $$ 
\end{defn}
\end{tcolorbox}
The Laplacian is the prototypical linear elliptic operator of $2$nd order. 
\begin{tcolorbox}[colback = white]
\begin{defn}
	We say $u$ is \textbf{Harmonic} if $\Delta u = 0$. 
\end{defn}
\end{tcolorbox}
\begin{tcolorbox}[colback = white]
	\textbf{Properties}
	\begin{enumerate}[label = \roman*)]
		\item $\Delta$ is linear. If $c_1,c_2 \in \R$, then 
			$$ \Delta \left( c_1 u + c_2 v \right) = c_1 \Delta u + c_2 \Delta v. $$ 
		\item $\Delta$ is self adjoint. If $u,v$ completely supported in $\Omega$ i.e. $0$ near the boundary, 
			$$ \int_\Omega \left( \Delta u \right) v = - \int_\Omega \grad u \cdot \grad v = \int_\Omega u \left( \Delta v \right). $$ 
	\item $\Delta$ is translationally invariant. If $\Delta u  =f$, then
	$$ \Delta \left( u(x-a) \right) = f(x-a) = \left( \Delta u \right) (x-a).$$
\item $\Delta$ is rotationally invariant. If $A \in O(n)$ then $u_A = u(Ax)$ satisfies
	$$ \Delta u_A = f(Ax) = \left( \Delta u \right) (Ax). $$ 
	\item If $\lambda \neq 0$, then 
		$$ \Delta u(\lambda x) = \lambda^2 \left( \Delta u \right) \left( \lambda x \right). $$ 
\end{enumerate}
\end{tcolorbox}
We now give some examples of simple harmonic functions. 
\begin{tcolorbox}[colback = white]
\begin{example}
\end{example}
	\begin{enumerate}[label = \roman*)]
		\item 	Constant functions. $\Delta (1) = 0$
\item 	Linear functions. $\Delta( a\cdot x + b) = 0$. 
\item Radial Harmonic functions? $u = u(|x|)= u(r)$. The radial laplacian is given as
	$$ \Delta u = \frac{ \partial^2 u}{ \partial r^2 } + \frac{ n-1 }{ r } \frac{ \partial u }{ \partial r } = \frac{ \partial  }{ \partial r } \left( r^{n-1} \frac{ \partial u  }{ \partial r } \right). $$ 
Therefore if $\Delta u = 0$, then 
			$$ \frac{ \partial u  }{ \partial r  }= cr^{1-n} .$$
So 
			$$ u(r) = \begin{cases}
				r^{2-n} & n>2
				 \\ \log r & n=2
			\end{cases} $$ 
Notice that $u$ blows up at $0$, so $u$ not harmonic at $0$. In fact, $\Delta u = c \delta_{\{0\}}$ 
	\end{enumerate}
\end{tcolorbox}
\begin{tcolorbox}[colback = white]
\begin{defn}
A function $u$ on $\Omega$ is said to be: 
\begin{enumerate}[label = \roman*)]
	\item  \textbf{Subharmonic} if $\Delta u \geq 0$
	\item \textbf{Superharmonic} if $\Delta u \leq 0$. 
\end{enumerate}
\end{defn}
\end{tcolorbox}
\begin{tcolorbox}[colback = white]
\begin{example}
\end{example} $u(x) = \frac{ 1 }{ 2 }|x|^2$ is subharmonic. 
\end{tcolorbox}
Given $u$ harmonic and $v$ subharmonic on $\Omega$, with $u|_{\bd \Omega} = v|_{\bd \Omega}$, we have that $v \leq u$. So for given boundary data on $\Omega$, a harmonic function satisfying this data is the "biggest" subharmonic function this data. The following results make this more precise. 
\begin{tcolorbox}
\begin{lemma}
	\textbf{Mean Value Property}
\end{lemma}
If $u$ satisfies $\Delta u \geq 0$ in $\Omega$, then for any $x_0 \in \Omega$ and for any $r>0$ so that $B_r(x_0) \subset \Omega$, then:
\begin{enumerate}[label = \roman*)]
	\item $$u(x_0) \leq \frac{ 1 }{ n \omega_n } \int_{S^{n-1}}u(x_0 + r\omega) d\sigma(\omega) $$
	\item $$ u(x_0) \leq \frac{ 1 }{ \omega_n r^n } \int_{B_r(x_0)} u(y) dy. $$ 
\end{enumerate}
\end{tcolorbox}
\begin{proof}
	We first translate so that $x_0 = 0$, and rescale so that $\tilde{u}(x) = u(rx)$, so that we work on $B_1(0)$. We define the quantity 
	$$ g(s) = \frac{ 1 }{ |B_1(0)| } \int_{\bd B_1} u(s\omega) d \sigma(\omega). $$
	Clearly we have that $g(s) \to u(0)$ as $s \to 0$. We take the the derivative in $s$ of $u(s\omega)$. Get
	$$ \frac{ \partial  }{ \partial s } u(s \omega) = \omega \cdot \grad u (s\omega) , $$ 
Where $\omega$ is the unit normal to $S^{n-1}$. 
So,
\begin{align*}
	\frac{ d }{ ds } \int_{\bd B_1} u(s \omega ) d \sigma(\omega) & = \int_{\bd B_1} \grad_\omega u(sw) 
	\\ & = \int_{B_1} \grad \cdot \grad u(sx) dx \tag{Divergence Theorem}
	\\ &= s^2 \int_{B_1} \left( \Delta u \right) (sx) dx \geq 0.
\end{align*}
So 
	$$ u(0) \leq \frac{ 1 }{ |\bd B_1| } \int_{B_1} u(\omega) d \sigma(\omega). $$ 
Now, 
	$$ \int_{B_1} u(x) dx = \int_0^1 \int_{\bd B_1} u(s\omega) s^{n-1} ds d\sigma \geq \left| \int_0^1 u(0) s^{n-1} ds \right| \cdot |\bd B_1| = \frac{ 1 }{ n  } |\bd B_1| u(0). $$ 
As desired. 
\end{proof}
\begin{tcolorbox}
\begin{cor}
\end{cor}
If $\Delta u =0$, then 
\begin{enumerate}[label = \roman*)]
	\item $$ u(x_0) = \frac{ 1 }{ n \omega_n } \int_{S^{n-1}} u(x_0 + r \omega) d \sigma(\omega). $$
	\item  $$ u(x_0) = \frac{ 1 }{ \omega r^n } \int-{B_r(x_0)} u(y) dy. $$ 
\end{enumerate}
\end{tcolorbox}
\begin{proof}
If $\Delta u = 0$, then $\pm u$ is subharmonic.
\end{proof}
\begin{tcolorbox}
\begin{cor}
	\textbf{Strong Maximum Principle}
\end{cor}
	If $\Delta u \geq 0$ on $\Omega$, where $\Omega$ is open and connected, and $\ol{\Omega}$ is compact,  then 
	$$ u(x) \leq \sup_{x\in \bd \Omega} u, $$ 
	with equality holding if and only if $u$ is constant. 
\end{tcolorbox}
	\begin{proof}
		Take $x_0 \in \Omega$ so that $u(x_0) = \sup_\Omega u$. Choose $r>0$ sufficiently small so that $B_r(x_0) \subset \Omega$. Then
		$$ u(x_0) \leq \frac{ 1 }{ \omega_n r^n }\int_{B_r(x_0)} u \leq \sup u. $$ 
Since $u(x_0) = \sup u$ get that $u = \sup u$ on $B_r(x_0)$. Define
		$$ \Sigma = \left\{ x\in \Omega : u(x) = \sup_\Omega u \right\} .$$ 
		$\Sigma $ is open by above, and is closed by continuity of $u$. Therefore $\Sigma = \Omega$ by connectedness. 
	\end{proof}
\begin{tcolorbox}
\begin{cor}
	\textbf{Comparison Principle}
\end{cor}
	Suppose $\Omega$ open, $\ol{\Omega}$ compact. Then if $\Delta u = 0, \Delta v \geq 0$, and 
$u|_{\bd \Omega} = v|_{\bd \Omega}$, then $u(x) \geq v(x)$ for all $x$ and equality holds if and only if $u = v$. 
\end{tcolorbox}
\begin{proof}
	$\Delta (v-u) \geq 0$ and $v-u|_{\bd \Omega} = 0$, so $v-u \leq 0$ in $\Omega$ follows from the strong maximum principle. 
\end{proof}
\begin{tcolorbox}
\begin{cor}
	\textbf{Uniquness of Solution to Dirichlet Problem}
\end{cor}
	If $\Omega \subset \R^n$ open, $\ol{\Omega}$ compact, $g: \bd \Omega \to \R$, then there is at most one unique function satisfying:
	$$ \begin{cases}
		\Delta u & = 0
		\\ u|_{\bd \Omega} & = g
	\end{cases} $$ 
\end{tcolorbox}
\begin{proof}
Let $u_1, u_2$ both satisfy the Dirichlet Problem. Apply the strong maximum principle to $u_1-u_2$, 
$u_2-u_1$. 
\end{proof}
\begin{tcolorbox}
\begin{lemma}
	\textbf{Higher Order Regularity}
\end{lemma}
	If $\Delta u =0$ in $\Omega$, then $u \in C^\infty (\Omega)$. 
\end{tcolorbox}
\begin{proof}
	Let $\varphi(x) = \varphi(|x|)$ be a smooth function so that $\varphi \geq 0$, $\varphi \equiv 0$ for $|x|\geq 1$, and 
	$$ n \omega_n \int_0^1 r^{n-1} \varphi(r) dr = 1 = \int_{B_1(0)} \varphi(x) dx. $$ 
	For $y\in \R^n$ consider $ \frac{ 1 }{ \ep^n } \varphi \left( \frac{ x-y }{ \ep } \right)$ supported in $B_\ep (y)$. Note that 
	$$ \frac{ 1 }{ \ep^n  }\in_{B_\ep (y)} \varphi \left( \frac{ x-y }{ \ep } \right)dx = 1. $$ 
	Let $\Omega_\ep = \left\{ y\in \Omega : d(, \bd \Omega) > \ep \right\}$. 
	Define 
	$$ v(y) = \frac{ 1 }{ \ep^n  }\int u(x) \varphi \left( \frac{ x-y }{ \ep }  \right)dx. $$ 
(This is called the "Mollification" of $u$). Note that this is $C^\infty$ since $\varphi$ is. 
	We claim that $v(y) = u(y)$. By translation, assume that $y=0$. 
	Then, 
	\begin{align*}
		v(0) & = \frac{ 1 }{ \ep^n  }\int u(x) \varphi (|x|/\ep) dx
		\\ & = \frac{ 1 }{ \ep^n } \int_0^\ep \int_{S^{n-1}} u(s \omega) \varphi(s/\ep) ds d\sigma(\omega)
		\\ & = n|B_1(0)| u(0) \frac{ 1 }{ \ep^n  }\int_0^\ep \phi (s/\ep)s^{n-1} ds \tag{Mean Value Property}
		\\ & = u(0) \tag{by definition of $\varphi$}
	\end{align*}
\end{proof}
In general we will see if $\Delta u = f$, then $u$ has two more derivatives than $f$. 
\begin{tcolorbox}
\begin{lemma}
	\textbf{Gradient Estimate}
\end{lemma}
	Suppose $u$ harmonic on $B_r(0) \subset \R^n$. Then 
	$$ |\grad u|(0) \leq \frac{ n }{ r } \sup_{\bd B_r} |u|. $$ 
\end{tcolorbox}
	\begin{proof}
	Since $u$ is $C^\infty$ we can differentiate the equation $\Delta u = 0$ to get $\Delta u_i = 0$. By the mean value property, we have
		$$ \frac{ \partial u }{ \partial x_i }(0) = \frac{ 1 }{ \omega_n r^n }\int_{B_r} \frac{ \partial u }{ \partial x_i } dy = \frac{ 1 }{ \omega_n r^n } \int_{B_r} \grad \cdot (u e_i)dy. $$
Where $e_i$ is the $i$th standard basis vector. 
By the divergence theorem, 
		$$ \int_{B_r} \grad \cdot \left( u e_i \right) dy = \int_{\bd B_r} ue_i \cdot \vec{\nu} d\sigma $$ 
with $\vec{\nu} = \frac{ x }{ |x| }$. 
Therefore
\begin{align*}
	\left| \frac{ \partial u }{ \partial x_i  }(0) \right| & \leq \frac{ 1 }{ \omega_n r^n } \int_{\bd B_r} |u| d \sigma 
	\\ & \leq \frac{ 1 }{ \omega_n r^n } n \omega_n r^{n-1} \sup_{\bd B_r} |u|
	\\ & = \frac{ n }{ r }\sup_{\bd B_r} |u|
\end{align*}
		By rotation, we can assume that $\grad u(0) = |\grad u(0)| e_1$, so the result follows. 
	\end{proof}
\begin{tcolorbox}
\begin{cor}
	\textbf{Liouville Theorem}
\end{cor}
If $u$ harmonic on all of $\R^n$ and $u$ bounded then $u$ is constant. 
\end{tcolorbox}
\begin{proof}
For any $x\in \R^n$, $r>0$ we have
	$$ |\grad u(x)| \leq \frac{ n }{ r }\sup_{\bd B_r(x)} |u| \leq \frac{ n }{ r } \sup_{\R^n} |u| \leq \frac{ nC }{ r } $$
for some $C$. Taking $r\to \infty$ tells us that $|\grad u(x)| = 0$. Since $x$ arbitrary, $u$ is constant. 
\end{proof}
\begin{tcolorbox}
\begin{lemma}
	\textbf{Real Analycity}
\end{lemma}
Suppose that $\Delta u = 0$ in $\Omega \subset \R^n$. Then $u$ analytic in $\Omega$. 
\end{tcolorbox}
\begin{proof}
	By translating and rescaling, we can assume that $\Delta u = 0$ in $B_1(0)$. It suffices to show that $u$ has a convergent power series at $0$.
We make the following claim: 
	For all $m \in \N$, for any multi-index $\alpha = (\alpha_1, \dots, \alpha_n)$ with size $m$,
	$$ \sup_{B_{1/2}(0)} \left| D^\alpha u(x) \right| \leq (2ne)^m m!\sup_{B_1}|u|.$$ 
	Fix an $m$. Let $r_k = \frac{ 1 }{ 2 } + \frac{ m-k }{ 2m }$. Consider $B_{r_m} (0) = B_{1/2}(0) \subset B_{r_{m-1}} (0) \subset \dots \subset B_1(0) = B_{r_0}(0)$. 
	Note that if $x\in B_{r_k}(0)$, then $B_{1/2m}(x) \subset B_{r_{k-1}}$
	By the gradient estimate, we have that 
	$$ \sup_{{B_{1/2}}(0)} \left| D^\alpha u \right| \leq \frac{ n }{ 1/2m } \sup_{B_{r_{m-1}}} \left| D^{\alpha^\prime} u \right|. $$ 
	Where $\alpha^\prime  = (\alpha_1 , \dots , \alpha_i - 1, \dots , \alpha_n)$ for some index $i$. We iterate this estimate and get
	$$ \sup_{B_{1/2}(0)} \left| D^\alpha u  \right| \leq n^m 2^m m^m \sup_{B_1} |u|. $$ 
	By Sterlings formula, $m^m \leq m! e^m$ so
	$$ \sup_{B_{1/2}(0)} \left| D^\alpha \right| \leq (2ne)^m m! \sup_{B_1(0)}|u|. $$
	This proves the claim. 
	Now to prove the lemma, let $|x| < \frac{ 1 }{ 4ne }$. Let $P_m(x)$ be the $m$-th order taylor polynomial at $0$. Consider $\sup_{B_{ \frac{ 1 }{ 4ne }}(0)} |u-P_m|$. By a rotation, we can assume that the $sup$ is achieved at $x = (x_1,0,  \dots , 0)$ , $|x_1| \leq \frac{ 1 }{ 4ne }$. 
	By Taylor's Theorem, there exists a $\theta \in [0,1]$ so that 
	\begin{align*}
	\left| u(x) - P_m(x) \right| & = \frac{ x^{m+1} }{ (m+1)! } \left| D^{m+1} u(\theta x) \right|
		\\ & \leq x^{m+1} (2ne)^{m+1} \sup_{ B_1(0) } |u|
		\\ & \leq \frac{ 1 }{ 2^{m+1} } \sup_{B_1(0)} |u| \to 0. 
	\end{align*}
\end{proof}
\begin{tcolorbox}
\begin{lemma}
	\textbf{Strong Unique Continuation}
\end{lemma}
	If $u,v$ harmonic functions in $B_1(0)$, and $D^\alpha u(0) = D^\alpha v (0)$ for all $\alpha, |\alpha| \geq 0$, then $u = v$. 
\end{tcolorbox}
\begin{proof}
$u$ and $v$ are real analytic. 
\end{proof}
Aside: How do we prove the excistence of harmonic functions?
\begin{tcolorbox}[breakable, colback = white]
	\textbf{Strategies}
\begin{enumerate}[label = \roman*)]
	\item Viscocity or Envelope Approach:
		\newline Idea: By comparison principle, we know that if $u|_{\bd \Omega} = v|_{\bd \Omega}$ and $\Delta v \geq \Delta u$, ten $v \geq u$ in $\Omega$. So if $u$ is the largest subharmonic function with given boundary values, it must be harmonic. 
		So
		$$ u = \sup \left\{ v: \ol{\Omega} \to \R ,v\in C^0 : v|_{\bd \Omega} = g, \Delta v \geq 0 \right\}. $$
		The upside of this strategy is it is robust, only requires comparison principle. On the other hand it may only produce a continuous function, not $C^2$. 
	\item Energy Methods: 
		\newline Idea: Consider the energy functional :
		$$ v \mapsto \frac{ 1 }{ 2 } \int_{\Omega} \left| \grad v \right|^2  = E(v) $$
		on the space of functions satisfying $v|_{\bd \Omega} = g$. Note that if $u$ is a critical point of $E$ then for any smooth $\phi$ with compact support in $\Omega$ we have
		$$ 0 = \frac{ d }{ dt } \Big|_{t = 0} E(u + t\phi) = \int \grad u \cdot \grad \phi. $$ If we can integrate by parts, then 
		$$ \int_\Omega \grad u \cdot \grad \phi = \int_\Omega \Delta u \cdot \phi = 0 \implies  \Delta u = 0.]
	$$ 
		The strategy is to minimize $E(v)$. Downsides: 
		To guarantee that $E(v)$ has a minimum, we need some form of compactness for the space of fucntions $v$ (Sobolev Spaces). The solution will also be some weekly differentiable function; we need to prove regularity. This only works for equations of divergence type. The upside is that it doesnt require maximum principle, and works for elliptic systems. (Calculus of Variations)
	\item Method of Continuity/ A priori Estimates. 
Idea: start with some equation you can solve and deform it into  the equation you want to solve. eg:
$$ \begin{cases}
	\Delta u & = 0
	\\ u|_{\bd \Omega} & = g.
\end{cases} $$ 
		Chose some function $v_0$ so that $v_0|_{\bd \Omega} = g, \Delta v_0 = f_0$. 
		Consider:
		$$ (*_t) \begin{cases}
			\Delta v_t & = (1-t)f_0
			\\ v_t |_{\bd \Omega} &= g
		\end{cases} $$ 
		Let $I = \left\{ t \in [0,1] : (*_t) \text{ has a smooth solution} \right\}	$

\end{enumerate}
We make the following observations:
\begin{enumerate}[label = \roman*)]
	\item $0\in I$ by construction .
	\item I is open, by the Implicit Function Theorem for Banach Spaces
	\item I is closed. This is the main step!!!!!
		The task is to show that $v_t$ satisfy uniform estimates and take limits. (if $ \left| v_t \right|_{C^3}$ then  $v_t \to v_{t^\ast} \in C^2$ and so the equation holds. )
Upside: Always works with mooth functions. The downside is that we need to proce a lot of estimates to get excistence. We also need to deal with more general equation, 
$\Delta v = f$ rather than $\Delta v = 0$. The point is that excistence of solutions to PDE's is intimately tied to estimates for solutions. 
\end{enumerate}
\end{tcolorbox}
Here is an example of weak regularity $\implies $ strong regularity. 
\begin{tcolorbox}
\begin{lemma}
	Suppose $u: \Omega \to \R$, $\int u^2 < \infty$ and $\int u \cdot \Delta \varphi = 0$ for all $\varphi \in C^\infty(\Omega)$ with compact support. Then $u\in C^\infty$ and $\Delta u = 0$. 
\end{lemma}
\end{tcolorbox}
\begin{proof}
	Idea: make a clever choice for $\varphi$ ($n>2$ for simplicity). Recall that $\Delta \left( |x|^{2-n} \right) = 0$ for $x \neq 0$. Consider the function:
	$$ \rho_r(x) = \begin{cases}
		|x|^{2-n} & |x| \geq r 

		\\
		r^{2-n} + \frac{ n-2 }{ 2 }r^{-n} \left( r^2 - |x|^2 \right) & |x| < r
	\end{cases} $$
	Note that $\rho_r \in C^1$ but not $C^2$. Furthermore, $ \left| D^2 \rho_r \right|$ bounded so $\rho_r \in C^{1,1}$. 
	Now taking $r_2 < r_1 <1$, consider $\varphi= \rho_{r_1} - \rho_{r_2}$. We have that 
	$$ \Delta \varphi = \frac{ n-2 }{ r_2^n  } \chi_{B_{2_r}} - \frac{ n-2 }{ r_1^n }\chi_{B_{r_1}}. $$ 
	We claim that $0 = \int_{B_1} u\cdot  \Delta \varphi $. First , let $\psi_n$ be a sequence of smooth compactly supported functions so that $ \left| D^2 \psi_n \right| <C$ and $\Delta \psi_n \to \Delta \varphi$. This can be done since $\Delta \varphi $ is bounded and compactly supported. 
Thus for all $n$, we have:
$$ 0 = \int_\Omega u \cdot \Delta \psi_n .  $$
We have that $|u \Delta \psi_n 
	|\leq Cn |u|$ so by the dominated convergence theorem $\int u \Delta \psi_n \to \int u \Delta \varphi$. 
\end{proof}
\begin{tcolorbox}[colback = white]
\begin{defn}
	We define $$\eL^1_{loc}(\Omega) = \left\{ u: \Omega \to \R \text{ integrable}: \forall p \in \Omega , \exists r>0 \st \int_{B_r(p)} |u| <\infty \right\}$$
\end{defn}
\end{tcolorbox}
We are going to define a notion of weak differentiability using integration by parts. 
\begin{tcolorbox}[colback = white]
\begin{defn}
	Say $u,v \in \eL^1_{loc}(\Omega)$, and $\alpha = \left( \alpha_1, \dots , \alpha_n \right)$ is a multi-index of length $|\alpha| =  \sum_i \alpha_i$. We say that $v$ is the weak derivative of $u$, written 
	$$ v = D^\alpha u = \frac{ \partial^{|\alpha|} u }{ \partial x_1^{\alpha_1} \dots \partial x_n^{\alpha_n} } $$ if: 
	$$ \int_\Omega u D^\alpha \varphi = (-1)^{|\alpha|} \int_\Omega v \varphi dx$$
	for all $\varphi \in C_c^\infty(\Omega)$. 
\end{defn}
\end{tcolorbox}
The idea is that if $u$ is already $|\alpha|$ times differentiable, then $v = D^\alpha u$ by integration by parts. 
Furthermore, if for some $v, \tilde{v}$ the definition holds then $v = \tilde{v}$ a.e.
\begin{tcolorbox}
\begin{lemma}
	Suppose that $v , \tilde{v}$ are $\alpha$'th weak derivatives of $u$. Then $v = \tilde{v} $ a.e.
\end{lemma}

\end{tcolorbox}
\begin{proof}
Let $\varphi \in C_c^\infty$. Then, 
	$$ \int v \varphi = (-1)^{|\alpha|} \int u D^\alpha \varphi = \int \tilde{v} \varphi $$ 
	so $\int (v - \tilde{v} ) \varphi = 0 $ for all $\varphi \in C_c^\infty$. Therefore $v = \tilde{v} $ a.e. 
\end{proof}
\begin{tcolorbox}[breakable, colback = white]
\begin{example}
	let $u(x) = |x|$. $u$ is not differentiable at $0$. We claim that $u$ is weak differentiable, and 
	$$ \frac{ du }{ dx } = H(x) = 
	\begin{cases}
		-1 & x \leq 0 \\ 1 & x>0
	\end{cases} $$ 
	Take $\varphi \in C_c^\infty (-1,1)$. Then, 
	\begin{align*}
		\int_{-1}^1 u \frac{ \partial \varphi }{ \partial x } & = \int_{-1}^0 (-x) \frac{ \partial \varphi }{ \partial x } + \int_0^1 x \frac{ \partial \varphi }{ \partial x } 
		\\ & = -x \varphi \Big|_{-1}^0 + \int_{-1}^0 \varphi  + x \varphi\Big|_{0}^1 - \int_0^1 \varphi 
		\\ & = - \int_{-1}^1 H(x) \varphi 
	\end{align*}
	We now claim that $H$ is not weakly differentiable. 
	Take $\varphi \in C_c^\infty(-1,1)$, then 
\begin{align*}
	\int_{-1}^1 H(x) \varphi(x) & = -\int_{-1}^0 \frac{ \partial \varphi  }{ \partial x } + \int_0^1 \frac{ \partial \varphi  }{ \partial x } 
	\\ & = -2 \varphi(0)
\end{align*}
	Thus if $ \frac{ dH }{ dx } = v$, then $v(x) = 0$ for all $x \neq 0$. So 
	$$ \int v(x) \varphi(x) = 2 \varphi(0) $$ 
	This is not in $L_{loc}^1$. In fact, $ \frac{ dH }{ dx }$ exists as a distribution. 
\end{example}
\end{tcolorbox}
Onwards, we will fix $1 \leq p \leq\infty$ and let $k \geq 0$. Recall that 
$$ \eL^p(\Omega) = \left\{ u : \Omega \to \R : \int_{\Omega} |u|^p < \infty \right\} .$$
\begin{tcolorbox}[colback = white]
\begin{defn}
	The \textbf{Sobolev Space} $W^{k,p}(\Omega)$ is the set of $u\in \eL^1_{loc}$ such that for all multi-indices $\alpha$ with  $|\alpha|\leq k$ $D^\alpha u$ exists and belongs to $\eL^p(\Omega)$. If $p = 2$, it is common to write $W^{k,2} = H^k$ where $H$ stands for "hilbert space".
\end{defn}
\end{tcolorbox}
\begin{tcolorbox}[colback = white]
\begin{defn}
	We define a norm on $W^{k,2p}(\Omega)$ by:
	$$ \norm{u}_{W^{k,p}(\Omega)} =  \begin{cases}
	\sum_{|\alpha| \leq k} \left(  \int_{\Omega} \left| D^\alpha u \right|^p \right)^{	\frac{ 1 }{ p }	 } &  p < \infty
		\\ \sum_{ |\alpha| \leq k} \norm{D^\alpha u}_{\eL^\infty(\Omega)} & p = \infty 
	\end{cases}$$ 
	Where $ \norm{v}_{\eL^\infty} = ess \sup_\Omega |v|$. 
\end{defn}
	Note $ \left( W^{k,p} (\Omega), \norm{\cdot}_{W^{k,p}(\Omega)} \right)$ is a normed vector space. 
\end{tcolorbox}
\begin{tcolorbox}[colback = white]
\begin{defn}
	Let $ \left\{ u_m \right\}  $ be a sequence in $W^{k,p}(\Omega)$. 
	\begin{enumerate}[label = \roman*)]
		\item We say $u_m$ converges to $u$ in $W^{k,p}( \Omega)$ if $ \norm{u_m - u}_{W^{k,p}} \to 0$ as $m \to \infty$. 
		\item We say $u_m \to u$ in $W^{k,p}_{loc}(\Omega)$ if $u_m \to u$ in $W^{k,p}(V)$ for all $V \ssubset \Omega$. 
		\item We denote $W_o^{k,p}(\Omega)$, the closure of $C_c^\infty(\Omega)$ in $W^{k,p}(\Omega)$. 
	\end{enumerate}
	Roughly speaking, $W_o^{k,p}$ is the set of all functions $u\in W^{k,p}$ such that $D^\alpha u = 0 $ on $\bd \Omega$ for all $|\alpha| \leq k$. 
\end{defn}
\end{tcolorbox}

\begin{tcolorbox}
\begin{thm}
Elementary Properties of Sobolev Spaces
\end{thm}
	Assume $u,v \in W^{k,p}(\Omega)$, $|\alpha| \leq k$. Then: 
	\begin{enumerate}[label = \roman*)]
		\item $D^\alpha u \in W^{k-|\alpha|, p}(\Omega)$ and $D^\beta (D^\alpha u) = D^\alpha(D^\beta u) = D^{\beta + \alpha} u$, for all $\alpha, \beta$ with $|\alpha| + |\beta| \leq k$. 
		\item If $\Omega^\prime \subset \Omega$ open then $u\in W^{k,p}(\Omega^\prime)$. 
		\item If $\eta \in C_c^\infty(\Omega)$, then $\eta u \in W^{k,p}(\Omega)$, and
			$D^\alpha (\eta u) = \sum_{\beta \leq \alpha} \pmat{|\alpha |\\ |\beta|} D^\beta \eta D^{\alpha - \beta} u$. 
	\end{enumerate}
\end{tcolorbox}
\begin{proof}
aaa
\end{proof}
\begin{tcolorbox}[colback = white]
\begin{defn}
	Let $\X$ be a vector space over a field (say $\R$ or $\C$). A \textbf{norm} on $\X$ is a map
	$$ \norm{\cdot} : \X \to [0 , \infty) $$ such that
\begin{enumerate}[label = \roman*)]
	\item $\norm{u+v} \leq \norm{u} + \norm{v}.$
	\item $\norm{\lambda u} = |\lambda| \cdot \norm{u}, \forall \lambda \in \F$
	\item $\norm{u} = 0 \iff u = 0$
\end{enumerate}	
	If $(\X, \norm{\cdot})$ is a normed vector space, can define a metrix space structure on $\X$ by $d(x,y) = \norm{x -y}$. This gives rise to a topology and a notion of convergence. 
\end{defn}
\end{tcolorbox}
\begin{tcolorbox}[colback = white]
\begin{defn}
 Banach Space
\end{defn}
	A Banach space is a vector space $\X$ over field ($\R$ or $\C$) such that $(\X, \norm{\cdot })$ defines a complete metric space. (sequence converges iff sequence is cauchy). 
	
\end{tcolorbox}
\begin{tcolorbox}[colback = white]
\begin{example}
\end{example}
\begin{enumerate}[label = \roman*)]
	\item $\R^n$ with $ \norm{x} = \sqrt{x\cdot x}$ is a banach space ( in fact all finite dimensional normed vectors spaces are banach spaces. )
	\item $C^0([0,1])$, continuous functions on $[0,1]$, with $ \norm{f} = \sup |f|$. 
	\item $\X = \left\{ f \in C^0([0,1]) : \frac{ df }{ dx } \text{ exists on } (0,1) \right\}$. Give $\X$ the $ \norm{\cdot}_{C^0}$ norm. 
\end{enumerate}
\end{tcolorbox}
Claim: $ \left( C^1([0,1]), \norm{\cdot} \right)$ is not complete. Consider
$$ f_\ep(x) = \begin{cases}
	-x  & x \leq - \ep
	\\ \frac{ 1 }{ 2\ep }x^2 + \frac{ \ep }{ 2 } & -\ep \leq x \leq \ep
	\\ x & x \geq \ep
\end{cases}  $$ 
Then, $ \left\{ f_{1/n} \right\}$ is cauchy but $f_{1/n} \to |x| \not \in \X$. 
\begin{tcolorbox}[colback = white]
\begin{thm}
	$W^{k,p}$ is a banach space. 
\end{thm}
\end{tcolorbox}
\begin{proof}
	Follows from the fact that $\eL^p(\Omega)$ is a banach space. 
\end{proof}
\begin{tcolorbox}[colback = white]
\begin{thm}
Local Approximation Theorem
\end{thm}
	Let $\ep >0$, $\Omega_\ep = \left\{ x \in \Omega : d(x, \bd \Omega) > \ep \right\}. $
	If $u\in W^{k,p}$ then there exists $u_\ep \in C^\infty(\Omega_\ep)$ so that $u_\ep \to u$ in $W^{k,p}_{loc}(\Omega)$. 
\end{tcolorbox}
\begin{proof}
	The proof is based on mollification. Let $\eta(x) = \eta(|x|)$ be a $C^\infty$ function such that $\eta \geq 0$, $\eta \equiv 0$ for $|x|\geq 1$ and $\int_{\R^n} \eta(|x|) dx = 1$. Define $\eta_\ep(x) = \frac{ 1 }{ \ep^n } \eta \left( \frac{ x }{ \ep } \right)$. Note that: $$ \begin{cases}
		\int_{\R^n} \eta_\ep(x) dx = 1
		\\ \text{supp} \eta_\ep(x) \subset B_\ep(0)
	\end{cases} $$
	Claim: if $u \in \eL^p(\Omega)$, then $u_\ep(x) = \int_\Omega \eta_\ep(x-y) u(y) dy = \eta_\ep \ast u$ is defined in $\Omega_\ep$ and $u_\ep \in C^\infty(\Omega_\ep)$ and $u_\ep \to u$ in $\eL^p_{loc}(\Omega)$ as $\ep \to 0$.
\begin{enumerate}[label = \roman*)]
	\item 
	First, note that the integral is well defined since :
	$$ \left| \int_\Omega \eta_\ep(x-y)u(y) dy \right| \leq \sup |\eta_\ep| \int_{B_\ep(x)} |u| dy \leq C \norm{u}_{\eL^p}.$$
	Where the second inequality follows from Cauchy-Schwartz. 
\item Let $h$ be small, $e_i = (0, \dots , 1 ,\dots 0)$ standard basis vector. We have:
	$$ \frac{ 1 }{ h } u_\ep(x+ h e_i) - u_\ep(x)) = \int_\Omega \frac{ 1 }{ h } \left( \eta_\ep(x + he_i - y) - \eta_\ep (x-y) \right)u(y) dy.  $$ 
		Since $\eta_\ep$ is in $C^\infty$, $ \frac{ 1 }{ h } \left( \eta_\ep(x + he_i - y) - \eta_\ep(x-y) \right)  \rightrightarrows \frac{ \partial \eta_\ep }{ \partial x_i  }(x-y)$. Thus 
		$$ \frac{ \partial u_\ep }{ \partial x_i }(x) = \int \frac{ \partial \eta_\ep  }{ \partial x_i }(x-y) u(y) dy.$$
		Applying this reasoning inductively, we have that $u_\ep \in C^\infty$. 
		\end{enumerate}
		Next, we claim that if $u \in \eL^p(\Omega)$, then $u_\ep = \eta_\ep \ast u \in \eL^p_{loc} (\Omega) and u_\ep \to u$ in $\eL^p_{loc}(\Omega)$. To prove this, we need $3$ steps.
		\begin{enumerate}[label = \roman*)]
			\item continuous functions
			\item $u_\ep \in \eL_{loc}^p(\Omega_ep)$
			\item $u_\ep \to u$ in $\eL^p_{loc}(\Omega)$. 
		\end{enumerate}
		Step 1: Say $u$ is continuous. Then $u_\ep \to u$ uniformly in compact sets. 
		\begin{proof}
		If $u$ is continious, then for all $V \ssubset \Omega_\ep$,
	\begin{align*}
		|u_\ep(x) - u(x)| & = \left| \int \eta_\ep(x-y) \left( u(y) - u(x) \right) dy \right|
		\\ & = \left| \int_{|w|\leq 1} \eta_\ep(w) \left( u(x+ \ep w) - u(x) \right)dw \right|\tag{sub $y = x + \ep w$}
	\end{align*}
			By uniform continuity, $ \left| u(x + \ep w) - u(x) \right| \leq \delta$ for all $|w| \leq 1$, for $x\in V$. 
			Therefore $|u_\ep(x) - u(x)| \leq \delta$ since $\int \eta  = 1$. So $u_\ep \rightrightarrows u$ uniformly on $V$. 
		\end{proof}
		Step 2: If $u\in \eL_{loc}^p(\Omega)$, then for all $V \ssubset W \ssubset \Omega $we have $ \norm{u_\ep}_{\eL^p(v)} \leq \norm{u}_{L^p(w)}. $
	\begin{proof}
		If $x\in V$, for $\ep$ sufficiently small, 
		\begin{align*}
			|u_\ep(x)|  & = \left| \int_{y\in \Omega} \eta_\ep (x-y) u(y) dy \right|
			\\ & \leq \int_{y\in \Omega} \left| \eta_\ep(x-y) \right||u(y)| dy
			\\ & \leq \left( \int \eta_\ep(x-y) |u(y)|^p dy \right)^{1/p} \overbrace{\left( \int \eta(x-y) dy \right)^{1- 1/p}}^{=1}
		\end{align*}
		Thus:
		$$ \int_{x\in V} |u_\ep(x)|^p \leq \int_V \int_W \eta_\ep(x-y) |u(y)|^p dy dx = \int_W |u(y)|^p dy $$ 
		Since $\int_V \eta_\ep(x-y) \leq 1$. 
	\end{proof}
	Step 3: $u_\ep \to u$ in $\eL^p_{loc}(\Omega)$. 
	\begin{proof}
		Fix $\tilde{V} \ssubset V \ssubset \Omega_\ep$. Fix $\delta >0$. 
		By a general result from measure theory, can find $g\in C(V)$ such that $ \norm{u-g}_{\eL^p(V)} <  \delta$. Then. 
		\begin{align*}
			\norm{u - u_\ep}_{\eL(\tilde{V})} & \leq \norm{u - g}_{L^p(\tilde{V})} + \norm{u_\ep - g_\ep}_{\eL^p(\tilde{V})} + \norm{g_\ep - g}_{\eL^p(\tilde{V})}\
			\\ & \leq 2 \norm{u - g}_{\eL^p(V)} + |\tilde{v}|^{1/p} \sup_{\tilde{V}} |g_\ep - g| \tag{step 2, Cauchy-Schwartz}
		\end{align*}
	\end{proof}
	Since $g_\ep \to g$ uniformly on $\tilde{V}$ by step 1, and $ \norm{u - g}_{\eL^P(\tilde{V})} < \delta$ we get that $u_\ep \to u$ in $\eL^p(\tilde{V})$ as desired. 

\end{proof}
\end{document}
