\documentclass[12pt, a4paper]{article}
\usepackage[lmargin =0.75 in, 
rmargin=0.75in, 
tmargin=1in,
bmargin=0.5in]{geometry}
\geometry{letterpaper}
\usepackage{tikz-cd}
\usepackage{amsmath}
\usepackage{amssymb}
\usepackage{blindtext}
\usepackage{titlesec}
\usepackage{enumitem}
\usepackage{fancyhdr}
\usepackage{amsthm}
\usepackage{graphicx}
\usepackage{cool}
\usepackage{thmtools}
\usepackage{hyperref}
\usepackage{tcolorbox}
\graphicspath{ }					%path to an image

%-------- sexy font ------------%
%\usepackage{libertine}
%\usepackage{libertinust1math}

%\usepackage{mlmodern}				% very nice and classic
%\usepackage[utopia]{mathdesign}
%\usepackage[T1]{fontenc}


\usepackage{mlmodern}
\usepackage{eulervm}
%\usepackage{tgtermes} 				%times new roman
%-------- sexy font ------------%


% Problem Styles
%====================================================================%


\newtheorem{problem}{Problem}


\theoremstyle{definition}
\newtheorem{thm}{Theorem}
\newtheorem{lemma}{Lemma}
\newtheorem{prop}{Proposition}
\newtheorem{cor}{Corollary}
\newtheorem{fact}{Fact}
\newtheorem{defn}{Definition}
\newtheorem{example}{Example}
\newtheorem{question}{Question}

\newtheorem{manualprobleminner}{Problem}

\newenvironment{manualproblem}[1]{%
	\renewcommand\themanualprobleminner{#1}%
	\manualprobleminner
}{\endmanualprobleminner}

\newcommand{\penum}{ \begin{enumerate}[label=\bf(\alph*), leftmargin=0pt]}
	\newcommand{\epenum}{ \end{enumerate} }

% Math fonts shortcuts
%====================================================================%

\newcommand{\ring}{\mathcal{R}}
\newcommand{\N}{\mathbb{N}}                           % Natural numbers
\newcommand{\Z}{\mathbb{Z}}                           % Integers
\newcommand{\R}{\mathbb{R}}                           % Real numbers
\newcommand{\C}{\mathbb{C}}                           % Complex numbers
\newcommand{\F}{\mathbb{F}}                           % Arbitrary field
\newcommand{\Q}{\mathbb{Q}}                           % Arbitrary field
\newcommand{\PP}{\mathcal{P}}                         % Partition
\newcommand{\M}{\mathcal{M}}                         % Mathcal M
\newcommand{\eL}{\mathcal{L}}                         % Mathcal L
\newcommand{\T}{\mathbb{T}}                         % Mathcal T
\newcommand{\U}{\mathcal{U}}                         % Mathcal U\\
\newcommand{\V}{\mathcal{V}}                         % Mathcal V

% symbol shortcuts
%====================================================================%

\newcommand{\bd}{\partial}
\newcommand{\grad}{\nabla}
\newcommand{\lam}{\lambda}
\newcommand{\imp}{\implies}
\newcommand{\all}{\forall}
\newcommand{\exs}{\exists}
\newcommand{\delt}{\delta}
\newcommand{\ep}{\varepsilon}
\newcommand{\ra}{\rightarrow}
\newcommand{\vph}{\varphi}

\newcommand{\ol}{\overline}
\newcommand{\f}{\frac}
\newcommand{\lf}{\lfrac}
\newcommand{\df}{\dfrac}

% bracketting shortcuts
%====================================================================%
\newcommand{\abs}[1]{\left| #1 \right|}
\newcommand{\babs}[1]{\Big|#1\Big|}
\newcommand{\bound}{\Big|}
\newcommand{\BB}[1]{\left(#1\right)}
\newcommand{\dd}{\mathrm{d}}
\newcommand{\artanh}{\mathrm{artanh}}
\newcommand{\Med}{\mathrm{Med}}
\newcommand{\Cov}{\mathrm{Cov}}
\newcommand{\Corr}{\mathrm{Corr}}
\newcommand{\tr}{\mathrm{tr}}
\newcommand{\Range}[1]{\mathrm{range}(#1)}
\newcommand{\Null}[1]{\mathrm{null}(#1)}
\newcommand{\lan}{\left\langle}
\newcommand{\ran}{\right\rangle}
\newcommand{\norm}[1]{\left\lVert#1\right\rVert}
\newcommand{\inn}[1]{\lan#1\ran}
\newcommand{\op}[1]{\operatorname{#1}}
\newcommand{\bmat}[1]{\begin{bmatrix}#1\end{bmatrix}}
\newcommand{\pmat}[1]{\begin{pmatrix}#1\end{pmatrix}}
\newcommand{\vmat}[1]{\begin{vmatrix}#1\end{vmatrix}}

\newcommand{\amogus}{{\bigcap}\kern-0.8em\raisebox{0.3ex}{$\subset$}}
\newcommand{\Note}{\textbf{Note: }}
\newcommand{\Aside}{{\bf Aside: }}
%restriction
%\newcommand{\op}[1]{\operatorname{#1}}
%\newcommand{\done}{$$\mathcal{QED}$$}

%====================================================================%


\setlength{\parindent}{0pt}      	% No paragraph indentations
\pagestyle{fancy}
\fancyhf{}							% fancy header

\setcounter{secnumdepth}{0}			% sections are numbered but numbers do not appear
\setcounter{tocdepth}{2} 			% no subsubsections in toc

%template
%====================================================================%
%\begin{manualproblem}{1}
%Spivak.
%\end{manualproblem}

%\begin{proof}[Solution]
%\end{proof}

%----------- or -----------%

%\begin{problem} 		
%\end{problem}	

%\penum
%	\item
%\epenum
%====================================================================%


\newcommand{\Course}{MAT482}
\newcommand{\hwNumber}{0}

%preamble

\title{Elliptic PDEs}
\author{A.N.}
\date{\today}
\lhead{\Course A\hwNumber}
\rhead{\thepage}
%\cfoot{\thepage}


%====================================================================%
\begin{document}

\maketitle
\begin{tcolorbox}[colback = white]
\begin{defn}
	If $u$ is $C^2$ in $\Omega\subset \R^n$, we define the \textbf{Laplacian} of $u$ to be:
	$$ \Delta u = \sum_{i=1}^n \frac{ \partial^2 }{ \partial x_i^2 } u = \grad \cdot (\grad u) $$ 
\end{defn}
\end{tcolorbox}
The Laplacian is the prototypical linear elliptic operator of $2$nd order. 
\begin{tcolorbox}[colback = white]
\begin{defn}
	We say $u$ is \textbf{Harmonic} if $\Delta u = 0$. 
\end{defn}
\end{tcolorbox}
\begin{tcolorbox}[colback = white]
	\textbf{Properties}
	\begin{enumerate}[label = \roman*)]
		\item $\Delta$ is linear. If $c_1,c_2 \in \R$, then 
			$$ \Delta \left( c_1 u + c_2 v \right) = c_1 \Delta u + c_2 \Delta v. $$ 
		\item $\Delta$ is self adjoint. If $u,v$ completely supported in $\Omega$ i.e. $0$ near the boundary, 
			$$ \int_\Omega \left( \Delta u \right) v = - \int_\Omega \grad u \cdot \grad v = \int_\Omega u \left( \Delta v \right). $$ 
	\item $\Delta$ is translationally invariant. If $\Delta u  =f$, then
	$$ \Delta \left( u(x-a) \right) = f(x-a) = \left( \Delta u \right) (x-a).$$
\item $\Delta$ is rotationally invariant. If $A \in O(n)$ then $u_A = u(Ax)$ satisfies
	$$ \Delta u_A = f(Ax) = \left( \Delta u \right) (Ax). $$ 
	\item If $\lambda \neq 0$, then 
		$$ \Delta u(\lambda x) = \lambda^2 \left( \Delta u \right) \left( \lambda x \right). $$ 
\end{enumerate}
\end{tcolorbox}
We now give some examples of simple harmonic functions. 
\begin{tcolorbox}[colback = white]
\begin{example}
\end{example}
	\begin{enumerate}[label = \roman*)]
		\item 	Constant functions. $\Delta (1) = 0$
\item 	Linear functions. $\Delta( a\cdot x + b) = 0$. 
\item Radial Harmonic functions? $u = u(|x|)= u(r)$. The radial laplacian is given as
	$$ \Delta u = \frac{ \partial^2 u}{ \partial r^2 } + \frac{ n-1 }{ r } \frac{ \partial u }{ \partial r } = \frac{ \partial  }{ \partial r } \left( r^{n-1} \frac{ \partial u  }{ \partial r } \right). $$ 
Therefore if $\Delta u = 0$, then 
			$$ \frac{ \partial u  }{ \partial r  }= cr^{1-n} .$$
So 
			$$ u(r) = \begin{cases}
				r^{2-n} & n>2
				 \\ \log r & n=2
			\end{cases} $$ 
Notice that $u$ blows up at $0$, so $u$ not harmonic at $0$. In fact, $\Delta u = c \delta_{\{0\}}$ 
	\end{enumerate}
\end{tcolorbox}
\begin{tcolorbox}[colback = white]
\begin{defn}
A function $u$ on $\Omega$ is said to be: 
\begin{enumerate}[label = \roman*)]
	\item  \textbf{Subharmonic} if $\Delta u \geq 0$
	\item \textbf{Superharmonic} if $\Delta u \leq 0$. 
\end{enumerate}
\end{defn}
\end{tcolorbox}
\begin{tcolorbox}[colback = white]
\begin{example}
\end{example} $u(x) = \frac{ 1 }{ 2 }|x|^2$ is subharmonic. 
\end{tcolorbox}
Given $u$ harmonic and $v$ subharmonic on $\Omega$, with $u|_{\bd \Omega} = v|_{\bd \Omega}$, we have that $v \leq u$. So for given boundary data on $\Omega$, a harmonic function satisfying this data is the "biggest" subharmonic function this data. The following results make this more precise. 
\begin{tcolorbox}
\begin{lemma}
	\textbf{Mean Value Property}
\end{lemma}
If $u$ satisfies $\Delta u \geq 0$ in $\Omega$, then for any $x_0 \in \Omega$ and for any $r>0$ so that $B_r(x_0) \subset \Omega$, then:
\begin{enumerate}[label = \roman*)]
	\item $$u(x_0) \leq \frac{ 1 }{ n \omega_n } \int_{S^{n-1}}u(x_0 + r\omega) d\sigma(\omega) $$
	\item $$ u(x_0) \leq \frac{ 1 }{ \omega_n r^n } \int_{B_r(x_0)} u(y) dy. $$ 
\end{enumerate}
\begin{proof}
	We first translate so that $x_0 = 0$, and rescale so that $\tilde{u}(x) = u(rx)$, so that we work on $B_1(0)$. We define the quantity 
	$$ g(s) = \frac{ 1 }{ |B_1(0)| } \int_{\bd B_1} u(s\omega) d \sigma(\omega). $$
	Clearly we have that $g(s) \to u(0)$ as $s \to 0$. We take the the derivative in $s$ of $u(s\omega)$. Get
	$$ \frac{ \partial  }{ \partial s } u(s \omega) = \omega \cdot \grad u (s\omega) , $$ 
Where $\omega$ is the unit normal to $S^{n-1}$. 
So,
\begin{align*}
	\frac{ d }{ ds } \int_{\bd B_1} u(s \omega ) d \sigma(\omega) & = \int_{\bd B_1} \grad_\omega u(sw) 
	\\ & = \int_{B_1} \grad \cdot \grad u(sx) dx \tag{Divergence Theorem}
	\\ &= s^2 \int_{B_1} \left( \Delta u \right) (sx) dx \geq 0.
\end{align*}
So 
	$$ u(0) \leq \frac{ 1 }{ |\bd B_1| } \int_{B_1} u(\omega) d \sigma(\omega). $$ 
Now, 
	$$ \int_{B_1} u(x) dx = \int_0^1 \int_{\bd B_1} u(s\omega) s^{n-1} ds d\sigma \geq \left| \int_0^1 u(0) s^{n-1} ds \right| \cdot |\bd B_1| = \frac{ 1 }{ n  } |\bd B_1| u(0). $$ 
As desired. 
\end{proof}
\end{tcolorbox}
\begin{tcolorbox}
\begin{cor}
\end{cor}
If $\Delta u =0$, then 
\begin{enumerate}[label = \roman*)]
	\item $$ u(x_0) = \frac{ 1 }{ n \omega_n } \int_{S^{n-1}} u(x_0 + r \omega) d \sigma(\omega). $$
	\item  $$ u(x_0) = \frac{ 1 }{ \omega r^n } \int-{B_r(x_0)} u(y) dy. $$ 
\end{enumerate}
\begin{proof}
If $\Delta u = 0$, then $\pm u$ is subharmonic.
\end{proof}
\end{tcolorbox}
\begin{tcolorbox}
\begin{cor}
	\textbf{Strong Maximum Principle}
\end{cor}
	If $\Delta u \geq 0$ on $\Omega$, where $\Omega$ is open and connected, and $\ol{\Omega}$ is compact,  then 
	$$ u(x) \leq \sup_{x\in \bd \Omega} u, $$ 
	with equality holding if and only if $u$ is constant. 
	\begin{proof}
		Take $x_0 \in \Omega$ so that $u(x_0) = \sup_\Omega u$. Choose $r>0$ sufficiently small so that $B_r(x_0) \subset \Omega$. Then
		$$ u(x_0) \leq \frac{ 1 }{ \omega_n r^n }\int_{B_r(x_0)} u \leq \sup u. $$ 
Since $u(x_0) = \sup u$ get that $u = \sup u$ on $B_r(x_0)$. Define
		$$ \Sigma = \left\{ x\in \Omega : u(x) = \sup_\Omega u \right\} .$$ 
		$\Sigma $ is open by above, and is closed by continuity of $u$. Therefore $\Sigma = \Omega$ by connectedness. 
	\end{proof}
\end{tcolorbox}
\begin{tcolorbox}
\begin{cor}
	\textbf{Comparison Principle}
\end{cor}
	Suppose $\Omega$ open, $\ol{\Omega}$ compact. Then if $\Delta u = 0, \Delta v \geq 0$, and 
$u|_{\bd \Omega} = v|_{\bd \Omega}$, then $u(x) \geq v(x)$ for all $x$ and equality holds if and only if $u = v$. 
\begin{proof}
	$\Delta (v-u) \geq 0$ and $v-u|_{\bd \Omega} = 0$, so $v-u \leq 0$ in $\Omega$ follows from the strong maximum principle. 
\end{proof}
\end{tcolorbox}
\begin{tcolorbox}
\begin{cor}
	\textbf{Uniquness of Solution to Dirichlet Problem}
\end{cor}
	If $\Omega \subset \R^n$ open, $\ol{\Omega}$ compact, $g: \bd \Omega \to \R$, then there is at most one unique function satisfying:
	$$ \begin{cases}
		\Delta u & = 0
		\\ u|_{\bd \Omega} & = g
	\end{cases} $$ 
\begin{proof}
Let $u_1, u_2$ both satisfy the Dirichlet Problem. Apply the strong maximum principle to $u_1-u_2$, 
$u_2-u_1$. 
\end{proof}
\end{tcolorbox}
\begin{tcolorbox}
\begin{lemma}
	\textbf{Higher Order Regularity}
\end{lemma}
	If $\Delta u =0$ in $\Omega$, then $u \in C^\infty (\Omega)$. 
\begin{proof}
	Let $\varphi(x) = \varphi(|x|)$ be a smooth function so that $\varphi \geq 0$, $\varphi \equiv 0$ for $|x|\geq 1$, and 
	$$ n \omega_n \int_0^1 r^{n-1} \varphi(r) dr = 1 = \int_{B_1(0)} \varphi(x) dx. $$ 
	For $y\in \R^n$ consider $ \frac{ 1 }{ \ep^n } \varphi \left( \frac{ x-y }{ \ep } \right)$ supported in $B_\ep (y)$. Note that 
	$$ \frac{ 1 }{ \ep^n  }\in_{B_\ep (y)} \varphi \left( \frac{ x-y }{ \ep } \right)dx = 1. $$ 
	Let $\Omega_\ep = \left\{ y\in \Omega : d(, \bd \Omega) > \ep \right\}$. 
	Define 
	$$ v(y) = \frac{ 1 }{ \ep^n  }\int u(x) \varphi \left( \frac{ x-y }{ \ep }  \right)dx. $$ 
(This is called the "Mollification" of $u$). Note that this is $C^\infty$ since $\varphi$ is. 
	We claim that $v(y) = u(y)$. By translation, assume that $y=0$. 
	Then, 
	\begin{align*}
		v(0) & = \frac{ 1 }{ \ep^n  }\int u(x) \varphi (|x|/\ep) dx
		\\ & = \frac{ 1 }{ \ep^n } \int_0^\ep \int_{S^{n-1}} u(s \omega) \varphi(s/\ep) ds d\sigma(\omega)
		\\ & = n|B_1(0)| u(0) \frac{ 1 }{ \ep^n  }\int_0^\ep \phi (s/\ep)s^{n-1} ds \tag{Mean Value Property}
		\\ & = u(0) \tag{by definition of $\varphi$}
	\end{align*}
\end{proof}
\end{tcolorbox}
In general we will see if $\Delta u = f$, then $u$ has two more derivatives than $f$. 
\begin{tcolorbox}
\begin{lemma}
	\textbf{Gradient Estimate}
\end{lemma}
	Suppose $u$ harmonic on $B_r(0) \subset \R^n$. Then 
	$$ |\grad u|(0) \leq \frac{ n }{ r } \sup_{\bd B_r} |u|. $$ 
	\begin{proof}
	Since $u$ is $C^\infty$ we can differentiate the equation $\Delta u = 0$ to get $\Delta u_i = 0$. By the mean value property, we have
		$$ \frac{ \partial u }{ \partial x_i }(0) = \frac{ 1 }{ \omega_n r^n }\int_{B_r} \frac{ \partial u }{ \partial x_i } dy = \frac{ 1 }{ \omega_n r^n } \int_{B_r} \grad \cdot (u e_i)dy. $$
Where $e_i$ is the $i$th standard basis vector. 
By the divergence theorem, 
		$$ \int_{B_r} \grad \cdot \left( u e_i \right) dy = \int_{\bd B_r} ue_i \cdot \vec{\nu} d\sigma $$ 
with $\vec{\nu} = \frac{ x }{ |x| }$. 
Therefore
\begin{align*}
	\left| \frac{ \partial u }{ \partial x_i  }(0) \right| & \leq \frac{ 1 }{ \omega_n r^n } \int_{\bd B_r} |u| d \sigma 
	\\ & \leq \frac{ 1 }{ \omega_n r^n } n \omega_n r^{n-1} \sup_{\bd B_r} |u|
	\\ & = \frac{ n }{ r }\sup_{\bd B_r} |u|
\end{align*}
		By rotation, we can assume that $\grad u(0) = |\grad u(0)| e_1$, so the result follows. 
	\end{proof}
\end{tcolorbox}
\begin{tcolorbox}
\begin{cor}
	\textbf{Liouville Theorem}
\end{cor}
If $u$ harmonic on all of $\R^n$ and $u$ bounded then $u$ is constant. 
\begin{proof}
For any $x\in \R^n$, $r>0$ we have
	$$ |\grad u(x)| \leq \frac{ n }{ r }\sup_{\bd B_r(x)} |u| \leq \frac{ n }{ r } \sup_{\R^n} |u| \leq \frac{ nC }{ r } $$
for some $C$. Taking $r\to \infty$ tells us that $|\grad u(x)| = 0$. Since $x$ arbitrary, $u$ is constant. 
\end{proof}
\end{tcolorbox}
\begin{tcolorbox}
\begin{lemma}
	\textbf{Real Analycity}
\end{lemma}
Suppose that $\Delta u = 0$ in $\Omega \subset \R^n$. Then $u$ analytic in $\Omega$. 
\begin{proof}
	By translating and rescaling, we can assume that $\Delta u = 0$ in $B_1(0)$. It suffices to show that $u$ has a convergent power series at $0$.
We make the following claim: 
	For all $m \in \N$, for any multi-index $\alpha = (\alpha_1, \dots, \alpha_n)$ with size $m$, z
\end{proof}
\end{tcolorbox}
\end{document}
