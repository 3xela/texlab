\documentclass[12pt, a4paper]{article}
\usepackage[lmargin =0.5 in, 
rmargin=0.5in, 
tmargin=1in,
bmargin=0.5in]{geometry}
\geometry{letterpaper}
\usepackage{tikz-cd}
\usepackage{amsmath}
\usepackage{amssymb}
\usepackage{blindtext}
\usepackage{titlesec}
\usepackage{enumitem}
\usepackage{fancyhdr}
\usepackage{amsthm}
\usepackage{graphicx}
\usepackage{cool}
\usepackage{thmtools}
\usepackage{hyperref}
\graphicspath{ }					%path to an image

%-------- sexy font ------------%
%\usepackage{libertine}
%\usepackage{libertinust1math}

%\usepackage{mlmodern}				% very nice and classic
%\usepackage[utopia]{mathdesign}
%\usepackage[T1]{fontenc}


\usepackage{mlmodern}
\usepackage{eulervm}
%\usepackage{tgtermes} 				%times new roman
%-------- sexy font ------------%


% Problem Styles
%====================================================================%


\newtheorem{problem}{Problem}


\theoremstyle{definition}
\newtheorem{thm}{Theorem}
\newtheorem{lemma}{Lemma}
\newtheorem{prop}{Proposition}
\newtheorem{cor}{Corollary}
\newtheorem{fact}{Fact}
\newtheorem{defn}{Definition}
\newtheorem{example}{Example}
\newtheorem{question}{Question}

\newtheorem{manualprobleminner}{Problem}

\newenvironment{manualproblem}[1]{%
	\renewcommand\themanualprobleminner{#1}%
	\manualprobleminner
}{\endmanualprobleminner}

\newcommand{\penum}{ \begin{enumerate}[label=\bf(\alph*), leftmargin=0pt]}
	\newcommand{\epenum}{ \end{enumerate} }

% Math fonts shortcuts
%====================================================================%

\newcommand{\ring}{\mathcal{R}}
\newcommand{\N}{\mathbb{N}}                           % Natural numbers
\newcommand{\Z}{\mathbb{Z}}                           % Integers
\newcommand{\R}{\mathbb{R}}                           % Real numbers
\newcommand{\C}{\mathbb{C}}                           % Complex numbers
\newcommand{\F}{\mathbb{F}}                           % Arbitrary field
\newcommand{\Q}{\mathbb{Q}}                           % Arbitrary field
\newcommand{\PP}{\mathcal{P}}                         % Partition
\newcommand{\M}{\mathcal{M}}                         % Mathcal M
\newcommand{\eL}{\mathcal{L}}                         % Mathcal L
\newcommand{\T}{\mathbb{T}}                         % Mathcal T
\newcommand{\U}{\mathcal{U}}                         % Mathcal U\\
\newcommand{\V}{\mathcal{V}}                         % Mathcal V

% symbol shortcuts
%====================================================================%

\newcommand{\bd}{\partial}
\newcommand{\grad}{\nabla}
\newcommand{\lam}{\lambda}
\newcommand{\imp}{\implies}
\newcommand{\all}{\forall}
\newcommand{\exs}{\exists}
\newcommand{\delt}{\delta}
\newcommand{\ep}{\varepsilon}
\newcommand{\ra}{\rightarrow}
\newcommand{\vph}{\varphi}

\newcommand{\ol}{\overline}
\newcommand{\f}{\frac}
\newcommand{\lf}{\lfrac}
\newcommand{\df}{\dfrac}

% bracketting shortcuts
%====================================================================%
\newcommand{\abs}[1]{\left| #1 \right|}
\newcommand{\babs}[1]{\Big|#1\Big|}
\newcommand{\bound}{\Big|}
\newcommand{\BB}[1]{\left(#1\right)}
\newcommand{\dd}{\mathrm{d}}
\newcommand{\artanh}{\mathrm{artanh}}
\newcommand{\Med}{\mathrm{Med}}
\newcommand{\Cov}{\mathrm{Cov}}
\newcommand{\Corr}{\mathrm{Corr}}
\newcommand{\tr}{\mathrm{tr}}
\newcommand{\Range}[1]{\mathrm{range}(#1)}
\newcommand{\Null}[1]{\mathrm{null}(#1)}
\newcommand{\lan}{\langle}
\newcommand{\ran}{\rangle}
\newcommand{\norm}[1]{\left\lVert#1\right\rVert}
\newcommand{\inn}[1]{\lan#1\ran}
\newcommand{\op}[1]{\operatorname{#1}}
\newcommand{\bmat}[1]{\begin{bmatrix}#1\end{bmatrix}}
\newcommand{\pmat}[1]{\begin{pmatrix}#1\end{pmatrix}}
\newcommand{\vmat}[1]{\begin{vmatrix}#1\end{vmatrix}}

\newcommand{\amogus}{{\bigcap}\kern-0.8em\raisebox{0.3ex}{$\subset$}}
\newcommand{\Note}{\textbf{Note: }}
\newcommand{\Aside}{{\bf Aside: }}
%restriction
%\newcommand{\op}[1]{\operatorname{#1}}
%\newcommand{\done}{$$\mathcal{QED}$$}

%====================================================================%


\setlength{\parindent}{0pt}      	% No paragraph indentations
\pagestyle{fancy}
\fancyhf{}							% fancy header

\setcounter{secnumdepth}{0}			% sections are numbered but numbers do not appear
\setcounter{tocdepth}{2} 			% no subsubsections in toc

%template
%====================================================================%
%\begin{manualproblem}{1}
%Spivak.
%\end{manualproblem}

%\begin{proof}[Solution]
%\end{proof}

%----------- or -----------%

%\begin{problem} 		
%\end{problem}	

%\penum
%	\item
%\epenum
%====================================================================%


\newcommand{\Course}{MAT482}
\newcommand{\hwNumber}{3}

%preamble

\title{MAT482 HW3}
\author{A.N.}
\date{\today}
\lhead{\Course A\hwNumber}
\rhead{\thepage}
%\cfoot{\thepage}


%====================================================================%
\begin{document}

\maketitle

\begin{problem}
\end{problem}
\penum
\item First we have that: 
\begin{align*}
	\frac{|f(x) g(x) - f(y)g(y)|}{|x-y|^\alpha}& = \frac{|f(x)g(x) + f(x)g(y)-f(x)g(y) - f(y)g(y)|}{|x-y|^\alpha}
	\\ & = \frac{|f(x) \left[  g(x) - g(y) \right]]  + g(y) \left[ f(x) - f(y)\right]|}{|x-y|^\alpha}
	\\ & \leq |f(x)| \frac{|g(x) - g(y)|}{|x-y|^\alpha} + |g(y)|\frac{|f(x) - f(y)|}{|x-y|^\alpha} \tag{By Triangle Inequality}
	\\ & \leq M_f \frac{|g(x) - g(y)|}{|x-y|^\alpha} + M_g \frac{|f(x)-f(y)|}{|x-y|^\alpha}\tag{By Continuity of $f,g$}
	\\ & 
\end{align*}
The above quantity is finite for all $x\neq y$ since $f,g\in C^\alpha$. We conclude that $fg\in C^\alpha$.
\item 
	Since $\{f_n\}$ is bounded in $C^\alpha$ it must be bounded in $C^0$ as well. Since $\norm{f_n}_{C^\alpha} \leq 1$ we have that for all $n$, $x\neq y\in \Omega$, 
	$$|f_n(x) - f_n(y)| \leq |x-y|^\alpha\implies |f_n(x) - f_n(y)| \leq M |x-y|,$$
Since H\"older Continuity implies  uniform continuity. Therefore $\{f_n\}$ is an equicontinuous family in $C^0$. By Arzela-Ascoli's Theorem there exists a subsequence $\{f_{n_k}\}$ so that $f_{n_k} \rightrightarrows f$ to some $f\in C^0$. 
We claim that $f\in C^\beta$. Since $C^\beta$ is a Banach Space, it is enough to show that $\{f_{n_k}\}$ is cauchy. 
First relabel the sequence $\{f_{n_k}\}$ as $\{f_n\}$. By definition, we have:
$$\norm{f_n - f_m}_{C^\beta} = \sup_{\Omega} |f_m - f_n| + \sup_{x\neq y} \frac{|(f_n-f_m)(x) - (f_n-f_m)(y)|}{|x-y|^\beta}.$$
The first term can clearly be made less than $\ep$ just by uniform convergence. It remains to show the same is true for the second term. We compute:
\begin{align*}
	\sup_{x\neq y} \frac{|(f_n-f_m)(x) - (f_n-f_m)(y)|}{|x-y|^\beta} & = \left(\sup_{x\neq y} \frac{|(f_n-f_m)(x) - (f_n-f_m)(y)|}{|x-y|^\alpha} \right)^{\frac{\beta}{\alpha}} |(f_n-f_m)(x)-(f_n-f_m)(y)|^{1 - \frac{\beta}{\alpha}}
	\\ & \leq 2\left[f_n - f_m \right]^\frac{\beta}{\alpha}_{C^\alpha} \cdot \sup_{\Omega} |f_n-f_m|^{1 - \frac{\beta}{\alpha}}
	\\ & \leq 2^{1+ \frac{\beta}{\alpha}} \sup_\Omega |f_n - f_m|^{1 - \frac{\beta}{\alpha}} .\tag{By Triangle Inequality}
\end{align*}
We can also make the above quantity small than $\ep$ for sufficiently large $n,m$. The sequence is cauchy and thus converges in $C^\beta$. 
\epenum
\newpage
\begin{problem}
\end{problem}
\penum 
\item  
\begin{enumerate}[label = \roman*)]
	\item Since $\tilde{u}$ is a rescaled and translated $u$, we must have that $\tilde{u}$ is defined on $B_{1}(0)$ if and only if $u$ is defined on $B_{d_p}(p)$. Our selection of $d_p$ makes it so that $B_{d_p}(p) \subset \Omega$. So our function $\tilde{u}$ makes sense. 
\item Since $\tilde{u}|_{B_{1/2}(0)} = u|_{B_{d_p/2}(p)}$, we compute that 
\begin{align*}
	\norm{\tilde{u}}_{C^2(B_{1/2}(0))}& = \sum_{|\alpha| \leq 2} \sup_{B_{1/2}(0)} |D^\alpha \tilde{u}|
	\\ & =\sum_{|\alpha|\leq 2} \frac{dp^{\alpha}}{M} \sup_{B_{d_p/2}(p)} |D^\alpha u(x)|
	\\ & = \sum_{|\alpha| = 0} \frac{1}{M} \sup_{B_{d_p/2}(p)} |u(x)| + \sum_{|\alpha| = 1} \frac{d_p}{M} \sup_{B_{d_p/2}(p)} |D^\alpha u(x)| + \sum_{|\alpha |=2} \frac{d_p^2}{M} \sup_{B_{d_p/2}(p)} |D^\alpha u(x)|
	\\ & \leq \frac{1}{M} + \sum_{|\alpha| = 1} \frac{d_p}{M} \sup_{B_{d_p/2}(p)} \left| \int_{B_{dp/2}(p)} D^{\alpha+1} u(x) dx\right|  + C(n) \tag{since $|u|\leq 1$, $\sup d_p^2|D^2u(p)| =M $}
	\\ & = \frac{1}{M} + C_1(n) + C_2(n) \tag{since $d^2_p|D^\alpha+2| \leq M$}
\end{align*}
In a similar fashion we compute :
		$$|D^2 \tilde{u}(x)| = |D^2 M^{-1} u(p+d_p x))| = M^{-1} d_p^2 |D^2u(p+d_p x)|.$$
At $x = 0$ we have that 
		$$|D^2\tilde{u}(0)| = M^{-1}d_p^2 |D^2u(p)| = 1.$$
	\item Using the chain rule we compute $a^{ij}\tilde{u}_{ij}$. 
		$$a^{ij}\tilde{u}_{ij} = M^{-1} a^{ij} (u(p+d_px))_{ij} = M^{-1}d_p^2 a^{ij}\tilde{u}_{ij} = M^{-1}d^2_p f(p + d_p x).$$
\item Since $|u|\leq 1$, 
	$$|\tilde{u}| = |M^{-1}u(p + dp_m x)|\leq M^{-1}. $$
\end{enumerate}
\item Suppose that lemma $0.2$ does not hold, that is we have for some function $u$ that for all $C$,
	$$\sup_{\Omega}d_x^2 |D^2u(x)| > C \left( \norm{f}_{C^\alpha}+ \norm{u}_{C^0}\right).$$
We can find a sequence $\{u_k\}$ so that if $C=m$, $a^{ij}(u_m)_{ij} = f_m$, 
$$\sup_{\Omega} d_x^2 |D^2u_m(x)| > k \left(\norm{f_m}_{C^\alpha} + \norm{u}_{C^0} \right).$$
Let $p_m$, $M_m$ be such that
$$\sup_{\Omega} d_x^2 |D^2u_m| = d_{p_m}^2 |D^2u_m(p_m)| = M_m.$$
By hypothesis, we must have that $M_m \to \infty$ as $m \to \infty$. Define $\tilde{u}_m = M_m^{-1}u_m(p_m + xdp_m)$. All of the results from part $a)$ hold, in particular $\norm{u_m}_{C^2(B_{1/2}(0)}$ is uniformly bounded, and $|u_m| \leq 1$ on $B_1(0)$. We have that $a^{ij}(\tilde{u})_{ij} = f_m(p_m + dp_m x)$. So by Theorem 0.1, we have:
$$(\min\{d_x, d_y\} )^\alpha \frac{\left|D^2 \tilde{u}_m(x) - D^2\tilde{u}_mu(y) \right|}{|x-y|^\alpha}\leq C\left(\norm{f_m}_{C^\alpha(\Omega^\prime)} + \norm{\tilde{u}_m}_{C^2(\Omega^\prime)} \right).$$
By taking the $\sup$ over $x,y\in B_{1/2}(0)$, we get that 
$$\sup_{x\neq y}\frac{\left|D^2\tilde{u}_m(x) - D^2 \tilde{u}_m(y) \right|}{|x-y|^\alpha}\leq C_1.$$
Since the $\tilde{u}_m$ are uniformly bounded in $C^2$, and have holder constant $C_1$ they must be in $C^{2,\alpha}$. Therefore $\{\tilde{u}_m\}$ is bounded. By Arzela-Ascoli, we can find a convergent subsequence, $\{\tilde{u}_{m_k}\}$ with limit $\tilde{u}_\infty$. We have that 
$$\norm{\tilde{u}_\infty}_{C^0} = \lim_{m_k \to \infty}\norm{\tilde{u}_{m_k}}_{C^0} \leq \lim_{m_k \to \infty} M_{m_k}^{-1} = 0.$$
However,
$$|D^2\tilde{u}_\infty(0)| = \lim_{m_k \to \infty} |D^2\tilde{u}_{m_k}(0)| = 1.$$
This is a contradiction. 
\epenum
\newpage
\begin{problem}
\end{problem}
Since $\ol{u}$ agrees with a harmonic function on the upper and lower half planes, it is enough to show that it is harmonic along $x_n= 0$. Since $\ol{u}$ is continuous, we check that it has mean value property for $x_n = 0$. Take a point $x = (x_1, \dots , x_{n-1} , 0)$. We compute: 
\begin{align*}
	\frac{1}{\omega_n r^n} \int_{B_r(x)} \ol{u}(y) dy & = \frac{1}{\omega_n r^n} \left[\int_{B_r(x)\cap \{x_n >0\}} \ol{u}(y) dy + \int_{B_r\cap\{ x_n <0\}} \ol{u}(y) dy \right]
	\\ & = \frac{1}{\omega_n r^n} \left[ \int_{B_r\cap \{x>0\}} u(y)dy + \int_{B_r(x) \cap \{x_n<0\}} -u(y_1, \dots , -y_n)dy \right]
	\\ & = \frac{1}{\omega_n r^n} \left[ \int_{B_r\cap \{x>0\}} u(y)dy + \int_{B_r(x) \cap \{x_n >0\}} -u(y) dy  \right] \tag{Change of Variables}
	\\ & = 0 
	\\ & = u(x)
\end{align*}
So $u$ is continiuous, has mean value proprty, and agrees everywhere with a harmonic function. We conclude $u$ is harmonic. 
\newpage
\begin{problem}
\end{problem}
\penum
\item Since $\norm{f}_{C^\alpha} + \norm{u}_{C^2} + \norm{\phi}_{C^\alpha,2}$ we must have that $\norm{f}_{C^\alpha} + \norm{u}_{C^2} \leq 1$. By theorem 0.1 we have that 
	$$\left( \min \{d_{p_m} ,d_{q_m} \}\right)^\alpha \frac{|D^2u(p_m) - D^2u(q_m)| }{|p_m-q_m|^\alpha} \leq C \implies d_{p_m}^\alpha M \leq C \implies d_{p_m}^\alpha \leq CM^{-1}.$$
Since $\norm{u}_{C^2} \leq 1$, we must have that by triangle inequality:
$$|D^2u(p_m) - D^2u(q_m)|\leq 2.$$
Multiplying both sides by $\frac{r_m^\alpha}{r_m^\alpha}$, we get that:
$$r^\alpha_m \frac{|D^2u(p_m) - D^2u(q_m)|}{r_m^\alpha}\leq 2 \implies r_m^\alpha M \leq 2 \implies r_m^\alpha \leq 2M^{-1}.$$
Since $M \geq m$, as $m \to infty$ we have that $d_{p_m}^\alpha, r_m^\alpha \to 0$. 
\item 
\begin{enumerate}[label = \roman*) ]
	\item Note that $\tilde{v}_m$ is a sum of $\tilde{u}_m$ and a polynomial in $x$ along with some constant terms. Therefore it will be defined exactly where $\tilde{u}_m$ is defined. Since $u_m$ is defined on $B_1(e_n)$, we must have that $\tilde{v}_m$ is defined on the ball of radius $1/r_m$ centered about $e_n$. Thus we wish to show that if $x\in \Gamma_m$, then $(r_mx_n - 1)^2 + \sum_{i}^{n-1}(r_mx_i)^2 \leq 1$. 
We have that 
		$$-1 + \sum_{i}^{n-1} (r_mx_i)^2 \leq r_mx_n - 1 \leq 0.$$
Squaring and flipping the directions of the inequalities, we have that 
$$ \left(-1 + \sum_{i}^{n-1} (r_mx_i)^2  \right)^2 \geq (r_m x_n- 1)^2 \geq 0. $$
Adding $\sum_{i}^{n-1} (r_m x_i)^2$ to both sides we get that:
		$$1 - \sum_{i}^{n-1} (r_mx_i)^2  + \left(\sum_{i}^{n-1} (r_mx_i)^2 \right)^2 \geq (r_mx_n-1)^2 + \sum_{i}^{n-1}(r_mx_i)^2.$$
Since $\sum_{i}^{n-1} (r_mx_i)^2 \leq 1/2$, the lefthand side is less than 1. 
\item We have $v_m(x) =  \tilde{u}_m(x) - \tilde{u}_m(0) + - \sum_i x_i \frac{\tilde{u}_m(0)}{\partial x_i}  - \frac{1}{2} \sum_{i,j} \frac{\partial^2\tilde{u}_m}{\partial x_i \partial x_j}(0) x_ix_j.$
It is easy to see that 
$$v_m(0) = \tilde{u}_m(0) - \tilde{u}_m(0) = 0.$$
As well as
$$Dv_m(0) = D\tilde{u}_m(0) - D\tilde{u}_m(0) =0 ,$$
since the degree 2 polynomial just evaluates to $0$, and the linear term has a gradient of $D\tilde{u}_m(0)$.  
Finally we have that 
$$D^2 v_m(0) = D^2\tilde{u}_m(0) - D^2 \tilde{u}_m(0). $$
This is since the constant and linear terms vanish and the hessian of the quadratic term is excactly $D^2\tilde{u}_m(0)$. Furthermore for $x,y\in \Gamma_m$, we have:
\begin{align*}
 |D^2 v_m(x) - D^2v_m(y)| & = |D^2\tilde{u}_m(x) - D^2\tilde{u}_m(y)|
 \\ & = |D^2(M^{-1}r_m^{-2-\alpha}u(r_m x)) - D^2(M^{-1}r_m^{-2-\alpha}u(r_m y))|
 \\ & = M^{-1}r_m^{-\alpha}|D^2u_m(r_mx) - D^2u_m(r_my)|
 \\ & \leq |x-y|^\alpha \tag{by assumption at points $r_mx,r_my$}
\end{align*}
\end{enumerate}
\epenum
\newpage
\begin{problem}
\end{problem}
\newpage
\end{document}
