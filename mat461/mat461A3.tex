\documentclass[12pt, a4paper]{article}
\usepackage[lmargin =0.5 in, 
rmargin=0.5in, 
tmargin=1in,
bmargin=0.5in]{geometry}
\geometry{letterpaper}
\usepackage{tikz-cd}
\usepackage{amsmath}
\usepackage{amssymb}
\usepackage{blindtext}
\usepackage{titlesec}
\usepackage{enumitem}
\usepackage{fancyhdr}
\usepackage{amsthm}
\usepackage{graphicx}
\usepackage{cool}
\usepackage{thmtools}
\usepackage{hyperref}
\graphicspath{ }					%path to an image

%-------- sexy font ------------%
%\usepackage{libertine}
%\usepackage{libertinust1math}

%\usepackage{mlmodern}				% very nice and classic
%\usepackage[utopia]{mathdesign}
%\usepackage[T1]{fontenc}


\usepackage{mlmodern}
\usepackage{eulervm}
%\usepackage{tgtermes} 				%times new roman
%-------- sexy font ------------%


% Problem Styles
%====================================================================%


\newtheorem{problem}{Problem}


\theoremstyle{definition}
\newtheorem{thm}{Theorem}
\newtheorem{lemma}{Lemma}
\newtheorem{prop}{Proposition}
\newtheorem{cor}{Corollary}
\newtheorem{fact}{Fact}
\newtheorem{defn}{Definition}
\newtheorem{example}{Example}
\newtheorem{question}{Question}

\newtheorem{manualprobleminner}{Problem}

\newenvironment{manualproblem}[1]{%
	\renewcommand\themanualprobleminner{#1}%
	\manualprobleminner
}{\endmanualprobleminner}

\newcommand{\penum}{ \begin{enumerate}[label=\bf(\alph*), leftmargin=0pt]}
	\newcommand{\epenum}{ \end{enumerate} }

% Math fonts shortcuts
%====================================================================%

\newcommand{\ring}{\mathcal{R}}
\newcommand{\N}{\mathbb{N}}                           % Natural numbers
\newcommand{\Z}{\mathbb{Z}}                           % Integers
\newcommand{\R}{\mathbb{R}}                           % Real numbers
\newcommand{\C}{\mathbb{C}}                           % Complex numbers
\newcommand{\F}{\mathbb{F}}                           % Arbitrary field
\newcommand{\Q}{\mathbb{Q}}                           % Arbitrary field
\newcommand{\PP}{\mathcal{P}}                         % Partition
\newcommand{\M}{\mathcal{M}}                         % Mathcal M
\newcommand{\eL}{\mathcal{L}}                         % Mathcal L
\newcommand{\T}{\mathbb{T}}                         % Mathcal T
\newcommand{\U}{\mathcal{U}}                         % Mathcal U\\
\newcommand{\V}{\mathcal{V}}                         % Mathcal V

% symbol shortcuts
%====================================================================%

\newcommand{\bd}{\partial}
\newcommand{\grad}{\nabla}
\newcommand{\lam}{\lambda}
\newcommand{\imp}{\implies}
\newcommand{\all}{\forall}
\newcommand{\exs}{\exists}
\newcommand{\delt}{\delta}
\newcommand{\ep}{\varepsilon}
\newcommand{\ra}{\rightarrow}
\newcommand{\vph}{\varphi}

\newcommand{\ol}{\overline}
\newcommand{\f}{\frac}
\newcommand{\lf}{\lfrac}
\newcommand{\df}{\dfrac}

% bracketting shortcuts
%====================================================================%
\newcommand{\abs}[1]{\left| #1 \right|}
\newcommand{\babs}[1]{\Big|#1\Big|}
\newcommand{\bound}{\Big|}
\newcommand{\BB}[1]{\left(#1\right)}
\newcommand{\dd}{\mathrm{d}}
\newcommand{\artanh}{\mathrm{artanh}}
\newcommand{\Med}{\mathrm{Med}}
\newcommand{\Cov}{\mathrm{Cov}}
\newcommand{\Corr}{\mathrm{Corr}}
\newcommand{\tr}{\mathrm{tr}}
\newcommand{\Range}[1]{\mathrm{range}(#1)}
\newcommand{\Null}[1]{\mathrm{null}(#1)}
\newcommand{\lan}{\left\langle}
\newcommand{\ran}{\right\rangle}
\newcommand{\norm}[1]{\left\lVert#1\right\rVert}
\newcommand{\inn}[1]{\lan#1\ran}
\newcommand{\op}[1]{\operatorname{#1}}
\newcommand{\bmat}[1]{\begin{bmatrix}#1\end{bmatrix}}
\newcommand{\pmat}[1]{\begin{pmatrix}#1\end{pmatrix}}
\newcommand{\vmat}[1]{\begin{vmatrix}#1\end{vmatrix}}

\newcommand{\amogus}{{\bigcap}\kern-0.8em\raisebox{0.3ex}{$\subset$}}
\newcommand{\Note}{\textbf{Note: }}
\newcommand{\Aside}{{\bf Aside: }}
%restriction
%\newcommand{\op}[1]{\operatorname{#1}}
%\newcommand{\done}{$$\mathcal{QED}$$}

%====================================================================%


\setlength{\parindent}{0pt}      	% No paragraph indentations
\pagestyle{fancy}
\fancyhf{}							% fancy header

\setcounter{secnumdepth}{0}			% sections are numbered but numbers do not appear
\setcounter{tocdepth}{2} 			% no subsubsections in toc

%template
%====================================================================%
%\begin{manualproblem}{1}
%Spivak.
%\end{manualproblem}

%\begin{proof}[Solution]
%\end{proof}

%----------- or -----------%

%\begin{problem} 		
%\end{problem}	

%\penum
%	\item
%\epenum
%====================================================================%


\newcommand{\Course}{MAT461}
\newcommand{\hwNumber}{3}

%preamble

\title{MAT461 HW 3}
\author{A.N.}
\date{\today}
\lhead{\Course A\hwNumber}
\rhead{\thepage}
%\cfoot{\thepage}


%====================================================================%
\begin{document}

\maketitle

\begin{problem}
% problem number 1
\end{problem}
\penum 
\item We first make the following change of coordiantes: 
	$$ \begin{cases}
		x &= R \sin \varphi \cos \theta
		\\y &= R \sin \varphi \sin \theta
		\\ z &= R \cos \varphi
	\end{cases}$$ 
The tangent space coordinates are therefore:
$$ \begin{cases}
	\dot{x} & = R \left( \dot{\varphi} \cos \varphi \cos \theta - \dot{\theta} \sin \varphi \sin \theta  \right)
	\\ \dot{y} & = R \left( \dot{\varphi} \cos \varphi \sin \theta + \dot{\theta} \sin \varphi \cos \theta \right)
	\\ \dot{z} & = -R \dot{\varphi} \sin \varphi
\end{cases}$$ 
Thus we can compute the lagrangian in spherical coordinates. 
\begin{align*}
	\eL ( \theta, \dot{\theta}, \varphi, \dot{\varphi}) & = \frac{ R^2}{ 2 } \left( \left( \dot{\varphi} \cos \varphi \cos \theta - \dot{\theta} \sin \varphi \sin \theta \right)^2 + \left( \dot{\varphi} \cos \varphi \sin \theta+ \dot{\theta} \sin \varphi \cos \theta \right)^2 + \dot{\varphi}^2 \sin^2 \varphi  \right) + gR \cos \varphi
	\\ & = \frac{ R^2 }{ 2 } \left( \dot{\varphi}^2 + \dot{\theta}^2 \sin^2 \varphi \right) + gR \cos \varphi.
\end{align*}
\item We now show that $\partial_\theta$ is a symmetry of the lagrangian. Note that this vector field has a flow of $$h_s(\theta , \varphi) = (\theta + s, \varphi).$$
It follows that since $\eL$ is constant in $\theta$,
$$ \eL(h_s(\theta, \varphi), \dot{\theta}, \dot{\varphi}) = \eL (\theta, \varphi, \dot{\theta}, \dot{\varphi}). $$ 
By N\"oethers Theorem, the conserved current is:
$$ J = \frac{ \partial \eL  }{ \partial \dot{q}  } \frac{ d h_s }{ ds }\Big|_{s=0}  = R^2 \dot{\varphi  } d \varphi + R^2 \dot{\theta} \sin^2 \varphi d \theta (\partial_\theta, 0) = R^2 \dot{\theta} \sin^2 \varphi.$$ 
It remains to show that $J$ is the z-component of angular momentum. Since angular momentum is given by $r \times \dot{r}$, and we have already computed these quantities in $a)$, we see that the $z$ component is given by:
\begin{align*} (r \times \dot{r})_z  &= x \dot{y} - y \dot{x} 
	\\ &=  R^2 \left( \sin \theta \sin \varphi \right) \left( \dot{\varphi} \cos \varphi \sin \theta + \dot{\theta} \sin \varphi \cos \theta \right)- R^2\left( \sin \varphi \sin \theta \right) \left( \dot{\varphi} \cos \varphi \cos \theta - \dot{\theta} \sin \varphi \sin \theta \right)
	\\ & = R^2 \dot{\theta} \sin^2 \varphi.
\end{align*}
This is exactly what we expect. 
\item The momentum $J$ is given by $J  = R^2 \dot{\theta} \sin^2 \varphi$. Our Lagrangian takes the form of $\eL = T - U$ , so the energy is given as 
	$$ E = T+U = \frac{ R^2 }{ 2 } \left( \dot{\varphi}^2 + \dot{\theta}^2 \sin^2 \varphi \right)- gR \cos \varphi. $$ 
	Therefore the energy momentum map is given as 
	$$ \mu = (J,E) = \left( R^2 \dot{\theta} \sin^2 \varphi, \frac{ R^2 }{ 2 } \left( \dot{\varphi}^2 + \dot{\theta}^2 \sin^2 \varphi \right) - gR \cos \varphi \right). $$ 
We now determine the critical points by checking where $dJ \wedge dE $ vanishse. We compute: 
$$ dJ = \left(R^2 \sin^2 \varphi \right) d \dot{\theta} +  \left(2 R \dot{\theta } \sin \varphi \cos \varphi \right)d \varphi.$$ 
$$ dE = \left( R^2 \dot{\theta}^2 \sin \varphi \cos \varphi + gR \sin \varphi \right)d\varphi + \left( R^2 \dot{\theta} \sin^2 \varphi \right)d\dot{\theta} +  R^2 \dot{\varphi}^2 d\dot{\varphi} $$
We now find $dJ \wedge dE$:
$$ dJ \wedge dE = \left( -R^4 \dot{\theta}^2 \sin^3 \varphi \cos \varphi + gR^3 \sin^3 \varphi \right) d \dot{\theta} \wedge d \varphi  + \left(   R^4 \dot{\varphi}^2 \sin^2 \varphi\right)d \dot{\theta} \wedge d \dot{\varphi} + \left( 2R^3 \dot{\varphi}^2 \sin\varphi \cos \varphi \right)d\varphi \wedge d \dot{\varphi}.$$ 
This will vanish exactly when 
\begin{align*}
	-R^4 \dot{\theta}^2 \sin^3 \varphi \cos \varphi + gR^3 \sin^3 \varphi & = 0
	\\ R^4 \dot{\varphi}^2 \sin^2 \varphi & = 0
	\\ 2R^3 \dot{\varphi}^2 \sin \varphi \cos \varphi & =0.
\end{align*}
This implies that $\varphi = n\pi $ for $n\in \Z$. 
We now determine the image of the critical points. Evaluating $\mu$, we get that 
$$ \mu = \left( 0 , \frac{ R^2 }{ 2 }\dot{\varphi}^2 \pm gR \right) $$ 
\epenum
\newpage
\begin{problem}
% problem number 2
\end{problem}
\newpage


\end{document}
