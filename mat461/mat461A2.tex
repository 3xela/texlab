\documentclass[12pt, a4paper]{article}
\usepackage[lmargin =0.5 in, 
rmargin=0.5in, 
tmargin=1in,
bmargin=0.5in]{geometry}
\geometry{letterpaper}
\usepackage{tikz-cd}
\usepackage{amsmath}
\usepackage{amssymb}
\usepackage{blindtext}
\usepackage{titlesec}
\usepackage{enumitem}
\usepackage{fancyhdr}
\usepackage{amsthm}
\usepackage{graphicx}
\usepackage{cool}
\usepackage{thmtools}
\usepackage{hyperref}
\graphicspath{ }					%path to an image

%-------- sexy font ------------%
%\usepackage{libertine}
%\usepackage{libertinust1math}

%\usepackage{mlmodern}				% very nice and classic
%\usepackage[utopia]{mathdesign}
%\usepackage[T1]{fontenc}


\usepackage{mlmodern}
\usepackage{eulervm}
%\usepackage{tgtermes} 				%times new roman
%-------- sexy font ------------%


% Problem Styles
%====================================================================%


\newtheorem{problem}{Problem}


\theoremstyle{definition}
\newtheorem{thm}{Theorem}
\newtheorem{lemma}{Lemma}
\newtheorem{prop}{Proposition}
\newtheorem{cor}{Corollary}
\newtheorem{fact}{Fact}
\newtheorem{defn}{Definition}
\newtheorem{example}{Example}
\newtheorem{question}{Question}

\newtheorem{manualprobleminner}{Problem}

\newenvironment{manualproblem}[1]{%
	\renewcommand\themanualprobleminner{#1}%
	\manualprobleminner
}{\endmanualprobleminner}

\newcommand{\penum}{ \begin{enumerate}[label=\bf(\alph*), leftmargin=0pt]}
	\newcommand{\epenum}{ \end{enumerate} }

% Math fonts shortcuts
%====================================================================%

\newcommand{\ring}{\mathcal{R}}
\newcommand{\N}{\mathbb{N}}                           % Natural numbers
\newcommand{\Z}{\mathbb{Z}}                           % Integers
\newcommand{\R}{\mathbb{R}}                           % Real numbers
\newcommand{\C}{\mathbb{C}}                           % Complex numbers
\newcommand{\F}{\mathbb{F}}                           % Arbitrary field
\newcommand{\Q}{\mathbb{Q}}                           % Arbitrary field
\newcommand{\PP}{\mathcal{P}}                         % Partition
\newcommand{\M}{\mathcal{M}}                         % Mathcal M
\newcommand{\eL}{\mathcal{L}}                         % Mathcal L
\newcommand{\T}{\mathbb{T}}                         % Mathcal T
\newcommand{\U}{\mathcal{U}}                         % Mathcal U\\
\newcommand{\V}{\mathcal{V}}                         % Mathcal V

% symbol shortcuts
%====================================================================%

\newcommand{\bd}{\partial}
\newcommand{\grad}{\nabla}
\newcommand{\lam}{\lambda}
\newcommand{\imp}{\implies}
\newcommand{\all}{\forall}
\newcommand{\exs}{\exists}
\newcommand{\delt}{\delta}
\newcommand{\ep}{\varepsilon}
\newcommand{\ra}{\rightarrow}
\newcommand{\vph}{\varphi}

\newcommand{\ol}{\overline}
\newcommand{\f}{\frac}
\newcommand{\lf}{\lfrac}
\newcommand{\df}{\dfrac}

% bracketting shortcuts
%====================================================================%
\newcommand{\abs}[1]{\left| #1 \right|}
\newcommand{\babs}[1]{\Big|#1\Big|}
\newcommand{\bound}{\Big|}
\newcommand{\BB}[1]{\left(#1\right)}
\newcommand{\dd}{\mathrm{d}}
\newcommand{\artanh}{\mathrm{artanh}}
\newcommand{\Med}{\mathrm{Med}}
\newcommand{\Cov}{\mathrm{Cov}}
\newcommand{\Corr}{\mathrm{Corr}}
\newcommand{\tr}{\mathrm{tr}}
\newcommand{\Range}[1]{\mathrm{range}(#1)}
\newcommand{\Null}[1]{\mathrm{null}(#1)}
\newcommand{\lan}{\left\langle}
\newcommand{\ran}{\right\rangle}
\newcommand{\norm}[1]{\left\lVert#1\right\rVert}
\newcommand{\inn}[1]{\lan#1\ran}
\newcommand{\op}[1]{\operatorname{#1}}
\newcommand{\bmat}[1]{\begin{bmatrix}#1\end{bmatrix}}
\newcommand{\pmat}[1]{\begin{pmatrix}#1\end{pmatrix}}
\newcommand{\vmat}[1]{\begin{vmatrix}#1\end{vmatrix}}

\newcommand{\amogus}{{\bigcap}\kern-0.8em\raisebox{0.3ex}{$\subset$}}
\newcommand{\Note}{\textbf{Note: }}
\newcommand{\Aside}{{\bf Aside: }}
%restriction
%\newcommand{\op}[1]{\operatorname{#1}}
%\newcommand{\done}{$$\mathcal{QED}$$}

%====================================================================%


\setlength{\parindent}{0pt}      	% No paragraph indentations
\pagestyle{fancy}
\fancyhf{}							% fancy header

\setcounter{secnumdepth}{0}			% sections are numbered but numbers do not appear
\setcounter{tocdepth}{2} 			% no subsubsections in toc

%template
%====================================================================%
%\begin{manualproblem}{1}
%Spivak.
%\end{manualproblem}

%\begin{proof}[Solution]
%\end{proof}

%----------- or -----------%

%\begin{problem} 		
%\end{problem}	

%\penum
%	\item
%\epenum
%====================================================================%


\newcommand{\Course}{MAT461}
\newcommand{\hwNumber}{2}

%preamble

\title{MAT461 A2}
\author{A.N.}
\date{\today}
\lhead{\Course A\hwNumber}
\rhead{\thepage}
%\cfoot{\thepage}


%====================================================================%
\begin{document}
\maketitle{}


\begin{problem}
\end{problem}
\penum
\item We compute the fiber derivatives of each of the given maps. 
	\begin{enumerate}[label = \roman*)]
		\item $\eL  (x , \dot{x}) = \pi^\ast V(x,\dot{x})$. Since $\pi^\ast V$ is constant in $\dot{x}$ we compute: 
			$$F \eL(x,\dot{x}) =\left(x, \frac{\partial \pi^\ast V}{\partial \dot{x}}\right) = (x,0).$$
		\item $\eL (x,\dot{x}) = A_i\dot{x}^i$. This is linear in $\dot{x}$ so 
			$$F \eL(x,\dot{x}) = \left( x, \frac{\partial A_i \dot{x}^i}{\partial \dot{x}}\right) = (x, A_1, \dots , A_n).$$
		\item $\eL(x,\dot{x}) =  \frac{1}{2}g_{ij}(x) \dot{x}^i \dot{x}^j.$
			$$F \eL(x,\dot{x}) =\left(x,\frac{1}{2} \frac{\partial g_{ij}\dot{x}^i \dot{x}^j}{\partial \dot{x} }\right)  = (x,g_{1j}(x) \dot{x}^j, \dots , g_{nj}\dot{x}^j ) = (x, g(x) \dot{x}).$$
		\item $\eL(x, \dot{x}) =\frac{1}{k!} K_{i_1, \dots , i_k}\dot{x}^{i_1} \cdot \dots \cdot \dot{x}^{i_k}.$ We first compute the partial derivatives of $f_K$ with respect to $\dot{x}^m$.
			$$\frac{\partial \dot{x}^{i_1} \dots  \dot{x}^{i_k}}{\partial \dot{x}^m} =\begin{cases}  0 & \text{ $i_j\neq m$ for all $j$}
			\\  \frac{\dot{x}^{i_1} \dots \dot{x}^{i_j}}{\dot{x}^m}& \text{if $i_j = m$ at some $j$. }\end{cases}	$$
	Therefore the fiber derivative of $f_K$ is:
			$$F \eL(x,\dot{x}) = \left(x, \frac{1}{k!} K_{i_1, \dots , i_k} \frac{x^{i_1} \dots x^{i_k}}{\dot{x}^1}, \dots , \frac{1}{k!}   K_{i_1, \dots , i_k} \frac{\dot{x}^{i_1} \cdots \dot{x}^{i_k}}{\dot{x}^n}\right)$$
Where it is understood that the coefficients that do not have an $\dot{x}^m$ dependance are 0. 
	\end{enumerate}
\item First note that $\eL$ will have the following fiber derivative: 
	$$F\eL(x, \dot{x}) = (x, g(x)\cdot \dot{x} + A(x)).$$
This is a smooth mapping since $g,A$ both are. Furthermore, since $g$ is nondegenerate it has a smooth inverse given as follows: 
$$H(x,y) = (x, g^{-1}(x) \cdot(y - A(x))).$$
This will be smooth since it is composed of smooth functions. It remains to verify that it is in fact the inverse of $g$:
$$H\circ F \eL(x,y) = H ( x, g(x) + A(x)) =(x , g^{-1}(x) \left(g(x) \cdot \dot{x} + A(x) - A(x) \right))  = (x, \dot{x}).$$
The composition $F \eL \circ H$ is computed similarly. Therefore $F \eL$ is a diffeomorphism. 
\item We compute the Euler Lagrange equations for our system: 
	$$\frac{d}{dt} \left( \frac{\partial \eL}{\partial \dot{x}} \right) = \frac{\partial \eL}{\partial x}.$$
	We already have computed $\frac{\partial \eL}{\partial \dot{x}^i}$ as
	$$ \frac{\partial \eL}{\partial \dot{x}^i} = g_{ij}  \dot{x}^j + A_i(x).$$
Taking the time derivative along a path $x$, we get that 
$$\frac{d}{dt} \left(\frac{\partial \eL}{\partial \dot{x}^i} \right) =\partial_k g_{ij}\dot{x}^k \dot{x}^j +g_{ij}\ddot{x}^{j} + \partial_j A_i \dot{x}^j.$$
Similarly we compute: 
$$\frac{\partial \eL}{\partial x^i} = \frac{1}{2} \partial_i g_{pq} \dot{x}^p \dot{x}^q +\partial_i A_j \dot{x}^j - \partial_i V.$$
Setting these equal to eachother we obtain the equations of motion. 
$$\partial_k g_{ij}\dot{x}^k \dot{x}^j +g_{ij}\ddot{x}^{j} + \partial_j A_i \dot{x}^j=\frac{1}{2} \partial_i g_{pq} \dot{x}^p \dot{x}^q + \partial_i A_j \dot{x}^j - \partial_i V. $$
\item The following equivalences hold: 
\begin{align*}
	  \text{L regular} & \iff \left(x, \frac{\partial \eL}{\partial \dot{x}}  \right)\text{ is a local diffeomorphism.}
	 \\ & \iff \pmat{I & 0 \\ * & \frac{\partial^2 \eL}{\partial \dot{x}^i \partial \dot{x}^j}}
	 \text{ is a linear isomorphism, by the inverse function theorem.}
	 \\ & \iff \pmat{\frac{\partial^2 \eL}{\partial {x}^i \partial {x}^j}} \text{ is non degenerate.}
\end{align*}
\item For $\eL = T_g + f_A - \pi^\ast V$, we compute the energy as:
$$E = \inn{(x, \dot{x}), (x, F \eL(x, \dot{x})} - L(x, \dot{x})=\inn{\dot{x}, g_{ij}(x) \cdot \dot{x} + A(x)} - T_g - f_A + \pi^\ast V = T_g +V. $$
We claim that this is conserved. Let $x(t)$ be a solution curve. We compute the derivative of energy along $x$. 
\begin{align*}
	\frac{d}{dt}E & = \frac{d}{dt} \left( \frac{1}{2}g_{ij}(x) \dot{x}_i \dot{x}_j + V(x) \right)
	\\ & = \frac{1}{2} \left(\grad g_{ij} (x) \cdot \dot{x} \right)\dot{x}^i \dot{x}^j+ \frac{1}{2} g_{ij}\ddot{x}^i \dot{x}^i + \frac{1}{2}g_{ij}\dot{x}^i \ddot{x}^j +  \grad V(x) \cdot \dot{x}
	\\ & = \left( \frac{1}{2}(\grad g_{ij}\cdot \dot{x}) \dot{x} + g_{ij}\ddot{x}^j + \grad V\right)\cdot \dot{x}
	\\ & = \left(  \frac{1}{2}(\grad g_{ij}\cdot \dot{x}) \dot{x} + g_{ij}\ddot{x} + \frac{1}{2}(\grad g_{ij} \cdot{x})\dot{x} - (\grad g_{ij} \cdot \dot{x}) \cdot{x} + dA(\dot{x}, -) - g_{ij}\ddot{x}   \right) \cdot \dot{x} \tag{By Substituting EoM}
	\\ & = dA(\dot{x}, \dot{x})
	\\ & = 0.
\end{align*}
By the way here is a simpler way to do it for a general (time independant) Lagrangian:
\begin{align*}
	\frac{d}{dt}E &= \frac{d}{dt}\left( \inn{ \frac{\partial \eL}{\partial \dot{x}}, \dot{x}} - \eL(x,\dot{x})\right)
	\\ & = \frac{d}{dt} \frac{\partial \eL}{\partial \dot{x}}\dot{x} + \frac{\partial \eL}{\partial \dot{x} }\ddot{x}- \left( \frac{\partial \eL}{\partial x}\dot{x} + \frac{\partial \eL }{\partial \dot{x}}\ddot{x} \right)
	\\ & = \left( \frac{d}{dt} \frac{\partial \eL}{\partial \dot{x}} - \frac{\partial \eL }{\partial x} \right)\dot{x} 
	\\ & = 0 \tag{By Euler-Lagrange Equations.}
\end{align*}
\epenum
\newpage
\begin{problem}
\end{problem}
\penum
\item We compute the Euler-Lagrange Equations: 
	$$\frac{d}{dt} \left(\frac{\partial \eL}{\partial \dot{x}^j} \right) = \frac{d}{dt}A_j(x) = \partial_i A_j(x) \dot{x}^i,$$
and
$$\frac{\partial \eL}{\partial x_j} = \partial_j A_i(x) \dot{x}^i.$$
These must be equal so we have that:
$$(\partial_i A_j(x) - \partial_jA_i(x))\dot{x}^i = 0.$$
This will give motion that is perpendicular to the vector fields given by $ \left(\partial_i A_j(x) -\partial_j A_i(x) \right)e^i$ for all $j$. 
Suppose that $x(t)$ is a solution and we reparametrize it to $x(t(s))$. Then by the chain rule we have that:
$$\left(\partial_i A_j(x(t(s)) - \partial_j A_i(x(t(s)) \right)\dot{x(t(s))}^i =(\partial_i A_j (x) - \partial_j A_i(x))\dot{x}^i \cdot \dot{t}  =0.$$
Therefore our reparametrization of $x$ is a solution as well. 
\item The equations of motion can be written in the following way: 
	$$\bmat{(\partial_i A_1 - \partial_1  A_i) \dot{x}^i  \\ \vdots \\ (\partial_i A_n - \partial_n A_i) \dot{x}^i} = 0.$$
This is the same as saying that as a $1-form$, $dA(\dot{x},- ) = 0$ for a critical path $\dot{x}$, since each row of the vector given above is of the form $dA(\dot{x}, e^i)$. 
\item If $A = xdy - ydx$, then $f_A = x\dot{y} - y\dot{x}$. We compute the Euler-Lagrange equations as:
	$$\frac{d}{dt} \left( \frac{\partial f_A}{\partial \dot{x}} \right) =-\dot{y}  ,\quad   \frac{ \partial f_A	 }{\partial x  } = \dot{y}$$
	and
	$$ \frac{ d }{dt  } \left( \frac{ \partial f_A }{\partial \dot{y}  } \right) =\dot{x}, \quad  \frac{ \partial f_A }{\partial y  } = - \dot{x}    $$ 
	Therefore we have that $\dot{x} = \dot{y} = 0$. The only critical paths are constant and so no such paths can exist from $(0,0)$ to $(1,0)$. 
\item In coordinates we have $f_A (\theta, \dot{\theta}) = \dot{\theta}$. By Euler-Lagrange equations, the equations of motion satisfy
	$$ \frac{d}{dt} \left(\frac{\partial f_A}{\partial \dot{\theta}} \right) = 0  , \quad \frac{\partial f_A}{\partial \theta} = 0 .  $$ 
	Under this lagrangian every path is critical.  So every path connecting a point to itself is critical.  
\item We write $A = \dot{z} - y \dot{x}$. We compute out the Euler-Lagrange equations:  
	$$ \frac{d}{dt} \left( \frac{\partial f_A}{\partial \dot{x}}  \right) = -\dot{y}, \quad \frac{\partial f_A}{\partial x}  = 0. $$ 
	$$ \frac{d}{dt} \left( \frac{\partial f_A}{\partial \dot{y}}  \right)  = 0 , \quad \frac{\partial f_A}{\partial y}  = -\dot{x}  $$ 
	$$ \frac{d}{dt} \left( \frac{\partial f_A}{\partial \dot{z}}  \right) = 0 ,  \quad \frac{\partial f_A}{\partial z}    = 0$$ 
	The restrictions are that $y=y_0, x=x_0$. Therefore any path of the form $(x_0, y_0,z(t))$ is critical. A critical trajectory between $ \left(x_1,y_1,z_1 \right) $ and $ \left(x_0, y_0, z_0 \right) $ exists when $y_0 = y_1, x_0 = x_1$.  
\item Let $A = A_i(x) dx^i$. We compute the fiber derivative of $f_A$:
	$$ FL(x,\dot{x}) = \left(x, \frac{\partial f_A}{\partial \dot{x}} \right) = \left(x, A(x) \right) . $$ 
	This will never be an isomorphism for any choice of $A$, since this will output a $1-$dimensional subspace. 
\item We compute the Euler-Lagrange equations for $\eL + f_A$:
	$$ \frac{d}{dt} \left( \frac{\partial \eL}{\partial \dot{x}^i} + \frac{\partial f_A}{\partial\dot{x}^i} \right) = \frac{\partial \eL}{\partial x^i} + \frac{\partial f_A}{\partial x^i}  \implies \frac{d}{dt} \left( \frac{\partial \eL}{\partial \dot{x}^i } \right) = \frac{\partial \eL}{\partial x^i} + \left( \frac{\partial f_A}{\partial x^i} - \frac{d}{dt} \left( \frac{\partial f_A}{\partial \dot{x}^i} \right)\right).$$
	We have already computed that 
	$$  \frac{\partial f_A}{\partial x^i} - \frac{d}{dt} \left( \frac{\partial f_A}{\partial \dot{x}^i} \right) = \left[\partial_j A_i - \partial_i A_j \right]\dot{x}^j = 0 $$
	since $dA =0$.
Therefore a path is critical for $\eL+f_A$ if and only if it is critical for $\eL$. We compute the energy as follows: 
$$ E(\eL + f_A) = \inn{ \frac{\partial \eL}{\partial \dot{x}} + \frac{\partial f_A}{\partial \dot{x}},\dot{x} } - \eL - f_A = \inn{ \frac{\partial \eL}{\partial \dot{x}}, \dot{x}} + f_A - \eL - f_A = E(\eL). $$
\epenum
\newpage
\begin{problem}
\end{problem}
Let $\eL_1 = T_{gE}$, $\eL_2 = T_{g} - V$. Solutions of $\eL_1$ will extremize the action
$$ S[x] = \int_{s_0}^{s_1}  \frac{1}{2}{g_E}_{ij}(x) \dot{x}^i \dot{x}^j ds,  $$
and similarly solutions of $\eL_2$ will extremize the actions 
$$S[y] = \int_{t_0}^{t_1} \frac{1}{2}g_{ij}(y)\dot{y}^i \dot{y}^j - V(y) dt.$$
We make the change of variables $ds = \frac{1}{2}(E-V(x))dt$, 
\begin{align*}
	S[y] & = \int_{s_0}^{s_1} g_{ij}(y)\dot{y}^i \dot{y}^j \cdot \left( \frac{ds}{dt}\right)^2\cdot \frac{2ds}{(E-V)} - \int_{t_0}^{t_1} E - \frac{1}{2}g_{ij}\dot{y}^i \dot{y}^j dt \tag{Since $T_g + V = E$}
	\\ & =  \int_{s_0}^{s_1}\frac{1}{4}(E-V(x)) g_{ij}(y)\dot{y}^i \dot{y}^j ds - \int_{t_0}^{t_1} E - \frac{1}{2}g_{ij}\dot{y}^i \dot{y}^j dt \tag{Simplifying}
	\\ & = \int_{s_0}^{s_1} \frac{1}{4}(E-V)g_{ij}(y) \dot{y}^i \dot{y}^j ds + \int_{s_0}^{s_1} \frac{1}{4} (E-V)g_{ij}(y) \dot{y}^i \dot{y}^j ds - E(t_1-t_0) \tag{Making the same substitution in second integral}
	\\ & = \int_{s_0}^{s_1} g_E(y) \dot{y}^i \dot{y}^j ds - E(t_1-t_0).
\end{align*}
Therefore if $S[y]$ is extremized, so will be $S[x]$, after our reparametrization. Therefore solutions to $\eL_2$ can be reparametrized to give a solution to $\eL_2$. A similar argument with the inverse reparametrization gives the reverse. 
\newpage
\begin{problem}
\end{problem}
\penum
\item The form $g_p$ is clearly linear and symmetric. We must show that it is positive definite. Consider a basis for $T_pX$, $\{e^i\}$. For any $\theta \in S^1$, we have a basis for $T_\theta S^1$ given by $1$, since $T_\theta S^1 \cong\R$. It is sufficient to verify that $g_P$ is positive definite
on vectors of the form $(e^i, 1)$.
$$g_p((e^i, 1),(e^i,1)) = g(x)(e^i, e^i) + (A + d\theta )(e^i,1) \cdot (A + d\theta)(e^i, 1) =g_{ii}(x) + \left(A_i(x)  + 1\right)^2. $$
This will always be positive since $g$ is a metric. Finally we show that this metric is non degenerate. Clearly if $ (v,c) = (0,0)$, $g_P((v,c),(v,c)) = 0$. Suppose that $g_P((v,c),(v,c)) = 0$. Then, 
$$0 = g(x)v^{i}v^{j}  +  \left(Av + c\right)^2.$$
Since $g_{ij} \dot{x}^i \dot{x}^j \geq 0$ we must have that $v = 0$ for the above quantity to be $0$. From this it follows that $c^2 = 0$ and clearly $c = 0$. 
\item For $\eL = T_g + f_A$ we have that $\theta$ will obviously be a cyclic variable; this lagrangian does not depend on $\theta$. The corresponding conserved quantity will be
	$$ \frac{d}{dt} \left( \frac{\partial \eL}{\partial \dot{\theta}} \right) = \frac{d}{dt}0 = 0.$$
So $0$ is conserved. 
We now give expressions for the generalized momenta:
$$ \frac{\partial \eL}{\partial \dot{x}^i}  = g_{ij}\dot{x}^j+ A_i. $$
For $T_{g_p}$, we can write this as 
	$$ T_{g_p} =  \frac{1}{2} \left(g_{ij}(x)\dot{x}^i \dot{x}^j + A_iA_j\dot{x}^i \dot{x}^j + 2\dot{\theta} A_i \dot{x}^i + \dot{\theta}^2 \right).$$
We can see once again that this lagrangian does not depend on $\theta$ ; it is a cyclic variable with corresponding conserved quantity
$$ 0 =  \frac{d}{dt} \left( \frac{\partial T_{g_p}}{\partial \dot{\theta}} \right) =  \frac{d}{dt} \left(A_i(x) \dot{x}^i + \dot{\theta}\right) = \partial_j A_i \dot{x}^i \dot{x}^j  + A_i \ddot{x}^i  + \ddot{\theta} . $$
We now give expression or the generalized momenta for $T_{g_p}$: 
$$ \frac{\partial T_{g_p}}{\partial \dot{x}^i} = g_{ij}\dot{x}^j + A_i A_j\dot{x}^j  + \dot{\theta} A_i  $$
\item We first compute the equations of motion for $\eL = T_g + f_A$:
	$$ \begin{cases}
		\frac{\partial \eL}{\partial x^i} &= \frac{1}{2}\partial_i g_{pq} \dot{x}^p \dot{x}^q + \partial_i A_j \dot{x}^j
		\\ \frac{d}{dt} \left( \frac{\partial \eL}{\partial \dot{x}^i} \right) &=  \partial_k g_{ij} \dot{x}^k\dot{x}^j + g_{ij} \ddot{x}^j + \partial_j A_i \dot{x}^j
	\end{cases}$$	
	Similarly we compute for $T_{g_P}$:
	$$ \begin{cases}
		\frac{\partial T_{g_p}}{\partial x^i} & = \frac{1}{2}\partial_i g_{pq} \dot{x}^p \dot{x}^q + (\partial_i A_p)A_q \dot{x}^p \dot{x}^q + \dot{\theta} \partial_i A_j \dot{x}^j
		\\ \frac{d}{dt} \left( \frac{\partial T_{g_p}}{\partial \dot{x}^i} \right) & = g_{ij} \ddot{x}^j + \partial_k g_{ij} \dot{x}^k \dot{x}^i + \partial_j A_i A_j \dot{x}^j \dot{x}^i + A_i \partial_k A_j \dot{x}^k \dot{x}^j + A_i A_j \ddot{x}^j + \ddot{\theta} A_i +  \dot{\theta} \partial_j A_i \dot{x}^j.
	\end{cases}$$
Note that we can rearrange the second equation as follows: 
\begin{align*}
	& g_{ij} \ddot{x}^j + \partial_k g_{ij} \dot{x}^k \dot{x}^i + \partial_j A_i A_j \dot{x}^j \dot{x}^i + A_i \partial_k A_j \dot{x}^k \dot{x}^j + A_i A_j \ddot{x}^j + \ddot{\theta} A_i +  \dot{\theta} \partial_j A_i \dot{x}^j
	\\ & =  g_{ij} \ddot{x}^j + \partial_k g_{ij} \dot{x}^k \dot{x}^i + \partial_j A_i A_j \dot{x}^j \dot{x}^i  +   \dot{\theta} \partial_j A_i \dot{x}^j + A_i \left(\ddot{\theta} + A_j \ddot{x}^j + \partial_k A_j \dot{x}^k \dot{x}^j  \right)
	\\ & =  g_{ij} \ddot{x}^j + \partial_k g_{ij} \dot{x}^k \dot{x}^i + \partial_j A_i A_j \dot{x}^j \dot{x}^i  +   \dot{\theta} \partial_j A_i \dot{x}^j \tag{Since conserved.}
\end{align*}
The equations of motion are given by:
$$  \frac{1}{2}\partial_i g_{pq} \dot{x}^p \dot{x}^q + (\partial_i A_p)A_q \dot{x}^p \dot{x}^q + \dot{\theta} \partial_i A_j \dot{x}^j =  g_{ij} \ddot{x}^j + \partial_k g_{ij} \dot{x}^k \dot{x}^i + \partial_j A_i A_j \dot{x}^j \dot{x}^i  +   \dot{\theta} \partial_j A_i \dot{x}^j. $$
We can rearrange this further:
$$ \frac{1}{2}\partial_i g_{pq} \dot{x}^p \dot{x}^q - g_{ij}\ddot{x}^j - \partial_k g_{ij}\dot{x}^k \dot{x}^j + \left(A_i\dot{x}^i + \dot{\theta} \right) \left(\partial_i A_j \dot{x}^j - \partial_j A_i \dot{x}^j \right) = 0.$$
Since $A_i\dot{x}^i + \dot{\theta}$ is a constant, we obtain equations of motions:
$$ \frac{1}{2}\partial_i g_{pq} \dot{x}^p \dot{x}^q - g_{ij}\ddot{x}^j - \partial_k g_{ij}\dot{x}^k \dot{x}^j + C \left(\partial_i A_j \dot{x}^j - \partial_j A_i \dot{x}^j \right) = 0.$$
For some parameter $C$, which we can set to $1$. Note that this is exactly the equations of motion on $X$ given by $\eL = T_g + f_A$. Conversely, given an equation of motion on $X$ of $\eL$, we can embed $X$ into $P$, and it will satisfy the equations of motion of $T_{g_p}$ by specifiying $\dot{\theta} = 1 - A_i\dot{x}$. 
\epenum
\newpage
\end{document}
