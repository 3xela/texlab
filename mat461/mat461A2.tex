\documentclass[12pt, a4paper]{article}
\usepackage[lmargin =0.5 in, 
rmargin=0.5in, 
tmargin=1in,
bmargin=0.5in]{geometry}
\geometry{letterpaper}
\usepackage{tikz-cd}
\usepackage{amsmath}
\usepackage{amssymb}
\usepackage{blindtext}
\usepackage{titlesec}
\usepackage{enumitem}
\usepackage{fancyhdr}
\usepackage{amsthm}
\usepackage{graphicx}
\usepackage{cool}
\usepackage{thmtools}
\usepackage{hyperref}
\graphicspath{ }					%path to an image

%-------- sexy font ------------%
%\usepackage{libertine}
%\usepackage{libertinust1math}

%\usepackage{mlmodern}				% very nice and classic
%\usepackage[utopia]{mathdesign}
%\usepackage[T1]{fontenc}


\usepackage{mlmodern}
\usepackage{eulervm}
%\usepackage{tgtermes} 				%times new roman
%-------- sexy font ------------%


% Problem Styles
%====================================================================%


\newtheorem{problem}{Problem}


\theoremstyle{definition}
\newtheorem{thm}{Theorem}
\newtheorem{lemma}{Lemma}
\newtheorem{prop}{Proposition}
\newtheorem{cor}{Corollary}
\newtheorem{fact}{Fact}
\newtheorem{defn}{Definition}
\newtheorem{example}{Example}
\newtheorem{question}{Question}

\newtheorem{manualprobleminner}{Problem}

\newenvironment{manualproblem}[1]{%
	\renewcommand\themanualprobleminner{#1}%
	\manualprobleminner
}{\endmanualprobleminner}

\newcommand{\penum}{ \begin{enumerate}[label=\bf(\alph*), leftmargin=0pt]}
	\newcommand{\epenum}{ \end{enumerate} }

% Math fonts shortcuts
%====================================================================%

\newcommand{\ring}{\mathcal{R}}
\newcommand{\N}{\mathbb{N}}                           % Natural numbers
\newcommand{\Z}{\mathbb{Z}}                           % Integers
\newcommand{\R}{\mathbb{R}}                           % Real numbers
\newcommand{\C}{\mathbb{C}}                           % Complex numbers
\newcommand{\F}{\mathbb{F}}                           % Arbitrary field
\newcommand{\Q}{\mathbb{Q}}                           % Arbitrary field
\newcommand{\PP}{\mathcal{P}}                         % Partition
\newcommand{\M}{\mathcal{M}}                         % Mathcal M
\newcommand{\eL}{\mathcal{L}}                         % Mathcal L
\newcommand{\T}{\mathbb{T}}                         % Mathcal T
\newcommand{\U}{\mathcal{U}}                         % Mathcal U\\
\newcommand{\V}{\mathcal{V}}                         % Mathcal V

% symbol shortcuts
%====================================================================%

\newcommand{\bd}{\partial}
\newcommand{\grad}{\nabla}
\newcommand{\lam}{\lambda}
\newcommand{\imp}{\implies}
\newcommand{\all}{\forall}
\newcommand{\exs}{\exists}
\newcommand{\delt}{\delta}
\newcommand{\ep}{\varepsilon}
\newcommand{\ra}{\rightarrow}
\newcommand{\vph}{\varphi}

\newcommand{\ol}{\overline}
\newcommand{\f}{\frac}
\newcommand{\lf}{\lfrac}
\newcommand{\df}{\dfrac}

% bracketting shortcuts
%====================================================================%
\newcommand{\abs}[1]{\left| #1 \right|}
\newcommand{\babs}[1]{\Big|#1\Big|}
\newcommand{\bound}{\Big|}
\newcommand{\BB}[1]{\left(#1\right)}
\newcommand{\dd}{\mathrm{d}}
\newcommand{\artanh}{\mathrm{artanh}}
\newcommand{\Med}{\mathrm{Med}}
\newcommand{\Cov}{\mathrm{Cov}}
\newcommand{\Corr}{\mathrm{Corr}}
\newcommand{\tr}{\mathrm{tr}}
\newcommand{\Range}[1]{\mathrm{range}(#1)}
\newcommand{\Null}[1]{\mathrm{null}(#1)}
\newcommand{\lan}{\langle}
\newcommand{\ran}{\rangle}
\newcommand{\norm}[1]{\left\lVert#1\right\rVert}
\newcommand{\inn}[1]{\lan#1\ran}
\newcommand{\op}[1]{\operatorname{#1}}
\newcommand{\bmat}[1]{\begin{bmatrix}#1\end{bmatrix}}
\newcommand{\pmat}[1]{\begin{pmatrix}#1\end{pmatrix}}
\newcommand{\vmat}[1]{\begin{vmatrix}#1\end{vmatrix}}

\newcommand{\amogus}{{\bigcap}\kern-0.8em\raisebox{0.3ex}{$\subset$}}
\newcommand{\Note}{\textbf{Note: }}
\newcommand{\Aside}{{\bf Aside: }}
%restriction
%\newcommand{\op}[1]{\operatorname{#1}}
%\newcommand{\done}{$$\mathcal{QED}$$}

%====================================================================%


\setlength{\parindent}{0pt}      	% No paragraph indentations
\pagestyle{fancy}
\fancyhf{}							% fancy header

\setcounter{secnumdepth}{0}			% sections are numbered but numbers do not appear
\setcounter{tocdepth}{2} 			% no subsubsections in toc

%template
%====================================================================%
%\begin{manualproblem}{1}
%Spivak.
%\end{manualproblem}

%\begin{proof}[Solution]
%\end{proof}

%----------- or -----------%

%\begin{problem} 		
%\end{problem}	

%\penum
%	\item
%\epenum
%====================================================================%


\newcommand{\Course}{MAT461}
\newcommand{\hwNumber}{2}

%preamble

\title{MAT461 A2}
\author{A.N.}
\date{\today}
\lhead{\Course A\hwNumber}
\rhead{\thepage}
%\cfoot{\thepage}


%====================================================================%
\begin{document}
\maketitle{}


\begin{problem}
\end{problem}
\penum
\item We compute the fiber derivatives of each of the given maps. 
	\begin{enumerate}[label = \roman*)]
		\item $\eL  (x , \dot{x}) = \pi^\ast V(x,\dot{x})$. Since $\pi^\ast V$ is constant in $\dot{x}$ we compute: 
			$$F \eL(x,\dot{x}) =\left(x, \frac{\partial \pi^\ast V}{\partial \dot{x}}\right) = (x,0).$$
		\item $\eL (x,\dot{x}) = A_i\dot{x}^i$. This is linear in $\dot{x}$ so 
			$$F \eL(x,\dot{x}) = \left( x, \frac{\partial A_i \dot{x}^i}{\partial \dot{x}}\right) = (x, A_1, \dots , A_n).$$
		\item $\eL(x,\dot{x}) =  \frac{1}{2}g_{ij}(x) \dot{x}^i \dot{x}^j.$
			$$F \eL(x,\dot{x}) =\left(x,\frac{1}{2} \frac{\partial g_{ij}\dot{x}^i \dot{x}^j}{\partial \dot{x} }\right)  = (x,g_{1j}(x) \dot{x}^j, \dots , g_{nj}\dot{x}^j ) = (x, g(x) \dot{x}).$$
		\item $\eL(x, \dot{x}) =\frac{1}{k!} K_{i_1, \dots , i_k}\dot{x}^{i_1} \cdot \dots \cdot \dot{x}^{i_k}.$ We first compute the partial derivatives of $f_K$ with respect to $\dot{x}^m$.
			$$\frac{\partial \dot{x}^{i_1} \dots  \dot{x}^{i_k}}{\partial \dot{x}^m} =\begin{cases}  0 & \text{ $i_j\neq m$ for all $j$}
			\\  \frac{\dot{x}^{i_1} \dots \dot{x}^{i_j}}{\dot{x}^m}& \text{if $i_j = m$ at some $j$. }\end{cases}	$$
	Therefore the fiber derivative of $f_K$ is:
			$$F \eL(x,\dot{x}) = \left(x, \frac{1}{k!} K_{i_1, \dots , i_k} \frac{x^{i_1} \dots x^{i_k}}{\dot{x}^1}, \dots , \frac{1}{k!}   K_{i_1, \dots , i_k} \frac{\dot{x}^{i_1} \cdots \dot{x}^{i_k}}{\dot{x}^n}\right)$$
Where it is understood that the coefficients that do not have an $\dot{x}^m$ dependance are 0. 
	\end{enumerate}
\item First note that $\eL$ will have the following fiber derivative: 
	$$F\eL(x, \dot{x}) = (x, g(x)\cdot \dot{x} + A(x)).$$
This is a smooth mapping since $g,A$ both are. Furthermore, since $g$ is nondegenerate it has a smooth inverse given as follows: 
$$H(x,y) = (x, g^{-1}(x) \cdot(y - A(x))).$$
This will be smooth since it is composed of smooth functions. It remains to verify that it is in fact the inverse of $g$:
$$H\circ F \eL(x,y) = H ( x, g(x) + A(x)) =(x , g^{-1}(x) \left(g(x) \cdot \dot{x} + A(x) - A(x) \right))  = (x, \dot{x}).$$
The composition $F \eL \circ H$ is comptuted similarly. Therefore $F \eL$ is a diffeomorphism. 
\item We compute the Euler Lagrange equations for our system: 
	$$\frac{d}{dt} \left( \frac{\partial \eL}{\partial \dot{x}} \right) = \frac{\partial \eL}{\partial x}.$$
	We already have computed $\frac{\partial \eL}{\partial \dot{x}^i}$ as
	$$g_{ij} \dot{x}^j + A_i(x).$$
Taking the time derivative along a path $x$, we get that 
$$\frac{d}{dt} \left(\frac{\partial \eL}{\partial \dot{x}^i} \right) =\partial_kg_{ij}\dot{x}^k \dot{x}^j +g^{ij}\ddot{x}^{i} + \partial_j A_i \dot{x}^j.$$
Similarly we compute: 
$$\frac{\partial \eL}{\partial x^i} = \partial_i g_{pq} \dot{x}^p \dot{x}^q +\partial_i A_j \dot{x}^j + \partial_i V.$$
Setting these equal to eachother we obtain the equations of motion. 
$$ \partial_k g_{ij} \dot{x}^k \dot{x}^j + g_{ij} \ddot{x}^i + \partial_j A_i \dot{x}^j = \partial_i g_{pq} \dot{x}^p \dot{x}^q + \partial_i A_j \dot{x}^j + \partial_i V. $$
\item The following equivalences hold: 
\begin{align*}
	  \text{L regular} & \iff \left(x, \frac{\partial \eL}{\partial \dot{x}}  \right)\text{ is a local diffeomorphism.}
	 \\ & \iff \pmat{I & 0 \\ * & \frac{\partial^2 \eL}{\partial \dot{x}^i \partial \dot{x}^j}}
	 \text{ is a linear isomorphism, by the inverse function theorem.}
	 \\ & \iff \pmat{\frac{\partial^2 \eL}{\partial {x}^i \partial {x}^j}} \text{ is non degenerate.}
\end{align*}
\item For $\eL = T_g + f_A - \pi^\ast V$, we compute the energy as:
	$$E = \inn{(x, \dot{x}), (x, F \eL(x, \dot{x})} - L(x, \dot{x})=\inn{\dot{x}, g_{ij}(x) \cdot \dot{x} + A(x)} - T_g - F_A + \pi^\ast V = V(x). $$
	Since $V(x)$ is independant of time, this will be conserved along any solution curve. 
\epenum
\newpage
\begin{problem}
\end{problem}
\penum
\item We compute the Euler-Lagrange Equations: 
	$$\frac{d}{dt} \left(\frac{\partial \eL}{\partial \dot{x}^j} \right) = \frac{d}{dt}A_j(x) = \partial_i A_j(x) \dot{x}^i,$$
and
$$\frac{\partial \eL}{\partial x_j} = \partial_j A_i(x) \dot{x}^i.$$
These must be equal so we have that:
$$(\partial_i A_j(x) - \partial_jA_i(x))\dot{x}^i = 0.$$
This will give motion that is (...).
Suppose that $x(t)$ is a solution and we reparametrize it to $x(t(s))$. Then by the chain rule we have that:
$$\left(\partial_i A_j(x(t(s)) - \partial_j A_i(x(t(s)) \right)\dot{x(t(s))}^i =(\partial_i A_j (x) - \partial_j A_i(x))\dot{x}^i \cdot \dot{t}  =0.$$
Therefore our reparametrization of $x$ is a solution as well. 
\item The equations of motion can be written in the following way: 
	$$\bmat{(\partial_i A_1 - \partial_1  A_i) \dot{x}^i  \\ \vdots \\ (\partial_i A_n - \partial_n A_i) \dot{x}^i} = 0$$
\item 
\item 
\item 
\item
\item 
\epenum
\newpage
\begin{problem}
\end{problem}
We determine the equations of motion for $\eL_1 = T_{g_E} = \frac{1}{2}(E-V(x)) g_{ij}(x)\dot{x}^i \dot{x}^j$: 

\begin{align*}
\frac{d}{dt} \left(\frac{\partial \eL_1}{\partial \dot{x}^k}\right) &= \frac{d}{dt} \left(\frac{1}{2}(E-V(x))g_{kj}(x) \dot{x}^j \right)
\\ & = -\grad V(x) \cdot \dot{x} g_{ki}(x)\dot{x}^i  +  \left( \frac{g_{ij}(x)}{\partial x^k} \dot{x}^k \dot{x}^i + g_{ij}(x)\ddot{x}^j \right).
\end{align*}
And,
$$\frac{\partial \eL_1}{\partial x^k} =-\frac{1}{2} \frac{\partial V}{\partial x^k}  g_{ij}\dot{x}^i \dot{x}^j  + \frac{1}{2}(E - V(x)) \frac{\partial g_{ij}(x)}{\partial x^k} \dot{x}^i \dot{x}^j.$$
Therefore the equations of motion satisfy: 
$$-\frac{1}{2} \grad V(x) \cdot \dot{x} g_{ki}(x)\dot{x}^i  + \frac{1}{2} \left( \frac{g_{ij}(x)}{\partial x^k} \dot{x}^k \dot{x}^i + g_{ij}(x)\ddot{x}^j \right) =  -\frac{1}{2} \frac{\partial V}{\partial x^k}  g_{ij}\dot{x}^i \dot{x}^j  + \frac{1}{2}(E - V(x)) \frac{\partial g_{ij}(x)}{\partial x^k} \dot{x}^i \dot{x}^j$$
Similarly for $\eL_2 = T_g - V$, we compute:
	$$\frac{d}{dt} \left( \frac{\partial \eL_2}{\partial \dot{x}^k}\right)  = \frac{d}{dt} \left( g_{ki}(x) \dot{x}^i \right) 
 = \frac{ \partial g_{ki}(x)}{\partial x^i} \dot{x}^i \dot{x}^j + g_{ki}(x) \ddot{x}^i $$
\newpage
\begin{problem}
\end{problem}
\penum
\item The form $g_p$ is clearly linear and symmetric. We must show that it is positive definite. Consider a basis for $T_pX$, $\{e^i\}$. For any $\theta \in S^1$, we have a basis for $T_\theta S^1$ given by $1$, since $T_\theta S^1 \cong\R$. It is sufficient to verify that $g_P$ is positive definite
on vectors of the form $(e^i, 1)$.
$$g_p((e^i, 1),(e^i,1)) = g(x)(e^i, e^i) + (A + d\theta )(e^i,1) \cdot (A + d\theta)(e^i, 1) =g_{ii}(x) + A_i(x)^2  + 1. $$
This will always be positive since $g$ is a metric. Finally we show that this metric is non degenerate. Clearly if $u = (v,c)$, $g_P((v,c),(v,c)) = 0$. Suppose that $g_P((v,c),(v,c)) = 0$. Then, 
$$0 = g(x)v^{i}v^{j}  +  Av^2 + c^2 \implies c=v=0.$$
\item 
\item 
\epenum
\newpage

\end{document}
