\documentclass[letterpaper]{article}
\usepackage[letterpaper,margin=1in,footskip=0.25in]{geometry}
\usepackage[utf8]{inputenc}
\usepackage{amsmath}
\usepackage{amsthm}
\usepackage{amssymb, pifont}
\usepackage{mathrsfs}
\usepackage{enumitem}
\usepackage{fancyhdr}
\usepackage{hyperref}

\pagestyle{fancy}
\fancyhf{}
\rhead{MAT 367}
\lhead{Assignment 1}
\rfoot{Page \thepage}

\setlength\parindent{24pt}
\renewcommand\qedsymbol{$\blacksquare$}

\DeclareMathOperator{\T}{\mathcal{T}}
\DeclareMathOperator{\V}{\mathcal{V}}
\DeclareMathOperator{\U}{\mathcal{U}}
\DeclareMathOperator{\Prt}{\mathbb{P}}
\DeclareMathOperator{\R}{\mathbb{R}}
\DeclareMathOperator{\N}{\mathbb{N}}
\DeclareMathOperator{\Z}{\mathbb{Z}}
\DeclareMathOperator{\Q}{\mathbb{Q}}
\DeclareMathOperator{\C}{\mathbb{C}}
\DeclareMathOperator{\ep}{\varepsilon}
\DeclareMathOperator{\identity}{\mathbf{0}}
\DeclareMathOperator{\card}{card}
\newcommand{\suchthat}{;\ifnum\currentgrouptype=16 \middle\fi|;}

\newtheorem{lemma}{Lemma}

\newcommand{\bd}{\partial}
\newcommand{\tr}{\mathrm{tr}}
\newcommand{\ra}{\rightarrow}
\newcommand{\lan}{\langle}
\newcommand{\ran}{\rangle}
\newcommand{\norm}[1]{\left\lVert#1\right\rVert}
\newcommand{\inn}[1]{\lan#1\ran}
\newcommand{\ol}{\overline}
\begin{document} \noindent Q1:
\begin{description}
\item{$(a) \implies (b)$} \newline  Suppose $a$ holds. Take $a\in M$, with open neighbourhood $U$ and function $f$ which satisfies the assumption. We can regard our function $f$ as a map from $U\cap (\R_y^{k} \times \R_z^{n-k})$ to $\R^{n-k}$ with coordinates permuted so that $ \Big[ \frac{\partial f}{\partial z} \Big]$ has rank $n-k$. 
By the implicit function theorem there is a $C^r$ function from $g: U^\prime \subset \R_y^{k} \to V^\prime \subset \R_z^{n-k}$ which satisfies $f(x,g(x)) = 0$ and $(a_{k+1}, \dots, a_n) = g(a_{1} , \dots , a_{k} )$. This is the desired graph, and it will parametrize some neighbourhood of the point $a$ on $M$ since $f=0$ exactly on the manifold. 
\item{$(b) \implies (a)$} \newline  
Now suppose that condition $b$ holds. Let $U$ be a neighbourhood of $M$ such that $M$ is the graph of function $g$. We define $F(y,z): U \to \R^{n-k}$ by $F(y,z) =g(y) -z$. We have that $F\in C^r$ and $F=0$ if and only if $(y,z)\in M\cap U$. Furthermore, we have that $$DF = \Big[ Dg \Big| -I \Big]$$ which will have a rank of $n-k$ on $M\cap U$ since the identity is of maximal rank.
\item{$(a) \implies (c)$} \newline 
Suppose condition $c$ holds. Pick coordinates $(x,y)$ such that when we regard $f$ as a map from $U\cap (\R_x^k \times \R_y^{n-k})$ we have that $\Big[ \frac{\partial f}{\partial y} \Big]$ is non singular. Define our function $h$ as $h(x,y) = (x,f(x,y)).$ We have that $$Dh = \begin{bmatrix}
    I & 0 \\ * & \frac{\partial f}{\partial y}
\end{bmatrix}$$
This is invertible by assumption, hence $h$ is a $C^r$ diffeomorphism by the inverse function theorem. Furthermore, if $(x,y)\in M$ then $$h(x,y) = (x,f(x,y)) = (x,0).$$ The point $x\in \R^k$ so $h$ is of desired form. 
\item{$(c)\implies (a)$} \newline
We now show that $c$ implies $a$. Given $h: U \to V$, a diffeomorphism satisfying the hypothesis of $c$, define our function $f$ as $\pi_{n-k}\circ h$, where $\pi_{n-k}$ is the projection onto the last $n-k$ coordinates. If $x\in M\cap U$, then $$\pi_{n-k}\circ h(x) = \pi_{n-k} (y_1\dots y_k, 0 \dots 0) =0.$$ Note that since $h$ has full rank as it is a diffeomorphism from a subsets of $\R^n$ to $\R^n$, it must have full rank. Composing it with $\pi_{n-k}$ leaves the composition with rank $n-k$. 
\item{$(c) \implies (d)$} \newline 
Given a diffeomorphism $h: M \to \R^n$ Next we show that condition $c$ implies $d$. Let $h,U,V$ be as given by $c$. Define the set $W$ as the projection onto the first $k$ coordinates of the set $V$. 
We take our coordinate patch $\varphi : = h^{-1} \circ \iota_k$ where $\iota_k$ is the injection from $\R^k$ into $\R^n$. Note that $\varphi$ is injective, since it is the composition of two injective functions. 
We also have that $$D\varphi = D(h^{-1}\circ \iota_k) \cdot D\iota_k.$$ The rank of $D(h^\prime \circ \iota_k)$ is $n$ since $h^{-1}$ is a diffeomorphism, and the rank of $D\iota_k$ is $k$. Hence their product will be rank $k$ as well.
Furthermore, by definition of $\varphi$ we have that $$\varphi(W) = h^{-1}(V \times \{0\}) = M\cap U.$$ 
It remains to show that $\varphi $ is continuous with respect to the subspace topology. Take $\Omega \subset W$ to be any open set. Then, $$\varphi(\Omega) = h^{-1}(\iota (\Omega)).$$ The inclusion is continuous with respect to the subspace toplogy, and since $h^{-1}$ is a continuous bijection it is as well. Hence their composition must be as well. 
\item{$(d) \implies (c)$} \newline
Finally we show that $d$ implies $c$. On $W \times \R_y^{n-k}$ define $l(x,y) = \varphi(x) + (0,y).$ This is a map into a manifold, and we have that $$l^\prime(x,y) = \begin{bmatrix}
    \frac{\partial \varphi}{\partial x} & 0 \\ 0 & I
\end{bmatrix}$$ Since $\frac{\partial \varphi}{\partial x}$ has rank $k$, $l^\prime(x,y)$ is invertible and so $l(x,y)$ is invertible on some neighbourhood $V \subset W \times \R^{n-k}$. Set $U = l(V)$. It remains to show that if $V^\prime \subset V$ open, then $l|_{V^\prime \cap \R^k}$ is contained in some $U^\prime \cap M \subset U \cap M$ open. Since on $V\cap \R^k$, $l$ agrees with $\varphi$, it is enough to show that $\varphi(V^\prime \cap \R^k ) $ is open in $M$. Indeed since $\varphi$ has a continuous inverse and is 1-1, it must carry open sets to open sets. So $\varphi(V^\prime \cap \R^k)$ must be open in $M \cap U$. There exists some $U^{\prime \prime}$ so that $\varphi(V^\prime \cap \R^k) = U^{\prime \prime} \cap M.$ Set $V^{\prime \prime } = l^{-1}(U^{\prime \prime})$
\end{description}
\end{document}
