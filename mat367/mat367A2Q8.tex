\documentclass[letterpaper]{article}
\usepackage[letterpaper,margin=1in,footskip=0.25in]{geometry}
\usepackage[utf8]{inputenc}
\usepackage{amsmath}
\usepackage{amsthm}
\usepackage{amssymb, pifont}
\usepackage{mathrsfs}
\usepackage{enumitem}
\usepackage{fancyhdr}
\usepackage{hyperref}

\pagestyle{fancy}
\fancyhf{}
\rhead{MAT 367}
\lhead{Assignment 2}
\rfoot{Page \thepage}

\setlength\parindent{24pt}
\renewcommand\qedsymbol{$\blacksquare$}

\DeclareMathOperator{\T}{\mathcal{T}}
\DeclareMathOperator{\V}{\mathcal{V}}
\DeclareMathOperator{\U}{\mathcal{U}}
\DeclareMathOperator{\Prt}{\mathbb{P}}
\DeclareMathOperator{\R}{\mathbb{R}}
\DeclareMathOperator{\N}{\mathbb{N}}
\DeclareMathOperator{\Z}{\mathbb{Z}}
\DeclareMathOperator{\Q}{\mathbb{Q}}
\DeclareMathOperator{\C}{\mathbb{C}}
\DeclareMathOperator{\ep}{\varepsilon}
\DeclareMathOperator{\identity}{\mathbf{0}}
\DeclareMathOperator{\card}{card}
\newcommand{\suchthat}{;\ifnum\currentgrouptype=16 \middle\fi|;}

\newtheorem{lemma}{Lemma}

\newcommand{\bd}{\partial}
\newcommand{\tr}{\mathrm{tr}}
\newcommand{\ra}{\rightarrow}
\newcommand{\lan}{\langle}
\newcommand{\ran}{\rangle}
\newcommand{\norm}[1]{\left\lVert#1\right\rVert}
\newcommand{\inn}[1]{\lan#1\ran}
\newcommand{\ol}{\overline}
\begin{document} 
\noindent Q8a: 
By $Q4$ it is enough to show that the identity $id: S^1 \to S^1/\{z=-z\}$ is a diffeomorphism.
Consider coordinate charts on $S^1$ given by stereographic projection, $\psi_x, \psi_y$. 
Let $\R P^1$ have the atlas induced by the quotient mapping of projection of $S^1$, with 
coordinate charts given as $\varphi_x (x,y) = sgn(x)y$ for $x\neq 0$ and $\psi_y(x,y) = sgn(y)x$ for $y\neq 0$.
The inverses of the projections are given as $\varphi_x^{-1}(x) = (\sqrt{1-x^2}, x)$. Similarly for $\varphi_y^{-1}(y)$.
These are smooth on the domain since we exclude $x,y = \pm 1$. 
The inverses of $\psi_{x,y}$ are given as in $Q5$. We verify that the transition maps are smooth. Indeed, 
by question $5$ they are smooth. Since we can identify $S^1$ with $\R P^1$, by question $7$ they are diffeomorphic. 
\newline \\ Q8b: Given the charts on $\C P^1$ as $$U_1= \{[x,y] : x \neq 0\}, U_2 = \{[x,y] : y \neq 0 \},$$ 
with coordinate charts $\varphi_1([x,y]) = \frac{y}{x}$, $\varphi_2([x,y]) = \frac{x}{y}$. We note that when $x,y\neq 0$ we have that the transition map $\phi_1 \circ \phi_2^{-1}= \frac{1}{z}$. 
By question $7$, the charts on the riemann sphere and on $\C P^1$ have the same transition map and hence induce the same equivalence relation on points in $\C$, with the
equivalence relation defined as in $Q7$. Therefore the induced quotient space is the same and so $\C P^1$ is biholomorphic to $\S^2$.  

\end{document}