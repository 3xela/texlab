\documentclass[letterpaper]{article}
\usepackage[letterpaper,margin=1in,footskip=0.25in]{geometry}
\usepackage[utf8]{inputenc}
\usepackage{amsmath}
\usepackage{amsthm}
\usepackage{amssymb, pifont}
\usepackage{mathrsfs}
\usepackage{enumitem}
\usepackage{fancyhdr}
\usepackage{hyperref}

\pagestyle{fancy}
\fancyhf{}
\rhead{MAT 367}
\lhead{Assignment 2}
\rfoot{Page \thepage}

\setlength\parindent{24pt}
\renewcommand\qedsymbol{$\blacksquare$}

\DeclareMathOperator{\smooth}{\mathcal{C}^\infty}
\DeclareMathOperator{\T}{\mathcal{T}}
\DeclareMathOperator{\V}{\mathcal{V}}
\DeclareMathOperator{\U}{\mathcal{U}}
\DeclareMathOperator{\Prt}{\mathbb{P}}
\DeclareMathOperator{\R}{\mathbb{R}}
\DeclareMathOperator{\N}{\mathbb{N}}
\DeclareMathOperator{\Z}{\mathbb{Z}}
\DeclareMathOperator{\Q}{\mathbb{Q}}
\DeclareMathOperator{\C}{\mathbb{C}}
\DeclareMathOperator{\ep}{\varepsilon}
\DeclareMathOperator{\identity}{\mathbf{0}}
\DeclareMathOperator{\card}{card}
\newcommand{\suchthat}{;\ifnum\currentgrouptype=16 \middle\fi|;}

\newtheorem{lemma}{Lemma}

\newcommand{\bd}{\partial}
\newcommand{\tr}{\mathrm{tr}}
\newcommand{\ra}{\rightarrow}
\newcommand{\lan}{\langle}
\newcommand{\ran}{\rangle}
\newcommand{\norm}[1]{\left\lVert#1\right\rVert}
\newcommand{\inn}[1]{\lan#1\ran}
\newcommand{\ol}{\overline}
\begin{document} \noindent Q6: 
$a \implies b$: Suppose that $f:M\to N$ is a diffeomorphism.
 Suppose that $\psi_1 : V \to \R^n$ is a coordinate chart of $\mathcal{B}$. 
 Then we have that $\psi_1\circ f$ is a homeomorphism from $f^{-1}(V)$ to $\R^n$ since $f$ is a homeomorphism.
Now given $\varphi$, coordinate chart of $M$, $(\psi_1 \circ f) \circ \varphi^{-1}$ is a diffeomorphism since $f$ is.  
If $\psi_2$ is another chart of $N$, then $$(\psi_2 \circ f) \circ (\psi_1\circ f)^{-1} = \psi_2 \circ f \circ f^{-1} \circ \psi_1^{-1} = \psi_2 \circ \psi_1^{-1}\in \smooth.$$
Now if $\psi\circ f$ is a coordinate chart of $\mathcal{A}$, $\psi \circ f \circ f^{-1} = \psi$ is a homeomorphism on $N$ into $\R^n$. 
We can see that for any other coordinate chart $\psi_2$ on $N$, $(\psi\circ f)\circ(\psi_2 \circ f)^{-1}$ is $\smooth$, since they are charts on $\mathcal{A}$, but $$(\psi\circ f)\circ(\psi_2 \circ f)^{-1} = \psi \circ \psi_2^{-1}.$$ 
So $\psi$ is $\smooth$ related to any other chart $\psi$. 
\newline \\ $b \implies c$: Suppose $g \in \smooth$ on $(N, \mathcal{B})$. Then for any chart $\psi\in \mathcal{B}$, we have that $g\circ \psi^{-1} \in \smooth$. Then $$g\circ f \circ \varphi^{-1} = (g\circ \psi^{-1}) \circ (\psi \circ f \circ \varphi^{-1}).$$ By assumption $\psi \circ f$ is a chart on $(M , \mathcal{A})$, so its composition with $\varphi^{-1}$ is smooth. 
We have that $g\circ f \circ \varphi^{-1}$ is the composition of smooth functions hence smooth. Now suppose that $g\circ f$ is $\smooth$ on $(M, \mathcal{A})$. 
Therefore, $$g \circ f \circ \varphi^{-1} = (g \circ \psi^{-1}) \circ ( \psi \circ f \circ \varphi^{-1}). $$ We have that $( \psi \circ f \circ \varphi^{-1})$ is a diffeomorphism by assumption, so $g\circ \psi$ is $\smooth$.
\newline \\ $c \implies a$. Take $g = \psi \in \mathcal{B}$. We know that $\psi$ is $\smooth$ on $(N, \mathcal{B})$, so we must have that $\psi \circ f$ is smooth on $(M, \mathcal{A})$ i.e. $\psi \circ f \circ \varphi^{-1}$ is smooth. Since $f$ is a homeomorphism, we conclude that $f$ is a diffeomorphism.  
\end{document}