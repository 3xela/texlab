\documentclass[letterpaper]{article}
\usepackage[letterpaper,margin=1in,footskip=0.25in]{geometry}
\usepackage[utf8]{inputenc}
\usepackage{amsmath}
\usepackage{amsthm}
\usepackage{amssymb, pifont}
\usepackage{mathrsfs}
\usepackage{enumitem}
\usepackage{fancyhdr}
\usepackage{hyperref}

\pagestyle{fancy}
\fancyhf{}
\rhead{MAT 367}
\lhead{Assignment 2}
\rfoot{Page \thepage}

\setlength\parindent{24pt}
\renewcommand\qedsymbol{$\blacksquare$}

\DeclareMathOperator{\T}{\mathcal{T}}
\DeclareMathOperator{\V}{\mathcal{V}}
\DeclareMathOperator{\U}{\mathcal{U}}
\DeclareMathOperator{\Prt}{\mathbb{P}}
\DeclareMathOperator{\R}{\mathbb{R}}
\DeclareMathOperator{\N}{\mathbb{N}}
\DeclareMathOperator{\Z}{\mathbb{Z}}
\DeclareMathOperator{\Q}{\mathbb{Q}}
\DeclareMathOperator{\C}{\mathbb{C}}
\DeclareMathOperator{\ep}{\varepsilon}
\DeclareMathOperator{\identity}{\mathbf{0}}
\DeclareMathOperator{\card}{card}
\newcommand{\suchthat}{;\ifnum\currentgrouptype=16 \middle\fi|;}

\newtheorem{lemma}{Lemma}

\newcommand{\bd}{\partial}
\newcommand{\tr}{\mathrm{tr}}
\newcommand{\ra}{\rightarrow}
\newcommand{\lan}{\langle}
\newcommand{\ran}{\rangle}
\newcommand{\norm}[1]{\left\lVert#1\right\rVert}
\newcommand{\inn}[1]{\lan#1\ran}
\newcommand{\ol}{\overline}
\begin{document} 
\noindent Q1:
Let $M$ be a $k$ dimensonal submanifold of $\R^n$. Since $M$ is a subspace of $\R^n$, it inherits second countability and is hausdorff. 
Therefore it is sufficient to show that given two submanifold coordinate charts $(U_1, \varphi_1^{-1}), (U_2, \varphi_2^{-1})$ , the transition map $\varphi_2^{-1} \circ \varphi_1$ is a diffeomorphism. 
Define the maps $h_i: \R_x^k \times \R_y^{n-k} \to \R^n$ by $$h_i(x,y)  = \varphi_i(x) + (0,y).$$ Similarly as in A1, we have that $h_i$ is a diffeomorphism. Therefore the composition $$h_2^{-1} \circ h_1$$ is a diffeomorphism of $\R^n$. 
If $\iota_x$ is the inclusion of $\R_x^k$ into $\R_{x,y}^n$ into the first $k$ coordinates and $\pi_x$ is the projection onto the $x$ coordinates, the composition, 
$$ \pi_x \circ (h_2^{-1} \circ h_1) \circ  \iota_x  = \pi_x(h_2^{-1}\circ (\varphi_1(x), 0))= \pi_x(\varphi_2^{-1}(\varphi_1(x), 0)) = \varphi_2^{-1}\circ \varphi_1(x).$$
This is a diffeomorphism since it has rank $k$. fhu0dfhwofn
\end{document}