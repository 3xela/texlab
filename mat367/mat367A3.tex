\documentclass[12pt, a4paper]{article}
\usepackage[lmargin =0.5 in, 
rmargin=0.5in, 
tmargin=1in,
bmargin=0.5in]{geometry}
\geometry{letterpaper}
\usepackage{tikz-cd}
\usepackage{amsmath}
\usepackage{amssymb}
\usepackage{blindtext}
\usepackage{titlesec}
\usepackage{enumitem}
\usepackage{fancyhdr}
\usepackage{amsthm}
\usepackage{graphicx}
\usepackage{cool}
\usepackage{thmtools}
\usepackage{hyperref}
\graphicspath{ }					%path to an image

%-------- sexy font ------------%
%\usepackage{libertine}
%\usepackage{libertinust1math}

%\usepackage{mlmodern}				% very nice and classic
%\usepackage[utopia]{mathdesign}
%\usepackage[T1]{fontenc}


\usepackage{mlmodern}
\usepackage{eulervm}
%\usepackage{tgtermes} 				%times new roman
%-------- sexy font ------------%


% Problem Styles
%====================================================================%


\newtheorem{problem}{Problem}


\theoremstyle{definition}
\newtheorem{thm}{Theorem}
\newtheorem{lemma}{Lemma}
\newtheorem{prop}{Proposition}
\newtheorem{cor}{Corollary}
\newtheorem{fact}{Fact}
\newtheorem{defn}{Definition}
\newtheorem{example}{Example}
\newtheorem{question}{Question}

\newtheorem{manualprobleminner}{Problem}

\newenvironment{manualproblem}[1]{%
	\renewcommand\themanualprobleminner{#1}%
	\manualprobleminner
}{\endmanualprobleminner}

\newcommand{\penum}{ \begin{enumerate}[label=\bf(\alph*), leftmargin=0pt]}
	\newcommand{\epenum}{ \end{enumerate} }

% Math fonts shortcuts
%====================================================================%

\newcommand{\ring}{\mathcal{R}}
\newcommand{\N}{\mathbb{N}}                           % Natural numbers
\newcommand{\Z}{\mathbb{Z}}                           % Integers
\newcommand{\R}{\mathbb{R}}                           % Real numbers
\newcommand{\C}{\mathbb{C}}                           % Complex numbers
\newcommand{\F}{\mathbb{F}}                           % Arbitrary field
\newcommand{\Q}{\mathbb{Q}}                           % Arbitrary field
\newcommand{\PP}{\mathcal{P}}                         % Partition
\newcommand{\M}{\mathcal{M}}                         % Mathcal M
\newcommand{\eL}{\mathcal{L}}                         % Mathcal L
\newcommand{\T}{\mathcal{T}}                         % Mathcal T
\newcommand{\U}{\mathcal{U}}                         % Mathcal U\\
\newcommand{\V}{\mathcal{V}}                         % Mathcal V

% symbol shortcuts
%====================================================================%

\newcommand{\lam}{\lambda}
\newcommand{\imp}{\implies}
\newcommand{\all}{\forall}
\newcommand{\exs}{\exists}
\newcommand{\delt}{\delta}
\newcommand{\ep}{\varepsilon}
\newcommand{\ra}{\rightarrow}
\newcommand{\vph}{\varphi}

\newcommand{\ol}{\overline}
\newcommand{\f}{\frac}
\newcommand{\lf}{\lfrac}
\newcommand{\df}{\dfrac}

% bracketting shortcuts
%====================================================================%
\newcommand{\abs}[1]{\left| #1 \right|}
\newcommand{\babs}[1]{\Big|#1\Big|}
\newcommand{\bound}{\Big|}
\newcommand{\BB}[1]{\left(#1\right)}
\newcommand{\dd}{\mathrm{d}}
\newcommand{\artanh}{\mathrm{artanh}}
\newcommand{\Med}{\mathrm{Med}}
\newcommand{\Cov}{\mathrm{Cov}}
\newcommand{\Corr}{\mathrm{Corr}}
\newcommand{\tr}{\mathrm{tr}}
\newcommand{\Range}[1]{\mathrm{range}(#1)}
\newcommand{\Null}[1]{\mathrm{null}(#1)}
\newcommand{\lan}{\langle}
\newcommand{\ran}{\rangle}
\newcommand{\norm}[1]{\left\lVert#1\right\rVert}
\newcommand{\inn}[1]{\lan#1\ran}
\newcommand{\op}[1]{\operatorname{#1}}
\newcommand{\bmat}[1]{\begin{bmatrix}#1\end{bmatrix}}
\newcommand{\pmat}[1]{\begin{pmatrix}#1\end{pmatrix}}
\newcommand{\vmat}[1]{\begin{vmatrix}#1\end{vmatrix}}

\newcommand{\amogus}{{\bigcap}\kern-0.8em\raisebox{0.3ex}{$\subset$}}
\newcommand{\Note}{\textbf{Note: }}
\newcommand{\Aside}{{\bf Aside: }}
%restriction
%\newcommand{\op}[1]{\operatorname{#1}}
%\newcommand{\done}{$$\mathcal{QED}$$}

%====================================================================%


\setlength{\parindent}{0pt}      	% No paragraph indentations
\pagestyle{fancy}
\fancyhf{}							% fancy header

\setcounter{secnumdepth}{0}			% sections are numbered but numbers do not appear
\setcounter{tocdepth}{2} 			% no subsubsections in toc

%template
%====================================================================%
%\begin{manualproblem}{1}
%Spivak.
%\end{manualproblem}

%\begin{proof}[Solution]
%\end{proof}

%----------- or -----------%

%\begin{problem} 		
%\end{problem}	

%\penum
%	\item
%\epenum
%====================================================================%


\newcommand{\Course}{MAT367 }
\newcommand{\hwNumber}{3}

%preamble

\title{a}
\author{A.N.}
\date{\today}
\lhead{\Course A\hwNumber}
\rhead{\thepage}
%\cfoot{\thepage}


%====================================================================%
\begin{document}
	\begin{problem}
	\end{problem}
Endow $\R P^2$ with the smooth structure given by quotienting $S^2$ in the usual way. We compute the local derivative of $f$ as 
$$Df(\vph_1^{-1})(\vph([x,y,z]))  = D(zy, z\sqrt{1-y^2-z^2} , y \sqrt{1-y^2-z^2}) = \begin{bmatrix}
	z & y \\ \frac{-zy}{\sqrt{1-y^2-z^2}} & \frac{1-y^2-2z^2}{\sqrt{1-y^2-z^2}} \\ \frac{1-2y^2 - z^2}{\sqrt{1-y^2-z^2}} & \frac{-zy}{\sqrt{1-y^2-z^2}}.
\end{bmatrix}$$
The mapping $f$ will fail to be an immersion at points $(y,z)$ where the columns are linearly dependant i.e. their cross product is 0. For notation set $x = \sqrt{1-y^2-z^2}$. We aim to solve $$\Big(z^2+y^2 -x^2, \frac{y(x^2-y^2) + z^2y}{x} , \frac{z(x^2-z^2) + zy^2}{x} \Big) = 0.$$
The constraint on the first coordinate implies that $$y^2+z^2=\frac{1}{2}$$ and the second and third coordinate constraints imply that $y$ or $z$ is 0 but not both. Substituting back into $\vph^{-1}$ we get four points where $f$ fails to be an immersion in this chart, $$\Big[\frac{1}{\sqrt{2}}, \frac{1}{\sqrt{2}} ,0 \Big], \Big[\frac{1}{\sqrt{2}} , - \frac{1}{\sqrt{2}}, 0 \Big], \Big[\frac{1}{\sqrt{2}} , 0,  \frac{1}{\sqrt{2}} \Big], \Big[\frac{1}{\sqrt{2}} , 0 ,  - \frac{1}{\sqrt{2}} \Big]. $$
We repeat this process for $\vph_2$, seeing that $$Df(\vph_2^{-1})(\vph_2([x,y,z])) = \begin{bmatrix}
	\frac{-xz}{\sqrt{1-x^2-z^2}} & \frac{1-x^2-2z^2}{\sqrt{1-x^2-z^2}} \\ z & x \\ \frac{1-2x^2-z^2}{\sqrt{1-x^2-z^2}} & \frac{-xz}{\sqrt{1-x^2-z^2}} 
\end{bmatrix}.$$
A similar computation reveals that this matrix has rank less than $2$ at $$ \Big[ \frac{1}{\sqrt{2}}, \frac{1}{\sqrt{2}} , 0 \Big], \Big[\frac{-1}{\sqrt{2}}, \frac{1}{\sqrt{2}} , 0 \Big] , \Big[0 \frac{1}{\sqrt{2}} , \frac{1}{\sqrt{2}} \Big], \Big[ 0, \frac{1}{\sqrt{2}}, \frac{-1}{\sqrt{2}}\Big] . $$
Finally, for $\vph_3$ we have that $$ Df(\vph_3^{-1})(\vph([x,y,z])) = \begin{bmatrix}
	\frac{-xy}{\sqrt{1-x^2-y^2}} & \frac{1-x^2-2y^2}{\sqrt{1-x^2-y^2}} \\ \frac{1-2x^2-y^2}{\sqrt{1-x^2-y^2}} & \frac{-xy}{\sqrt{1-x^2-y^2}} \\ y & x
\end{bmatrix}. $$ This will fail to be an immersion at the points $$ \Big[0, \frac{1}{\sqrt{2}} , \frac{1}{\sqrt{2}} \Big], \Big[ 0 , \frac{-1}{\sqrt{2}}, \frac{1}{\sqrt{2}} \Big], \Big[ \frac{1}{\sqrt{2}}, 0 , \frac{1}{\sqrt{2}}\Big] , \Big[\frac{-1}{\sqrt{2}}, 0 , \frac{1}{\sqrt{2}} \Big]. $$ We conclude that the mapping $f$ will not be an immersion at $$\Big[0, \frac{1}{\sqrt{2}} , \frac{1}{\sqrt{2}} \Big], \Big[ 0 , \frac{-1}{\sqrt{2}}, \frac{1}{\sqrt{2}} \Big], \Big[ \frac{1}{\sqrt{2}}, 0 , \frac{1}{\sqrt{2}}\Big] , \Big[\frac{-1}{\sqrt{2}}, 0 , \frac{1}{\sqrt{2}} \Big] , \Big[\frac{1}{\sqrt{2}}, \frac{1}{\sqrt{2}} ,0 \Big], \Big[\frac{1}{\sqrt{2}} , - \frac{1}{\sqrt{2}}, 0 \Big].$$ The images of these points will be $$\Big(\pm \frac{1}{2}, 0,0 \Big), \Big(0 , \pm \frac{1}{2}, 0\Big), \Big(0,0, \pm \frac{1}{2} \Big). $$
\newpage
\begin{problem}
\end{problem}
\penum
\item Let $f$ be the map given. We show that the following hold to conclude that it is indeed an imbedding. 
\begin{enumerate}[label = \roman*)]
	\item $f$ is injective
	\item $f$ is an immersion
	\item $f$ is a homeomorphism onto its image.  
\end{enumerate}
First we show that $f$ is injective. If $$[y, 0] = [x, 0]$$
then $(y,0)$ and $(x,0)$ are either equal or andipodal. Clearly we must have that $x\sim y$ and $[x] = [y]$. Hence $f$ is injective. Now we claim that $f$ is an immersion. Endow $\R P^n$ and $\R P^{n+1}$ with the smooth structure induced by quotienting the sphere by antipodal points. Then if $\psi_j$ and $\vph_i$ are the standard charts on $\R P^{n+1}$ and $\R P^n$, for $j \neq i$ we have that $$(\psi_j \circ f \circ \vph_i^{-1})(x_1 , \dots , x_n) =(x_1,  \dots ,\hat{x}_j , \dots , \sqrt{1- x_1^2 - \dots x_n^2}, \dots , x_n, 0  ) ,$$ which will evaluate as $$D(\psi_j \circ f \circ \vph_i^{-1}) = \begin{bmatrix}
	1 & 0 & 0 & \cdots &0
	\\ 0 & 1 & 0 & \cdots& 0
	\\ \vdots & \vdots & \ddots & \vdots & \vdots 
	\\ \frac{-2x_1}{\sqrt{1- x_1^2 - \dots x_n^2}} & \cdots &\frac{-2x_{n-1}}{\sqrt{1- x_1^2 - \dots x_n^2}} & \frac{-2x_n}{\sqrt{1- x_1^2 - \dots x_n^2}}&0 
	\\ \vdots & \vdots & \vdots & \vdots & 0 
	\\ 0 & \cdots & \cdots & 1 & 0
\end{bmatrix}.$$ This will be of rank $n$. If $i=j$ then the Jacobian matrix will be $$D(\psi_j \circ f \circ \vph_j^{-1}) = \begin{bmatrix}
I & 0 \\ 0 & 0 
\end{bmatrix},$$ where $I$ is the $n\times n$ identity matrix. 
Therefore $f$ is an immersion. It remains to show that it is a homeomorphism onto its image. First, note that $f$ must be continuous since $f$ is smooth, and $\psi_j,\vph_i$ cover the manifold, the gluing lemma gives us the desired result. Furthermore since $f$ is injective, we have that $f$ is a bijection onto $f(\R P^n)$. Since $\R P^n$ is compact and hausdorff, it follows from topology that $f$ is a homeomorphism onto its image. Therefore $f$ defines an imbedding of $\R P^n$ in $\R P^{n+1}$.
\item We first check that the Segre imbedding is in fact an imbedding. First we show that it is an immersion. Regard $\C P^1$ and $\C P^3$ as quotients of complex spheres of same dimension. Let $\{\vph_i , U_i\}, \{\psi_j , V_j\}$ be atlases on $\C P^1, \C P^3$ with coordinates given by projection. We compute that $S$ looks like $$\psi_1 \circ S \circ (\vph_1^{-1}, \vph_1^{-1}) (z,w)  = (w\sqrt{1-z^2}, z\sqrt{1-w^2}, \sqrt{1-w^2}\sqrt{1-z^2}),$$ and the differential will be $$D(\psi_1 \circ S \circ (\vph_1^{-1}, \vph_1^{-1})) (z,w) = \begin{bmatrix}
	\frac{-zw}{\sqrt{1-z^2}} & \sqrt{1-z^2} \\ 
	\\
	\sqrt{1-w^2} & \frac{-zw}{\sqrt{1-w^2}} \\
	\\  
	\frac{-z\sqrt{1-w^2}}{\sqrt{1-z^2}} & \frac{-w\sqrt{1-z^2}}{\sqrt{1-w^2}}
\end{bmatrix}.$$
This will have a complex rank 2. A similar computation for different choices of $\psi_i, \vph_j$ will yield the same result and so we conclude that $S$ is an immersion. We now claim that $S$ is a homeomorphism onto its image. First we show that $S$ is injective. Suppose that $$S([z_0,z_1],[w_0,w_1]) = S([u_0,u_1],[v_0, v_1]). $$  This gives us that $$[z_0w_0, z_1w_0, z_0w_1, z_1w_1] = [u_0v_0, u_1v_0, u_0v_1, u_1v_1].$$ By the equivalence relation we have that \begin{align*}
	z_0w_0 & = \pm u_0v_0
	\\ z_1w_0 &= \pm u_1 v_0
	\\ z_0w_1 & =  \pm u_0v_1
	\\ z_1w_1 & = \pm u_1v_1
\end{align*} which implies that $[z_0,z_1] = [u_0,u_1]$ and $[w_0, w_1] = [v_0,v_1]$. Furthermore since $S$ is smooth, and since $\psi_j, \vph_i$ cover our manifolds, the gluing lemma implies that $S$ is continuous. Since $S$ is a bijection onto its image, and $\C P^1 \times \C P^1$ is compact and hausdorff, and $\C P^3$ is hausdorff we have that $S$ must be a homeomorphism onto its image. 
Define the generalized Segre imbedding as $S: [x_0, \dots , x_j] \times [y_0, \dots , y_k] \mapsto [x_iy_l ] $ where $x_iy_l$ is the vector given with entries ranging over all possible products of $x_i$ with $y_l$. We claim that $S$ is an imbedding. Let $\vph$ be a chart of $\C P^j$, $\psi $ be a chart of $C P^k$ and $\lambda$ be a chart of $\C P^{(j+1)(k+1) - 1}$. We have that $$\lambda \circ S \circ (\vph^{-1} , \psi^{-1})(z, y) = (z_0y_0, \dots z_0\sqrt{1-y_0^2 - \dots}, \dots \widehat{z_hy_k} , \dots z_jy_k),$$ and will have a differential of $$D(\lambda \circ S \circ (\vph^{-1}, \psi^{-1}))(z,y) = \begin{bmatrix}
	y_0 & 0 &  \dots & \dots & z_0 & 0 & \dots & \dots & 0 \\
	y_1 & 0 & \dots & \dots & 0 & z_0 & \dots & \dots & 0 \\ 
	\vdots & \vdots & \vdots & \vdots & \vdots & \vdots & \vdots & \vdots & \vdots \\
	0 & \cdots & \sqrt{(1-y_0^2 - \dots)} & \cdots &  \frac{-z_0 y_1}{\sqrt{1-y_0^2 - \dots }} & \cdots & \cdots & \cdots & \frac{-z_0y_k}{\sqrt{1-y_0^2 - \dots }} \\ \vdots & \vdots & \vdots & \vdots & \vdots & \vdots & \vdots & \vdots & \vdots \\
	0 & \cdots & \cdots & y_{n+1} & z_0 & 0 & \cdots & \cdots & 0 \\
	0 & y_0 & 0 & \cdots & z_1 & 0 & \cdots & 0 
\end{bmatrix}.$$ One can verify that this matrix has a complex rank of $j+k$. Hence $S$ is an immersion. By a similar argument as before, it is an imbedding.  
\epenum
\newpage
\begin{problem}
\end{problem}
Let $U$ be an open set around $B$ thats disjoint from $A$. We have that $U$ is a submanifold, and $A^c$ is an open covering of it. There exists a partition of unity $\{\psi_i\}$ subordinate to $A^c$. We have that $supp(\psi_i) \subset A^c$. Therefore the function $$f(p) = \begin{cases}

	\sum_{i} \psi_i(p) & p \not \in A
	\\ 0 & p \in A
\end{cases}$$
will be smooth and satisfies our requirements. 
\newpage
\begin{problem}
\end{problem}
Let $C$ be a closed subset of $\R^n$. Cover $\R^n \setminus C$ with a countable covering of open balls $\{B_n\}$. Each ball $B_i$ contains some compact set $C_i$, and we can take some smooth functions $f_i$ so that $f_i|_{C_i} = 1$ and $f_i=0$ outside of $B_i$.  Define $$f = \sum_{i} \frac{f_i}{2^i M_i}$$ where $M_i$ is the supremum of the absolute value of all mixed partials of orders less than or equal to $i$, of $f_i$. We claim that $f$ as defined is $0$ exactly on $C$. Notice that if $x\in C$, then each $f_i$ is 0 so $f(x)=0$. If $x \in C^c$, then it belongs to some $B_i$ and so $f(x) \geq \frac{f_i(x)}{2^i M_i} >0$. It remains to show that $f$ is smooth. By comparison test, we have that $$\sum_{i} \frac{f_i}{2^i M_i}$$ is an absolutely convergent series, hence $f$ is differentiable since each $f_i$ is. We claim that $f$ is smooth. Note that if we apply any mixed order partial derivative operator, we get that
 $$ \Big|\frac{\partial^\alpha}{\partial x^\alpha}f \Big| = \Big|\sum_i \frac{1}{2^i M_i}\frac{\partial^\alpha}{\partial x^\alpha} f_i \Big| \leq M_i \sum_{i}\frac{1}{2^i}.$$
 Therefore all mixed partials of $f$ exist hence $f$ is smooth. Now suppose that $C$ is a closed subset of a manifold. By the Whitney Imbedding theorem, there exists an imbedding $\psi : M \to \R^M$ for sufficiently large $M$. Since $\psi$ is a homeomorphism onto its image, we have that 
 $\psi(C)$ is a closed subset of $\R^M$. Choose $f$ as per above defined on $\psi(M)$ so that $f^{-1}\{(0)\}= \psi(C)$. Then the smooth function $f\circ \psi : M\to \R$ will suffice. 
 \newpage
 \begin{problem}
 \end{problem}
\penum
\item Let $X_0 \in M(m,n;k).$ Let $v_{i_1},  \dots ,  v_{i_k}$ be the $k$ linearly independant columns. Choose a column permutation matrix $Q$ that sends $v_{i_1} \dots v_{i_k}$ to the first $k$ columns. Now let $u_{j_1}, \dots , u_{j_k}$ be the $k$ linearly independant rows of $X_0Q$. Take $P$ to be a permutation matrix which sends $u_{j_1}, \dots , u_{j_k}$ to the first $k$ rows. Our matrix $PX_0Q$ will be of the form $$PX_0Q = \bmat{ A & B \\ C & D}	$$ where $A$ is an invertible $k$ by $k$ matrix. 
\item Since $\det(A)$ is a smooth polynomial in the entries of $A$, if $det(A) \neq 0$ we can find a sufficiently small $\ep$ so that $det(A_0) \neq 0$ when the entries of $A - A_0$ are less than $\ep$. 
\item Suppose that $$Y = \bmat{A & B \\ C& D}$$ for $A$ $k\times k$ and nonsingular. Suppose that $Y$ is rank $k$. Then for some matrix $$X = \bmat{I & 0 \\ Z & 0},$$ we have that $$XY = \bmat{A & B \\ 0 & 0 }.$$ Computing the matrix multiplication, we see that $$ XY = \bmat{A & B \\ 0 & 0 } = \bmat{A & B \\ ZA+C & ZB+D}.$$ This implies that $Z = -CA^{-1},$ and so $-CA^{-1}B+D=0$ as desired. Now suppose that $D = CA^{-1}B$. Then we have that $$\bmat{I& 0 \\ -CA^{-1} & I} \bmat{A & B \\ C & CA^{-1}B } = \bmat{A & B \\ 0 & 0}.$$ Since $A$ has rank $k$, $Y$ must as well. 
\item Define the map $f:\R^{nm} \to \R^{(m-k)(n-k)}$ by $$f \left( \bmat{A & B \\ C& D} \right) =D - CA^{-1}B.$$
We can always take a matrix of rank $k$ to be in this form, and in some neighbourhood of $A$ this matrix will be of the same form by $a,b,c$. Evidently by $c)$ this will vanish exactly when $\bmat{A & B \\ C& D}$ has rank $k$. The zero set of this function will be a neighbourhood of $M(n,m;k)$, and $f$ will have full rank since it consists of linear terms. Therefore the dimension of this manifold will be $$nm - (m-k)(n-k) = k(m+n-k)$$
\epenum
\newpage
\begin{problem}
\end{problem}
\penum
\item Matrix multiplication is an algebraic operation, hence smooth. Similarly, the inverse of a matrix is a polynomial in its entries, so it is smooth as well. Since $GL_n(\R) = \det^{-1}( \R \setminus \{0\} )$ we have that $GL_n(\R)$ is an open subset of $\R^{n^2}$ and hence has dimension of $n^2$. 
\item $O(n)$ is the set of all matrices satisfying $A^\perp = A^{-1}$ or equivalently $A^\perp A = I$. Notice that this is a lie subgroup of $GL_n(\R)$. Consider the mapping $f: GL_n(\R) \to Sym_n(\R)$ defined by $$f(A ) = A^\perp A.$$ We claim that $I$ is a regular value of $f$, which would imply that $O(n)$ is a manifold since $f^{-1}(I) = O(n)$. By the computation done in tutorial, we have that $Df_A(X) = A^\perp X + X^\perp A. $ We claim that this is surjective for $A\in f^{-1}(I)$. Let $Y\in Sym_n(\R)$. Then taking $X = \frac{1}{2} AY$ will solve the equation. So $I$ is a regular value and so $O(n)$ is a manifold of dimension $$dim(GL_n(\R)) - dim( Sym_n(\R)) = n^2 - \frac{1}{2}n(n+1) = \frac{1}{2}n(n-1)$$
\epenum
\newpage
\begin{problem}
\end{problem}
\penum
\item Consider the mapping $f: \R \to S^1$ defined by $$x\mapsto e^{2\pi i x}.$$ We have that $f^\prime  = 2\pi i f(x).$ This is nonzero so $f$ is a submersion. Clearly this is not a diffeomorphism since it is periodic on $\R$, yet $S^1$ and $\R$ are both $1-$manifolds. 
\item Let $a\in M$. Taking a suitable chart $(\vph, U)$ around $a$, and a chart $(\psi, V)$ around $f(a)$  consider the following commutative diagram: 
$$\begin{tikzcd}
	{TM_a} && {TN_{f(a)}} \\
	{T\R^n_{\vph(a)}} && {T\R^m_{\psi(f(a))}}
	\arrow["{f_{\ast a}}"', from=1-1, to=1-3]
	\arrow["{\vph_{\ast a}}"', from=1-1, to=2-1]
	\arrow["{\psi_{\ast f(a)}}", from=1-3, to=2-3]
	\arrow["{\psi_{\ast f(a)} \circ f_{\ast a} \circ \vph^{-1}_{\ast \vph(a)}}"', from=2-1, to=2-3]
\end{tikzcd}$$ This diagram commutes, and since $\psi_{\ast a}$ and $\vph_{\ast a}$ are isomorphisms, we have that $n=m$. Furthermore, We have that the mapping $$\psi_{\ast f(a)} \circ f_{\ast a} \circ \vph^{-1}_{\ast \vph(a)} = (\psi \circ f \circ \vph^{-1})_{\ast a}$$ is an isomorphism. So $(\psi \circ f \circ \vph^{-1})$ is a diffeomorphism by the inverse function theorem. So $f$ must be a diffeomorphism. 
\item First note that $f$ is injective and continuous. Hence it is an open mapping. Therefore $f(M)$ is open in $N$. Since $M$ is compact then so is $f(M)$. Therefore $f(M)$ is closed and open and nonempty. So $f(M) = N$. We have that $f:M \to N$ is a bijection. By $b)$ $f$ must be a diffeomorphism. 
\epenum 
\newpage
\begin{problem}
\end{problem}
\penum
\item For $X\in M(n, \R)$, let $X_\C = X\otimes_{\C}1$ be the complexification of the matrix $X$. We have that by linear algebra, $$\det(I+tX_{\C}) = t^{n} \det(t^{-1}I - (-X_{\C})) = t^n \left(t^{-n} + (Tr(X_{\C})) t^{-n+1} + \dots	\right)= 1+ (Tr(X_{\C}))t + \dots.$$
This is a polynomial in $t$, so differentiating at $t=0$ gives us that $$\frac{d}{dt} \det(I + tX_{\C}) = tr(X_{\C}).$$ Since $tr(X) = tr(X_{\C})$ we obtain the desired result. 
\item We have that $$f(A+tX) = \det(A+tX) = \det(A)det(I+A^{-1}X).$$
By part $a)$ we have that $Df(A)X = det(A)tr(A^{-1}X)$. This is a linear map in $X$. Thus we are done.
\item We claim that $f$ is a submersion. It is sufficient to show that $Df(A)X$ is a surjective mapping onto $\R$. Given $A\in GL_n(\R)$ and $c\in \R$ we wish to find an $X$ so that $$det(A) tr(A^{-1}X) = c.$$ Taking $X = \frac{c}{n\det(A)}A$ gives us $$\det(A) tr \left(A^{-1} \frac{c}{n\det(A)}A \right) =\frac{c}{n} tr(I) = c.$$ Therefore $f$ is a submersion. 
\item By results from A1Q2, we have that the tangent space to $I$ is given by the kernel of $Df(I).$ So $$X \in TM_{I} \iff Df(I)X = 0 \iff \det(I)tr(I^{-1}X)=0 \iff tr(X)=0.$$
\epenum
\end{document}