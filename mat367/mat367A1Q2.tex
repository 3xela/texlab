\documentclass[letterpaper]{article}
\usepackage[letterpaper,margin=1in,footskip=0.25in]{geometry}
\usepackage[utf8]{inputenc}
\usepackage{amsmath}
\usepackage{amsthm}
\usepackage{amssymb, pifont}
\usepackage{mathrsfs}
\usepackage{enumitem}
\usepackage{fancyhdr}
\usepackage{hyperref}

\pagestyle{fancy}
\fancyhf{}
\rhead{MAT 367}
\lhead{Assignment 1}
\rfoot{Page \thepage}

\setlength\parindent{24pt}
\renewcommand\qedsymbol{$\blacksquare$}

\DeclareMathOperator{\T}{\mathcal{T}}
\DeclareMathOperator{\V}{\mathcal{V}}
\DeclareMathOperator{\U}{\mathcal{U}}
\DeclareMathOperator{\Prt}{\mathbb{P}}
\DeclareMathOperator{\R}{\mathbb{R}}
\DeclareMathOperator{\N}{\mathbb{N}}
\DeclareMathOperator{\Z}{\mathbb{Z}}
\DeclareMathOperator{\Q}{\mathbb{Q}}
\DeclareMathOperator{\C}{\mathbb{C}}
\DeclareMathOperator{\ep}{\varepsilon}
\DeclareMathOperator{\identity}{\mathbf{0}}
\DeclareMathOperator{\card}{card}
\newcommand{\suchthat}{;\ifnum\currentgrouptype=16 \middle\fi|;}

\newtheorem{lemma}{Lemma}

\newcommand{\bd}{\partial}
\newcommand{\tr}{\mathrm{tr}}
\newcommand{\ra}{\rightarrow}
\newcommand{\lan}{\langle}
\newcommand{\ran}{\rangle}
\newcommand{\norm}[1]{\left\lVert#1\right\rVert}
\newcommand{\inn}[1]{\lan#1\ran}
\newcommand{\ol}{\overline}
\begin{document} 
\noindent
\begin{description}
\item{$(d) \subset TM_a$} \newline 
First we show that $(d) \subset TM_a$. Let $v\in D\varphi(b) ( \R_b^k).$ There therefore must be some $u\in \R^k$ so that $D \varphi(b)\cdot u = v$. 
Consider the mapping $\gamma(t)=\varphi(b+ tu)$ for $t\in (-\delta, \delta)$ for sufficiently small $\delta$ so that the image of the interval is contained in the domain of $\varphi$. 
We have that $\gamma(0) = \varphi(b+0\cdot u) = a$, and $$\gamma^\prime(t) = \varphi^\prime(b+tu) = \varphi^\prime(b+tu) \cdot u.$$ Evaluating at $t=0$, 
get that $$\gamma^\prime(0) = \varphi^\prime(b)\cdot u = v.$$ Therefore $v\in TM_a$. 
\item{$TM_a \subset (c)$} \newline
Let $v\in TM_a$. Let $\gamma(t)$ be such that $\gamma(0) = a$ and $\gamma^\prime(0) = v$. 
Define $w = h^\prime(a)\cdot v$, since $h$ is a diffeomorphism it makes sense to say that $v = h^\prime(a)^{-1}\cdot w$. Take $$\gamma_2(t) = h\circ(\gamma) = (u_1(t), \dots , u_k(t), 0 ,\dots ,0). $$ 
Note that we have that $\gamma_2(0) =h(a) $ and $\gamma_2^\prime(0) =w$. Note that $w\in (\R^k\times \{0\})_{h(a)}$. Since $h^\prime(a)^{-1}w=v$ we can conclude that $TM_a\subset (c)$. 
\item{$(c) \subset (d)$}\newline
Next we show that $(c) \subset (d)$. We have shown previously that $\varphi = h^{-1} \circ \iota_k$. Let $v \in Dh(a)^{-1}((\R^k\times \{0\})_{h(a)}).$ There must be a $w = (w_1, \dots , w_k , 0, \dots ,0).$ such $h^{-1}(a)w = v$. 
Therefore $$D\varphi = h^\prime(\iota(a)) \cdot \iota = (h\circ \iota)^{-1}\circ \iota_k.$$ 
We therefore can write $v = D\varphi (w_1, \dots , w_k)$. This is exactly what we wanted to show. 
\item{$TM_a \subset (b)$} \newline 
Next we show that $(TM_a) \subset (b)$. Let $\gamma(t)$, $a\in M$, $v\in TM_a$ be given in the usual way. Let $(y,g(y))$ be the graph of some function $g$ which agrees with $M$ on some open neighbourhood of $a$. 
Therefore at some point $y_0\in \R^k$ we have that $(b,c)=a =(y_0,g(y_0)).$ It follows that if we define $\gamma_k = \pi_{k}\circ \gamma$, then $F(t) = (\gamma_k(t), g(\gamma_k(t)))\subset M$ and $(\gamma_k(0), g(\gamma_k(0))) = a$.
Taking the differential of $F$ we have that $$F^\prime(t) = [\gamma_k^\prime (t) | g^\prime(\gamma_k(t))\cdot \gamma^\prime(t)].$$ Evaluating at $0$, and only looking at the block with $g^\prime$, we have that $g^\prime(\gamma_k(0))\cdot \gamma_k^\prime(0)= g^\prime(b)\cdot v_k$. 
Therefore we can write $v = (v_k,g^\prime(b)\cdot v_k)$. This is the desired result. 
\item{$(b) \subset (a)$} \newline 
We now claim that $(b) \subset (a)$. Let $(y,Dg(b)\cdot y)$ belong to $(b)$. Construct the function $F(y,z) = g(y) - z$. 
Similarly to Q1, we have that $F$ is $0$ exactly on the subset of $M$ that is locally the graph of $g$. We compute that $$DF(b,c) = \Big[\frac{\partial g}{\partial y}(b) | -I \Big].$$ A vector in $(b)$ can be written in the form $(v, Dg(b)\cdot v).$ We verify that $$Df(b,c) \cdot (v, Dg(b)\cdot v) = \Big[\frac{\partial g}{\partial y}(b) | -I \Big]\cdot \begin{pmatrix}
    v \\ Dg(b)\cdot v
\end{pmatrix} = \frac{\partial g}{\partial y}(b) \cdot v - Dg(b)\cdot v = 0.$$
Thus we are done. 
\item{$TM_a \subset (a)$} \newline 
We now show that $TM_a \subset (a)$. Let $v\in TM_a$ with corresponding curve $\gamma$ defined on $(-\delta,\delta)$ with $\delta$ chosen sufficiently small so that $f(\gamma((-\delta, \delta))) = 0$. 
Then since $f(\gamma(t))$ is constant we have that $$Df(\gamma(t)) \cdot \gamma^\prime(t) = 0 \implies Df(\gamma(0))\cdot \gamma^\prime(0)=0 \implies Df(a)\cdot v = 0,$$ as desired. 
\item{$(a) \subset (b)$} \newline 
Finally we show that $(a) \subset(b). $ Take $f$ as in the definition in Q1, and $v\in \ker Df(a).$
Permute coordinates of $\R^n$ so that $\frac{\partial f}{\partial y}$ has rank $n-k$. 
The implicit function guarantees the excistence of some $g(x)$ which implicitly solves $f(x,y)=0$. We have that $$Df(x,g(x)) = Df(x,g(x))\cdot [I|Dg(x)].$$ Evaluating at the point $a=(b,c)$, we have that $$Df(a) = Df(b,g(b))\cdot [I|Dg(b)].$$ Since $v \in \ker Df(a),$ we must have that 
$[I|Dg(b)]\cdot v \in \ker Df(a).$ This is the same as saying as $$(\pi_k v, Dg(b) \cdot \pi_k v) \in \ker Df(a).$$ This is exactly what we wanted to show. 
We now claim that the containments do not depend on our choice or construction of each function. We can just repeat the proccess of the proof with any other given function that satisfies the same hypothesis.
Our end result is that we have equality with $TM_a$, and so $TM_a$ is independant of function chosen. 

\end{description}
\end{document}