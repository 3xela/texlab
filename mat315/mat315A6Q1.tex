\documentclass[letterpaper]{article}
\usepackage[letterpaper,margin=1in,footskip=0.25in]{geometry}
\usepackage[utf8]{inputenc}
\usepackage{amsmath}
\usepackage{amsthm}
\usepackage{amssymb, pifont}
\usepackage{mathrsfs}
\usepackage{enumitem}
\usepackage{fancyhdr}
\usepackage{hyperref}

\pagestyle{fancy}
\fancyhf{}
\rhead{MAT 315}
\lhead{Assignment 6}
\rfoot{Page \thepage}

\setlength\parindent{24pt}
\renewcommand\qedsymbol{$\blacksquare$}

\DeclareMathOperator{\F}{\mathbb{F}}
\DeclareMathOperator{\T}{\mathcal{T}}
\DeclareMathOperator{\V}{\mathcal{V}}
\DeclareMathOperator{\U}{\mathcal{U}}
\DeclareMathOperator{\Prt}{\mathbb{P}}
\DeclareMathOperator{\R}{\mathbb{R}}
\DeclareMathOperator{\N}{\mathbb{N}}
\DeclareMathOperator{\Z}{\mathbb{Z}}
\DeclareMathOperator{\Q}{\mathbb{Q}}
\DeclareMathOperator{\C}{\mathbb{C}}
\DeclareMathOperator{\ep}{\varepsilon}
\DeclareMathOperator{\identity}{\mathbf{0}}
\DeclareMathOperator{\card}{card}
\newcommand{\suchthat}{;\ifnum\currentgrouptype=16 \middle\fi|;}

\newtheorem{lemma}{Lemma}

\newcommand{\Mod}[1]{\ \mathrm{mod}\ (#1)}
\newcommand{\tr}{\mathrm{tr}}
\newcommand{\ra}{\rightarrow}
\newcommand{\lan}{\langle}
\newcommand{\ran}{\rangle}
\newcommand{\norm}[1]{\left\lVert#1\right\rVert}
\newcommand{\inn}[1]{\lan#1\ran}
\newcommand{\ol}{\overline}
\begin{document}
\noindent Q1a: Note since $$(1-x)(1+x+ \dots + x^{q-1}) = (1-x^q)$$ Any root of the cyclotomic polynomial is also root of $1-x^q$. We see that $\Phi_q(0)\equiv 1 \text{mod (p)}$. Therefore by FlT the roots of $\Phi_q(x)$ must satisfy $x^{p-1}= 1$. We now check cases when $p\equiv 1 \text{ mod (q)}$, $p=q$ or if neither are true. 
First consider the case where $p\equiv 1 \text{ mod (1)}$, then there is some $k$ such that $kq = p-1$
\begin{align*}
x^{p-1}-1 & = x^{kq-1}-1
\\ & = (x^q)^k-1
\\ & = (x^q -1)(x^{kq-q} + x^{kq-2q} + \dots +1)
\\ & = (x^{q}-1)(x^{p-1-q} + \dots + 1)
\end{align*} Hence by Lagranges theorem the first term must have at most $q$ roots and the second term must have at most $p-1-q$ roots. By Fermats little theorem, $x^{p-1}-1$ has $p-1$ roots mod $p$, so therefore $x^q-1$ has $q$ roots and $x^{p-1-q} + \dots + 1$ has $p-1-q$ roots. Now since $(x-1)\Phi_q(x)$ has $p$ roots, and $\Phi_q(1) \equiv q \text{ mod(p)}$ which is not 0, so we must have that $\Phi_q(x)$ has $p-1$ roots. If $ p\not\equiv 1 \text{ mod (q)}$, so $\gcd(q,p-1)=1$. Thus by bezouts identity, there exists integers $u,v$ with $u(p-1) + v(q)=1$. Therefore, if $x$ is a root of $\Phi_q(x)$, then $x^1 = x^{u(p-1) + v(q)} = (x^{p-1})^u \cdot (x^{q})^v $. Now if $p=q$ then $\Phi_q(1) = 0$ mod $(p)$ so it has 1 root. Otherwise, $\Phi_q(1)\neq 0$ so $\Phi_q(x)$ has no roots.
\newline \\ Q1b: By the chinese remainder theorem, any solution to $x^{18} + 4x^{14}+3x+10 \equiv 0 \Mod{21}$ must also be a solution to $x^{18} + 4x^{14}+3x+10 \equiv 0 \Mod{3}$ and $x^{18} + 4x^{14}+3x+10 \equiv 0 \Mod{7}$. By corollary 4.4, $x^{3k}\equiv x \Mod{3}$ for all $k\in \Z$. Therefore we can reduce our polynomials to $$x^{18} + 4x^{14}+3x+10 \equiv (x^{3\cdot 3})^2 + x^2+1 \equiv (2x^2-1)\equiv 1-x^2 \equiv(1-x)(1+x)  \Mod{3}$$ which has a solution of $x\equiv 1 \Mod{3}$ and $x\equiv 2 \Mod{3}$. Now for the $\Mod{7}$ polynomial, 
$$x^{18} + 4x^{14}+3x+10 \equiv (x^{7})^2\cdot x^4 + 4(x^7)^2 + 3x + 10 \equiv x^6+4x^2 +3x+3 \equiv 0 \Mod{7}$$
By checking $x\in \{0,1,2,3,4,5,6\}$, we see that $x\equiv 3 \Mod{7}$ and $x\equiv 5 \Mod{7}$ are both solutions. 
By chinese remainder theorem, the solutions are $x\equiv 5,10,17,19 \Mod{21}$. 
\newline \\ Q1c: We wish to solve $x^{59}+ 2x^{40}+5x^{25}+x^{15}+17\equiv 0 \Mod{221} $. By the chinese remainder theorem, any solution to this will also be a solution to $x^{59}+ 2x^{40}+5x^{25}+x^{15}+17\equiv 0\Mod{13}$ and $x^{59}+ 2x^{40}+5x^{25}+x^{15}+17\equiv 0 \Mod{17}$. We will first proceed with the first equivalency. Using corollary 4.4, we have that 
$$x^{59}+ 2x^{40}+5x^{25}+x^{15}+17\equiv (x^{13})^4\cdot x^7 + 2(x^{13})^3 \cdot x + 5(x^{13})x^{12} + x^{13}\cdot x^2 + 17 \equiv x^{11} + 2x^4+x^3+5x+4\equiv 0 \Mod{13}$$
Taking $x\equiv 1\Mod{13}$ will satisfy this. For$\Mod{17}$, we have that 
$$x^{59}+ 2x^{40}+5x^{25}+x^{15}+17 \equiv x^{59}+ 2x^{40}+5x^{25}+x^{15}\equiv 0 \Mod{17}$$
We see taking $x\equiv 0 \Mod{17}$ will satisfy. Therefore by the chinese remainder theorem, a solution to the polynomial$\Mod{221}$ is $x\equiv 170 \Mod{221}$. 
\newline \\ Q1d: To find a solution to $55x^{19} + 3x^{14} + x^2 + 55\equiv 0 \Mod{66}$. By the Chinese remainder theorem, any solution to this must be a simultanous solution to $55x^{19} + 3x^{14} + x^2 + 55\equiv 0\Mod{2}$, $55x^{19} + 3x^{14} + x^2 + 55\equiv 0\Mod{3}$, $55x^{19} + 3x^{14} + x^2 + 55\equiv 0\Mod{11}$. We will first look for \newline solutions in the mod(2) case. We see that 
$$55x^{19} + 3x^{14} + x^2 + 55\equiv x^{19}+ x^{14} + x^2 +1 \equiv 0 \Mod{2}$$ 
The only solution is $x\equiv 1 \Mod{2}$ is a solution. Now we look at the$\Mod{3}$ case. 
$$55x^{19} + 3x^{14} + x^2 + 55\equiv x^{19} + x^2 + 1 \equiv x^2+x+1\equiv 0 \Mod{3}$$
We see that the only solution is $x\equiv 1 \Mod{3}$. Finally, we check$\Mod{11}$. 
$$55x^{19} + 3x^{14} + x^2 + 55\equiv 3x^{11}x^3 + x^2 \equiv 3x^4 + x^2 \equiv x^2(3x^2+1)\equiv 0 \Mod{11}$$ By checking each $x\in \Z / 11 \Z$ we see that $3x^2+1\not \equiv 0$ for all $x$. Thus we conclude that the only solution mod 11 is $x\equiv 0 \Mod{11}$. 
Therefore, by the chinese remainder theorem, the solution will be $x\equiv 55 \Mod{66}$. 
\end{document}