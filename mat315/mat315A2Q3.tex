\documentclass[letterpaper]{article}
\usepackage[letterpaper,margin=1in,footskip=0.25in]{geometry}
\usepackage[utf8]{inputenc}
\usepackage{amsmath}
\usepackage{amsthm}
\usepackage{amssymb, pifont}
\usepackage{mathrsfs}
\usepackage{enumitem}
\usepackage{fancyhdr}
\usepackage{hyperref}

\pagestyle{fancy}
\fancyhf{}
\rhead{MAT 315}
\lhead{Assignment 2}
\rfoot{Page \thepage}

\setlength\parindent{24pt}
\renewcommand\qedsymbol{$\blacksquare$}

\DeclareMathOperator{\U}{\mathcal{U}}
\DeclareMathOperator{\Prt}{\mathbb{P}}
\DeclareMathOperator{\R}{\mathbb{R}}
\DeclareMathOperator{\N}{\mathbb{N}}
\DeclareMathOperator{\Z}{\mathbb{Z}}
\DeclareMathOperator{\Q}{\mathbb{Q}}
\DeclareMathOperator{\C}{\mathbb{C}}
\DeclareMathOperator{\ep}{\varepsilon}
\DeclareMathOperator{\identity}{\mathbf{0}}
\DeclareMathOperator{\card}{card}
\newcommand{\suchthat}{;\ifnum\currentgrouptype=16 \middle\fi|;}

\newtheorem{lemma}{Lemma}

\newcommand{\tr}{\mathrm{tr}}
\newcommand{\ra}{\rightarrow}
\newcommand{\lan}{\langle}
\newcommand{\ran}{\rangle}
\newcommand{\norm}[1]{\left\lVert#1\right\rVert}
\newcommand{\inn}[1]{\lan#1\ran}
\newcommand{\ol}{\overline}
\begin{document}
\noindent 
Q3a: Since $\frac{ab}{\gcd(a,b)}=lcm(a,b)$, we know from Q2a that $gcd(1485,1745)=5$. Hence $lcm(1485,1745) = \frac{(1485)(1745)}{5} = 518265$
\newline \\ Q3b: We begin by proving the reverse implication. Suppose that $c$ is a multiple of the $lcm(a,c)=l$. Then for some $k\in \Z$, $c = k\cdot l$. By the definition of the $lcm(a,b)$, we have $a\cdot q_1 = l$ and $b\cdot q_2 = l$ for some $q_1,q_2\in \Z$. Substituting in our equalities we see that $c = k \cdot q_1 \cdot a = k\cdot q_2 \cdot b$ and so $c$ is a common multiple of $a$ and $b$. 
Now suppose that $c$ is a common multiple of $a,b$. Then for some $q_1,q_2$, $c=q_1 a = q_2 b$ We see that $$ \frac{c}{lcm(a,b)} = \frac{c}{\frac{ab}{\gcd(a,b)}} = \frac{c(av+bu)}{ab} = \frac{c(au+bv)}{\frac{c}{q_1}\cdot \frac{c}{q_2}} = \frac{q_1q_2(au+bv)}{c} = \frac{q_1q_2 a u+ q_1q_2bv}{c} = q_2u+q_1v$$
This is an integer so we conclude $c$ is a multiple of $lcm(a,b)$. 
\newline \\ Q3c: This statement is not necessarily true. Consider $a=b=c=2$. Then $\gcd(2,2,2)=2$ and $lcm(2,2,2)=2$, but $lcm(2,2,2)\cdot \gcd(2,2,2) \neq 2\cdot 2\cdot 2$
\newline \\ Q3d: We claim that $\frac{abc}{\gcd(a,b,c)} = lcm(a,b,c)$ if and only iff $a,b,c$ are pairwise co-prime. 
We first need the following lemma: 
\newline Lemma: Suppose that we have $a_1\dots a_k$ each with a prime factorization $a_j = \Pi_{i=1}^n p_i^{e^j_i}$. Then $gcd(a_1,\dots a_k) = \Pi_{i=1}^n p_i^{min(e^1_i,\dots e^n_i)}$ and $lcm(a_1\dots a_k) = \Pi_{i=1}^n p_i^{max(e^1_i,\dots e^n_i)}$
\newline pf: Clearly, by our choice of gcd, $gcd(a_1\dots a_k)|a_1\dots a_k$ by construction. It must also be the minimum divisor, since any smaller choice must divide the quantity $\Pi_{i=1}^n p_i^{min(e^1_i,\dots e^n_i)}$ but it would fail to divide at least one $a_i$. Similarly, for lcm, our choice for lcm will be divisible by each $a_i$, but if there was a lower number its prime factorization would not include all the prime factors of each $a_i$. $\qed$
\newline \\ We proceed with the forward implication. 
It suffices to check that if $a,b,c$ are not coprime, then the equality will not hold. We can write $a,b,c$ in the following way: $$a=\Pi_{i=1}^k p_i^{e_i}, b=\Pi_{i=1}^k p_i^{e_i}, c=\Pi_{i=1}^k p_i^{l_i}$$ for primes $p_i$ and $e_i,f_i,l_i\in \N \cup \{0\}$. We compute that $$\frac{abc}{\gcd(a,b,c)} = \frac{\Pi_{i=1}^k p_i^{e_i+f_i+l_i}}{\Pi_{i=1}^k p_i^{min(e_i,f_i,l_i)}} = \Pi_i^{k} p_i^{e_i+f_i+l_i-min(e_i,f_i,l_i)}$$
By lemma, $lcm(a,c,b) = \Pi_{i=1}^k p_i^{max(e_i,f_i,l_i)}$. By assumption, $a,b,c$ are not coprime, so for some $i\in{1,\dots k}$, we have that at least two of $e_i,f_i,l_i$ are strictly more than zero. We now notice that the equality $$e_i+f_i+l_i-min(e_i,f_i,l_i) = max(e_i,f_i,l_i)$$ will never hold, for some $i$, since on the left side we add 3 positive numbers and subtract off one of them, and on the right side we have only one positive number, which is included in the sum on the left. 
Therefore, the exponents on some $p_i$ in the expansion of $\frac{abc}{\gcd(a,b,c)}$ will be different than in the expansion of $lcm(a,b,c)$ and so they will be different. We now prove the reverse implication. Suppose that $a,b,c$ are co-prime. Then we evaluate $$\frac{abc}{\gcd(a,b,c)} = \frac{abc}{\gcd(\gcd(a,b),c)} = \frac{abc}{\gcd(a,c)} = abc = lcm(a,b,c)$$ where the equality $lcm(a,b,c)=abc$ holds my the lemma applied to comprime numbers $a,b,c$.
\end{document}