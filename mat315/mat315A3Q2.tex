\documentclass[letterpaper]{article}
\usepackage[letterpaper,margin=1in,footskip=0.25in]{geometry}
\usepackage[utf8]{inputenc}
\usepackage{amsmath}
\usepackage{amsthm}
\usepackage{amssymb, pifont}
\usepackage{mathrsfs}
\usepackage{enumitem}
\usepackage{fancyhdr}
\usepackage{hyperref}

\pagestyle{fancy}
\fancyhf{}
\rhead{MAT 315}
\lhead{Assignment 3}
\rfoot{Page \thepage}

\setlength\parindent{24pt}
\renewcommand\qedsymbol{$\blacksquare$}

\DeclareMathOperator{\U}{\mathcal{U}}
\DeclareMathOperator{\Prt}{\mathbb{P}}
\DeclareMathOperator{\R}{\mathbb{R}}
\DeclareMathOperator{\N}{\mathbb{N}}
\DeclareMathOperator{\Z}{\mathbb{Z}}
\DeclareMathOperator{\Q}{\mathbb{Q}}
\DeclareMathOperator{\C}{\mathbb{C}}
\DeclareMathOperator{\ep}{\varepsilon}
\DeclareMathOperator{\identity}{\mathbf{0}}
\DeclareMathOperator{\card}{card}
\newcommand{\suchthat}{;\ifnum\currentgrouptype=16 \middle\fi|;}

\newtheorem{lemma}{Lemma}

\newcommand{\tr}{\mathrm{tr}}
\newcommand{\ra}{\rightarrow}
\newcommand{\lan}{\langle}
\newcommand{\ran}{\rangle}
\newcommand{\norm}[1]{\left\lVert#1\right\rVert}
\newcommand{\inn}[1]{\lan#1\ran}
\newcommand{\ol}{\overline}
\begin{document}
\noindent Q2: We claim $m^\frac{1}{n}\in \Q$ if and only if $m=k^n$ for some $k \in \N$. We will prove the forward implication first. 
Suppose that $m^\frac{1}{n}\in \Q$. Then for  $a,b\in \N$ we have that $m= \frac{a^n}{b^n}$. Therefore, $b^n\cdot m = a^n$. By the fundamental theorem of arithemtic, $a,b$ and $m$ admit a unique prime factorization of the form $a= \Pi_{i=1}^n p_i^{e_i},b=\Pi_{i=1}^k p_i^{f_i}$ and $m=\Pi_{i=1}^k p_i^{g_i}$ for primes $p_i$, and $e_i,f_i,g_i \in \N$. We see that $$\Pi_{i=1}^k p_i^{g_i} \cdot \Pi_{i=1}^k p_i^{ne_i} = \Pi_{i=1}^k p_i^{g_i+ne_i} = \Pi_{i=1}^k p_i^{nf_i}$$
We see that $g_i$ must be a multiple of $n$, and we conclude that $m$ is an $n'th$ power. 
Now suppose that $m=k^n$ for some k. We have that $({k^n})^{\frac{1}{n}} = k$. Thus we are done.
\end{document}