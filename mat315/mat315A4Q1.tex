\documentclass[letterpaper]{article}
\usepackage[letterpaper,margin=1in,footskip=0.25in]{geometry}
\usepackage[utf8]{inputenc}
\usepackage{amsmath}
\usepackage{amsthm}
\usepackage{amssymb, pifont}
\usepackage{mathrsfs}
\usepackage{enumitem}
\usepackage{fancyhdr}
\usepackage{hyperref}

\pagestyle{fancy}
\fancyhf{}
\rhead{MAT 315}
\lhead{Assignment 4}
\rfoot{Page \thepage}

\setlength\parindent{24pt}
\renewcommand\qedsymbol{$\blacksquare$}

\DeclareMathOperator{\T}{\mathcal{T}}
\DeclareMathOperator{\V}{\mathcal{V}}
\DeclareMathOperator{\U}{\mathcal{U}}
\DeclareMathOperator{\Prt}{\mathbb{P}}
\DeclareMathOperator{\R}{\mathbb{R}}
\DeclareMathOperator{\N}{\mathbb{N}}
\DeclareMathOperator{\Z}{\mathbb{Z}}
\DeclareMathOperator{\Q}{\mathbb{Q}}
\DeclareMathOperator{\C}{\mathbb{C}}
\DeclareMathOperator{\ep}{\varepsilon}
\DeclareMathOperator{\identity}{\mathbf{0}}
\DeclareMathOperator{\card}{card}
\newcommand{\suchthat}{;\ifnum\currentgrouptype=16 \middle\fi|;}

\newtheorem{lemma}{Lemma}

\newcommand{\tr}{\mathrm{tr}}
\newcommand{\ra}{\rightarrow}
\newcommand{\lan}{\langle}
\newcommand{\ran}{\rangle}
\newcommand{\norm}[1]{\left\lVert#1\right\rVert}
\newcommand{\inn}[1]{\lan#1\ran}
\newcommand{\ol}{\overline}
\begin{document}
\noindent Q1i: We wish to solve $3x \equiv 5$(mod 7). We see that $\gcd(3,7)=1$. Since $1|5$, by theorem 3 this will have a solution. We can check and see that $[3(4)]_7=[5]_7$. So $x=4$ satisfies this equation. By theorem 3.7, we will have a general solution of the form $x=4+7t$.
\newline \\ Q1ii: We see that $\gcd(12,22) =2$ and $2\nmid 15$. Thus by theorem 3.7 this linear congruence has no solution.
\newline \\ Q1iii: We see that $\gcd(19,50)=1$ so a solution exists. Taking $x=18$ we see that $19\cdot 18 \equiv 342 \equiv 45 $mod $50$. 
\newline \\ Q1iv: From 1c we know that $x=19$ will solve. By lemma 3.9, we also have that $18x \equiv 42 $mod $50$ is equivalent to $9x\equiv 21 $mod 25. This is solved by $x=19$ as well. 
\newline Q1bi: Let $n=4\cdot 3\cdot 5 = 60$. Let $c_1=15,c_2=20,c_3=12$. We want to find $d_i=x$ such that $c_ix \equiv 1 $mod $(n_i)$. Take $d_1=3, d_2= 2, d_3 = 3$. Thus by the CRT the solution will be $x = 1\cdot 3\cdot 15 + 2\cdot 20\cdot 2 + 3\cdot 3\cdot 12=53 mod 60$
\newline Q1bii: Let $n=7\cdot 9 \cdot 4 = 252$. Let $c_1=36,c_2=28,c_3=63$. We want to find $d=x_i$ such that $c_ix\equiv 1 mod(n_i)$. Take $d_1=1,d_2=1,d_3=3$. Then the particular solution is $$x_0=2\cdot 1\cdot 36 + 7\cdot 1\cdot 28 + 3\cdot 3\cdot 63 = 79mod(252)$$
\newline Q1c: Suppose the remainders of the given number $x$ when divided by $3,5,7$ are $a_1,a_2,a_3$. We have that $n=105$, $n_1=3,n_2=5,n_3=7$, $c_1=35,c_2=21,c_3=15$. We must find $x=d_i$ which solve $c_i x \equiv 1 mod (a_i)$. Take $d_1=-1,d_2=1,d_3=1$. By the proof of theorem 3.10 the solution will be $$x_0= a_1 c_1 d_1 + a_2 d_2 c_2 + a_3 c_3 d_3 = -35a_1+21a_2+13a_3$$ This is in mod(105) so this formula will work for all numbers below 100.
\end{document}