\documentclass[letterpaper]{article}
\usepackage[letterpaper,margin=1in,footskip=0.25in]{geometry}
\usepackage[utf8]{inputenc}
\usepackage{amsmath}
\usepackage{amsthm}
\usepackage{amssymb, pifont}
\usepackage{mathrsfs}
\usepackage{enumitem}
\usepackage{fancyhdr}
\usepackage{hyperref}

\pagestyle{fancy}
\fancyhf{}
\rhead{MAT 315}
\lhead{Assignment 7}
\rfoot{Page \thepage}

\setlength\parindent{24pt}
\renewcommand\qedsymbol{$\blacksquare$}

\DeclareMathOperator{\F}{\mathbb{F}}
\DeclareMathOperator{\T}{\mathcal{T}}
\DeclareMathOperator{\V}{\mathcal{V}}
\DeclareMathOperator{\U}{\mathcal{U}}
\DeclareMathOperator{\Prt}{\mathbb{P}}
\DeclareMathOperator{\R}{\mathbb{R}}
\DeclareMathOperator{\N}{\mathbb{N}}
\DeclareMathOperator{\Z}{\mathbb{Z}}
\DeclareMathOperator{\Q}{\mathbb{Q}}
\DeclareMathOperator{\C}{\mathbb{C}}
\DeclareMathOperator{\ep}{\varepsilon}
\DeclareMathOperator{\identity}{\mathbf{0}}
\DeclareMathOperator{\card}{card}
\newcommand{\suchthat}{;\ifnum\currentgrouptype=16 \middle\fi|;}

\newtheorem{lemma}{Lemma}

\newcommand{\Mod}[1]{\ \mathrm{mod}\ (#1)}
\newcommand{\tr}{\mathrm{tr}}
\newcommand{\ra}{\rightarrow}
\newcommand{\lan}{\langle}
\newcommand{\ran}{\rangle}
\newcommand{\norm}[1]{\left\lVert#1\right\rVert}
\newcommand{\inn}[1]{\lan#1\ran}
\newcommand{\ol}{\overline}
\begin{document}
\noindent Q1a: We have that Alice's public key is $(3225997,13)$. We compute $\phi(3225997) = 1696\cdot 1900 = 3222400$. To find Alice's private key, we wish to find an $f_1$ such that $$e_1f_1 \equiv 13f_1 \equiv 1 \mod{3222400}, \gcd(f_1,13)=1$$
By python code (allowed on Piazza), we compute that $f_1=247877$ and $\gcd(247877,13)=1$. Therefore, Alice's private key is $(3225887,247877)$
\newline \\ Q1b: Given that Bob's private key is $(3250447,17)$, we wish to compute his public key $e_2$ by solving $$e_2f_2\equiv 17e_2 \equiv 1 \mod{\phi(3250447)}$$
Using python, we see that $e_2=954953$. Hence Bob's public key is $(3250447,954953)$
\newline \\ Q1c: We compute the encryption of 7 as $7^{13}\equiv 2642506\mod(3225887)$ using google. 
\newline \\ Q1d: We compute the sign of $11$ as $11^{17} \equiv 2494952 \equiv \mod{3250447}$
\end{document}