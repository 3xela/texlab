\documentclass[letterpaper]{article}
\usepackage[letterpaper,margin=1in,footskip=0.25in]{geometry}
\usepackage[utf8]{inputenc}
\usepackage{amsmath}
\usepackage{amsthm}
\usepackage{amssymb, pifont}
\usepackage{mathrsfs}
\usepackage{enumitem}
\usepackage{fancyhdr}
\usepackage{hyperref}

\pagestyle{fancy}
\fancyhf{}
\rhead{MAT 315}
\lhead{Assignment 5}
\rfoot{Page \thepage}

\setlength\parindent{24pt}
\renewcommand\qedsymbol{$\blacksquare$}

\DeclareMathOperator{\F}{\mathbb{F}}
\DeclareMathOperator{\T}{\mathcal{T}}
\DeclareMathOperator{\V}{\mathcal{V}}
\DeclareMathOperator{\U}{\mathcal{U}}
\DeclareMathOperator{\Prt}{\mathbb{P}}
\DeclareMathOperator{\R}{\mathbb{R}}
\DeclareMathOperator{\N}{\mathbb{N}}
\DeclareMathOperator{\Z}{\mathbb{Z}}
\DeclareMathOperator{\Q}{\mathbb{Q}}
\DeclareMathOperator{\C}{\mathbb{C}}
\DeclareMathOperator{\ep}{\varepsilon}
\DeclareMathOperator{\identity}{\mathbf{0}}
\DeclareMathOperator{\card}{card}
\newcommand{\suchthat}{;\ifnum\currentgrouptype=16 \middle\fi|;}

\newtheorem{lemma}{Lemma}

\newcommand{\tr}{\mathrm{tr}}
\newcommand{\ra}{\rightarrow}
\newcommand{\lan}{\langle}
\newcommand{\ran}{\rangle}
\newcommand{\norm}[1]{\left\lVert#1\right\rVert}
\newcommand{\inn}[1]{\lan#1\ran}
\newcommand{\ol}{\overline}
\begin{document}
\noindent Q2a: Since $\F_p(x) / s(x)\F_p(x)$ is a ring, to show that it is a field it suffices to show that it contains multiplicative inverses for nonzero elements. Let $[a(x)]_{s(x)} \in \F_p(x) / s(x)\F_p(x)$ be nonzero. 
Let $d(x) = gcd(a(x),s(x))$. By A4Q3d, we have that $d(x)|s(x)$. Therefore, either $d(x) = c\cdot s(x)$ or $d(x)=c$. Note that we can not have $d(x)=c\cdot s(x)$ since we would have that $c \cdot s(x) | a(x)$ and so $[a(x)]_{s(x)}  \equiv 0$, which can not happen by assumption. Thus $d(x)=c$. Without loss of generality, we can assume that $d(x)=1$. Thus for some $u(x),v(x)$ we can write $1 = u(x)a(x)+ v(x)s(x)$. We get that $v(x)s(x) = 1-u(x)a(x)$. Therefore, $s(x)|1-u(x)a(x)$ and so $[1]_{s(x)}\equiv [u(x)]_{s(x)} \cdot [a(x)]_{s(x)}$. Take $[a(x)]_{s(x)}^{-1} = [u(x)]_{s(x)}$. Thus $\F_p(x) / s(x)\F_p(x)$ is a field. 
\newline \\ Q2b: It is sufficient to show that $f(x)$ not prime if and only if there exists a $c$ such that $f(c)=0$. We proceed with the forward implication. Suppose that $f(x)$ is not prime. Then there exists some $q(x),p(x)\in \F_p[x]$ whose product is $f(x)$ and whose degrees are both stritcly less than that of $f(x)$. Thus we will either have that the degrees of $p(x),q(x)$ are either both 1 or 1 and 2. WLOG assume that $deg(p)=1$. Thus $p(x)=x-c$ for some $c\in \F_p$. We can therefore write $f(x)=(x-c)q(x)$. We see that $f(c)=0$. Now we prove the reverse implication. If there exists some $c$ where $f(c)=0$, then we can write $f(x)=(x-c)q(x)$ for some $q(x)$. Thus $f(x)$ has two divisors. 
\newline \\ Q2c: Since $deg(f)=3$, we can verify if it is a prime polynomial using 2b by checking if it has any roots in $\F_5$. Indeed, $s(1)\equiv 4,s(2)\equiv 2,s(3)\equiv 2,s(4)\equiv 4,s(0)\equiv 3$. This polynomial hsa no roots and therefore is prime. Therefore using the proof of 2a, and the euclidean algorithm for polynomials, we can find the inverse of $[x^3+3x+2]_{s(x)}$ by finding polynomials $u(x),v(x)$ which satisfy $u(x)(x^3+3x+2)+v(x)s(x)=1$. We can take $u(x)=3$ and $v(x)=3x$. By the proof of $2a$, $[3]_{s(x)}$ will be the desired inverse. 
\newline \\ Q2d: We can verify easily that $t(x)$ has no roots in $\F_5$. By 2b it must be a prime polynomial. From 2c, we see that the polynomial $v(x)=2x$ will satisfy the proof 2a, and will be our desired inverse. 
\end{document} 