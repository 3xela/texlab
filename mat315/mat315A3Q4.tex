\documentclass[letterpaper]{article}
\usepackage[letterpaper,margin=1in,footskip=0.25in]{geometry}
\usepackage[utf8]{inputenc}
\usepackage{amsmath}
\usepackage{amsthm}
\usepackage{amssymb, pifont}
\usepackage{mathrsfs}
\usepackage{enumitem}
\usepackage{fancyhdr}
\usepackage{hyperref}

\pagestyle{fancy}
\fancyhf{}
\rhead{MAT 315}
\lhead{Assignment 3}
\rfoot{Page \thepage}

\setlength\parindent{24pt}
\renewcommand\qedsymbol{$\blacksquare$}

\DeclareMathOperator{\U}{\mathcal{U}}
\DeclareMathOperator{\Prt}{\mathbb{P}}
\DeclareMathOperator{\R}{\mathbb{R}}
\DeclareMathOperator{\N}{\mathbb{N}}
\DeclareMathOperator{\Z}{\mathbb{Z}}
\DeclareMathOperator{\Q}{\mathbb{Q}}
\DeclareMathOperator{\C}{\mathbb{C}}
\DeclareMathOperator{\ep}{\varepsilon}
\DeclareMathOperator{\identity}{\mathbf{0}}
\DeclareMathOperator{\card}{card}
\newcommand{\suchthat}{;\ifnum\currentgrouptype=16 \middle\fi|;}

\newtheorem{lemma}{Lemma}

\newcommand{\tr}{\mathrm{tr}}
\newcommand{\ra}{\rightarrow}
\newcommand{\lan}{\langle}
\newcommand{\ran}{\rangle}
\newcommand{\norm}[1]{\left\lVert#1\right\rVert}
\newcommand{\inn}[1]{\lan#1\ran}
\newcommand{\ol}{\overline}
\begin{document}
Q4: We first claim that $1000^k$ has a remainder of $(-1)^k$ when divided by either 7 or 13. We see that 
\begin{align*}
    1000^k & = (1001-1)^k
    \\ & = (7\cdot 11 \cdot 13 -1)^k
    \\ & = \sum_{i=0}^k \begin{pmatrix} k \\ i \end{pmatrix} (7\cdot 11 \cdot 13)^{k-i}\cdot (-1)^{i}
    \\ & = 7\cdot 13 \cdot \Bigg[\sum_{i=1}^{k-1}\begin{pmatrix} k \\ i \end{pmatrix} (7\cdot 13)^{k-i-1} \cdot 11^{k-i} (-1)^i \Bigg] + (-1)^k
\end{align*}Therefore, $1000^k$ has a remainder of $(-1)^k$ when divided by 7 or 13. Therefore, the divisibility rule for 7 or 13 will be as follows. If $a=\sum_{k=0}^n a_k\cdot 1000^k$, for $a_k\in 1,2 \dots 999$. Therefore the divisibility rule is $a$ is divisible by 7 or 13 if $\sum_{k=1}^n a_k (-1)^k$ divides by 7 or 13, respectively. 
\end{document}