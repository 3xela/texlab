\documentclass[letterpaper]{article}
\usepackage[letterpaper,margin=1in,footskip=0.25in]{geometry}
\usepackage[utf8]{inputenc}
\usepackage{amsmath}
\usepackage{amsthm}
\usepackage{amssymb, pifont}
\usepackage{mathrsfs}
\usepackage{enumitem}
\usepackage{fancyhdr}
\usepackage{hyperref}

\pagestyle{fancy}
\fancyhf{}
\rhead{MAT 315}
\lhead{Assignment 4}
\rfoot{Page \thepage}

\setlength\parindent{24pt}
\renewcommand\qedsymbol{$\blacksquare$}

\DeclareMathOperator{\F}{\mathbb{F}}
\DeclareMathOperator{\T}{\mathcal{T}}
\DeclareMathOperator{\V}{\mathcal{V}}
\DeclareMathOperator{\U}{\mathcal{U}}
\DeclareMathOperator{\Prt}{\mathbb{P}}
\DeclareMathOperator{\R}{\mathbb{R}}
\DeclareMathOperator{\N}{\mathbb{N}}
\DeclareMathOperator{\Z}{\mathbb{Z}}
\DeclareMathOperator{\Q}{\mathbb{Q}}
\DeclareMathOperator{\C}{\mathbb{C}}
\DeclareMathOperator{\ep}{\varepsilon}
\DeclareMathOperator{\identity}{\mathbf{0}}
\DeclareMathOperator{\card}{card}
\newcommand{\suchthat}{;\ifnum\currentgrouptype=16 \middle\fi|;}

\newtheorem{lemma}{Lemma}

\newcommand{\tr}{\mathrm{tr}}
\newcommand{\ra}{\rightarrow}
\newcommand{\lan}{\langle}
\newcommand{\ran}{\rangle}
\newcommand{\norm}[1]{\left\lVert#1\right\rVert}
\newcommand{\inn}[1]{\lan#1\ran}
\newcommand{\ol}{\overline}
\begin{document}
\noindent 3a: Suppose that $a(x)\equiv b(x)$ mod $n(x)$. Then by the polynomial division algorithm, there exists $q_1(x),q_2(x)$ such that $a(x) = q_1(x)n(x)+r(x)$ and $b(x)=q_2(x)n(x)+r(x)$. Computing their difference, we see 
$$a(x)-b(x) = q_1(x)n(x)-q_2(x)n(x)=[q_1(x)-q_2(x)]n(x)$$ We see that $n(x)$ divides $a(x)-b(x)$. Now suppose that $n(x)|a(x)-b(x)$. 
There exists some polynomials $q_1(x),q_2(x),r_1(x),r_2(x)$ such that $a(x)=q_1(x)n(x)+r_1(x)$ and $b(x)= q_2(x)n(x)+r_2(x)$ with $deg(r_1(x))<deg(q_1(x))$ and $deg(r_2(x)<deg(q_2(x))$. Since $n(x)|a(x)-b(x)$ there exists some $q(x)$ where $n(x)q(x) = a(x)-b(x)$. Now we compute that $$a(x)-b(x)-(q_1(x)n(x)-q_2(x)n(x)) = r_1(x)-r_2(x) = (q(x)-q_1(x)+q_2(x))n(x)$$ Assume that $(q(x)-q_1(x)+q_2(x))\neq 0$. 
This implies that $deg(r_1(x)-r_2(x))\geq deg(n(x))deg(q(x)-q_1(x)+q_2(x)) \geq deg(n(x))$. This is a contradiction. Thus $a(x)\equiv b(x)$mod $n(x)$
\newline \\ 3b: Since $a(x)\equiv b(x)$ mod $n(x)$ when they have the same remainder after division by $n(x)$, to count $ \F_p[X] / n(x) \F_p[x]$ it suffices to count how many polynomials with coefficients in $\F_p$ exist with degree less than $deg(n(x))$. There are $deg(n(x))$ possible terms in each polynomial, and each term has a choice of $p$ coefficients. Thus by basic counting $|\F_p[X] / n(x) \F_p[x]|= p^{deg(n(x))}$
\newline \\ 3c: Suppose that $a(x)\equiv a^\prime(x)$mod $n(x)$ and $b(x)\equiv b^\prime(x)$mod $n(x)$. By 3a we know that there exists $q_1(x)$ and $q_2(x)$ such that $a(x)-a^\prime(x) = q_1(x)n(x)$ and $b(x)-b^\prime(x)= q_2(x)n(x)$. We compute 
$$[a(x)+b(x)]-[a^\prime(x)+b^\prime(x)] = [q_1(x)+q_2(x)]n(x)$$
Which implies that $a(x)+b(x)\equiv a^\prime(x)+b^\prime(x)$mod $n(x)$. Hence addition is well defined. We now check multiplication. We compute 
\begin{align*}
    a(x)b(x)-a^\prime(x)b^\prime(x) & = (a^\prime(x)+q_1(x)n(x))(b^\prime(x)+q_2(x)n(x))-a^\prime(x)b^\prime(x)
    \\ & = a^\prime(x)b^\prime(x)+ a^\prime(x)q_2(x)n(x) + b^\prime(x)q_1(x)n(x) + q_1(x)q_2(x)n^2(x)-a^\prime(x)b^\prime(x)
    \\ & = n(x)[a^\prime(x)q_2(x)+b^\prime(x)q_1(x) + q_1(x)q_2(x)n(x)]
\end{align*}
Therefore, $a(x)b(x)\equiv a^\prime(x) b^\prime(x)$mod $n(x)$ hence multiplication is well defined. 
\newline \\ 3d: Suppose that $gcd(a(x),b(x))= d(x) \nmid a(x)$. By the division algorithm there exists polynomials $q(x), r(x)$ such that $a(x) = q(x)d(x)+r(x)$. Since $d(x) = u(x)a(x)+v(x)b(x)$ for some $u(x),v(x)\in \F_p[x]$, we can rewrite $$r(x)= a(x)-q(x)d(x) = a(x)-q(x)[u(x)a(x)+v(x)b(x)] = (1-q(x)u(x))a(x)+(-q(x)v(x))b(x)$$ 
Let $r\in \F_p$ be the leading coefficient of $r(x)$. We have that $gcd(p,r)=1$ and so by corrollary 3.8, $rx\equiv 1$mod$p$ will have a solution $y$. If we multiply $r(x)$ by $y$ then we have that $$y\cdot r(x) = y\cdot (1-q(u)u(x))a(x) + y\cdot (-q(x)v(x))b(x)$$
Since we multiplied by the inverse of $r$, $\cdot r(x) $ is not monic. By the division algorithm, $deg(r(x))<deg(d(x))$, and so $deg(y \cdot r(x))<deg(d(x))$ which is a contradiction. Therefore, $d(x)|a(x)$ and similarly $d(x)|b(x)$. 
\end{document}