\documentclass[letterpaper]{article}
\usepackage[letterpaper,margin=1in,footskip=0.25in]{geometry}
\usepackage[utf8]{inputenc}
\usepackage{amsmath}
\usepackage{amsthm}
\usepackage{amssymb, pifont}
\usepackage{mathrsfs}
\usepackage{enumitem}
\usepackage{fancyhdr}
\usepackage{hyperref}

\pagestyle{fancy}
\fancyhf{}
\rhead{MAT 315}
\lhead{Assignment 3}
\rfoot{Page \thepage}

\setlength\parindent{24pt}
\renewcommand\qedsymbol{$\blacksquare$}

\DeclareMathOperator{\U}{\mathcal{U}}
\DeclareMathOperator{\Prt}{\mathbb{P}}
\DeclareMathOperator{\R}{\mathbb{R}}
\DeclareMathOperator{\N}{\mathbb{N}}
\DeclareMathOperator{\Z}{\mathbb{Z}}
\DeclareMathOperator{\Q}{\mathbb{Q}}
\DeclareMathOperator{\C}{\mathbb{C}}
\DeclareMathOperator{\ep}{\varepsilon}
\DeclareMathOperator{\identity}{\mathbf{0}}
\DeclareMathOperator{\card}{card}
\newcommand{\suchthat}{;\ifnum\currentgrouptype=16 \middle\fi|;}

\newtheorem{lemma}{Lemma}

\newcommand{\tr}{\mathrm{tr}}
\newcommand{\ra}{\rightarrow}
\newcommand{\lan}{\langle}
\newcommand{\ran}{\rangle}
\newcommand{\norm}[1]{\left\lVert#1\right\rVert}
\newcommand{\inn}[1]{\lan#1\ran}
\newcommand{\ol}{\overline}
\begin{document}
\noindent Q1a: Suppose that $p|a^k$. By Corr 2.2 $p|a$. Therefore, by prime factorization of $a$, we must have that $p^k|a^k$. 
This does not hold when $p$ is composite. Consider when $p=4,a=2,k=4$. Then $4|16$ but $4\nmid 2$. 
\newline \\ Q1b: It is easy to check that $132 = 2^2 \cdot 3\cdot 11$, $400 = 2^4 \cdot 5^2$,  and $1995 = 3\cdot 5\cdot 7 \cdot 19$. By the formula for the gcd of 2 numbers (pg. 23 jones and jones), we see $\gcd(132,400) = 2^2 = 4$, $\gcd(132,1995) = 3$, $\gcd(400,1995) = 5$,$\gcd(132,400,1995) = \gcd(\gcd(132,400),1995) = \gcd(4,1995) = 1$
\newline \\ Q1c: i) This is true. Given that $gcd(a,p^2) = p$, this means that for some $k\in \Z$, $kp=a$. Therefore, $$\gcd(a^2,p^2)=\gcd(k^2p^2,p^2) = p^2\gcd(k^2,1) = p^2$$
\newline ii) False. Consider $a=p$ and $b=p^3$, then we have that $\gcd(p,p^2) = p$, $\gcd(p^3,p^2)=p^2$, but $\gcd(p^4,p^4)=p^4$. 
\newline \\ iii) This is true. Since $\gcd(a,p^2)=p$, there exists some $k\in \Z $, with $k$ coprime to $p$ such that $kp=a$. Similarly, there is some $l \in \Z$ with the the same proprty but $lp = b$. Therefore, the product $kl$ is also coprime with $p$ and $p^2$ by the contrapositive of lemma 2.1 b. Thus we have $$\gcd(ab,p^4) = \gcd(klp^2,p^4) = p^2\gcd(kl,p^2) = p^2$$
\newline iv) False. Take $a=p^2-p$. First we claim for $p$ prime, $\gcd(p,p-1)=1$. Suppose that $k|p$ and $k|p-1$. By Corr 1.4, $k|p-(p-1)=1$, so $k=1$. Then we have that $\gcd(p^2-p,p^2) =p \gcd(p-1,p)=p$. We also have that $\gcd(a+p,p^2) = \gcd(p^2,p^2)=p^2$. 
\end{document}