\documentclass[letterpaper]{article}
\usepackage[letterpaper,margin=1in,footskip=0.25in]{geometry}
\usepackage[utf8]{inputenc}
\usepackage{amsmath}
\usepackage{amsthm}
\usepackage{amssymb, pifont}
\usepackage{mathrsfs}
\usepackage{enumitem}
\usepackage{fancyhdr}
\usepackage{hyperref}

\pagestyle{fancy}
\fancyhf{}
\rhead{MAT 315}
\lhead{Assignment 5}
\rfoot{Page \thepage}

\setlength\parindent{24pt}
\renewcommand\qedsymbol{$\blacksquare$}

\DeclareMathOperator{\F}{\mathbb{F}}
\DeclareMathOperator{\T}{\mathcal{T}}
\DeclareMathOperator{\V}{\mathcal{V}}
\DeclareMathOperator{\U}{\mathcal{U}}
\DeclareMathOperator{\Prt}{\mathbb{P}}
\DeclareMathOperator{\R}{\mathbb{R}}
\DeclareMathOperator{\N}{\mathbb{N}}
\DeclareMathOperator{\Z}{\mathbb{Z}}
\DeclareMathOperator{\Q}{\mathbb{Q}}
\DeclareMathOperator{\C}{\mathbb{C}}
\DeclareMathOperator{\ep}{\varepsilon}
\DeclareMathOperator{\identity}{\mathbf{0}}
\DeclareMathOperator{\card}{card}
\newcommand{\suchthat}{;\ifnum\currentgrouptype=16 \middle\fi|;}

\newtheorem{lemma}{Lemma}

\newcommand{\Mod}[1]{\ \mathrm{mod}\ (#1)}
\newcommand{\tr}{\mathrm{tr}}
\newcommand{\ra}{\rightarrow}
\newcommand{\lan}{\langle}
\newcommand{\ran}{\rangle}
\newcommand{\norm}[1]{\left\lVert#1\right\rVert}
\newcommand{\inn}[1]{\lan#1\ran}
\newcommand{\ol}{\overline}
\begin{document}
\noindent Q3a: Suppose that for $f(x)\in \Z[x]$ there exists some $g(x)$ where $f(x)g(x)=1$. Taking the degrees of both sides we have that the degree of $f$ and $g$ must both be 0, and hence $f$ and $g$ are constant. Furthermore, if the product of two integers are 1, then they must both be 1 or $-1$. Therefore, $f(x)= \pm 1$
\newline \\ Q3bi: Consider the polynomial $f(x) = (x-3)(x-4) = x^2-7x+12$. This is clearly primitive since $\gcd(1,-7,12)=1$ but it can be written as the product of two linear polynomials.
\newline \\ Q3bii: Consider the polynomial $f(x)= 2x^2+6$. This polynomial is irreducible over $\Q$ since it does not split and hence can not be prime. It is clearly not primitive, since $\gcd(2,6)= 2\neq 1$. $f$ is not irreducible in $\Z[x]$, since $f(x)= 2(x^2+3)$
\newline \\ Q3c: Since $f(r)=0$, for some $g$ we can write $f(x)=(x-\frac{a}{b})g(x)$. Let $g(x) = \sum_{i=0}^n c_i x^i$. We claim that $b|c_i$ for each $i$. Indeed, we can verify that by expanding the product, $$f(x)=(x-\frac{a}{c})g(x) = \sum_{i=0}^n c_i x^{i+1} - \sum_{i=0}^n \frac{a}{b}c_ix^i$$
Since the coefficients of $f$ are in $\Z$, we have that $c_i\cdot \frac{a}{b}\in \Z$ and so we conclude that $b|c_i$. Therefore we can write $c_i = b\cdot d_i$. Therefore 
$$f(x) = (x-\frac{a}{b})\sum_{i=0}^n c_i x^i = (x-\frac{a}{b})b \sum_{i=0}^n d_i x^i = (bx-a)\sum_{i=0}^n d_ix^i$$
As desired. 
\newline \\ Q3d: We know that when we multiple two polynomials, $c_{j_0+i_0} = \sum_{k=0}^{i_0+j_0} a_{k}b_{i_0+j_0-k} $. Assume that $j_0>i_0$ and we see that $p$ will divide the first $i_0$ terms, but at the $j_0'th$ term $p \nmid a_{j_0}\cdot b_{i_0}$
\end{document}