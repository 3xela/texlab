\documentclass[letterpaper]{article}
\usepackage[letterpaper,margin=1in,footskip=0.25in]{geometry}
\usepackage[utf8]{inputenc}
\usepackage{amsmath}
\usepackage{amsthm}
\usepackage{amssymb, pifont}
\usepackage{mathrsfs}
\usepackage{enumitem}
\usepackage{fancyhdr}
\usepackage{hyperref}

\pagestyle{fancy}
\fancyhf{}
\rhead{MAT 315}
\lhead{Assignment 2}
\rfoot{Page \thepage}

\setlength\parindent{24pt}
\renewcommand\qedsymbol{$\blacksquare$}

\DeclareMathOperator{\U}{\mathcal{U}}
\DeclareMathOperator{\Prt}{\mathbb{P}}
\DeclareMathOperator{\R}{\mathbb{R}}
\DeclareMathOperator{\N}{\mathbb{N}}
\DeclareMathOperator{\Z}{\mathbb{Z}}
\DeclareMathOperator{\Q}{\mathbb{Q}}
\DeclareMathOperator{\C}{\mathbb{C}}
\DeclareMathOperator{\ep}{\varepsilon}
\DeclareMathOperator{\identity}{\mathbf{0}}
\DeclareMathOperator{\card}{card}
\newcommand{\suchthat}{;\ifnum\currentgrouptype=16 \middle\fi|;}

\newtheorem{lemma}{Lemma}

\newcommand{\tr}{\mathrm{tr}}
\newcommand{\ra}{\rightarrow}
\newcommand{\lan}{\langle}
\newcommand{\ran}{\rangle}
\newcommand{\norm}[1]{\left\lVert#1\right\rVert}
\newcommand{\inn}[1]{\lan#1\ran}
\newcommand{\ol}{\overline}
\begin{document}
\noindent Q5a: By corrollary 1.2 we have that for any $a,b\in Z$, there exists unique $q,r\in \Z$ so that $a=qb+r$ and $|r|< |b|$.
We check 2 cases. First if $|r|\leq \frac{|b|}{2}$ then we are done. If not, that is if $\frac{|b|}{2}<r<|b|$ then we do the following. If $b>0$, then we have $a=b(q+1)+(r-b)$. We have that $|r-b|< \frac{|b|}{2}$. 
Now if $b<0$, then we have $a=b(q+1)+(r+b)$ and $|r+b|<\frac{|b|}{2}$.
We now claim uniqueness. Suppose that $a=q_1 b+r_1 = q_2 b +r_2$. We have that $b(q_2-q_1)=r_1-r_2$. Suppose that $q_1\neq q_2$, then we have that $|q_2-q_1|\geq 1$. This implies that $|r_1-r_2|\geq b$. But since $|r_1|,|r_2|<\frac{|b|}{2}$, this can never happen. Hence $p_1=p_2$ and $r_1=r_2$.
The new updated Euclidean Algorithm is as follows: 
\begin{align*}
    a & =q_1b+r_1
   \\ b & = q_2 r_1 + r_2 
   \\ r_1 & = q_3 r_2 + r_3
   \\ \vdots
   \\ r_n = 0
\end{align*}
At the i'th step, the remainder term $r_i$ will be bounded above by $\frac{|b|}{2^i}$. 
\newline \\ Q5b: We will compute $\gcd(1066,1492)$ and $\gcd(1485,1745)$. First, we will use the Euclidean Algorithim: 
\begin{align*}
    1492 & = 1066 + 426
    \\ 1066 & = 2\cdot 426 + 214
    \\ 426  & = 214 + 212
    \\ 214 & = 212 + 2
    \\ & 212 = 106\cdot 2
\end{align*}We see in 5 steps that $\gcd(1066,1492) =2 $. We will now compute this with our new least remainder algorithim. 
\begin{align*}
    1492 &= 1066 + 426
    \\ 1066 & =  3\cdot 426 -212
    \\ 426 &   = -2\cdot (-212) + 2
    \\ -212 & = -106\cdot 2
\end{align*} We get the same result but in 4 steps.
Now for $\gcd(1485,1745)$, we know from previously that the Euclidean algorithim will return 5 as our result in 6 steps. Using the least remainders algorithim;
\begin{align*}
    1745 & = 1485 + 260
    \\ 1485 & = 6\cdot 260 -75
    \\ 260 & = -3 \cdot(-75) + 35
    \\ 75 & = 2\cdot 35 + 5
    \\ 35 & = 5\cdot 7
\end{align*}
This terminates in $5$ steps. This is faster than the euclidean algorithim. 
\newline \\ Q5c: In general, suppose that the algorithim terminates in $n$ steps. Since $r_n=0$ and $r_i<\frac{|b|}{2^i}$ we will have that $\frac{|b|}{2^n}<1$ and so $|b|<2^n$. Therefore the number of steps, $n$, bounds above the quantity $\log_2(|b|)$. So The number of steps will be $n= \lceil\log_2(|b|) \rceil$
\end{document}