\documentclass[letterpaper]{article}
\usepackage[letterpaper,margin=1in,footskip=0.25in]{geometry}
\usepackage[utf8]{inputenc}
\usepackage{amsmath}
\usepackage{amsthm}
\usepackage{amssymb, pifont}
\usepackage{mathrsfs}
\usepackage{enumitem}
\usepackage{fancyhdr}
\usepackage{hyperref}

\pagestyle{fancy}
\fancyhf{}
\rhead{MAT 315}
\lhead{Assignment 9}
\rfoot{Page \thepage}

\setlength\parindent{24pt}
\renewcommand\qedsymbol{$\blacksquare$}

\DeclareMathOperator{\T}{\mathcal{T}}
\DeclareMathOperator{\V}{\mathcal{V}}
\DeclareMathOperator{\U}{\mathcal{U}}
\DeclareMathOperator{\Prt}{\mathbb{P}}
\DeclareMathOperator{\R}{\mathbb{R}}
\DeclareMathOperator{\N}{\mathbb{N}}
\DeclareMathOperator{\Z}{\mathbb{Z}}
\DeclareMathOperator{\Q}{\mathbb{Q}}
\DeclareMathOperator{\C}{\mathbb{C}}
\DeclareMathOperator{\ep}{\varepsilon}
\DeclareMathOperator{\identity}{\mathbf{0}}
\DeclareMathOperator{\card}{card}
\newcommand{\suchthat}{;\ifnum\currentgrouptype=16 \middle\fi|;}

\newtheorem{lemma}{Lemma}

\newcommand{\bd}{\partial}
\newcommand{\tr}{\mathrm{tr}}
\newcommand{\ra}{\rightarrow}
\newcommand{\lan}{\langle}
\newcommand{\ran}{\rangle}
\newcommand{\norm}[1]{\left\lVert#1\right\rVert}
\newcommand{\inn}[1]{\lan#1\ran}
\newcommand{\ol}{\overline}
\begin{document}
\noindent 7.14 4a: By the proof of 7.13, we have that $$16^2=6+5^3\cdot 2$$
We have that $q=2,r=16,p=5,i=3$ with $k$ unknown. So therefore we have that $s=16+5^4k$ and so $s^2 = r^2+2rp^ik+p^{2i}k^2 \equiv a+(q+2+rk)p^{i}\mod{p^{i+1}}$
So we want to solve the following: 
$$q+ 2rk \equiv 0\mod{5} \implies 2+2(16)k\equiv 0 \mod{5}$$ And we see that $k=4$ solves. Therefore $s=16+5^3\cdot 4 = 516$. Thus the square roots of $6 \mod{54}$ are $\pm 516$. 
\newline \\ 7.16a: We see that $41 \equiv 3^2 \mod{2^5}$ with $r=3$ and so $3^2= 41+2^5\cdot -1$. We see that $q=-1$ when $k=1$. Therefore $q+rk = -1+3\cdot 1 = 2$. Thus $s=13+2^4\cdot 1=19$ is a square root of $41\mod{2^6}$. Multiplying by $\pm 1$ and $\pm 31$ we get that $\pm 19$ and $\pm 13$ are our desired square roots. 
\newline \\ 7.18: We know that $168 = 3\cdot 7\cdot 8$. By the previous results we will work with $25 \equiv 1 \mod{2^3}, 25 \equiv 1 \mod{3}, 25\equiv 4 \mod{7}$. These will have solutions of $s\equiv 1 \mod{2}, s\equiv \pm 1\mod{3}, s\equiv \pm 2\mod{7}$. By CRT, the solutions are $s\equiv \pm (5,19,23,37,47,61,65,79) \mod{168}$. 
\newline \\ 7.26: We have that $513 = 3^3 \cdot 19$. Hence the square roots of $7\mod{3^3}$ is $s = \pm 13\mod{3^3}$ and $7\mod{19}$ will have square root of $s = \pm 8 \mod{19}$. By CRT the square root will be $s = \pm 68, 122\mod{513} 68$
\end{document}