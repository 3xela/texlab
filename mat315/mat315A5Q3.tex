\documentclass[letterpaper]{article}
\usepackage[letterpaper,margin=1in,footskip=0.25in]{geometry}
\usepackage[utf8]{inputenc}
\usepackage{amsmath}
\usepackage{amsthm}
\usepackage{amssymb, pifont}
\usepackage{mathrsfs}
\usepackage{enumitem}
\usepackage{fancyhdr}
\usepackage{hyperref}

\pagestyle{fancy}
\fancyhf{}
\rhead{MAT 315}
\lhead{Assignment 5}
\rfoot{Page \thepage}

\setlength\parindent{24pt}
\renewcommand\qedsymbol{$\blacksquare$}

\DeclareMathOperator{\F}{\mathbb{F}}
\DeclareMathOperator{\T}{\mathcal{T}}
\DeclareMathOperator{\V}{\mathcal{V}}
\DeclareMathOperator{\U}{\mathcal{U}}
\DeclareMathOperator{\Prt}{\mathbb{P}}
\DeclareMathOperator{\R}{\mathbb{R}}
\DeclareMathOperator{\N}{\mathbb{N}}
\DeclareMathOperator{\Z}{\mathbb{Z}}
\DeclareMathOperator{\Q}{\mathbb{Q}}
\DeclareMathOperator{\C}{\mathbb{C}}
\DeclareMathOperator{\ep}{\varepsilon}
\DeclareMathOperator{\identity}{\mathbf{0}}
\DeclareMathOperator{\card}{card}
\newcommand{\suchthat}{;\ifnum\currentgrouptype=16 \middle\fi|;}

\newtheorem{lemma}{Lemma}

\newcommand{\tr}{\mathrm{tr}}
\newcommand{\ra}{\rightarrow}
\newcommand{\lan}{\langle}
\newcommand{\ran}{\rangle}
\newcommand{\norm}[1]{\left\lVert#1\right\rVert}
\newcommand{\inn}[1]{\lan#1\ran}
\newcommand{\ol}{\overline}
\begin{document}
\noindent Q3a: We will show that $ev$ is a ring homomorphism in several steps. First we claim that it is a group homomorphism from additive group $(\F_p[x],+)$ to $(Fun(\F_p,\F_p),+)$. First, note that $$ev([0(x)]_p)(c) = ev([0]_p)(c) = [0]_p(c) = \sum_{k=0}^n 0\cdot c^k = [0]_p$$ 
Now we show that it preserves the structure of addition. Let $a(x),b(x)\in \F_p[x]$. We see that
$$ev(a(x)+b(x))(c) = ev(a+b)(c) = \sum_{k=0}^n (a_k+b_k)c^k = \sum_{k=0}^n a_k c^k + \sum_{k=0}^n b_k c^k = ev(a(x))(c) + ev(b(x))(c)$$
We now show that it sends the multiplicative identity to the multiplicative identity. 
$$ev(1(x))(c) = ev(1)(c) = \sum_{k=0}^n 1\cdot c^0 = [1]_p$$
Finally it remains to show it preserves the structure of multiplication. Let $a(x),b(x)\in \F_p[x]$
$$ev(a(x)\cdot b(x))(c) = ev(a\cdot b)(c) = (a\cdot b)(c) = a(c)\cdot b(c) = ev(a(x))(c)\cdot ev(b(x))(c)$$ 
Therefore, $ev$ is a ring homomorphism. 
\newline \\ Q3b: Let $q(x) = x^p-x$. To show $\widetilde{ev}$ is well defined it must be shown that if $[f(x)]_{q(x)} = [g(x)]_{q(x)}$, then $\widetilde{ev}([f(x)]_{q(x)}) = \widetilde{ev}([g(x)]_{q(x)})$. Suppose that $[f(x)]_{q(x)} = [g(x)]_{q(x)}$. Then by the euclidian algorithm for polynomials, there exists $p_1(x),p_2(x),r(x)$ such that $f(x) = p_1(x)q(x) + r(x)$ and $g(x) = p_2(x)q(x)+r(x)$. 
Therefore, $$\widetilde{ev}([f(x)]_{q(x)})(c) = ev(r(x))(c) = \widetilde{ev}([g(x)]_{q(x)})$$
Hence this map is well defined. Note that it is also a ring homomorphism by almost the exact same reasoning as in 3a, since it is a field as well. 
\newline \\ Q3c: Let $x^p-x=q(x)$ Suppose that $\widetilde{ev}([f(x)]_{q(x)}) = \widetilde{ev}([g(x)]_{q(x)})$. This is the same as saying that $\widetilde{ev}([f(x)]_{q(x)} - [g(x)]_{q(x)})=0$. By definition of $\widetilde{ev}$, we have that $ev(f-g)(c)=0$ for all $c$. Therefore, $x,(x-1),\dots (x-(p-1))$ each divide $f(x)-g(x)$. We now claim that for $a\neq b$, $x-a$ is coprime to $x-b$. Indeed, we see that $$(a-b)^{-1}(x-b)-(a-b)^{-1}(x-a) = 1$$ We further assert that if for some polynomials, $a_1(x)\dots a_n(x)$ mutually coprime, if $a_i(x)|p(x)$ then \newline $a_1(x)\cdot \dots a_n(x)|p(x)$. We will prove this by induction. For the case when $n=2$, this is true by fact 3. Now suppose that it holds for $n$. We want to show that this is true for $n+1$. By assumption, $a_1(x)\cdot \dots a_n(x)|p(x)$. It is enough to show that $\gcd(a_1(x)\cdot \dots \cdot a_n(x),a_{n+1})=1$. 
We know that there exists $u_i(x),v_i(x)$ such that $u_i(x)a_i(x) + v_i(x)a_{n+1}(x)=1$. Multiplying each of these equations together, get
\begin{align*}
1 & =\prod_{i=1}^n (u_i(x) a_i(x) + v_i(x) a_{n+1}(x))
\\ & = P(x)a_1(x)\cdot \dots \cdot a_{n}(x) + Q(x)a_{n+1}(x)
\end{align*} For some polynomials $P(x),Q(x)$. Therefore, they are coprime and the claim is proven. Therefore $x(x-1)\dots (x-(p-1))|f(x)-g(x)$ and so $x^p-x|f(x)-g(x)$. We can therefore conclude that $[f(x)]_{q(x)} = [g(x)]_{q(x)}$. 
\newline \\ Q3d: It is sufficient to show the cardinalities of the domain and co-domain are equal. By A4Q3b, \newline $|\F_p[x] / x^p-x \F_p[x]|=p^p$. We claim the cardinality of $Fun(\F_p,\F_p)$ is the same. Indeed, for $f\in Fun(\F_p,\F_p)$, it will have $p$ possible inputs, and each input has $p$ possible outputs. Therefore there are $p^p$ possible functions. Therefore, we can conclude that $\widetilde{ev}$ is a ring isomorphism. 
\newline \\ Q3e: Since $\widetilde{ev}$ is a bijection it has an inverse. Therefore, for any $g\in Fun(\F_p,\F_p)$, we can apply $\widetilde{ev}^{-1}$ to $g$ and get a polynomial in $\F_p[x] / q(x) \F_p[x]$
\end{document}