\documentclass[letterpaper]{article}
\usepackage[letterpaper,margin=1in,footskip=0.25in]{geometry}
\usepackage[utf8]{inputenc}
\usepackage{amsmath}
\usepackage{amsthm}
\usepackage{amssymb, pifont}
\usepackage{mathrsfs}
\usepackage{enumitem}
\usepackage{fancyhdr}
\usepackage{hyperref}

\pagestyle{fancy}
\fancyhf{}
\rhead{MAT 315}
\lhead{Assignment 5}
\rfoot{Page \thepage}

\setlength\parindent{24pt}
\renewcommand\qedsymbol{$\blacksquare$}

\DeclareMathOperator{\F}{\mathbb{F}}
\DeclareMathOperator{\T}{\mathcal{T}}
\DeclareMathOperator{\V}{\mathcal{V}}
\DeclareMathOperator{\U}{\mathcal{U}}
\DeclareMathOperator{\Prt}{\mathbb{P}}
\DeclareMathOperator{\R}{\mathbb{R}}
\DeclareMathOperator{\N}{\mathbb{N}}
\DeclareMathOperator{\Z}{\mathbb{Z}}
\DeclareMathOperator{\Q}{\mathbb{Q}}
\DeclareMathOperator{\C}{\mathbb{C}}
\DeclareMathOperator{\ep}{\varepsilon}
\DeclareMathOperator{\identity}{\mathbf{0}}
\DeclareMathOperator{\card}{card}
\newcommand{\suchthat}{;\ifnum\currentgrouptype=16 \middle\fi|;}

\newtheorem{lemma}{Lemma}

\newcommand{\tr}{\mathrm{tr}}
\newcommand{\ra}{\rightarrow}
\newcommand{\lan}{\langle}
\newcommand{\ran}{\rangle}
\newcommand{\norm}[1]{\left\lVert#1\right\rVert}
\newcommand{\inn}[1]{\lan#1\ran}
\newcommand{\ol}{\overline}
\begin{document}
\noindent Q1a: To show that $\alpha$ is a ring isomorphism, we will show that it is a group homomorphism with respect to polynomial addition, then show it maps the identity to the identity, and that it respects multiplication in the ring. Observe
\begin{align*}
\alpha([a(x)]_{f(x)g(x)} + [b(x)]_{f(x)g(x)}) & = \alpha([a(x)+b(x)]_{f(x)g(x)}) 
\\ & = ([a(x)+b(x)]_{f(x)},[a(x)+b(x)]_{g(x)})
\\ & = ([a(x)]_{f(x)} + [b(x)]_{f(x)}, [a(x)]_{g(x)} + [b(x)]_{g(x)})
\\ & = ([a(x)]_{f(x)},[a(x)]_{g(x)}) + ([b(x)]_{f(x)},[b(x)]_{g(x)})
\\ & = \alpha([a(x)]_{f(x)g(x)}) + \alpha([b(x)]_{f(x)g(x)})
\end{align*}
We now show it respects the identity element; 
\begin{align*}
\alpha([1]_{f(x)g(x)}) & = ([1]_{f(x)},1_{g(x)})
\end{align*}
Which is the identity element in the product ring. Finally we show that the mapping $\alpha$ respects products. 
\begin{align*}
\alpha([a(x)]_{f(x)g(x)}\cdot [b(x)]_{f(x)g(x)}) & = \alpha([a(x)\cdot b(x)]_{f(x)g(x)})
\\ & = ([a(x)\cdot b(x)]_{f(x)},[a(x)\cdot b(x)]_{g(x)})
\\ & = ([a(x)]_{f(x)}\cdot[b(x)]_{f(x)}, [a(x)]_{g(x)}\cdot [b(x)]_{g(x)})
\\ & = ([a(x)]_{f(x)} , [a(x)]_{g(x)})\cdot ([b(x)]_{f(x)},[b(x)]_{g(x)})])
\\ & = \alpha([a(x)]_{f(x)g(x)}) \cdot \alpha([b(x)]_{f(x)g(x)})
\end{align*}
As desired. Finally we will show that it is a bijection. We claim that the domain and codomain have the same carinality. Indeed, 
\begin{align*}
|\F_p(x) / f(x)g(x)\F_p(x)|  & = p^{deg(f(x)g(x))} \tag{A4 Q3b}
\\ & = p^{deg(f(x)) + deg(g(x))} \tag{by properties of polynomials}
\\ & = p^{deg(f(x))} \cdot p^{deg(g(x))}
\\ & = |\F_p(x) / f(x)\F_p(x)|\cdot |\F_p(x) / g(x) \F_p(x)|
\end{align*} Thus it suffices to show that $\alpha$ is a ring injection. Suppose that $\alpha([a(x)]_{f(x)g(x)}) = \alpha([b(x)]_{f(x)g(x)})$. This implies that $[a(x)_{f(x)}] = [b(x)_{f(x)}]$ and $[a(x)_{g(x)}] = [b(x)_{g(x)}]$
Therefore, $f(x)|a(x)-b(x)$ and $g(x)|a(x)-b(x)$. Since $f(x)$ and $g(x)$ are coprime, we have that $f(x)g(x)|a(x)-b(x)$. Therefore, $[a(x)]_{f(x)g(x)} = [b(x)]_{f(x)g(x)}$, and we conclude that $\alpha$ is an injection. 
\newline \\ Q1b: Since the polynomaisl $f(x),g(x)$ are coprime, there exists $z(x),y(x)$ such that $z(x)f(x) + y(x)g(x)=1$. We see that $y(x)g(x) = 1 - z(x)f(x)$ or equivalently, $[y(x)g(x)]_{f(x)} = [1]_{f(x)}$. By almost exactly the same argument we have that $[z(x)f(x)]_g(x) = [1]_{g(x)}$ 
\newline \\ Q1c: Let $c(x) = a(x)y(x)g(x) + b(x)z(x)f(x)$. We can verify that 
\begin{align*}
[c(x)]_{f(x)} & = [a(x)y(x)g(x) + b(x)z(x)f(x)]_{f(x)}
\\ & = [a(x)]_f(x) \cdot [y(x)g(x)]_f(x) + [b(x)z(x)f(x)]_{f(x)}
\\ & = [a(x)]_{f(x)}
\end{align*} By almost the exact same computation we can verify that $[c(x)]_{g(x)} = [b(x)]_{g(x)}$
\end{document}