\documentclass[letterpaper]{article}
\usepackage[letterpaper,margin=1in,footskip=0.25in]{geometry}
\usepackage[utf8]{inputenc}
\usepackage{amsmath}
\usepackage{amsthm}
\usepackage{amssymb, pifont}
\usepackage{mathrsfs}
\usepackage{enumitem}
\usepackage{fancyhdr}
\usepackage{hyperref}

\pagestyle{fancy}
\fancyhf{}
\rhead{MAT 315}
\lhead{Assignment 7}
\rfoot{Page \thepage}

\setlength\parindent{24pt}
\renewcommand\qedsymbol{$\blacksquare$}

\DeclareMathOperator{\F}{\mathbb{F}}
\DeclareMathOperator{\T}{\mathcal{T}}
\DeclareMathOperator{\V}{\mathcal{V}}
\DeclareMathOperator{\U}{\mathcal{U}}
\DeclareMathOperator{\Prt}{\mathbb{P}}
\DeclareMathOperator{\R}{\mathbb{R}}
\DeclareMathOperator{\N}{\mathbb{N}}
\DeclareMathOperator{\Z}{\mathbb{Z}}
\DeclareMathOperator{\Q}{\mathbb{Q}}
\DeclareMathOperator{\C}{\mathbb{C}}
\DeclareMathOperator{\ep}{\varepsilon}
\DeclareMathOperator{\identity}{\mathbf{0}}
\DeclareMathOperator{\card}{card}
\newcommand{\suchthat}{;\ifnum\currentgrouptype=16 \middle\fi|;}

\newtheorem{lemma}{Lemma}

\newcommand{\Mod}[1]{\ \mathrm{mod}\ (#1)}
\newcommand{\tr}{\mathrm{tr}}
\newcommand{\ra}{\rightarrow}
\newcommand{\lan}{\langle}
\newcommand{\ran}{\rangle}
\newcommand{\norm}[1]{\left\lVert#1\right\rVert}
\newcommand{\inn}[1]{\lan#1\ran}
\newcommand{\ol}{\overline}
\begin{document}
\noindent Q3a: Since $f_3 = \gcd(f_1,f_2)$, there must exist some $u,v$ with $f_3 = v \cdot f_1 + u\cdot f_2$ evaluating at $\alpha$ we get that $$f_3(\alpha) = u(\alpha)f_1(\alpha) + v(\alpha)f_2(\alpha) = 0\cdot u(\alpha) + 0\cdot v(\alpha) =0$$
\newline \\ Q3b: We define $M = \{ deg(p(x)) : p(\alpha)=0\}$. This is non-empty by definition, since $\alpha$ is algebraic. It is also bounded below by the property of the degrees of polynomials. Hence by the well ordering principle, there exists a minimal element corresonding to some polynomial $m_{\alpha}(x)$. \\ Claim 1: $m_\alpha$ is irreducible. 
\newline \\ pf: Suppose not. Then there exist some non constant polynomials, $u(x),v(x)$ such that $u(x)\cdot v(x)=m_\alpha(x)$ By definition of $m_{\alpha}$, we know that $$0=m_{\alpha}=u(\alpha)\cdot v(\alpha)$$ So either $u(\alpha)=0$ or $v(\alpha)=0$. Since $u,v$ are assumed to be non constant, we have that $deg(u)<deg(m_\alpha)$ and $deg(v)<deg(m_\alpha)$. This contradicts minimality of $m_\alpha \qed$ 
\\ \newline Claim 2: Uniqueness of $m_\alpha$ 
\newline \\ pf: Suppose that there exists some other $v_\alpha$ of minimal degree with $v_\alpha(\alpha)=0$. Let $m_{\alpha} = \sum_{k=0}^n m_k x^k$ and $v_\alpha = \sum_{k=0}^n v_k x^k$, with $v_n,m_n \neq 0$. Note that any linear combination of $m_\alpha$ and $v_\alpha $ will also vanish at $\alpha$. With this in mind observe the following: 
$$v_\alpha(x) + (-v_n\cdot m_n^{-1})m_{\alpha}(x) = \sum_{k=0}^{n-1} v_{k} - \frac{v_n}{m_n}m_k x^k$$ By construction, this polynomial has degree strictly less than that of $v_\alpha,m_\alpha$ and will also vanish at $\alpha$, contradicting minimailty of $m_\alpha\qed$
\newline Note that WLOG we can always rescale $m_\alpha$ to be monic. 
\\ \newline Claim 3: For any $f$ which vanishes at $\alpha$ is divisible by $m_\alpha$. 
\newline pf: Consider $\gcd(f,m_\alpha)= q(x)$. By previously proven results, we have that $q(x)|m_{\alpha}(x)$. Hence $deg(q(x))\leq deg(m_\alpha(x))$. By part $3a$ we have that $q(\alpha)=0$ and from uniqueness and minimality of $m_\alpha(x)$ we have that $q(x) = m_{\alpha}(x)$ and hence $m_\alpha(x)|f(x) \qed$
\\ \newline Q3c: From basic calculus we have that such an $r$ satisfying $r^3 = m$ must be an irrational number. By 3b, if $q$ is a polynomial where $q(r)=0$, then $x^3 -m |q(x)$. Hence it is sufficient to show that $q(x)$ can not be of degree 1 or 2. If $q$ is degree 1, then for some $a,b \in \Q$ we have that $$q(r) = ar + b = 0 \implies r =- \frac{b}{a}$$ contradicting that $r$ is irrational. 
If $q$ is of degree 2, then we have that $x^3-r^3 = (x-r) (x^2+xr+r^2)$. So $q=x^2+xr+r^2$. These coefficients are not in $\Q$ so no such polynomial can exist. 
\\ \newline Q3d: Consider the $m=2,n=9$. It is clear that there is no integer satisfying $k^9=2$. We see that $2^\frac{1}{9}$ is indeed a root of $x^9-2$, but it will also be the root of the polynomial $x^3-2^{\frac{1}{3}}$
\end{document}