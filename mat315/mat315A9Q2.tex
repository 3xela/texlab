\documentclass[letterpaper]{article}
\usepackage[letterpaper,margin=1in,footskip=0.25in]{geometry}
\usepackage[utf8]{inputenc}
\usepackage{amsmath}
\usepackage{amsthm}
\usepackage{amssymb, pifont}
\usepackage{mathrsfs}
\usepackage{enumitem}
\usepackage{fancyhdr}
\usepackage{hyperref}

\pagestyle{fancy}
\fancyhf{}
\rhead{MAT 315}
\lhead{Assignment 9}
\rfoot{Page \thepage}

\setlength\parindent{24pt}
\renewcommand\qedsymbol{$\blacksquare$}

\DeclareMathOperator{\T}{\mathcal{T}}
\DeclareMathOperator{\V}{\mathcal{V}}
\DeclareMathOperator{\U}{\mathcal{U}}
\DeclareMathOperator{\Prt}{\mathbb{P}}
\DeclareMathOperator{\R}{\mathbb{R}}
\DeclareMathOperator{\N}{\mathbb{N}}
\DeclareMathOperator{\Z}{\mathbb{Z}}
\DeclareMathOperator{\Q}{\mathbb{Q}}
\DeclareMathOperator{\C}{\mathbb{C}}
\DeclareMathOperator{\ep}{\varepsilon}
\DeclareMathOperator{\identity}{\mathbf{0}}
\DeclareMathOperator{\card}{card}
\newcommand{\suchthat}{;\ifnum\currentgrouptype=16 \middle\fi|;}

\newtheorem{lemma}{Lemma}

\newcommand{\bd}{\partial}
\newcommand{\tr}{\mathrm{tr}}
\newcommand{\ra}{\rightarrow}
\newcommand{\lan}{\langle}
\newcommand{\ran}{\rangle}
\newcommand{\norm}[1]{\left\lVert#1\right\rVert}
\newcommand{\inn}[1]{\lan#1\ran}
\newcommand{\ol}{\overline}
\begin{document}
\noindent Q2ai: First note that by previous results, we have that $|M^\times|= p^{\deg(s(x))}-1$. Hence the order of any element can not exceed $p^{\deg(s(x))}-1$. Since we have that $M^\times$ is cyclic there must exist some element of order exactly $p^{\deg(s(x))}-1$. 
\newline \\ 2aii: Let $\alpha$ be a root of $\Phi_{p^m-1}(x)$. Then by HW8Q2a, we have that $\alpha\in M^\times$ and therefore by HW8Q3 we have that the order of $\alpha$ is $p^m-1$. It follows by Lagranges theorem that $p^m-1 |p^{\deg(s(X))}-1$ and clearly 
\newline \\ 2aiii: It has been shown that $\gcd(p^a-1,p^b-1)=p^{\gcd(a,b)}-1$. The euclidean algorithim then gives us that $\gcd(a,b) = \gcd(b,r)$. Hence we have that $$gcd(p^a-1,p^b-1) = p^{\gcd(b,r)}-1 \leq p^{r}-1$$
Since $\gcd(b,r)\leq r$. It follows that if $p^b-1 | p^a-1$, then $\gcd(a,b)= \gcd(b,r)=b$ but by assumption $r<b$. Hence $p^a-1\nmid p^b-1$. Reasoning contrapositively, by above if $m\nmid \deg(s(x))$ then $p^m-1 \nmid p^{\deg(s(x))}-1$. 
\newline \\ Q2iv: Since $alpha$ is a root of $\Phi_{p^m-1}$ it must have an order of $p^m-1$. Therefore, $\alpha^{p^m-1}=1$ and so $\alpha^{p^m}=\alpha$.  
\end{document}