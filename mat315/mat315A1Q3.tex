\documentclass[letterpaper]{article}
\usepackage[letterpaper,margin=1in,footskip=0.25in]{geometry}
\usepackage[utf8]{inputenc}
\usepackage{amsmath}
\usepackage{amsthm}
\usepackage{amssymb, pifont}
\usepackage{mathrsfs}
\usepackage{enumitem}
\usepackage{fancyhdr}
\usepackage{hyperref}

\pagestyle{fancy}
\fancyhf{}
\rhead{MAT 315}
\lhead{Assignment 1}
\rfoot{Page \thepage}

\setlength\parindent{24pt}
\renewcommand\qedsymbol{$\blacksquare$}

\DeclareMathOperator{\U}{\mathcal{U}}
\DeclareMathOperator{\Prt}{\mathbb{P}}
\DeclareMathOperator{\R}{\mathbb{R}}
\DeclareMathOperator{\N}{\mathbb{N}}
\DeclareMathOperator{\Z}{\mathbb{Z}}
\DeclareMathOperator{\Q}{\mathbb{Q}}
\DeclareMathOperator{\C}{\mathbb{C}}
\DeclareMathOperator{\ep}{\varepsilon}
\DeclareMathOperator{\identity}{\mathbf{0}}
\DeclareMathOperator{\card}{card}
\newcommand{\suchthat}{;\ifnum\currentgrouptype=16 \middle\fi|;}

\newtheorem{lemma}{Lemma}

\newcommand{\tr}{\mathrm{tr}}
\newcommand{\ra}{\rightarrow}
\newcommand{\lan}{\langle}
\newcommand{\ran}{\rangle}
\newcommand{\norm}[1]{\left\lVert#1\right\rVert}
\newcommand{\inn}[1]{\lan#1\ran}
\newcommand{\ol}{\overline}
\begin{document}
\noindent Q3a: \\
Let $P(n)$ denote $\sum_{k=1}^n k^3 = \big( \frac{(n)(n+1)}{2} \big)^2$. We proceed by induction on $n$. When $n=1$, $$\sum_{k=1}^1 1^3 = (\frac{(1)(2)}{2})^2 =1$$. Thus $P(1)$ is true. Now suppose the formula holds for $n$. We compute
\begin{align*}
\sum_{k=1}^{n+1} k^3 & = \sum_{k=1}^n k^3 + (n+1)^3
\\ & = \bigg(\frac{(n)(n+1)}{2}\bigg)^2+(n+1)^3 \tag{by induction hypothesis}
\\ & = (n+1)^2\bigg[ \frac{n^2}{4}+ (n+1)\bigg]
\\ & = (n+1)^2 \bigg( \frac{n^2+4n+4}{4} \bigg)
\\ & = (n+1)^2 \frac{(n+2)^2}{4}
\\ & = \bigg(  \frac{(n+1)(n+2)}{2}\bigg)^2
\end{align*}
We have that $P(1)$ is true and $P(n)\implies P(n+1)$. By the principle of induction, $P(n)$ is true for all $n\in N$. 
\\ Q3b: \\
Let $P(n)$ denote $n \leq 2^n$. We proceed by induction on $n$. $P(1)$ corresponds to $1\leq 2$. This is clearly true. Now Suppose that $P(n)$ is true, i.e. $n\leq 2^n$, Multiplying by $2$, we have that $2n \leq 2^{n+1}$. For $n \geq1$, $n+1 \leq 2n$, so we will have $n+1 \leq 2^{n+1} $. We have $P(1)$ true and $P(n)\implies P(n+1)$ so by the principle of induction $P(n)$ is true for all $n$. 
\\ Q3c: \\ Let $P(n) = "\text{n has a binary representation}"$. We proceed by strong induction on $n$. We see that $P(1)=2^0=1$. Now suppose that $P(1)\dots P(n)$ each have binary representations. We will consider 2 separate cases, one in which $n$ is even and one in which $n$ is odd. First, when $n$ is even, we know that $2^0$ is not included in our representation, since it is the only odd power of 2. 
Hence $n+1$ is represented as $n+2^0$. If $n$ is odd, then $n+1$ will be even and will be represented in the following way. $n+1$ even implies $n+1=2k$ for some $k<n$. By assumption $k$ will have a binary representation, say $k= 2^{i_1} + \dots 2^{i_j}$ for unique natural $i_1\dots i_j$. We have that $2k = 2(2^{i_1} + \dots 2^{i_j}) = 2^{i_1 +1} \dots 2^{i_j +1}=n+1$. Since We translate the list $i_1\dots i_j$ by 1 at each point, we preserve the uniqueness. Thus, $n+1$ will have a binary representation. Since $P(1)$ is true, and $P(1)\dots P(n)\implies P(n+1)$, $P(n)$ is true for all $n\in \N$ by the principle of Strong Induction. 
\end{document}