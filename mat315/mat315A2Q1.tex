\documentclass[letterpaper]{article}
\usepackage[letterpaper,margin=1in,footskip=0.25in]{geometry}
\usepackage[utf8]{inputenc}
\usepackage{amsmath}
\usepackage{amsthm}
\usepackage{amssymb, pifont}
\usepackage{mathrsfs}
\usepackage{enumitem}
\usepackage{fancyhdr}
\usepackage{hyperref}

\pagestyle{fancy}
\fancyhf{}
\rhead{MAT 315}
\lhead{Assignment 2}
\rfoot{Page \thepage}

\setlength\parindent{24pt}
\renewcommand\qedsymbol{$\blacksquare$}

\DeclareMathOperator{\U}{\mathcal{U}}
\DeclareMathOperator{\Prt}{\mathbb{P}}
\DeclareMathOperator{\R}{\mathbb{R}}
\DeclareMathOperator{\N}{\mathbb{N}}
\DeclareMathOperator{\Z}{\mathbb{Z}}
\DeclareMathOperator{\Q}{\mathbb{Q}}
\DeclareMathOperator{\C}{\mathbb{C}}
\DeclareMathOperator{\ep}{\varepsilon}
\DeclareMathOperator{\identity}{\mathbf{0}}
\DeclareMathOperator{\card}{card}
\newcommand{\suchthat}{;\ifnum\currentgrouptype=16 \middle\fi|;}

\newtheorem{lemma}{Lemma}

\newcommand{\tr}{\mathrm{tr}}
\newcommand{\ra}{\rightarrow}
\newcommand{\lan}{\langle}
\newcommand{\ran}{\rangle}
\newcommand{\norm}[1]{\left\lVert#1\right\rVert}
\newcommand{\inn}[1]{\lan#1\ran}
\newcommand{\ol}{\overline}
\begin{document}
\noindent Q1a: Since $a|b$, we have for some $k\in \Z$ , $ak=z$, and since $b|c$ for some $l\in \Z$, $lb=c$. Therefore, $c = lb = lka$ and so $a|c$. 
\newline \\ Q1b: Since $a|b$, we have for some $k\in \Z$, $ak=z$, and since $c|d$ we have for some $l\in \Z$, $lc = d$. Therefore, we have $bd = (ka)(lc) = (kl)(ac) $ and so we conclude that $ ac|bd$
\newline \\ Q1c: Let $m\neq 0 $. We note that $a|c \iff ak=b \text{ for some k }\in Z \iff mak = bm \iff ma|mb$
\newline \\ Q1d: Since $d|a$, for some $k$, $dk = a$. Since $a\neq 0$, we have that $k\neq 0$. So thus $$|dk|=|a| \implies |k||d| = |a| \implies |d|\leq |a|$$ With equality holding when $|k|=1$



\end{document}