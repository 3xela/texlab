\documentclass[letterpaper]{article}
\usepackage[letterpaper,margin=1in,footskip=0.25in]{geometry}
\usepackage[utf8]{inputenc}
\usepackage{amsmath}
\usepackage{amsthm}
\usepackage{amssymb, pifont}
\usepackage{mathrsfs}
\usepackage{enumitem}
\usepackage{fancyhdr}
\usepackage{hyperref}

\pagestyle{fancy}
\fancyhf{}
\rhead{MAT 315}
\lhead{Assignment 8}
\rfoot{Page \thepage}

\setlength\parindent{24pt}
\renewcommand\qedsymbol{$\blacksquare$}

\DeclareMathOperator{\T}{\mathcal{T}}
\DeclareMathOperator{\V}{\mathcal{V}}
\DeclareMathOperator{\U}{\mathcal{U}}
\DeclareMathOperator{\Prt}{\mathbb{P}}
\DeclareMathOperator{\R}{\mathbb{R}}
\DeclareMathOperator{\N}{\mathbb{N}}
\DeclareMathOperator{\Z}{\mathbb{Z}}
\DeclareMathOperator{\Q}{\mathbb{Q}}
\DeclareMathOperator{\C}{\mathbb{C}}
\DeclareMathOperator{\ep}{\varepsilon}
\DeclareMathOperator{\identity}{\mathbf{0}}
\DeclareMathOperator{\card}{card}
\newcommand{\suchthat}{;\ifnum\currentgrouptype=16 \middle\fi|;}

\newtheorem{lemma}{Lemma}

\newcommand{\bd}{\partial}
\newcommand{\tr}{\mathrm{tr}}
\newcommand{\ra}{\rightarrow}
\newcommand{\lan}{\langle}
\newcommand{\ran}{\rangle}
\newcommand{\norm}[1]{\left\lVert#1\right\rVert}
\newcommand{\inn}[1]{\lan#1\ran}
\newcommand{\ol}{\overline}
\begin{document}
\noindent Q1a: By theorem 6.7c, it is sufficient to check whether 2 is a primitive root of $p^2$, for $p \in \{3,5,7,11,13,17,19,23\}$. Furthermore, by lemma 6.4, it suffices to show whether or not $2^{\frac{\phi(p^2)}{q}} \neq 1$ in $U_{p^2}$ for $q$ which divide $\phi(p^2)$. We first compute $\phi(3^2) = 6$. This will have prime divisors of $2,3$. Therefore, we have that $2^\frac{6}{2}=2^3=8$ which is not 1. Similarly, we have that $2^\frac{6}{3}=2^2=4$ which is also not 1. Hence $2$ is a generator of $U_{3^e}$. Next we evaluate $\phi(5^2)=20$. We see that $2,5$ are prime and divide $\phi(5^2)$. We see that $2^\frac{20}{2}=2^{10}=-1$ and $2^{\frac{20}{5}} = 2^4=16$. Neither of these are equal to 1 in $U_{5^2}$. Next, we see that $\phi(7^2) = 42$.
The unique prime divisors of $42$ are $2,3,7$. We can verify that $2^{\frac{42}{2}}=2^{21}=1$, so $2$ is not a generator of the group $U_{7^2}$. Next, we compute $\phi(11^2) = 110$. This will have unique prime divisors of 2,5,10. One can easily verify that $2^\frac{110}{2},2^\frac{110}{5},2^\frac{110}{11}$ are not 1 in $U_{11^2}$. Next, we compute $\phi(13^2)=156$. This will have unique prime factors of $2,3,13$. We can easily verify that $2^\frac{156}{2},2^\frac{156}{3},2^\frac{156}{13}$ are not 1 in $U_{13^2}$. Next, we compute $\phi(17^2)=272$. This will have prime divisors of 2 and 17. We can verify that $2^\frac{272}{2}=1$ in $U_{17^2}$. Therefore, 2 is not a generator of $U_{17^2}$. Next, $\phi(19^2)=342$. This will have prime divisors of $2,3,19$. We see that $2^\frac{342}{2},2^\frac{342}{3},2^\frac{342}{19}$ are not 1 in $U_{19^2}$. Finally, we compute $\phi(23^2)=506$. This will have unique prime factors of $2,11,23$. We can check that $2^\frac{506}{2}=1$ in $U_{23^2}$. 
\newline \\ Q1b: First we compute $\phi(18)$ as 6. The only prime divisors are 2 and 3. Thus any primitive root $a$ of $U_{18}$ must satisfy $a^2\neq 1$ and $a^3\neq 1$ in $U_{18}$. We can verify using that the only numbers that satisfy this are $5,11\in U_{18}$. Similarly, for $U_{27}$, we compute $\phi(27)=18$. So any primitive root $a$ of $U_{27}$ must satisfy $a^9\neq 1$ and $a^6\neq 1$. The only solutions to this are $2,5,11,14,20,23 \in U_{27}$. 
\newline \\ Q1ci : If we let $h=g+rp$ for $r\in \{1,2\dots p-1\}$, we have that $h^{p-1}=1-rpg^{p-2}$ by the binomial theorem. Furthermore, since $r$ is coprimewith $p$, we have that $rpg^{p-2}$ is not 0 in $U_{p^2}$. Therefore by lemma 6.2 we have that it is a primitive root. Since this is true for all $r$, we have that there will be $p-1$ primitive roots in $U_{p^2}$ for each root in $U_p$.
\newline \\ Q1cii: We can check that $\phi(25)=20$ which is divisible by 2 and 5. Therefore, we can check that $2^\frac{20}{2}\neq 1 $ and $2^\frac{20}{5}\neq 1$ in $U_{25}$. hence 2 is a primitive root,  
\end{document}