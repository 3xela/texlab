\documentclass[letterpaper]{article}
\usepackage[letterpaper,margin=1in,footskip=0.25in]{geometry}
\usepackage[utf8]{inputenc}
\usepackage{amsmath}
\usepackage{amsthm}
\usepackage{amssymb, pifont}
\usepackage{mathrsfs}
\usepackage{enumitem}
\usepackage{fancyhdr}
\usepackage{hyperref}

\pagestyle{fancy}
\fancyhf{}
\rhead{MAT 315}
\lhead{Assignment 3}
\rfoot{Page \thepage}

\setlength\parindent{24pt}
\renewcommand\qedsymbol{$\blacksquare$}

\DeclareMathOperator{\U}{\mathcal{U}}
\DeclareMathOperator{\Prt}{\mathbb{P}}
\DeclareMathOperator{\R}{\mathbb{R}}
\DeclareMathOperator{\N}{\mathbb{N}}
\DeclareMathOperator{\Z}{\mathbb{Z}}
\DeclareMathOperator{\Q}{\mathbb{Q}}
\DeclareMathOperator{\C}{\mathbb{C}}
\DeclareMathOperator{\ep}{\varepsilon}
\DeclareMathOperator{\identity}{\mathbf{0}}
\DeclareMathOperator{\card}{card}
\newcommand{\suchthat}{;\ifnum\currentgrouptype=16 \middle\fi|;}

\newtheorem{lemma}{Lemma}

\newcommand{\tr}{\mathrm{tr}}
\newcommand{\ra}{\rightarrow}
\newcommand{\lan}{\langle}
\newcommand{\ran}{\rangle}
\newcommand{\norm}[1]{\left\lVert#1\right\rVert}
\newcommand{\inn}[1]{\lan#1\ran}
\newcommand{\ol}{\overline}
\begin{document}
\noindent Q3a: We will prove the contrapositive. If $a$ is odd, then we have that $a^m$ is also odd, for any $m\in N$. Therefore, $a^m+1$ is even, and it has factors. 
Now, if $m$ is a power of 2, we can write it as $m=2^n\cdot q$ for some odd $q$. Consider the polynomial $f(t) = t^q+1$.  This polynomial has a root $t=-1$ and thus it splits. 
Letting $t=x^{2^n}$, we see that $g(x)=f(x^{2^n})=x^m+1$ has a proper factor $x^{2^n}$. Letting $x=a$ we see that $a^{2^n}+1$ is a proper factor of $2^m+1$ thus it cannot be prime. 
\newline \\ Q3b: First suppose that $a>2$. Then we have that $a^m-1 = (a-1)(a^{m-1} + a^{m-2}+ \dots + 1)$. Therefore, $a-1| a^m-1$. Since $(a-1)>1$we have that $a^m-1$ is composite. Therefore $a=2$
Now suppose that $m$ is not prime. Therefore, $m=qp$ for some $q,p\neq 1$. Then we have that 
\begin{align*}
    a^m-1 & = a^{pq}-1
    \\ & = a^{p^q}-1
    \\ & = (a^p-1)(a^{p^{q-1}} + a^{p^{q-2}} + \dots +a^p+1)
\end{align*}
Thus $(a^p-1)|(a^m-1)$. Thus it can not be prime. Therefore if $m>1$ and $a^n-1$ is prime, $a\geq 2$ and $m$ is prime. 
\end{document}