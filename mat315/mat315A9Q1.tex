\documentclass[letterpaper]{article}
\usepackage[letterpaper,margin=1in,footskip=0.25in]{geometry}
\usepackage[utf8]{inputenc}
\usepackage{amsmath}
\usepackage{amsthm}
\usepackage{amssymb, pifont}
\usepackage{mathrsfs}
\usepackage{enumitem}
\usepackage{fancyhdr}
\usepackage{hyperref}

\pagestyle{fancy}
\fancyhf{}
\rhead{MAT 315}
\lhead{Assignment 9}
\rfoot{Page \thepage}

\setlength\parindent{24pt}
\renewcommand\qedsymbol{$\blacksquare$}

\DeclareMathOperator{\T}{\mathcal{T}}
\DeclareMathOperator{\V}{\mathcal{V}}
\DeclareMathOperator{\U}{\mathcal{U}}
\DeclareMathOperator{\Prt}{\mathbb{P}}
\DeclareMathOperator{\R}{\mathbb{R}}
\DeclareMathOperator{\N}{\mathbb{N}}
\DeclareMathOperator{\Z}{\mathbb{Z}}
\DeclareMathOperator{\Q}{\mathbb{Q}}
\DeclareMathOperator{\C}{\mathbb{C}}
\DeclareMathOperator{\ep}{\varepsilon}
\DeclareMathOperator{\identity}{\mathbf{0}}
\DeclareMathOperator{\card}{card}
\newcommand{\suchthat}{;\ifnum\currentgrouptype=16 \middle\fi|;}

\newtheorem{lemma}{Lemma}

\newcommand{\bd}{\partial}
\newcommand{\tr}{\mathrm{tr}}
\newcommand{\ra}{\rightarrow}
\newcommand{\lan}{\langle}
\newcommand{\ran}{\rangle}
\newcommand{\norm}[1]{\left\lVert#1\right\rVert}
\newcommand{\inn}[1]{\lan#1\ran}
\newcommand{\ol}{\overline}
\begin{document}
\noindent Q1a: From previous results, we know that $\Phi_8(x) = x^4+1$. By assumption $s(x)$ is a factor of $\Phi_8(x)$, and $\xi$ is a root of $s(x)$ it must also be a root of $\Phi_8(x)$. Hence $\xi^4+1=0$. Therefore $\xi^4 = -1$, and so $\xi^8=1$. Therefore $\xi$ is nonzero. Hence by Lagranges theorem, the order of the subgroup generated by $\xi$ must must divide 8. Suppose the order is not 8. Then we have that if $m=ord(\xi)$, then $$\xi^k = \xi^{k^{\frac{4}{k}}}+1=1+1=0$$
Hence $2 = 0\mod{p}$, and so p is 2. This is a contradiction, since we assume that $p$ is odd. Therefore, the order of $\xi$ is 8. 
\newline \\ 1b: Let $\tau = \xi+ \xi^{-1}$. Since $\xi^4+1=0$, we have that $\xi^{-2}(\xi^4+1)=0$ and so $\xi^2+\xi^{-2}=0$. Therefore, $$(\xi+\xi^{-1})-2(\xi\cdot \xi^{-1})=0$$ Which implies that $\tau^2=2\cdot 1 = 2$ 
\newline \\ 1c: By Eulers Criterion, we have that $$(\frac{2}{p}) = 2^{\frac{p-1}{2}} = \tau^{2\frac{p-1}{2}} = \tau^{p-1}$$ Where the second equality follows from $1b$. 
\newline \\ 1d: We will use the result that if $p = \pm 1 \mod{8}$, $\tau^p=\tau$, but that we can discard the case when $p = \pm 3\mod{8}$ in our proof. First if $p=\pm 1\mod{8}$, then $$\tau^p = \xi^{8^k}\xi^{\pm 1}+ \xi^{8^k}\xi^{\pm 1} = \xi^{\pm 1} + \xi^{\pm 1} = \tau$$ Now if $p = \pm 3 \mod{8}$, then $$\tau^p = \xi^{8^k}\xi^{\pm 3} + \xi^{8^{-k}}\xi^{\pm 3} = \xi^{\pm 3} + \xi^{\pm 3} =\xi^3 + \xi^{-3} = \xi = \xi^{-1}$$
Thus if $\tau^p=\tau$, $$\xi^6+1 = \xi^4 + \xi^2 \implies \xi^4(\xi^2)+1 = -1+\xi^2$$
Which implies that $2\xi^2=2$, which can not happen since the order is 8. We now prove the result. 
\newline \\ $\implies$
\newline \\ Suppose that $p = \pm 1 \mod{8}$. Then we have that $p=8k+1$, and so $p^2-1= (8k+1)^2-1 = 64k^2+16k$. It is clear that $\frac{p^2-1}{8} = 8k^2+2k$ which is 0 mod $2$. 
\newline \\ $\impliedby$
\newline \\ Suppose that $\frac{p^2-1}{8} = 0\mod{2}$. Therefore $16|p^2-1 = (p-1)(p+1)$.If we have that $4|p-1$ and $4|p+1$. Hence $p\equiv 3\equiv 1\mod{4}$. Which can not be the case. Therefore $8|p+1$ or $8|p-1$, because $2|(p+1)$ and so $p = \pm 1 \mod{4}$. 
\newline \\ Q1e: By $c,d$, we have that $(\frac{2}{p})=1 \iff \frac{p^2-1}{8}\equiv 0 \mod{2}$, or $(\frac{2}{p}) \equiv -1 \mod{2} \equiv \frac{p^2-1}{8}\equiv 1 \mod{2}$. Thus, $(\frac{2}{p}) = (-1)^{\frac{p^2-1}{8}}$   
\end{document}