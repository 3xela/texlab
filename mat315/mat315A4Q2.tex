\documentclass[letterpaper]{article}
\usepackage[letterpaper,margin=1in,footskip=0.25in]{geometry}
\usepackage[utf8]{inputenc}
\usepackage{amsmath}
\usepackage{amsthm}
\usepackage{amssymb, pifont}
\usepackage{mathrsfs}
\usepackage{enumitem}
\usepackage{fancyhdr}
\usepackage{hyperref}

\pagestyle{fancy}
\fancyhf{}
\rhead{MAT 315}
\lhead{Assignment 4}
\rfoot{Page \thepage}

\setlength\parindent{24pt}
\renewcommand\qedsymbol{$\blacksquare$}

\DeclareMathOperator{\T}{\mathcal{T}}
\DeclareMathOperator{\V}{\mathcal{V}}
\DeclareMathOperator{\U}{\mathcal{U}}
\DeclareMathOperator{\Prt}{\mathbb{P}}
\DeclareMathOperator{\R}{\mathbb{R}}
\DeclareMathOperator{\N}{\mathbb{N}}
\DeclareMathOperator{\Z}{\mathbb{Z}}
\DeclareMathOperator{\Q}{\mathbb{Q}}
\DeclareMathOperator{\C}{\mathbb{C}}
\DeclareMathOperator{\ep}{\varepsilon}
\DeclareMathOperator{\identity}{\mathbf{0}}
\DeclareMathOperator{\card}{card}
\newcommand{\suchthat}{;\ifnum\currentgrouptype=16 \middle\fi|;}

\newtheorem{lemma}{Lemma}

\newcommand{\tr}{\mathrm{tr}}
\newcommand{\ra}{\rightarrow}
\newcommand{\lan}{\langle}
\newcommand{\ran}{\rangle}
\newcommand{\norm}[1]{\left\lVert#1\right\rVert}
\newcommand{\inn}[1]{\lan#1\ran}
\newcommand{\ol}{\overline}
\begin{document}
\noindent Q2a: We want to compute $[5^{4100}]_{36}$. We first set $m = \lfloor log_2(4100) \rfloor = 12$. Next, we compute $[r_i] = [5^{2^i}]_{36}$ for $i$ from 1 to $m$. When $i$ is odd, we see that $[r_i]_{36}=25$ and for even $i$, $[r]_{36}=13$. Next, we represent 4100 as a binary number, $4100 =2^{12}+2^2$. Finally, the last step in our algorithm is to solve $[5^{4100}]_{36} = [r_{12}]_{36}\cdot [r_2]_{36} = [25]_{36}$
\newline \\ Q2b: We want to compute $[2^{32790}]_{77}$. We first set $m=\lfloor log_2(32790) \rfloor = 15$. Next we compute $[r_i] = [2^{2^i}]_{77}$ for $i$ from 1 to $m$. We see that $[r_1]_{77}=4$, $[r_2]_{77}=16$, $[r_3]_{77} = 25$ and $[r_4]_{77}=9$. This pattern will repeat as we increase $i$. We now write $32790 = 2^{15} + 2^4 + 2^2 + 2^1$, and by out algorithm we have that $[2^{32790}]_{77} = [r_{15}\cdot r_4 \cdot r_2 \cdot r_1]_{77} = [1]_{77}$. 
\newline \\ Q2c: We can set $m = \lfloor log_2(50) \rfloor = 5$. Then we compute $[(x+1)^{2^i}]_{x^3+x^2+1}$ for $i$ ranging from 1 to $m$. We see that $(x+1)^2\equiv x^2+1$, $(x+1)^{2^2} \equiv x^2+x$ and $(x+1)^{2^3}\equiv x+1$ with the pattern repeating. In binary expansion, we have that $50=2^5+2^4+2^1$ and so we compute $(x+1)^{50}\equiv (x^2+2)(x^2+1)(x^2+1) \equiv x $mod $x^3+x^2+1$
\newline \\ Q2d: Set $m= \lfloor log_2(200) \rfloor = 7$. Then we compute $[x^{2^i}]_{x^3+x+1}$ for $i$ ranging from 1 to $m$. We see $[x^{2^1}]_{x^3+x+1}=x^2$, $[x^{2^2}]_{x^3+x+1}=-x^2-x$, $[x^{2^3}]_{x^3+x+1} = 2x+3$, $[x^{2^4}]_{x^3+x+1}=-x^2+2x-1$, $[x^{2^5}]_{x^3+x+1}=-x$, repeating every 5 i. We know that $200$ has a binary expansion of $200=2^7+2^6+2^3$. Therefore, by our algorithm, $$[x^{200}]_{x^3+x+1} \equiv(-x-x^2)(x^2)(2x+3)\equiv 2x^2+x+1 mod(x^3+x+1)$$
\end{document}