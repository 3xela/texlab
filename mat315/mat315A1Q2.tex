\documentclass[letterpaper]{article}
\usepackage[letterpaper,margin=1in,footskip=0.25in]{geometry}
\usepackage[utf8]{inputenc}
\usepackage{amsmath}
\usepackage{amsthm}
\usepackage{amssymb, pifont}
\usepackage{mathrsfs}
\usepackage{enumitem}
\usepackage{fancyhdr}
\usepackage{hyperref}

\pagestyle{fancy}
\fancyhf{}
\rhead{MAT 315}
\lhead{Assignment 1}
\rfoot{Page \thepage}

\setlength\parindent{24pt}
\renewcommand\qedsymbol{$\blacksquare$}

\DeclareMathOperator{\U}{\mathcal{U}}
\DeclareMathOperator{\Prt}{\mathbb{P}}
\DeclareMathOperator{\R}{\mathbb{R}}
\DeclareMathOperator{\N}{\mathbb{N}}
\DeclareMathOperator{\Z}{\mathbb{Z}}
\DeclareMathOperator{\Q}{\mathbb{Q}}
\DeclareMathOperator{\C}{\mathbb{C}}
\DeclareMathOperator{\ep}{\varepsilon}
\DeclareMathOperator{\identity}{\mathbf{0}}
\DeclareMathOperator{\card}{card}
\newcommand{\suchthat}{;\ifnum\currentgrouptype=16 \middle\fi|;}

\newtheorem{lemma}{Lemma}

\newcommand{\tr}{\mathrm{tr}}
\newcommand{\ra}{\rightarrow}
\newcommand{\lan}{\langle}
\newcommand{\ran}{\rangle}
\newcommand{\norm}[1]{\left\lVert#1\right\rVert}
\newcommand{\inn}[1]{\lan#1\ran}
\newcommand{\ol}{\overline}
\begin{document}
\noindent Q2a: Prove $(2)\iff (2^\prime)$ \\ $"\implies"$ \\
Let $P(n)$ be the statement $1,\dots n \in  B$. P(1) is true since $1 \in B$. If $P(n)$ is true, then we have that $1,\dots n \in B$, so $1\dots n,n+1 \in B$ and so $P(n+1)$ is true. $P(1)$ is true and $P(n)\implies P(n+1)$, so $P(n)$ is true for all $n\in \N$ by (2). Thus $1,\dots n \in B$ for all $n\in N$, so $B = \N$. 
\\ $"\impliedby"$\\ Given $P(1) \dots P(n)$, take $B= \{ n\in \N : P(1) \dots P(n) true \}$. Then we have that $1\in B$, since $P(1)$ is true, and if $1,\dots n \in B$ then $P(1),\dots P(n)$ is true so $P(n+1)$ is true and so $n+1\in B$ hence $B=\N$. So $P(n)$ true for all $n\in N$. 
\\ Q2b: \\ Let $B= \{ n\in \N: P(1)\dots P(n)\text{ true} \}$. Since $P(1)$ is true by assumption, we have that $1\in B$. Since $P(n)\implies P(n+1)$, by our assumption $(2^\prime)$ we know that if $1\dots n\in B$ then $n+1\in B$, so $B=\N$. Thus $P(n)$ is true for all $n\in \N$.
\end{document}
