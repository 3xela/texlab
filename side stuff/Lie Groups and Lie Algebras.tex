\documentclass[12pt, a4paper]{article}
\usepackage[lmargin =0.5 in, 
rmargin=0.5in, 
tmargin=1in,
bmargin=0.5in]{geometry}
\geometry{letterpaper}
\usepackage{tikz-cd}
\usepackage{amsmath}
\usepackage{amssymb}
\usepackage{blindtext}
\usepackage{titlesec}
\usepackage{enumitem}
\usepackage{fancyhdr}
\usepackage{amsthm}
\usepackage{graphicx}
\usepackage{cool}
\usepackage{thmtools}
\usepackage{hyperref}
\graphicspath{ }					%path to an image

%-------- sexy font ------------%
%\usepackage{libertine}
%\usepackage{libertinust1math}

%\usepackage{mlmodern}				% very nice and classic
%\usepackage[utopia]{mathdesign}
%\usepackage[T1]{fontenc}


\usepackage{mlmodern}
\usepackage{eulervm}
%\usepackage{tgtermes} 				%times new roman
%-------- sexy font ------------%


% Problem Styles
%====================================================================%


\newtheorem{problem}{Problem}


\theoremstyle{definition}
\newtheorem{thm}{Theorem}
\newtheorem{lemma}{Lemma}
\newtheorem{prop}{Proposition}
\newtheorem{cor}{Corollary}
\newtheorem{fact}{Fact}
\newtheorem{defn}{Definition}
\newtheorem{example}{Example}
\newtheorem{question}{Question}

\newtheorem{manualprobleminner}{Problem}

\newenvironment{manualproblem}[1]{%
	\renewcommand\themanualprobleminner{#1}%
	\manualprobleminner
}{\endmanualprobleminner}

\newcommand{\penum}{ \begin{enumerate}[label=\bf(\alph*), leftmargin=0pt]}
	\newcommand{\epenum}{ \end{enumerate} }

% Math fonts shortcuts
%====================================================================%

\newcommand{\ring}{\mathcal{R}}
\newcommand{\N}{\mathbb{N}}                           % Natural numbers
\newcommand{\Z}{\mathbb{Z}}                           % Integers
\newcommand{\R}{\mathbb{R}}                           % Real numbers
\newcommand{\C}{\mathbb{C}}                           % Complex numbers
\newcommand{\F}{\mathbb{F}}                           % Arbitrary field
\newcommand{\Q}{\mathbb{Q}}                           % Arbitrary field
\newcommand{\PP}{\mathcal{P}}                         % Partition
\newcommand{\M}{\mathcal{M}}                         % Mathcal M
\newcommand{\eL}{\mathcal{L}}                         % Mathcal L
\newcommand{\T}{\mathcal{T}}                         % Mathcal T
\newcommand{\U}{\mathcal{U}}                         % Mathcal U\\
\newcommand{\V}{\mathcal{V}}                         % Mathcal V

% symbol shortcuts
%====================================================================%

\newcommand{\lam}{\lambda}
\newcommand{\imp}{\implies}
\newcommand{\all}{\forall}
\newcommand{\exs}{\exists}
\newcommand{\delt}{\delta}
\newcommand{\ep}{\varepsilon}
\newcommand{\ra}{\rightarrow}
\newcommand{\vph}{\varphi}

\newcommand{\ol}{\overline}
\newcommand{\f}{\frac}
\newcommand{\lf}{\lfrac}
\newcommand{\df}{\dfrac}

% bracketting shortcuts
%====================================================================%
\newcommand{\abs}[1]{\left| #1 \right|}
\newcommand{\babs}[1]{\Big|#1\Big|}
\newcommand{\bound}{\Big|}
\newcommand{\BB}[1]{\left(#1\right)}
\newcommand{\dd}{\mathrm{d}}
\newcommand{\artanh}{\mathrm{artanh}}
\newcommand{\Med}{\mathrm{Med}}
\newcommand{\Cov}{\mathrm{Cov}}
\newcommand{\Corr}{\mathrm{Corr}}
\newcommand{\tr}{\mathrm{tr}}
\newcommand{\Range}[1]{\mathrm{range}(#1)}
\newcommand{\Null}[1]{\mathrm{null}(#1)}
\newcommand{\lan}{\langle}
\newcommand{\ran}{\rangle}
\newcommand{\norm}[1]{\left\lVert#1\right\rVert}
\newcommand{\inn}[1]{\lan#1\ran}
\newcommand{\op}[1]{\operatorname{#1}}
\newcommand{\bmat}[1]{\begin{bmatrix}#1\end{bmatrix}}
\newcommand{\pmat}[1]{\begin{pmatrix}#1\end{pmatrix}}
\newcommand{\vmat}[1]{\begin{vmatrix}#1\end{vmatrix}}

\newcommand{\amogus}{{\bigcap}\kern-0.8em\raisebox{0.3ex}{$\subset$}}
\newcommand{\Note}{\textbf{Note: }}
\newcommand{\Aside}{{\bf Aside: }}
%restriction
%\newcommand{\op}[1]{\operatorname{#1}}
%\newcommand{\done}{$$\mathcal{QED}$$}

%====================================================================%


\setlength{\parindent}{0pt}      	% No paragraph indentations
\pagestyle{fancy}
\fancyhf{}							% fancy header

\setcounter{secnumdepth}{0}			% sections are numbered but numbers do not appear
\setcounter{tocdepth}{2} 			% no subsubsections in toc

%template
%====================================================================%
%\begin{manualproblem}{1}
%Spivak.
%\end{manualproblem}

%\begin{proof}[Solution]
%\end{proof}

%----------- or -----------%

%\begin{problem} 		
%\end{problem}	

%\penum
%	\item
%\epenum
%====================================================================%


\newcommand{\Course}{MAT IDK }
\newcommand{\hwNumber}{X}

%preamble

\title{Lie Groups and Lie Algebras}
\author{A.N.}
\date{\today}
\lhead{\Course A\hwNumber}
\rhead{\thepage}
%\cfoot{\thepage}


%====================================================================%
\begin{document}
	\maketitle
	\newpage
	\begin{problem}
		Exercise 1.7
\end{problem} For $A\in Sp(n)$, $A$ must belong to $GL(n, \mathbb{H}) \cap O(4n)$. We identify each $x= a+ ib+jc+kd\in \mathbb{H}$ with a $4x4$ real matrix, and a $2x2$ complex matrix in the following way: 
$$x \sim \bmat{a & -d &c & b \\ d&a&-b&c \\ -c&b&a&d \\ -b & -c & -d & a} \sim \bmat{\alpha & - \ol{\beta} \\ \beta& \ol{\alpha}}. $$ Where $\alpha = a+ id, \beta = -c- i b$. Thus we can regard $A$ as an element of $GL(2n, \C)\cap O(2n) = U(2n)$. Conjugation by $J$ flips the order of the columns, and swaps signs This corresponds to conjugation of each $x$. So $\overline{A} = JAJ^{-1}$. Conversely orthogonality of $A$ implies that it can be written as an invertible matrix over quaterions. So it must be in $Sp(n)$. 
\begin{problem}
Exercise 1.8
\end{problem}
Suppose $A\in SO(2m)$. $A$ has determinant $1$, preserves the norm, and hence preserves the inner product. From mat247 we know that there exists a change of basis matrix $O$ so that $OAO^{-1}$ is of the desired form. We claim that $O\in SO(2m)$. Since $\left(OAO^{-1}\right)^T  = \left(OAO^{-1}\right)^{-1}$, and $A\in SO(2m)$ we must have that $O\in SO(2m)$. If $A\in SO(2m+1)$ then $1$ must be a root of the characteristic polynomial, i.e. there is a 1-dim $A$ stable subspace $U$. We can therefore write $\R^{2m} = U\oplus W$ for some even dimensional $W$. Apply the previous argument to $A|_W$. Therefore we have a continuous map which surjects onto $SO(n)$ given by $$f(\theta_1, \dots, \theta_n) = \bmat{R(\theta_1) & 0 & \cdots & 0  \\ 0 & R(\theta_2) &\cdots & 0 \\ 0 & \cdots & \ddots & 0 \\ 0 & 0 & \cdots &  R(\theta_\frac{n}{2}) }$$ for even $n$, and for odd $n$ we make the n'th row and column $0$ except at $(n,n)$ entry of the matrix. Since $f$ is continous and $SO(n)$ is the image of $f$, we have that $SO(n)$ is connected. 
\begin{problem}
	Exercise 1.9
\end{problem}
Since $G$ is a connected manifold, it must be path connected as well. Let $\gamma: [0,1] \to G$ be a path so $\gamma(0)=e$, $\gamma(1)=g$. Then the family $\left\{\gamma(t)U \right\}_{t\in [0,1]}$ gives an open covering of $\gamma([0,1])$. By compactness, we have
\end{document}