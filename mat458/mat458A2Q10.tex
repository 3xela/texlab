\documentclass[letterpaper]{article}
\usepackage[letterpaper,margin=1in,footskip=0.25in]{geometry}
\usepackage[utf8]{inputenc}
\usepackage{amsmath}
\usepackage{amsthm}
\usepackage{amssymb, pifont}
\usepackage{mathrsfs}
\usepackage{enumitem}
\usepackage{fancyhdr}
\usepackage{hyperref}

\pagestyle{fancy}
\fancyhf{}
\rhead{MAT 458}
\lhead{Assignment 2}
\rfoot{Page \thepage}

\setlength\parindent{24pt}
\renewcommand\qedsymbol{$\blacksquare$}

\DeclareMathOperator{\E}{\mathcal{E}}
\DeclareMathOperator{\M}{\mathcal{M}}
\DeclareMathOperator{\F}{\mathbb{F}}
\DeclareMathOperator{\T}{\mathcal{T}}
\DeclareMathOperator{\V}{\mathcal{V}}
\DeclareMathOperator{\U}{\mathcal{U}}
\DeclareMathOperator{\Prt}{\mathbb{P}}
\DeclareMathOperator{\R}{\mathbb{R}}
\DeclareMathOperator{\N}{\mathbb{N}}
\DeclareMathOperator{\Z}{\mathbb{Z}}
\DeclareMathOperator{\Q}{\mathbb{Q}}
\DeclareMathOperator{\C}{\mathbb{C}}
\DeclareMathOperator{\ep}{\varepsilon}
\DeclareMathOperator{\identity}{\mathbf{0}}
\DeclareMathOperator{\card}{card}
\newcommand{\suchthat}{;\ifnum\currentgrouptype=16 \middle\fi|;}

\newtheorem{lemma}{Lemma}

\newcommand{\tr}{\mathrm{tr}}
\newcommand{\ra}{\rightarrow}
\newcommand{\lan}{\langle}
\newcommand{\ran}{\rangle}
\newcommand{\norm}[1]{\left\lVert#1\right\rVert}
\newcommand{\inn}[1]{\lan#1\ran}
\newcommand{\ol}{\overline}
\newcommand{\ci}{i}
\newcommand{\X}{\mathfrak{X}}
\begin{document} \noindent 5.5.59: For $f \in \mathcal{H}$, we define $\delta = \inf\{\norm{f-v}: v\in K \}$. 
Let $\{v_n\}$ be a sequence in $K$ so that $\norm{f- v_n} \to \delta$. By the parrelelogram law we have that 
$$2(\norm{v_n-f}^2 + \norm{v_m - f}^2) = \norm{v_n-v_m}^2 + \norm{v_n+v_m -2f}^2. $$
By conevexity we have that $\frac{1}{2}(v_n+v_m) \in K$ for all $n,m$, therefore we get that 
$$\norm{y_n-y_m}^2 = 2\norm{v_n-f}^2 + 2\norm{v_m - f}^2 - 4 \norm{\frac{1}{2}(v_n+v_m) - f}^2 \leq 2\norm{v_n - f}^2 + 2\norm{v_m-f}^2 - 4\delta^2.$$
Taking $n,m\to \infty$ we get that the sequence must be cauchy. Therefore it must converge since $K$ is closed in $\mathcal{H}$. Let $u$ be the limit. We claim $u$ is unique. 
If not, let $u_1,u_2$ satisfy minimality. Then, $$2 \norm{f - \frac{1}{2}(u_1+u_2)} \leq \norm{f-u_1} + \norm{f-u_2}.$$
Minimality implies that this is an equality and so $u_1 = u_2$. 
We now claim that $\inn{f-u,v-u}\leq 0 $ for all $v\in K$. 
\end{document}