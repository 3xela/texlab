\documentclass[12pt, a4paper]{article}
\usepackage[lmargin =0.5 in, 
rmargin=0.5in, 
tmargin=1in,
bmargin=0.5in]{geometry}
\geometry{letterpaper}
\usepackage{tikz-cd}
\usepackage{amsmath}
\usepackage{amssymb}
\usepackage{blindtext}
\usepackage{titlesec}
\usepackage{enumitem}
\usepackage{fancyhdr}
\usepackage{amsthm}
\usepackage{graphicx}
\usepackage{cool}
\usepackage{thmtools}
\usepackage{hyperref}
\graphicspath{ }					%path to an image

%-------- sexy font ------------%
%\usepackage{libertine}
%\usepackage{libertinust1math}

%\usepackage{mlmodern}				% very nice and classic
%\usepackage[utopia]{mathdesign}
%\usepackage[T1]{fontenc}


\usepackage{mlmodern}
\usepackage{eulervm}
%\usepackage{tgtermes} 				%times new roman
%-------- sexy font ------------%


% Problem Styles
%====================================================================%


\newtheorem{problem}{Problem}


\theoremstyle{definition}
\newtheorem{thm}{Theorem}
\newtheorem{lemma}{Lemma}
\newtheorem{prop}{Proposition}
\newtheorem{cor}{Corollary}
\newtheorem{fact}{Fact}
\newtheorem{defn}{Definition}
\newtheorem{example}{Example}
\newtheorem{question}{Question}

\newtheorem{manualprobleminner}{Problem}

\newenvironment{manualproblem}[1]{%
	\renewcommand\themanualprobleminner{#1}%
	\manualprobleminner
}{\endmanualprobleminner}

\newcommand{\penum}{ \begin{enumerate}[label=\bf(\alph*), leftmargin=0pt]}
	\newcommand{\epenum}{ \end{enumerate} }

% Math fonts shortcuts
%====================================================================%

\newcommand{\ring}{\mathcal{R}}
\newcommand{\N}{\mathbb{N}}                           % Natural numbers
\newcommand{\Z}{\mathbb{Z}}                           % Integers
\newcommand{\R}{\mathbb{R}}                           % Real numbers
\newcommand{\C}{\mathbb{C}}                           % Complex numbers
\newcommand{\F}{\mathbb{F}}                           % Arbitrary field
\newcommand{\Q}{\mathbb{Q}}                           % Arbitrary field
\newcommand{\PP}{\mathcal{P}}                         % Partition
\newcommand{\M}{\mathcal{M}}                         % Mathcal M
\newcommand{\eL}{\mathcal{L}}                         % Mathcal L
\newcommand{\T}{\mathcal{T}}                         % Mathcal T
\newcommand{\U}{\mathcal{U}}                         % Mathcal U\\
\newcommand{\V}{\mathcal{V}}                         % Mathcal V

% symbol shortcuts
%====================================================================%

\newcommand{\lam}{\lambda}
\newcommand{\imp}{\implies}
\newcommand{\all}{\forall}
\newcommand{\exs}{\exists}
\newcommand{\delt}{\delta}
\newcommand{\ep}{\varepsilon}
\newcommand{\ra}{\rightarrow}
\newcommand{\vph}{\varphi}

\newcommand{\ol}{\overline}
\newcommand{\f}{\frac}
\newcommand{\lf}{\lfrac}
\newcommand{\df}{\dfrac}

% bracketting shortcuts
%====================================================================%
\newcommand{\abs}[1]{\left| #1 \right|}
\newcommand{\babs}[1]{\Big|#1\Big|}
\newcommand{\bound}{\Big|}
\newcommand{\BB}[1]{\left(#1\right)}
\newcommand{\dd}{\mathrm{d}}
\newcommand{\artanh}{\mathrm{artanh}}
\newcommand{\Med}{\mathrm{Med}}
\newcommand{\Cov}{\mathrm{Cov}}
\newcommand{\Corr}{\mathrm{Corr}}
\newcommand{\tr}{\mathrm{tr}}
\newcommand{\Range}[1]{\mathrm{range}(#1)}
\newcommand{\Null}[1]{\mathrm{null}(#1)}
\newcommand{\lan}{\langle}
\newcommand{\ran}{\rangle}
\newcommand{\norm}[1]{\left\lVert#1\right\rVert}
\newcommand{\inn}[1]{\lan#1\ran}
\newcommand{\op}[1]{\operatorname{#1}}
\newcommand{\bmat}[1]{\begin{bmatrix}#1\end{bmatrix}}
\newcommand{\pmat}[1]{\begin{pmatrix}#1\end{pmatrix}}
\newcommand{\vmat}[1]{\begin{vmatrix}#1\end{vmatrix}}

\newcommand{\amogus}{{\bigcap}\kern-0.8em\raisebox{0.3ex}{$\subset$}}
\newcommand{\Note}{\textbf{Note: }}
\newcommand{\Aside}{{\bf Aside: }}
%restriction
%\newcommand{\op}[1]{\operatorname{#1}}
%\newcommand{\done}{$$\mathcal{QED}$$}

%====================================================================%


\setlength{\parindent}{0pt}      	% No paragraph indentations
\pagestyle{fancy}
\fancyhf{}							% fancy header

\setcounter{secnumdepth}{0}			% sections are numbered but numbers do not appear
\setcounter{tocdepth}{2} 			% no subsubsections in toc

%template
%====================================================================%
%\begin{manualproblem}{1}
%Spivak.
%\end{manualproblem}

%\begin{proof}[Solution]
%\end{proof}

%----------- or -----------%

%\begin{problem} 		
%\end{problem}	

%\penum
%	\item
%\epenum
%====================================================================%


\newcommand{\Course}{MAT458 }
\newcommand{\hwNumber}{8}

%preamble

\title{MAT458 Solution Set}
\author{A.N.}
\date{\today}
\lhead{\Course A\hwNumber}
\rhead{\thepage}
%\cfoot{\thepage}


%====================================================================%
\begin{document}
\maketitle
\newpage
\begin{problem}
	Folland 5.3.42
\end{problem}
\penum
\item Let $f\in E_n$. That is for some $x_0\in [0,1]$ we have that $|f(x_0)- f(x)| \leq n|x_0-x|$. By Stone-Weierstrass, we can uniformly approximate $f$ with some piecewise linear function $h$ with slope $\pm 2n$. Thus for any $\ep>0$ we have an $h$ so that $\norm{f-h}_u < \ep $. We claim that $h\not \in E_n$. For any $x\in [0,1]$, $$\frac{|h(x_0) - h(x)|}{|x-x_0|} \geq 2n \implies |h(x_0) - h(x)|\geq 2n|x-n_0|  > n|x-x_0|.$$ Therefore $h$ is not in $E_n$, and so $E_n$ is nowhere dense in $C^0$. 
\item The countable union $E = \bigcup_{n=1}^\infty E_n$ is nowhere dense in $C^0$. It follows that the set $C^0 \setminus E$  is residual, and nonempty. Since $E$ is the set of all somewhere differentiable functions, the compliment must be set of nowhere differentiable functions. 
\epenum 
\begin{problem}
	Folland 5.3.27
\end{problem}
Let $x_n$ be an enumeration of the rationals. Define: $$E_n = \bigcup_{k=1}^\infty \left(x_k - \frac{1}{2^{k-1} n} , x_k - \frac{1}{2^{k-1} n}\right).$$ We have that $m(E_n) = \frac{1}{n}$. It follows from measure continuity that $m\left( \cap_{n=1}^\infty E_n\right)=0$. Since each $E_n$ is dense in $\R$, so is $E$ by Baire Category theorem. The compliment is nowhere dense. Take the compliement of $E$ as our desired set. 
\begin{problem}
	Extra Credit: 
\end{problem}
\begin{problem}
	Folland 5.3.32
\end{problem}
Consider the identity mapping $I : \left( \mathfrak{X}, \norm{\cdot }_1\right) \to \left( \mathfrak{X}, \norm{\cdot}_2 \right)$. $I$ is bijective, and continuous by assumption. It follows that the inverse is bounded by Folland Cor 5.11. So there exists some constant $C$ so that $\norm{\cdot}_2 \leq C \norm{\cdot}_1$. 
\begin{problem}
	(Extra Credit) Folland 5.3.33
\end{problem}
Suppose that there exists a sequence $\{a_n\}$, $a_i\geq 0$ so that $\sum_{n}a_n |c_n|< \infty$ if and only if $\{c_n\}$ is bounded. Define $T: B\left(	\N\right) \to L^1(\mu)$ as $Tf(n) = a_nf(n)$. We first claim that $\{g_n\}$ so that $g_n$ is nonzero for finitely many $n$ is dense in $L^1(\mu)$. Given some $h(n)\in L^1(\mu)$, for any $\ep>0$ there is some $N$ so that $\sum_{n\geq N}|h(n)| <\ep$. Define $$g = \begin{cases} h(n) & n< N \\ 0 & n\geq N \end{cases}.$$  Then, $\sum_{n\in \N} |h(n) - g(n)| = \sum_{n\geq N} |h(n)| < \ep$. Note however this family of functions is not dense in $B(\N)$, since if we take a constant sequence of $1$, then $|f(n) - 1|_1 = \infty$. By the uniform boundedness principle, we have that $\norm{Tf(n)} < \infty$ for all $n$, so $\sup_{n}\norm{Tf(n)} < \infty$. There exists some $c$ so that $a_nf(n) =Tf(n) \leq C$. Clearly, we can modify $f(n)$ so that this inequality breaks however. 
\begin{problem}
	Folland 5.3.37
\end{problem}
Let $\{x_n\}$ be a sequence converging to $x$. Let $\lim_{n\to \infty } Tx_n = y$. By continuity of linear functionals, we have that $f(Tx_n) \to f(Tx)$. We also have that $f(Tx_n) \to f(y)$. Since continuous linear functionals seperate points, we have $y=Tx$. By the closed graph theorem $T$ is bounded. 
\begin{problem}
	Folland 5.3.38
\end{problem}
Note that $T$ is linear. It remains to show that it is continuous. By the uniform boundedness principle, we there exists some constant $C$ so that $\sup_{n}\norm{T_n} \leq C$. Note that the following chain of inequalities holds: 
$$\norm{Tx} \leq \sup_{n} \norm{T_nx} \leq C\norm{x}.$$
Therefore $T$ is bounded and thus continuous. 
\begin{problem}
	Folland 5.3.39
\end{problem}
Let $B: \mathfrak{X}\times \mathfrak{Y} \to \mathfrak{Z}$ be a separately continuous linear map. By bilinearity, it is enough to show that there exists a constant $C$ so that $\norm{B(x,y)} \leq C \norm{x} \cdot \norm{y}$. Let $B_x(y) = B(x,y), B_y(x) = B(x,y)$. Since $B$ is seperately continuous, we have that there exists some $C_x,D_y$ so that $ \norm{B_x(y) }\leq C_x \norm{y}$,$ \norm{B_y(x)} \leq D_y \norm{x}$.  By uniform boundedness principle, there exist $C,D$ so that $\norm{B_y(x)} \leq C \norm{x}$ and $\norm{B_x(y)} \leq D\norm{y}$ for all $x,y$ respectively. Thus we have that $\sup_{\norm{x}, \norm{y} = 1} \norm{B(x,y)}\leq \max{(C,D)}$. Thus $B$ is continuous.
\begin{problem}
	Folland 5.3.40 (Principle of Condensation of Singularities)
\end{problem}
Suppose not. Then $\sup_k \left\{ \norm{T_{jk}x}: j\in \N \right\}< \infty$ for all $x\in \mathfrak{X}$. By uniform boundedness we have that $\sup_{k} \left\{\norm{T_{jk}} : j\in \N \right\}< \infty$. This contradicts the assumption. 
\begin{problem}
	Folland 5.4.47
\end{problem}
\penum
\item 
Suppose that $T_n \to T$ weakly. That is for all $f\in \mathfrak{Y}^\ast$, we have $fT_n x \to fTx$. Therefore $\sup_{n} \norm{f T_nx} < \infty$ for all $f$. By hahn banach, take $g$ so that $|g| = 1$, so $$\norm{T_n x} = |g(T_n x)|\leq \sup_{f\in \mathfrak{Y}^\ast, \norm{f} = 1} |f(T_nx)| =\sup_{f\in \mathfrak{Y}^\ast, \norm{f} = 1} |\hat{x} \circ (T_n^\ast \circ f)| \leq \sup_{f\in \mathfrak{Y}^\ast, \norm{f} = 1} |\hat{x} \circ T_n^\ast| \leq \norm{\hat{x} \circ T_n^\ast} < \infty.$$ By uniform boundedness princple, we have that $\sup_{n}\norm{\hat{x}\circ T_n^\ast} < \infty$. Since if $T_n\to T$ strongly implies weakly, we are done. 
\item Let $\{x_n\}$ be a weakly convergent sequence in $\mathfrak{X}$. That is for all $f\in \mathfrak{X}^\ast$, $f(x_n) \to f(x)$. Therefore $\norm{f(x_n)} \to \norm{f(x)}$. We also have that $\hat{x}_n(f) \to \hat{x}(f)$, with norms converging to $\norm{f(x)}$. Thus we have that $\sup_n \norm{\hat{x}_n(f)}<\infty$. Therefore $\norm{\hat{x}} = \norm{x} < \infty$. Now let $\{f_n\}$ be a sequence in $\mathfrak{X}$ converging to $f$ in the weark star topology i.e. $f_n(x) \to f(x)$ for all $x$. Thus we have that $$\sup_n \norm{f_n(x)} = \sup_n \norm{\hat{x}(f_n)}<\infty,$$ by convergence. Therefore taking $\norm{\hat{x}} =1$ we have that $f_n$ is bounded. 
\epenum
\begin{problem}
Folland 5.4.48
\end{problem}
\penum
\item Let $\{x_n\}$ be a sequence in $B$ converging to some $x\in B$. For any $f\in \mathfrak{X}$, $\norm{f(x_n)} = \norm{\hat{x}_n(f)} \leq \norm{\hat{x}_n}\leq 1$ by theorem $5.8d$. Therefore $\norm{\hat{x}(f)} = \norm{f(x)}\leq 1$. 
\item Let $E$ be a bounded set in $\mathfrak{X}$. Let $\inn{x_\alpha}$ be a net in $E$ converging to $x$ , and for $f\in \mathfrak{X}$, $f(x_\alpha)\to f(x)$. We have that $$\sup_{\alpha} \norm{f(x_\alpha)} = \sup_{\alpha} \norm{\hat{x}_\alpha(f)} = \sup_{\alpha}\norm{\hat{x}_\alpha}= \sup_{\alpha} \norm{x_\alpha}< \infty. $$
\item Let $F$ be a bounded subset of $\mathfrak{X}^\ast$. Let $\inn{f_\alpha}$ be a net in $F$ converging to $f$. Then for all $\norm{x}=1$ we have $$\sup_\alpha \norm{f_\alpha(x)} <C \implies \lim_{\alpha\to \infty} \norm{f_\alpha} < \infty. $$ 
\item Let $\inn{f_\alpha}$ be a net so that $\inn{f_i-f_j}\to 0$. Then for sufficiently large $n,m$ we have $\norm{f_n(x) - f_m(x)}\to 0$. So $\{f_n(x)\}$ is a cauchy sequence. It converges to some $f$ by 5.3.38. 
\epenum
\begin{problem}
	Folland 5.4.49
\end{problem}
\penum
\item It is sufficient to show that any element of the basis is unbounded. Elements of the basis take the form $$U_{f,\ep}(x)  = \{y\in \mathfrak{X} : |f(x)-f(y)| < \ep\}.$$ Taking any $v\in f^{-1}(0)$, we have that $x+ \alpha v \in U_{f, \ep}(x)$ for all scalars $\alpha$. Thus this set is unbounded. For the weak $\ast$ topology, the basis elements take the form $$V_{f, \ep} = \{g\in \mathfrak{X} : \norm{f-g} < \ep\}. $$ It is sufficient to show that these sets are unbounded. For all $f\in V_{f, \ep}$, $$\sup_{\norm{x} = 1} \norm{f(x) - g(x)}  = \sup_{\norm{x} = 1} \hat{x}(f-g) < \ep.$$
Takeing any $l$ so that $\hat{x}(l) = 0$ we have that $f+ \alpha l\in V_{f, \ep}$ for all $\alpha$. Thus this set is unbounded. 
\item If $E$ is a bounded subset of $\mathfrak{X}$, then so is its weak closure by 5.4.48b. By part a we have that the interiour must be empty. The same result follows for $F\subset \mathfrak{X}^\ast$ bounded by 5.4.48c and a. 
\item Let $E_n = \{x\in \mathfrak{X} : \norm{x} \leq n\}$. Each $E_n$ is nowhere dense in weak topology, and $\mathfrak{X} = \bigcup_{n} E_n$. So $\mathfrak{X}$ is meager in the weak topology. The result for $\mathfrak{X}^\ast$ is obtained in the exact same way. 
\item IDK ask rob 
\epenum
\begin{problem}
	Folland 5.4.50
\end{problem}
Let $\{x_n\}, \{q_n\}$ be enumerations of the dense subsets of $B,\Q$ respectively. Take $f_n \in \mathfrak{X}$ so that $f_x(x_n) = q_n$. Let $\ep>0$. Take $V_{f_n, \ep}$ as defined earlier. We claim that $\{V_{f_n, \ep}\}$ is a covering of $B^\ast$. Let $\norm{g} \leq  1$. Then at some $x\in B$, $g$ attains a maximum since $g$ bounded. Then, 
$$\norm{g(x) - f(x)} \leq \norm{g(x_n) - g(x) } + \norm{f(x_n) - f(x)} + \norm{f(x_m) - f(x_n)} + \norm{f(x_m)  - f(x)}.$$ Since the norms of all the operators are $1$, taking $m,n$ sufficiently large we can make each term less than $\frac{\ep}{4}$. Therefore we have a countable covering of $B^\ast$ by basis elements. Hence it is second countable. Since it is compact and Hausdorff, it must be metrizable by topology results. 
\begin{problem}
	Folland 5.4.51
\end{problem}
Let $\mathfrak{Y}\subset \mathfrak{X}$ be a vector subspace. Let $\{x_n\}$ be a sequence in $\mathfrak{Y}$ so that $\norm{x_n - x}\to 0$ implies $x\in \mathfrak{Y}$. Let $f\in \mathfrak{X}$. Then we have that $\norm{f(x_n) - f(x)}\leq C_f \norm{x_n -x}\to 0$. Conversely suppose that for all $f\in \mathfrak{X}^\ast$, $f(x_n) \to f(x)$, for $\{x_n\} \subset \mathfrak{Y}$ and $\norm{x_n - x}\to 0$. We claim that $x\in \mathfrak{Y}$. By theorem $5.8$ we can take $f$ so that $f|_{\mathfrak{Y}} = 0$. We have that $f(x) = 0$ and so we are done. 
\begin{problem}
Folland 5.4.52
\end{problem}
\penum
\item 
\epenum
\begin{problem}
	Folland 5.5.56
\end{problem}
The smallest closed subspace that contains $E$ is by definition $\overline{span(E)}$. We claim that $\overline{span(E)} = E^{\perp \perp}$.  First suppose that $v\in \ol{span(E)}$. Then there exists some sequence $\{v_n\}$ converging to $v$. For all $u\in E^\perp$, we have that $$\inn{v_n, u} = 0, $$ so $v_n\in E^{\perp \perp}$, and by continuity of the inner product, $v\in E^{\perp \perp }.$
Now suppose that $\{v_n\}$ is a sequence in $E^{\perp \perp}$ converging to some $v$. Then for all $u\in \ol{span(E)}$ we have that $$\inn{u,v_n}=0$$ and so $\{v_n\} \subset \ol{span(E)}$. By contiuity we have that $v\in \ol{span(E)}$. 
\begin{problem}
Folland 5.5.57
\end{problem}
\penum
\item
	We claim that $T^\ast = V^{-1} T^\dag V$. We see that it satisfies $$\inn{x, T^\ast y} = \inn{x, V^{-1} T^\dag V y} = \left(V V^{-1} T^\dag Vy \right)(x) = (T^\dag Vy)(x) = (Vy)(Tx) = \inn{Tx, y}. $$
We now claim uniqueness holds. If $S^\ast, T^\ast$ both satisfy the equality, then we have that
$$\inn{Tx, y} = \inn{x, T^\ast y} = \inn{x, S^\ast y}   \implies \inn{x, T^\ast - S^\ast y} = 0,\forall x,y \implies T^\ast = S^\ast.  $$
\item We first claim that $T^{\ast \ast} = T$. Notice that if $T^\ast$ satisfies $\inn{Tx, y} = \inn{x, T^\ast y}$, then we must have that 
$$\inn{T^\ast x, y} = \ol{\inn{y, T^\ast x}} = \ol{\inn{Ty, x }} = \inn{x, Ty}. $$
We now claim that $$\norm{T^\ast} = \norm{T}.$$ Observe that for any $x$, $$\norm{T^\ast} =\sup_{\norm{y}=1} \norm{T^\ast y} = \sup_{\norm{y} = \norm{x} = 1} \left|\inn{x, T^\ast y} \right| = \sup_{\norm{y} = \norm{x} = 1}|\inn{Tx,y}| = \norm{T}. $$
Next, we have that $$\inn{(aS+bT)x, y} = \inn{aSx, y} + \inn{bTx, y} = \inn{x, \ol{a} S^\ast y} + \inn{x, \ol{b} T^\ast y} = \inn{x, \ol{a}S^\ast +\ol{b}T^\ast y}.  $$
Finally, $$\inn{STx,y} = \inn{Tx, S^\ast y} = \inn{x, T^\ast S^\ast y}. $$
\item We first show that $R(T)^\perp= N(T^\ast)$. Let $y\in R(T)^\perp$. Then for all $x\in \mathcal{H}$,  we have $$0 = \inn{Tx, y} = \inn{x, T^\ast y}\implies T^\ast y = 0 . $$
	Conversely, if $y\in N(T^\ast)$, we have that for all $x\in \mathcal{H}$,  $$0 =\inn{x, T^\ast y} = \inn{Tx, y} \implies y\in R(T)^\perp. $$
	Next we claim that $N(T)^\perp = \ol{R(T^\ast)}$. Suppose for some $x\in N(T)$, and for all $y\in \mathcal{H}$, we have that $$0= \inn{T x, y} = \inn{x, T^\ast y} \implies \ol{R(T^\ast)} \subset N(T)^\perp. $$ Conversely, let $x\in N(T)^\perp$. Then we have that 
$$0 = \inn{Tx, y} = \inn{x, T^\ast y} \implies x\in \ol{R(T)}.$$
\item Suppose that $T$ is unitary. Then it must also be invertible. We claim that $T^\ast = T^{-1}$. We have that $$\inn{x,y} =\inn{Tx, Ty} = \inn{x, T^\ast Ty} \implies 0 = \inn{x, (T^\ast T -I)y}\implies T^\ast T = I. $$ Therefore we have that $T^{-1} = T^\ast$.
	Conversely suppose that $T$ is invertible with $T^{-1} = T^\ast$. Then we have that $$\inn{x,x} = \inn{x, T^\ast Tx} = \inn{Tx,Tx}. $$ By the polarization identity we have that $$\inn{Tx,Ty} = \inn{x,y}. $$
\epenum
\begin{problem}
	Folland 5.5.58
\end{problem}
\penum
\item First note that by definition of $P$, we have that $\inn{Px-x, Px} =0$. This implies that $\inn{Px,Px} = \inn{x,Px}.$ By Cauchy-Schwartz's inequality, we have that
$$\norm{Px}^2 = \inn{x,Px} \leq\norm{x} \cdot \norm{Px} \implies \norm{Px} \leq \norm{x}. $$
Therefore $\norm{P}\leq 1$, so $P\in \mathcal{L}(\mathcal{H}, \mathcal{H})$. We now claim that $P^\ast = P$. Note that $\inn{Px-x,Px}=0 = \inn{Px,Px-x}$ which implies that $\inn{Px,x} = \inn{x,Px}$. Therefore we have $$\inn{P^\ast x - Px, x}= \inn{P^\ast x, x} - \inn{Px, x} = \inn{x, P x} - \inn{Px,x} = 0.  $$
Since this holds for all $x$ we have that $P^\ast =P$. 
Finally, we have that $$\inn{Px,Px} = \inn{x,P^2x} = \inn{x,Px}\implies P^2 = P. $$
We next claim that $N(P) = M^\perp$ and $R(P) = M$. Note that by definition, $Px\in M$. Now if $y\in M$, since $y-Py\in M^\perp \cap M$ we have that $Py=y$. Therefore $R(P) = M$ and so by $57c)$ we must have $N(P)= M^\perp$. 
\item First suppose that $\{x_n\}\subset R(P)$ with limit $x$ and that $P$ satisfies the definition of a projection. We will have that $x\in R(P)$ if $Px =x$. We have that $$\lim_{n\to \infty} \inn{Px_n - x_n, x_n} = 0,  \forall n \implies \inn{Px-x,x} = 0$$ 
Therefore $R(P)$ is closed. We now claim that for all $x$, $\inn{Px-x, Px}=0$. Using the properties of $P$, we see that 
$$\inn{Px-x, Px} = \inn{P^2x - Px , x } = 0 \forall x. $$
Therefore such $P$ must be a projection. 
\item 
We claim that $Px = \sum_\alpha \inn{x, u_\alpha} u_\alpha$ satisfies $\inn{Px-x, Px} = 0$. This is clearly a continuous operator, so by the previous result we can conclude that $P$ is indeed a projection. 
\begin{align*} 
\inn{Px-x, Px}
	&= \inn{\sum_\alpha \inn{x,u_\alpha} u_\alpha - x, \sum_\alpha \inn{x, u_\alpha} u_\alpha} 
\\ & = \inn{\sum_\alpha \inn{x,u_\alpha} u_\alpha, \sum_\alpha \inn{x,u_\alpha} u_\alpha} - \inn{x, \sum_\alpha \inn{x,u_\alpha} u_\alpha}
	\\& =\sum_\alpha \inn{x,u_\alpha}^2 - \sum_\alpha\inn{x,u_\alpha}^2
	\\ & = 0
\end{align*}
As desired. 
\epenum
\begin{problem}
	Folland 5.5.59
\end{problem}
Assume that $0 \not \in K$. Let $\delta = \inf_{v\in K} \norm{v}$. Let $\{v_n\}$ be a sequence so that $\norm{v_n}\to \delta$. By closedness of $K$, we have that the limit $v\in K$. We now claim uniqueness of $v$. Suppose that $u$ is another vector that attains minimal norm. 
Test. 
\end{document}

