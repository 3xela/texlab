\documentclass[12pt, a4paper]{article}
\usepackage[lmargin =0.5 in, 
rmargin=0.5in, 
tmargin=1in,
bmargin=0.5in]{geometry}
\geometry{letterpaper}
\usepackage{tikz-cd}
\usepackage{amsmath}
\usepackage{amssymb}
\usepackage{blindtext}
\usepackage{titlesec}
\usepackage{enumitem}
\usepackage{fancyhdr}
\usepackage{amsthm}
\usepackage{graphicx}
\usepackage{cool}
\usepackage{thmtools}
\usepackage{hyperref}
\graphicspath{ }					%path to an image

%-------- sexy font ------------%
%\usepackage{libertine}
%\usepackage{libertinust1math}

%\usepackage{mlmodern}				% very nice and classic
%\usepackage[utopia]{mathdesign}
%\usepackage[T1]{fontenc}


\usepackage{mlmodern}
\usepackage{eulervm}
%\usepackage{tgtermes} 				%times new roman
%-------- sexy font ------------%


% Problem Styles
%====================================================================%


\newtheorem{problem}{Problem}


\theoremstyle{definition}
\newtheorem{thm}{Theorem}
\newtheorem{lemma}{Lemma}
\newtheorem{prop}{Proposition}
\newtheorem{cor}{Corollary}
\newtheorem{fact}{Fact}
\newtheorem{defn}{Definition}
\newtheorem{example}{Example}
\newtheorem{question}{Question}

\newtheorem{manualprobleminner}{Problem}

\newenvironment{manualproblem}[1]{%
	\renewcommand\themanualprobleminner{#1}%
	\manualprobleminner
}{\endmanualprobleminner}

\newcommand{\penum}{ \begin{enumerate}[label=\bf(\alph*), leftmargin=0pt]}
	\newcommand{\epenum}{ \end{enumerate} }

% Math fonts shortcuts
%====================================================================%

\newcommand{\ring}{\mathcal{R}}
\newcommand{\N}{\mathbb{N}}                           % Natural numbers
\newcommand{\Z}{\mathbb{Z}}                           % Integers
\newcommand{\R}{\mathbb{R}}                           % Real numbers
\newcommand{\C}{\mathbb{C}}                           % Complex numbers
\newcommand{\F}{\mathbb{F}}                           % Arbitrary field
\newcommand{\Q}{\mathbb{Q}}                           % Arbitrary field
\newcommand{\PP}{\mathcal{P}}                         % Partition
\newcommand{\M}{\mathcal{M}}                         % Mathcal M
\newcommand{\eL}{\mathcal{L}}                         % Mathcal L
\newcommand{\T}{\mathcal{T}}                         % Mathcal T
\newcommand{\U}{\mathcal{U}}                         % Mathcal U\\
\newcommand{\V}{\mathcal{V}}                         % Mathcal V

% symbol shortcuts
%====================================================================%

\newcommand{\lam}{\lambda}
\newcommand{\imp}{\implies}
\newcommand{\all}{\forall}
\newcommand{\exs}{\exists}
\newcommand{\delt}{\delta}
\newcommand{\ep}{\varepsilon}
\newcommand{\ra}{\rightarrow}
\newcommand{\vph}{\varphi}

\newcommand{\ol}{\overline}
\newcommand{\f}{\frac}
\newcommand{\lf}{\lfrac}
\newcommand{\df}{\dfrac}

% bracketting shortcuts
%====================================================================%
\newcommand{\abs}[1]{\left| #1 \right|}
\newcommand{\babs}[1]{\Big|#1\Big|}
\newcommand{\bound}{\Big|}
\newcommand{\BB}[1]{\left(#1\right)}
\newcommand{\dd}{\mathrm{d}}
\newcommand{\artanh}{\mathrm{artanh}}
\newcommand{\Med}{\mathrm{Med}}
\newcommand{\Cov}{\mathrm{Cov}}
\newcommand{\Corr}{\mathrm{Corr}}
\newcommand{\tr}{\mathrm{tr}}
\newcommand{\Range}[1]{\mathrm{range}(#1)}
\newcommand{\Null}[1]{\mathrm{null}(#1)}
\newcommand{\lan}{\langle}
\newcommand{\ran}{\rangle}
\newcommand{\norm}[1]{\left\lVert#1\right\rVert}
\newcommand{\inn}[1]{\lan#1\ran}
\newcommand{\op}[1]{\operatorname{#1}}
\newcommand{\bmat}[1]{\begin{bmatrix}#1\end{bmatrix}}
\newcommand{\pmat}[1]{\begin{pmatrix}#1\end{pmatrix}}
\newcommand{\vmat}[1]{\begin{vmatrix}#1\end{vmatrix}}

\newcommand{\amogus}{{\bigcap}\kern-0.8em\raisebox{0.3ex}{$\subset$}}
\newcommand{\Note}{\textbf{Note: }}
\newcommand{\Aside}{{\bf Aside: }}
%restriction
%\newcommand{\op}[1]{\operatorname{#1}}
%\newcommand{\done}{$$\mathcal{QED}$$}

%====================================================================%


\setlength{\parindent}{0pt}      	% No paragraph indentations
\pagestyle{fancy}
\fancyhf{}							% fancy header

\setcounter{secnumdepth}{0}			% sections are numbered but numbers do not appear
\setcounter{tocdepth}{2} 			% no subsubsections in toc

%template
%====================================================================%
%\begin{manualproblem}{1}
%Spivak.
%\end{manualproblem}

%\begin{proof}[Solution]
%\end{proof}

%----------- or -----------%

%\begin{problem} 		
%\end{problem}	

%\penum
%	\item
%\epenum
%====================================================================%


\newcommand{\Course}{MAT458 }
\newcommand{\hwNumber}{8}

%preamble

\title{a}
\author{A.N.}
\date{\today}
\lhead{\Course A\hwNumber}
\rhead{\thepage}
%\cfoot{\thepage}


%====================================================================%
\begin{document}
\begin{problem}
Folland	8.4.30
\end{problem}
Take any $x$ in the Lebesgue set of $f$. Then by fatou's lemma, we have 
$$\lim_{t\to 0} f^t(x) \geq \int \lim \inf _{t\to 0} \hat{f} (\xi)\Phi(t\xi) e^{2\pi i \xi \cdot x} d\xi = \int \hat{f} d\xi$$(Done with Robbert Liu)
\newpage 
\begin{problem}
	Folland 8.4.26
\end{problem}
\penum
\item Taking $\vph(x)$ as in $8.37$, then by the inverse fourier transform we get that 
$$e^{-\beta} = \int \frac{1}{\pi(1+t^2)}e^{-i\beta t}dt $$
\item 
We verify that the equality holds by the following computation: 
\begin{align*}
	\int_{-\infty}^\infty \frac{1}{\pi }\frac{1}{1+t^2} e^{-i\beta t} & = \int_{-\infty}^\infty \frac{e^{-i\beta t}}{\pi} \int_0^\infty e^{-(1+t^2)s}ds dt \tag{by hint}
	\\ & = \int_{0}^\infty \int_{-\infty }^\infty \frac{1}{\pi} e^{-i\beta t} e^{-(1+t^2)s}dt ds \tag{By Fubini-Tonelli}
	\\ & =  \int_{0}^\infty \int_{-\infty }^\infty \sqrt{\pi}e^{-s}e^{-i\beta \sqrt{\pi}x}e^{-\pi x^2s}dx ds \tag{By substituting $x = \frac{t}{\sqrt{\pi}}$}
	\\ & = \int_0^\infty \sqrt{\pi}e^{-s} \int_{-\infty}^\infty e^{\frac{-i\beta x 2\pi}{2\sqrt{\pi}}}e^{-\pi x^2s}
	\\ & = \int_0^\infty \sqrt{\pi}e^{-s} \int_{-\infty}^\infty e^{-2\pi i \xi} e^{-\pi x^2 s}dx ds 
	\\ & = \int_0^\infty (\pi s)^{-1/2} e^{-s}e^{\frac{-\beta^2}{4s}} ds
\end{align*}
\item By the previous results, 
\begin{align*}
	e^{-2\pi |\xi|} & = \int_0^\infty (\pi s)^{-1/2} e^{-s} e^{-\frac{\pi^2|\xi|^2}{s}}ds
	\\  & = \int_0^\infty (\pi s)^{-1/2} e^{-s} \int_\R \left( \frac{s}{\pi}\right)^{n/2}e^{-s|x|^2}e^{-2\pi i \xi \cdot x} dxds
	\\ & = \int_\R \int_0^\infty (\pi s)^{-1/2} e^{-s} \left( \frac{s}{\pi} \right)^{n/2} e^{-s|x|^2}e^{-2\pi i \xi \cdot x}ds dx
	\\ & = \int_{\R}e^{-2\pi i \xi \cdot x} \left( \int_0^\infty (s)^{\frac{n-1}{2}} e^{-\pi s} e^{-\pi s |x|^2} ds\right) dx 
	\\ & = \int_{\R}e^{-2\pi i \xi \cdot x} \frac{\Gamma \left(\frac{1}{2}(n+1) \right)}{\pi^{(n+1)/2}} \left(1 +|x|\right)^{-\frac{n+1}{2}} dx
\end{align*}
Which is exactly what we wanted to show. 
(Done with Robbert Liu)

\epenum
\newpage 
\begin{problem}
	Folland 8.4.28
\end{problem}
\penum
\item The following computation verifies what we wish to show:
\begin{align*}
	f\ast P_r(x) &= P_r\ast f(x) 
	\\ & = \int f(y) P_r(x-y)dy
	\\ & = \int f(y) \sum_{k\in \Z} r^{|k|}e^{2\pi i(x-y)} dy 
	\\ & = \sum_{k\in \Z} r^{|k|} e^{2\pi i x}\int_{\R} f(y)e^{-2\pi i y}dy
	\\ & = \sum_{k\in \Z} r^{|k|} e^{-2\pi i kx} \hat{f}(k)
\end{align*} 
\item We compute $P_r(x)$ as 
\begin{align*}
	P_r(x) & = \sum_{-\infty }^\infty r^{|k|}e^{2\pi i kx}
	\\ & = 1+ \sum_{1}^\infty r^{|k|}e^{2\pi i kx} + \sum_{-\infty}^{-1} r^{|k|} e^{2\pi i kx}
	\\ & = 1+ \sum_{1}^\infty \left( re^{2\pi i x}\right)^k + \sum_{1}^\infty \left(re^{-2\pi ix }\right)^k
	\\ & = 1 + \left(-1 + \frac{1}{1- re^{2\pi i x}}	\right) + \left(-1 + \frac{1}{1-re^{-2\pi i x}}\right)
	\\ & = -1 + \frac{1}{1 - re^{2\pi i x}} + \frac{1}{1- re^{2\pi i x}}
	\\ & = \frac{- \left(1- re^{2\pi i x}\right) \left(1 - re^{-2\pi i x}\right) + 1 - re^{2\pi i x} + 1 - re^{2\pi i x} }{\left(1 - re^{2\pi i x}\right) \left( 1- re^{-2\pi i x} \right)}
	\\ & = \frac{1-r^2}{1+r^2 - 2r\cos(2\pi x)}
\end{align*}

\epenum

\newpage
\begin{problem}
	Folland 8.4.31
\end{problem}
Using 8.37, and the dilation formula, we compute that $$\frac{\pi}{a} \cdot \frac{ 1+ e^{-2\pi a}}{1 - e^{-2\pi a}} = \frac{\pi}{a} \left(\sum_1^\infty e^{-2\pi a k} + \sum_{0}^\infty e^{-2\pi a k }  \right) = \frac{\pi}{a} \sum_{-\infty}^\infty e^{-2\pi a|k|} = \sum_{-\infty }^\infty \frac{1}{(k^2+a^2)}.$$ Therefore the following series of equalities hold: 
\begin{align*}
	2 \sum_{k=1}^\infty \frac{1}{k^2 + a^2} &= \frac{\pi}{a} \cdot \frac{1+e^{-2\pi a}}{1 - e^{-2\pi a}} - \frac{1}{a^2}
	\\ & = \frac{a\pi \left(1+ e^{-2\pi a}\right) - \left(1 - e^{2\pi a}\right)}{a^2\left(1 - e^{-2\pi a}\right)}
\end{align*}
To take the limit as $a\to 0$, we apply L'hoptals rule 3 times. The third derivative of the numerator with respect to $a$ is: $$4\pi^3 e^{-2\pi a} + 4\pi^3e^{-2\pi a}  + 4\pi^3e^{-2\pi a} + 8\pi^4ae^{-2\pi a} - 8\pi^3e^{- 2\pi a },$$ and the third derivative of the denominator is $$4\pi e^{-2\pi a}(2\pi^2 a^2 -6\pi a + 3).$$ At $a=0$ we have the quotient equal to $\frac{\pi^2}{3}$ and we conclude that $$2\sum_{1}^\infty \frac{1}{k^2} = \frac{\pi^2}{3}.$$
(Done with Charles Swaney)
\newpage 
\begin{problem}
	Folland 8.5.33
\end{problem}
\penum
\item We compute the convolution as: 
\begin{align*}
	f\ast F_m (x) & = \int f(y)F_m(x-y)dy 
	\\ & = \int f(y) \frac{1}{m+1} \sum_0^m \sum_{-k}^k e^{2\pi i k(x-y)}dy
	\\ & = \sum_{-k}^k \int f(y) \frac{m+1- |k|}{m+1} e^{-2\pi ik y} e^{2\pi ik x}dy \tag{by counting like terms }
	\\ & = \sum_{-k}^k e^{2\pi i kx}\frac{m+1-|k|}{m+1} \hat{f}(y)
\end{align*}
\item We compute $F_m$ as 
\begin{align*}
	F_m &= \frac{1}{m+1} \sum_{k=0}^m D_k
	\\ & = \frac{1}{m+1}\sum_{k=0}^m \frac{e^{(2k+1) \pi i x } - e^{-(2k+1) \pi i x}}{2i}
	\\ & = \frac{1}{m+1}\sum_{k=0}^m \frac{e^{(2k+1) \pi i x } - e^{-(2k+1) \pi i x}}{2i} \cdot \frac{e^{(2k+1) \pi i x } + e^{-(2k+1) \pi i x}}{e^{(2k+1) \pi i x } + e^{-(2k+1) \pi i x}}
	\\ & = \frac{1}{m+1} \sum_{k=0}^m \frac{{e^{(2k+1)\pi i x}}^2 - {e^{-(2k+1)\pi i x}}^2}{2i\left(e^{(2k+1) \pi i x } + e^{-(2k+1) \pi i x} \right)}
	\\ & = \frac{\sin^2([m+1]\pi x)}{(m+1)\sin^2(\pi x)} \tag{ by telescopic summations}
\end{align*}
\epenum
\newpage
\begin{problem}
	Folland 8.5.34
\end{problem}
Using the closed form for the Dirichlet kernel, we compute the limit as $m\to \infty$ of  $\norm{D_m}_1$ as: 
\begin{align*}
	\lim_{m\to \infty }\norm{D_m}_1 & = \lim_{m\to \infty }\int_{\R} \left| \frac{\sin([2m+1] \pi x)}{\sin(\pi x)}\right| dx
	\\ &= \lim_{m\to \infty} \int \left|\frac{\sin(y)}{\sin(y/\pi(2m+1))} \right|\frac{1}{\pi(2m+1)} dy \tag{substituting $y = \pi(2m+1)x$}
	\\ & = \int \left|\frac{\sin(y)}{y}\right|dy \tag{by Monotone Convergence Theorem}
	\\ & = \infty \tag{Mat157}
\end{align*}
\newpage 
\begin{problem}
Folland 8.5.35
\end{problem}
\penum
\item First we show that $\phi_m$ is linear. We compute that $$\phi_m(f+g) = S_m(f+g)(0) = \sum_{-m}^m \hat{f} + \hat{g} (k) = \sum_{-m}^m \hat{f}(k) + \sum_{-m}^m \hat{g}(k) = \phi_m(f) + \phi_m(g), $$ and $$\phi_m(\alpha f) = S_m(\alpha f(0)) = \alpha \sum_{-m}^m \hat{f}(k) = \alpha \phi_m(f). $$ We now claim that $\phi_m$ is continuous. For $|f|_u< \ep$, we have $$\norm{\phi_m(f)} = \norm{\sum_{-m}^m \hat{f}(k) } \leq \sum_{-m}^m \norm{\hat{f}} = 2m \norm{\hat{f}} = 2m \int_{\mathbb{T}} \left|f(x)e^{-2\pi i kx} \right| dx \leq 2m \ep m^\ast(\mathbb{T}). $$ 
Therefore $\phi_m \in C(\mathbb{T})^\ast$. We also have that by Young's Inequality that $\norm{\phi_m} \leq \norm{D_m}$, and that this equality will be attained when we take $f = 1$. Thus $\norm{\phi_m} = \norm{D_m}$. 
\item Suppose that the set of all such $f$ is not meager. Then by Uniform Boundedness Principle we have that the sequence $\sup_{f}\{S_mf(0)\}<\infty$. This contradicts part $8.5.34$ since as $m\to \infty$, $\norm{D_m} \to \infty$ and by part $a)$ so does $\norm{\phi_m}$. 
\item First note that the result from $b)$ holds for any $x\in T$, since if we replace $0$ with any point $x$ we still have that $\left|e^{2\pi i xk} \right| =1$. Then, by the Principle of Condensation of Singularities (Folland 5.3.40) there is a residual subset of $C(\mathbb{T})$ so that $\{S_mf(x)\}$ diverges on a dense subset of $\mathbb{T}$. 
\epenum
(Done with Petar Jovasevic)
	
	
\end{document}