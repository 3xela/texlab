\documentclass[letterpaper]{article}
\usepackage[letterpaper,margin=1in,footskip=0.25in]{geometry}
\usepackage[utf8]{inputenc}
\usepackage{amsmath}
\usepackage{amsthm}
\usepackage{amssymb, pifont}
\usepackage{mathrsfs}
\usepackage{enumitem}
\usepackage{fancyhdr}
\usepackage{hyperref}

\pagestyle{fancy}
\fancyhf{}
\rhead{MAT 458}
\lhead{Assignment 2}
\rfoot{Page \thepage}

\setlength\parindent{24pt}
\renewcommand\qedsymbol{$\blacksquare$}

\DeclareMathOperator{\E}{\mathcal{E}}
\DeclareMathOperator{\M}{\mathcal{M}}
\DeclareMathOperator{\F}{\mathbb{F}}
\DeclareMathOperator{\T}{\mathcal{T}}
\DeclareMathOperator{\V}{\mathcal{V}}
\DeclareMathOperator{\U}{\mathcal{U}}
\DeclareMathOperator{\Prt}{\mathbb{P}}
\DeclareMathOperator{\R}{\mathbb{R}}
\DeclareMathOperator{\N}{\mathbb{N}}
\DeclareMathOperator{\Z}{\mathbb{Z}}
\DeclareMathOperator{\Q}{\mathbb{Q}}
\DeclareMathOperator{\C}{\mathbb{C}}
\DeclareMathOperator{\ep}{\varepsilon}
\DeclareMathOperator{\identity}{\mathbf{0}}
\DeclareMathOperator{\card}{card}
\newcommand{\suchthat}{;\ifnum\currentgrouptype=16 \middle\fi|;}

\newtheorem{lemma}{Lemma}

\newcommand{\tr}{\mathrm{tr}}
\newcommand{\ra}{\rightarrow}
\newcommand{\lan}{\langle}
\newcommand{\ran}{\rangle}
\newcommand{\norm}[1]{\left\lVert#1\right\rVert}
\newcommand{\inn}[1]{\lan#1\ran}
\newcommand{\ol}{\overline}
\newcommand{\ci}{i}
\newcommand{\X}{\mathfrak{X}}
\begin{document} 5.5.58a: We have as given that $$ \inn{Px-x , Px} =\inn{Px, Px-x} =0.$$
Together with linearity this implies that for all $x$, $$\inn{Px,x} = \inn{x,Px}.$$ Thus we have that $P^\ast = P$. Furthermore, linearity implies that 
$$\inn{Px,Px} = \inn{P^2x,x} = \inn{Px,x}.$$ Therefore $P^2 = P.$
\newline \\ 5.5.58b: We first claim that $R(P)$ is closed. Let $\{x_n\}$ be a sequence in $R(P)$, that converges to some $x$. Then we have that $Px_n = x_n$, and so $$\inn{Px_n - x_n, x_n} = 0. $$ Inner product continuity implies that $$\inn{Px - x, x} = 0.$$ Therefore $Px=x$ and so $x\in R(P).$ Thus $R(P)$ is closed. We now verify that $\inn{Px-x, Px} =0$. Since $P^2 = P = P^\ast$.
$$\inn{Px-x,Px} = \inn{Px-x, P^2x} = \inn{P^2x - Px, Px} = 0.$$
\newline \\ 5.5.58c: It is enough to check that $Px = \sum\inn{x,u_\alpha} u_\alpha$ satisfies $\inn{Px-x, Px} = 0$.
\begin{align*} \inn{\sum\inn{x,u_\alpha} u_\alpha - x , \sum\inn{x,u_\alpha} u_\alpha} 
    &= \inn{ \sum\inn{x,u_\alpha} u_\alpha , \sum\inn{x,u_\alpha} u_\alpha} - \inn{x, \sum\inn{x,u_\alpha} u_\alpha} 
    \\ &  = \inn{\sum\inn{x,u_\alpha} u_\alpha, \sum\inn{x,u_\alpha} u_\alpha} - \sum \inn{x,u_\alpha}\cdot \sum\inn{x,u_\alpha} u_\alpha 
    \\ & = 0
\end{align*}
\end{document}