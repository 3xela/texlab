\documentclass[letterpaper]{article}
\usepackage[letterpaper,margin=1in,footskip=0.25in]{geometry}
\usepackage[utf8]{inputenc}
\usepackage{amsmath}
\usepackage{amsthm}
\usepackage{amssymb, pifont}
\usepackage{mathrsfs}
\usepackage{enumitem}
\usepackage{fancyhdr}
\usepackage{hyperref}

\pagestyle{fancy}
\fancyhf{}
\rhead{MAT 458}
\lhead{Assignment 4}
\rfoot{Page \thepage}

\setlength\parindent{24pt}
\renewcommand\qedsymbol{$\blacksquare$}

\DeclareMathOperator{\hil}{\mathcal{H}}
\DeclareMathOperator{\E}{\mathcal{E}}
\DeclareMathOperator{\M}{\mathcal{M}}
\DeclareMathOperator{\F}{\mathbb{F}}
\DeclareMathOperator{\T}{\mathcal{T}}
\DeclareMathOperator{\V}{\mathcal{V}}
\DeclareMathOperator{\U}{\mathcal{U}}
\DeclareMathOperator{\Prt}{\mathbb{P}}
\DeclareMathOperator{\R}{\mathbb{R}}
\DeclareMathOperator{\N}{\mathbb{N}}
\DeclareMathOperator{\Z}{\mathbb{Z}}
\DeclareMathOperator{\Q}{\mathbb{Q}}
\DeclareMathOperator{\C}{\mathbb{C}}
\DeclareMathOperator{\ep}{\varepsilon}
\DeclareMathOperator{\identity}{\mathbf{0}}
\DeclareMathOperator{\card}{card}
\newcommand{\suchthat}{;\ifnum\currentgrouptype=16 \middle\fi|;}

\newtheorem{lemma}{Lemma}

\newcommand{\tr}{\mathrm{tr}}
\newcommand{\ra}{\rightarrow}
\newcommand{\lan}{\langle}
\newcommand{\ran}{\rangle}
\newcommand{\norm}[1]{\left\lVert#1\right\rVert}
\newcommand{\inn}[1]{\lan#1\ran}
\newcommand{\ol}{\overline}
\newcommand{\ci}{i}
\newcommand{\X}{\mathfrak{X}}
\begin{document} \noindent 5a: 
First note that if we substitute $x = \frac{z}{y}$,  $$\int K(xy)x^{-\frac{1}{q}} dx = \int K(z)( \frac{z}{y})^{-\frac{1}{q}} \frac{1}{y}dz = y^{\frac{1}{q} -1} \int K(z) z^{\frac{1}{p}-1} dz = y^{- \frac{1}{p}} \phi(p^{-1}).$$
We now see that \begin{align*}
    \Big( \int K(xy)f(x)dx \Big)^p & \leq \Big( \int K(xy)x^{-\frac{1}{q}} dx \Big)^{\frac{p}{q}} \int x^{\frac{p}{q^2}}K(xy) {f(x)}^p dx \tag{by Holders inequality}
    \\ &=y^{-\frac{1}{q}}\phi(p^{-1})^{\frac{p}{q}}\int x^{\frac{p}{q^2}}K(xy) {f(x)}^p dx \tag{by above}
\end{align*}
Furthermore, we have that 
\begin{align*}
    \int \Big( \int K(xy)f(x)dx \Big)^p dy & \leq \int y^{-\frac{1}{q}} \phi(p^{-1})^\frac{p}{q} dy \int x^{\frac{p}{q^2}}K(xy) {f(x)}^p dx \tag{by tonelli and above}
    \\ & = \phi(p^{-1})^\frac{p}{q} \int \int y^{-\frac{1}{q}} x^{\frac{p^2}{q}}K(xy)f(x)^p dx dy 
    \\ & = \phi(p^{-1})^\frac{p}{q} \int x^{\frac{p^2}{q}}f(x)^p \int K(xy) y^{-\frac{1}{q}} dy dx 
    \\ & = \phi(p^{-1})^p \int x^{p-2} f(x)^p dx 
\end{align*}
Therefore by Holders inequality, 
$$\int \int K(xy)f(x)g(y) dx dy \leq \norm{g}_q \phi(p^{-1}) \int x^{p-2} f(x)^p dx.$$
\newline \\ Q5b: Using 5a, we get that $$\norm{Tf(x)}^2_2 = \int \Big| \int K(xy)f(y)dy \Big|^2 dx \leq \phi(\frac{1}{2})^2 \int |f(x)|^2 dx. $$
Therefore this operator is bounded and maps $L^2$ into $L^2$. 
\end{document}