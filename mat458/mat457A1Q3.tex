\documentclass[letterpaper]{article}
\usepackage[letterpaper,margin=1in,footskip=0.25in]{geometry}
\usepackage[utf8]{inputenc}
\usepackage{amsmath}
\usepackage{amsthm}
\usepackage{amssymb, pifont}
\usepackage{mathrsfs}
\usepackage{enumitem}
\usepackage{fancyhdr}
\usepackage{hyperref}

\pagestyle{fancy}
\fancyhf{}
\rhead{MAT 458}
\lhead{Assignment 1}
\rfoot{Page \thepage}

\setlength\parindent{24pt}
\renewcommand\qedsymbol{$\blacksquare$}

\DeclareMathOperator{\E}{\mathcal{E}}
\DeclareMathOperator{\M}{\mathcal{M}}
\DeclareMathOperator{\F}{\mathbb{F}}
\DeclareMathOperator{\T}{\mathcal{T}}
\DeclareMathOperator{\V}{\mathcal{V}}
\DeclareMathOperator{\U}{\mathcal{U}}
\DeclareMathOperator{\Prt}{\mathbb{P}}
\DeclareMathOperator{\R}{\mathbb{R}}
\DeclareMathOperator{\N}{\mathbb{N}}
\DeclareMathOperator{\Z}{\mathbb{Z}}
\DeclareMathOperator{\Q}{\mathbb{Q}}
\DeclareMathOperator{\C}{\mathbb{C}}
\DeclareMathOperator{\ep}{\varepsilon}
\DeclareMathOperator{\identity}{\mathbf{0}}
\DeclareMathOperator{\card}{card}
\newcommand{\suchthat}{;\ifnum\currentgrouptype=16 \middle\fi|;}

\newtheorem{lemma}{Lemma}

\newcommand{\tr}{\mathrm{tr}}
\newcommand{\ra}{\rightarrow}
\newcommand{\lan}{\langle}
\newcommand{\ran}{\rangle}
\newcommand{\norm}[1]{\left\lVert#1\right\rVert}
\newcommand{\inn}[1]{\lan#1\ran}
\newcommand{\ol}{\overline}
\newcommand{\ci}{i}
\newcommand{\X}{\mathfrak{X}}
\begin{document} \noindent Q3: We claim that there exists some $x\in \R$ satisfying $E \cap E+x = \emptyset.$ 
Define the set $A = \{x: E \cap E+ x = \emptyset \}.$ It is sufficient to show that $A$ is nowhere dense, 
i.e. $A$ closed with empty interiour. First we claim that $A^c$ is open. If $x\in A^c$, then $E\cap E+x  $ is empty. If $A^c$
were not empty then for all $\ep>0$, $B_{\ep}(x)$ is not contained in $A^c$. Since this holds for all $\ep$, we have that $x \not \in A^c$. A contradiction.
Finally it remains to show that the interiour of $A$ is empty. If not then for some $x \in A$ and $\ep >0$, $B_\ep(x) \subset A$. If $y$ is such that $y\in E,E+x$, we have that $y-x\in E$ and so $(y-x-\ep,y-x+\ep) \subset E$. Therefore $m(E)>0$. A contradiction. 
Since translation is a homeomorphism, we apply the previous result, with our homeomorphism being translation. 


\end{document}