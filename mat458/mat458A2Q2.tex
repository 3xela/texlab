\documentclass[letterpaper]{article}
\usepackage[letterpaper,margin=1in,footskip=0.25in]{geometry}
\usepackage[utf8]{inputenc}
\usepackage{amsmath}
\usepackage{amsthm}
\usepackage{amssymb, pifont}
\usepackage{mathrsfs}
\usepackage{enumitem}
\usepackage{fancyhdr}
\usepackage{hyperref}

\pagestyle{fancy}
\fancyhf{}
\rhead{MAT 458}
\lhead{Assignment 2}
\rfoot{Page \thepage}

\setlength\parindent{24pt}
\renewcommand\qedsymbol{$\blacksquare$}

\DeclareMathOperator{\E}{\mathcal{E}}
\DeclareMathOperator{\M}{\mathcal{M}}
\DeclareMathOperator{\F}{\mathbb{F}}
\DeclareMathOperator{\T}{\mathcal{T}}
\DeclareMathOperator{\V}{\mathcal{V}}
\DeclareMathOperator{\U}{\mathcal{U}}
\DeclareMathOperator{\Prt}{\mathbb{P}}
\DeclareMathOperator{\R}{\mathbb{R}}
\DeclareMathOperator{\N}{\mathbb{N}}
\DeclareMathOperator{\Z}{\mathbb{Z}}
\DeclareMathOperator{\Q}{\mathbb{Q}}
\DeclareMathOperator{\C}{\mathbb{C}}
\DeclareMathOperator{\ep}{\varepsilon}
\DeclareMathOperator{\identity}{\mathbf{0}}
\DeclareMathOperator{\card}{card}
\newcommand{\suchthat}{;\ifnum\currentgrouptype=16 \middle\fi|;}

\newtheorem{lemma}{Lemma}

\newcommand{\tr}{\mathrm{tr}}
\newcommand{\ra}{\rightarrow}
\newcommand{\lan}{\langle}
\newcommand{\ran}{\rangle}
\newcommand{\norm}[1]{\left\lVert#1\right\rVert}
\newcommand{\inn}[1]{\lan#1\ran}
\newcommand{\ol}{\overline}
\newcommand{\ci}{i}
\newcommand{\X}{\mathfrak{X}}
\begin{document} \noindent 5.4.48a: Let $\{x_n\} \subset B$ be a sequence converging to some $x$. We claim that for any $f\in \X^\ast$, 
$f(x_n) \to f(x)$. We have that $\norm{f(x_n)} = \norm{\hat{x}_n (f)} \leq 1$ by Theorem 5.8d. Therefore $\norm{\hat{x}(f)} = \norm{f(x)}\leq 1$. As desired. 
\newline \\ 5.4.48b: Let $\inn{x_\alpha}$ be a net in a bounded set $E$. Suppose that $f(x_\alpha)\to f(x)$. Then, we have that $\sup_\alpha \norm{f(x_\alpha)} = \sup_\alpha \norm{\hat{x}_\alpha (f)} = \sup_\alpha\norm{x_\alpha} < \infty.$ 
\newline \\ 5.4.48c: Let $\{f_n\}$ be a sequence in $F\subset \X^\ast$, $F$ bounded,  that weak converges to some $f$ in the weak closure. Then for all $\norm{x} =1$, we have that $$\sup_n \norm{f_n(x)} \leq C$$ for some $C$. Since $\norm{\cdot}$ 
is continuous, we have that $\norm{f} = \norm{\lim_{n\to \infty} f_n} \leq C$ 
\newline \\ 5.4.48.d: Let $\inn{f_\alpha}_{\alpha \in I}$ such that $\inn{f_i-f_j}_{(i,j)\in I^2} \to 0$. We have that for sufficiently large $n,m$, $\norm{f_n(x) - f_m(x)}\to 0$. Therefore $\inn{f_n(x)}$ is a cauchy sequence. Hence it pointwise converges to some $f\in \X^\ast$ by assigntment 1 question 7. 


\end{document}