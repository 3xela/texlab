\documentclass[letterpaper]{article}
\usepackage[letterpaper,margin=1in,footskip=0.25in]{geometry}
\usepackage[utf8]{inputenc}
\usepackage{amsmath}
\usepackage{amsthm}
\usepackage{amssymb, pifont}
\usepackage{mathrsfs}
\usepackage{enumitem}
\usepackage{fancyhdr}
\usepackage{hyperref}

\pagestyle{fancy}
\fancyhf{}
\rhead{MAT 458}
\lhead{Assignment 4}
\rfoot{Page \thepage}

\setlength\parindent{24pt}
\renewcommand\qedsymbol{$\blacksquare$}

\DeclareMathOperator{\hil}{\mathcal{H}}
\DeclareMathOperator{\E}{\mathcal{E}}
\DeclareMathOperator{\M}{\mathcal{M}}
\DeclareMathOperator{\F}{\mathbb{F}}
\DeclareMathOperator{\T}{\mathcal{T}}
\DeclareMathOperator{\V}{\mathcal{V}}
\DeclareMathOperator{\U}{\mathcal{U}}
\DeclareMathOperator{\Prt}{\mathbb{P}}
\DeclareMathOperator{\R}{\mathbb{R}}
\DeclareMathOperator{\N}{\mathbb{N}}
\DeclareMathOperator{\Z}{\mathbb{Z}}
\DeclareMathOperator{\Q}{\mathbb{Q}}
\DeclareMathOperator{\C}{\mathbb{C}}
\DeclareMathOperator{\ep}{\varepsilon}
\DeclareMathOperator{\identity}{\mathbf{0}}
\DeclareMathOperator{\card}{card}
\newcommand{\suchthat}{;\ifnum\currentgrouptype=16 \middle\fi|;}

\newtheorem{lemma}{Lemma}

\newcommand{\tr}{\mathrm{tr}}
\newcommand{\ra}{\rightarrow}
\newcommand{\lan}{\langle}
\newcommand{\ran}{\rangle}
\newcommand{\norm}[1]{\left\lVert#1\right\rVert}
\newcommand{\inn}[1]{\lan#1\ran}
\newcommand{\ol}{\overline}
\newcommand{\ci}{i}
\newcommand{\X}{\mathfrak{X}}
\begin{document} \noindent Q1a: Suppose that $R(T)$ is closed. By contiuity we have that 
we have that $N(T)$. Therefore $E \setminus N(T) \cong R(T)$.
$T$ factors as an isomorphism composed with a projection, $\tilde{T}\circ \pi$. 
Therefore we have that $$d(x, N(T)) \leq \norm{x + N(T)} = \norm{\tilde{T} ^{-1} \tilde{T} (x+N(T))} \leq C \norm{\tilde{T} (x+ N(T))} = C\norm{T(x)}.$$
Now suppose that for some $C$ we have $$d(x,N(T)) \leq C\norm{T(x)}.$$ Let $\{x_n\}$ be a sequence such that ,
$T(x_n) \to y$. Then, $$\norm{x_n - x_m} \leq C \norm{T(x_n) - T(x_m)} \to 0$$ as $n,m \to \infty$. 
Therefore by completeness $x_n \to x$ and so $T(x_n) \to T(x)$. Thus $R(T)$ is closed. 
\newline \\ 1b: Define the mapping $$T: G \times L ,  (x,y) \mapsto x-y.$$ This has kernel $G\cap L$ so we apply $1a$ to conclude that $T(G,L)$ is closed. 
Equivalently we have that $G+L$ is closed.  
 
\end{document}