\documentclass[letterpaper]{article}
\usepackage[letterpaper,margin=1in,footskip=0.25in]{geometry}
\usepackage[utf8]{inputenc}
\usepackage{amsmath}
\usepackage{amsthm}
\usepackage{amssymb, pifont}
\usepackage{mathrsfs}
\usepackage{enumitem}
\usepackage{fancyhdr}
\usepackage{hyperref}

\pagestyle{fancy}
\fancyhf{}
\rhead{MAT 458}
\lhead{Assignment 2}
\rfoot{Page \thepage}

\setlength\parindent{24pt}
\renewcommand\qedsymbol{$\blacksquare$}

\DeclareMathOperator{\E}{\mathcal{E}}
\DeclareMathOperator{\M}{\mathcal{M}}
\DeclareMathOperator{\F}{\mathbb{F}}
\DeclareMathOperator{\T}{\mathcal{T}}
\DeclareMathOperator{\V}{\mathcal{V}}
\DeclareMathOperator{\U}{\mathcal{U}}
\DeclareMathOperator{\Prt}{\mathbb{P}}
\DeclareMathOperator{\R}{\mathbb{R}}
\DeclareMathOperator{\N}{\mathbb{N}}
\DeclareMathOperator{\Z}{\mathbb{Z}}
\DeclareMathOperator{\Q}{\mathbb{Q}}
\DeclareMathOperator{\C}{\mathbb{C}}
\DeclareMathOperator{\ep}{\varepsilon}
\DeclareMathOperator{\identity}{\mathbf{0}}
\DeclareMathOperator{\card}{card}
\newcommand{\suchthat}{;\ifnum\currentgrouptype=16 \middle\fi|;}

\newtheorem{lemma}{Lemma}

\newcommand{\tr}{\mathrm{tr}}
\newcommand{\ra}{\rightarrow}
\newcommand{\lan}{\langle}
\newcommand{\ran}{\rangle}
\newcommand{\norm}[1]{\left\lVert#1\right\rVert}
\newcommand{\inn}[1]{\lan#1\ran}
\newcommand{\ol}{\overline}
\newcommand{\ci}{i}
\newcommand{\X}{\mathfrak{X}}
\begin{document} \noindent 5.5.61: 
We show that the set $\{h_{nm}\}$ satisfies mutual orthongonality and is normalized to $1$. Observe that: 
\begin{align*} \inn{h_{nm}, h_{pq}} & = \int_{X \times Y} \ol{h_{nm}} h_{pq}  d(\mu \times \nu)
    \\ & = \int_{X \times Y} \ol{f_n(x)} \ol{g_m(y)}f_p(x)g_p(y) d(\mu \times \nu)
    \\ & = \int_{X} \int_Y \ol{f_n(x)}f_p(x)\ol{g_m(y)}g_q(y) d\nu d\mu \tag{by Fubini-Tonelli's Theorem}
    \\ & = \Big[\int_{X} \ol{f_n(x)}f_p(x) d\mu \Big] \cdot \Big[ \int_Y \ol{g_m(y)}g_q(y) d\nu\Big]
    \\ & = \delta_{np} \cdot \delta_{mq} \tag{Since $f_i,g_j$ form orthonormal basis}
\end{align*}
This is exactly what we wanted to show. Now we claim that the set $\{h_{nm}\}$ spans $L^2(\mu \times \nu)$.
We check that this satisfies the properties $(a)-(c)$ of theorem $5.27$ of folland. First suppose 
$k\in L^2(\mu \times \nu)$ such that $\inn{k, h_{nm}} =0$ for all $(n,m)$. We have that 
$$ 0= \int_{X \times Y} \ol{k(x,y)} h_{nm} d(\mu \times \nu) = \int_X f_n(x)\Big[ \int_Y \ol{k(x,y)} g_m(y) d\nu \Big]d\mu = \int_Y g_m(y) \Big[ \int_X \ol{k(x,y)} f_n(x) d\mu \Big] d\nu$$
Which implies that for all $m,n$, 
$$ 0 = \int_{Y} \ol{k(x,y)} g_{m}(y) d\nu = \int_X \ol{k(x,y)} f_n(x) d\mu.$$ Letting $k_x(y) = k(x,y)$, and $k_y(x) = k(x,y)$. Using orthonormality we get that $0= k_x(y)=k_y(x) = k(x,y)$. 
We now show that Parseval's Identity holds. By Bessels Identity, it is sufficient to show that $|k|^2 \leq \sum_{n,m} |\inn{k,h_{nm}}|^2$ Let $k\in L^2(\mu \times \nu)$. 
Then, we write $$\sum_{(n,m)} \Big| \inn{k, h_{nm}} \Big|^2 =\sum_{(n,m)} \Big| \int_X f_n(x) \Big[\int_Y \ol{k(x,y)}g_m(y) d\nu \Big] d\mu \Big|^2 = \sum_{(n,m)} \Big| \int_Y g_m(y) \Big[ \int_X \ol{k(x,y)}f_n(x) d\mu\Big] d\nu \Big| $$
Using orthonormality, we get that $$ \Big|\int \ol{k(x,y)} g_m(y) \Big|^2 = \Big|\int \ol{k(x,y)} f_n(x)\Big|^2.$$ 
This implies that $$|k|^2 = \int_{X \times Y} \ol{k(x,y)}k(x,y)\inn{h_{nm}, h_{nm}} d(\mu \times \nu) \leq \sum_{(n,m)} \Big| \inn{k(x,y) , f_n(x)g_m(x)} \Big|^2 $$
From orthonormality of $f,g$. 

\end{document}