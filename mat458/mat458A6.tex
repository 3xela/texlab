\documentclass[12pt, a4paper]{article}
\usepackage[lmargin =0.5 in, 
rmargin=0.5in, 
tmargin=1in,
bmargin=0.5in]{geometry}
\geometry{letterpaper}
\usepackage{tikz-cd}
\usepackage{amsmath}
\usepackage{amssymb}
\usepackage{blindtext}
\usepackage{titlesec}
\usepackage{enumitem}
\usepackage{fancyhdr}
\usepackage{amsthm}
\usepackage{graphicx}
\usepackage{cool}
\usepackage{thmtools}
\usepackage{hyperref}
\graphicspath{ }					%path to an image

%-------- sexy font ------------%
%\usepackage{libertine}
%\usepackage{libertinust1math}

%\usepackage{mlmodern}				% very nice and classic
%\usepackage[utopia]{mathdesign}
%\usepackage[T1]{fontenc}


\usepackage{mlmodern}
\usepackage{eulervm}
%\usepackage{tgtermes} 				%times new roman
%-------- sexy font ------------%


% Problem Styles
%====================================================================%


\newtheorem{problem}{Problem}


\theoremstyle{definition}
\newtheorem{thm}{Theorem}
\newtheorem{lemma}{Lemma}
\newtheorem{prop}{Proposition}
\newtheorem{cor}{Corollary}
\newtheorem{fact}{Fact}
\newtheorem{defn}{Definition}
\newtheorem{example}{Example}
\newtheorem{question}{Question}

\newtheorem{manualprobleminner}{Problem}

\newenvironment{manualproblem}[1]{%
	\renewcommand\themanualprobleminner{#1}%
	\manualprobleminner
}{\endmanualprobleminner}

\newcommand{\penum}{ \begin{enumerate}[label=\bf(\alph*), leftmargin=0pt]}
	\newcommand{\epenum}{ \end{enumerate} }

% Math fonts shortcuts
%====================================================================%

\newcommand{\ring}{\mathcal{R}}
\newcommand{\N}{\mathbb{N}}                           % Natural numbers
\newcommand{\Z}{\mathbb{Z}}                           % Integers
\newcommand{\R}{\mathbb{R}}                           % Real numbers
\newcommand{\C}{\mathbb{C}}                           % Complex numbers
\newcommand{\F}{\mathbb{F}}                           % Arbitrary field
\newcommand{\Q}{\mathbb{Q}}                           % Arbitrary field
\newcommand{\PP}{\mathcal{P}}                         % Partition
\newcommand{\M}{\mathcal{M}}                         % Mathcal M
\newcommand{\eL}{\mathcal{L}}                         % Mathcal L
\newcommand{\T}{\mathcal{T}}                         % Mathcal T
\newcommand{\U}{\mathcal{U}}                         % Mathcal U\\
\newcommand{\V}{\mathcal{V}}                         % Mathcal V

% symbol shortcuts
%====================================================================%

\newcommand{\lam}{\lambda}
\newcommand{\imp}{\implies}
\newcommand{\all}{\forall}
\newcommand{\exs}{\exists}
\newcommand{\delt}{\delta}
\newcommand{\ep}{\varepsilon}
\newcommand{\ra}{\rightarrow}
\newcommand{\vph}{\varphi}

\newcommand{\ol}{\overline}
\newcommand{\f}{\frac}
\newcommand{\lf}{\lfrac}
\newcommand{\df}{\dfrac}

% bracketting shortcuts
%====================================================================%
\newcommand{\abs}[1]{\left| #1 \right|}
\newcommand{\babs}[1]{\Big|#1\Big|}
\newcommand{\bound}{\Big|}
\newcommand{\BB}[1]{\left(#1\right)}
\newcommand{\dd}{\mathrm{d}}
\newcommand{\artanh}{\mathrm{artanh}}
\newcommand{\Med}{\mathrm{Med}}
\newcommand{\Cov}{\mathrm{Cov}}
\newcommand{\Corr}{\mathrm{Corr}}
\newcommand{\tr}{\mathrm{tr}}
\newcommand{\Range}[1]{\mathrm{range}(#1)}
\newcommand{\Null}[1]{\mathrm{null}(#1)}
\newcommand{\lan}{\langle}
\newcommand{\ran}{\rangle}
\newcommand{\norm}[1]{\left\lVert#1\right\rVert}
\newcommand{\inn}[1]{\lan#1\ran}
\newcommand{\op}[1]{\operatorname{#1}}
\newcommand{\bmat}[1]{\begin{bmatrix}#1\end{bmatrix}}
\newcommand{\pmat}[1]{\begin{pmatrix}#1\end{pmatrix}}
\newcommand{\vmat}[1]{\begin{vmatrix}#1\end{vmatrix}}

\newcommand{\amogus}{{\bigcap}\kern-0.8em\raisebox{0.3ex}{$\subset$}}
\newcommand{\Note}{\textbf{Note: }}
\newcommand{\Aside}{{\bf Aside: }}
%restriction
%\newcommand{\op}[1]{\operatorname{#1}}
%\newcommand{\done}{$$\mathcal{QED}$$}

%====================================================================%


\setlength{\parindent}{0pt}      	% No paragraph indentations
\pagestyle{fancy}
\fancyhf{}							% fancy header

\setcounter{secnumdepth}{0}			% sections are numbered but numbers do not appear
\setcounter{tocdepth}{2} 			% no subsubsections in toc

%template
%====================================================================%
%\begin{manualproblem}{1}
%Spivak.
%\end{manualproblem}

%\begin{proof}[Solution]
%\end{proof}

%----------- or -----------%

%\begin{problem} 		
%\end{problem}	

%\penum
%	\item
%\epenum
%====================================================================%


\newcommand{\Course}{MAT458 }
\newcommand{\hwNumber}{6}

%preamble

\title{a}
\author{A.N.}
\date{\today}
\lhead{\Course A\hwNumber}
\rhead{\thepage}
%\cfoot{\thepage}


%====================================================================%
\begin{document}
	\begin{problem} Folland 8.1.3
	\end{problem}
\penum
\item We proceed by induction on $k$. For $k=1$ we have $$\eta^{1}(t) = \frac{1}{t^2}e^{-\frac{1}{t}} = P_1 \left( 1/t \right) \eta(t). $$
Suppose the result holds for $k$. Then, $$\eta^{(k+1)}(t) = \left( P_k\left(1/t \right) \eta(t)\right)^{(1)} = \left(P_k(1/t)\right)^\prime \eta(t) + P_{k}(1/t)P_1(1/t)\eta(t) = \eta(t) \left(P_k(1/t)^\prime + P_k(1/t)P_1(1/t) \right).$$ This is what we wanted to show. 
\item First we claim that $\lim_{t\to 0}\eta^{(1)}(t) = 0$. 
$$\lim_{t\to 0} \eta^{1}(t) = \lim_{t\to 0} \frac{1}{t^2}e^{-1/t} = \lim_{y\to \infty} y^2e^{-y} =0$$ by L'Hopitals Rule. This is true for all $k$ by induction and L'Hopitals rule. 
\epenum
\newpage
\begin{problem} Folland 8.1.4
\end{problem} First note that such $f$ must belong to $L^1_{loc}$, since on any compact set $K$ we have that $$\int_K |f| dx \leq \norm{f}_{\infty}m(K) <\infty. $$
Define $A_rf(x)$ as $$A_rf(x) = \frac{1}{m(B(r,x))}\int_{B(r,x)} f(y)dy.$$ We claim that $\lim_{r\to 0}A_rf(x) = g(x)$ is uniformly continuous. we have that: 
\begin{align*}
	|g(x)-g(y)| & = \lim_{r\to 0} \frac{1}{B_x(r)}\abs{ \int_{B_x(r)} f(z)dz - \int_{B_y(r)} f(z)dz } 
\\ & = \lim_{r\to 0} \frac{1}{B_x(r)}\int_{B_x(r)} \tau_y f(z)-f(z)dz
\\ & \leq \norm{\tau_y f(x) - f(x) }_u
\end{align*} Which can be made arbitrarily small. 
 As $r\to 0$, $A_rf(x) \to f(x)$. We have $$\norm{A_rf - f}_{\infty} \leq \norm{A_rf - \tau_y f}_{\infty} + \norm{\tau_y f - f}_{\infty},$$ which can be made arbitrarily small since $\norm{A_rf - \tau_y f}_{\infty} \to 0$ uniformly as $y\to 0, r\to 0$. Therefore $f$ agrees with $\lim_{r\to 0} A_rf(x)$ except on a set of measure 0. 
\newpage
\begin{problem}
Folland 8.2.6
\end{problem}
The following chain of inequalities holds: 
\begin{align*}
	|f\ast g(x)|^r & = \left|\int f(y)g(x-y)dy \right|^r
	\\ & \leq \left(\int \left|f(y) \right|  \left|g(x-y) \right| \right)^r
	\\ & = \left(\int \abs{f(x)}^{1+p/q-p/q} \abs{g(x-y)}^{1+q/p-q/p} dy \right)^r
	\\ & = \left( \int \abs{f(y)}^{p/r} \abs{g(x-y)}^{q/r} \abs{f(y)}^{(r-p)/r} \abs{g(x-y)}^{(r-q)/r} dy \right)^r
	\\ &  = \left( \int \left(\abs{f(y)}^{p} \abs{g(x-y)}^q \right)^{1/r} \abs{f(y)}^{(r-p)/r} \abs{g(x-y)}^{(r-q)/r} dy \right)^r
	\\ & \leq \norm{ f(y)^{p} g(x-y)^p }_r^r \cdot \norm{f(y)^{\frac{r-p}{p}}}_{\frac{pr}{r-p}}^r \cdot \norm{g(x-y)^{\frac{r-q}{q}} }^r_{\frac{qr}{r-q}} \tag{by Generalized Holders Inequality}
	\\ & \leq \norm{f}^{r-p}_p \norm{g}^{r-q}_{q} \int |f(y)|^p|g(x-y)|^q dy
\end{align*}
Therefore 
\begin{align*}
\int |f\ast g(x)|^r dx  & \leq \norm{f}_p^{r-p} \norm{g}_{q}^{r-q} \int \int |f(y)|^p |g(x-y)|^q dy dx \\ & =\norm{f}_p^{r-p} \norm{g}_{q}^{r-q} \int|f(y)|^p dy \int|g(x-y)|^q dx \tag{by Fubini-Tonelli} 
\\ & = \norm{f}_{p}^r \norm{g}^r_{q} 
\end{align*}
As desired. 
\newpage
\begin{problem}
	Folland 8.2.7
\end{problem}
Since $g$ has compact support, then for every multi index $\alpha, |\alpha|\leq k$ we have $\partial^\alpha g\in C_c(\R^n)$. Thus $$\partial^\alpha f\ast g(x) = \int_{\R^n} \partial^\alpha g(x-y) f(y) dy$$ will exist, since $f\in L^1_{loc}$ and $\partial^\alpha g$ is compactly supported. 
\newpage
\begin{problem}
	Folland 8.2.8
\end{problem}
By the fundamental theorem of calculus, $\partial_j(f\ast g)$ exists in the regular sense. We now claim that it equals $(\partial_j f)\ast g$. 
We compute that 
\begin{align*}
	&\lim_{y\to 0} \norm{y^{-1}\left(\tau_{-y}f\ast g  - f\ast g\right) - (\partial_i f) \ast g }
	\\ &=  \lim_{y\to 0 } \norm{y^{-1} \left[ \int f(x+y e_j - z) g(z) dz - \int f(x-z)g(z)dz \right] - \int \partial_j f(x-z)g(z)dz }
	\\ & = \lim_{y\to 0} \norm{y^{-1} \int \left[\abs{f(x+ye_j -z) -f(x-z) - y \partial_j f(x-z) }\right] \abs{g(z) }dz } 
	\\ & \leq \lim_{y\to 0} \left[ \norm{y^{-1} (\tau_{-y}f - f) - \partial_j f}_p \norm{g}_{q}  \right] \tag{By Holders Inequality}
	\\ & = 0\tag{since $f$ is strong $L^p$ differentiable}
\end{align*}
\newpage 
\begin{problem}
	Folland 8.2.9
\end{problem}
First suppose that $f^\prime$ exists almost everywhere.Then, by taking any $g\in C_c$ with $\int g = 1$ we have by Folland 8.2.8 $f\ast g$ is $L^p$ differentiable in the usual sense. We also have that $(f\ast g_t)^\prime = f^\prime \ast g_t$. By Folland Theorem 8.14 we have $f^\prime \ast g_t\to f^\prime $ as $t\to 0$. Therefore we have that $f$ is absolutely continuous by the Fundamental Theorem of Lebesgue integrals. Conversely suppose that $f$ is absolutely continuous on bounded intervals. We can write using the fundamental theorem of Calculus, $$\frac{f(x+y)-f(x)}{y} - f^\prime(x)= \frac{1}{y} \int_0^y \left[f^\prime(x+t) - f^\prime(x)\right] dx .$$ 
Taking the $L^p$ norm we have $$\norm{\frac{1}{y} \int_0^y \left[f^\prime(x+t) - f^\prime(x)\right] dx }_p \leq \norm{\frac{1}{y}\int_0^y f^\prime(x) dt}_p + \norm{\frac{1}{y} \int_{0}^y f^\prime(x+t) dt} \to 0 \text{ as y $\to 0$}$$
\newpage
\begin{problem}
	Folland 8.2.10
\end{problem}
We can write our integral as $$\abs{f\ast \phi_t(x)} \leq \int \abs{f(x-y)} \abs{\phi_t(y)}dy = \int_{|x|\leq t}\abs{f(x-y)} \abs{\phi_t(y)}dy + \sum_{k=0}^\infty \int_{2^kt \leq |x| \leq 2^{k+1}t}\abs{f(x-y)} \abs{\phi_t(y)}dy.$$ We bound each term in the following way: 
\begin{align*}
	\int_{|x|\leq t}\abs{f(x-y)} \abs{\phi_t(y)}dy & = t^{-n} \int_{|x|\leq t} \abs{f(x-y)} \abs{\phi(t^{-1}y)} dy
	\\ & \leq Ct^{-n} \int_{|x|\leq t} |f(x-y)|dy
	\\ & \leq C \frac{m(B_x(1))}{m(B_x(t))} \int_{B_x(t)} \abs{f(y)} dy 
	\\ & = Cm(B_x(1))Hf(x). 
\end{align*}
For The second summand, we estimate that 
\begin{align*}
	\sum_{k=0}^\infty \int_{2^kt \leq |x| \leq 2^{k+1}t}\abs{f(x-y)} \abs{\phi_t(y)}dy & \leq \sum_{k=0}^\infty Ct^{-n}\int_{2^kt \leq |x| \leq 2^{k+1}t} |f(x-y)| \left(1+|t^{-1}y|\right)^{-n-\ep} dy
	\\ & \leq \sum_{k=0}^\infty Cm(B_x(1)) \left(2^k\right)^{-n-\ep} Hf(x) 
	\\ & \leq 2Cm(B_x(1))Hf(x). 
\end{align*} 
Therefore $M_\phi(f) \leq C\cdot Hf(x)$. 
\newpage
\begin{problem}
	Folland 8.2.11
\end{problem}
\penum
\item Let $\mathcal{J}$ be an ideal in $L^1$. Let $\ol{\mathcal{J}}$ be its closure. Take any sequence $\{f_n\}$ in $\mathcal{J}$ with limit $f$. Then by youngs inequality with $p=1$, we have for any $g\in L^1$, $$\norm{f_n\ast g}_1 \leq \norm{f_n}_1 \norm{g}_1\implies \norm{f\ast g}_1 \leq \norm{f}_1\norm{g}_1.$$
\item Let $U$ be the subspace. First, suppose that $g\in C_c$. Then, for a finite partition of the support of $g$ we estimate $f \ast g$ by $$\sum_{i=1}^n f(x-y_j) g(y_i) (y_{i+1} - y_i)  = \sum_{i=1}^n (y_{i+1} - y_i) \tau_{y_i}f(x) \leq \sum_{i}^n \int_{(y_{i}, y_{i+1}]}|f(x-y)|dy <\infty.$$
Conversely, consider the sequence $\{\phi_{1/n}\}$ with $\int \phi = 1$, then by prop 8.6, theorem 8.14 $$f \ast \tau_y \phi_{1/n} \to \tau_y f. $$ Since $I$ is closed we have that this is in $I$. 
\epenum
\newpage
\begin{problem}
	60 [Extra Credit]
\end{problem}
Take $\{U_i\}$ to be a countable covering of $\R^n \setminus E$. Take $f_i$ to be smooth,  $>0$ on a compact set contained inside $U_i$, and $0 $ outside of $U_i$. Define $$f = \sum_{i=1}^\infty \frac{f_i}{2^i M_i},$$ where $M_i = \sup_{\alpha \leq i}\abs{\partial^\alpha f_i }$. This sequence absolutely and uniformly converges. The partial derivatives of all orders of $f$ are bounded, so $f\in C^\infty$. Furthermore $f(x) = 0$ if and only if $f_i(x)=0$ for all $i$ i.e. $x\in E$. Thus we are done. 
\end{document}