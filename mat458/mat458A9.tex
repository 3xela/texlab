\documentclass[12pt, a4paper]{article}
\usepackage[lmargin =0.5 in, 
rmargin=0.5in, 
tmargin=1in,
bmargin=0.5in]{geometry}
\geometry{letterpaper}
\usepackage{tikz-cd}
\usepackage{amsmath}
\usepackage{amssymb}
\usepackage{blindtext}
\usepackage{titlesec}
\usepackage{enumitem}
\usepackage{fancyhdr}
\usepackage{amsthm}
\usepackage{graphicx}
\usepackage{cool}
\usepackage{thmtools}
\usepackage{hyperref}
\graphicspath{ }					%path to an image

%-------- sexy font ------------%
%\usepackage{libertine}
%\usepackage{libertinust1math}

%\usepackage{mlmodern}				% very nice and classic
%\usepackage[utopia]{mathdesign}
%\usepackage[T1]{fontenc}


\usepackage{mlmodern}
\usepackage{eulervm}
%\usepackage{tgtermes} 				%times new roman
%-------- sexy font ------------%


% Problem Styles
%====================================================================%


\newtheorem{problem}{Problem}


\theoremstyle{definition}
\newtheorem{thm}{Theorem}
\newtheorem{lemma}{Lemma}
\newtheorem{prop}{Proposition}
\newtheorem{cor}{Corollary}
\newtheorem{fact}{Fact}
\newtheorem{defn}{Definition}
\newtheorem{example}{Example}
\newtheorem{question}{Question}

\newtheorem{manualprobleminner}{Problem}

\newenvironment{manualproblem}[1]{%
	\renewcommand\themanualprobleminner{#1}%
	\manualprobleminner
}{\endmanualprobleminner}

\newcommand{\penum}{ \begin{enumerate}[label=\bf(\alph*), leftmargin=0pt]}
	\newcommand{\epenum}{ \end{enumerate} }

% Math fonts shortcuts
%====================================================================%

\newcommand{\ring}{\mathcal{R}}
\newcommand{\N}{\mathbb{N}}                           % Natural numbers
\newcommand{\Z}{\mathbb{Z}}                           % Integers
\newcommand{\R}{\mathbb{R}}                           % Real numbers
\newcommand{\C}{\mathbb{C}}                           % Complex numbers
\newcommand{\F}{\mathbb{F}}                           % Arbitrary field
\newcommand{\Q}{\mathbb{Q}}                           % Arbitrary field
\newcommand{\PP}{\mathcal{P}}                         % Partition
\newcommand{\M}{\mathcal{M}}                         % Mathcal M
\newcommand{\eL}{\mathcal{L}}                         % Mathcal L
\newcommand{\T}{\mathcal{T}}                         % Mathcal T
\newcommand{\U}{\mathcal{U}}                         % Mathcal U\\
\newcommand{\V}{\mathcal{V}}                         % Mathcal V

% symbol shortcuts
%====================================================================%

\newcommand{\lam}{\lambda}
\newcommand{\imp}{\implies}
\newcommand{\all}{\forall}
\newcommand{\exs}{\exists}
\newcommand{\delt}{\delta}
\newcommand{\ep}{\varepsilon}
\newcommand{\ra}{\rightarrow}
\newcommand{\vph}{\varphi}

\newcommand{\ol}{\overline}
\newcommand{\f}{\frac}
\newcommand{\lf}{\lfrac}
\newcommand{\df}{\dfrac}

% bracketting shortcuts
%====================================================================%
\newcommand{\abs}[1]{\left| #1 \right|}
\newcommand{\babs}[1]{\Big|#1\Big|}
\newcommand{\bound}{\Big|}
\newcommand{\BB}[1]{\left(#1\right)}
\newcommand{\dd}{\mathrm{d}}
\newcommand{\artanh}{\mathrm{artanh}}
\newcommand{\Med}{\mathrm{Med}}
\newcommand{\Cov}{\mathrm{Cov}}
\newcommand{\Corr}{\mathrm{Corr}}
\newcommand{\tr}{\mathrm{tr}}
\newcommand{\Range}[1]{\mathrm{range}(#1)}
\newcommand{\Null}[1]{\mathrm{null}(#1)}
\newcommand{\lan}{\langle}
\newcommand{\ran}{\rangle}
\newcommand{\norm}[1]{\left\lVert#1\right\rVert}
\newcommand{\inn}[1]{\lan#1\ran}
\newcommand{\op}[1]{\operatorname{#1}}
\newcommand{\bmat}[1]{\begin{bmatrix}#1\end{bmatrix}}
\newcommand{\pmat}[1]{\begin{pmatrix}#1\end{pmatrix}}
\newcommand{\vmat}[1]{\begin{vmatrix}#1\end{vmatrix}}

\newcommand{\amogus}{{\bigcap}\kern-0.8em\raisebox{0.3ex}{$\subset$}}
\newcommand{\Note}{\textbf{Note: }}
\newcommand{\Aside}{{\bf Aside: }}
%restriction
%\newcommand{\op}[1]{\operatorname{#1}}
%\newcommand{\done}{$$\mathcal{QED}$$}

%====================================================================%


\setlength{\parindent}{0pt}      	% No paragraph indentations
\pagestyle{fancy}
\fancyhf{}							% fancy header

\setcounter{secnumdepth}{0}			% sections are numbered but numbers do not appear
\setcounter{tocdepth}{2} 			% no subsubsections in toc

%template
%====================================================================%
%\begin{manualproblem}{1}
%Spivak.
%\end{manualproblem}

%\begin{proof}[Solution]
%\end{proof}

%----------- or -----------%

%\begin{problem} 		
%\end{problem}	

%\penum
%	\item
%\epenum
%====================================================================%


\newcommand{\Course}{MAT458 }
\newcommand{\hwNumber}{8}

%preamble

\title{a}
\author{A.N.}
\date{\today}
\lhead{\Course A\hwNumber}
\rhead{\thepage}
%\cfoot{\thepage}


%====================================================================%
\begin{document}
\begin{problem}
	Folland 8.6.39
\end{problem}
First assume that $\mu$ is not the finite sum of point masses. Then we have that 
$$\hat{\mu}(k) = \int_{\mathbb{T}} e^{-2\pi i kx} d\mu(x) \leq \frac{1}{k} \int_{\mathbb{T}}e^{2\pi i y}d\mu(y) = \frac{1}{k}<1. $$
If $\mu$ is a linear combination of point masses with the given properties, we have that 
$$\hat{\mu}(jm) =  \int e^{-2\pi i jm x } d\mu = \sum_{k=1}^n a_k \int e^{-2\pi i jmx} d \delta_{\frac{\alpha+(k-1)}{m}} = \sum_{k=1}^n a_k e^{-2\pi i j(\alpha + k-1)} =\sum_{k=1}^n e^{-2\pi i j \alpha} = e^{-2\pi i j \alpha} $$
\newpage 
\begin{problem}
	Folland 8.6.40
\end{problem}
Let $\mu\in M(\R^n)$. Let $\{\phi_t\}$ be an approximate identity. Then we have that $\phi_t \ast \mu\in L^1$ by prop $8.49$. Thus for every $g$, we have that $$\int g d(\phi_t \ast \mu) \to \int g d\mu,$$ so for sufficiently small $t$ we can take $|\phi_t \ast \mu - \mu|<\ep$ in the weak $*$ topology. 
\newpage 
\begin{problem}
	Folland 8.6.41
\end{problem}
It is sufficient to show that $\Delta$ is vaguely dense in $L^1$ since $L^1$ is vaguely dense in $M(\R^n)$ by $8.6.40$. Take $f\in C_c(\R^n)$. Then for any $g\in C_0(\R^n)$, we have that $fg$ is riemann integrable. Therefore for $\ep>0$, choose a partition $\{R_i\}$ so that 
$$\left| \int fg - \sum_{i=1}^n Vol(R_i)\sup_{R_i}(fg) \right|<\ep . $$ Therefore we can take $\mu\in \Delta$ as $\mu = \sum_{i=1}^n Vol(R_i)\sup_{R_i}(fg) \delta_{y_i}$ for some $y_i \in R_i$.
\newpage
\begin{problem}
	Folland 8.7.43
\end{problem}
We rewrite our PDE as $(1-\partial^2)u = f$. Applying the fourier transformation and inverting, we get the condition that $$\hat{u} = \frac{1}{1 - \xi^2}\hat{f}.$$ We also verify that $$\int \frac{1}{2}e^{-|x|}\cdot e^{2\pi i \xi x} dx =\frac{1}{1- \xi^2}. $$ It follows that the solution to the PDE will be given as $f\ast \phi$. We can verify this by a straightforward computation to see that $$u - u^{\prime \prime} = f\ast (\phi - \phi^{\prime \prime}) = f \ast \delta = f. $$ As long as $f\in L^1$ this solution will make sense, since Fourier inversion is defined.  
\newpage 
\begin{problem}
	Folland 8.7.44
\end{problem}
First we show that $u(x,t) = f\ast G_t(x)$ is well defined. Taking $\ep$ sufficiently small, we have that $$|u(x,t) | = |f\ast G_t(x)| = \left|\int f(y) G_t(x-y) dy\right| \leq \int |f(y)| G_t(x-y) dy \leq \int C_\ep e^{\ep |x^2|}G_t(x-y) dy <\infty.$$
Next we claim that $\lim_{t\to 0}u(x,t) = f(x) $ a.e. Take $V$ as a bounded open set. By Urysohns lemma, take $\phi $ which is $1$ on $V$. We write $f = \phi f+ (1-\phi )f$.  Since $G_t(x)$ is an approximate identity, we have that $(1-\phi)f \ast G_t(x) \to 0$ on $V$, and $\phi f \ast G_t(x) \to \phi f(x) = f(x)$ on $V$. Therefore $\lim_{t\to 0}u(x,t) = f(x) $ a.e.
We now check that is satisfies the PDE. We compute that: 
\begin{align*}(\partial_t - \Delta)(f\ast G_t(x)) 
	& = (\partial_t - \Delta) \left(\int f(y)G_t(x-y)dy  \right)
	\\ & =\int f(y) \left[\partial_t (4\pi t)^{-\frac{n}{2}} e^{-\frac{|x-y|^2}{4t}} -\Delta (4\pi t)^{-\frac{n}{2}} e^{-\frac{|x-y|}{4t}} \right]dy
	\\ & = \int f(y) \left[- \frac{n}{2}4\pi  e^{- \frac{-|x-y|}{4t}}+ (4\pi t)^{-\frac{n}{2}}\frac{|x-y|^2}{4t^2}e^{-\frac{|x-y|^2}{4t}} + (4\pi t)^{-\frac{n}{2}}\right] \\& + \left[ - (4\pi t)^{-\frac{n}{2}}\frac{|x-y|^2}{4t^2} e^{-\frac{|x-y|^2}{4t}}  (4\pi t)^{-\frac{n}{2}} \frac{n}{2t} e^{-\frac{|x-y|^2}{4t}}\right] dy
	\\ & = 0. 
\end{align*}
\newpage
\begin{problem}
	Folland 8.7.45
\end{problem}
Using $8.55$, we can write the solution as $$u(x,t) = \partial_t(f\ast W_t(x)) + g\ast W_t(x),$$ where $W_t = \left[\frac{\sin(2\pi t |\xi|)}{2\pi |\xi|}\right]^\vee$. By exercise $15a$ we know that $W_t = \frac{1}{2}\chi_{[-t,t]}$. So we compute that
\begin{align*}
	u(x,t) &= \frac{1}{2}\left( \partial_t \int f(x-s)\chi_{[-t,t]} ds \right) + \frac{1}{2} \left( \int g(x-s)\chi_{[-t,t]}ds \right)
	\\ & = \frac{1}{2} \partial_t \int_{x-t}^{x+t}f(s)ds + \frac{1}{2} \int_{x-t}^{x-t}g(s)ds
	\\ & =\frac{1}{2} \left[f(x+t) + f(x-t)\right] + \frac{1}{2}\int_{x-t}^{x+t}g(s) ds \tag{by FTC}
\end{align*}
\newpage 
\begin{problem}
	Folland 9.1.1
\end{problem}
\penum
\item If $f_n \to f$ in $L^p$ norm we have that $\inn{f_n,\phi} \to \inn{f,\phi}$ for any $\phi \in L^q$. If $f_n \to f$ weakly then for $\phi \in L^q$, $$\inn{g,\tau_x(f_n-f)} = |g\ast f_n-f| \leq \norm{g}_1 \norm{f_n-f} \to 0.$$
\item Take $h\in C_c^\infty$ Then, we have that $$\int |f_n||h| \leq \int |g|||h|.$$
Therefore by the DCT, we have that $\int f_n h \to \int f h$ for all $h\in C_c^\infty$
\item Consider $\{f_n\}$, the growing steeples. Then $f_n \to 0$ pointwise, but if we take $g = 1$ on $[0,1]$ and $0$ outside of some open interval, we have that $\int f_n g >0$ for all $n$ but $\int fg = 0$. 
\epenum
\newpage 
\begin{problem}
	Folland 9.1.5
\end{problem}
We verify that $f^\prime$ satisfies $\inn{f^\prime , \phi} = - \inn{f, \phi^\prime}.$
\begin{align*}
	\inn{f^\prime , \phi} & = \int \left(\frac{df}{dx} + \sum^m \left( f(x_j+) - f(x_j-)\right)\tau_{x_j}\delta	\right) \phi dx
	\\ & = -\int_{\R \setminus{x_1, \dots x_m}} f \phi^\prime dx +  \int_{\R} \sum^m \left( f(x_j+)- f(x_j -) \right) \tau_{x_j}\delta dx
	\\ & = -\int_{\R \setminus{x_1, \dots x_m}} f \phi^\prime dx - \int \sum^m \left( f(x_j+)- f(x_j -) \right) \phi^\prime(x_j) dx
	\\ & = -  \int f \phi^\prime dx
\end{align*}
\end{document}
