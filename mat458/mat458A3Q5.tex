\documentclass[letterpaper]{article}
\usepackage[letterpaper,margin=1in,footskip=0.25in]{geometry}
\usepackage[utf8]{inputenc}
\usepackage{amsmath}
\usepackage{amsthm}
\usepackage{amssymb, pifont}
\usepackage{mathrsfs}
\usepackage{enumitem}
\usepackage{fancyhdr}
\usepackage{hyperref}

\pagestyle{fancy}
\fancyhf{}
\rhead{MAT 458}
\lhead{Assignment 2}
\rfoot{Page \thepage}

\setlength\parindent{24pt}
\renewcommand\qedsymbol{$\blacksquare$}

\DeclareMathOperator{\hil}{\mathcal{H}}
\DeclareMathOperator{\E}{\mathcal{E}}
\DeclareMathOperator{\M}{\mathcal{M}}
\DeclareMathOperator{\F}{\mathbb{F}}
\DeclareMathOperator{\T}{\mathcal{T}}
\DeclareMathOperator{\V}{\mathcal{V}}
\DeclareMathOperator{\U}{\mathcal{U}}
\DeclareMathOperator{\Prt}{\mathbb{P}}
\DeclareMathOperator{\R}{\mathbb{R}}
\DeclareMathOperator{\N}{\mathbb{N}}
\DeclareMathOperator{\Z}{\mathbb{Z}}
\DeclareMathOperator{\Q}{\mathbb{Q}}
\DeclareMathOperator{\C}{\mathbb{C}}
\DeclareMathOperator{\ep}{\varepsilon}
\DeclareMathOperator{\identity}{\mathbf{0}}
\DeclareMathOperator{\card}{card}
\newcommand{\suchthat}{;\ifnum\currentgrouptype=16 \middle\fi|;}

\newtheorem{lemma}{Lemma}

\newcommand{\tr}{\mathrm{tr}}
\newcommand{\ra}{\rightarrow}
\newcommand{\lan}{\langle}
\newcommand{\ran}{\rangle}
\newcommand{\norm}[1]{\left\lVert#1\right\rVert}
\newcommand{\inn}[1]{\lan#1\ran}
\newcommand{\ol}{\overline}
\newcommand{\ci}{i}
\newcommand{\X}{\mathfrak{X}}
\begin{document} \noindent 5.5.65: Suppose $l^2(A)$ is unitarily isomorphic to $l^2(B).$ 
We have that $\{\phi_\alpha\}_{\alpha \in A}$ where $\phi_\alpha(x) = \delta_{x \alpha} $ forms an orthonormal basis for $l^2(A)$. 
If $U$ is our unitary isomorphism, $\{U(\phi_\alpha)\}$ forms an orthonormal basis for $l^2(B)$ since $U$ preserves the inner product, and is a bijection.
There is a natural bijection between $\{\phi_\alpha\}$ and $A$ by $\phi_\alpha \leftrightarrow \alpha$. The same reasoning with $\{\psi_\beta\}_{\beta \in B}$ 
shows that $card(A) = card(B)$. Now suppose that $card(A) = card(B)$. Construct $\{\phi_\alpha\}_{\alpha \in A}$ and $\{\psi_\beta\}_{\beta \in B}$ as above. 
If $h$ is a bijection between $A$ and $B$, define $U(\phi_\alpha)= \psi_{h(\beta)}$ and extend linearly. 
We have that $U$ sends an orthonormal basis to another orthonormal basis and is a bijection. Thus we are done. 


\end{document}