\documentclass[letterpaper]{article}
\usepackage[letterpaper,margin=1in,footskip=0.25in]{geometry}
\usepackage[utf8]{inputenc}
\usepackage{amsmath}
\usepackage{amsthm}
\usepackage{amssymb, pifont}
\usepackage{mathrsfs}
\usepackage{enumitem}
\usepackage{fancyhdr}
\usepackage{hyperref}

\pagestyle{fancy}
\fancyhf{}
\rhead{MAT 458}
\lhead{Assignment 1}
\rfoot{Page \thepage}

\setlength\parindent{24pt}
\renewcommand\qedsymbol{$\blacksquare$}

\DeclareMathOperator{\E}{\mathcal{E}}
\DeclareMathOperator{\M}{\mathcal{M}}
\DeclareMathOperator{\F}{\mathbb{F}}
\DeclareMathOperator{\T}{\mathcal{T}}
\DeclareMathOperator{\V}{\mathcal{V}}
\DeclareMathOperator{\U}{\mathcal{U}}
\DeclareMathOperator{\Prt}{\mathbb{P}}
\DeclareMathOperator{\R}{\mathbb{R}}
\DeclareMathOperator{\N}{\mathbb{N}}
\DeclareMathOperator{\Z}{\mathbb{Z}}
\DeclareMathOperator{\Q}{\mathbb{Q}}
\DeclareMathOperator{\C}{\mathbb{C}}
\DeclareMathOperator{\ep}{\varepsilon}
\DeclareMathOperator{\identity}{\mathbf{0}}
\DeclareMathOperator{\card}{card}
\newcommand{\suchthat}{;\ifnum\currentgrouptype=16 \middle\fi|;}

\newtheorem{lemma}{Lemma}

\newcommand{\tr}{\mathrm{tr}}
\newcommand{\ra}{\rightarrow}
\newcommand{\lan}{\langle}
\newcommand{\ran}{\rangle}
\newcommand{\norm}[1]{\left\lVert#1\right\rVert}
\newcommand{\inn}[1]{\lan#1\ran}
\newcommand{\ol}{\overline}
\newcommand{\ci}{i}
\newcommand{\X}{\mathfrak{X}}
\begin{document}
Q7: Since $\lim_n T_n x$ exists for all $x$, uniform bounded principle tells us that there is some $C$ so that $\sup_n \norm{T_n} \leq C$. 
Since limits and $T_n$ are linear, the definition of $Tx = \lim_n T_n x$ implies that $T$ is linear. It is sufficient to show that $T$ is continous. Let $\{x_m\}$ be a sequence converging to $x$. 
We compute that $$\norm{Tx_m - Tx} \leq \norm{(Tx_m - T_nx_m) + (T_n x_m - T_n x) + (T_nx - Tx)} \leq \norm{Tx_m - T_n x_m} + \norm{T_nx_m - T_n x} + \norm{T_nx - Tx}.$$
Uniform boundedness gives us $\norm{T_n x_m - T_n x} \leq C \norm{x_m - x}$, and the definition of $T$ tells us that the other terms on the right hand side go to $0$ for large $n,m$. Thus $Tx_m \to Tx$ as desired.  

\end{document}