\documentclass[letterpaper]{article}
\usepackage[letterpaper,margin=1in,footskip=0.25in]{geometry}
\usepackage[utf8]{inputenc}
\usepackage{amsmath}
\usepackage{amsthm}
\usepackage{amssymb, pifont}
\usepackage{mathrsfs}
\usepackage{enumitem}
\usepackage{fancyhdr}
\usepackage{hyperref}

\pagestyle{fancy}
\fancyhf{}
\rhead{MAT 458}
\lhead{Assignment 2}
\rfoot{Page \thepage}

\setlength\parindent{24pt}
\renewcommand\qedsymbol{$\blacksquare$}

\DeclareMathOperator{\E}{\mathcal{E}}
\DeclareMathOperator{\M}{\mathcal{M}}
\DeclareMathOperator{\F}{\mathbb{F}}
\DeclareMathOperator{\T}{\mathcal{T}}
\DeclareMathOperator{\V}{\mathcal{V}}
\DeclareMathOperator{\U}{\mathcal{U}}
\DeclareMathOperator{\Prt}{\mathbb{P}}
\DeclareMathOperator{\R}{\mathbb{R}}
\DeclareMathOperator{\N}{\mathbb{N}}
\DeclareMathOperator{\Z}{\mathbb{Z}}
\DeclareMathOperator{\Q}{\mathbb{Q}}
\DeclareMathOperator{\C}{\mathbb{C}}
\DeclareMathOperator{\ep}{\varepsilon}
\DeclareMathOperator{\identity}{\mathbf{0}}
\DeclareMathOperator{\card}{card}
\newcommand{\suchthat}{;\ifnum\currentgrouptype=16 \middle\fi|;}

\newtheorem{lemma}{Lemma}

\newcommand{\tr}{\mathrm{tr}}
\newcommand{\ra}{\rightarrow}
\newcommand{\lan}{\langle}
\newcommand{\ran}{\rangle}
\newcommand{\norm}[1]{\left\lVert#1\right\rVert}
\newcommand{\inn}[1]{\lan#1\ran}
\newcommand{\ol}{\overline}
\newcommand{\ci}{i}
\newcommand{\X}{\mathfrak{X}}
\begin{document} \noindent 5.4.52a: The inclusion $M \subset N^0$ is clear since if $Tx =0$ each $f_i$ is certainly 0. 
Now suppose that $\varphi \in N^0$. Consider the mapping into $C^{n+1}$ defined by $ x \mapsto (\varphi(x), T(x))$. The
image of this map must be an $n$ dimensional subspace. By Hahn-Banach there exists a linear functional $g \in (\C^{n+1})^\ast$ that is 
$0$ on the image of $(\varphi, T)$ and nonzero on the remaining 1-d subspace. If $v_1 \dots v_n$ is a basis for the image of $(\varphi, T)$ and $v_{n+1}$ is a basis for the subspace on which $g$ is nonzer0, we have that 
$$0 = g(\varphi(x), f_1(x) , \dots f_n(x))= g(e_{n+1})\varphi(x) + \sum_{i}f_i(x) g(e_i) \implies \varphi(x)= - \sum_{i}\frac{g(e_i)}{g(e_{n+1}) f_i(x)}.$$
Thus $\varphi$ is in the span. By exercise 23, we have that $M^\ast \cong (\X / N)^\ast. $
\newline \\ 5.4.52b: We have that the projection map $\pi_1: \X \to M$ is an isometric embedding by 5.4.23. Therefore so is $\pi_2 : M^\ast \to (\X / N)^{\ast \ast}.$ Take $\ep >0$. 

\end{document}