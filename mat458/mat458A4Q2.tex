\documentclass[letterpaper]{article}
\usepackage[letterpaper,margin=1in,footskip=0.25in]{geometry}
\usepackage[utf8]{inputenc}
\usepackage{amsmath}
\usepackage{amsthm}
\usepackage{amssymb, pifont}
\usepackage{mathrsfs}
\usepackage{enumitem}
\usepackage{fancyhdr}
\usepackage{hyperref}

\pagestyle{fancy}
\fancyhf{}
\rhead{MAT 458}
\lhead{Assignment 4}
\rfoot{Page \thepage}

\setlength\parindent{24pt}
\renewcommand\qedsymbol{$\blacksquare$}

\DeclareMathOperator{\hil}{\mathcal{H}}
\DeclareMathOperator{\E}{\mathcal{E}}
\DeclareMathOperator{\M}{\mathcal{M}}
\DeclareMathOperator{\F}{\mathbb{F}}
\DeclareMathOperator{\T}{\mathcal{T}}
\DeclareMathOperator{\V}{\mathcal{V}}
\DeclareMathOperator{\U}{\mathcal{U}}
\DeclareMathOperator{\Prt}{\mathbb{P}}
\DeclareMathOperator{\R}{\mathbb{R}}
\DeclareMathOperator{\N}{\mathbb{N}}
\DeclareMathOperator{\Z}{\mathbb{Z}}
\DeclareMathOperator{\Q}{\mathbb{Q}}
\DeclareMathOperator{\C}{\mathbb{C}}
\DeclareMathOperator{\ep}{\varepsilon}
\DeclareMathOperator{\identity}{\mathbf{0}}
\DeclareMathOperator{\card}{card}
\newcommand{\suchthat}{;\ifnum\currentgrouptype=16 \middle\fi|;}

\newtheorem{lemma}{Lemma}

\newcommand{\tr}{\mathrm{tr}}
\newcommand{\ra}{\rightarrow}
\newcommand{\lan}{\langle}
\newcommand{\ran}{\rangle}
\newcommand{\norm}[1]{\left\lVert#1\right\rVert}
\newcommand{\inn}[1]{\lan#1\ran}
\newcommand{\ol}{\overline}
\newcommand{\ci}{i}
\newcommand{\X}{\mathfrak{X}}
\begin{document} \noindent Q2: 
Let $T$ be a surjective linear mapping from banach spaces $E$ to $F$. Suppose that $T$ has a right inverse $S$. We claim that $R(S)$ is the compliment of $N(T)$. First we show that 
$$R(S) \cap N(T) = \{0\}. $$ Let $v\in R(S) \cap N(T)$. For some $u, Su = v$. We also have that $Tv = TSu = id_F u =0.$ Therefore $u = 0$ and so $v=0$. 
We now claim that $E = R(S)+N(T)$. If $v\in E$, then $u = STv - v \in \ker T$. Therefore $v = STv - u$. 
Now suppose that $N(T)$ has a compliment in $E$. Let $G$ be the compliment. We can write any $v\in E$ as $v = x+y$ for $x\in N(T), y\in G$. Define the right inverse $S$ as $S(Tv) = x$. 
We see that $T(S(Tv)) = T(S(Tx)) = Tx.$ 
\end{document}