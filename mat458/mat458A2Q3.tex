\documentclass[letterpaper]{article}
\usepackage[letterpaper,margin=1in,footskip=0.25in]{geometry}
\usepackage[utf8]{inputenc}
\usepackage{amsmath}
\usepackage{amsthm}
\usepackage{amssymb, pifont}
\usepackage{mathrsfs}
\usepackage{enumitem}
\usepackage{fancyhdr}
\usepackage{hyperref}

\pagestyle{fancy}
\fancyhf{}
\rhead{MAT 458}
\lhead{Assignment 2}
\rfoot{Page \thepage}

\setlength\parindent{24pt}
\renewcommand\qedsymbol{$\blacksquare$}

\DeclareMathOperator{\E}{\mathcal{E}}
\DeclareMathOperator{\M}{\mathcal{M}}
\DeclareMathOperator{\F}{\mathbb{F}}
\DeclareMathOperator{\T}{\mathcal{T}}
\DeclareMathOperator{\V}{\mathcal{V}}
\DeclareMathOperator{\U}{\mathcal{U}}
\DeclareMathOperator{\Prt}{\mathbb{P}}
\DeclareMathOperator{\R}{\mathbb{R}}
\DeclareMathOperator{\N}{\mathbb{N}}
\DeclareMathOperator{\Z}{\mathbb{Z}}
\DeclareMathOperator{\Q}{\mathbb{Q}}
\DeclareMathOperator{\C}{\mathbb{C}}
\DeclareMathOperator{\ep}{\varepsilon}
\DeclareMathOperator{\identity}{\mathbf{0}}
\DeclareMathOperator{\card}{card}
\newcommand{\suchthat}{;\ifnum\currentgrouptype=16 \middle\fi|;}

\newtheorem{lemma}{Lemma}

\newcommand{\tr}{\mathrm{tr}}
\newcommand{\ra}{\rightarrow}
\newcommand{\lan}{\langle}
\newcommand{\ran}{\rangle}
\newcommand{\norm}[1]{\left\lVert#1\right\rVert}
\newcommand{\inn}[1]{\lan#1\ran}
\newcommand{\ol}{\overline}
\newcommand{\ci}{i}
\newcommand{\X}{\mathfrak{X}}
\begin{document} \noindent 5.4.49a: It is sufficient to show that any element of the basis i.e. sets of the form 
$$U_{f, \varepsilon}(x) = \{y\in \X : |f(x)-f(y)|< \varepsilon\}$$ is unbounded. Take any $v\in f^{-1}(\{0\})$ nonzero. Then for all $\alpha$, $\alpha v+ x\in U_{f, \varepsilon}$.
For the weak * topology, the basis elements take the form $V_{f, \ep} =\{g\in X^\ast : \norm{f-g} < \ep\}.$
It is sufficient to show that these sets are unbounded. For all $f\in V_{f, \ep }$, $\sup_{\norm{x} = 1} \norm{f(x)- g(x)} = \sup{\hat{x} (f-g)}<\ep$.
Taking $l$ nonzero such that $\hat{x}(l) =0$, we get that for all scalars $\alpha$, $f+ \alpha l \in V_{f,\ep}$. This is unbounded. 
\newline \\ 5.4.49b: If $E \subset \X$ is a bounded subset, then so is its weak closure by Q2b. However by part $a$ we have that the interiour must be empty. The result for bounded $F \subset \X^\ast$ follows in the same way. 
\newline \\ 5.4.49c: We can $E_n = \{x: \norm{x}\leq n\}$. We have that $\X = \bigcup_{n\in \N} E_n$. Thus by $b$, $\X$ is meager in the weak topology. 
Defining $F_n = \{f\in \X^\ast : \norm{f}\leq n \}$. We have that analogously $\X^\ast$ is meager in the weak * topology. 
\newline \\ 5.4.49d: Suppose that some translation invariant metric $d$ defines the weak * topology on $\X^\ast$. Let $\inn{f_n}$ be a cauchy sequence. Then for any $V_{f, \ep}$ as defined in $a$, we have that (this problem set is too long i cant finish it sorry. )
\end{document}