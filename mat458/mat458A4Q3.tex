\documentclass[letterpaper]{article}
\usepackage[letterpaper,margin=1in,footskip=0.25in]{geometry}
\usepackage[utf8]{inputenc}
\usepackage{amsmath}
\usepackage{amsthm}
\usepackage{amssymb, pifont}
\usepackage{mathrsfs}
\usepackage{enumitem}
\usepackage{fancyhdr}
\usepackage{hyperref}

\pagestyle{fancy}
\fancyhf{}
\rhead{MAT 458}
\lhead{Assignment 4}
\rfoot{Page \thepage}

\setlength\parindent{24pt}
\renewcommand\qedsymbol{$\blacksquare$}

\DeclareMathOperator{\hil}{\mathcal{H}}
\DeclareMathOperator{\E}{\mathcal{E}}
\DeclareMathOperator{\M}{\mathcal{M}}
\DeclareMathOperator{\F}{\mathbb{F}}
\DeclareMathOperator{\T}{\mathcal{T}}
\DeclareMathOperator{\V}{\mathcal{V}}
\DeclareMathOperator{\U}{\mathcal{U}}
\DeclareMathOperator{\Prt}{\mathbb{P}}
\DeclareMathOperator{\R}{\mathbb{R}}
\DeclareMathOperator{\N}{\mathbb{N}}
\DeclareMathOperator{\Z}{\mathbb{Z}}
\DeclareMathOperator{\Q}{\mathbb{Q}}
\DeclareMathOperator{\C}{\mathbb{C}}
\DeclareMathOperator{\ep}{\varepsilon}
\DeclareMathOperator{\identity}{\mathbf{0}}
\DeclareMathOperator{\card}{card}
\newcommand{\suchthat}{;\ifnum\currentgrouptype=16 \middle\fi|;}

\newtheorem{lemma}{Lemma}

\newcommand{\tr}{\mathrm{tr}}
\newcommand{\ra}{\rightarrow}
\newcommand{\lan}{\langle}
\newcommand{\ran}{\rangle}
\newcommand{\norm}[1]{\left\lVert#1\right\rVert}
\newcommand{\inn}[1]{\lan#1\ran}
\newcommand{\ol}{\overline}
\newcommand{\ci}{i}
\newcommand{\X}{\mathfrak{X}}
\begin{document} \noindent Q3: Suppose that there is some $S$ so that $S\circ T = id_E$. 
First we claim that $R(T)$ is closed. Suppose that $x_n \to x,$ $Tx_n \to y$. Then we have that $Tz = y$ for some $z$. 
By uniqueness of limits and injectivity we have that $Tx_n \to Tx$. Now let $S$ be the left inverse of $T$. We claim that $\ker S$ is the compliment of 
$R(T)$. If $v\in \ker S\cap R(T)$, then $Sv = 0$ but for some $w, Tw = v$ so $0= STw = id_E w$ so $w=0$ and so $v=0$. 
Now if $v\in F$, $TSv - v = u \in \ker S $ so $v = TSv - u$. Suppose that $R(T)$ is closed and admits a compliment $G$. Every $v\in F$
can be written as $v = x+y$, $x\in R(T), y\in G$. Since $x\in R(T)$ we can write $x = Tu$. Define $Sv = x$. This will be our left inverse since $STx = x$. 
\end{document}