\documentclass[letterpaper]{article}
\usepackage[letterpaper,margin=1in,footskip=0.25in]{geometry}
\usepackage[utf8]{inputenc}
\usepackage{amsmath}
\usepackage{amsthm}
\usepackage{amssymb, pifont}
\usepackage{mathrsfs}
\usepackage{enumitem}
\usepackage{fancyhdr}
\usepackage{hyperref}

\pagestyle{fancy}
\fancyhf{}
\rhead{MAT 458}
\lhead{Assignment 2}
\rfoot{Page \thepage}

\setlength\parindent{24pt}
\renewcommand\qedsymbol{$\blacksquare$}

\DeclareMathOperator{\E}{\mathcal{E}}
\DeclareMathOperator{\M}{\mathcal{M}}
\DeclareMathOperator{\F}{\mathbb{F}}
\DeclareMathOperator{\T}{\mathcal{T}}
\DeclareMathOperator{\V}{\mathcal{V}}
\DeclareMathOperator{\U}{\mathcal{U}}
\DeclareMathOperator{\Prt}{\mathbb{P}}
\DeclareMathOperator{\R}{\mathbb{R}}
\DeclareMathOperator{\N}{\mathbb{N}}
\DeclareMathOperator{\Z}{\mathbb{Z}}
\DeclareMathOperator{\Q}{\mathbb{Q}}
\DeclareMathOperator{\C}{\mathbb{C}}
\DeclareMathOperator{\ep}{\varepsilon}
\DeclareMathOperator{\identity}{\mathbf{0}}
\DeclareMathOperator{\card}{card}
\newcommand{\suchthat}{;\ifnum\currentgrouptype=16 \middle\fi|;}

\newtheorem{lemma}{Lemma}

\newcommand{\tr}{\mathrm{tr}}
\newcommand{\ra}{\rightarrow}
\newcommand{\lan}{\langle}
\newcommand{\ran}{\rangle}
\newcommand{\norm}[1]{\left\lVert#1\right\rVert}
\newcommand{\inn}[1]{\lan#1\ran}
\newcommand{\ol}{\overline}
\newcommand{\ci}{i}
\newcommand{\X}{\mathfrak{X}}
\begin{document} \noindent 5.5.62a:
Suppose that $f\in L^2(\mu)$. For $\ep>0$, by Lusins theorem there is a compact set $E_{\ep}$ with $\mu(E_{\ep}) < \ep$ so that the function $f|_{E_{\ep}}$ is continuous. 
Let $f_{\ep}$ be the continuous extension, which exists by Tietze extension theorem. Since $f,f_{\ep}$ agree on $[0,1] \setminus E_{\ep}$, we see that $\int_{E_{\ep}} |f- f_{\ep}|^2 \to 0  $ as $\ep \to 0$ since $|f-f_{\ep}|^2$ is integrable. 
Thus we are done. 
\newline \\ b: Since the polynomials are dense in $C[0,1]$ by elementary real analysis results, we have that the polynomials are dense in $L^2(\mu)$. 
\newline \\ c: Consider the set of polynomials with integer coefficients, $P$. This forms a function algebra, vanishes nowhere and seperates points. 
It is therefore dense in the set of all polynomials, so dense in $L^2(0,1)$. Furthermore, it is countable, since we can write $P$ as the union of 
the set of polynomials with degree less than $n$, for all $n$. Each of these sets is countable since there is a bijection with $\Z^n$. 
Apply Gram-Schmidt Prodedure to $P$ to get $O$, an orthonormal set of vectors. Since each $v\in O$ is a linear combination of vectors in $P$, we have that $span\{O\}$ is dense in $L^2([0,1])$. 
Therefore for each $f\in L^2([0,1])$, we have that we can write $f = \sum_{v_i\in O}a_i v_i$. Therefore $L^2([0,1])$ is separable. 
\newline \\ d: We have that for all $n\in \Z$, $L^2([n,n+1])$ is seperable by applying the same proof from $c$ to a translated interval. 
By 5.5.60 we have that $L^2(\R) $ is seperable, since $\R = \bigcup_n [n,n+1]$, and each $L^2([n,n+1])$ is seperable. 
\newline \\ e: By 5.5.61, since $\R^n = \R \otimes \R \dots \otimes \R$, and $L^2(\R)$ is seperable, so is $L^2(\R^n)$. 
\end{document}