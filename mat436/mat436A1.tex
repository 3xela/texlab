\documentclass[letterpaper]{article}
\usepackage[letterpaper,margin=1in,footskip=0.25in]{geometry}
\usepackage[utf8]{inputenc}
\usepackage{amsmath}
\usepackage{amsthm}
\usepackage{amssymb, pifont}
\usepackage{mathrsfs}
\usepackage{enumitem}
\usepackage{fancyhdr}
\usepackage{hyperref}

\pagestyle{fancy}
\fancyhf{}
\rhead{MAT 436}
\lhead{Assignment 1}
\rfoot{Page \thepage}

\setlength\parindent{24pt}
\renewcommand\qedsymbol{$\blacksquare$}

\DeclareMathOperator{\F}{\mathbb{F}}
\DeclareMathOperator{\T}{\mathcal{T}}
\DeclareMathOperator{\V}{\mathcal{V}}
\DeclareMathOperator{\U}{\mathcal{U}}
\DeclareMathOperator{\Prt}{\mathbb{P}}
\DeclareMathOperator{\R}{\mathbb{R}}
\DeclareMathOperator{\N}{\mathbb{N}}
\DeclareMathOperator{\Z}{\mathbb{Z}}
\DeclareMathOperator{\Q}{\mathbb{Q}}
\DeclareMathOperator{\C}{\mathbb{C}}
\DeclareMathOperator{\ep}{\varepsilon}
\DeclareMathOperator{\identity}{\mathbf{0}}
\DeclareMathOperator{\card}{card}
\newcommand{\suchthat}{;\ifnum\currentgrouptype=16 \middle\fi|;}

\newtheorem{lemma}{Lemma}

\newcommand{\tr}{\mathrm{tr}}
\newcommand{\ra}{\rightarrow}
\newcommand{\lan}{\langle}
\newcommand{\ran}{\rangle}
\newcommand{\norm}[1]{\left\lVert#1\right\rVert}
\newcommand{\inn}[1]{\lan#1\ran}
\newcommand{\ol}{\overline}
\newcommand{\ci}{i}
\begin{document}
\noindent
Q1: Bee flying problem: \\
Since the trains are approaching eachother at $50m/s$, and they start at a distance of $100m$ apart, they will meet in $1$ second according to basic kinematics. Since the bee travels at $100m/s$, it will move a total of $100m$ before the trains collide. 
\newline \\ 
Q2: Derive Gauss' summation for first n integers: \\ We claim that the following formula holds $$\sum_{i=1}^n i = \frac{n(n+1)}{2}$$ We proceed by induction. For $n=1$ it is clear that $\sum_{i=1}^1 i =1$. Suppose that the formula holds for $n$. We claim that this implies that the formula is true for $n+1$. Observe: 
$$\sum_{i=1}^{n+1} i = \sum_{i=1}^n i +(n+1) = \frac{n(n+1)}{2} + (n+1) = \frac{n^2 +3n+2}{2} = \frac{(n+1)(n+2)}{2}$$
We conclude that the formula holds for all $n$ by the principle of induction. Thus we see that Gauss' summation of integers from 1 to 100 was quite simple he only had to compute: 
$$\sum_{i=1}^{100} i = \frac{100(100+1)}{2} = 5050$$ 
\newline \\ Q3: Suppose there is a tennis tournament with 128 players, how many matches are played in total?
\\ We can compute this as a finite geometric sum. $$\text{total played} = \sum_{i=1}^7 128 \frac{1}{2}^i = 64 \sum_{i=0}^{7} \frac{1}{2}^{i-1} = 64(\frac{1-(\frac{1}{2})^7}{1 - \frac{1}{2}}) = 127$$
Alternatively, we could compute out the sum $64+ 32 + \dots + 1 = 127$. 
\end{document}