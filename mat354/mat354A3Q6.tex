\documentclass[letterpaper]{article}
\usepackage[letterpaper,margin=1in,footskip=0.25in]{geometry}
\usepackage[utf8]{inputenc}
\usepackage{amsmath}
\usepackage{amsthm}
\usepackage{amssymb, pifont}
\usepackage{mathrsfs}
\usepackage{enumitem}
\usepackage{fancyhdr}
\usepackage{hyperref}

\pagestyle{fancy}
\fancyhf{}
\rhead{MAT 354}
\lhead{Assignment 3}
\rfoot{Page \thepage}

\setlength\parindent{24pt}
\renewcommand\qedsymbol{$\blacksquare$}

\DeclareMathOperator{\Qu}{\mathcal{Q}_8}
\DeclareMathOperator{\F}{\mathbb{F}}
\DeclareMathOperator{\T}{\mathcal{T}}
\DeclareMathOperator{\V}{\mathcal{V}}
\DeclareMathOperator{\U}{\mathcal{U}}
\DeclareMathOperator{\Prt}{\mathbb{P}}
\DeclareMathOperator{\R}{\mathbb{R}}
\DeclareMathOperator{\N}{\mathbb{N}}
\DeclareMathOperator{\Z}{\mathbb{Z}}
\DeclareMathOperator{\Q}{\mathbb{Q}}
\DeclareMathOperator{\C}{\mathbb{C}}
\DeclareMathOperator{\ep}{\varepsilon}
\DeclareMathOperator{\identity}{\mathbf{0}}
\DeclareMathOperator{\card}{card}
\newcommand{\suchthat}{;\ifnum\currentgrouptype=16 \middle\fi|;}

\newtheorem{lemma}{Lemma}

\newcommand{\tr}{\mathrm{tr}}
\newcommand{\ra}{\rightarrow}
\newcommand{\lan}{\langle}
\newcommand{\ran}{\rangle}
\newcommand{\norm}[1]{\left\lVert#1\right\rVert}
\newcommand{\inn}[1]{\lan#1\ran}
\newcommand{\ol}{\overline}
\newcommand{\ci}{i}
\begin{document}
\noindent Q6:
Suppose that in some neighbourhood $U_{z_0}$ of $z_0\in I$, $f$ has a power series given by $$f(z) = \sum_{n=0}^\infty a_n (z-z_0)^n$$ with a radius of convergence of $r$. We claim that $f(z)$ absolutely converges on a ball in $\C$ given by $|z-z_0|<r$. 
We compute that $$|\sum_{n=0}^\infty a_n (z-z_0)^n| \leq \sum_{n=0}^\infty |a_n|\cdot |(z-z_0)|^n < \infty$$
We now claim that the coefficients of the power series are the same regardless of which $z_0$ you choose. Take $z_0,z_1 \in I$ such that the neighborhoods containing them where the power series of $f$ converges are not disjoint. Take $z_2\in I$ in this intersection. There must be some power series expansion. By analytic continuation, it must agree with the power series of $f$ at $z_0$ and $z_1$. Hence the power series is the same along $I$, and the neighborhood around $I$ where it converges. If we take the unions of each $B_{r}(z_0)$ we have a connected open set containing $I$ on which $f$ is analytic. 
\end{document}