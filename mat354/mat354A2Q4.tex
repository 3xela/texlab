\documentclass[letterpaper]{article}
\usepackage[letterpaper,margin=1in,footskip=0.25in]{geometry}
\usepackage[utf8]{inputenc}
\usepackage{amsmath}
\usepackage{amsthm}
\usepackage{amssymb, pifont}
\usepackage{mathrsfs}
\usepackage{enumitem}
\usepackage{fancyhdr}
\usepackage{hyperref}

\pagestyle{fancy}
\fancyhf{}
\rhead{MAT 354}
\lhead{Assignment 2}
\rfoot{Page \thepage}

\setlength\parindent{24pt}
\renewcommand\qedsymbol{$\blacksquare$}

\DeclareMathOperator{\Qu}{\mathcal{Q}_8}
\DeclareMathOperator{\F}{\mathbb{F}}
\DeclareMathOperator{\T}{\mathcal{T}}
\DeclareMathOperator{\V}{\mathcal{V}}
\DeclareMathOperator{\U}{\mathcal{U}}
\DeclareMathOperator{\Prt}{\mathbb{P}}
\DeclareMathOperator{\R}{\mathbb{R}}
\DeclareMathOperator{\N}{\mathbb{N}}
\DeclareMathOperator{\Z}{\mathbb{Z}}
\DeclareMathOperator{\Q}{\mathbb{Q}}
\DeclareMathOperator{\C}{\mathbb{C}}
\DeclareMathOperator{\ep}{\varepsilon}
\DeclareMathOperator{\identity}{\mathbf{0}}
\DeclareMathOperator{\card}{card}
\newcommand{\suchthat}{;\ifnum\currentgrouptype=16 \middle\fi|;}

\newtheorem{lemma}{Lemma}

\newcommand{\tr}{\mathrm{tr}}
\newcommand{\ra}{\rightarrow}
\newcommand{\lan}{\langle}
\newcommand{\ran}{\rangle}
\newcommand{\norm}[1]{\left\lVert#1\right\rVert}
\newcommand{\inn}[1]{\lan#1\ran}
\newcommand{\ol}{\overline}
\newcommand{\ci}{i}
\begin{document}
\noindent
Q4: Note that by definition of $\cos(z)$, we can write $$\cos(z) = \frac{e^{iz} + e^{-iz}}{2}$$ We wish to find a $z$ such that for all $c\in \C$ $$\frac{e^{iz} + e^{-iz}}{2}=c$$
Take $z = -i \log(-c - \sqrt{c^2-1})$. We compute that:
\begin{align*}
    \frac{e^{iz} + e^{-iz}}{2} & = \frac{e^{i (-i \log(-c - \sqrt{c^2-1}))} + e^{-i(-i \log(-c - \sqrt{c^2-1}))}}{2}
    \\ & = \frac{e^{\log(-c - \sqrt{c^2-1})} + e^{-\log(-c - \sqrt{c^2-1})}}{2}
    \\ & = \frac{\frac{-c - \sqrt{c^2-1}}{1} + \frac{1}{-c - \sqrt{c^2-1}}}{2}
    \\ & =  \frac{1}{2}\cdot \frac{c^2 -2c\sqrt{c^2-1} + c^2-1+1}{-c - \sqrt{c^2-1}}
    \\ & = \frac{1}{2} \cdot \frac{2c(-c - \sqrt{c^2-1})}{-c-\sqrt{c^2-1}}
    \\ & = c
\end{align*}
We conclude that $\cos(z)$ is surjective onto the complex plane. 

\end{document}