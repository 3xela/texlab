\documentclass[letterpaper]{article}
\usepackage[letterpaper,margin=1in,footskip=0.25in]{geometry}
\usepackage[utf8]{inputenc}
\usepackage{amsmath}
\usepackage{amsthm}
\usepackage{amssymb, pifont}
\usepackage{mathrsfs}
\usepackage{enumitem}
\usepackage{fancyhdr}
\usepackage{hyperref}

\pagestyle{fancy}
\fancyhf{}
\rhead{MAT 354}
\lhead{Assignment 2}
\rfoot{Page \thepage}

\setlength\parindent{24pt}
\renewcommand\qedsymbol{$\blacksquare$}

\DeclareMathOperator{\Qu}{\mathcal{Q}_8}
\DeclareMathOperator{\F}{\mathbb{F}}
\DeclareMathOperator{\T}{\mathcal{T}}
\DeclareMathOperator{\V}{\mathcal{V}}
\DeclareMathOperator{\U}{\mathcal{U}}
\DeclareMathOperator{\Prt}{\mathbb{P}}
\DeclareMathOperator{\R}{\mathbb{R}}
\DeclareMathOperator{\N}{\mathbb{N}}
\DeclareMathOperator{\Z}{\mathbb{Z}}
\DeclareMathOperator{\Q}{\mathbb{Q}}
\DeclareMathOperator{\C}{\mathbb{C}}
\DeclareMathOperator{\ep}{\varepsilon}
\DeclareMathOperator{\identity}{\mathbf{0}}
\DeclareMathOperator{\card}{card}
\newcommand{\suchthat}{;\ifnum\currentgrouptype=16 \middle\fi|;}

\newtheorem{lemma}{Lemma}

\newcommand{\tr}{\mathrm{tr}}
\newcommand{\ra}{\rightarrow}
\newcommand{\lan}{\langle}
\newcommand{\ran}{\rangle}
\newcommand{\norm}[1]{\left\lVert#1\right\rVert}
\newcommand{\inn}[1]{\lan#1\ran}
\newcommand{\ol}{\overline}
\newcommand{\ci}{i}
\begin{document}
\noindent
Q3: Note that we can write that $$\sum_{n=1}^\infty  \frac{z^n}{1-z^n}= \sum_{n=1}^\infty \frac{z^n}{(1-z)(1+ z + \dots z^{n-1})}$$
We will compute the ratio test. $$\lim_{n \to \infty} \Bigg|\frac{\frac{z^{n+1}}{(1-z)(1+ z + \dots z^n)}}{\frac{z^n}{(1-z)(1+ z + \dots + z^{n-1})}} \Bigg| = \lim_{n \to \infty} \frac{|z+ z^2 + \cdots+  z^{n}|}{|1+ z + \cdots + z^n |}$$
We let $r = z+ z^2 + \dots + z^n$ and rewrite the equation as $$\lim_{n\to \infty} \frac{|r|}{|1+r|}$$ If we have that $|z|<1$, then this ratio will converge to a quantity less than one. If we have that $|z|\geq 1$, then it will approach 1 and hence diverge. 
\end{document}