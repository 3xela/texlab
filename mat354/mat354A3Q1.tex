\documentclass[letterpaper]{article}
\usepackage[letterpaper,margin=1in,footskip=0.25in]{geometry}
\usepackage[utf8]{inputenc}
\usepackage{amsmath}
\usepackage{amsthm}
\usepackage{amssymb, pifont}
\usepackage{mathrsfs}
\usepackage{enumitem}
\usepackage{fancyhdr}
\usepackage{hyperref}

\pagestyle{fancy}
\fancyhf{}
\rhead{MAT 354}
\lhead{Assignment 3}
\rfoot{Page \thepage}

\setlength\parindent{24pt}
\renewcommand\qedsymbol{$\blacksquare$}

\DeclareMathOperator{\Qu}{\mathcal{Q}_8}
\DeclareMathOperator{\F}{\mathbb{F}}
\DeclareMathOperator{\T}{\mathcal{T}}
\DeclareMathOperator{\V}{\mathcal{V}}
\DeclareMathOperator{\U}{\mathcal{U}}
\DeclareMathOperator{\Prt}{\mathbb{P}}
\DeclareMathOperator{\R}{\mathbb{R}}
\DeclareMathOperator{\N}{\mathbb{N}}
\DeclareMathOperator{\Z}{\mathbb{Z}}
\DeclareMathOperator{\Q}{\mathbb{Q}}
\DeclareMathOperator{\C}{\mathbb{C}}
\DeclareMathOperator{\ep}{\varepsilon}
\DeclareMathOperator{\identity}{\mathbf{0}}
\DeclareMathOperator{\card}{card}
\newcommand{\suchthat}{;\ifnum\currentgrouptype=16 \middle\fi|;}

\newtheorem{lemma}{Lemma}

\newcommand{\tr}{\mathrm{tr}}
\newcommand{\ra}{\rightarrow}
\newcommand{\lan}{\langle}
\newcommand{\ran}{\rangle}
\newcommand{\norm}[1]{\left\lVert#1\right\rVert}
\newcommand{\inn}[1]{\lan#1\ran}
\newcommand{\ol}{\overline}
\newcommand{\ci}{i}
\begin{document}
\noindent
Q1a: Writing $u+ iv = w = cos(z)$, we compute that 
\begin{align*}
    u+ iv & = cos(z)
    \\ & = \frac{e^{iz} - e^{-iz}}{2}
    \\ & = \frac{e^{i(x+iy)} + e^{-i(x+iy)}}{2}
    \\ & = \frac{e^{ix} \cdot e^{-y} + e^{i(-x)} \cdot e^{y}}{2}
    \\ & = \frac{(\cos(x) + i\sin(x))e^{-y} + (\cos(x) - i \sin(x))e^y}{2}
    \\ & = \frac{\cos(x)(e^y + e^{-y})}{2} + i \frac{-\sin(x)(e^{y} - e^{-y})}{2}
    \\ & = \cos(x)\cosh(y) - i\sin(x)\sinh(y). 
\end{align*}
We conclude that $$u = \cos(x) \cosh(x), v = - \sin(x)\sinh(y)$$
\newline \\ \noindent Q1b: Fix some $y\in \R$, and we let $\alpha = \cosh(y), \beta = \sinh(y)$. We have that $$(u(x),v(x)) = (\alpha \cos(x), \beta \sin(x)). $$
Note that this satisfies the elipse equation $$\frac{u(x)^2}{\alpha^2} + \frac{v(x)^2}{\beta^2} = 1.$$
We claim that for $x\in (0,2\pi]$, $(u(x), v(x))$ is injective. Suppose that for some $x,y \in (0, 2\pi]$ we have that $$(u(x), v(x)) = (u(y), v(y)). $$ This would imply that $$\cos(x) = \cos(y), \sin(x) = \sin(y). $$ The first equation implies that either $x=y$ or $x = 2\pi -y$. Suppose that $x = 2\pi - y$.
This yields $$\sin(x) = \sin(2\pi - x) = -\sin(x)$$
Which implies that $x = \pi $ or $x = 2 \pi$. If $x = \pi$, $y= \pi$. If $x = 2\pi$ then $y = 0$, which can not happen. Thus we conclude that $x=y$ and hence this chain is injective. 
\newline \\ \noindent Q1c: For fixed $x$, define $\alpha = \cos(x), \beta = -\sin(x)$. We have that $$(u(y), v(y)) = (\alpha \cosh(y), \beta \sinh(y)). $$ Note that this satisfies the hyperbola equation $$\frac{u(x)^2}{\alpha^2} - \frac{v(x)^2}{\beta^2} = 1. $$ Since $\cosh(y)> 0$ for all $y\in \R$, if $\alpha>0$ this will define the right branch and if $\alpha<0$ this will define the left branch. 
\end{document}