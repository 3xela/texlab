\documentclass[letterpaper]{article}
\usepackage[letterpaper,margin=1in,footskip=0.25in]{geometry}
\usepackage[utf8]{inputenc}
\usepackage{amsmath}
\usepackage{amsthm}
\usepackage{amssymb, pifont}
\usepackage{mathrsfs}
\usepackage{enumitem}
\usepackage{fancyhdr}
\usepackage{hyperref}

\pagestyle{fancy}
\fancyhf{}
\rhead{MAT 354}
\lhead{Assignment 1}
\rfoot{Page \thepage}

\setlength\parindent{24pt}
\renewcommand\qedsymbol{$\blacksquare$}

\DeclareMathOperator{\Qu}{\mathcal{Q}_8}
\DeclareMathOperator{\F}{\mathbb{F}}
\DeclareMathOperator{\T}{\mathcal{T}}
\DeclareMathOperator{\V}{\mathcal{V}}
\DeclareMathOperator{\U}{\mathcal{U}}
\DeclareMathOperator{\Prt}{\mathbb{P}}
\DeclareMathOperator{\R}{\mathbb{R}}
\DeclareMathOperator{\N}{\mathbb{N}}
\DeclareMathOperator{\Z}{\mathbb{Z}}
\DeclareMathOperator{\Q}{\mathbb{Q}}
\DeclareMathOperator{\C}{\mathbb{C}}
\DeclareMathOperator{\ep}{\varepsilon}
\DeclareMathOperator{\identity}{\mathbf{0}}
\DeclareMathOperator{\card}{card}
\newcommand{\suchthat}{;\ifnum\currentgrouptype=16 \middle\fi|;}

\newtheorem{lemma}{Lemma}

\newcommand{\tr}{\mathrm{tr}}
\newcommand{\ra}{\rightarrow}
\newcommand{\lan}{\langle}
\newcommand{\ran}{\rangle}
\newcommand{\norm}[1]{\left\lVert#1\right\rVert}
\newcommand{\inn}[1]{\lan#1\ran}
\newcommand{\ol}{\overline}
\newcommand{\ci}{i}
\begin{document}
\noindent
Q7a: By the fundamental theorem of algebra, $P(z)$ factors fully over $\C$. We can write $$P(z) = c (z-b_1)\dots (z-b_n)$$
We compute the derivative of $P^\prime(z)$ as 
$$P^\prime(z) = \sum_{j=1}^n \prod_{i=1, j\neq i}^n (z_i - b_i)$$
Taking the quotient of these quantities we get 
$$\frac{P^\prime(z)}{P(z)} = \sum_{i=1}^n \frac{1}{(z_i-b_i)}$$
\newline \\ 
Q7b: Suppose that we had $$0 = \sum_{k=1}^n \frac{1}{(z-b_k)}$$
Multuplying each $\frac{1}{z-b_k}$ term by $\frac{\ol{z}-\ol{b_k}}{\ol{z} - \ol{b_k}}$, we see that 
$$0 = \sum_{k=1}^n \frac{\ol{z} - \ol{b_k}}{|z-b_k|^2} = \Big( \sum_{k=1}^n \frac{1}{|z-b_k|^2} \Big)\ol{z} - \sum_{k=1}^n \frac{\ol{b_k}}{|z-b_k|^2}$$
We conclude that 
$$\Big( \sum_{k=1}^n \frac{1}{|z-b_k|^2} \Big)\ol{z} = \sum_{k=1}^n \frac{\ol{b_k}}{|z-b_k|^2}$$
\newline \\ 
Q7c: We claim that and $z$ satisfying $P^\prime(z)=0$ is a convex linear combination of each $b_k$. Using the result from 7b and applying the complex conjugation, we get that 
$$\Big( \sum_{k=1}^n \frac{1}{|z-b_k|^2}   \Big)z = \sum_{k=1}^n \frac{b_k}{|z-b_k|^2}$$
Since $\sum_{k=1}^n \frac{1}{|z-b_k|^2}$ is nonzero we take no issue with writing the equation as 
$$z = \frac{\sum_{k=1}^n \frac{b_k}{|z-b_k|^2}}{\Big( \sum_{k=1}^n \frac{1}{|z-b_k|^2} \Big)} $$
For all $i$ denote $\frac{1}{|z-b_i|^2}$ as $c_i$. 
We see that $$z = \frac{\sum_{k=1}^n bk \cdot c_k}{\sum_{k=1}^n c_k} = \frac{1}{\sum_{k=1}^n c_k} \sum_{k=1}^n b_k c_k $$
Evaluating for the sum of the coefficients on the $b_k's$ we get that $$\frac{1}{\sum_{k=1}^n c_k} \sum_{k=1}c_k =1$$ Since the coefficient on each $b_k$ is positive, and they sum to $1$, we can conclude that $z$ is a convex linear combination of each $b_i$. Since the $b_k$'s form a convex hull, we have that $z$ must belong to the convex hull generated by $\{b_i \}$. 
\end{document}