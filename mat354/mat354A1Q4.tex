\documentclass[letterpaper]{article}
\usepackage[letterpaper,margin=1in,footskip=0.25in]{geometry}
\usepackage[utf8]{inputenc}
\usepackage{amsmath}
\usepackage{amsthm}
\usepackage{amssymb, pifont}
\usepackage{mathrsfs}
\usepackage{enumitem}
\usepackage{fancyhdr}
\usepackage{hyperref}

\pagestyle{fancy}
\fancyhf{}
\rhead{MAT 354}
\lhead{Assignment 1}
\rfoot{Page \thepage}

\setlength\parindent{24pt}
\renewcommand\qedsymbol{$\blacksquare$}

\DeclareMathOperator{\F}{\mathbb{F}}
\DeclareMathOperator{\T}{\mathcal{T}}
\DeclareMathOperator{\V}{\mathcal{V}}
\DeclareMathOperator{\U}{\mathcal{U}}
\DeclareMathOperator{\Prt}{\mathbb{P}}
\DeclareMathOperator{\R}{\mathbb{R}}
\DeclareMathOperator{\N}{\mathbb{N}}
\DeclareMathOperator{\Z}{\mathbb{Z}}
\DeclareMathOperator{\Q}{\mathbb{Q}}
\DeclareMathOperator{\C}{\mathbb{C}}
\DeclareMathOperator{\ep}{\varepsilon}
\DeclareMathOperator{\identity}{\mathbf{0}}
\DeclareMathOperator{\card}{card}
\newcommand{\suchthat}{;\ifnum\currentgrouptype=16 \middle\fi|;}

\newtheorem{lemma}{Lemma}

\newcommand{\tr}{\mathrm{tr}}
\newcommand{\ra}{\rightarrow}
\newcommand{\lan}{\langle}
\newcommand{\ran}{\rangle}
\newcommand{\norm}[1]{\left\lVert#1\right\rVert}
\newcommand{\inn}[1]{\lan#1\ran}
\newcommand{\ol}{\overline}
\newcommand{\ci}{i}
\begin{document}
\noindent
Q4a: 
Assume wlog that $Q(z) = z^n + \dots c_0 = (z-a_1)\dots(z-a_n)$. By partial fraction decomposition there exists $d_1\dots d_n$ which satisfy 
$$\frac{P(z)}{Q(z)} = \frac{d_1}{(z-a_1)} + \dots + \frac{d_n}{(z-a_n)} $$
The Heaviside Cover up Method (MAT157) tells us that each $d_i$ is determined in the following way. 
$$d_i = \frac{P(a_i)}{\Pi_{j=1, j\neq i}^n (a_i -a_j)}$$
Therefore we see that 
\begin{align*}
    \frac{P(z)}{Q(z)} & = \sum_{i=1}^n \frac{d_i}{(z-a_i)}
    \\ & = \sum_{i=1}^n \frac{P(a_i)}{(z-a_i)\Pi_{j=1, j\neq i}^n (a_i -a_j)}
\end{align*}
Since $Q^\prime(z) = \sum_{i=1}^n \Pi_{j=1,j\neq i}^n (z-a_j)$ from the product rule, we have that $Q^\prime(a_i) = \Pi_{j=1,j\neq i}(a_i - a_j)$
Therefore we get that $$\frac{P(z)}{Q(z)} =\sum_{i=1}^n \frac{P(a_i)}{Q^\prime(a_i)(z-a_i)}$$
\newline \\ Q4b: We define our polynomial $P(z)$ as $$P(z) = Q(z) \sum_{j=1}^n \frac{b_j}{Q^\prime(a_j)(z-a_j)}$$ We claim that this indeed satisfies $P(a_k)=b_k$ and it is uniquely determined by $Q(z)$. 
We see that 
$$
    P(a_k) = \sum_{j=1}^n \frac{b_j \cdot \Pi_{l=1}^n(a_k-a_l)}{\Pi_{i=1, i \neq j}(a_i-a_j)\cdot (a_k -a_j)}
$$
At the index $j=k$ we have that polynomial evaluates to $b_k$, at every other point the numerator evaluates to 0. Hence $P(a_k) = b_k$.
We now claim uniqueness. From linear algebra, we know that any polynomial of degree $n$ is completely determined by its values on at least $n+1$ points. Hence we have that $P_1(z) = P(z)$. 
\end{document}