\documentclass[letterpaper]{article}
\usepackage[letterpaper,margin=1in,footskip=0.25in]{geometry}
\usepackage[utf8]{inputenc}
\usepackage{amsmath}
\usepackage{amsthm}
\usepackage{amssymb, pifont}
\usepackage{mathrsfs}
\usepackage{enumitem}
\usepackage{fancyhdr}
\usepackage{hyperref}

\pagestyle{fancy}
\fancyhf{}
\rhead{MAT 354}
\lhead{Assignment 2}
\rfoot{Page \thepage}

\setlength\parindent{24pt}
\renewcommand\qedsymbol{$\blacksquare$}

\DeclareMathOperator{\Qu}{\mathcal{Q}_8}
\DeclareMathOperator{\F}{\mathbb{F}}
\DeclareMathOperator{\T}{\mathcal{T}}
\DeclareMathOperator{\V}{\mathcal{V}}
\DeclareMathOperator{\U}{\mathcal{U}}
\DeclareMathOperator{\Prt}{\mathbb{P}}
\DeclareMathOperator{\R}{\mathbb{R}}
\DeclareMathOperator{\N}{\mathbb{N}}
\DeclareMathOperator{\Z}{\mathbb{Z}}
\DeclareMathOperator{\Q}{\mathbb{Q}}
\DeclareMathOperator{\C}{\mathbb{C}}
\DeclareMathOperator{\ep}{\varepsilon}
\DeclareMathOperator{\identity}{\mathbf{0}}
\DeclareMathOperator{\card}{card}
\newcommand{\suchthat}{;\ifnum\currentgrouptype=16 \middle\fi|;}

\newtheorem{lemma}{Lemma}

\newcommand{\tr}{\mathrm{tr}}
\newcommand{\ra}{\rightarrow}
\newcommand{\lan}{\langle}
\newcommand{\ran}{\rangle}
\newcommand{\norm}[1]{\left\lVert#1\right\rVert}
\newcommand{\inn}[1]{\lan#1\ran}
\newcommand{\ol}{\overline}
\newcommand{\ci}{i}
\begin{document}
\noindent
Q2: We first take note that $$\frac{2}{(1-w)^3} =  \frac{\partial^2}{\partial w^2} \frac{1}{1-w}$$
We can therefore have that $$\frac{2}{(1-w)^3} =  \frac{\partial^2}{\partial w^2} \cdot \sum_{n=0}^\infty w^n$$
Applying termwise differentiation, get that $$\frac{2}{(1-w)^3} = \sum_{n=2}^\infty n(n-1)w^n$$
Substituting $w = z^2$, get that 
$$\frac{2}{(1-z^2)^3} = \sum_{n=2}^\infty n(n-1)z^{2n}$$
We now multiply by $z^2$ and conclude that 
$$\frac{z^2}{(1-z^2)^3} = \sum_{n=2}^\infty \frac{n(n-1)}{2}z^{2n+2}$$
Now using Hadamards Formula, we compute the radius of convergence as 
\begin{align*}
    \frac{1}{R} & = \lim_{n \to \infty} \sup \sqrt[n]{\Big|\frac{n(n-1)}{2} \Big|}
    \\ & = \lim_{n \to \infty} \sup \sqrt[n]{n(n-1)} \cdot \lim_{n \to \infty} \sup \sqrt[n]{\frac{1}{2}}
    \\ & =1
\end{align*}Hence this power series converges on $|z|<1$ and diverges everywhere else. 
\end{document}