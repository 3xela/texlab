\documentclass[letterpaper]{article}
\usepackage[letterpaper,margin=1in,footskip=0.25in]{geometry}
\usepackage[utf8]{inputenc}
\usepackage{amsmath}
\usepackage{amsthm}
\usepackage{amssymb, pifont}
\usepackage{mathrsfs}
\usepackage{enumitem}
\usepackage{fancyhdr}
\usepackage{hyperref}

\pagestyle{fancy}
\fancyhf{}
\rhead{MAT 354}
\lhead{Assignment 2}
\rfoot{Page \thepage}

\setlength\parindent{24pt}
\renewcommand\qedsymbol{$\blacksquare$}

\DeclareMathOperator{\Qu}{\mathcal{Q}_8}
\DeclareMathOperator{\F}{\mathbb{F}}
\DeclareMathOperator{\T}{\mathcal{T}}
\DeclareMathOperator{\V}{\mathcal{V}}
\DeclareMathOperator{\U}{\mathcal{U}}
\DeclareMathOperator{\Prt}{\mathbb{P}}
\DeclareMathOperator{\R}{\mathbb{R}}
\DeclareMathOperator{\N}{\mathbb{N}}
\DeclareMathOperator{\Z}{\mathbb{Z}}
\DeclareMathOperator{\Q}{\mathbb{Q}}
\DeclareMathOperator{\C}{\mathbb{C}}
\DeclareMathOperator{\ep}{\varepsilon}
\DeclareMathOperator{\identity}{\mathbf{0}}
\DeclareMathOperator{\card}{card}
\newcommand{\suchthat}{;\ifnum\currentgrouptype=16 \middle\fi|;}

\newtheorem{lemma}{Lemma}

\newcommand{\tr}{\mathrm{tr}}
\newcommand{\ra}{\rightarrow}
\newcommand{\lan}{\langle}
\newcommand{\ran}{\rangle}
\newcommand{\norm}[1]{\left\lVert#1\right\rVert}
\newcommand{\inn}[1]{\lan#1\ran}
\newcommand{\ol}{\overline}
\newcommand{\ci}{i}
\begin{document}
\noindent
Q1a: Using Hadamards Formula we compute:
\begin{align*}
    \frac{1}{R} & = \lim_{n \to \infty} \sup \sqrt[n]{|a_n|}
    \\ & = \lim_{n \to \infty} \sup \sqrt[n]{|q^{n^2}|}
    \\ & = \lim_{n \to \infty} \sup \sqrt[n]{|q|^{n^2}}
    \\ & = \lim_{n \to \infty} \sup |q|^n
    \\ & = 0 \tag{since $|q|<1$ and hence its power approaches 0}
\end{align*}
We conclude that $R = \infty$ i.e. the power series converges everywhere
\newline \\ \noindent
Q1b: Using Hadamards Formula, we compute:
\begin{align*}
    \frac{1}{R} & = \lim_{n \to \infty} \sup \sqrt[n]{|a_n|}
    \\ & = \lim_{n \to \infty} \sup \sqrt[n]{|n^p|}
    \\ & = \lim_{n \to \infty} \sup \sqrt[n]{|n|}^p 
    \\ & = (\lim_{n \to \infty} \sup \sqrt[n]{|n|})^p
    \\ & = 1
\end{align*} Hence $R=1$. 
\newline \\ \noindent
Q1c: Using Hadamards formula, we compute: $$\frac{1}{R} = \lim_{n \to \infty} sup \sqrt[n]{|a_n|} = \lim_{n\to \infty} \sup \{\sqrt[2n+1]{a^{2n+1}} , \sqrt[2n]{b^{2n}}\} = \lim_{n \to \infty } \sup \{a,b\} = \max\{a,b\}$$
Therefore we get that $$R = \frac{1}{\max\{a,b\}}$$

\end{document}