\documentclass[letterpaper]{article}
\usepackage[letterpaper,margin=1in,footskip=0.25in]{geometry}
\usepackage[utf8]{inputenc}
\usepackage{amsmath}
\usepackage{amsthm}
\usepackage{amssymb, pifont}
\usepackage{mathrsfs}
\usepackage{enumitem}
\usepackage{fancyhdr}
\usepackage{hyperref}

\pagestyle{fancy}
\fancyhf{}
\rhead{MAT 354}
\lhead{Assignment 1}
\rfoot{Page \thepage}

\setlength\parindent{24pt}
\renewcommand\qedsymbol{$\blacksquare$}

\DeclareMathOperator{\F}{\mathbb{F}}
\DeclareMathOperator{\T}{\mathcal{T}}
\DeclareMathOperator{\V}{\mathcal{V}}
\DeclareMathOperator{\U}{\mathcal{U}}
\DeclareMathOperator{\Prt}{\mathbb{P}}
\DeclareMathOperator{\R}{\mathbb{R}}
\DeclareMathOperator{\N}{\mathbb{N}}
\DeclareMathOperator{\Z}{\mathbb{Z}}
\DeclareMathOperator{\Q}{\mathbb{Q}}
\DeclareMathOperator{\C}{\mathbb{C}}
\DeclareMathOperator{\ep}{\varepsilon}
\DeclareMathOperator{\identity}{\mathbf{0}}
\DeclareMathOperator{\card}{card}
\newcommand{\suchthat}{;\ifnum\currentgrouptype=16 \middle\fi|;}

\newtheorem{lemma}{Lemma}

\newcommand{\tr}{\mathrm{tr}}
\newcommand{\ra}{\rightarrow}
\newcommand{\lan}{\langle}
\newcommand{\ran}{\rangle}
\newcommand{\norm}[1]{\left\lVert#1\right\rVert}
\newcommand{\inn}[1]{\lan#1\ran}
\newcommand{\ol}{\overline}
\newcommand{\ci}{i}
\begin{document}
\noindent
Q3: We will show that any such function $f$ which preserves norms and maps $0$ to $0$ is a rotation or a rotation and conjugation. By properties of $f$, we have that 
$$|f(z) - f(1) |^2 = |z-1|^2$$
Using the properties of the norm, we know that 
$$[\ol{f(z)} - \ol{f(1)}][f(z) -f(1)] = [\ol{z} - 1][z-1]$$
Expanding, we see that 
$$|f(z)|^2 - \ol{f(z)} f(1) - \ol{f(1)} f(z) + |f(1)|^2 = |z|^2 - \ol{z} - z +1$$
Using the distance preserving properties, we get that 
$$\ol{f(z)} f(1) + \ol{f(1)}f(z) = \ol{z} + z = 2Re(z)$$
Since $|f(1)| =1$, we can write $f(1) = e^{\ci \theta}$ for some $\theta$. We get that 
$$\ol{f(z)} e^{\ci \theta} + f(z) e^{- \ci \theta} = \ol{f(z) e^{- \ci \theta }} + f(z)e^{- \ci \theta} =2Re(z)$$
If we let $z= a+ \ci b$ We get that 
$$2Re(f(z) e^{-\ci \theta}) = 2Re(z) = 2a$$
And so,
$$Re(f(z) e^{-\ci \theta}) = a = Re(z)$$
By the norm preserving property, we see that 
$$Re(f(z) e^{-\ci \theta})^2 + Im(f(z) e^{-\ci \theta})^2 = a^2+ b^2$$
And thus 
$$ Im(f(z) e^{-\ci \theta}) = \pm b $$
We see that if it is the case that $Im(f(z) e^{-\ci \theta}) = b$, then $f(z) = e^{\ci \theta} (a+\ci b)$ and so $f$ is a rotation by $\theta$. If $Im(f(z) e^{-\ci \theta}) = -b$ then we have that $f(z) = e^{\ci \theta} ( a- \ci b)$, which is exactly a rotation by $\theta$ and complex conjugation. Note that it is not the case that for some $z_1$ that $f(z_1) = \ol{e^{i \theta}z_1}$ but for all other $z$, $f(z) = e^{i \theta}z$ since $f$ would no longer be continuous, since we can find a neighbourhood of $z_1$ that does not get carried to a neighbourhood of $f(z_1)$. 

\end{document}