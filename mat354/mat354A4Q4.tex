\documentclass[letterpaper]{article}
\usepackage[letterpaper,margin=1in,footskip=0.25in]{geometry}
\usepackage[utf8]{inputenc}
\usepackage{amsmath}
\usepackage{amsthm}
\usepackage{amssymb, pifont}
\usepackage{mathrsfs}
\usepackage{enumitem}
\usepackage{fancyhdr}
\usepackage{hyperref}

\pagestyle{fancy}
\fancyhf{}
\rhead{MAT 354}
\lhead{Assignment 4}
\rfoot{Page \thepage}

\setlength\parindent{24pt}
\renewcommand\qedsymbol{$\blacksquare$}

\DeclareMathOperator{\Qu}{\mathcal{Q}_8}
\DeclareMathOperator{\F}{\mathbb{F}}
\DeclareMathOperator{\T}{\mathcal{T}}
\DeclareMathOperator{\V}{\mathcal{V}}
\DeclareMathOperator{\U}{\mathcal{U}}
\DeclareMathOperator{\Prt}{\mathbb{P}}
\DeclareMathOperator{\R}{\mathbb{R}}
\DeclareMathOperator{\N}{\mathbb{N}}
\DeclareMathOperator{\Z}{\mathbb{Z}}
\DeclareMathOperator{\Q}{\mathbb{Q}}
\DeclareMathOperator{\C}{\mathbb{C}}
\DeclareMathOperator{\ep}{\varepsilon}
\DeclareMathOperator{\identity}{\mathbf{0}}
\DeclareMathOperator{\card}{card}
\newcommand{\suchthat}{;\ifnum\currentgrouptype=16 \middle\fi|;}

\newtheorem{lemma}{Lemma}

\newcommand{\tr}{\mathrm{tr}}
\newcommand{\ra}{\rightarrow}
\newcommand{\lan}{\langle}
\newcommand{\ran}{\rangle}
\newcommand{\norm}[1]{\left\lVert#1\right\rVert}
\newcommand{\inn}[1]{\lan#1\ran}
\newcommand{\ol}{\overline}
\newcommand{\ci}{i}
\begin{document}
\noindent Q4: Take $\gamma$ to be sufficiently large. Using Cauchy's Integral formula, we can write $g(z)$ as 
$$g(z) = \frac{1}{2\pi i} \sum_{m=0}^\infty z^m \int_{\gamma} \frac{f(\zeta)}{\zeta^{m+1}}d\zeta - \frac{1}{2 \pi i} \int_{\gamma} \frac{f(\zeta)}{\zeta-z} \cdot \frac{z^n}{\zeta^n} $$
We can write $$\frac{1}{\zeta - z} = \frac{1}{\zeta} \sum_{m=0}^\infty \frac{z^m}{\zeta^m}.$$
We have that $g(z)$ becomes $$g(z) = \frac{1}{2\pi i} \sum_{m=0}^\infty z^m \int_\gamma \frac{f(\zeta)}{\zeta^{m+1}} d\zeta - \frac{1}{2 \pi i} \sum_{m=n}^\infty z^{m} \int_{\gamma} \frac{f(\zeta)}{\zeta^{m+1}} d\zeta = \frac{1}{2\pi i} \Big( \sum_{m=0}^{n-1}z^{m} \int_{\gamma} \frac{f(\zeta)}{\zeta^{m+1}} d\zeta \Big)$$
Which is clearly a polynomial of degree $n-1$. Using Cauchy's Integral Formula to construct a power series for $f$, we have that for $k \leq n-1$, $$g^{(k)}(0) = \frac{k!}{2\pi i} \int_{\gamma}\frac{f(\zeta)}{\zeta^{k+1}} d \zeta = f^{(k)}(0). $$ As desired. 

\end{document}