\documentclass[letterpaper]{article}
\usepackage[letterpaper,margin=1in,footskip=0.25in]{geometry}
\usepackage[utf8]{inputenc}
\usepackage{amsmath}
\usepackage{amsthm}
\usepackage{amssymb, pifont}
\usepackage{mathrsfs}
\usepackage{enumitem}
\usepackage{fancyhdr}
\usepackage{hyperref}

\pagestyle{fancy}
\fancyhf{}
\rhead{MAT 354}
\lhead{Assignment 5}
\rfoot{Page \thepage}

\setlength\parindent{24pt}
\renewcommand\qedsymbol{$\blacksquare$}

\DeclareMathOperator{\Qu}{\mathcal{Q}_8}
\DeclareMathOperator{\F}{\mathbb{F}}
\DeclareMathOperator{\T}{\mathcal{T}}
\DeclareMathOperator{\V}{\mathcal{V}}
\DeclareMathOperator{\U}{\mathcal{U}}
\DeclareMathOperator{\Prt}{\mathbb{P}}
\DeclareMathOperator{\R}{\mathbb{R}}
\DeclareMathOperator{\N}{\mathbb{N}}
\DeclareMathOperator{\Z}{\mathbb{Z}}
\DeclareMathOperator{\Q}{\mathbb{Q}}
\DeclareMathOperator{\C}{\mathbb{C}}
\DeclareMathOperator{\ep}{\varepsilon}
\DeclareMathOperator{\identity}{\mathbf{0}}
\DeclareMathOperator{\card}{card}
\newcommand{\suchthat}{;\ifnum\currentgrouptype=16 \middle\fi|;}

\newtheorem{lemma}{Lemma}

\newcommand{\tr}{\mathrm{tr}}
\newcommand{\ra}{\rightarrow}
\newcommand{\lan}{\langle}
\newcommand{\ran}{\rangle}
\newcommand{\norm}[1]{\left\lVert#1\right\rVert}
\newcommand{\inn}[1]{\lan#1\ran}
\newcommand{\ol}{\overline}
\newcommand{\ci}{i}
\begin{document}
\noindent Q4a: We wish to compute the following integral using residue calculus, $$\int_{[0,2\pi]} \cos^{2n}\theta d\theta.$$
Substituting $\cos(z) = \frac{1}{2}(z+\frac{1}{z})$ on $S^1$, we get the integral $$\frac{-i}{2^{2n}} \int_{S^1} \frac{1}{z} \Big( \frac{z^2+1}{z}\Big)^{2n} dz.$$
Using the binomial expansion, we compute that $$\frac{1}{z}\Big(\frac{z^2+1}{z} \Big)^{2n} = \frac{1}{z} \sum_{k=0}^{2n}\frac{(z^2)^{2n-k}}{z^{2n}}. $$
Notice that the only place in this expansion that we have a term of the form $\frac{1}{z}$ will be when $k=n$. Thus the laurent series for each polynomial will have a zero coefficient on $a_{-1}$ except when $\begin{pmatrix}
    2n \\ n
\end{pmatrix}$ Hence the residue of the integral will be $\begin{pmatrix}
    2n \\ n
\end{pmatrix}$ Thus by the residue theorem, we have that $$\int_{[0,2\pi]} \cos^{2n}(\theta) d\theta = \frac{2\pi i\cdot -i \cdot \begin{pmatrix} 2n \\ n \end{pmatrix} }{2^{2n}} = \frac{\pi \begin{pmatrix} 2n \\ n \end{pmatrix} }{2^{2n-1}}$$
\newline \\ Q4b: We wish to compute $$\int_{[0,\infty]} \frac{x^2-a^2}{x^2+a^2}\frac{\sin(x)}{x}dx. $$ The integrand is even so it is equal to $$\frac{1}{2} \int_{\R} \frac{x^2-a^2}{x^2+a^2}\frac{\sin(x)}{x}dx.$$ We will compute this integral as $$Im \Big( \frac{1}{2} \int_{\C} \frac{z^2-a^2}{z^2+a^2}\frac{e^{iz}}{z}dz \Big). $$ This makes sense to integrate since $$\Big|\frac{z^2-a^2}{z^2+a^2} \Big|< \infty, \forall z\in \C .$$ 
 Note that the integrand has poles at $\pm ia, 0$. Thus by the residue formula we have $$\frac{1}{2}\int_{\C} \frac{z^2-a^2}{z^2+a^2}\frac{e^{iz}}{z}dz = 2\pi i\sum Res(f) = \pi i \Big( \frac{(-a^2 - a^2)e^{-a}}{-3a^2 + a^2} + \frac{-a^2 e^{0}}{2a^2} = \pi ie^{-a} - \frac{\pi}{2} i \Big)$$
Therefore $$\int_{[0,\infty]} \frac{x^2-a^2}{x^2+a^2} \frac{\sin(x)}{x}dx  = \pi e^{-a} - \frac{\pi}{2}.$$
\end{document}