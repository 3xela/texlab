\documentclass[letterpaper]{article}
\usepackage[letterpaper,margin=1in,footskip=0.25in]{geometry}
\usepackage[utf8]{inputenc}
\usepackage{amsmath}
\usepackage{amsthm}
\usepackage{amssymb, pifont}
\usepackage{mathrsfs}
\usepackage{enumitem}
\usepackage{fancyhdr}
\usepackage{hyperref}

\pagestyle{fancy}
\fancyhf{}
\rhead{MAT 354}
\lhead{Assignment 5}
\rfoot{Page \thepage}

\setlength\parindent{24pt}
\renewcommand\qedsymbol{$\blacksquare$}

\DeclareMathOperator{\Qu}{\mathcal{Q}_8}
\DeclareMathOperator{\F}{\mathbb{F}}
\DeclareMathOperator{\T}{\mathcal{T}}
\DeclareMathOperator{\V}{\mathcal{V}}
\DeclareMathOperator{\U}{\mathcal{U}}
\DeclareMathOperator{\Prt}{\mathbb{P}}
\DeclareMathOperator{\R}{\mathbb{R}}
\DeclareMathOperator{\N}{\mathbb{N}}
\DeclareMathOperator{\Z}{\mathbb{Z}}
\DeclareMathOperator{\Q}{\mathbb{Q}}
\DeclareMathOperator{\C}{\mathbb{C}}
\DeclareMathOperator{\ep}{\varepsilon}
\DeclareMathOperator{\identity}{\mathbf{0}}
\DeclareMathOperator{\card}{card}
\newcommand{\suchthat}{;\ifnum\currentgrouptype=16 \middle\fi|;}

\newtheorem{lemma}{Lemma}

\newcommand{\tr}{\mathrm{tr}}
\newcommand{\ra}{\rightarrow}
\newcommand{\lan}{\langle}
\newcommand{\ran}{\rangle}
\newcommand{\norm}[1]{\left\lVert#1\right\rVert}
\newcommand{\inn}[1]{\lan#1\ran}
\newcommand{\ol}{\overline}
\newcommand{\ci}{i}
\begin{document}
\noindent Q6a: Writing $f(x+iz) = w(x,y)+ iv(x,y)$, we get that $$F = w^2 + v^2 = u^2. $$ Using the chain rule, we compute that $$\frac{\partial F}{\partial x} = 2u \cdot u_x,$$ and so $$u_x = \frac{\partial F}{\partial x}\cdot \frac{1}{2u}.$$
Using this formula, we get that $$u_x = \frac{1}{2|f|} \cdot \Big[ 2w\cdot w_x+ 2v\cdot v_x \Big] = \frac{1}{|f|} Re\Big( w\cdot w_x + i u\cdot v_x -iv\cdot u_x + v\cdot v_x \Big) = \frac{1}{|f|} Re(\ol{f}f^\prime). $$
Similarly we have that $$u_y = \frac{1}{2u} \frac{\partial F}{\partial y}.$$
We run through an almost identical computation to see that $$u_y = \frac{1}{2|f|} \Big[ 2w \cdot w_y + 2v \cdot v_y \Big] = - \frac{1}{|f|} Im \Big( w \cdot v_y - i w \cdot w_y -iv \cdot v_y - v \cdot w_y \Big) = -\frac{1}{|f|}Im(\ol{f} f^\prime).$$
\newline \\ We now wish to compute $$F_{xx} + F_{yy}.$$
Using simple calculus, we can write this as $$F_{xx} + F_{yy} = 2u_x^2 + 2u u_{xx} + 2u_y^2 + 2u_{yy} = 2[u_x^2 + u_y^2]^2 + 2u[u_{xx} + u_{yy}].$$
We compute that 
\begin{align*}
    F_{xx} + F_{yy} & = 2[u_x^2 + u_y^2]^2 + 2u[u_{xx} + u_{yy}] 
    \\ & = \frac{2}{u^2} \Big[Re(\ol{f} f^\prime)^2 + Im(\ol{f} f^\prime)^2 \Big] + 2u \Big[\Big(\frac{ww_x + vv_x}{u} \Big)_x +  \Big( \frac{ww_y + vv_y}{u} \Big)_y\Big]
    \\ & = \frac{2}{u^2} \Big[(v^2+w^2)(w_x^2 + v_x^2) \Big] + 2u \Big[ \frac{u(w_x^2+v_2^2+ w_y^2 + v_y^2)}{u^2} - u\frac{(w_x^2 + v_x^2)}{u^3} \Big] \tag{simplifying above }
    \\ & = 4(w_x^2 + v_x^2)
    \\ & = 4|f^\prime(z)|
\end{align*}
\newline \\ Q6b: Since $g$ is holomorphic, its real and imaginary parts are both holomorphic and thus both harmonic. Thus using the previous result, we get $$0 = Re(g(z))^2_{xx} + Re (g(z))^2+{yy} = |f(z)|^2_{xx} + |f(z)|^2_{yy} = 4|f^\prime(z)|^2.$$ Thus we have that for all $z$, $|f^\prime(z)|=0$ and so $f^\prime(z)=0. $ It has been shown that any such holomorphic $f$ is necessarily constant. Furthermore, once again by previous results we have that $$Re(g(z))_x = Re(g(z))_y =0.$$ Holomorphicity of $g$ implies that $g$ is constant, from the Cauchy-Riemann Equations.  
\end{document}