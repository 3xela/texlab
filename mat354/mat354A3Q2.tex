\documentclass[letterpaper]{article}
\usepackage[letterpaper,margin=1in,footskip=0.25in]{geometry}
\usepackage[utf8]{inputenc}
\usepackage{amsmath}
\usepackage{amsthm}
\usepackage{amssymb, pifont}
\usepackage{mathrsfs}
\usepackage{enumitem}
\usepackage{fancyhdr}
\usepackage{hyperref}

\pagestyle{fancy}
\fancyhf{}
\rhead{MAT 354}
\lhead{Assignment 3}
\rfoot{Page \thepage}

\setlength\parindent{24pt}
\renewcommand\qedsymbol{$\blacksquare$}

\DeclareMathOperator{\Qu}{\mathcal{Q}_8}
\DeclareMathOperator{\F}{\mathbb{F}}
\DeclareMathOperator{\T}{\mathcal{T}}
\DeclareMathOperator{\V}{\mathcal{V}}
\DeclareMathOperator{\U}{\mathcal{U}}
\DeclareMathOperator{\Prt}{\mathbb{P}}
\DeclareMathOperator{\R}{\mathbb{R}}
\DeclareMathOperator{\N}{\mathbb{N}}
\DeclareMathOperator{\Z}{\mathbb{Z}}
\DeclareMathOperator{\Q}{\mathbb{Q}}
\DeclareMathOperator{\C}{\mathbb{C}}
\DeclareMathOperator{\ep}{\varepsilon}
\DeclareMathOperator{\identity}{\mathbf{0}}
\DeclareMathOperator{\card}{card}
\newcommand{\suchthat}{;\ifnum\currentgrouptype=16 \middle\fi|;}

\newtheorem{lemma}{Lemma}

\newcommand{\tr}{\mathrm{tr}}
\newcommand{\ra}{\rightarrow}
\newcommand{\lan}{\langle}
\newcommand{\ran}{\rangle}
\newcommand{\norm}[1]{\left\lVert#1\right\rVert}
\newcommand{\inn}[1]{\lan#1\ran}
\newcommand{\ol}{\overline}
\newcommand{\ci}{i}
\begin{document}
\noindent
Q2: First let $\alpha,\beta$ be the radii of the concentric circles, and assume that $\alpha<\beta$. If $$f = \frac{az+ b}{cz+d}$$ is a fractional linear transformation with the supposed properties, we can assume without loss of generality that the circles $S_{\alpha}, S_{\beta}$ are centered about 0, and $f(S_\alpha)$ and $f(S_\beta)$ will also be centered around 0, since we can always translate the images to 0 while $f$ will remain a fractional linear transformation.
Consider any pair of lines $l_1(t), l_2(t) $ which intersect at a right angle and pass through the origin. 
Note that they must also intersect at $\infty$. Since $f$ is a conformal mapping and it takes lines to lines, we have that $f(l_1(t))$ and $f(l_2(t))$ must also intersect at a right angle at the origin. 
Therefore one of $2$ cases must hold. Either $f(0) =0, f(\infty) = \infty , $ or $f(\infty) = 0$ and $f(0) = \infty$. Consider the first case. If we have that $f(0)=0$, this implies that $b=0$. Furthermore since $f^{-1}(\infty) = - \frac{d}{c} = \infty$, this implies that $c = 0$. Hence $f$ takes the form $$f(z) = \frac{az}{d} = \lambda z$$ for $\lambda = \frac{a}{d}$. We compute that $$\frac{f(\beta)}{f(\alpha)} = \frac{\lambda \beta}{\lambda \alpha} = \frac{\beta}{\alpha}$$
Now consider the second case. If we have that $f(0) = \infty$, then it must be that $d = 0$. Since $f^{-1}(0) = -\frac{b}{a} = \infty$, this implies that $a =0$. Hence $f$ takes the form $$f(z) = \frac{b}{cz} = \frac{\lambda}{z}. $$
We can evaluate that $$\frac{f(\beta)}{f(\alpha)} = \frac{\frac{\lambda}{\beta}}{\frac{\lambda}{\alpha}} = \frac{\alpha}{\beta}$$
The ratio is conserved. Thus we are done. 
\end{document}