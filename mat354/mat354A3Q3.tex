\documentclass[letterpaper]{article}
\usepackage[letterpaper,margin=1in,footskip=0.25in]{geometry}
\usepackage[utf8]{inputenc}
\usepackage{amsmath}
\usepackage{amsthm}
\usepackage{amssymb, pifont}
\usepackage{mathrsfs}
\usepackage{enumitem}
\usepackage{fancyhdr}
\usepackage{hyperref}

\pagestyle{fancy}
\fancyhf{}
\rhead{MAT 354}
\lhead{Assignment 3}
\rfoot{Page \thepage}

\setlength\parindent{24pt}
\renewcommand\qedsymbol{$\blacksquare$}

\DeclareMathOperator{\Qu}{\mathcal{Q}_8}
\DeclareMathOperator{\F}{\mathbb{F}}
\DeclareMathOperator{\T}{\mathcal{T}}
\DeclareMathOperator{\V}{\mathcal{V}}
\DeclareMathOperator{\U}{\mathcal{U}}
\DeclareMathOperator{\Prt}{\mathbb{P}}
\DeclareMathOperator{\R}{\mathbb{R}}
\DeclareMathOperator{\N}{\mathbb{N}}
\DeclareMathOperator{\Z}{\mathbb{Z}}
\DeclareMathOperator{\Q}{\mathbb{Q}}
\DeclareMathOperator{\C}{\mathbb{C}}
\DeclareMathOperator{\ep}{\varepsilon}
\DeclareMathOperator{\identity}{\mathbf{0}}
\DeclareMathOperator{\card}{card}
\newcommand{\suchthat}{;\ifnum\currentgrouptype=16 \middle\fi|;}

\newtheorem{lemma}{Lemma}

\newcommand{\tr}{\mathrm{tr}}
\newcommand{\ra}{\rightarrow}
\newcommand{\lan}{\langle}
\newcommand{\ran}{\rangle}
\newcommand{\norm}[1]{\left\lVert#1\right\rVert}
\newcommand{\inn}[1]{\lan#1\ran}
\newcommand{\ol}{\overline}
\newcommand{\ci}{i}
\begin{document}
\noindent
Q3: Note that we can write $w$ as a composition of conformal mappings in the following way. We define $u = 1-z, \xi = u^{\frac{1}{4}}, v = \frac{1-\xi}{1+\xi}$. We see that $$w = v \circ \xi \circ u = \frac{1-(1-z)^{\frac{1}{4}}}{ 1+ (1-z)^{\frac{1}{4}}}$$
We wish to find a $z$ such that $Re(w(z))=0$. Note that since the FLT $w(\xi) = \frac{1+ \xi}{1-\xi}$ acts on the region $\{x+iy :x\geq 0,x\geq y, y\geq -x  \}$ onto the lens shaped region homomorhpically, it must take the boundary to the boundary. Therefore to find the height of the region, we solve for when $$Re(w(a+ia)) =0$$
We know that $$Re(w(a+ia)) = 0 \iff \frac{1- (a+ia)}{1+(a+ia)} + \frac{1-(a-ia)}{1+ (a-ia)} =0.$$
Which implies that $$(1-a-ia)(1+a-ia) + (1-a+ia)(1+a+ia)=0. $$
Solving gives us that $$a = \pm \frac{1}{\sqrt{2}}$$
We take the positive root since we are working over the right half of the complex plane. 
We therefore compute $$w(\frac{1}{\sqrt{2}} + i \frac{1}{\sqrt{2}}) = -i(\sqrt{2}-1),$$ 
and similarly $$w(\frac{1}{\sqrt{2}} - i \frac{1}{\sqrt{2}}) = i(\sqrt{2}-1)$$
Hence the height of this lense is $2\sqrt{2} -2$.
\end{document}