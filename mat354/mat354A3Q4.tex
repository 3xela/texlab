\documentclass[letterpaper]{article}
\usepackage[letterpaper,margin=1in,footskip=0.25in]{geometry}
\usepackage[utf8]{inputenc}
\usepackage{amsmath}
\usepackage{amsthm}
\usepackage{amssymb, pifont}
\usepackage{mathrsfs}
\usepackage{enumitem}
\usepackage{fancyhdr}
\usepackage{hyperref}

\pagestyle{fancy}
\fancyhf{}
\rhead{MAT 354}
\lhead{Assignment 3}
\rfoot{Page \thepage}

\setlength\parindent{24pt}
\renewcommand\qedsymbol{$\blacksquare$}

\DeclareMathOperator{\Qu}{\mathcal{Q}_8}
\DeclareMathOperator{\F}{\mathbb{F}}
\DeclareMathOperator{\T}{\mathcal{T}}
\DeclareMathOperator{\V}{\mathcal{V}}
\DeclareMathOperator{\U}{\mathcal{U}}
\DeclareMathOperator{\Prt}{\mathbb{P}}
\DeclareMathOperator{\R}{\mathbb{R}}
\DeclareMathOperator{\N}{\mathbb{N}}
\DeclareMathOperator{\Z}{\mathbb{Z}}
\DeclareMathOperator{\Q}{\mathbb{Q}}
\DeclareMathOperator{\C}{\mathbb{C}}
\DeclareMathOperator{\ep}{\varepsilon}
\DeclareMathOperator{\identity}{\mathbf{0}}
\DeclareMathOperator{\card}{card}
\newcommand{\suchthat}{;\ifnum\currentgrouptype=16 \middle\fi|;}

\newtheorem{lemma}{Lemma}

\newcommand{\tr}{\mathrm{tr}}
\newcommand{\ra}{\rightarrow}
\newcommand{\lan}{\langle}
\newcommand{\ran}{\rangle}
\newcommand{\norm}[1]{\left\lVert#1\right\rVert}
\newcommand{\inn}[1]{\lan#1\ran}
\newcommand{\ol}{\overline}
\newcommand{\ci}{i}
\begin{document}
\noindent Q4: We first map $\C \setminus [-1,1]$ to $\C \setminus [-\infty, 0]$ using the conformal mapping $f_1(z)$ defined by $$z \mapsto \frac{z+1}{z-1}.$$
Now we map $C\setminus [-\infty , 0]$ to $\C \setminus \{z= x+ iy : x<0, \arg(z) \in [\pi - \arccos(r), \pi + \arccos(r)] \}$, using the conformal mapping $f_2(z)$ defined by $$z \mapsto z^{1 - \frac{\arccos(r)}{\pi}}.$$
We choose the power of $z$ such that the boundary of $C \setminus \{z= x+ iy : x<0, \arg(z) \in [\pi - \arccos(r), \pi + \arccos(r)] \}$ gets sent to the boundary of the lense by our choice of $f_4$. 
We now apply $f_3(z) = -z$, which is conformal, to rotate the plane. We finally apply $$f_4(z) = \frac{1-z}{1+z}$$ to transform this unbounded region into the compliment of the lense. Hence we take $$f = f_4 \circ f_3 \circ f_2 \circ f_1,$$ to be our conformal mapping of the given region 
\end{document}