\documentclass[12pt, a4paper]{article}
\usepackage[lmargin =0.5 in, 
rmargin=0.5in, 
tmargin=1in,
bmargin=0.5in]{geometry}
\geometry{letterpaper}
\usepackage{amsmath}
\usepackage{amssymb}
\usepackage{blindtext}
\usepackage{titlesec}
\usepackage{enumitem}
\usepackage{fancyhdr}
\usepackage{amsthm}
\usepackage{graphicx}
\usepackage{cool}
\usepackage{thmtools}
\usepackage{hyperref}
\graphicspath{ }					%path to an image

%-------- sexy font ------------%
%\usepackage{libertine}
%\usepackage{libertinust1math}

%\usepackage{mlmodern}				% very nice and classic
%\usepackage[utopia]{mathdesign}
%\usepackage[T1]{fontenc}


\usepackage{mlmodern}
\usepackage{eulervm}
%\usepackage{tgtermes} 				%times new roman
%-------- sexy font ------------%


% Problem Styles
%====================================================================%


\newtheorem{problem}{Problem}


\theoremstyle{definition}
\newtheorem{thm}{Theorem}
\newtheorem{lemma}{Lemma}
\newtheorem{prop}{Proposition}
\newtheorem{cor}{Corollary}
\newtheorem{fact}{Fact}
\newtheorem{defn}{Definition}
\newtheorem{example}{Example}
\newtheorem{question}{Question}

\newtheorem{manualprobleminner}{Problem}

\newenvironment{manualproblem}[1]{%
	\renewcommand\themanualprobleminner{#1}%
	\manualprobleminner
}{\endmanualprobleminner}

\newcommand{\penum}{ \begin{enumerate}[label=\bf(\alph*), leftmargin=0pt]}
	\newcommand{\epenum}{ \end{enumerate} }

% Math fonts shortcuts
%====================================================================%

\newcommand{\ring}{\mathcal{R}}
\newcommand{\N}{\mathbb{N}}                           % Natural numbers
\newcommand{\Z}{\mathbb{Z}}                           % Integers
\newcommand{\R}{\mathbb{R}}                           % Real numbers
\newcommand{\C}{\mathbb{C}}                           % Complex numbers
\newcommand{\F}{\mathbb{F}}                           % Arbitrary field
\newcommand{\Q}{\mathbb{Q}}                           % Arbitrary field
\newcommand{\PP}{\mathcal{P}}                         % Partition
\newcommand{\M}{\mathcal{M}}                         % Mathcal M
\newcommand{\eL}{\mathcal{L}}                         % Mathcal L
\newcommand{\T}{\mathcal{T}}                         % Mathcal T
\newcommand{\U}{\mathcal{U}}                         % Mathcal U\\
\newcommand{\V}{\mathcal{V}}                         % Mathcal V

% symbol shortcuts
%====================================================================%

\newcommand{\lam}{\lambda}
\newcommand{\imp}{\implies}
\newcommand{\all}{\forall}
\newcommand{\exs}{\exists}
\newcommand{\delt}{\delta}
\newcommand{\eps}{\varepsilon}
\newcommand{\ra}{\rightarrow}

\newcommand{\ol}{\overline}
\newcommand{\f}{\frac}
\newcommand{\lf}{\lfrac}
\newcommand{\df}{\dfrac}

% bracketting shortcuts
%====================================================================%
\newcommand{\abs}[1]{\left| #1 \right|}
\newcommand{\babs}[1]{\Big|#1\Big|}
\newcommand{\bound}{\Big|}
\newcommand{\BB}[1]{\left(#1\right)}
\newcommand{\dd}{\mathrm{d}}
\newcommand{\artanh}{\mathrm{artanh}}
\newcommand{\Med}{\mathrm{Med}}
\newcommand{\Cov}{\mathrm{Cov}}
\newcommand{\Corr}{\mathrm{Corr}}
\newcommand{\tr}{\mathrm{tr}}
\newcommand{\Range}[1]{\mathrm{range}(#1)}
\newcommand{\Null}[1]{\mathrm{null}(#1)}
\newcommand{\lan}{\langle}
\newcommand{\ran}{\rangle}
\newcommand{\norm}[1]{\left\lVert#1\right\rVert}
\newcommand{\inn}[1]{\lan#1\ran}
\newcommand{\op}[1]{\operatorname{#1}}
\newcommand{\bmat}[1]{\begin{bmatrix}#1\end{bmatrix}}
\newcommand{\pmat}[1]{\begin{pmatrix}#1\end{pmatrix}}
\newcommand{\vmat}[1]{\begin{vmatrix}#1\end{vmatrix}}

\newcommand{\amogus}{{\bigcap}\kern-0.8em\raisebox{0.3ex}{$\subset$}}
\newcommand{\Note}{\textbf{Note: }}
\newcommand{\Aside}{{\bf Aside: }}
%restriction
%\newcommand{\op}[1]{\operatorname{#1}}
%\newcommand{\done}{$$\mathcal{QED}$$}

%====================================================================%


\setlength{\parindent}{0pt}      	% No paragraph indentations
\pagestyle{fancy}
\fancyhf{}							% fancy header

\setcounter{secnumdepth}{0}			% sections are numbered but numbers do not appear
\setcounter{tocdepth}{2} 			% no subsubsections in toc

%template
%====================================================================%
%\begin{manualproblem}{1}
%Spivak.
%\end{manualproblem}

%\begin{proof}[Solution]
%\end{proof}

%----------- or -----------%

%\begin{problem} 		
%\end{problem}	

%\penum
%	\item
%\epenum
%====================================================================%


\newcommand{\Course}{MAT347 }
\newcommand{\hwNumber}{16}

%preamble

\title{a}
\author{A.N.}
\date{\today}

\lhead{\Course A\hwNumber}
\rhead{\thepage}
%\cfoot{\thepage}


%====================================================================%
\begin{document}
\begin{problem}
\end{problem}
To find all possible rational canonical forms of a matrix $A$, we first find all the ways to partition $c_A(x)$ into polynomials $a_1(x), \dots ,a_k(x)$ so that $a_i(x)|a_{i+1}(x)$ and $\prod a_i(x) = c_A(x)$, then find the corresponding companion matrices for each $a_i$. The following polynomials are all of which that satisfy the requirements: 
\begin{align*}
p_1(x) & = x^5+3x^4+3x^2+x^2
\\ p_2(x) & = (x)|(x^4+3x^3+3x^2+x)
\\ p_3(x) & = (x^2+x)|(x^3+2x^2+x)
\\ p_4(x) & = (x+1)|(x^4+2x^3+x^2)
\\ p_5(x) & = (x+1)|(x^2+x)|(x^2+x)
\\ p_6(x) & = (x+1)|(x+1)|(x^3+x^2).
\end{align*} 
The corresponding RCF matrices will be:
\begin{align*}
	M_1 &= \bmat{0&0 & 0&0&0\\ 1 & 0 & 0 & 0 &0\\0&1&0&0&-1\\0&0&1&0&-3\\0&0&0&1&-3 }
	\\ M_2& = \bmat{0&0&0&0&0 \\ 0&0&0&0&0\\ 0&1&0&0&-1\\ 0&0&1&0&-3\\0&0&0&1&-3 }
	\\ M_3 & = \bmat{0&0&0&0&0\\ 1&-1&0&0&0 \\ 0&0&0&0&0 \\ 0&0&1&0&-1 \\ 0&0&0&1&-2}
	\\ M_4 & = \bmat{-1&0&0&0&0 \\ 0&0&0&0&0 \\ 0&1&0&0&0\\ 0&0&1&0&-1 \\ 0&0&0&1&-2}
	\\ M_5 & = \bmat{-1&0&0&0&0 \\ 0&0&0&0&0 \\ 0&1&-1&0&0\\ 0&0&0&0&0 \\ 0&0&0&1&-1}
	\\ M_6 & = \bmat{-1 & 0&0&0&0 \\ 0&-1&0&0&0 \\ 0&0& 0&0&0 \\ 0&0&1&0&0 \\ 0&0&0&1&-1}
\end{align*}
\newpage 
\begin{problem}
\end{problem}
We claim that the minimal polynomial of $B$ is $$m_B(x) = (x-1)^3(x-2)^2(x-3)^3.$$ The exponents are chosen so that they coincide with the size of the largest Jordan block of the corresponding eigenvalue. This will be the minimal exponent required to annihilate the corresponding Jordan blocks, since we can regard $B$ as acting on subspaces independantly. Therefore $m_B(x)$ is the minimal polynomial.
\newpage
\begin{problem}
\end{problem} 
Consider the matrices $$A= \bmat{0&1&0&0 \\ 0&0&0&0 \\ 0&0&0&1 \\ 0&0&0&0}, B = \bmat{0&1&0&0 \\ 0&0&0&0 \\ 0&0&0&0 \\ 0&0&0&0}.$$ Both $A,B$ have characteristic polynomial $x^4$, and minimal polynomial $x^2$. They are clearly not conjugate since they have different ranks. 
\newpage 
\begin{problem}
\end{problem}
Note that $C$ has characteristic polynomial of $c_C(x) = (x-1)(x-2)^2$ simply by computing the determinant of $xI-C$. By row reducing we see that the minimal polynomial must be $(x-1)(x-2)^2 = x^3-5x^2+8x-4$, since we can row reduce into Jordan Canonical form.  Thus the rational canonical form will be $$RCF_C = \bmat{0&0&4\\ 1&0&-8\\ 0&1&5}$$
\end{document}