\documentclass[letterpaper]{article}
\usepackage[letterpaper,margin=1in,footskip=0.25in]{geometry}
\usepackage[utf8]{inputenc}
\usepackage{amsmath}
\usepackage{amsthm}
\usepackage{amssymb, pifont}
\usepackage{mathrsfs}
\usepackage{enumitem}
\usepackage{fancyhdr}
\usepackage{hyperref}

\pagestyle{fancy}
\fancyhf{}
\rhead{MAT 347}
\lhead{Assignment 2}
\rfoot{Page \thepage}

\setlength\parindent{24pt}
\renewcommand\qedsymbol{$\blacksquare$}

\DeclareMathOperator{\Qu}{\mathcal{Q}_8}
\DeclareMathOperator{\F}{\mathbb{F}}
\DeclareMathOperator{\T}{\mathcal{T}}
\DeclareMathOperator{\V}{\mathcal{V}}
\DeclareMathOperator{\U}{\mathcal{U}}
\DeclareMathOperator{\Prt}{\mathbb{P}}
\DeclareMathOperator{\R}{\mathbb{R}}
\DeclareMathOperator{\N}{\mathbb{N}}
\DeclareMathOperator{\Z}{\mathbb{Z}}
\DeclareMathOperator{\Q}{\mathbb{Q}}
\DeclareMathOperator{\C}{\mathbb{C}}
\DeclareMathOperator{\ep}{\varepsilon}
\DeclareMathOperator{\identity}{\mathbf{0}}
\DeclareMathOperator{\card}{card}
\newcommand{\suchthat}{;\ifnum\currentgrouptype=16 \middle\fi|;}

\newtheorem{lemma}{Lemma}

\newcommand{\normal}{\trianglelefteq}
\newcommand{\tr}{\mathrm{tr}}
\newcommand{\ra}{\rightarrow}
\newcommand{\lan}{\langle}
\newcommand{\ran}{\rangle}
\newcommand{\norm}[1]{\left\lVert#1\right\rVert}
\newcommand{\inn}[1]{\lan#1\ran}
\newcommand{\ol}{\overline}
\newcommand{\ci}{i}
\begin{document}
\noindent
Q3: We first claim that every nontrivial subgroup of $\Z$ must be infinite. If $H\leq G$ with $|H|>1$, then take $a\in H, a\neq 0$. We must have for any $n\in \Z$, $an\in H$ since multiplication with integers is equivalent to repeated addition or subtraction. Therefore $H$ must be infinite. We claim that the only subgroups of $\Z$ are $n\Z$ for $n\in \{0,1,2 \dots \}$. Let $H$ be a nontrivial subgroup of $\Z$. We define $Y = \{\gcd(|g|,|h|): g,h\in H\}$. $Y$ is a nonempty set and this is bounded below by $0$, hence we can apply the well ordering principle. There must exist a minimal element $d\in Y$. We now claim that $H = d\Z$. Note that by bezouts identity there exists $a,b\in \Z$ such that $d = ag + bh$ for some $g,h\in \Z$, namely the $g,h$ satisfying $\gcd(g,h) =d$. We now claim that $H = d\Z$. Suppose that there is some $a\in H$ that cannot be written as $dz =a$ for some $z\in H$. This would imply that $\gcd(d,a)<d$ contradicting minimality of $d$. Hence we have that any subgroup of $\Z$ must be of the form $d\Z$.
\end{document}