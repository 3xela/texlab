\documentclass[letterpaper]{article}
\usepackage[letterpaper,margin=1in,footskip=0.25in]{geometry}
\usepackage[utf8]{inputenc}
\usepackage{amsmath}
\usepackage{amsthm}
\usepackage{amssymb, pifont}
\usepackage{mathrsfs}
\usepackage{enumitem}
\usepackage{fancyhdr}
\usepackage{hyperref}

\pagestyle{fancy}
\fancyhf{}
\rhead{MAT 347}
\lhead{Assignment 2}
\rfoot{Page \thepage}

\setlength\parindent{24pt}
\renewcommand\qedsymbol{$\blacksquare$}

\DeclareMathOperator{\Qu}{\mathcal{Q}_8}
\DeclareMathOperator{\F}{\mathbb{F}}
\DeclareMathOperator{\T}{\mathcal{T}}
\DeclareMathOperator{\V}{\mathcal{V}}
\DeclareMathOperator{\U}{\mathcal{U}}
\DeclareMathOperator{\Prt}{\mathbb{P}}
\DeclareMathOperator{\R}{\mathbb{R}}
\DeclareMathOperator{\N}{\mathbb{N}}
\DeclareMathOperator{\Z}{\mathbb{Z}}
\DeclareMathOperator{\Q}{\mathbb{Q}}
\DeclareMathOperator{\C}{\mathbb{C}}
\DeclareMathOperator{\ep}{\varepsilon}
\DeclareMathOperator{\identity}{\mathbf{0}}
\DeclareMathOperator{\card}{card}
\newcommand{\suchthat}{;\ifnum\currentgrouptype=16 \middle\fi|;}

\newtheorem{lemma}{Lemma}

\newcommand{\normal}{\trianglelefteq}
\newcommand{\tr}{\mathrm{tr}}
\newcommand{\ra}{\rightarrow}
\newcommand{\lan}{\langle}
\newcommand{\ran}{\rangle}
\newcommand{\norm}[1]{\left\lVert#1\right\rVert}
\newcommand{\inn}[1]{\lan#1\ran}
\newcommand{\ol}{\overline}
\newcommand{\ci}{i}
\begin{document}
\noindent
Q2: Clearly we have that $C_{15}$ and $\inn{e}$ are both subgroups of $C_{15}$. By Lagrange's theorem, we have that any subgroup of $C_{15}$ must have order of $3$ or $5$. Taking $$H_1 = \inn{3} = \{0,3,6,9,12 \}$$ and $$H_2 = \inn{5} = \{0,5,10\}$$
We claim that these are the only subgroups of order $5$ and $3$ respectively. Note that since $C_{15}$ is cyclic, every subgroup of it can be generated by 1 element. From A1Q4ii, we know that every nonidentity element of $C_{15}$ which is coprime to 15 is a generator of $C_{15}$. Therefore the elements $3,5,6,9,10,12$ do not generate $C_{15}$. We can verify through simple computation that
\begin{align*}
    \inn{3} & = H_1
    \\ \inn{5} &= H_2
    \\ \inn{6} &= \{6,12,3,9,0\} = H_1
    \\ \inn{9} & = \{9,3,12,6,0\} = H_1
    \\ \inn{10} & = \{ 10,5,0\} = H_2
    \\ \inn{12} & = \{12,9,6,3,0\} = H_1
\end{align*}
We see that the only subgroups which are not $\{e\}$ and $C_{15}$ are $H_1,H_2$.
\end{document}