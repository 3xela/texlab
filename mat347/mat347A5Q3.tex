\documentclass[letterpaper]{article}
\usepackage[letterpaper,margin=1in,footskip=0.25in]{geometry}
\usepackage[utf8]{inputenc}
\usepackage{amsmath}
\usepackage{amsthm}
\usepackage{amssymb, pifont}
\usepackage{mathrsfs}
\usepackage{enumitem}
\usepackage{fancyhdr}
\usepackage{hyperref}

\pagestyle{fancy}
\fancyhf{}
\rhead{MAT 347}
\lhead{Assignment 5}
\rfoot{Page \thepage}

\setlength\parindent{24pt}
\renewcommand\qedsymbol{$\blacksquare$}

\DeclareMathOperator{\Qu}{\mathcal{Q}_8}
\DeclareMathOperator{\F}{\mathbb{F}}
\DeclareMathOperator{\T}{\mathcal{T}}
\DeclareMathOperator{\V}{\mathcal{V}}
\DeclareMathOperator{\U}{\mathcal{U}}
\DeclareMathOperator{\Prt}{\mathbb{P}}
\DeclareMathOperator{\R}{\mathbb{R}}
\DeclareMathOperator{\N}{\mathbb{N}}
\DeclareMathOperator{\Z}{\mathbb{Z}}
\DeclareMathOperator{\Q}{\mathbb{Q}}
\DeclareMathOperator{\C}{\mathbb{C}}
\DeclareMathOperator{\ep}{\varepsilon}
\DeclareMathOperator{\identity}{\mathbf{0}}
\DeclareMathOperator{\card}{card}
\newcommand{\suchthat}{;\ifnum\currentgrouptype=16 \middle\fi|;}

\newtheorem{lemma}{Lemma}

\newcommand{\normal}{\trianglelefteq}
\newcommand{\tr}{\mathrm{tr}}
\newcommand{\ra}{\rightarrow}
\newcommand{\lan}{\langle}
\newcommand{\ran}{\rangle}
\newcommand{\norm}[1]{\left\lVert#1\right\rVert}
\newcommand{\inn}[1]{\lan#1\ran}
\newcommand{\ol}{\overline}
\newcommand{\ci}{i}
\begin{document}
\noindent Q3: Let $\sigma$ have a disjoint cycle decomposition of $$\sigma = \sigma_1 \dots \sigma_k. $$ We claim that if each $\sigma_i$ is conjugate to its inverse, then so is $\sigma$. Let $\tau_i$ be the permutation which sends $\sigma_i \to \sigma_i^{-1}$ by conjugating. Note that the $\tau_i$ must be disjoint from eachother, since it can only act on the elements each $\sigma_i$ acts on. We have that 
$$\sigma^{-1} = \sigma_k^{-1} \dots \sigma_{1}^{-1} = (\tau_k \sigma_k \tau_k^{-1}) \dots (\tau_1 \sigma_1 \tau_1^{-1}) = (\tau_k \dots \tau_1) (\sigma_k \dots \sigma_1)(\tau_1^{-1} \dots \tau_{k}^{-1}) = (\tau_k \dots \tau_1) ( \sigma) (\tau_k \dots \tau_1)^{-1} $$
Hence without loss of generality we assume that $\sigma = (a_1 a_2 \dots a_k)$. We know that $\sigma^{-1} = (a_1 a_k \dots a_2)$. Take $\tau = (a_2 a_k)(a_3 a_{k-1}) \dots$. We compute that $$\tau \sigma = (a_1 a_k) (a_2 a_{k-1}) \dots = \sigma^{-1}\tau, $$
As desired.  
\end{document}