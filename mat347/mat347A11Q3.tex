\documentclass[letterpaper]{article}
\usepackage[letterpaper,margin=1in,footskip=0.25in]{geometry}
\usepackage[utf8]{inputenc}
\usepackage{amsmath}
\usepackage{amsthm}
\usepackage{amssymb, pifont}
\usepackage{mathrsfs}
\usepackage{enumitem}
\usepackage{fancyhdr}
\usepackage{hyperref}

\pagestyle{fancy}
\fancyhf{}
\rhead{MAT 347}
\lhead{Assignment 11}
\rfoot{Page \thepage}

\setlength\parindent{24pt}
\renewcommand\qedsymbol{$\blacksquare$}

\DeclareMathOperator{\Qu}{\mathcal{Q}_8}
\DeclareMathOperator{\F}{\mathbb{F}}
\DeclareMathOperator{\T}{\mathcal{T}}
\DeclareMathOperator{\V}{\mathcal{V}}
\DeclareMathOperator{\U}{\mathcal{U}}
\DeclareMathOperator{\Prt}{\mathbb{P}}
\DeclareMathOperator{\R}{\mathbb{R}}
\DeclareMathOperator{\N}{\mathbb{N}}
\DeclareMathOperator{\Z}{\mathbb{Z}}
\DeclareMathOperator{\Q}{\mathbb{Q}}
\DeclareMathOperator{\C}{\mathbb{C}}
\DeclareMathOperator{\ep}{\varepsilon}
\DeclareMathOperator{\identity}{\mathbf{0}}
\DeclareMathOperator{\card}{card}
\newcommand{\suchthat}{;\ifnum\currentgrouptype=16 \middle\fi|;}

\newtheorem{lemma}{Lemma}

\newcommand{\normal}{\triangleleft}
\newcommand{\normaleq}{\trianglelefteq}
\newcommand{\tr}{\mathrm{tr}}
\newcommand{\ra}{\rightarrow}
\newcommand{\lan}{\langle}
\newcommand{\ran}{\rangle}
\newcommand{\norm}[1]{\left\lVert#1\right\rVert}
\newcommand{\inn}[1]{\lan#1\ran}
\newcommand{\ol}{\overline}
\newcommand{\ci}{i}
\newcommand{\ring}{\mathcal{R}}
\begin{document} 
\noindent Q3i:
Write $$2 = (a+b\sqrt{-n})(c+d\sqrt{-n}).$$ Taking the norms of both sides, get that $$4 = N(2) = N(a+b \sqrt{-n}) N(c+d\sqrt{-n}) = (a^2+nb^2)(c^2+nd^2).$$ 
The terms on the right can either be both $2$ or $1$ and $4$. Since $n>3$ they can not both be $2$. Therefore one must have norm $1$ and another must have norm $4$. 
So we have that $2 = \pm 1 \cdot \pm 2$. Hence $2$ is irreducible. Next we will show that $\sqrt{-n}$ is irreducible.  We write $$\sqrt{-n} = (a+b\sqrt{-n})(c + d\sqrt{-n}).$$
Taking norms, we have that $$n = N(\sqrt{-n}) = (a^2+nb^2)(c^2+nd^2).$$ If $\sqrt{-n}$ is not irreducible then both of the numbers on the right are not $1$ or $n$. But this implies that $b,d=0$. Thus $n = a^2c^2$. 
A contradiction. Finally we suppose that $$1+ \sqrt{-n} = (a+b\sqrt{-n})(c+d\sqrt{-n} ). $$ Taking norms, we see that $$1+n^2 = (a^2+nb^2)(c^2+nd^2) = a^2c^2 + n(a^2d^2 + b^2c^2) + b^2d^2n^2.$$ 
This yield $a^2c^2 = 1$, $b^2d^2=1$, and $a^2d^2 + b^2c^2=0$. This is impossible however, since this implies that $a^2d^2= b^2c^2 =0$ and since $\Z$ is an integral domain this implies that either 
$a^2$ or $d^2$ is $0$, and $b^2$ or $c^2$ is $0$, which will break the other equalities. 
\newline \\ Q3ii: Suppose that $\sqrt{-n}$ and $1+\sqrt{-n}$ are both prime. Then they will generate prime ideals, and so by Prop. 13 (p.255 dummite and foote), 
we have that $\Z[\sqrt{-n}]/ (\sqrt{-n})$ and $Z[\sqrt{-n}]/ (1+\sqrt{-n})$ must be integral domains. 
In $\Z[\sqrt{-n}]/ \sqrt{-n}$ we can write every element as $a+b\sqrt{-n}$. If $a\geq n$, then we can write $a = pn + r$ for some $p,r$. Since $pn \in (\sqrt{-n})$,
 we have that $a+b\sqrt{-n}$ is the same as $r$ in the quotient ring. Thus this quotient ring is isomorphic to $Z/ nZ$. Similarly, $Z[\sqrt{-n}]/ (1+\sqrt{-n}) \cong \Z/(n+1)\Z$. These rings can not both be integral domains, since one of $n,n+1$ is even, and hence can be factored into a product of 2 and another integer. These are both zero divisors. 
\end{document}