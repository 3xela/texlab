\documentclass[letterpaper]{article}
\usepackage[letterpaper,margin=1in,footskip=0.25in]{geometry}
\usepackage[utf8]{inputenc}
\usepackage{amsmath}
\usepackage{amsthm}
\usepackage{amssymb, pifont}
\usepackage{mathrsfs}
\usepackage{enumitem}
\usepackage{fancyhdr}
\usepackage{hyperref}

\pagestyle{fancy}
\fancyhf{}
\rhead{MAT 347}
\lhead{Assignment 14}
\rfoot{Page \thepage}

\setlength\parindent{24pt}
\renewcommand\qedsymbol{$\blacksquare$}

\DeclareMathOperator{\Qu}{\mathcal{Q}_8}
\DeclareMathOperator{\F}{\mathbb{F}}
\DeclareMathOperator{\T}{\mathcal{T}}
\DeclareMathOperator{\V}{\mathcal{V}}
\DeclareMathOperator{\U}{\mathcal{U}}
\DeclareMathOperator{\Prt}{\mathbb{P}}
\DeclareMathOperator{\R}{\mathbb{R}}
\DeclareMathOperator{\N}{\mathbb{N}}
\DeclareMathOperator{\Z}{\mathbb{Z}}
\DeclareMathOperator{\Q}{\mathbb{Q}}
\DeclareMathOperator{\C}{\mathbb{C}}
\DeclareMathOperator{\ep}{\varepsilon}
\DeclareMathOperator{\identity}{\mathbf{0}}
\DeclareMathOperator{\card}{card}
\newcommand{\suchthat}{;\ifnum\currentgrouptype=16 \middle\fi|;}

\newtheorem{lemma}{Lemma}

\newcommand{\normal}{\triangleleft}
\newcommand{\normaleq}{\trianglelefteq}
\newcommand{\tr}{\mathrm{tr}}
\newcommand{\ra}{\rightarrow}
\newcommand{\lan}{\langle}
\newcommand{\ran}{\rangle}
\newcommand{\norm}[1]{\left\lVert#1\right\rVert}
\newcommand{\inn}[1]{\lan#1\ran}
\newcommand{\ol}{\overline}
\newcommand{\ci}{i}
\newcommand{\ring}{\mathcal{R}}
\begin{document} \noindent Q4i: 
First consider any permutation $\sigma \in S_3$. Let $w\in W$. We find that $$\sigma \cdot w = \sum_{i=1}^3 w_i e_{\sigma(i)}.$$
Since the coefficients do not change we have that $\sigma \cdot w \in W$. Now for any $\sum_{i}a_i \sigma_i$ we have that $$ \Big(\sum_i a_i \sigma_i\Big) \cdot w = \sum_{i} a_i\sigma_i \cdot w.$$
Since each $\sigma_i \cdot w \in W$, and $W$ is a subspace the entire expression must also belong to $W$. 
\newline \\ Q4ii: Note that the cardinality of $\F_3^3$ is 27 by a combinatorics argument. Similarly, $W$ has a cardinality of $9$. If there was a direct sum decomposition of $\F_3^3$ as $W\oplus V$ for some $V$, then $V$ must contain $0$ but no other vectors in $W$ , so $|V|= 19$. Since $V$ must be a subgroup of $\F_3^3$ we must have that $19|27$ which is absurd. 
\end{document}
