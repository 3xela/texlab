\documentclass[letterpaper]{article}
\usepackage[letterpaper,margin=1in,footskip=0.25in]{geometry}
\usepackage[utf8]{inputenc}
\usepackage{amsmath}
\usepackage{amsthm}
\usepackage{amssymb, pifont}
\usepackage{mathrsfs}
\usepackage{enumitem}
\usepackage{fancyhdr}
\usepackage{hyperref}

\pagestyle{fancy}
\fancyhf{}
\rhead{MAT 347}
\lhead{Assignment 3}
\rfoot{Page \thepage}

\setlength\parindent{24pt}
\renewcommand\qedsymbol{$\blacksquare$}

\DeclareMathOperator{\Qu}{\mathcal{Q}_8}
\DeclareMathOperator{\F}{\mathbb{F}}
\DeclareMathOperator{\T}{\mathcal{T}}
\DeclareMathOperator{\V}{\mathcal{V}}
\DeclareMathOperator{\U}{\mathcal{U}}
\DeclareMathOperator{\Prt}{\mathbb{P}}
\DeclareMathOperator{\R}{\mathbb{R}}
\DeclareMathOperator{\N}{\mathbb{N}}
\DeclareMathOperator{\Z}{\mathbb{Z}}
\DeclareMathOperator{\Q}{\mathbb{Q}}
\DeclareMathOperator{\C}{\mathbb{C}}
\DeclareMathOperator{\ep}{\varepsilon}
\DeclareMathOperator{\identity}{\mathbf{0}}
\DeclareMathOperator{\card}{card}
\newcommand{\suchthat}{;\ifnum\currentgrouptype=16 \middle\fi|;}

\newtheorem{lemma}{Lemma}

\newcommand{\normal}{\trianglelefteq}
\newcommand{\tr}{\mathrm{tr}}
\newcommand{\ra}{\rightarrow}
\newcommand{\lan}{\langle}
\newcommand{\ran}{\rangle}
\newcommand{\norm}[1]{\left\lVert#1\right\rVert}
\newcommand{\inn}[1]{\lan#1\ran}
\newcommand{\ol}{\overline}
\newcommand{\ci}{i}
\begin{document}
\noindent
Q2i: It is sufficient to show that for any $a,b\in L$, we have that $ab^{-1} \in L$. 
We write $a = h_1k_1$ and $b = h_2,k_2$ and compute that 
\begin{align*}
    ab^{-1} &= h_1k_1 k_2^{-1}h_2^{-1}
    \\ &=h_1 ek_1ek_2^{-1}h_2^{-1}
    \\ & = h_1(h_2^{-1}h_2)k_1(h_2^{-1}h_2)k_2^{-1}h_2^{-1}
    \\ & =(h_1h_2^{-1})(h_2k_1h_2^{-1})(h_2k_2^{-1}h_2^{-1}) \tag{by generalized associativity}
\end{align*}
We have that $h_1h_2^{-1}\in H$ and $(h_2k_1h_2^{-1})(h_2k_2^{-1}h_2^{-1}) \in K$ since $h_2 \in Norm_G(K)$. 
We conclude that $L$ is a subgroup
\newline \\ \noindent Q2ii: Let $l = hk\in L$. We evaluate that 
\begin{align*}
    Kl & = Khk
    \\ & = hKk \tag{since h is in $Norm_G(K)$}
    \\ & = hK \tag{since kK = K}
    \\ & = gkK \tag{since K = kK}
    \\ & = lK
\end{align*} Hence $K$ is normal in $L$. 
\end{document}