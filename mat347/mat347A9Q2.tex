\documentclass[letterpaper]{article}
\usepackage[letterpaper,margin=1in,footskip=0.25in]{geometry}
\usepackage[utf8]{inputenc}
\usepackage{amsmath}
\usepackage{amsthm}
\usepackage{amssymb, pifont}
\usepackage{mathrsfs}
\usepackage{enumitem}
\usepackage{fancyhdr}
\usepackage{hyperref}

\pagestyle{fancy}
\fancyhf{}
\rhead{MAT 347}
\lhead{Assignment 9}
\rfoot{Page \thepage}

\setlength\parindent{24pt}
\renewcommand\qedsymbol{$\blacksquare$}

\DeclareMathOperator{\Qu}{\mathcal{Q}_8}
\DeclareMathOperator{\F}{\mathbb{F}}
\DeclareMathOperator{\T}{\mathcal{T}}
\DeclareMathOperator{\V}{\mathcal{V}}
\DeclareMathOperator{\U}{\mathcal{U}}
\DeclareMathOperator{\Prt}{\mathbb{P}}
\DeclareMathOperator{\R}{\mathbb{R}}
\DeclareMathOperator{\N}{\mathbb{N}}
\DeclareMathOperator{\Z}{\mathbb{Z}}
\DeclareMathOperator{\Q}{\mathbb{Q}}
\DeclareMathOperator{\C}{\mathbb{C}}
\DeclareMathOperator{\ep}{\varepsilon}
\DeclareMathOperator{\identity}{\mathbf{0}}
\DeclareMathOperator{\card}{card}
\newcommand{\suchthat}{;\ifnum\currentgrouptype=16 \middle\fi|;}

\newtheorem{lemma}{Lemma}

\newcommand{\normal}{\triangleleft}
\newcommand{\normaleq}{\trianglelefteq}
\newcommand{\tr}{\mathrm{tr}}
\newcommand{\ra}{\rightarrow}
\newcommand{\lan}{\langle}
\newcommand{\ran}{\rangle}
\newcommand{\norm}[1]{\left\lVert#1\right\rVert}
\newcommand{\inn}[1]{\lan#1\ran}
\newcommand{\ol}{\overline}
\newcommand{\ci}{i}
\begin{document} 
\noindent Q2: It has been proven that an ideal $I$ is maximal if and only if $R/ I$ is a field. Thus we will show that $\R[x]/(x^2+1)$ is a field. Note that any polynomial of degree greater than 2 will be equivalent to a linear polynomial, since we can subtract off a sufficiently large multiple of $x^2+1$. Elements of the ring $\R[x]/ (x^2+1)$ will therefore take the form of $a+bx$. We claim that $\R[x]/(x^2+1) \cong \C$. If we take $a+bx,c+dx$ in this ring, we have that their sum will be $$(a+bx)+(c+dx) = (a+c)+x(b+d).$$
We evaluate their product as $$(a+bx)(c+dx) = ac + adx + cbx + dbx^2 = (ac- bd)+x(ad+cb).$$
We have an isomorphism between $\R[x]/(x^2+1)$ to $\C$ given by $\varphi(a+bx) = a+bi$. The structures of addition and multiplication will be preserved by our computations above. 
\end{document}

