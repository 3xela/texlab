\documentclass[letterpaper]{article}
\usepackage[letterpaper,margin=1in,footskip=0.25in]{geometry}
\usepackage[utf8]{inputenc}
\usepackage{amsmath}
\usepackage{amsthm}
\usepackage{amssymb, pifont}
\usepackage{mathrsfs}
\usepackage{enumitem}
\usepackage{fancyhdr}
\usepackage{hyperref}

\pagestyle{fancy}
\fancyhf{}
\rhead{MAT 347}
\lhead{Assignment 2}
\rfoot{Page \thepage}

\setlength\parindent{24pt}
\renewcommand\qedsymbol{$\blacksquare$}

\DeclareMathOperator{\Qu}{\mathcal{Q}_8}
\DeclareMathOperator{\F}{\mathbb{F}}
\DeclareMathOperator{\T}{\mathcal{T}}
\DeclareMathOperator{\V}{\mathcal{V}}
\DeclareMathOperator{\U}{\mathcal{U}}
\DeclareMathOperator{\Prt}{\mathbb{P}}
\DeclareMathOperator{\R}{\mathbb{R}}
\DeclareMathOperator{\N}{\mathbb{N}}
\DeclareMathOperator{\Z}{\mathbb{Z}}
\DeclareMathOperator{\Q}{\mathbb{Q}}
\DeclareMathOperator{\C}{\mathbb{C}}
\DeclareMathOperator{\ep}{\varepsilon}
\DeclareMathOperator{\identity}{\mathbf{0}}
\DeclareMathOperator{\card}{card}
\newcommand{\suchthat}{;\ifnum\currentgrouptype=16 \middle\fi|;}

\newtheorem{lemma}{Lemma}

\newcommand{\normal}{\trianglelefteq}
\newcommand{\tr}{\mathrm{tr}}
\newcommand{\ra}{\rightarrow}
\newcommand{\lan}{\langle}
\newcommand{\ran}{\rangle}
\newcommand{\norm}[1]{\left\lVert#1\right\rVert}
\newcommand{\inn}[1]{\lan#1\ran}
\newcommand{\ol}{\overline}
\newcommand{\ci}{i}
\begin{document}
\noindent
Q4: 
Consider $G=\mathcal{Q}_8$. As per A1Q4, we have classified all the subgroups of $\mathcal{Q}_8$. We know from properties of multiplication in $\mathcal{Q}_8$ that for any $g\in \mathcal{Q}_8$, $H \leq G$ that $gH = Hg$. For example, we can compute that 
\begin{align*}
    \\ k \inn{j} &= \{k ,-i, -k, i \}
    \\ \inn{j}k & = \{k,i,-k, -i \}
\end{align*}
We see that the left and right cosets are equal, although we know that this group is not commutative. We claim that a subgroup is normal if and only if the left and right cosets coinside. We proceed with the forward implication. Suppose $H \normal G$. We see that for all $g\in G$, 
$$gHg^{-1} = H \implies gHg^{-1}g = Hg \implies gH = Hg$$
As desired. Now suppose that $H \leq G$, and $gH = Hg$. Applying $g^{-1}$ to the right side we get that $gHg^{-1} = H$. Hence $H$ is normal. 
\end{document}