\documentclass[letterpaper]{article}
\usepackage[letterpaper,margin=1in,footskip=0.25in]{geometry}
\usepackage[utf8]{inputenc}
\usepackage{amsmath}
\usepackage{amsthm}
\usepackage{amssymb, pifont}
\usepackage{mathrsfs}
\usepackage{enumitem}
\usepackage{fancyhdr}
\usepackage{hyperref}

\pagestyle{fancy}
\fancyhf{}
\rhead{MAT 347}
\lhead{Assignment 8}
\rfoot{Page \thepage}

\setlength\parindent{24pt}
\renewcommand\qedsymbol{$\blacksquare$}

\DeclareMathOperator{\Qu}{\mathcal{Q}_8}
\DeclareMathOperator{\F}{\mathbb{F}}
\DeclareMathOperator{\T}{\mathcal{T}}
\DeclareMathOperator{\V}{\mathcal{V}}
\DeclareMathOperator{\U}{\mathcal{U}}
\DeclareMathOperator{\Prt}{\mathbb{P}}
\DeclareMathOperator{\R}{\mathbb{R}}
\DeclareMathOperator{\N}{\mathbb{N}}
\DeclareMathOperator{\Z}{\mathbb{Z}}
\DeclareMathOperator{\Q}{\mathbb{Q}}
\DeclareMathOperator{\C}{\mathbb{C}}
\DeclareMathOperator{\ep}{\varepsilon}
\DeclareMathOperator{\identity}{\mathbf{0}}
\DeclareMathOperator{\card}{card}
\newcommand{\suchthat}{;\ifnum\currentgrouptype=16 \middle\fi|;}

\newtheorem{lemma}{Lemma}

\newcommand{\normal}{\trianglelefteq}
\newcommand{\tr}{\mathrm{tr}}
\newcommand{\ra}{\rightarrow}
\newcommand{\lan}{\langle}
\newcommand{\ran}{\rangle}
\newcommand{\norm}[1]{\left\lVert#1\right\rVert}
\newcommand{\inn}[1]{\lan#1\ran}
\newcommand{\ol}{\overline}
\newcommand{\ci}{i}
\begin{document}
\noindent Q1i: Consider the group $C_n$ for $n$ even. The subgroup $\inn{\ol{2}}$ is invariant under any automorphism, since previously we have characterized automorphisms of $C_n$ with multiplication by any number coprime to $n$.
\newline \\ Q1ii: Consider the group $\mathcal{Q}_8$ with normal subgroup $\inn{i}$. If we consider the map $\varphi$ defined by $\varphi(i)=j, \varphi(j)=k, \varphi(k)=i$. This will be an automorphism since it simply relabels the generators of the group. However $$\varphi(\inn{i}) = \inn{j}. $$ Thus this subgroup can not be characteristic.
We now give another example of a normal subgroup that is not characteristic. Consider the group $G = (\Q, +)$, with subgroup $H = \{\frac{n}{2}: n\in \Z \}$. Since $G$ is abelian we have that $H \normal G$. Consider the map $\varphi:G \to G$ defined by $\varphi(x) = 2x$. This is an linear operation so it must preserve the additive structure of $\Q$. Furthermore it is invertible since it is just multiplication by a scalar. We see that $\varphi(H) = \Z$. Hence $H$ is not characteristic. 
\end{document}