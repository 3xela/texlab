\documentclass[letterpaper]{article}
\usepackage[letterpaper,margin=1in,footskip=0.25in]{geometry}
\usepackage[utf8]{inputenc}
\usepackage{amsmath}
\usepackage{amsthm}
\usepackage{amssymb, pifont}
\usepackage{mathrsfs}
\usepackage{enumitem}
\usepackage{fancyhdr}
\usepackage{hyperref}

\pagestyle{fancy}
\fancyhf{}
\rhead{MAT 347}
\lhead{Assignment 2}
\rfoot{Page \thepage}

\setlength\parindent{24pt}
\renewcommand\qedsymbol{$\blacksquare$}

\DeclareMathOperator{\Qu}{\mathcal{Q}_8}
\DeclareMathOperator{\F}{\mathbb{F}}
\DeclareMathOperator{\T}{\mathcal{T}}
\DeclareMathOperator{\V}{\mathcal{V}}
\DeclareMathOperator{\U}{\mathcal{U}}
\DeclareMathOperator{\Prt}{\mathbb{P}}
\DeclareMathOperator{\R}{\mathbb{R}}
\DeclareMathOperator{\N}{\mathbb{N}}
\DeclareMathOperator{\Z}{\mathbb{Z}}
\DeclareMathOperator{\Q}{\mathbb{Q}}
\DeclareMathOperator{\C}{\mathbb{C}}
\DeclareMathOperator{\ep}{\varepsilon}
\DeclareMathOperator{\identity}{\mathbf{0}}
\DeclareMathOperator{\card}{card}
\newcommand{\suchthat}{;\ifnum\currentgrouptype=16 \middle\fi|;}

\newtheorem{lemma}{Lemma}

\newcommand{\normal}{\trianglelefteq}
\newcommand{\tr}{\mathrm{tr}}
\newcommand{\ra}{\rightarrow}
\newcommand{\lan}{\langle}
\newcommand{\ran}{\rangle}
\newcommand{\norm}[1]{\left\lVert#1\right\rVert}
\newcommand{\inn}[1]{\lan#1\ran}
\newcommand{\ol}{\overline}
\newcommand{\ci}{i}
\begin{document}
\noindent
Q1a: We claim that if $G = \Z$ and $H = 3\Z$, then $H \leq G$. Suppose that $a,b \in H$. We can write them in the form $a = 3n$ and $b = 3m$. It is sufficient to verify that $ab^{-1}\in H$. A simple computation verifies that indeed, $$ab^{-1} = a + (-b) = 3n + (-3m) = 3(n-m) \in H$$
We have shown the desired result. 
\newline \\ \noindent 
Q1b: Similarly to 1a, we will show that for any $q,r\in H$, $qr^{-1} \in H$. $$qr^{-1} = q \cdot \frac{1}{r} = \frac{q}{r}$$
Since both $r$ is positive and rational, its reciprocal is positive and rational as well, and positive rationals are closed under multiplication. Hence $H \leq G $. 
\newline \\ \noindent Q1c: Suppose that $H$ was a subgroup of $D_{6}$ containing every reflection. There are 3 possible reflections that can be put on a triangle. Additionally, $H$ must contain $e$ to be considered a subgroup. Therefore $|H| =4$. This is not possible however, since $|D_{6}| = 6$ and $4$ does not divide $6$. Hence by Lagranges theorem, this cannot be a subgroup. 
\end{document}