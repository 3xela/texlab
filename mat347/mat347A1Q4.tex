\documentclass[letterpaper]{article}
\usepackage[letterpaper,margin=1in,footskip=0.25in]{geometry}
\usepackage[utf8]{inputenc}
\usepackage{amsmath}
\usepackage{amsthm}
\usepackage{amssymb, pifont}
\usepackage{mathrsfs}
\usepackage{enumitem}
\usepackage{fancyhdr}
\usepackage{hyperref}

\pagestyle{fancy}
\fancyhf{}
\rhead{MAT 347}
\lhead{Assignment 1}
\rfoot{Page \thepage}

\setlength\parindent{24pt}
\renewcommand\qedsymbol{$\blacksquare$}

\DeclareMathOperator{\F}{\mathbb{F}}
\DeclareMathOperator{\T}{\mathcal{T}}
\DeclareMathOperator{\V}{\mathcal{V}}
\DeclareMathOperator{\U}{\mathcal{U}}
\DeclareMathOperator{\Prt}{\mathbb{P}}
\DeclareMathOperator{\R}{\mathbb{R}}
\DeclareMathOperator{\N}{\mathbb{N}}
\DeclareMathOperator{\Z}{\mathbb{Z}}
\DeclareMathOperator{\Q}{\mathbb{Q}}
\DeclareMathOperator{\C}{\mathbb{C}}
\DeclareMathOperator{\ep}{\varepsilon}
\DeclareMathOperator{\identity}{\mathbf{0}}
\DeclareMathOperator{\card}{card}
\newcommand{\suchthat}{;\ifnum\currentgrouptype=16 \middle\fi|;}

\newtheorem{lemma}{Lemma}

\newcommand{\tr}{\mathrm{tr}}
\newcommand{\ra}{\rightarrow}
\newcommand{\lan}{\langle}
\newcommand{\ran}{\rangle}
\newcommand{\norm}[1]{\left\lVert#1\right\rVert}
\newcommand{\inn}[1]{\lan#1\ran}
\newcommand{\ol}{\overline}
\newcommand{\ci}{i}
\begin{document}
\noindent
Q4i:We claim that $[1],[5],[7],[11]$ generate $C_{12}$ and no other elements do. We can verify using the group operation that indeed 
\begin{align*}
    \inn{1} & = \{1,2,3,4,5,6,7,8,9,10,11,0\}
    \\ \inn{5} & = \{5,10,3,8,1,6,11,4,9,2,7,0\}
    \\ \inn{7} & = \{7,2,9,4,11,6,1,8,3,10,5,0\}
    \\ \inn{11} & = \{11,10,9,8,7,6,5,4,3,2,1,0 \}
\end{align*}
We claim that any other element of $C_{12}$ will not generate $C_{12}$. Note that $1,5,7,11$ are the only integers less than 12 which are coprime to 12. Assume that $n$ is not coprime to 12. We will show that $|\inn{n}| < 12$. Since $n<12$ and not coprime, for some integer $k<12$ we have that $nk = 12$. We see that 
$$\inn{n} = \{n ,2n \dots kn \} $$
We see that $|\inn{n}| = k$ which is strictly less than 12. Hence any element of $C_{12}$ which is not coprime to $12$ will not generate $C_{12}$
\newline \\ 
Q4ii: We claim that every $k<n$ that is coprime to $n$ generates $C_{n}$. By bezouts identity we have that there exists integers $a,b$ such that $ak + nb =1$. If we had any $c\in C_{n}$, we can write $c = (ac)k + n(bc)$. Therefore we see that $$[c]_n = [(ac)k + n(bc)]_n =  [(ac)k]_n + [nbc]_n = [(ac)k]$$
Thus $k$ generates $C_n$ 
\end{document}