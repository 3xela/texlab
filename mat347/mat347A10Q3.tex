\documentclass[letterpaper]{article}
\usepackage[letterpaper,margin=1in,footskip=0.25in]{geometry}
\usepackage[utf8]{inputenc}
\usepackage{amsmath}
\usepackage{amsthm}
\usepackage{amssymb, pifont}
\usepackage{mathrsfs}
\usepackage{enumitem}
\usepackage{fancyhdr}
\usepackage{hyperref}

\pagestyle{fancy}
\fancyhf{}
\rhead{MAT 347}
\lhead{Assignment 10}
\rfoot{Page \thepage}

\setlength\parindent{24pt}
\renewcommand\qedsymbol{$\blacksquare$}

\DeclareMathOperator{\Qu}{\mathcal{Q}_8}
\DeclareMathOperator{\F}{\mathbb{F}}
\DeclareMathOperator{\T}{\mathcal{T}}
\DeclareMathOperator{\V}{\mathcal{V}}
\DeclareMathOperator{\U}{\mathcal{U}}
\DeclareMathOperator{\Prt}{\mathbb{P}}
\DeclareMathOperator{\R}{\mathbb{R}}
\DeclareMathOperator{\N}{\mathbb{N}}
\DeclareMathOperator{\Z}{\mathbb{Z}}
\DeclareMathOperator{\Q}{\mathbb{Q}}
\DeclareMathOperator{\C}{\mathbb{C}}
\DeclareMathOperator{\ep}{\varepsilon}
\DeclareMathOperator{\identity}{\mathbf{0}}
\DeclareMathOperator{\card}{card}
\newcommand{\suchthat}{;\ifnum\currentgrouptype=16 \middle\fi|;}

\newtheorem{lemma}{Lemma}

\newcommand{\normal}{\triangleleft}
\newcommand{\normaleq}{\trianglelefteq}
\newcommand{\tr}{\mathrm{tr}}
\newcommand{\ra}{\rightarrow}
\newcommand{\lan}{\langle}
\newcommand{\ran}{\rangle}
\newcommand{\norm}[1]{\left\lVert#1\right\rVert}
\newcommand{\inn}[1]{\lan#1\ran}
\newcommand{\ol}{\overline}
\newcommand{\ci}{i}
\newcommand{\ring}{\mathcal{R}}
\begin{document} 
\noindent Q3: We first show the forward implication. Suppose that $a_nx^n + \dots a_1x + a_0$ is unital. Then for some polynomial $b_mx^m + \dots b_1x + b_0$ we have that $$1= (a_nx^n + \dots a_1x + a_0)(b_mx^m + \dots b_1x + b_0) = (a_nb_m x^{m+n} + \dots a_0 b_0),$$ which implies that $a_0b_0=1$ i.e. $a_0$ is unital. We now claim that $a_nx^n + \dots a_1x$ is nilpotent.
Observe that $$1+ \mathcal{N}(\ring)  = (a_0 +P )+ \mathcal{N}(\ring) \cdot (b_0 + Q) + \mathcal{N}(\ring) =a_0b_0 + \mathcal{N}(\ring) + a_0Q \mathcal{N}(\ring) + b_0P + \mathcal{N}(\ring) + PQ + \mathcal{N}(\ring).$$ 
This implies that $P,Q \in \mathcal{N}(\ring)$ as desired.
We now prove the converse direction. Suppose that $a_0 + P = a_0 + a_1x + \dots  + a_nx^n$ with $a_0$ unital and $P$ nilpotent. Then we have that for any nilpotent polynomial $Q$ for some $m$, $$1= 1- Q^m = (1-Q)(1+Q + \dots Q^{m-1}).$$ Setting $Q = -a_0^{-1}P$ we get that $$1 = (1+ a_0^{-1}P)(1+ \dots (-a_0)^{m-1} P^{m-1}) \implies a_0 = (a_0 + P)(1+ \dots (-a_0)^{m-1} P^{m-1}),$$
which implies that $$1 = (a_0+P)[a_0(1+ \dots (-a_0)^{m-1} P^{m-1})].$$ Hence $a_0+P$ has an inverse. 

\end{document}