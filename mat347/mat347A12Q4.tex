\documentclass[letterpaper]{article}
\usepackage[letterpaper,margin=1in,footskip=0.25in]{geometry}
\usepackage[utf8]{inputenc}
\usepackage{amsmath}
\usepackage{amsthm}
\usepackage{amssymb, pifont}
\usepackage{mathrsfs}
\usepackage{enumitem}
\usepackage{fancyhdr}
\usepackage{hyperref}

\pagestyle{fancy}
\fancyhf{}
\rhead{MAT 347}
\lhead{Assignment 13}
\rfoot{Page \thepage}

\setlength\parindent{24pt}
\renewcommand\qedsymbol{$\blacksquare$}

\DeclareMathOperator{\Qu}{\mathcal{Q}_8}
\DeclareMathOperator{\F}{\mathbb{F}}
\DeclareMathOperator{\T}{\mathcal{T}}
\DeclareMathOperator{\V}{\mathcal{V}}
\DeclareMathOperator{\U}{\mathcal{U}}
\DeclareMathOperator{\Prt}{\mathbb{P}}
\DeclareMathOperator{\R}{\mathbb{R}}
\DeclareMathOperator{\N}{\mathbb{N}}
\DeclareMathOperator{\Z}{\mathbb{Z}}
\DeclareMathOperator{\Q}{\mathbb{Q}}
\DeclareMathOperator{\C}{\mathbb{C}}
\DeclareMathOperator{\ep}{\varepsilon}
\DeclareMathOperator{\identity}{\mathbf{0}}
\DeclareMathOperator{\card}{card}
\newcommand{\suchthat}{;\ifnum\currentgrouptype=16 \middle\fi|;}

\newtheorem{lemma}{Lemma}

\newcommand{\normal}{\triangleleft}
\newcommand{\normaleq}{\trianglelefteq}
\newcommand{\tr}{\mathrm{tr}}
\newcommand{\ra}{\rightarrow}
\newcommand{\lan}{\langle}
\newcommand{\ran}{\rangle}
\newcommand{\norm}[1]{\left\lVert#1\right\rVert}
\newcommand{\inn}[1]{\lan#1\ran}
\newcommand{\ol}{\overline}
\newcommand{\ci}{i}
\newcommand{\ring}{\mathcal{R}}
\begin{document} \noindent Q4: For $P>3$, we employ the binomial formula to compute that
 $$f(x) = \frac{ \sum_{k=1}^p \begin{pmatrix} p \\ k  \end{pmatrix} x^{p-k}(-3)^{k} + 3^p }{x} =  \sum_{k=1}^p \begin{pmatrix} p \\ k  \end{pmatrix} x^{p-k-1}(-3)^k = x^{p-1} + \dots + p(-3)^{p-1}.$$
 We see that $p$ divides every coefficient except on the leading term, and $p^2$ does not divide the constant term. 
 Thus by Eisenstiens criteron this polynomial is irreducible for $p>3$. For $p=3$, we can simply compute $$f(x) = x^2-9x+27.$$
 If this polynomial splits, it must split into two linear factors or equivalently have two roots in $\Z$. The quadratic formula tells us that the roots are $x = \frac{1}{2}(9 \pm i 3\sqrt{3})$, which are not in $\Z$. Thus this polynomial is irreducible. 
 
\end{document}