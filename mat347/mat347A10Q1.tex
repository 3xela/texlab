\documentclass[letterpaper]{article}
\usepackage[letterpaper,margin=1in,footskip=0.25in]{geometry}
\usepackage[utf8]{inputenc}
\usepackage{amsmath}
\usepackage{amsthm}
\usepackage{amssymb, pifont}
\usepackage{mathrsfs}
\usepackage{enumitem}
\usepackage{fancyhdr}
\usepackage{hyperref}

\pagestyle{fancy}
\fancyhf{}
\rhead{MAT 347}
\lhead{Assignment 10}
\rfoot{Page \thepage}

\setlength\parindent{24pt}
\renewcommand\qedsymbol{$\blacksquare$}

\DeclareMathOperator{\Qu}{\mathcal{Q}_8}
\DeclareMathOperator{\F}{\mathbb{F}}
\DeclareMathOperator{\T}{\mathcal{T}}
\DeclareMathOperator{\V}{\mathcal{V}}
\DeclareMathOperator{\U}{\mathcal{U}}
\DeclareMathOperator{\Prt}{\mathbb{P}}
\DeclareMathOperator{\R}{\mathbb{R}}
\DeclareMathOperator{\N}{\mathbb{N}}
\DeclareMathOperator{\Z}{\mathbb{Z}}
\DeclareMathOperator{\Q}{\mathbb{Q}}
\DeclareMathOperator{\C}{\mathbb{C}}
\DeclareMathOperator{\ep}{\varepsilon}
\DeclareMathOperator{\identity}{\mathbf{0}}
\DeclareMathOperator{\card}{card}
\newcommand{\suchthat}{;\ifnum\currentgrouptype=16 \middle\fi|;}

\newtheorem{lemma}{Lemma}

\newcommand{\normal}{\triangleleft}
\newcommand{\normaleq}{\trianglelefteq}
\newcommand{\tr}{\mathrm{tr}}
\newcommand{\ra}{\rightarrow}
\newcommand{\lan}{\langle}
\newcommand{\ran}{\rangle}
\newcommand{\norm}[1]{\left\lVert#1\right\rVert}
\newcommand{\inn}[1]{\lan#1\ran}
\newcommand{\ol}{\overline}
\newcommand{\ci}{i}
\begin{document} 
\noindent Q1i: We claim the element $x^2y$ is in $I\cap J$ but not in $IJ$. Note that it belongs to both the ideal $I$ and $J$, since we can take $y\cdot x^2$, or $x\cdot xy$ respectively. However, every element of $IJ$ can be written as $$\sum_{i}a_i \cdot x^3y$$ for some $a_i\in \mathcal{R}.$ In this polynomial, $x$ must have a degree of at least $3.$ Thus $x^2y \not \in IJ.$
\newline \\ Q1ii: Consider the ideals $I = (2), J=(4).$ It is clear that $I\cap J =J$, since every multiple of $4$ is also a multiple of $2$. However, we have that $IJ$ is all elements of the form $$\sum_{i}a_i (2\cdot 4) = 8\cdot \sum_{i}a_i,$$ for $a_i\in \Z$. Thus we have that $IJ = (8).$  

\end{document}