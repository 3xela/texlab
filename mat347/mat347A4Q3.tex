\documentclass[letterpaper]{article}
\usepackage[letterpaper,margin=1in,footskip=0.25in]{geometry}
\usepackage[utf8]{inputenc}
\usepackage{amsmath}
\usepackage{amsthm}
\usepackage{amssymb, pifont}
\usepackage{mathrsfs}
\usepackage{enumitem}
\usepackage{fancyhdr}
\usepackage{hyperref}

\pagestyle{fancy}
\fancyhf{}
\rhead{MAT 347}
\lhead{Assignment 4}
\rfoot{Page \thepage}

\setlength\parindent{24pt}
\renewcommand\qedsymbol{$\blacksquare$}

\DeclareMathOperator{\Qu}{\mathcal{Q}_8}
\DeclareMathOperator{\F}{\mathbb{F}}
\DeclareMathOperator{\T}{\mathcal{T}}
\DeclareMathOperator{\V}{\mathcal{V}}
\DeclareMathOperator{\U}{\mathcal{U}}
\DeclareMathOperator{\Prt}{\mathbb{P}}
\DeclareMathOperator{\R}{\mathbb{R}}
\DeclareMathOperator{\N}{\mathbb{N}}
\DeclareMathOperator{\Z}{\mathbb{Z}}
\DeclareMathOperator{\Q}{\mathbb{Q}}
\DeclareMathOperator{\C}{\mathbb{C}}
\DeclareMathOperator{\ep}{\varepsilon}
\DeclareMathOperator{\identity}{\mathbf{0}}
\DeclareMathOperator{\card}{card}
\newcommand{\suchthat}{;\ifnum\currentgrouptype=16 \middle\fi|;}

\newtheorem{lemma}{Lemma}

\newcommand{\normal}{\trianglelefteq}
\newcommand{\tr}{\mathrm{tr}}
\newcommand{\ra}{\rightarrow}
\newcommand{\lan}{\langle}
\newcommand{\ran}{\rangle}
\newcommand{\norm}[1]{\left\lVert#1\right\rVert}
\newcommand{\inn}[1]{\lan#1\ran}
\newcommand{\ol}{\overline}
\newcommand{\ci}{i}
\begin{document}
\noindent Q3: We claim that for any isomorphism $\varphi$ between two groups $G,H$, for all $g\in G$, $|\phi(g)| = |g|$. Let $|g|=n$. It is clear that $$\varphi(x)^n=\varphi(x^n) = \varphi(e) =e$$ Suppose now that for some $m<n$, we have that $\varphi(x)^m =e$. Since $\varphi$ is injective this implies that $x^m = e$. Contradicting the order of $x$. Now suppose that $n>m$. Without loss of generality, assume that $m$ is not an integer multiple of $n$. We can write $m = kn+r$ for some nonzero $r$. We get that $$e = \varphi(x)^{m} = \varphi(x^{m}) = \varphi(x^{kn+r}) = \varphi(x^kn)\varphi(x^r) = \varphi(x^r)$$
Once again this contradicts the minimality of $n$. Now if there were an isomorphism between $C_2\times C_2$ and $C_4$, it would have to preserve the order of every element in $C_2\times C_2$. However, $C_2\times C_2 = \{(0,0),(0,1),(1,0),(1,1)\}$. Notice that every element but $e$ has an order of 2. However in $C_4 = \{0,1,2,3\}$ we have two elements, $1,3$ which have order 4. Hence no isomorphism can exist between them. 
\end{document}