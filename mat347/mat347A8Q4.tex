\documentclass[letterpaper]{article}
\usepackage[letterpaper,margin=1in,footskip=0.25in]{geometry}
\usepackage[utf8]{inputenc}
\usepackage{amsmath}
\usepackage{amsthm}
\usepackage{amssymb, pifont}
\usepackage{mathrsfs}
\usepackage{enumitem}
\usepackage{fancyhdr}
\usepackage{hyperref}

\pagestyle{fancy}
\fancyhf{}
\rhead{MAT 347}
\lhead{Assignment 8}
\rfoot{Page \thepage}

\setlength\parindent{24pt}
\renewcommand\qedsymbol{$\blacksquare$}

\DeclareMathOperator{\Qu}{\mathcal{Q}_8}
\DeclareMathOperator{\F}{\mathbb{F}}
\DeclareMathOperator{\T}{\mathcal{T}}
\DeclareMathOperator{\V}{\mathcal{V}}
\DeclareMathOperator{\U}{\mathcal{U}}
\DeclareMathOperator{\Prt}{\mathbb{P}}
\DeclareMathOperator{\R}{\mathbb{R}}
\DeclareMathOperator{\N}{\mathbb{N}}
\DeclareMathOperator{\Z}{\mathbb{Z}}
\DeclareMathOperator{\Q}{\mathbb{Q}}
\DeclareMathOperator{\C}{\mathbb{C}}
\DeclareMathOperator{\ep}{\varepsilon}
\DeclareMathOperator{\identity}{\mathbf{0}}
\DeclareMathOperator{\card}{card}
\newcommand{\suchthat}{;\ifnum\currentgrouptype=16 \middle\fi|;}

\newtheorem{lemma}{Lemma}

\newcommand{\normal}{\triangleleft}
\newcommand{\normaleq}{\trianglelefteq}
\newcommand{\tr}{\mathrm{tr}}
\newcommand{\ra}{\rightarrow}
\newcommand{\lan}{\langle}
\newcommand{\ran}{\rangle}
\newcommand{\norm}[1]{\left\lVert#1\right\rVert}
\newcommand{\inn}[1]{\lan#1\ran}
\newcommand{\ol}{\overline}
\newcommand{\ci}{i}
\begin{document} 
\noindent Q4: Note that there are $(3^2-1)(3^2-3)$ invertible 2 by 2 matrices, since the first column has $3^2-1$ choices (we exclude the 0 column), and the second column can not be a scalar multiple. So there are $3^2-3$ different choices for the second column. We divide this product by 2 since half of the matrices have a determinant of 1. Thus $|SL(2,\F_3)| = 24 = 2^3\cdot 3. $ By Sylows theorem, the number of $3$ subgroups must be either $1,4,7...$. We can check that the following subgroups are of order 3: 
\begin{align*}
    \Big\langle {\begin{bmatrix}0 & 2 \\ 1 & 2 \end{bmatrix}} \Big\rangle & = \Big \{{\begin{bmatrix}0 & 2 \\ 1 & 2 \end{bmatrix}} , {\begin{bmatrix}  2&1  \\ 2 & 0 \end{bmatrix}}, e \Big \}
    \\ \Big\langle {\begin{bmatrix}1 & 1 \\ 0 & 1 \end{bmatrix}} \Big\rangle & = \Big \{{\begin{bmatrix}1 & 1 \\ 0 & 1 \end{bmatrix}} , {\begin{bmatrix}  1&2  \\ 0 & 1 \end{bmatrix}}, e \Big \}
    \\ \Big\langle {\begin{bmatrix}2 & 2 \\ 1 & 0 \end{bmatrix}} \Big\rangle & = \Big \{{\begin{bmatrix}2 & 2 \\ 1 & 0 \end{bmatrix}} , {\begin{bmatrix}  0&1  \\ 2 & 2 \end{bmatrix}}, e \Big \}
    \\ \Big\langle {\begin{bmatrix}1 & 0 \\ 2 & 1 \end{bmatrix}} \Big\rangle & = \Big \{{\begin{bmatrix}1 & 0 \\ 2 & 1 \end{bmatrix}} , {\begin{bmatrix}  1&0  \\ 1 & 1 \end{bmatrix}}, e \Big \}
\end{align*}
We claim that there are no other subgroups of order 3 i.e. there are 4 Sylow $3$ subgroups. If $P$ is a Sylow $3-$group, note that $n_3(G) = \frac{|G|}{|N_G(P|}$. Furthermore, if $a$ normalizes $P$ then so does $2a$ since $$(2a)P (2a)^{-1} = 4aPa^{-1} = aPa^{-1}.$$ We also know that each element of $P$ normalizes $P$ so $|N_G(P)| \geq 5$. This is a subgroup of $G$ so be Lagranges Theorem, we have that $N_G(P)\geq 6$. Thus $n_3(G) \leq 4$. But we have produced 4 $3$ subgroups hence we conclude that the list we have produced is the complete characterization of $3$ subgroups of $G$. 
\end{document}