\documentclass[12pt, a4paper]{article}
\usepackage[lmargin =0.5 in, 
rmargin=0.5in, 
tmargin=1in,
bmargin=0.5in]{geometry}
\geometry{letterpaper}
\usepackage{amsmath}
\usepackage{amssymb}
\usepackage{blindtext}
\usepackage{titlesec}
\usepackage{enumitem}
\usepackage{fancyhdr}
\usepackage{amsthm}
\usepackage{graphicx}
\usepackage{cool}
\usepackage{thmtools}
\usepackage{hyperref}
\graphicspath{ }					%path to an image

%-------- sexy font ------------%
%\usepackage{libertine}
%\usepackage{libertinust1math}

%\usepackage{mlmodern}				% very nice and classic
%\usepackage[utopia]{mathdesign}
%\usepackage[T1]{fontenc}


\usepackage{mlmodern}
\usepackage{eulervm}
%\usepackage{tgtermes} 				%times new roman
%-------- sexy font ------------%


% Problem Styles
%====================================================================%


\newtheorem{problem}{Problem}


\theoremstyle{definition}
\newtheorem{thm}{Theorem}
\newtheorem{lemma}{Lemma}
\newtheorem{prop}{Proposition}
\newtheorem{cor}{Corollary}
\newtheorem{fact}{Fact}
\newtheorem{defn}{Definition}
\newtheorem{example}{Example}
\newtheorem{question}{Question}

\newtheorem{manualprobleminner}{Problem}

\newenvironment{manualproblem}[1]{%
	\renewcommand\themanualprobleminner{#1}%
	\manualprobleminner
}{\endmanualprobleminner}

\newcommand{\penum}{ \begin{enumerate}[label=\bf(\alph*), leftmargin=0pt]}
	\newcommand{\epenum}{ \end{enumerate} }

% Math fonts shortcuts
%====================================================================%

\newcommand{\ring}{\mathcal{R}}
\newcommand{\N}{\mathbb{N}}                           % Natural numbers
\newcommand{\Z}{\mathbb{Z}}                           % Integers
\newcommand{\R}{\mathbb{R}}                           % Real numbers
\newcommand{\C}{\mathbb{C}}                           % Complex numbers
\newcommand{\F}{\mathbb{F}}                           % Arbitrary field
\newcommand{\Q}{\mathbb{Q}}                           % Arbitrary field
\newcommand{\PP}{\mathcal{P}}                         % Partition
\newcommand{\M}{\mathcal{M}}                         % Mathcal M
\newcommand{\eL}{\mathcal{L}}                         % Mathcal L
\newcommand{\T}{\mathcal{T}}                         % Mathcal T
\newcommand{\U}{\mathcal{U}}                         % Mathcal U\\
\newcommand{\V}{\mathcal{V}}                         % Mathcal V

% symbol shortcuts
%====================================================================%

\newcommand{\lam}{\lambda}
\newcommand{\imp}{\implies}
\newcommand{\all}{\forall}
\newcommand{\exs}{\exists}
\newcommand{\delt}{\delta}
\newcommand{\eps}{\varepsilon}
\newcommand{\ra}{\rightarrow}

\newcommand{\ol}{\overline}
\newcommand{\f}{\frac}
\newcommand{\lf}{\lfrac}
\newcommand{\df}{\dfrac}

% bracketting shortcuts
%====================================================================%
\newcommand{\abs}[1]{\left| #1 \right|}
\newcommand{\babs}[1]{\Big|#1\Big|}
\newcommand{\bound}{\Big|}
\newcommand{\BB}[1]{\left(#1\right)}
\newcommand{\dd}{\mathrm{d}}
\newcommand{\artanh}{\mathrm{artanh}}
\newcommand{\Med}{\mathrm{Med}}
\newcommand{\Cov}{\mathrm{Cov}}
\newcommand{\Corr}{\mathrm{Corr}}
\newcommand{\tr}{\mathrm{tr}}
\newcommand{\Range}[1]{\mathrm{range}(#1)}
\newcommand{\Null}[1]{\mathrm{null}(#1)}
\newcommand{\lan}{\langle}
\newcommand{\ran}{\rangle}
\newcommand{\norm}[1]{\left\lVert#1\right\rVert}
\newcommand{\inn}[1]{\lan#1\ran}
\newcommand{\op}[1]{\operatorname{#1}}
\newcommand{\bmat}[1]{\begin{bmatrix}#1\end{bmatrix}}
\newcommand{\pmat}[1]{\begin{pmatrix}#1\end{pmatrix}}
\newcommand{\vmat}[1]{\begin{vmatrix}#1\end{vmatrix}}

\newcommand{\amogus}{{\bigcap}\kern-0.8em\raisebox{0.3ex}{$\subset$}}
\newcommand{\Note}{\textbf{Note: }}
\newcommand{\Aside}{{\bf Aside: }}
%restriction
%\newcommand{\op}[1]{\operatorname{#1}}
%\newcommand{\done}{$$\mathcal{QED}$$}

%====================================================================%


\setlength{\parindent}{0pt}      	% No paragraph indentations
\pagestyle{fancy}
\fancyhf{}							% fancy header

\setcounter{secnumdepth}{0}			% sections are numbered but numbers do not appear
\setcounter{tocdepth}{2} 			% no subsubsections in toc

%template
%====================================================================%
%\begin{manualproblem}{1}
%Spivak.
%\end{manualproblem}

%\begin{proof}[Solution]
%\end{proof}

%----------- or -----------%

%\begin{problem} 		
%\end{problem}	

%\penum
%	\item
%\epenum
%====================================================================%


\newcommand{\Course}{MAT347 }
\newcommand{\hwNumber}{17}

%preamble

\title{a}
\author{A.N.}
\date{\today}

\lhead{\Course A\hwNumber}
\rhead{\thepage}
%\cfoot{\thepage}


%====================================================================%
\begin{document}
	\begin{problem}
	\end{problem}
Since $f$ is a polynomial of degree 2, it is irreducible if and only if it does not have any roots. Any root $x = \frac{a}{b}$ must satisfy $a,b|1$. Therefore if any roots of $f$ exist they must be $x = \pm1$. However $f(1)=3$ and $f(-1) =1$. Thus $f$ is not reducible and $\Q[x]/ (f(x))$ is a field. We claim that $\Q[x]/ (f(x)) \cong \Q(\sqrt{-3})$, and $\ol{x}$ can be identified with $-\frac{1}{2} + \frac{\sqrt{-3}}{2}$. Under this identification, we have for $g\in \Q[x]/ (f(x))$, $$g = a+b\ol{x} = a+ b\left(-\frac{1}{2} + \frac{\sqrt{-3}}{2}\right)  = (a - \frac{b}{2}) + \frac{b}{2}(\sqrt{-3}) \in \Q(\sqrt{-3}). $$
Similarly for $z = a+b\sqrt{-3}$, 
$$z = a+b\sqrt{-3} = a+b(2\ol{x} +1) = (b+a) + 2b(\ol{x}).$$ 

We conclude $\Q[x]/ (f(x)) \cong \Q(\sqrt{-3})$. 
\newpage 
\begin{problem}
\end{problem}
Suppose that $K(\sqrt{a}) = K(\sqrt{b})$. Without loss of generality, assume that $\sqrt{a}, \sqrt{b} \not \in K$. Then certainly there exists some $c,d \in K$ with $\sqrt{a} = c+d\sqrt{b}. $ We claim that $c = 0$. If not, then we have that $$a = c^2+ 2cd\sqrt{b} + d^2b \implies \sqrt{b} \in K.$$ A similar argument for $\sqrt{b} = c^\prime + d^\prime \sqrt{a}$ will yield the same contradiction. Therefore $\sqrt{a} = d\sqrt{b}$ for some $d\in K$. Therefore $$a = y^2b \implies ab = y^2b^2.$$ So $ab$ is a square. Conversely suppose that $ab = c^2$ for some $c$. Then, $c = \sqrt{a}\sqrt{b}$ since $b^2a = cb$. Thus we have that $\sqrt{a} = \frac{c}{b}\sqrt{b}$. Any expression of the form $x+y\sqrt{a} = x+ y\frac{c}{b}\sqrt{b}$ holds. Similarly we have $x+y\sqrt{b} = x+ \frac{b}{c}\sqrt{a}$. 
\newpage
\begin{problem}
\end{problem}
Let $\alpha = \sqrt{2} + \sqrt{3}.$ We have that $$\alpha^2 = 5+ 2\sqrt{6} \implies \alpha^2 - 5 = 2\sqrt{6} \implies (\alpha^2 - 5)^2 - 24 =0.$$ Take $f(x) = (x^2-5)^2 - 24$. By construction $f$ will satisfy $f(\alpha) = 0$. 
\newpage 
\begin{problem}
\end{problem}
We compute the cubes of elements in $\F_7$: 
\begin{align*}
	0^3 & = 0
	\\ 1^3 & = 1
	\\ 2^3 & = 1
	\\ 3^3 & = 6
	\\ 4^3 & = 1
	\\ 5^3 & = 6
	\\ 6^3 & = 6
\end{align*}
The polynomial $x^3+2 $ is degree 3 so it is irreducible if it has a root. No such roots exist in $\F_7$ since a root $\beta$ must satisfy $\beta^3 = 5$, which cannot happen by our above computation. Suppose that $\alpha$ is a root in $\F_7[x]/ (q(x))$. Then, $2\alpha$ and $4\alpha$ will also be solutions to $x^3+2 =0$ since $2^3=4^3=1$. 
\end{document}