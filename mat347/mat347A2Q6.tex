\documentclass[letterpaper]{article}
\usepackage[letterpaper,margin=1in,footskip=0.25in]{geometry}
\usepackage[utf8]{inputenc}
\usepackage{amsmath}
\usepackage{amsthm}
\usepackage{amssymb, pifont}
\usepackage{mathrsfs}
\usepackage{enumitem}
\usepackage{fancyhdr}
\usepackage{hyperref}

\pagestyle{fancy}
\fancyhf{}
\rhead{MAT 347}
\lhead{Assignment 2}
\rfoot{Page \thepage}

\setlength\parindent{24pt}
\renewcommand\qedsymbol{$\blacksquare$}

\DeclareMathOperator{\Qu}{\mathcal{Q}_8}
\DeclareMathOperator{\F}{\mathbb{F}}
\DeclareMathOperator{\T}{\mathcal{T}}
\DeclareMathOperator{\V}{\mathcal{V}}
\DeclareMathOperator{\U}{\mathcal{U}}
\DeclareMathOperator{\Prt}{\mathbb{P}}
\DeclareMathOperator{\R}{\mathbb{R}}
\DeclareMathOperator{\N}{\mathbb{N}}
\DeclareMathOperator{\Z}{\mathbb{Z}}
\DeclareMathOperator{\Q}{\mathbb{Q}}
\DeclareMathOperator{\C}{\mathbb{C}}
\DeclareMathOperator{\ep}{\varepsilon}
\DeclareMathOperator{\identity}{\mathbf{0}}
\DeclareMathOperator{\card}{card}
\newcommand{\suchthat}{;\ifnum\currentgrouptype=16 \middle\fi|;}

\newtheorem{lemma}{Lemma}

\newcommand{\normal}{\trianglelefteq}
\newcommand{\tr}{\mathrm{tr}}
\newcommand{\ra}{\rightarrow}
\newcommand{\lan}{\langle}
\newcommand{\ran}{\rangle}
\newcommand{\norm}[1]{\left\lVert#1\right\rVert}
\newcommand{\inn}[1]{\lan#1\ran}
\newcommand{\ol}{\overline}
\newcommand{\ci}{i}
\begin{document}
\noindent
Q6i: If $G = \Z / 12\Z$, we claim that the only subgroups of order 2 and 3 are $H_1 =\{0,6\} $ and $H_2 =\{0,4,8\}$ respectively. We claim that these are the only subgroups of order 2 and 3 respectively. First note that in any order 2 subgroup of $G$, we must have that $e=0$ belongs to the subgroup, and some other element $a$ satisfying $a = a^{-1}$. Hence we have that $$12= a+a^{-1} = a+a\implies a =6$$
So a subgroup of order $2$ must be $\{0,6\}$. Now consider a subgroup of order 3. It must clearly contain the identity, and two other elements that are inverses of eachother. If we have that $H = \{0,a,b\}$, we know that $a^{-1} = b$, but also that $a+a=b$. This implies that 
$$12 = a+b = a+a+a = 3a\implies a = 4$$ Therefore the only subgroups of $G$ with order $2$ and $3$ are $H_1$ and $H_2$. 
\newline \\  \noindent 
Q6ii: Sicne $2$ does not divide $45$, we have that there does not exist a subgroup of order $2$ in $G = \Z / 45 \Z$. We claim that the only subgroup of $G$ with order 3 is $H = \{0,15,30\}$. The proof that this is the only subgroup of order 3 follows the same procedure as in Q6ii. We conclude that $H$ is the only subgroup of $G$. 
\newline \\ \noindent 
Q6iii: Let $G = D_{12}$, We claim that the only subgroups of order $2$ are $H_1 = \{e, \rho^{3}\}$, $H_2 =\{e,\sigma\}$ and $H_3 = \{e, \sigma \rho^{3}\}$. It is clear that these are the only such subgroups, since they must be generated by an element with order 2. It is clear that these are the only subgroups of order 2 since they are generated by the only elements of $D_{12}$ that have an order of 2. We now claim that the only subgroup of order 3, is $H_4 = \{e,\rho^2, \rho^4\}$. We know that $\sigma$ can not belong to a subgroup with odd order, since $|\sigma| = 2$. Hence it must be be generated by only rotations. Clearly the only rotation satisfying $a^3 = e$ is $a = \rho^2$. 
\newline \\ \noindent
Q6iv: Let $G = D_{10}$. We know by lagranges theorem that there does not exist a subgroup of order 3, since 3 does not divide 10. Thus there may only exist subgroups of order 2. We claim that the only subgroups of order 2 in $D_{10}$ is $H_1 = \{e,\sigma\}$. We require that a subgroup of order 2 must be generated by an element with order 2. Clearly $\sigma$ satisfies this. No other element of $D_{10}$ can satisfy this, since a pentagon has an odd number of sides, hence no rotation or rotation and reflection can be applied successively to obtain the identity transformation. 
\end{document}