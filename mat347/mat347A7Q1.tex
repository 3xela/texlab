\documentclass[letterpaper]{article}
\usepackage[letterpaper,margin=1in,footskip=0.25in]{geometry}
\usepackage[utf8]{inputenc}
\usepackage{amsmath}
\usepackage{amsthm}
\usepackage{amssymb, pifont}
\usepackage{mathrsfs}
\usepackage{enumitem}
\usepackage{fancyhdr}
\usepackage{hyperref}

\pagestyle{fancy}
\fancyhf{}
\rhead{MAT 347}
\lhead{Assignment 7}
\rfoot{Page \thepage}

\setlength\parindent{24pt}
\renewcommand\qedsymbol{$\blacksquare$}

\DeclareMathOperator{\Qu}{\mathcal{Q}_8}
\DeclareMathOperator{\F}{\mathbb{F}}
\DeclareMathOperator{\T}{\mathcal{T}}
\DeclareMathOperator{\V}{\mathcal{V}}
\DeclareMathOperator{\U}{\mathcal{U}}
\DeclareMathOperator{\Prt}{\mathbb{P}}
\DeclareMathOperator{\R}{\mathbb{R}}
\DeclareMathOperator{\N}{\mathbb{N}}
\DeclareMathOperator{\Z}{\mathbb{Z}}
\DeclareMathOperator{\Q}{\mathbb{Q}}
\DeclareMathOperator{\C}{\mathbb{C}}
\DeclareMathOperator{\ep}{\varepsilon}
\DeclareMathOperator{\identity}{\mathbf{0}}
\DeclareMathOperator{\card}{card}
\newcommand{\suchthat}{;\ifnum\currentgrouptype=16 \middle\fi|;}

\newtheorem{lemma}{Lemma}

\newcommand{\normal}{\trianglelefteq}
\newcommand{\tr}{\mathrm{tr}}
\newcommand{\ra}{\rightarrow}
\newcommand{\lan}{\langle}
\newcommand{\ran}{\rangle}
\newcommand{\norm}[1]{\left\lVert#1\right\rVert}
\newcommand{\inn}[1]{\lan#1\ran}
\newcommand{\ol}{\overline}
\newcommand{\ci}{i}
\begin{document}
\noindent Q1: First note that by the class equation, we have that $$1 \neq |Z(G)|.$$ Since $|Z(G)| \mid p^2$, suppose that $|Z(G)| = p$. We therefore have that $|G / Z(G)| =p$. 
Hence, $G/Z(G) \cong C_p$. Taking some $x \notin Z(G)$, we can present $G/ Z(G) = \{e, \ol{x}, \dots \ol{x}^{p-1}\}.$ Note that $|\ol{x}| = p$ therefore $x^p \in Z(G).$ We must also have that $|x| = p$ or $p^2$. If $|x| = p^2$, we are done since $G \cong C_{p^2}.$ If we have that $|x| =p$, we write $$G = \bigcup_{k=1}^{p} x^k Z(G).$$
Since $|Z(G)| = p$, we have that it must be cyclic. We can therefore write $Z(G) = \{e,z,z^2 , \dots , z^{p-1}\}$ for some $z\in Z(G).$ Furtherfore, we have that $G = \{x^i z^j: 0\leq i,j\leq p\}.$ If we take any $h,g\in G$, we can express them $g = x^a z^b, h = x^c z^d$. We compute that $$gh = x^a z^b x^c z^d = x^{a+c}z^{c+d} = x^c z^d x^a z^b = hg$$
Hence we get that $G$ is abelian. 
\end{document}