\documentclass[letterpaper]{article}
\usepackage[letterpaper,margin=1in,footskip=0.25in]{geometry}
\usepackage[utf8]{inputenc}
\usepackage{amsmath}
\usepackage{amsthm}
\usepackage{amssymb, pifont}
\usepackage{mathrsfs}
\usepackage{enumitem}
\usepackage{fancyhdr}
\usepackage{hyperref}

\pagestyle{fancy}
\fancyhf{}
\rhead{MAT 347}
\lhead{Assignment 9}
\rfoot{Page \thepage}

\setlength\parindent{24pt}
\renewcommand\qedsymbol{$\blacksquare$}

\DeclareMathOperator{\Qu}{\mathcal{Q}_8}
\DeclareMathOperator{\F}{\mathbb{F}}
\DeclareMathOperator{\T}{\mathcal{T}}
\DeclareMathOperator{\V}{\mathcal{V}}
\DeclareMathOperator{\U}{\mathcal{U}}
\DeclareMathOperator{\Prt}{\mathbb{P}}
\DeclareMathOperator{\R}{\mathbb{R}}
\DeclareMathOperator{\N}{\mathbb{N}}
\DeclareMathOperator{\Z}{\mathbb{Z}}
\DeclareMathOperator{\Q}{\mathbb{Q}}
\DeclareMathOperator{\C}{\mathbb{C}}
\DeclareMathOperator{\ep}{\varepsilon}
\DeclareMathOperator{\identity}{\mathbf{0}}
\DeclareMathOperator{\card}{card}
\newcommand{\suchthat}{;\ifnum\currentgrouptype=16 \middle\fi|;}

\newtheorem{lemma}{Lemma}

\newcommand{\normal}{\triangleleft}
\newcommand{\normaleq}{\trianglelefteq}
\newcommand{\tr}{\mathrm{tr}}
\newcommand{\ra}{\rightarrow}
\newcommand{\lan}{\langle}
\newcommand{\ran}{\rangle}
\newcommand{\norm}[1]{\left\lVert#1\right\rVert}
\newcommand{\inn}[1]{\lan#1\ran}
\newcommand{\ol}{\overline}
\newcommand{\ci}{i}
\begin{document} 
\noindent Q4i: We know from matrix multiplication that the left ideal of $E_{ij}$ will be all matrices which are zero everywhere except possibly the $j'th$ column.
\newline \\ Q4ii: Similarly to $4i$, we have that the right ideal will be given by matrices with zero entries everywhere except possibly the $i'th$ row. 
\newline \\ Q4iii: From properties of matrix multiplication, we know that the two sided ideal generated by $E_{ij}$ will be the set of all matrices with nonzero entries everywhere except possibly the $ij$'th component of the matrix.
\newline \\ Q4iv: Given a matrix $M\in \mathcal{R}$, we consider the two sided ideal generated by it, $RMR$. Any matrix in $RMR$ can be written as the sum of matrices in the form $rMs$ for $r,s\in \mathcal{R}.$ We can choose $r,s$ such that we get a matrix $K$ which is the same rank,$k$ as $M$, with $k$ $1's$ on the diagonal. From here we can generate every matrix with rank less than or equal to $M$. Thus we are done.   

\end{document}