\documentclass[letterpaper]{article}
\usepackage[letterpaper,margin=1in,footskip=0.25in]{geometry}
\usepackage[utf8]{inputenc}
\usepackage{amsmath}
\usepackage{amsthm}
\usepackage{amssymb, pifont}
\usepackage{mathrsfs}
\usepackage{enumitem}
\usepackage{fancyhdr}
\usepackage{hyperref}

\pagestyle{fancy}
\fancyhf{}
\rhead{MAT 347}
\lhead{Assignment 3}
\rfoot{Page \thepage}

\setlength\parindent{24pt}
\renewcommand\qedsymbol{$\blacksquare$}

\DeclareMathOperator{\Qu}{\mathcal{Q}_8}
\DeclareMathOperator{\F}{\mathbb{F}}
\DeclareMathOperator{\T}{\mathcal{T}}
\DeclareMathOperator{\V}{\mathcal{V}}
\DeclareMathOperator{\U}{\mathcal{U}}
\DeclareMathOperator{\Prt}{\mathbb{P}}
\DeclareMathOperator{\R}{\mathbb{R}}
\DeclareMathOperator{\N}{\mathbb{N}}
\DeclareMathOperator{\Z}{\mathbb{Z}}
\DeclareMathOperator{\Q}{\mathbb{Q}}
\DeclareMathOperator{\C}{\mathbb{C}}
\DeclareMathOperator{\ep}{\varepsilon}
\DeclareMathOperator{\identity}{\mathbf{0}}
\DeclareMathOperator{\card}{card}
\newcommand{\suchthat}{;\ifnum\currentgrouptype=16 \middle\fi|;}

\newtheorem{lemma}{Lemma}

\newcommand{\normal}{\trianglelefteq}
\newcommand{\tr}{\mathrm{tr}}
\newcommand{\ra}{\rightarrow}
\newcommand{\lan}{\langle}
\newcommand{\ran}{\rangle}
\newcommand{\norm}[1]{\left\lVert#1\right\rVert}
\newcommand{\inn}[1]{\lan#1\ran}
\newcommand{\ol}{\overline}
\newcommand{\ci}{i}
\begin{document}
\noindent
Q6i: We claim that $G^\prime \trianglelefteq G$. Any element of $G^\prime$ must be of the form $b = ghg^{-1}h^{-1}$. We claim that for any $a\in G$, $aba^{-1}\in G^\prime$. Observe: 
$$aba^{-1} = a gh g^{-1}h^{-1}a = ageheg^{-1}eh^{-1}a^{-1} =(aga^{-1})(aha^{-1})(ag^{-1}a^{-1})(ah^{-1}a^{-1}) $$ We notice that this in the desired form, since the first and third terms are inverses of eachother, and the second and fourth terms are inverses of eachother. 
We conclude that $G^\prime \trianglelefteq G$
\newline \\ \noindent 
Q6ii: We claim that $G / G^\prime$ is an abelian group. It is equivalent to show that for any $a,b\in G $, $$abG^\prime = ba G^\prime \iff ab(ba)^{-1}G^\prime = G^\prime \iff aba^{-1}b^{-1}G^\prime = G^\prime$$
The last equality is clearly true since $aba^{-1}b^{-1}\in G$. 
\newline \\ \noindent 
Q6iii: Let $N\trianglelefteq G$ be such that $G / N$ is abelian. We claim that $G^\prime \subseteq N$. Let $a\in G^\prime$. It must take the form $a = ghg^{-1}h^{-1}$ for some $g,h\in G$. We evaluate that 
\begin{align*} aN &= (ghg^{-1}h^{-1})N 
    \\ & = (gN) (hN) (g^{-1}N) (h^{-1}N) 
    \\ & =(gN)(g^{-1}N)(hN)(h^{-1}N) 
    \\ & = (gg^{-1})N(hh^{-1}N) 
    \\ & = (eN)(eN)
    \\ & = eeN
    \\ &= eN 
    \\ &= N 
\end{align*}
Therefore we have that $a\in N$ and we conclude that $G^\prime \subset N$
\end{document}