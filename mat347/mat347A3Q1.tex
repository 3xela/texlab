\documentclass[letterpaper]{article}
\usepackage[letterpaper,margin=1in,footskip=0.25in]{geometry}
\usepackage[utf8]{inputenc}
\usepackage{amsmath}
\usepackage{amsthm}
\usepackage{amssymb, pifont}
\usepackage{mathrsfs}
\usepackage{enumitem}
\usepackage{fancyhdr}
\usepackage{hyperref}

\pagestyle{fancy}
\fancyhf{}
\rhead{MAT 347}
\lhead{Assignment 3}
\rfoot{Page \thepage}

\setlength\parindent{24pt}
\renewcommand\qedsymbol{$\blacksquare$}

\DeclareMathOperator{\Qu}{\mathcal{Q}_8}
\DeclareMathOperator{\F}{\mathbb{F}}
\DeclareMathOperator{\T}{\mathcal{T}}
\DeclareMathOperator{\V}{\mathcal{V}}
\DeclareMathOperator{\U}{\mathcal{U}}
\DeclareMathOperator{\Prt}{\mathbb{P}}
\DeclareMathOperator{\R}{\mathbb{R}}
\DeclareMathOperator{\N}{\mathbb{N}}
\DeclareMathOperator{\Z}{\mathbb{Z}}
\DeclareMathOperator{\Q}{\mathbb{Q}}
\DeclareMathOperator{\C}{\mathbb{C}}
\DeclareMathOperator{\ep}{\varepsilon}
\DeclareMathOperator{\identity}{\mathbf{0}}
\DeclareMathOperator{\card}{card}
\newcommand{\suchthat}{;\ifnum\currentgrouptype=16 \middle\fi|;}

\newtheorem{lemma}{Lemma}

\newcommand{\normal}{\trianglelefteq}
\newcommand{\tr}{\mathrm{tr}}
\newcommand{\ra}{\rightarrow}
\newcommand{\lan}{\langle}
\newcommand{\ran}{\rangle}
\newcommand{\norm}[1]{\left\lVert#1\right\rVert}
\newcommand{\inn}[1]{\lan#1\ran}
\newcommand{\ol}{\overline}
\newcommand{\ci}{i}
\begin{document}
\noindent
Q1i: 
It is sufficient to show that for any $\alpha\in G$, $\alpha H = H \alpha$. First suppose that $\alpha \in G - H$. Since the index of $H$ is 2, we know that we can obtain a partition of $G$ as $G = H \sqcup \alpha H$. Similarly, we can obtain the partition of $G = H \sqcup H \alpha$. This implies that for $\alpha \in G - H$, $\alpha H = H \alpha$. Now suppose that $\alpha \in H$. Since $H$ is a group which inherits its multiplication from $G$, we have that $\alpha H = H \alpha$. We have that $\alpha H = H \alpha$ for all $\alpha\in G$, hence we apply our result from assignment 2, question 4 and conclude that $H \trianglelefteq G$.
\newline \\ \noindent 
Q1ii:  We write $n = ak$ for some positive integer $a$. Take $G = \Z / n\Z$, and $H = \inn{[a]} = \{[a], [2a], \dots [ka] \}$. $H$ is clearly a subgroup of $G$ with order $k$, and since $G$ is abelian, it follows that every subgroup is normal in $G$. 
\end{document}