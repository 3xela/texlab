\documentclass[letterpaper]{article}
\usepackage[letterpaper,margin=1in,footskip=0.25in]{geometry}
\usepackage[utf8]{inputenc}
\usepackage{amsmath}
\usepackage{amsthm}
\usepackage{amssymb, pifont}
\usepackage{mathrsfs}
\usepackage{enumitem}
\usepackage{fancyhdr}
\usepackage{hyperref}

\pagestyle{fancy}
\fancyhf{}
\rhead{MAT 347}
\lhead{Assignment 1}
\rfoot{Page \thepage}

\setlength\parindent{24pt}
\renewcommand\qedsymbol{$\blacksquare$}

\DeclareMathOperator{\F}{\mathbb{F}}
\DeclareMathOperator{\T}{\mathcal{T}}
\DeclareMathOperator{\V}{\mathcal{V}}
\DeclareMathOperator{\U}{\mathcal{U}}
\DeclareMathOperator{\Prt}{\mathbb{P}}
\DeclareMathOperator{\R}{\mathbb{R}}
\DeclareMathOperator{\N}{\mathbb{N}}
\DeclareMathOperator{\Z}{\mathbb{Z}}
\DeclareMathOperator{\Q}{\mathbb{Q}}
\DeclareMathOperator{\C}{\mathbb{C}}
\DeclareMathOperator{\ep}{\varepsilon}
\DeclareMathOperator{\identity}{\mathbf{0}}
\DeclareMathOperator{\card}{card}
\newcommand{\suchthat}{;\ifnum\currentgrouptype=16 \middle\fi|;}

\newtheorem{lemma}{Lemma}

\newcommand{\tr}{\mathrm{tr}}
\newcommand{\ra}{\rightarrow}
\newcommand{\lan}{\langle}
\newcommand{\ran}{\rangle}
\newcommand{\norm}[1]{\left\lVert#1\right\rVert}
\newcommand{\inn}[1]{\lan#1\ran}
\newcommand{\ol}{\overline}
\newcommand{\ci}{i}
\begin{document}
\noindent
Q2i: Let $g\in G$. Suppose that $a,b$ are both multiplicative inverses of $g$. Using the properties of group multiplication, we get that 
$$bg = e \implies (bg)a = ea \implies b(ga) = a \implies b=a$$
Hence the multiplicative identity is unique. 
\newline \\ 
Q2ii: Let $g,h\in G$. We let $(gh)^{-1}=a$. We see that 
\begin{align*}
    (gh)a &= e
    \\ g^{-1}(gh)a &= g^{-1}e \tag{multiply left sides by $g^{-1}$}
    \\ (g^{-1}g)(ha) &= g^{-1} \tag{by associativity and identity}
    \\ ha &= g^{-1} \tag{by inverse}
    \\ (h^{-1} h) a &=h^{-1} g^{-1} \tag{multiply left sides by $h^{-1}$}
    \\ a &=h^{-1}g^{-1} \tag{by inverse}
\end{align*}
Thus we see that $a=(gh)^{-1} = h^{-1}g^{-1}$ as desired. 
\newline \\ 
Q2iii: First, note that by multiplying $gh =e$ by $h^{-1}$ to the right, we get that $g = h^{-1}$. Similarly, when we multiply $gh = e$ by $g^{-1}$ to the left we get that $h = g^{-1}$. We can now verify that indeed
\begin{align*}
    gh & = e
    \\ (hg)h &= he \tag{multiply both sides by $h$}
    \\ (hg)(hg)&= (he)g \tag{numtiply both sides by g, associativity}
    \\ (g^{-1}g)(h^{-1}h) & =hg 
    \\ ee& =hg \tag{by inverse}
    \\ e&=hg \tag{by identity}
\end{align*} Thus we have that $gh = e = hg$ as desired. 
\end{document}