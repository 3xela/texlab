\documentclass[letterpaper]{article}
\usepackage[letterpaper,margin=1in,footskip=0.25in]{geometry}
\usepackage[utf8]{inputenc}
\usepackage{amsmath}
\usepackage{amsthm}
\usepackage{amssymb, pifont}
\usepackage{mathrsfs}
\usepackage{enumitem}
\usepackage{fancyhdr}
\usepackage{hyperref}

\pagestyle{fancy}
\fancyhf{}
\rhead{MAT 347}
\lhead{Assignment 5}
\rfoot{Page \thepage}

\setlength\parindent{24pt}
\renewcommand\qedsymbol{$\blacksquare$}

\DeclareMathOperator{\Qu}{\mathcal{Q}_8}
\DeclareMathOperator{\F}{\mathbb{F}}
\DeclareMathOperator{\T}{\mathcal{T}}
\DeclareMathOperator{\V}{\mathcal{V}}
\DeclareMathOperator{\U}{\mathcal{U}}
\DeclareMathOperator{\Prt}{\mathbb{P}}
\DeclareMathOperator{\R}{\mathbb{R}}
\DeclareMathOperator{\N}{\mathbb{N}}
\DeclareMathOperator{\Z}{\mathbb{Z}}
\DeclareMathOperator{\Q}{\mathbb{Q}}
\DeclareMathOperator{\C}{\mathbb{C}}
\DeclareMathOperator{\ep}{\varepsilon}
\DeclareMathOperator{\identity}{\mathbf{0}}
\DeclareMathOperator{\card}{card}
\newcommand{\suchthat}{;\ifnum\currentgrouptype=16 \middle\fi|;}

\newtheorem{lemma}{Lemma}

\newcommand{\normal}{\trianglelefteq}
\newcommand{\tr}{\mathrm{tr}}
\newcommand{\ra}{\rightarrow}
\newcommand{\lan}{\langle}
\newcommand{\ran}{\rangle}
\newcommand{\norm}[1]{\left\lVert#1\right\rVert}
\newcommand{\inn}[1]{\lan#1\ran}
\newcommand{\ol}{\overline}
\newcommand{\ci}{i}
\begin{document}
\noindent Q4: 
This result is false. Consider $D_8 = G$, with $H = Z(G) = \inn{\rho^2}$ and $H^\prime = \{e,\sigma\}$. We have that $\varphi: H \to H^\prime$ is an isomorphism when we define $\varphi(\rho^2) = \sigma$. Furthermore, we have that $H = Z(G) \triangleleft G$. However, we have that $$G / H  = \{\ol{e}, \ol{\rho}, \ol{\sigma}, \ol{\sigma \rho} \}, $$ and $$G / H^\prime = \{\ol{e}, \ol{\rho}, \ol{\rho^2}, \ol{\rho^3}  \}. $$
These groups are certainly not isomorphic, since every element in $G/ H$ has an order of $2$ or $1$, while $\ol{\rho^3} \in G / H^\prime$ has an order of $3$. By the proof of A4Q3 we know that isomorphisms must preserve the order of elements. Hence no isomorphism between these groups can exist. 
\end{document}