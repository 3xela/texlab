\documentclass[letterpaper]{article}
\usepackage[letterpaper,margin=1in,footskip=0.25in]{geometry}
\usepackage[utf8]{inputenc}
\usepackage{amsmath}
\usepackage{amsthm}
\usepackage{amssymb, pifont}
\usepackage{mathrsfs}
\usepackage{enumitem}
\usepackage{fancyhdr}
\usepackage{hyperref}

\pagestyle{fancy}
\fancyhf{}
\rhead{MAT 347}
\lhead{Assignment 7}
\rfoot{Page \thepage}

\setlength\parindent{24pt}
\renewcommand\qedsymbol{$\blacksquare$}

\DeclareMathOperator{\Qu}{\mathcal{Q}_8}
\DeclareMathOperator{\F}{\mathbb{F}}
\DeclareMathOperator{\T}{\mathcal{T}}
\DeclareMathOperator{\V}{\mathcal{V}}
\DeclareMathOperator{\U}{\mathcal{U}}
\DeclareMathOperator{\Prt}{\mathbb{P}}
\DeclareMathOperator{\R}{\mathbb{R}}
\DeclareMathOperator{\N}{\mathbb{N}}
\DeclareMathOperator{\Z}{\mathbb{Z}}
\DeclareMathOperator{\Q}{\mathbb{Q}}
\DeclareMathOperator{\C}{\mathbb{C}}
\DeclareMathOperator{\ep}{\varepsilon}
\DeclareMathOperator{\identity}{\mathbf{0}}
\DeclareMathOperator{\card}{card}
\newcommand{\suchthat}{;\ifnum\currentgrouptype=16 \middle\fi|;}

\newtheorem{lemma}{Lemma}

\newcommand{\normal}{\trianglelefteq}
\newcommand{\tr}{\mathrm{tr}}
\newcommand{\ra}{\rightarrow}
\newcommand{\lan}{\langle}
\newcommand{\ran}{\rangle}
\newcommand{\norm}[1]{\left\lVert#1\right\rVert}
\newcommand{\inn}[1]{\lan#1\ran}
\newcommand{\ol}{\overline}
\newcommand{\ci}{i}
\begin{document}
\noindent Q2: Let $|G|= p^\alpha$. We first claim that if for some $i$, $Z_i(G) = Z_{i+1}(G)$ then $Z_i(G)=Z_{i+1}(G) = G$. By assumption, $|Z_i(G) / Z_{i+1}(G)| = 1$. 
Therefore we have that $|Z(G / Z_i(G))| = 1$. This implies that $Z_i(G) = G$ by Lagranges Theorem. Hence $Z_i(G) = Z_{i+1}(G) = G$. We now claim that the sequence $\{ |Z_i(G)| \}$ is strictly increasing up until some $k$ from which point on we have equality. We will proceed by induction. First consider $|Z_2(G) / Z_1(G)|.$ By definition the equality $$|Z(G / Z_1(G))| = |Z_2(G) / Z_1(G)|$$ must hold. If we let $|Z_1(G)| = p^\beta$ for $\beta < \alpha$, we have that for some $\gamma \leq  \alpha - \beta$, $$p^\gamma = |Z_2(G) / Z_1(G)| = |Z_2(G)| \cdot p^{-\alpha}$$ Which is equivalent to $|Z_2(G)| = p^{\gamma + \alpha}$. Since $|Z_2(G)|$ must divide $p^\alpha$ it must also be a power of $p$. Now if $\gamma = \alpha - \beta$ we get that $|Z_2(G)| = p^\alpha$ and we conclude that $Z_2(G) = G$. If not, then we have that $|Z_2(G)| > |Z(G)|$. Now suppose that this holds for up until some $i$. We will show that $|Z_i(G)| \leq |Z_{i+1}(G)|$, with equality signaling that $Z_{i+1}(G)=G$. We let $|Z_{i}(G)| = p^{\beta_i},$ and note that $p^{\gamma_i} = |Z(G / Z_{i}(G) )| \geq p^{\alpha - \beta_i} $ But also that $$ |Z_{i+1}(G)| = p^{\gamma_i + \beta_i}$$. When $\gamma = \alpha - \beta_i$ we have equality and conclude $|Z_{i+1}(G)| = p^\alpha$. If the inequality is strict, then we get that $p^{\gamma_i + \beta_i } =|Z_{i+1}(G)| > p^{\beta_i} = |Z_{i}(G)|$.
Hence the sequence $\{|Z_i(G)| \}$ is strictly increasing, and since it is bounded above by $p^\alpha$ is must attain $p^\alpha$ eventually, from then on it will be constant. 

\end{document}