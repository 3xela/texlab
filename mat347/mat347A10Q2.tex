\documentclass[letterpaper]{article}
\usepackage[letterpaper,margin=1in,footskip=0.25in]{geometry}
\usepackage[utf8]{inputenc}
\usepackage{amsmath}
\usepackage{amsthm}
\usepackage{amssymb, pifont}
\usepackage{mathrsfs}
\usepackage{enumitem}
\usepackage{fancyhdr}
\usepackage{hyperref}

\pagestyle{fancy}
\fancyhf{}
\rhead{MAT 347}
\lhead{Assignment 10}
\rfoot{Page \thepage}

\setlength\parindent{24pt}
\renewcommand\qedsymbol{$\blacksquare$}

\DeclareMathOperator{\Qu}{\mathcal{Q}_8}
\DeclareMathOperator{\F}{\mathbb{F}}
\DeclareMathOperator{\T}{\mathcal{T}}
\DeclareMathOperator{\V}{\mathcal{V}}
\DeclareMathOperator{\U}{\mathcal{U}}
\DeclareMathOperator{\Prt}{\mathbb{P}}
\DeclareMathOperator{\R}{\mathbb{R}}
\DeclareMathOperator{\N}{\mathbb{N}}
\DeclareMathOperator{\Z}{\mathbb{Z}}
\DeclareMathOperator{\Q}{\mathbb{Q}}
\DeclareMathOperator{\C}{\mathbb{C}}
\DeclareMathOperator{\ep}{\varepsilon}
\DeclareMathOperator{\identity}{\mathbf{0}}
\DeclareMathOperator{\card}{card}
\newcommand{\suchthat}{;\ifnum\currentgrouptype=16 \middle\fi|;}

\newtheorem{lemma}{Lemma}

\newcommand{\normal}{\triangleleft}
\newcommand{\normaleq}{\trianglelefteq}
\newcommand{\tr}{\mathrm{tr}}
\newcommand{\ra}{\rightarrow}
\newcommand{\lan}{\langle}
\newcommand{\ran}{\rangle}
\newcommand{\norm}[1]{\left\lVert#1\right\rVert}
\newcommand{\inn}[1]{\lan#1\ran}
\newcommand{\ol}{\overline}
\newcommand{\ci}{i}
\newcommand{\ring}{\mathcal{R}}
\begin{document} 
\noindent
Q2: We first claim that $\mathcal{N}(\mathcal{R})$ is an ideal. It is clear that if $r^n=0$, then for any $a\in \ring$ we have $$(ar)^n = a^n r^n = 0.$$
Now suppose that $r,s \in \mathcal{N}(\ring)$ and $r^n=s^m=0.$ Then by the binomial expansion we compute that $$(r+s)^{m+n} = \sum_{i=1}^{m+n} \begin{pmatrix}
    m+n \\ i
\end{pmatrix} r^{m+n-i}s^{i} = 0,$$ since until $i=m$, $r^{m+n-i}=0$, and for $i >m$, $s^i=0$. Thus $\mathcal{N}(\ring)$ is an ideal. Now let $r\in \ring/ \mathcal{N}(\ring)$ be a nilpotent element. Then we have that for some sufficiently large $n$, $$ r^n + \mathcal{N}(\ring) = 0+ \mathcal{N}(\ring).$$ Thus we have that $r^n \in \mathcal{N}(\ring).$ Hence for some $k$, $r^{n^{k}} = 0$ i.e. $r^{nk}=0$ and so $r\in \mathcal{N}(\ring).$ Thus $\mathcal{N}(\ring)$ contains no nonzero nilpotent elements.
\end{document} 