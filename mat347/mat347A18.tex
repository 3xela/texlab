\documentclass[12pt, a4paper]{article}
\usepackage[lmargin =0.5 in, 
rmargin=0.5in, 
tmargin=1in,
bmargin=0.5in]{geometry}
\geometry{letterpaper}
\usepackage{amsmath}
\usepackage{amssymb}
\usepackage{blindtext}
\usepackage{titlesec}
\usepackage{enumitem}
\usepackage{fancyhdr}
\usepackage{amsthm}
\usepackage{graphicx}
\usepackage{cool}
\usepackage{thmtools}
\usepackage{hyperref}
\graphicspath{ }					%path to an image

%-------- sexy font ------------%
%\usepackage{libertine}
%\usepackage{libertinust1math}

%\usepackage{mlmodern}				% very nice and classic
%\usepackage[utopia]{mathdesign}
%\usepackage[T1]{fontenc}


\usepackage{mlmodern}
\usepackage{eulervm}
%\usepackage{tgtermes} 				%times new roman
%-------- sexy font ------------%


% Problem Styles
%====================================================================%


\newtheorem{problem}{Problem}


\theoremstyle{definition}
\newtheorem{thm}{Theorem}
\newtheorem{lemma}{Lemma}
\newtheorem{prop}{Proposition}
\newtheorem{cor}{Corollary}
\newtheorem{fact}{Fact}
\newtheorem{defn}{Definition}
\newtheorem{example}{Example}
\newtheorem{question}{Question}

\newtheorem{manualprobleminner}{Problem}

\newenvironment{manualproblem}[1]{%
	\renewcommand\themanualprobleminner{#1}%
	\manualprobleminner
}{\endmanualprobleminner}

\newcommand{\penum}{ \begin{enumerate}[label=\bf(\alph*), leftmargin=0pt]}
	\newcommand{\epenum}{ \end{enumerate} }

% Math fonts shortcuts
%====================================================================%

\newcommand{\ring}{\mathcal{R}}
\newcommand{\N}{\mathbb{N}}                           % Natural numbers
\newcommand{\Z}{\mathbb{Z}}                           % Integers
\newcommand{\R}{\mathbb{R}}                           % Real numbers
\newcommand{\C}{\mathbb{C}}                           % Complex numbers
\newcommand{\F}{\mathbb{F}}                           % Arbitrary field
\newcommand{\Q}{\mathbb{Q}}                           % Arbitrary field
\newcommand{\PP}{\mathcal{P}}                         % Partition
\newcommand{\M}{\mathcal{M}}                         % Mathcal M
\newcommand{\eL}{\mathcal{L}}                         % Mathcal L
\newcommand{\T}{\mathcal{T}}                         % Mathcal T
\newcommand{\U}{\mathcal{U}}                         % Mathcal U\\
\newcommand{\V}{\mathcal{V}}                         % Mathcal V

% symbol shortcuts
%====================================================================%

\newcommand{\lam}{\lambda}
\newcommand{\imp}{\implies}
\newcommand{\all}{\forall}
\newcommand{\exs}{\exists}
\newcommand{\delt}{\delta}
\newcommand{\eps}{\varepsilon}
\newcommand{\ra}{\rightarrow}

\newcommand{\ol}{\overline}
\newcommand{\f}{\frac}
\newcommand{\lf}{\lfrac}
\newcommand{\df}{\dfrac}

% bracketting shortcuts
%====================================================================%
\newcommand{\abs}[1]{\left| #1 \right|}
\newcommand{\babs}[1]{\Big|#1\Big|}
\newcommand{\bound}{\Big|}
\newcommand{\BB}[1]{\left(#1\right)}
\newcommand{\dd}{\mathrm{d}}
\newcommand{\artanh}{\mathrm{artanh}}
\newcommand{\Med}{\mathrm{Med}}
\newcommand{\Cov}{\mathrm{Cov}}
\newcommand{\Corr}{\mathrm{Corr}}
\newcommand{\tr}{\mathrm{tr}}
\newcommand{\Range}[1]{\mathrm{range}(#1)}
\newcommand{\Null}[1]{\mathrm{null}(#1)}
\newcommand{\lan}{\langle}
\newcommand{\ran}{\rangle}
\newcommand{\norm}[1]{\left\lVert#1\right\rVert}
\newcommand{\inn}[1]{\lan#1\ran}
\newcommand{\op}[1]{\operatorname{#1}}
\newcommand{\bmat}[1]{\begin{bmatrix}#1\end{bmatrix}}
\newcommand{\pmat}[1]{\begin{pmatrix}#1\end{pmatrix}}
\newcommand{\vmat}[1]{\begin{vmatrix}#1\end{vmatrix}}

\newcommand{\amogus}{{\bigcap}\kern-0.8em\raisebox{0.3ex}{$\subset$}}
\newcommand{\Note}{\textbf{Note: }}
\newcommand{\Aside}{{\bf Aside: }}
%restriction
%\newcommand{\op}[1]{\operatorname{#1}}
%\newcommand{\done}{$$\mathcal{QED}$$}

%====================================================================%


\setlength{\parindent}{0pt}      	% No paragraph indentations
\pagestyle{fancy}
\fancyhf{}							% fancy header

\setcounter{secnumdepth}{0}			% sections are numbered but numbers do not appear
\setcounter{tocdepth}{2} 			% no subsubsections in toc

%template
%====================================================================%
%\begin{manualproblem}{1}
%Spivak.
%\end{manualproblem}

%\begin{proof}[Solution]
%\end{proof}

%----------- or -----------%

%\begin{problem} 		
%\end{problem}	

%\penum
%	\item
%\epenum
%====================================================================%


\newcommand{\Course}{MAT347 }
\newcommand{\hwNumber}{18}

%preamble

\title{a}
\author{A.N.}
\date{\today}

\lhead{\Course A\hwNumber}
\rhead{\thepage}
%\cfoot{\thepage}


%====================================================================%
\begin{document}
	\begin{problem}
	\end{problem}
We factor the polynomial $f(x)$ as: 
$$ f(x) = x^6-9 = (x^3-3)(x^3+3) = (x - \sqrt[3]{3})(x - \omega\sqrt[3]{3})(x-\omega^2\sqrt[3]{3}) (x + \sqrt[3]{3})(x +\omega\sqrt[3]{3})(x + \omega^2 \sqrt[3]{3}),$$
where $\omega$ is third root of unity. Take $E = \Q(\omega, \sqrt[3]{3})$. We claim that this is the splitting field of $f(x)$. By above computation, $f$ splits over $E$. This field is also minimal, since it is the smallest field extension which contains $\sqrt[3]{3}$ and $\omega$. The degree of this extension is thus $6$ since adjoining $\omega$ is a degree $2$ extension, and adjoining $\sqrt[3]{3}$ is a degree $3$ extension. 
\newpage 
\begin{problem}
\end{problem}
Suppose that $x^d-1$ divides $x^n-1$ in $\Q[x]$. Then every root of $x^d-1$ is also a root of $x^n-1$. Let $\xi$ be a primitive $d'th$ root of unity. Then we have that $\xi^d =1$. Since $\xi$ is also a root of $x^n-1$ we have that $\xi^n =1$. Since $\xi$ is primitive we must have that $d|n$. Conversely suppose that $d|n$. We can write $$x^m-1 = \prod_{b|m} \Phi_{b}(x).$$
Since $d|n$ every $\Phi_b(x)$ that appears in the product expansion of $x^d-1$ will also appear in the expansion of $x^n-1$. Therefore the quotient $$\frac{x^n-1}{x^d-1} = \prod_{b|n, b\geq d} \Phi_b(x).$$ Since each $\Phi_b(x)\in \Q[x]$, the quotient is as well.  
\newpage
\begin{problem}
\end{problem}
Define $f(x) = x^{p^n-1}-1$ in $\F_{p^n}[x]$. We compute its formal derivative as 
$$Df(x) = (p^n-1)x^{p^n-2}= -(x^{p^n-2}) = 0 \iff x=0.$$ Therefore $f(x)$ has $p^n-1$ distinct nonzero roots, since $0$ is clearly not a root of $f$. We conclude that $f(x) = \prod_{x\in \F_{p^n}^\times}(x-u). $ Thus we have $$f(0) = -1 = \prod_{u\in \F_{p^n}^\times}-u \implies (-1)^{p^n}= \prod_{u\in \F_{p^n}^\times} u.$$
Taking $p\neq 2$ and $n=1$ we deduce Wilsons Theorem: $$(-1)^{p} = -1 = \prod_{u\in F_p^\times} u = (p-1)(p-2)\dots (2) = (p-1)!.$$
\newpage
\begin{problem}
\end{problem}
Suppose for the sake of contradiction that $E/\Q$ is a finite extension but contains infinitely many (distinct) roots of unity.  Then there must be an infinite subset of roots of unity with distinct orders. Thus the extension $E/ \Q$ must be infinite. A contradiction. 
\newpage 
\begin{problem}
\end{problem}
Suppose that an isomorphism $\varphi: \Q(\sqrt{p}) \to \Q(\sqrt{q})$ exists for distinct $p,q$. Then $$\varphi(p) = \varphi( \overbrace{1+\dots + 1}^{\text{p times}}) = \varphi(1)+ \dots \varphi(1) = 1+ \dots + 1 = p.$$ Let $\varphi(\sqrt{p}) = x$. Then $\varphi(\sqrt{p})^2 = \varphi(p) = x^2$. So $p = x^2$ in $\Q(\sqrt{q})$ i.e. $x = \sqrt{p}$. This is impossible clearly. 
\newpage 
\begin{problem}
\end{problem}
To determine the Galois group of $f(x) = x^3-3x+1$ we first determine its discriminant. We have that $s_2 = -1$ and $s_2 = -3$.  It follows that the discriminant is $D = -4(-3)^3-27(-1)^2 = 81$. This is a square over $\Q$ so we have that $Gal(f(x)) = A_3. $
\end{document}