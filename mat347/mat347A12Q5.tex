\documentclass[letterpaper]{article}
\usepackage[letterpaper,margin=1in,footskip=0.25in]{geometry}
\usepackage[utf8]{inputenc}
\usepackage{amsmath}
\usepackage{amsthm}
\usepackage{amssymb, pifont}
\usepackage{mathrsfs}
\usepackage{enumitem}
\usepackage{fancyhdr}
\usepackage{hyperref}

\pagestyle{fancy}
\fancyhf{}
\rhead{MAT 347}
\lhead{Assignment 13}
\rfoot{Page \thepage}

\setlength\parindent{24pt}
\renewcommand\qedsymbol{$\blacksquare$}

\DeclareMathOperator{\Qu}{\mathcal{Q}_8}
\DeclareMathOperator{\F}{\mathbb{F}}
\DeclareMathOperator{\T}{\mathcal{T}}
\DeclareMathOperator{\V}{\mathcal{V}}
\DeclareMathOperator{\U}{\mathcal{U}}
\DeclareMathOperator{\Prt}{\mathbb{P}}
\DeclareMathOperator{\R}{\mathbb{R}}
\DeclareMathOperator{\N}{\mathbb{N}}
\DeclareMathOperator{\Z}{\mathbb{Z}}
\DeclareMathOperator{\Q}{\mathbb{Q}}
\DeclareMathOperator{\C}{\mathbb{C}}
\DeclareMathOperator{\ep}{\varepsilon}
\DeclareMathOperator{\identity}{\mathbf{0}}
\DeclareMathOperator{\card}{card}
\newcommand{\suchthat}{;\ifnum\currentgrouptype=16 \middle\fi|;}

\newtheorem{lemma}{Lemma}

\newcommand{\normal}{\triangleleft}
\newcommand{\normaleq}{\trianglelefteq}
\newcommand{\tr}{\mathrm{tr}}
\newcommand{\ra}{\rightarrow}
\newcommand{\lan}{\langle}
\newcommand{\ran}{\rangle}
\newcommand{\norm}[1]{\left\lVert#1\right\rVert}
\newcommand{\inn}[1]{\lan#1\ran}
\newcommand{\ol}{\overline}
\newcommand{\ci}{i}
\newcommand{\ring}{\mathcal{R}}
\begin{document} \noindent Q5: First note that since $\F_3$ is a field, constant numbers are irreducible. 
Next, we have that linear terms $x,(x-1),(x-2)$ are all irreducible, since they are all linear. This is also an exhaustive list of all degree 1 irreducible polynomials,
since they each have a unique root. Note that for degree 2 and 3 polynomials, it is sufficient to check when they are monic, and their constant term is nonzero. 
The complete list of degree 2 monic polynomials over $\F_3$ are: 
\begin{align*}
    x^2+ 1\\ x^2+2 \\ x^2+x+1 \\ x^2+x+2\\ x^2+2x+1 \\ x^2+2x+2
\end{align*}
Applying Prop. 9 and Prop 10, it is enough to evaluate each polynomial on $\F_3$, and the ones which do not vanish
will therefore be irreducible. We see that the only such polynomials are $$x^2+1, x^2+x+2, x^2+2x+2.$$ 
Finally, we will find the degree $3$ irreducible polynomials. Similarly to the degree 2 case, we use Prop. 9 and Prop. 10. 
We can find such degree $3$ polynomials by checking all degree 3 monic polynommials who do not vanish at 0. 
First note that the two simplest cases that are irreducible are $$x(x-1)(x-2)+1 = x^3+2x+1, x(x-1)(x-2)+2= x^3+2x+2.$$ 
These are irreducible by construction. They will never be 0.
Checking the remaining degree polynomials, we have that the following are irreducible: 
\begin{align*} x^3+x^2+2 \\ x^3+x^2+x+2 \\ x^3+2x^2+x+1 \\x^3+2x^2+2x+2 \\ x^3+x^2+2x+1 \\ x^3+2x^2+1   \end{align*}
Thus by brute force these are all the irreducible polynomials over $\F_3$ of degree less than 3. 
\end{document}