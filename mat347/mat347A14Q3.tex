\documentclass[letterpaper]{article}
\usepackage[letterpaper,margin=1in,footskip=0.25in]{geometry}
\usepackage[utf8]{inputenc}
\usepackage{amsmath}
\usepackage{amsthm}
\usepackage{amssymb, pifont}
\usepackage{mathrsfs}
\usepackage{enumitem}
\usepackage{fancyhdr}
\usepackage{hyperref}

\pagestyle{fancy}
\fancyhf{}
\rhead{MAT 347}
\lhead{Assignment 14}
\rfoot{Page \thepage}

\setlength\parindent{24pt}
\renewcommand\qedsymbol{$\blacksquare$}

\DeclareMathOperator{\Qu}{\mathcal{Q}_8}
\DeclareMathOperator{\F}{\mathbb{F}}
\DeclareMathOperator{\T}{\mathcal{T}}
\DeclareMathOperator{\V}{\mathcal{V}}
\DeclareMathOperator{\U}{\mathcal{U}}
\DeclareMathOperator{\Prt}{\mathbb{P}}
\DeclareMathOperator{\R}{\mathbb{R}}
\DeclareMathOperator{\N}{\mathbb{N}}
\DeclareMathOperator{\Z}{\mathbb{Z}}
\DeclareMathOperator{\Q}{\mathbb{Q}}
\DeclareMathOperator{\C}{\mathbb{C}}
\DeclareMathOperator{\ep}{\varepsilon}
\DeclareMathOperator{\identity}{\mathbf{0}}
\DeclareMathOperator{\card}{card}
\newcommand{\suchthat}{;\ifnum\currentgrouptype=16 \middle\fi|;}

\newtheorem{lemma}{Lemma}

\newcommand{\normal}{\triangleleft}
\newcommand{\normaleq}{\trianglelefteq}
\newcommand{\tr}{\mathrm{tr}}
\newcommand{\ra}{\rightarrow}
\newcommand{\lan}{\langle}
\newcommand{\ran}{\rangle}
\newcommand{\norm}[1]{\left\lVert#1\right\rVert}
\newcommand{\inn}[1]{\lan#1\ran}
\newcommand{\ol}{\overline}
\newcommand{\ci}{i}
\newcommand{\ring}{\mathcal{R}}
\begin{document} \noindent Q3i: 
Any vector in $V$ is of the form $v = \lambda(e_1 + \dots + e_n)$ for some $\lambda \in \C$. Applying any permutation $\sigma \in S_n$ we have that
$$\sigma \cdot v = \lambda (e_{\sigma(1)} + \dots + e_{\sigma(n)}) = v.$$
Therefore $V$ is invariant under $\C[S_n]$ if we extend by linearity over $\C$. 
\newline \\ Q3ii: Suppose $w\in W$. Then if for any permutation $\sigma \in S_n$, we have that $$\sigma \cdot w = \sum_{i}^n w_i e_{\sigma(i)}.$$
So $\sigma \cdot w \in W$ since this only relabels the basis vectors but does not adjust the coefficients. When we extend by linearity since $W$ is a subspace we still remain in $W$. 
\newline \\ Q3iii: We claim that $\C^n = V \oplus W$. Observe that if $v\in W \cap V$, then $v =\lambda(e_1 + \dots e_n)  $ and $n\lambda =0$ i.e. $\lambda =0$.
We now claim that any $u\in V$ can be written in the form $u =v+w $ for $v\in V$, $w\in W$. In the basis $\{e_1, \dots , e_n\}$ we write $u = u_1 e_1 + \dots + u_n e_n$. We see that by taking $\lambda = \frac{1}{n} \sum_{i}^n u_i$, and $w_i = u_i - \lambda$, we see $$u= \lambda (e_1 + \dots + e_n) + (u_1 - \lambda)e_1 + \dots + (u_n - \lambda) e_n .$$ 
The coefficients on our choice of $W$ work since $$\sum_{i=1}^n (u_i - \lambda ) = \sum_{i=1}^n u_i - n\lambda = 0. $$
hello
\end{document}