\documentclass[letterpaper]{article}
\usepackage[letterpaper,margin=1in,footskip=0.25in]{geometry}
\usepackage[utf8]{inputenc}
\usepackage{amsmath}
\usepackage{amsthm}
\usepackage{amssymb, pifont}
\usepackage{mathrsfs}
\usepackage{enumitem}
\usepackage{fancyhdr}
\usepackage{hyperref}

\pagestyle{fancy}
\fancyhf{}
\rhead{MAT 347}
\lhead{Assignment 9}
\rfoot{Page \thepage}

\setlength\parindent{24pt}
\renewcommand\qedsymbol{$\blacksquare$}

\DeclareMathOperator{\Qu}{\mathcal{Q}_8}
\DeclareMathOperator{\F}{\mathbb{F}}
\DeclareMathOperator{\T}{\mathcal{T}}
\DeclareMathOperator{\V}{\mathcal{V}}
\DeclareMathOperator{\U}{\mathcal{U}}
\DeclareMathOperator{\Prt}{\mathbb{P}}
\DeclareMathOperator{\R}{\mathbb{R}}
\DeclareMathOperator{\N}{\mathbb{N}}
\DeclareMathOperator{\Z}{\mathbb{Z}}
\DeclareMathOperator{\Q}{\mathbb{Q}}
\DeclareMathOperator{\C}{\mathbb{C}}
\DeclareMathOperator{\ep}{\varepsilon}
\DeclareMathOperator{\identity}{\mathbf{0}}
\DeclareMathOperator{\card}{card}
\newcommand{\suchthat}{;\ifnum\currentgrouptype=16 \middle\fi|;}

\newtheorem{lemma}{Lemma}

\newcommand{\normal}{\triangleleft}
\newcommand{\normaleq}{\trianglelefteq}
\newcommand{\tr}{\mathrm{tr}}
\newcommand{\ra}{\rightarrow}
\newcommand{\lan}{\langle}
\newcommand{\ran}{\rangle}
\newcommand{\norm}[1]{\left\lVert#1\right\rVert}
\newcommand{\inn}[1]{\lan#1\ran}
\newcommand{\ol}{\overline}
\newcommand{\ci}{i}
\begin{document} 
\noindent Q3: If $A\in I$ it will take the form of $$A = \begin{bmatrix}
    0 & \cdots & 0 \\ \vdots  & \vdots & \vdots \\ a_{i1} & \cdots & a_{in} \\ \vdots & \vdots & \vdots \\ 0 & \cdots & 0
\end{bmatrix}$$
 for some $i$. The set of all matrices in this form will be closed under addition since matrix multiplication is done entry-wise. We compute right multiplication with an element of $\mathcal{R}$ as 
 $$MA = \begin{bmatrix}
    0 & \cdots & 0 \\ \vdots  & \vdots & \vdots \\ a_{i1} & \cdots & a_{in} \\ \vdots & \vdots & \vdots \\ 0 & \cdots & 0
\end{bmatrix} \cdot \begin{bmatrix}
    b_{11} & \cdots & b_{1n} \\ \vdots & \vdots & \vdots \\ b_{n1} & \cdots & b_{nn}
 \end{bmatrix} = \begin{bmatrix}
    0 & \cdots & 0 \\ \vdots & \vdots & \vdots \\ \sum_{j=1}^n a_{ij}b_{j1} & \cdots & \sum_{j=1}^n a_{ij}b_{jn} \\ \vdots &\vdots & \vdots \\ 0 & \cdots & 0  
 \end{bmatrix} $$
 Which is of the form that we desire. Hence this is a right ideal. We compute that the left multiplication will be 
 $$AM = \begin{bmatrix}
    b_{11} & \cdots & b_{1n} \\ \vdots & \vdots & \vdots \\ b_{n1} & \cdots & b_{nn}
 \end{bmatrix} \cdot \begin{bmatrix}
    0 & \cdots & 0 \\ \vdots  & \vdots & \vdots \\ a_{i1} & \cdots & a_{in} \\ \vdots & \vdots & \vdots \\ 0 & \cdots & 0
\end{bmatrix}= \begin{bmatrix}
    b_{i1}a_{i1} & \cdots & b_{in}a_{i1} \\ \vdots & \vdots & \vdots \\ b_{in}a_{i1} & \cdots & b_{in}a_{in}

\end{bmatrix}$$ We note that this will not belong to the left ideal unless $n=1$ which is just multiplication of real numbers. 
\end{document}