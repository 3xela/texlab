\documentclass[letterpaper]{article}
\usepackage[letterpaper,margin=1in,footskip=0.25in]{geometry}
\usepackage[utf8]{inputenc}
\usepackage{amsmath}
\usepackage{amsthm}
\usepackage{amssymb, pifont}
\usepackage{mathrsfs}
\usepackage{enumitem}
\usepackage{fancyhdr}
\usepackage{hyperref}

\pagestyle{fancy}
\fancyhf{}
\rhead{MAT 347}
\lhead{Assignment 6}
\rfoot{Page \thepage}

\setlength\parindent{24pt}
\renewcommand\qedsymbol{$\blacksquare$}

\DeclareMathOperator{\Qu}{\mathcal{Q}_8}
\DeclareMathOperator{\F}{\mathbb{F}}
\DeclareMathOperator{\T}{\mathcal{T}}
\DeclareMathOperator{\V}{\mathcal{V}}
\DeclareMathOperator{\U}{\mathcal{U}}
\DeclareMathOperator{\Prt}{\mathbb{P}}
\DeclareMathOperator{\R}{\mathbb{R}}
\DeclareMathOperator{\N}{\mathbb{N}}
\DeclareMathOperator{\Z}{\mathbb{Z}}
\DeclareMathOperator{\Q}{\mathbb{Q}}
\DeclareMathOperator{\C}{\mathbb{C}}
\DeclareMathOperator{\ep}{\varepsilon}
\DeclareMathOperator{\identity}{\mathbf{0}}
\DeclareMathOperator{\card}{card}
\newcommand{\suchthat}{;\ifnum\currentgrouptype=16 \middle\fi|;}

\newtheorem{lemma}{Lemma}

\newcommand{\normal}{\trianglelefteq}
\newcommand{\tr}{\mathrm{tr}}
\newcommand{\ra}{\rightarrow}
\newcommand{\lan}{\langle}
\newcommand{\ran}{\rangle}
\newcommand{\norm}[1]{\left\lVert#1\right\rVert}
\newcommand{\inn}[1]{\lan#1\ran}
\newcommand{\ol}{\overline}
\newcommand{\ci}{i}
\begin{document} 
\noindent Q5i: Since any any $\phi \in Aut(G)$ is determined by $\phi(\ol{1}) = a \ol{1}$, for some $a\in \F_p$ it is sufficient to determine which $a\in \F_p$ will give an invertible map. Note that from number theory, the elements in $\F_p$ which have multiplicative inverses are the nonzero elements. Hence we can correspond every automorphism of $G$ with an element $\F_p^\times$ by identifiying $\phi$ with $\phi(1)$. Hence $$Aut(G) \cong F_p^\times $$
\newline \\ Q5ii: Similarly to 5i, an automorphism $\phi \in Aut(\Z / n \Z)$ will be determined by $\phi(\ol{1})= a \ol{1}. $ Hence the set of all automorphisms can be corresponded to the set of all $a\in \Z / n\Z $ which have multiplicative inverses. We know from number theory this is exactly the set of all $a\in Z / n\Z$ such that $\gcd(a,n)=1$ i.e. the unit group of $\Z / n\Z$. 


\end{document}