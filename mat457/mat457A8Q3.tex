\documentclass[letterpaper]{article}
\usepackage[letterpaper,margin=1in,footskip=0.25in]{geometry}
\usepackage[utf8]{inputenc}
\usepackage{amsmath}
\usepackage{amsthm}
\usepackage{amssymb, pifont}
\usepackage{mathrsfs}
\usepackage{enumitem}
\usepackage{fancyhdr}
\usepackage{hyperref}

\pagestyle{fancy}
\fancyhf{}
\rhead{MAT 457}
\lhead{Assignment 8}
\rfoot{Page \thepage}

\setlength\parindent{24pt}
\renewcommand\qedsymbol{$\blacksquare$}

\DeclareMathOperator{\E}{\mathcal{E}}
\DeclareMathOperator{\M}{\mathcal{M}}
\DeclareMathOperator{\F}{\mathbb{F}}
\DeclareMathOperator{\T}{\mathcal{T}}
\DeclareMathOperator{\V}{\mathcal{V}}
\DeclareMathOperator{\U}{\mathcal{U}}
\DeclareMathOperator{\Prt}{\mathbb{P}}
\DeclareMathOperator{\R}{\mathbb{R}}
\DeclareMathOperator{\N}{\mathbb{N}}
\DeclareMathOperator{\Z}{\mathbb{Z}}
\DeclareMathOperator{\Q}{\mathbb{Q}}
\DeclareMathOperator{\C}{\mathbb{C}}
\DeclareMathOperator{\ep}{\varepsilon}
\DeclareMathOperator{\identity}{\mathbf{0}}
\DeclareMathOperator{\card}{card}
\newcommand{\suchthat}{;\ifnum\currentgrouptype=16 \middle\fi|;}

\newtheorem{lemma}{Lemma}

\newcommand{\tr}{\mathrm{tr}}
\newcommand{\ra}{\rightarrow}
\newcommand{\lan}{\langle}
\newcommand{\ran}{\rangle}
\newcommand{\norm}[1]{\left\lVert#1\right\rVert}
\newcommand{\inn}[1]{\lan#1\ran}
\newcommand{\ol}{\overline}
\newcommand{\ci}{i}
\newcommand{\X}{\mathfrak{X}}
\begin{document}
Q3: If $f \equiv 0$ then this is clearly true. Therefore we can assume that $f$ is not identically 0. 
By the proof of Folland 6.10, we have that $$\norm{f}_q \leq \norm{f}_p^{\frac{p}{q}} \norm{f}_\infty^{1- \frac{p}{q}}.$$ 
Taking the $\lim \sup$ as $q\to \infty$ we get that $$\lim_{q \to \infty} \sup \norm{f}_q \leq \lim_{q\to \infty} \sup \norm{f}_p^{\frac{p}{q}} \norm{f}_\infty^{1- \frac{p}{q}} = \norm{f}_{\infty}.$$
Now choose $M$ with $0<M < \norm{f}_\infty$. We define the set $E_M = \{x: |f(x)| \geq M\}.$ Since $$\norm{f}_q^q = \int |f|^q d\mu,$$ monotonicity of the integral implies that $$M^q \mu(E_M) \leq \norm{f}_q^q$$
and so $$M \mu(E_m)^{\frac{1}{q}} \leq \norm{f}_q.$$ 
Therefore $$M \lim_{q\to \infty } \inf \mu(E_M)^\frac{1}{q} \leq \lim_{q \to \infty} \inf \norm{f}_q .$$ Taking the limit as $M \to \norm{f}_{\infty}$ yields $$ \norm{f}_\infty \leq \lim_{q\to \infty} \inf \norm{f}_q.$$ Thus we have that $$\lim_{q \to \infty } \sup \norm{f}_q \leq \norm{f}_\infty \leq \lim_{q\to \infty} \inf \norm{f}_q.$$ Thus we have equality. 
\end{document}