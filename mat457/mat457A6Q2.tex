\documentclass[letterpaper]{article}
\usepackage[letterpaper,margin=1in,footskip=0.25in]{geometry}
\usepackage[utf8]{inputenc}
\usepackage{amsmath}
\usepackage{amsthm}
\usepackage{amssymb, pifont}
\usepackage{mathrsfs}
\usepackage{enumitem}
\usepackage{fancyhdr}
\usepackage{hyperref}

\pagestyle{fancy}
\fancyhf{}
\rhead{MAT 457}
\lhead{Assignment 6}
\rfoot{Page \thepage}

\setlength\parindent{24pt}
\renewcommand\qedsymbol{$\blacksquare$}

\DeclareMathOperator{\E}{\mathcal{E}}
\DeclareMathOperator{\M}{\mathcal{M}}
\DeclareMathOperator{\F}{\mathbb{F}}
\DeclareMathOperator{\T}{\mathcal{T}}
\DeclareMathOperator{\V}{\mathcal{V}}
\DeclareMathOperator{\U}{\mathcal{U}}
\DeclareMathOperator{\Prt}{\mathbb{P}}
\DeclareMathOperator{\R}{\mathbb{R}}
\DeclareMathOperator{\N}{\mathbb{N}}
\DeclareMathOperator{\Z}{\mathbb{Z}}
\DeclareMathOperator{\Q}{\mathbb{Q}}
\DeclareMathOperator{\C}{\mathbb{C}}
\DeclareMathOperator{\ep}{\varepsilon}
\DeclareMathOperator{\identity}{\mathbf{0}}
\DeclareMathOperator{\card}{card}
\newcommand{\suchthat}{;\ifnum\currentgrouptype=16 \middle\fi|;}

\newtheorem{lemma}{Lemma}

\newcommand{\tr}{\mathrm{tr}}
\newcommand{\ra}{\rightarrow}
\newcommand{\lan}{\langle}
\newcommand{\ran}{\rangle}
\newcommand{\norm}[1]{\left\lVert#1\right\rVert}
\newcommand{\inn}[1]{\lan#1\ran}
\newcommand{\ol}{\overline}
\newcommand{\ci}{i}
\begin{document}
\noindent Q2: Since $F(x)$ is right continuous, and defined on $[a,b]$ we have that $F(x) \in L^1$. Furthermore we have that each $F_j \in L^1$. By Folland 2.23 b, it is sufficient to show that $$ \int_{(a,x]} F^\prime(t) dt  =  \int_{(a,x]} \sum_{j}F^\prime (t) dt. $$
Note that by Lebesgue Radon Nikodyn Theorem, we can write $$\mu_F = \nu_F + \rho_F = \nu_F + \int f dm ,$$ and for each $F_j$ we have $$\mu_{F_j} = \nu_{F_j} + \int f_j dm.$$ Summing over all $j$ we get that $$\nu_F + \int f dm = \mu_F = \sum_{j} \mu_{F_j} = \sum_{j} \nu_{F_j} + \sum_{j} \int f_j dm. $$ 
We claim that this is a Lebesgue-Radon-Nikodyn decomposition of $\mu_F$. Note that if $\mu_F(E)=0$, then for each $j$, $\mu_{F_j}(E) =0$. Therefore for each $f_j, \int_{E} f_j dm =0$. Hence their countable sum is $0$ as well. We now claim that $\sum_{j} \nu_{F_j} \bot m$. For each $\nu_{F_j}$ let $E_j$ be the set which it is $0$ on. We have that for $E_j^c$, $m|_{E_j^c} =0$. 
Take the set $E = \bigcap E_j$. If this intersection is empty then the result holds trivially. If not, then we have that $\sum_{j} \nu_{F_j} |_{E} = 0$. Similarly, $m|_{E^c}= 0$. Hence this is a L-R-N decomposition. 
Mutual singularity of $\nu_{F_i}$ implies that on an $a.e.$ set $$\int f dm = \sum_{j} \int f_j dm.$$ Furthermore, by DCT we have that $$\int f dm = \int \sum_{j} f_j dm.$$ Hence we have that $f = \sum_{j} f_j.$ Thus we compute that $$\int_{(a,x]} F^\prime(t) dt = \int_{(a,x]} d \mu_F = \int_{(a,x]} d \nu_F + f dm = \int_{(a,x]} d \sum_{j} d\nu_{F_j} + d \sum_{j} \mu_{F_j} = \int_{(a,x]} \sum_{j} F^\prime(t) dt $$

\end{document}