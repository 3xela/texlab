\documentclass[letterpaper]{article}
\usepackage[letterpaper,margin=1in,footskip=0.25in]{geometry}
\usepackage[utf8]{inputenc}
\usepackage{amsmath}
\usepackage{amsthm}
\usepackage{amssymb, pifont}
\usepackage{mathrsfs}
\usepackage{enumitem}
\usepackage{fancyhdr}
\usepackage{hyperref}

\pagestyle{fancy}
\fancyhf{}
\rhead{MAT 457}
\lhead{Assignment 1}
\rfoot{Page \thepage}

\setlength\parindent{24pt}
\renewcommand\qedsymbol{$\blacksquare$}

\DeclareMathOperator{\M}{\mathcal{M}}
\DeclareMathOperator{\F}{\mathbb{F}}
\DeclareMathOperator{\T}{\mathcal{T}}
\DeclareMathOperator{\V}{\mathcal{V}}
\DeclareMathOperator{\U}{\mathcal{U}}
\DeclareMathOperator{\Prt}{\mathbb{P}}
\DeclareMathOperator{\R}{\mathbb{R}}
\DeclareMathOperator{\N}{\mathbb{N}}
\DeclareMathOperator{\Z}{\mathbb{Z}}
\DeclareMathOperator{\Q}{\mathbb{Q}}
\DeclareMathOperator{\C}{\mathbb{C}}
\DeclareMathOperator{\ep}{\varepsilon}
\DeclareMathOperator{\identity}{\mathbf{0}}
\DeclareMathOperator{\card}{card}
\newcommand{\suchthat}{;\ifnum\currentgrouptype=16 \middle\fi|;}

\newtheorem{lemma}{Lemma}

\newcommand{\tr}{\mathrm{tr}}
\newcommand{\ra}{\rightarrow}
\newcommand{\lan}{\langle}
\newcommand{\ran}{\rangle}
\newcommand{\norm}[1]{\left\lVert#1\right\rVert}
\newcommand{\inn}[1]{\lan#1\ran}
\newcommand{\ol}{\overline}
\newcommand{\ci}{i}
\begin{document}
\noindent
Q1a: Since $\M$ is an infinite $\sigma$-algebra, there exists an infinite sequence of nonempty sets $\{ E_i\}$ belonging to $\M$ whose countable union and intersection belong to $\M$ and $E_i\in \M $ implies that $E_i^c \in \M$. We can generate a disjoint sequence of sets belonging to $\M$ in the following way. Define $F_1 = E_1$, and $F_n = \cup_{i=1}^n E_i \setminus \cup_{i=1}^{n-1} E_i $. We note that by construction, the $F_i$'s are disjoint from one another, and that they belong to $\M$, since they are built using unions and compliments. Note as well that not all $F_i$'s are empty, since we presume that $\M$ is infinite hence we can take distinct $E_i$. We have the desired result.
\newline \\ Q1b: We now claim that $\M$ is uncountable. It is sufficient to construct a subset of $\M$ that is uncountable, since we can find an injection from any such subset to $\M$. Let $\Prt(\N)$ be the power set of the naturals. Now  for each $\lambda\in \Prt(\N)$, define 
$$A_{\lambda} = \bigcup_{i\in \lambda} F_i$$
Each $A_\lambda$ is a countable union of sets in $\M$, hence it belongs to $\M$. Furthermore if we consider $\{A_\lambda\}_{\lambda \in \Prt(\N)}$, we can identify each $A_\lambda$ with $\lambda\in \Prt(\N)$ bijectively. It is a fact that $\Prt(\N)$ is uncountable, hence $\{A_\lambda \}_{\lambda \in \Prt(\N)}$ is our desired uncountable subset of $\M$. 
\end{document}