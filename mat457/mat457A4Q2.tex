\documentclass[letterpaper]{article}
\usepackage[letterpaper,margin=1in,footskip=0.25in]{geometry}
\usepackage[utf8]{inputenc}
\usepackage{amsmath}
\usepackage{amsthm}
\usepackage{amssymb, pifont}
\usepackage{mathrsfs}
\usepackage{enumitem}
\usepackage{fancyhdr}
\usepackage{hyperref}

\pagestyle{fancy}
\fancyhf{}
\rhead{MAT 457}
\lhead{Assignment 4}
\rfoot{Page \thepage}

\setlength\parindent{24pt}
\renewcommand\qedsymbol{$\blacksquare$}

\DeclareMathOperator{\E}{\mathcal{E}}
\DeclareMathOperator{\M}{\mathcal{M}}
\DeclareMathOperator{\F}{\mathbb{F}}
\DeclareMathOperator{\T}{\mathcal{T}}
\DeclareMathOperator{\V}{\mathcal{V}}
\DeclareMathOperator{\U}{\mathcal{U}}
\DeclareMathOperator{\Prt}{\mathbb{P}}
\DeclareMathOperator{\R}{\mathbb{R}}
\DeclareMathOperator{\N}{\mathbb{N}}
\DeclareMathOperator{\Z}{\mathbb{Z}}
\DeclareMathOperator{\Q}{\mathbb{Q}}
\DeclareMathOperator{\C}{\mathbb{C}}
\DeclareMathOperator{\ep}{\varepsilon}
\DeclareMathOperator{\identity}{\mathbf{0}}
\DeclareMathOperator{\card}{card}
\newcommand{\suchthat}{;\ifnum\currentgrouptype=16 \middle\fi|;}

\newtheorem{lemma}{Lemma}

\newcommand{\tr}{\mathrm{tr}}
\newcommand{\ra}{\rightarrow}
\newcommand{\lan}{\langle}
\newcommand{\ran}{\rangle}
\newcommand{\norm}[1]{\left\lVert#1\right\rVert}
\newcommand{\inn}[1]{\lan#1\ran}
\newcommand{\ol}{\overline}
\newcommand{\ci}{i}
\begin{document}
\noindent
We claim that $\rho$ is a metric on measurable functions on $X$. First, note that $$\rho(f,g) = \int \frac{|f-g|}{1+|f-g|}d \mu = \int \frac{|g-f|}{1+|g-f|} d\mu = \rho(g,f)$$
We now claim positive definiteness. First note that the function $$0\leq \frac{|f-g|}{1+|f-g|} \leq 1$$ for all $f,g$. Therefore we have that for all $f,g$, $$0\leq \int_X \frac{|f-g|}{1+|f-g|} \leq \int_X 1 = \mu(X) < \infty$$ We claim that $\rho(f,g)=0 $ if and only if $f=g$. 
First suppose that $\rho(f,g)=0$ i.e. $$\int_{X} \frac{|f-g|}{1+|f-g|} =0$$ We know from properties of the integral that if the integral of a positive function is 0, then the function must be 0 almost everywhere. Therefore $$\frac{|f-g|}{1+|f-g|}=0$$ almost everywhere. This is only satisfied when $|f-g|=0$ a.e., which is the same as $f=g$ a.e. 
If $f=g$ a.e. We denote $E$ to be the measure 0 set where they differ. Then $$\rho(f,g) = \int_X \frac{|f-g|}{1+|f-g|} = \int_E \frac{|f-g|}{1+|f-g|} + \int_{E^c} \frac{|f-g|}{1+|f-g|} = \int_{E} \frac{|f-g|}{1+|f-g|} =0$$
Where we use the fact that the integral over a measure 0 set will be 0. 
We now claim that the triangle inequality holds. Let $f,g,h$ be measurable functions. It is sufficient to show that $$\rho(g,h) + \rho(h,f) - \rho(f,g)\geq 0$$
We can compute that 
\begin{align*}
    \rho(g,h) + \rho(h,f) - \rho(f,g) & = \int \frac{|g-h|}{1+|g-h|} + \frac{|h-f|}{1+|h-f|} - \frac{|f-g|}{1+|f-g|} d\mu
    \\ & = \int \frac{2(|h-f|\cdot |g-h|) +(|h-f|\cdot |g-h|\cdot|f-g|) + |h-f| +|g-h| - |f-g|}{(1+|g-h|)(1+|h-f|)(1+|f-g|)}
    \\ & \geq \int \frac{2(|h-f|\cdot |g-h|) +(|h-f|\cdot |g-h|\cdot|f-g|)}{(1+|g-h|)(1+|h-f|)(1+|f-g|)} \tag{by triangle inequality}
    \\ & \geq 0 \tag{since this function is positive}
\end{align*}
Hence we conclude that $\rho$ is a metric. We now claim that $f_n \to f$ in measure if and only if $f_n \to f$ in the $\rho$ metric. We first claim that for any $f_n$, $\frac{|f_n-f|}{1+|f_n-f|} \in L^1$. It is clear, since $$\frac{|f_n-f|}{1+|f_n-f|} \leq 1 \implies \int_X \frac{|f_n-f|}{1+|f_n-f|} d\mu \leq \int_X 1 d\mu = \mu(X)<\infty$$
This function is therefore in $L^1$. Now we let $\varepsilon >0$, it has been proven in class that there exists a $\delta>0$ such that $\int_E |\frac{|f_n-f|}{1+|f_n-f|}| < \ep$ for any $E$ with $\mu(E)< \delta$. Take $E_{n} = \{x: |f_n-f| \geq \ep \}$. By convergence in measure, take $n$ sufficiently large so that $\mu(E_{n})<\delta$. We see that 
\begin{align*} \rho(f_n,f) &= \int_{X} \frac{|f_n-f|}{1+|f_n-f|} d\mu 
    \\ & = \int_{E_{n}} \frac{|f_n-f|}{1+|f_n-f|} d\mu + \int_{E_n^c} \frac{|f_n-f|}{1+|f_n-f|} d\mu
    \\ & \leq \ep + \int_{E_n^c} \frac{\ep}{1+|f_n-f|} d\mu
    \\ & \leq \ep + \ep \int_{X} 1 d \mu 
    \\ & \leq \ep + \ep(\mu(X))
\end{align*}
Since $\ep$ is arbitrary, we have that $\rho(f_n,f) \to 0$ as $n\to \infty$. Now suppose that $\rho(f_n,f)\to 0$ as $n\to \infty$. Let $\ep ,\delta >0$. Choose $n$ sufficiently large so that $\rho(f_n,f) <\delta$.  Define $E_{\ep, n} = \{x: |f_n-f| \geq \ep \}$. We have that
$$\rho(f_n,f)<\delta \implies \int_{E_{\ep,n}} \frac{|f_n-f|}{1+|f_n-f|} + \int_{(E_{\ep,n})^c} \frac{|f_n-f|}{1+|f_n-f|} <\delta$$
Positiveness of $\frac{|f_n-f|}{1+|f_n-f|}$ implies that $$\int_{E_{\ep,n}} \frac{|f_n-f|}{1+|f_n-f|}<\delta$$ Since for positive $F$, the mapping $E\mapsto \int_E F d\mu$ is a measure, this implies that $\mu(E_{\ep,n}) <\delta$ for arbitrarily large $n$, as desired. 
\end{document}