\documentclass[letterpaper]{article}
\usepackage[letterpaper,margin=1in,footskip=0.25in]{geometry}
\usepackage[utf8]{inputenc}
\usepackage{amsmath}
\usepackage{amsthm}
\usepackage{amssymb, pifont}
\usepackage{mathrsfs}
\usepackage{enumitem}
\usepackage{fancyhdr}
\usepackage{hyperref}

\pagestyle{fancy}
\fancyhf{}
\rhead{MAT 457}
\lhead{Assignment 7}
\rfoot{Page \thepage}

\setlength\parindent{24pt}
\renewcommand\qedsymbol{$\blacksquare$}

\DeclareMathOperator{\E}{\mathcal{E}}
\DeclareMathOperator{\M}{\mathcal{M}}
\DeclareMathOperator{\F}{\mathbb{F}}
\DeclareMathOperator{\T}{\mathcal{T}}
\DeclareMathOperator{\V}{\mathcal{V}}
\DeclareMathOperator{\U}{\mathcal{U}}
\DeclareMathOperator{\Prt}{\mathbb{P}}
\DeclareMathOperator{\R}{\mathbb{R}}
\DeclareMathOperator{\N}{\mathbb{N}}
\DeclareMathOperator{\Z}{\mathbb{Z}}
\DeclareMathOperator{\Q}{\mathbb{Q}}
\DeclareMathOperator{\C}{\mathbb{C}}
\DeclareMathOperator{\ep}{\varepsilon}
\DeclareMathOperator{\identity}{\mathbf{0}}
\DeclareMathOperator{\card}{card}
\newcommand{\suchthat}{;\ifnum\currentgrouptype=16 \middle\fi|;}

\newtheorem{lemma}{Lemma}

\newcommand{\tr}{\mathrm{tr}}
\newcommand{\ra}{\rightarrow}
\newcommand{\lan}{\langle}
\newcommand{\ran}{\rangle}
\newcommand{\norm}[1]{\left\lVert#1\right\rVert}
\newcommand{\inn}[1]{\lan#1\ran}
\newcommand{\ol}{\overline}
\newcommand{\ci}{i}
\begin{document}
\noindent Q1: We first define $\Omega = \{(x,y): a< x\leq y \leq b\}$. Using Fubini-Tonelli Theorem, we compute 
\begin{align*}
    \mu_F\times \mu_G(\Omega) &= \int_{(a,b]} \int_{(a,y]} dF(x)dG(y)
    \\ & = \int_{(a,b]} [F(y) - F(a)]dG(y)
    \\ & = \int_{(a,b]} \varDelta F dG(y) + \int_{(a,b]} F_{-}dG(y) - \int_{(a,b]} F(a)dG(y) \tag{since $F = \varDelta F + F_{-}$ and linearity}
    \\ & = \sum_{x, \varDelta F\neq 0}\int_{x} \varDelta F dG(x) + \int_{(a,b]} F_{-}dG(x) - F(a)[G(b) - G(a)] \tag{$\varDelta F$ nonzero on countable points}
    \\ & = \sum_{x, \varDelta F \neq 0} \varDelta F(x) \varDelta G(x) + \int_{(a,b]} F_{-}dG(x) - F(a) [ G(b) - G(a)] \tag{using the definition of $dG$}
\end{align*}
Similarly, we compute that 
\begin{align*}
    \mu_F \times \mu_G (\Omega) &= \int_{(a,b]} \int_{(x,b]} dG(y) dF(x)
    \\ & = \int_{(a,b]} [G(b) - G(x)]dF(x) \tag{take decreasing sequence to $(x,b]$, apply downward measure cont.}
    \\ & = \int_{(a,b]} G(b) dF(x) - \int_{(a,b]} \varDelta G(x)dF(x) - \int_{(a,b]} G_{-}(x) dF(x) 
    \\ & = G(b)[F(b)-F(a)] - \sum_{x, \varDelta G(x) \neq 0} \int_{x} \varDelta G(x)dF(x) - \int_{(a,b]} G_{-}dF(x) 
    \\ & = G(b)[F(b) - F(a)] - \sum_{x, \varDelta G(x) \neq 0 } \varDelta F(x) \varDelta G(x) -\int_{(a,b]} G_{-}dF(x)
\end{align*}
Taking the differences we get that $$G(b)F(b)- G(a)F(a) = \int_{(a,b]} F_{-}dG + \int_{(a,b]}G_{-}dF  + \sum_{a< x\leq b} \varDelta G(x) \varDelta F(x)$$
Since the points where either is $\varDelta G(x)$ or $\varDelta F(x)$ are zero will vanish in either of the equations we take the difference. 
\end{document}