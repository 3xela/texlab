\documentclass[letterpaper]{article}
\usepackage[letterpaper,margin=1in,footskip=0.25in]{geometry}
\usepackage[utf8]{inputenc}
\usepackage{amsmath}
\usepackage{amsthm}
\usepackage{amssymb, pifont}
\usepackage{mathrsfs}
\usepackage{enumitem}
\usepackage{fancyhdr}
\usepackage{hyperref}

\pagestyle{fancy}
\fancyhf{}
\rhead{MAT 457}
\lhead{Assignment 2}
\rfoot{Page \thepage}

\setlength\parindent{24pt}
\renewcommand\qedsymbol{$\blacksquare$}

\DeclareMathOperator{\E}{\mathcal{E}}
\DeclareMathOperator{\M}{\mathcal{M}}
\DeclareMathOperator{\F}{\mathbb{F}}
\DeclareMathOperator{\T}{\mathcal{T}}
\DeclareMathOperator{\V}{\mathcal{V}}
\DeclareMathOperator{\U}{\mathcal{U}}
\DeclareMathOperator{\Prt}{\mathbb{P}}
\DeclareMathOperator{\R}{\mathbb{R}}
\DeclareMathOperator{\N}{\mathbb{N}}
\DeclareMathOperator{\Z}{\mathbb{Z}}
\DeclareMathOperator{\Q}{\mathbb{Q}}
\DeclareMathOperator{\C}{\mathbb{C}}
\DeclareMathOperator{\ep}{\varepsilon}
\DeclareMathOperator{\identity}{\mathbf{0}}
\DeclareMathOperator{\card}{card}
\newcommand{\suchthat}{;\ifnum\currentgrouptype=16 \middle\fi|;}

\newtheorem{lemma}{Lemma}

\newcommand{\tr}{\mathrm{tr}}
\newcommand{\ra}{\rightarrow}
\newcommand{\lan}{\langle}
\newcommand{\ran}{\rangle}
\newcommand{\norm}[1]{\left\lVert#1\right\rVert}
\newcommand{\inn}[1]{\lan#1\ran}
\newcommand{\ol}{\overline}
\newcommand{\ci}{i}
\begin{document}
\noindent Q4a: We first will deal with the case that $m(E)< \infty$.We suppose for the sake of contradiction that there is some $\alpha\in (0,1)$ such that for all intervals $I$ such that $$m(E \cap I)< \alpha m(I)$$
From the measurability of $E$, for any $\varepsilon>0$, we can find some open cover of $E$ by intervals $\{I_n\}$ with $$m(E) \leq \sum_{n}m(I_n) \leq m(E) + \varepsilon$$ 
We mupltiply this inequality with $\alpha$ to get that $$\alpha m(E ) \leq \sum_n \alpha m(I_n) \leq \alpha m(E) + \alpha \varepsilon $$ 
Using our assumption, we get that $$\sum_n m(E \cap I_n) < \sum_{n} \alpha m(I_n)$$
Using subadditivity we get $$m(\bigcup_{n} E \cap I_n) \leq \sum_n m(E \cap I_n) $$ And since $E \subset \bigcup_{n} E \cap I_n$, we get $$m(E) < m(\bigcup_{n} E \cap I_n)$$
Combining these inequalities together we get that $$m(E) < \alpha m(E) + \alpha \varepsilon$$ 
We assume that $\alpha\in (0,1)$ and $\varepsilon$ is arbitrary, hence we obtain a contradiction. Now if $m(E) = \infty$, we can cover it with countable many $E_n$ which all have some $I^n$ corresponding to them, along with some $\alpha_n$. Take $I = \bigcup_n I_n$ and we are done. 
\newline \\ \noindent
Q4b: Suppose for the sake of contradiction that $E- E$ contains no interval centered about $0$. Then for all $\varepsilon>0$, there exists some $p\in (-\varepsilon, \varepsilon)$ so that $E+p \cap E = \emptyset$. Take $\alpha \in (\frac{3}{4}, 1)$. From 4a, there must exist some interval $I$ with $$m(E \cap I)\geq \alpha m(I)$$
Translation invariance of the lebesgue measure also guarantees that $$m(E\cap I + p) = m(E \cap I)$$
and our choice of $p$ guarantees that $$(E \cap I +p)\cap(E\cap I) = \emptyset$$
Hence using the property of the measure, we get that $$ m((E \cap I +p)\cup(E\cap I)) \geq 2\alpha m(I) $$
From properties of sets it is clear that we have $$(E\cap I+p) \cup (E\cap I) \subset (I+p)\cup I$$ and so $$m((E\cap I+p) \cup (E\cap I)) \leq m((I+p)\cup I )$$
Since $I$ is an interval, it follows that $m( I + p \cup I) = \mu(I) + |p|$. Since $|p|<\varepsilon$, along with the inequalities from above, and taking $\varepsilon<m(I)$ we get $$m(I) + \varepsilon \geq 2\alpha m(I)$$
If we take $\varepsilon = \frac{1}{2}m(I)$, 
We get that $$\frac{3}{2} m(I) \geq 2\alpha m(I) > \frac{3}{2} m(I)$$
A contradiction. We conclude that there exists an interval cenetered around $0$ contained in $E-E$. 
\end{document}