\documentclass[letterpaper]{article}
\usepackage[letterpaper,margin=1in,footskip=0.25in]{geometry}
\usepackage[utf8]{inputenc}
\usepackage{amsmath}
\usepackage{amsthm}
\usepackage{amssymb, pifont}
\usepackage{mathrsfs}
\usepackage{enumitem}
\usepackage{fancyhdr}
\usepackage{hyperref}

\pagestyle{fancy}
\fancyhf{}
\rhead{MAT 457}
\lhead{Assignment 2}
\rfoot{Page \thepage}

\setlength\parindent{24pt}
\renewcommand\qedsymbol{$\blacksquare$}

\DeclareMathOperator{\E}{\mathcal{E}}
\DeclareMathOperator{\M}{\mathcal{M}}
\DeclareMathOperator{\F}{\mathbb{F}}
\DeclareMathOperator{\T}{\mathcal{T}}
\DeclareMathOperator{\V}{\mathcal{V}}
\DeclareMathOperator{\U}{\mathcal{U}}
\DeclareMathOperator{\Prt}{\mathbb{P}}
\DeclareMathOperator{\R}{\mathbb{R}}
\DeclareMathOperator{\N}{\mathbb{N}}
\DeclareMathOperator{\Z}{\mathbb{Z}}
\DeclareMathOperator{\Q}{\mathbb{Q}}
\DeclareMathOperator{\C}{\mathbb{C}}
\DeclareMathOperator{\ep}{\varepsilon}
\DeclareMathOperator{\identity}{\mathbf{0}}
\DeclareMathOperator{\card}{card}
\newcommand{\suchthat}{;\ifnum\currentgrouptype=16 \middle\fi|;}

\newtheorem{lemma}{Lemma}

\newcommand{\tr}{\mathrm{tr}}
\newcommand{\ra}{\rightarrow}
\newcommand{\lan}{\langle}
\newcommand{\ran}{\rangle}
\newcommand{\norm}[1]{\left\lVert#1\right\rVert}
\newcommand{\inn}[1]{\lan#1\ran}
\newcommand{\ol}{\overline}
\newcommand{\ci}{i}
\begin{document}
\noindent 
Q2a: Let $\{B_n\}$ be an disjoint sequence of sets in $\mathcal{A}$. We define $$A_n = \bigcup_{i=1}^n B_i$$ We have that the $A_i's$ are an increasing sequence by construction. 
It is clear that each $A_i$ belongs to $\mathcal{A}$, since each is the union of sets in $\mathcal{A}$.  
It is given that $\mu$ is a finitely additive measure, so $$\mu(\bigcup_{j=1}^n B_j) = \sum_{j=1}^n \mu(B_j)$$ By the construction of $\{A_n\}$ it is also true that $$\mu(A_n) = \mu(\bigcup_{j=1}^n B_n)$$
Taking the limit we see that 
\begin{align*}
    \lim_{n \to \infty} \sum_{j=1}^n \mu(B_j)& = \lim_{n \to \infty} \mu(\bigcup_{j=1}^n B_j)
    \\ & = \lim_{n \to \infty } \mu(A_n) \tag{by construction of $A_n$}
    \\ & = \mu(\lim_{n \to \infty} \bigcup_{j=1}^n A_n) \tag{by given measure continuity}
    \\ & = \mu(\lim_{n \to \infty} \bigcup_{j=1}^n B_n) \tag{by definition of $B_n$}
\end{align*}
 We get the desired equality and conclude that $\mu$ is a premeasure. 
\newline \\ \noindent 
Q2b: Let $\{B_n\}$ be a sequence of disjoint sets in $\mathcal{A}$.  Since $\mu$ is finite, we have that downward measure continuity holds. We define a decreasing sequence of sets $\{A_n\}$ by $A_n = \bigcup_{i=n} B_i $. Note that this sequence is contained in $\mathcal{A}$ since it is the union of elements of $\mathcal{A}$. 
We note that by finiteness of $\mu$ we have that $$\sum_{i=1}^n \mu(B_i) = \mu(\bigcup_{i=1}^n B_i)$$
But also by the construction of $\{A_n\}$ we get that $$\sum_{i=1}^n \mu(B_i) = \sum_{i=1}^n \mu(A_i) - \mu(A_{i+1}) = \mu(A_1) - \mu(A_{n+1})$$
Taking the limit as $n \to \infty$, we get that 
\begin{align*}
    \sum_{i=1}^ \infty \mu(B_i) & =  \lim_{n \to \infty} \sum_{i=1}^n \mu(B_i)
    \\ & = \mu(A_1) - \lim_{n \to \infty} \mu(A_{n+1})
    \\ & = \mu(A_1) \tag{by assumption of measure of nested sets}
    \\ & = \mu(\bigcup_{i=1}^\infty B_i)
\end{align*}
We get the desired result and conclude that $\mu$ is a premeasure on $\mathcal{A}$
\end{document}