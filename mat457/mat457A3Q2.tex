\documentclass[letterpaper]{article}
\usepackage[letterpaper,margin=1in,footskip=0.25in]{geometry}
\usepackage[utf8]{inputenc}
\usepackage{amsmath}
\usepackage{amsthm}
\usepackage{amssymb, pifont}
\usepackage{mathrsfs}
\usepackage{enumitem}
\usepackage{fancyhdr}
\usepackage{hyperref}

\pagestyle{fancy}
\fancyhf{}
\rhead{MAT 457}
\lhead{Assignment 3}
\rfoot{Page \thepage}

\setlength\parindent{24pt}
\renewcommand\qedsymbol{$\blacksquare$}

\DeclareMathOperator{\E}{\mathcal{E}}
\DeclareMathOperator{\M}{\mathcal{M}}
\DeclareMathOperator{\F}{\mathbb{F}}
\DeclareMathOperator{\T}{\mathcal{T}}
\DeclareMathOperator{\V}{\mathcal{V}}
\DeclareMathOperator{\U}{\mathcal{U}}
\DeclareMathOperator{\Prt}{\mathbb{P}}
\DeclareMathOperator{\R}{\mathbb{R}}
\DeclareMathOperator{\N}{\mathbb{N}}
\DeclareMathOperator{\Z}{\mathbb{Z}}
\DeclareMathOperator{\Q}{\mathbb{Q}}
\DeclareMathOperator{\C}{\mathbb{C}}
\DeclareMathOperator{\ep}{\varepsilon}
\DeclareMathOperator{\identity}{\mathbf{0}}
\DeclareMathOperator{\card}{card}
\newcommand{\suchthat}{;\ifnum\currentgrouptype=16 \middle\fi|;}

\newtheorem{lemma}{Lemma}

\newcommand{\tr}{\mathrm{tr}}
\newcommand{\ra}{\rightarrow}
\newcommand{\lan}{\langle}
\newcommand{\ran}{\rangle}
\newcommand{\norm}[1]{\left\lVert#1\right\rVert}
\newcommand{\inn}[1]{\lan#1\ran}
\newcommand{\ol}{\overline}
\newcommand{\ci}{i}
\begin{document}
\noindent
Q2a: We first claim that $I(P)$ converges if $f\in L^1(\mu)$. First, note that given any partition we can without loss of generality assume that $y_0 = 0$, since we can always partition the interval containing 0 into two smaller intervals with lengths less than $\delta(P)$ by the fact that $\delta(P)< \infty$.
 We know that $$I(P) = \sum_{i = - \infty }^\infty y_i \mu(\{x: f(x) \in (y_i,y_{i+1}]  \}) = \sum_{i=0}^\infty y_i \mu(\{x: f(x) \in (y_i,y_{i+1}]  \}) + \sum_{i = - \infty}^0 y_i \mu(\{x: f(x) \in (y_i,y_{i+1}]  \})$$ 
It is sufficient to check that these sums do not diverge, so we can conclude that their sum will not diverge either. Suppose for the sake of contradiction that $$\sum_{i=0}^\infty y_i \mu(\{x: f(x) \in (y_i,y_{i+1}]  \}) = \infty$$ Then for any simple functions $\{ \phi_n \}$, we would have that $$\sum_{i=1}^n a_i \phi_i \leq \sum_{i=0}^\infty y_i \mu(\{x: f(x) \in (y_i,y_{i+1}]  \})$$
We can make this true for arbitrarily large $a_i$, hence we have that $\int f^+$ diverges. This contradicts the integrability of $f$. It remains to show that $\sum_{i = - \infty}^0 y_i \mu(\{x: f(x) \in (y_i,y_{i+1}]  \})$ converges. Suppose not. Then we have that $$|\sum_{i = - \infty}^0 y_i \mu(\{x: f(x) \in (y_i,y_{i+1}]  \}) |= \infty$$
For any simple functions $\{\phi_n\}$, we can take $a_i$ such that $(a_i- \delta(P)) \phi_i \leq f^{-}$, which we can do by integrability of $f^{-}$. Since $|\sum_{i = - \infty}^0 y_i \mu(\{x: f(x) \in (y_i,y_{i+1}]  \})|$ is unbounded, and must always be greater than or equal to $\sum_{i=1}^n a_i \phi_i$, we can choose $a_i$ arbitrarily large. Once again this contradicts the integrability of $f^-$. Hence we have that $I(P)$ is well defined. 
\newline \\ We now claim that $$\int f = \lim_{\delta(P) \to 0} I(P)$$
First we will show that $\lim_{\delta(P) \to 0} I(P) \geq \int f$. 
First note that for any interval, $(y_i,y_{i+1}]$ we have that $\delta(P) \geq y_{i+1} - y_i$, and for $x\in f^{-1}\{(y_i,y_{i+1})\}$ we get that $f(x) \leq y_{i+1} - (y_i - y_{i+1})$ and so we get that $y_i \geq f(x) - \delta(P)$. 
\begin{align*}
    I(P) & = \sum_{i= -\infty}^\infty \int_{f^{-1}\{ (y_i,y_{i+1})\} } y_i
    \\ & \geq \sum_{i = - \infty}^\infty \int_{f^{-1}\{ (y_i,y_{i+1})\} } f(x)- \delta(P)
    \\ & = \int_{\bigcup_{i = -\infty}^\infty (y_i ,y_{i+1}]} f(x) -\delta(P)
    \\ & = \int_{X} f d \mu - \int_{X} \delta(P)
    \\ & = \int_{X} f d \mu -  \mu(X)\delta(P)
\end{align*} Taking the limit as $\delta(P) \to 0$ we get that $\lim_{\delta(P) \to 0}I(P) \geq \int f$. 
We will now show that $\lim_{\delta(P) \to 0} I(P) \leq \int f$. Since $f(x) \geq y_i$, we have that $$\int f = \int f^+ - \int f^- \geq \sum_{i= 0}^\infty y_i \mu (\{x: f(x) \in (y_i,y_{i+1}] \} ) - \sum_{i = -\infty}^0 \mu (\{x: f(x) \in (y_i,y_{i+1}] \} )=I(P)$$
Taking the limit as $\delta(P)\to 0$ we get the desired inequality and conclude that $\int f = \lim_{\delta(P)\to 0} I(P)$
\newline \\ \noindent Q2b: We claim that $\int |f|$ is finite. We have that 
\begin{align*}
    \int |f| \leq \int_X \delta(P)d \mu = \mu(X)\delta(P) <\infty
\end{align*}
Since the value of the integral will be at most $\delta$
\end{document}