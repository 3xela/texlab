\documentclass[letterpaper]{article}
\usepackage[letterpaper,margin=1in,footskip=0.25in]{geometry}
\usepackage[utf8]{inputenc}
\usepackage{amsmath}
\usepackage{amsthm}
\usepackage{amssymb, pifont}
\usepackage{mathrsfs}
\usepackage{enumitem}
\usepackage{fancyhdr}
\usepackage{hyperref}

\pagestyle{fancy}
\fancyhf{}
\rhead{MAT 457}
\lhead{Assignment 5}
\rfoot{Page \thepage}

\setlength\parindent{24pt}
\renewcommand\qedsymbol{$\blacksquare$}

\DeclareMathOperator{\E}{\mathcal{E}}
\DeclareMathOperator{\M}{\mathcal{M}}
\DeclareMathOperator{\F}{\mathbb{F}}
\DeclareMathOperator{\T}{\mathcal{T}}
\DeclareMathOperator{\V}{\mathcal{V}}
\DeclareMathOperator{\U}{\mathcal{U}}
\DeclareMathOperator{\Prt}{\mathbb{P}}
\DeclareMathOperator{\R}{\mathbb{R}}
\DeclareMathOperator{\N}{\mathbb{N}}
\DeclareMathOperator{\Z}{\mathbb{Z}}
\DeclareMathOperator{\Q}{\mathbb{Q}}
\DeclareMathOperator{\C}{\mathbb{C}}
\DeclareMathOperator{\ep}{\varepsilon}
\DeclareMathOperator{\identity}{\mathbf{0}}
\DeclareMathOperator{\card}{card}
\newcommand{\suchthat}{;\ifnum\currentgrouptype=16 \middle\fi|;}

\newtheorem{lemma}{Lemma}

\newcommand{\tr}{\mathrm{tr}}
\newcommand{\ra}{\rightarrow}
\newcommand{\lan}{\langle}
\newcommand{\ran}{\rangle}
\newcommand{\norm}[1]{\left\lVert#1\right\rVert}
\newcommand{\inn}[1]{\lan#1\ran}
\newcommand{\ol}{\overline}
\newcommand{\ci}{i}
\begin{document}
\noindent
Q1: Since $\lim_{n \to \infty} \nu_n(E)$ exists and is finite for each set we have that $$\nu(\emptyset) = \lim_{n\to \infty} \nu_n(\emptyset) = \lim_{n\to \infty} 0 =0.$$ Now suppose that $\{E_m\}_{m=1}^\infty$ is a disjoint sequence of sets in $\mathcal{M}. $ 
We first claim that for any finite union of disjoint sets, $\nu(\bigcup_{m=1}^M E_m) = \sum_{m=1}^M \nu(E_m)$. We have that $$\nu(\bigcup_{m=1}^N E_m) = \lim_{n\to \infty} \nu_n(\bigcup_{m=1}^M E_m) = \lim_{n\to \infty} \sum_{m=1}^M \nu_n (E_m) = \sum_{m=1}^M \lim_{n\to \infty} \nu_n(E_m)  = \sum_{m=1}^M \nu(E_m),$$
using the fact that limits commute with finite sums. We now will prove that countable additivity holds. Let $\varepsilon>0$ and $M$ sufficiently large so that $$\nu(\bigcup_{m=1}^\infty E_m) = \sum_{m=1}^M \nu(E_m) + \nu(\bigcup_{m=M+1}^{\infty} E_m) \leq \sum_{m=1}^M \nu(E_m) + \ep $$
by setwise convergence. Furthermore, for all $n$, we have that $$\nu_n(\bigcup_{m=1}^M E_m) \leq \nu_n(\bigcup_{m=1}^\infty E_m) $$
Taking the limits and applying our first claim we have that $$\sum_{m=1}^M \nu(E_m) \leq \nu(\bigcup_{m=1}^\infty E_m)$$
Hence we have the inequality $$\sum_{m=1}^M \nu(E_m)\leq \nu(\bigcup_{m=1}^\infty E_m) \leq \sum_{m=1}^M \nu(E_m) + \nu(\bigcup_{m=M+1}^\infty E_m) \quad (1)$$
We claim that as $M\to \infty$, $\nu(\bigcup_{m=M+1}^\infty E_m)\to 0$. Since each $\nu_n$ is bounded, it is sufficient to show that $\nu_n(\bigcup_{m=M+1}^\infty E_m) \to 0$ as $M \to \infty$. Note that $(1)$ holds for $\nu_n$ as well since it is a measure, so we can take $\ep>0$ and choose $M$ large enough so that $\sum_{m=M+1}^\infty \nu_n(E_m) = \nu_n(\bigcup_{m=M+1}^\infty E_m) < \ep$/ By uniform absolute measure continuity, we have that there is some $\delta>0$
with $\mu(\bigcup_{m=M+1}^\infty E_m)  = \sum_{m=M+1}\mu(E_m) < \delta$. Since $\mu(X)<\infty$, we have that $\lim_{M\to \infty} \sum_{m=M+1}\mu(E_m)\to 0$, since it is convergent and so tails must converge. Therefore we have that $\sum_{m=M+1}^\infty \nu_n(E_m)\to 0$ as $M\to \infty$, and so $\nu(\bigcup_{m=M+1}^\infty E_m) \to 0 $ as $M\to \infty$. We apply the limit of $M\to \infty$ to $(1)$ and conclude that $$\sum_{m=1}^\infty \nu(E_m) = \nu(\bigcup_{m=1}^M E_m)$$
\end{document}