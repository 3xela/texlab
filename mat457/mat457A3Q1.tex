\documentclass[letterpaper]{article}
\usepackage[letterpaper,margin=1in,footskip=0.25in]{geometry}
\usepackage[utf8]{inputenc}
\usepackage{amsmath}
\usepackage{amsthm}
\usepackage{amssymb, pifont}
\usepackage{mathrsfs}
\usepackage{enumitem}
\usepackage{fancyhdr}
\usepackage{hyperref}

\pagestyle{fancy}
\fancyhf{}
\rhead{MAT 457}
\lhead{Assignment 3}
\rfoot{Page \thepage}

\setlength\parindent{24pt}
\renewcommand\qedsymbol{$\blacksquare$}

\DeclareMathOperator{\E}{\mathcal{E}}
\DeclareMathOperator{\M}{\mathcal{M}}
\DeclareMathOperator{\F}{\mathbb{F}}
\DeclareMathOperator{\T}{\mathcal{T}}
\DeclareMathOperator{\V}{\mathcal{V}}
\DeclareMathOperator{\U}{\mathcal{U}}
\DeclareMathOperator{\Prt}{\mathbb{P}}
\DeclareMathOperator{\R}{\mathbb{R}}
\DeclareMathOperator{\N}{\mathbb{N}}
\DeclareMathOperator{\Z}{\mathbb{Z}}
\DeclareMathOperator{\Q}{\mathbb{Q}}
\DeclareMathOperator{\C}{\mathbb{C}}
\DeclareMathOperator{\ep}{\varepsilon}
\DeclareMathOperator{\identity}{\mathbf{0}}
\DeclareMathOperator{\card}{card}
\newcommand{\suchthat}{;\ifnum\currentgrouptype=16 \middle\fi|;}

\newtheorem{lemma}{Lemma}

\newcommand{\tr}{\mathrm{tr}}
\newcommand{\ra}{\rightarrow}
\newcommand{\lan}{\langle}
\newcommand{\ran}{\rangle}
\newcommand{\norm}[1]{\left\lVert#1\right\rVert}
\newcommand{\inn}[1]{\lan#1\ran}
\newcommand{\ol}{\overline}
\newcommand{\ci}{i}
\begin{document}
\noindent
Q1: Let $\varepsilon>0$. Define $A_\varepsilon = \{x: |f(x)|<\varepsilon\}$. Choose $\delta = \frac{\ep}{1+ \ep}$. We cover $[0,1]$ with intervals of the form $(x - \frac{\delta}{2} , x + \frac{\delta}{2})$. By compactness, there exists finitely many $x_i$ corresponding to sets of the form $(x_i-\frac{\delta}{2}, x_i + \frac{\delta}{2})$ which cover $[0,1]$. 
It is sufficient to check that each set $(x_i - \frac{\delta}{2}, x_i + \frac{\delta}{2})\cap A_{\ep}$ is of measure 0 when mapped under $f$, since $$f(\bigcup_{i=1}^n A_{\ep} \cap (x_i - \frac{\delta}{2} , x_i + \frac{\delta}{2})) = \bigcup_{i=1}^n f(A_{\ep} \cap (x_i - \frac{\delta}{2} , x_i + \frac{\delta}{2}))$$
We have that $f$ is differentiable on $(x_i - \frac{\delta}{2} , x_i + \frac{\delta}{2}) \cap [0,1]$ and continuous on $[x_i - \frac{\delta}{2} , x_i + \frac{\delta}{2}] \cap [0,1]$. 
Hence we can apply the mean value theorem and get that $$\frac{|f(x_i + \frac{\delta}{2}) - f(x_i - \frac{\delta}{2}) |}{\delta} \leq \ep  $$
For convinience we let $a=f(x_i + \frac{\delta}{2})$ and $b = f(x_i - \frac{\delta}{2})$. Consider the interval $X = (\min(a,b) - \frac{\delta}{2}, \max(a,b) + \frac{\delta}{2})$. We have that $X \supset f(A_{\ep} \cap (x_i - \frac{\delta}{2}, x_i + \frac{\delta}{2}))$, since the image of connected intervals is a connected interval. We therefore get that $$m^\ast (f(A_{\ep} \cap (x_i - \frac{\delta}{2}, x_i + \frac{\delta}{2}))) \leq m^\ast(X) < \delta + \delta \cdot \ep = \ep$$
Therefore we have that $m^\ast(f(A_{\ep} \cap (x_i - \frac{\delta}{2}, x_i + \frac{\delta}{2})))$ is of measure 0, and hence $A_{\ep}$ is measure 0. We conclude that $f(\{x: |f^\prime(x)|=0 \})$ is measure 0. 
\end{document}