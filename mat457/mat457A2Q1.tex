\documentclass[letterpaper]{article}
\usepackage[letterpaper,margin=1in,footskip=0.25in]{geometry}
\usepackage[utf8]{inputenc}
\usepackage{amsmath}
\usepackage{amsthm}
\usepackage{amssymb, pifont}
\usepackage{mathrsfs}
\usepackage{enumitem}
\usepackage{fancyhdr}
\usepackage{hyperref}

\pagestyle{fancy}
\fancyhf{}
\rhead{MAT 457}
\lhead{Assignment 2}
\rfoot{Page \thepage}

\setlength\parindent{24pt}
\renewcommand\qedsymbol{$\blacksquare$}

\DeclareMathOperator{\E}{\mathcal{E}}
\DeclareMathOperator{\M}{\mathcal{M}}
\DeclareMathOperator{\F}{\mathbb{F}}
\DeclareMathOperator{\T}{\mathcal{T}}
\DeclareMathOperator{\V}{\mathcal{V}}
\DeclareMathOperator{\U}{\mathcal{U}}
\DeclareMathOperator{\Prt}{\mathbb{P}}
\DeclareMathOperator{\R}{\mathbb{R}}
\DeclareMathOperator{\N}{\mathbb{N}}
\DeclareMathOperator{\Z}{\mathbb{Z}}
\DeclareMathOperator{\Q}{\mathbb{Q}}
\DeclareMathOperator{\C}{\mathbb{C}}
\DeclareMathOperator{\ep}{\varepsilon}
\DeclareMathOperator{\identity}{\mathbf{0}}
\DeclareMathOperator{\card}{card}
\newcommand{\suchthat}{;\ifnum\currentgrouptype=16 \middle\fi|;}

\newtheorem{lemma}{Lemma}

\newcommand{\tr}{\mathrm{tr}}
\newcommand{\ra}{\rightarrow}
\newcommand{\lan}{\langle}
\newcommand{\ran}{\rangle}
\newcommand{\norm}[1]{\left\lVert#1\right\rVert}
\newcommand{\inn}[1]{\lan#1\ran}
\newcommand{\ol}{\overline}
\newcommand{\ci}{i}
\begin{document}
\noindent
Q1: Suppose that $\mu$ is a measure on $\mathcal{P}(\R)$, which only takes on values of $0$ or $1$. We first check the value of $\mu(\R)$. If it is the case that it is 0, $\mu$ must clearly be $0$ on any subset of $\R$. If it is 1, then we proceed. We write $\R = \bigcup_{n \in \Z} [n, n+1]$. Since $\mu$ can only take on values of $0$ and $1$, we have that $\mu([n,n+1])= 0$ for either all but one $n$ or some $n,n+1$. If it is nonzero for some consecutive intervals,$[n,n+1], [n+1,n+2]$, then we must have that $$\mu([n,n+1] \cup [n+1,n+2])= \mu([n,n+2])=1$$ This together with the fact that each interval has measure 1 implies that $$\mu([n,n+1] \cap [n+1,n+2]) = \mu(\{n+1\})=1 \quad(*)$$ Since if the intersection were measure 0, we would have two disjoint sets with each measure 1. Hence this case leads to the conclusion that $\mu = \delta_{x}$ for $x = n+1$. Now consider the case that only one $[n,n+1]$ has a measure of 1. We call $E_1 = [n,n+1]$. We now define a sequence of sets $\{E_i\}$ in the following way. If $E_i = [a,b]$, $$E_{i+1} =\begin{cases}
    [a, a + \frac{b-a}{2}], & \text{if $\mu([a, a + \frac{b-a}{2}])=1$}\\
            [a+ \frac{b-a}{2}, b], & \text{ if $\mu([a+ \frac{b-a}{2}, b])$=1}
\end{cases} $$
If at some point, we have that $\mu([a+ \frac{b-a}{2}, b]) = \mu([a, a + \frac{b-a}{2}])=1$ we terminate the sequence and using the same process as (*), we can conclude that $\mu = \delta_{a+ \frac{b-a}{2}}$. Note that it will never be the case that both are of measure 0, since their union will be $E_i$, which defined recursively implies that each $E_i$ is not measure 0. Furthermore, note that each $E_i \supset E_{i+1}$. Hence this is a decreasing sequence. We consider their intersection $$A = \bigcap_i E_i$$ Note that downward measure continuity implies that $\mu(A) =1$. Furthermore, from toplogy we know that the countable intersection of nested cauchy closed intervals in $\R$ is a singleton. Hence we have that $A = \{x\}$ for some point $x$. Therefore our measure $\mu$ is $0$ on every set except those containing the point $x$. Namely, $\mu = \delta_x$.  
\end{document}