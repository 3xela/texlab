\documentclass[letterpaper]{article}
\usepackage[letterpaper,margin=1in,footskip=0.25in]{geometry}
\usepackage[utf8]{inputenc}
\usepackage{amsmath}
\usepackage{amsthm}
\usepackage{amssymb, pifont}
\usepackage{mathrsfs}
\usepackage{enumitem}
\usepackage{fancyhdr}
\usepackage{hyperref}

\pagestyle{fancy}
\fancyhf{}
\rhead{MAT 457}
\lhead{Assignment 4}
\rfoot{Page \thepage}

\setlength\parindent{24pt}
\renewcommand\qedsymbol{$\blacksquare$}

\DeclareMathOperator{\E}{\mathcal{E}}
\DeclareMathOperator{\M}{\mathcal{M}}
\DeclareMathOperator{\F}{\mathbb{F}}
\DeclareMathOperator{\T}{\mathcal{T}}
\DeclareMathOperator{\V}{\mathcal{V}}
\DeclareMathOperator{\U}{\mathcal{U}}
\DeclareMathOperator{\Prt}{\mathbb{P}}
\DeclareMathOperator{\R}{\mathbb{R}}
\DeclareMathOperator{\N}{\mathbb{N}}
\DeclareMathOperator{\Z}{\mathbb{Z}}
\DeclareMathOperator{\Q}{\mathbb{Q}}
\DeclareMathOperator{\C}{\mathbb{C}}
\DeclareMathOperator{\ep}{\varepsilon}
\DeclareMathOperator{\identity}{\mathbf{0}}
\DeclareMathOperator{\card}{card}
\newcommand{\suchthat}{;\ifnum\currentgrouptype=16 \middle\fi|;}

\newtheorem{lemma}{Lemma}

\newcommand{\tr}{\mathrm{tr}}
\newcommand{\ra}{\rightarrow}
\newcommand{\lan}{\langle}
\newcommand{\ran}{\rangle}
\newcommand{\norm}[1]{\left\lVert#1\right\rVert}
\newcommand{\inn}[1]{\lan#1\ran}
\newcommand{\ol}{\overline}
\newcommand{\ci}{i}
\begin{document}
\noindent
Q1: First note that $|f_n-f|\leq |f_n| + |f| \leq 2|g|$. If we define $E_n(k) = \bigcup_{m=n}^\infty \{x: |f_m-f|\geq \frac{1}{k}\}$, and \newline $A(k) = \{x: 2|g| \geq \frac{1}{k}\}$. We see that $A(k)\supset E_n(K)$. We have that for fixed $k$, $E_n(k)$ is a decreasing sequence. We wish to show that $\mu(E_n(k))$ is of finite measure. Applying Markovs inequality to $A(k)$, we get that $$\mu(A(k)) \leq k \int_X 2|g|$$
Since $g\in L^1$ we have that $\mu(A(k)) <\infty$. Therefore we get that $\mu(E_1(k)) < \infty$. Since $\bigcap_{n=1}^\infty E_{n}(k) = \emptyset$, measure continuity implies that $\mu(E_n(k)) \to 0$ as $n\to \infty$. Given $\ep >0$, and $k\in \N$, choose $n_k$ sufficiently large so that $\mu(E_{n_k}(k))< \ep 2^{-k}$. Let $E = \bigcap_{k=1}^\infty E_{n_k}(k)$. We get that $\mu(E) < \ep$ and $|f_n(x)-f(x)|<\frac{1}{k}$ for $n>n_k$ and $x \notin E$. Therefore $f_n \to f$ uniformly on $E^c$.  
\end{document}