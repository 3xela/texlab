\documentclass[letterpaper]{article}
\usepackage[letterpaper,margin=1in,footskip=0.25in]{geometry}
\usepackage[utf8]{inputenc}
\usepackage{amsmath}
\usepackage{amsthm}
\usepackage{amssymb, pifont}
\usepackage{mathrsfs}
\usepackage{enumitem}
\usepackage{fancyhdr}
\usepackage{hyperref}

\pagestyle{fancy}
\fancyhf{}
\rhead{MAT 457}
\lhead{Assignment 5}
\rfoot{Page \thepage}

\setlength\parindent{24pt}
\renewcommand\qedsymbol{$\blacksquare$}

\DeclareMathOperator{\E}{\mathcal{E}}
\DeclareMathOperator{\M}{\mathcal{M}}
\DeclareMathOperator{\F}{\mathbb{F}}
\DeclareMathOperator{\T}{\mathcal{T}}
\DeclareMathOperator{\V}{\mathcal{V}}
\DeclareMathOperator{\U}{\mathcal{U}}
\DeclareMathOperator{\Prt}{\mathbb{P}}
\DeclareMathOperator{\R}{\mathbb{R}}
\DeclareMathOperator{\N}{\mathbb{N}}
\DeclareMathOperator{\Z}{\mathbb{Z}}
\DeclareMathOperator{\Q}{\mathbb{Q}}
\DeclareMathOperator{\C}{\mathbb{C}}
\DeclareMathOperator{\ep}{\varepsilon}
\DeclareMathOperator{\identity}{\mathbf{0}}
\DeclareMathOperator{\card}{card}
\newcommand{\suchthat}{;\ifnum\currentgrouptype=16 \middle\fi|;}

\newtheorem{lemma}{Lemma}

\newcommand{\tr}{\mathrm{tr}}
\newcommand{\ra}{\rightarrow}
\newcommand{\lan}{\langle}
\newcommand{\ran}{\rangle}
\newcommand{\norm}[1]{\left\lVert#1\right\rVert}
\newcommand{\inn}[1]{\lan#1\ran}
\newcommand{\ol}{\overline}
\newcommand{\ci}{i}
\begin{document}
\noindent
Q3: We first claim that $\mu_1 \leq \mu_2 \leq \mu_3$. Let $E\in \mathcal{M}$.Suppose $E_1 \dots E_n$ are disjoint with $\bigcup_{n=1}^N E_n = E. $ Then if $\{F_i\}_{i=1}^\infty$ are a disjoint sequence with $\bigcup_{i=1}^\infty F_i = E$ we have that $$\sum_{n=1}^N |\nu(E_n)| \leq \sum_{n=1}^\infty |\nu(F_n)|$$
Taking supremums yields that $\mu_1(E) \leq \mu_2(E) $. We now claim that $\mu_2(E)\leq \mu_3(E).$ Take any $E_1, E_2, \dots$ pairwise disjoint with $\bigcup_{n=1}^\infty E_n = E.$ 
We have that $$\sum_{n=1}^\infty |\nu(E_n)| = \sum_{n=1}^\infty |\int_{E_n} d \nu | = |\int_{E} d\nu| \leq \mu_{3}(E)$$
Hence we have $\mu_2(E)\leq \mu_3(E)$. Thus we have the chain of inequalities $\mu_1\leq \mu_2 \leq \mu_3$. We now claim that $\mu_3(E)= |\nu|(E).$ 
By prop 3.13 from Folland, we have that $$\mu_3(E) \leq \sup \Big\{\int_E |f|d|\nu| : |f|\leq 1, f \text{ measurable} \Big\} = \int_{X} \chi_E d|\nu| = |\nu|(E), $$
by approximating $|f|$ with simple functions.
We can also check that
\begin{align*}
    |\nu|(E) & = \Big|\int_E 1 d|\nu| \Big|
    \\ & = \Big| \int_E \big| \frac{d \nu}{d |\nu|} \big| d |\nu|  \Big| \tag{by prop 3.13}
    \\ & = \Big| \int_E \ol{\frac{d \nu}{d|\nu|}} \cdot \frac{d\nu}{d|\nu|} d|\nu| \Big| \tag{by definition of modulus}
    \\ & = \Big| \int_E\ol{\frac{d\nu}{d |\nu|}} d\nu \Big|
    \\ & \leq \mu_3(E) \tag{since $|\ol{\frac{d\nu}{d|\nu|}}| = |\frac{d\nu}{d|\nu|}|=1$ a.e. by prop 3.13}
\end{align*}
Hence $|\nu|(E) = \mu_3(E)$. Finally we will show that $\mu_3 \leq \mu_1$, which will prove the result. Let $E\in \mathcal{M}$, $f$ be any measurable function with $|f|\leq 1$, and let $\{E_n\}_{n=1}^N$ be any partition of $E$. 
Choose a simple function $\phi = \sum_{n=1}^N a_n \chi_{E_n}$ with $|a_n|\leq 1$ so that $\int_E |\phi - f|< \ep$. It is sufficient to show that the result holds for $\phi$. 
We have that $$\Big|\int_E \phi d\nu \Big| = \Big| \sum_{n=1}^N \int_{X} a_n \chi_{E_n} d\nu \Big| \leq \sum_{n=1}^N |a_n| \Big| \int_{E_n} 1 d\nu \Big| = \sum_{n=1}^N |a_n| |\nu(E_n)| \leq \sum_{n=1}^N |\nu(E_n)|$$
Taking supremums yields the desired inequality. Hence we have that $$|\nu|\leq \mu_1 \leq \nu_2 \leq |\nu|$$ and conclude that $$|\nu| = \mu_1 = \mu_2 $$
\end{document}
