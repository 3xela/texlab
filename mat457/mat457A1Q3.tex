\documentclass[letterpaper]{article}
\usepackage[letterpaper,margin=1in,footskip=0.25in]{geometry}
\usepackage[utf8]{inputenc}
\usepackage{amsmath}
\usepackage{amsthm}
\usepackage{amssymb, pifont}
\usepackage{mathrsfs}
\usepackage{enumitem}
\usepackage{fancyhdr}
\usepackage{hyperref}

\pagestyle{fancy}
\fancyhf{}
\rhead{MAT 457}
\lhead{Assignment 1}
\rfoot{Page \thepage}

\setlength\parindent{24pt}
\renewcommand\qedsymbol{$\blacksquare$}

\DeclareMathOperator{\E}{\mathcal{E}}
\DeclareMathOperator{\M}{\mathcal{M}}
\DeclareMathOperator{\F}{\mathbb{F}}
\DeclareMathOperator{\T}{\mathcal{T}}
\DeclareMathOperator{\V}{\mathcal{V}}
\DeclareMathOperator{\U}{\mathcal{U}}
\DeclareMathOperator{\Prt}{\mathbb{P}}
\DeclareMathOperator{\R}{\mathbb{R}}
\DeclareMathOperator{\N}{\mathbb{N}}
\DeclareMathOperator{\Z}{\mathbb{Z}}
\DeclareMathOperator{\Q}{\mathbb{Q}}
\DeclareMathOperator{\C}{\mathbb{C}}
\DeclareMathOperator{\ep}{\varepsilon}
\DeclareMathOperator{\identity}{\mathbf{0}}
\DeclareMathOperator{\card}{card}
\newcommand{\suchthat}{;\ifnum\currentgrouptype=16 \middle\fi|;}

\newtheorem{lemma}{Lemma}

\newcommand{\tr}{\mathrm{tr}}
\newcommand{\ra}{\rightarrow}
\newcommand{\lan}{\langle}
\newcommand{\ran}{\rangle}
\newcommand{\norm}[1]{\left\lVert#1\right\rVert}
\newcommand{\inn}[1]{\lan#1\ran}
\newcommand{\ol}{\overline}
\newcommand{\ci}{i}
\begin{document}
\noindent
Q3a: If we let $A_k = \bigcap _{n\geq k} E_n$ it is clear that $A_1 \subset A_2 \dots$. Hence by measure continuity, we get that 
$$\mu(\bigcup_{k\geq 1} A_k) = \lim_{k\to \infty} \mu(A_k) = \lim_{k \to \infty} \mu( \bigcap_{n \geq k} E_n)$$
However from the properties of the measure, namely $A \subset B$ implies that $\mu(A) \leq \mu(B)$, we can deduce that for any $k$, $$\mu(\bigcap_{n \geq k} E_n) \leq \inf_{n\geq k} \mu(E_n)$$
Since measure continuty holds, by applying limits we see that $$\lim \inf_n E_n = \mu (\bigcap_{k\geq 1} \bigcup_{n\geq k} E_n) = \lim_{k \to \infty} \mu(\bigcap_{n\geq k} E_n) \leq \lim_{k \to \infty} \inf_{n\geq k} \mu(E_n) = \lim \inf_{n} \mu(E_n) \qed$$
\newline \\ 
\noindent 
Q3b: 
If we define $A_k = \bigcup_{n\geq k}E_k$ we see that $A_1 \supset A_2 \dots$, and $\mu(A_1)< \infty$ as given. Hence we can apply measure continuity to get that 
$$\mu(\bigcap_{k \geq 1} A_k) = \lim_{k \to \infty} \mu(A_k) = \lim_{k \to \infty} \mu(\bigcup_{n\geq k}E_n)$$
Similarly to 3a, we can reason that $$\mu(\bigcup_{n\geq k} E_n) \geq \sup_{n\geq k}\mu(E_n)$$
Since measure continuity holds we can apply limits and conclude that 
$$\mu(\lim \sup_n E_n) = \mu(\bigcap_{k\geq 1} \bigcup_{n\geq k} E_n) = \lim_{k\to \infty} \mu(\bigcup_{n\geq k} E_n) \geq \lim_{k\to \infty}\sup_{n \geq k} \mu(E_n) = \lim \sup_n \mu(E_n) \qed$$
If we did not have the hypothesis that $\mu(\bigcup_{n=1}^{\infty}E_n)$ is finite this result would not hold. Consider the collection $\{E_n\}$ with $E_n = (-n, n]$. We see that $\mu(\lim \sup_n E_n) = \mu(E_1) =2$ but $\lim \sup_n \mu(E_n) = \infty$. It is certainly false that $2\geq \infty$. 
\end{document}