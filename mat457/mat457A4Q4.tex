\documentclass[letterpaper]{article}
\usepackage[letterpaper,margin=1in,footskip=0.25in]{geometry}
\usepackage[utf8]{inputenc}
\usepackage{amsmath}
\usepackage{amsthm}
\usepackage{amssymb, pifont}
\usepackage{mathrsfs}
\usepackage{enumitem}
\usepackage{fancyhdr}
\usepackage{hyperref}

\pagestyle{fancy}
\fancyhf{}
\rhead{MAT 457}
\lhead{Assignment 4}
\rfoot{Page \thepage}

\setlength\parindent{24pt}
\renewcommand\qedsymbol{$\blacksquare$}

\DeclareMathOperator{\E}{\mathcal{E}}
\DeclareMathOperator{\M}{\mathcal{M}}
\DeclareMathOperator{\F}{\mathbb{F}}
\DeclareMathOperator{\T}{\mathcal{T}}
\DeclareMathOperator{\V}{\mathcal{V}}
\DeclareMathOperator{\U}{\mathcal{U}}
\DeclareMathOperator{\Prt}{\mathbb{P}}
\DeclareMathOperator{\R}{\mathbb{R}}
\DeclareMathOperator{\N}{\mathbb{N}}
\DeclareMathOperator{\Z}{\mathbb{Z}}
\DeclareMathOperator{\Q}{\mathbb{Q}}
\DeclareMathOperator{\C}{\mathbb{C}}
\DeclareMathOperator{\ep}{\varepsilon}
\DeclareMathOperator{\identity}{\mathbf{0}}
\DeclareMathOperator{\card}{card}
\newcommand{\suchthat}{;\ifnum\currentgrouptype=16 \middle\fi|;}

\newtheorem{lemma}{Lemma}

\newcommand{\tr}{\mathrm{tr}}
\newcommand{\ra}{\rightarrow}
\newcommand{\lan}{\langle}
\newcommand{\ran}{\rangle}
\newcommand{\norm}[1]{\left\lVert#1\right\rVert}
\newcommand{\inn}[1]{\lan#1\ran}
\newcommand{\ol}{\overline}
\newcommand{\ci}{i}
\begin{document}
\noindent
Q4: Suppose that $f_n \to f$ in $L^1$. Let $E_k = \{x: |f_n-f| \geq \frac{1}{k}\}$. By markovs inequality, we have that $$\mu(E_k) \leq k \int_X |f_n-f|$$
As $n$ get sufficiently large the integral goes to $0$ and we conclude that $\lim_{n\to \infty} \mu(E_k)=0$. Hence $f_n \to f$ in measure. 
We now claim that $\{f_n\}$ is uniformly absolutely continuous. Let $\ep>0$. We take $N$ sufficiently large so that $\int_X |f_n-f| < \frac{\ep}{2}$ for $n \geq N$. By a result from class, there must exist some $\delta^\ast >0$ so that for all $E$ with $\mu(E) < \delta^\ast$, $\int_E |f| < \frac{\ep}{2}$. We therefore have that $$\int_{E} |f_n| \leq \int_{E} |f_n-f| + |f| < \frac{\ep}{2} + \frac{\ep}{2} = \ep$$ Now if $1 \leq n < N$, by the result from class for the same $\ep$ there exists $\delta_i$ so that $$\int_{E} |f_n| \leq \ep $$ for $\mu(E) < \delta_i$. We take $\delta$ to be the minimum of the $\delta_i$'s and $\delta^\ast$. Hence for all $E$, $\mu(E)<\delta$ we have that $\int_E |f_n| < \ep$ 
Now suppose the converse. Let $\ep>0$. By convergence in measure choose $n$ sufficiently large so that if $E = \{x: |f_n-f| > \ep \}$ then $\mu(E) < \frac{\ep}{2}$. Then we have that 
\begin{align*}
    \int_X |f_n - f| & = \int_E |f_n-f | + \int_{E^c} |f_n-f|
    \\ & \leq \int_E |f| + \int_E |f| + \int_{E^c} |f_n-f|
    \\ & \leq \frac{\ep}{2} + \frac{\ep}{2} + \int_{E^c} \ep
    \\ & < \ep (1+ \mu(X))
\end{align*}
Since $\ep$ is arbitrary, we conclude that $\int |f_n -f| \to 0$ and so $f_n \to f$. 
\end{document}