\documentclass[letterpaper]{article}
\usepackage[letterpaper,margin=1in,footskip=0.25in]{geometry}
\usepackage[utf8]{inputenc}
\usepackage{amsmath}
\usepackage{amsthm}
\usepackage{amssymb, pifont}
\usepackage{mathrsfs}
\usepackage{enumitem}
\usepackage{fancyhdr}
\usepackage{hyperref}

\pagestyle{fancy}
\fancyhf{}
\rhead{MAT 457}
\lhead{Assignment 8}
\rfoot{Page \thepage}

\setlength\parindent{24pt}
\renewcommand\qedsymbol{$\blacksquare$}

\DeclareMathOperator{\E}{\mathcal{E}}
\DeclareMathOperator{\M}{\mathcal{M}}
\DeclareMathOperator{\F}{\mathbb{F}}
\DeclareMathOperator{\T}{\mathcal{T}}
\DeclareMathOperator{\V}{\mathcal{V}}
\DeclareMathOperator{\U}{\mathcal{U}}
\DeclareMathOperator{\Prt}{\mathbb{P}}
\DeclareMathOperator{\R}{\mathbb{R}}
\DeclareMathOperator{\N}{\mathbb{N}}
\DeclareMathOperator{\Z}{\mathbb{Z}}
\DeclareMathOperator{\Q}{\mathbb{Q}}
\DeclareMathOperator{\C}{\mathbb{C}}
\DeclareMathOperator{\ep}{\varepsilon}
\DeclareMathOperator{\identity}{\mathbf{0}}
\DeclareMathOperator{\card}{card}
\newcommand{\suchthat}{;\ifnum\currentgrouptype=16 \middle\fi|;}

\newtheorem{lemma}{Lemma}

\newcommand{\tr}{\mathrm{tr}}
\newcommand{\ra}{\rightarrow}
\newcommand{\lan}{\langle}
\newcommand{\ran}{\rangle}
\newcommand{\norm}[1]{\left\lVert#1\right\rVert}
\newcommand{\inn}[1]{\lan#1\ran}
\newcommand{\ol}{\overline}
\newcommand{\ci}{i}
\newcommand{\X}{\mathfrak{X}}
\begin{document}
Q4: Suppose $\norm{f_n-f}_p \to 0$ as $n\to \infty$. Since $\norm{\cdot}_p$ is a norm, we have that the reverse triangle inequality holds, $i.e.$ we have $$|\norm{f_n}_p - \norm{f}_p|\leq \norm{f_n-f}_p.$$ Thus as $n\to \infty$,$|\norm{f_n}_p - \norm{f}_p|\to 0$ or equivalently $\norm{f_n}_p \to \norm{f}$. Conversely suppose that $\norm{f_n}_p \to \norm{f}_p $. We first claim that for any two functions $f,g$ the following inequality holds: $$2^{-p}|f+g|^p \leq |f|^p + g^p.$$ Indeed we have that $$|f+g|^p \leq 2^p \max{|f|^p,|g|^p} \leq 2^{p}(|f|^p + |g|^p).$$ Dividing by $2^{p}$ gives the desired inequality. Therefore we can write that $$2^{-p}|f_n - f|^p \leq |f|^p + |f_n|^p.$$ We now claim the Generalized Dominated Convergence Theorem. Given $f_n,g_n$ with $f_n \to f ,g_n \to g$ a.e. , $|f_n|\leq |g_n|$, and $\int {g_n} \to \int g$ then $\int f_n \to \int f$. Observe that by Fatou's lemma we have that $$ \int g- \int f \leq \lim \inf \int g_n - f_n = \lim_{n} \int g_n - \lim \sup \int f_n = \int g - \lim \sup \int f_n.$$
Similarly, we have that $$\int g + \int f \leq \lim \inf \int g_n + f_n = \lim_{n \to \infty} \int g_n + \lim \inf \int f_n  = \int g + \lim \inf \int f_n.$$ Thus we have that $$\lim \sup \int f_n \leq \int f \leq \lim \inf \int f_n.$$ And we conclude that $$\int f = \lim_{n\to \infty} \int f_n.$$ Since $\norm{f_n}_p \to \norm{f_n},$ and $2^{-p} |f_n - f|^p \leq |f|^p + |f_n|^p$, the generalized DCT tells us that $$\lim_{n\to \infty} \int 2^{-p}|f_n - f|^p d\mu = \int 2^{-p} \Big(\lim_{n\to \infty}  |f_n - f|\Big)^p = 0,$$ and thus $|f_n-f|\to 0$ as $n\to \infty$. 
\end{document}