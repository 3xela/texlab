\documentclass[letterpaper]{article}
\usepackage[letterpaper,margin=1in,footskip=0.25in]{geometry}
\usepackage[utf8]{inputenc}
\usepackage{amsmath}
\usepackage{amsthm}
\usepackage{amssymb, pifont}
\usepackage{mathrsfs}
\usepackage{enumitem}
\usepackage{fancyhdr}
\usepackage{hyperref}

\pagestyle{fancy}
\fancyhf{}
\rhead{MAT 457}
\lhead{Assignment 5}
\rfoot{Page \thepage}

\setlength\parindent{24pt}
\renewcommand\qedsymbol{$\blacksquare$}

\DeclareMathOperator{\E}{\mathcal{E}}
\DeclareMathOperator{\M}{\mathcal{M}}
\DeclareMathOperator{\F}{\mathbb{F}}
\DeclareMathOperator{\T}{\mathcal{T}}
\DeclareMathOperator{\V}{\mathcal{V}}
\DeclareMathOperator{\U}{\mathcal{U}}
\DeclareMathOperator{\Prt}{\mathbb{P}}
\DeclareMathOperator{\R}{\mathbb{R}}
\DeclareMathOperator{\N}{\mathbb{N}}
\DeclareMathOperator{\Z}{\mathbb{Z}}
\DeclareMathOperator{\Q}{\mathbb{Q}}
\DeclareMathOperator{\C}{\mathbb{C}}
\DeclareMathOperator{\ep}{\varepsilon}
\DeclareMathOperator{\identity}{\mathbf{0}}
\DeclareMathOperator{\card}{card}
\newcommand{\suchthat}{;\ifnum\currentgrouptype=16 \middle\fi|;}

\newtheorem{lemma}{Lemma}

\newcommand{\tr}{\mathrm{tr}}
\newcommand{\ra}{\rightarrow}
\newcommand{\lan}{\langle}
\newcommand{\ran}{\rangle}
\newcommand{\norm}[1]{\left\lVert#1\right\rVert}
\newcommand{\inn}[1]{\lan#1\ran}
\newcommand{\ol}{\overline}
\newcommand{\ci}{i}
\begin{document}
\noindent
Q2: We will proceed by the contrapositive. Suppose that $\mu \not \perp \nu$. Then there exists some maximal sets $E,F$ such that $\mu |_{E^c} \equiv 0$ and $\nu|_F^c \equiv 0$ but $E\cap F \neq \emptyset$ and $E\cap F$ is not measure 0. We compute that 
\begin{align*} 
\norm{\alpha \mu - (1 - \alpha) \nu}  &= \int_{X} \alpha \mu + (1-\alpha) \nu
\\ & = \int_E \alpha \mu + (1-\alpha) \nu + \int_F \alpha \mu + (1-\alpha) \nu - \int_{E\cap F} \alpha \mu + (1-\alpha) \nu
\\ & = \alpha +(1-\alpha) - \int_{E \cap F}\alpha \mu + (1-\alpha) \nu
\\ & = 1 - \int_{E \cap F}\alpha \mu + (1-\alpha) \nu
\end{align*}
Now we verify that $\int_{E \cap F}\alpha \mu + (1-\alpha) \nu >0$. Note that by definition of $E,F$ and since $\mu,\nu$ are both positive on their intersection, we have that $\alpha \mu +(1- \alpha \nu) >0$ on a nonempty set. 
Hence we have that $\int_{E \cap F}\alpha \mu + (1-\alpha) \nu >0$. Thus we conclude that if $ \mu \not \perp \nu$, then $\norm{\alpha \mu -(1 - \alpha) \nu} \neq 1$. 
\end{document}