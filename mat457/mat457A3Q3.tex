\documentclass[letterpaper]{article}
\usepackage[letterpaper,margin=1in,footskip=0.25in]{geometry}
\usepackage[utf8]{inputenc}
\usepackage{amsmath}
\usepackage{amsthm}
\usepackage{amssymb, pifont}
\usepackage{mathrsfs}
\usepackage{enumitem}
\usepackage{fancyhdr}
\usepackage{hyperref}

\pagestyle{fancy}
\fancyhf{}
\rhead{MAT 457}
\lhead{Assignment 3}
\rfoot{Page \thepage}

\setlength\parindent{24pt}
\renewcommand\qedsymbol{$\blacksquare$}

\DeclareMathOperator{\E}{\mathcal{E}}
\DeclareMathOperator{\M}{\mathcal{M}}
\DeclareMathOperator{\F}{\mathbb{F}}
\DeclareMathOperator{\T}{\mathcal{T}}
\DeclareMathOperator{\V}{\mathcal{V}}
\DeclareMathOperator{\U}{\mathcal{U}}
\DeclareMathOperator{\Prt}{\mathbb{P}}
\DeclareMathOperator{\R}{\mathbb{R}}
\DeclareMathOperator{\N}{\mathbb{N}}
\DeclareMathOperator{\Z}{\mathbb{Z}}
\DeclareMathOperator{\Q}{\mathbb{Q}}
\DeclareMathOperator{\C}{\mathbb{C}}
\DeclareMathOperator{\ep}{\varepsilon}
\DeclareMathOperator{\identity}{\mathbf{0}}
\DeclareMathOperator{\card}{card}
\newcommand{\suchthat}{;\ifnum\currentgrouptype=16 \middle\fi|;}

\newtheorem{lemma}{Lemma}

\newcommand{\tr}{\mathrm{tr}}
\newcommand{\ra}{\rightarrow}
\newcommand{\lan}{\langle}
\newcommand{\ran}{\rangle}
\newcommand{\norm}[1]{\left\lVert#1\right\rVert}
\newcommand{\inn}[1]{\lan#1\ran}
\newcommand{\ol}{\overline}
\newcommand{\ci}{i}
\begin{document}
\noindent
Q3: By Egorov's Theorem, there must exist some set $E_k$ such that $\mu(X \setminus E_k)< \frac{1}{k}$ and $f_n \to 0$ uniformly on $E_k$. 
We define the set $$E = \bigcup_{k=1}^\infty E_k$$
Since $X \setminus E \subset X\setminus E_k$ for all $k$, we have that $\mu(X \setminus E) =0$. Since the $f_n$'s converge to 0 on $E$, for any $m\in \N$, there is some $n_m \in \N$ that satisfies $$|f_n(x)|<\frac{1}{2^m}$$ for all $x\in X$ and $n\geq n_m$. We can assume that each $n_m \leq n_{m+1}$ since we can choose the minimum such $n_m$ satisfying the above. We define $C_n = m$ for $n_m \leq n < n_{m+1}$. We have that $C_n$ will approach $\infty$, since if it were not then it would be constant at some point and $|f_n(x)|$ would not approach 0. Now if we take $\varepsilon>0$ and chose a $m$ sufficiently large so that $m2^{-m}< \varepsilon$. Take $n \geq n_m$, and $l\geq m$ such that $n_{l} \leq n \leq n_{l+1}$. Therefore if $x\in E$, we get that $$|C_nf_n(x)| = |lf_n(x)| \leq l 2^{-l}\leq m2^{-m} < \ep$$


\end{document}