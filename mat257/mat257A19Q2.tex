\documentclass[letterpaper]{article}
\usepackage[letterpaper,margin=1in,footskip=0.25in]{geometry}
\usepackage[utf8]{inputenc}
\usepackage{amsmath}
\usepackage{amsthm}
\usepackage{amssymb, pifont}
\usepackage{mathrsfs}
\usepackage{enumitem}
\usepackage{fancyhdr}
\usepackage{hyperref}

\pagestyle{fancy}
\fancyhf{}
\rhead{MAT 257}
\lhead{Assignment 19}
\rfoot{Page \thepage}

\setlength\parindent{24pt}
\renewcommand\qedsymbol{$\blacksquare$}

\DeclareMathOperator{\T}{\mathcal{T}}
\DeclareMathOperator{\V}{\mathcal{V}}
\DeclareMathOperator{\U}{\mathcal{U}}
\DeclareMathOperator{\Prt}{\mathbb{P}}
\DeclareMathOperator{\R}{\mathbb{R}}
\DeclareMathOperator{\N}{\mathbb{N}}
\DeclareMathOperator{\Z}{\mathbb{Z}}
\DeclareMathOperator{\Q}{\mathbb{Q}}
\DeclareMathOperator{\C}{\mathbb{C}}
\DeclareMathOperator{\ep}{\varepsilon}
\DeclareMathOperator{\identity}{\mathbf{0}}
\DeclareMathOperator{\card}{card}
\newcommand{\suchthat}{;\ifnum\currentgrouptype=16 \middle\fi|;}

\newtheorem{lemma}{Lemma}

\newcommand{\bd}{\partial}
\newcommand{\tr}{\mathrm{tr}}
\newcommand{\ra}{\rightarrow}
\newcommand{\lan}{\langle}
\newcommand{\ran}{\rangle}
\newcommand{\norm}[1]{\left\lVert#1\right\rVert}
\newcommand{\inn}[1]{\lan#1\ran}
\newcommand{\ol}{\overline}
\begin{document}
\noindent Q2a: The orientation of $M$ requires that our tangent vector is inward pointing. Hence we take the tangent vector to be $\xi = \Big(\begin{pmatrix} x \\ 0 \\ z \end{pmatrix}, \begin{pmatrix} 0 \\ y\\ 0\end{pmatrix} \Big)$. 
\newline \\ Q2b: Using the chain $c:I\to \bd M, c(t) = (-sin(2\pi t), 0 ,-cos(2\pi t))$, we compute $\int_{\bd M}\omega$ as 
\begin{align*}
    \int_{\bd M} \omega & = \int_{C} \omega 
    \\ & = \int_{I} c^\ast\omega
    \\ & = \int_{[0,1]} -3\sin(2\pi t)\cdot -2\pi \sin(2 \pi t) dt
    \\ & = 3\pi
\end{align*}
\newline \\ Q2c: We will compute $\int_{M}d\omega$ now
\begin{align*}
    \int_{M} d\omega & = \int_{M -\bd M}d\omega 
    \\ & = \int_{\alpha} \omega
    \\ & = \int_{u^2+v^2<1} \alpha^\ast \omega
    \\ & = \int_{u^2+v^1<1} 3du\wedge dv 
    \\ & = 3\pi
\end{align*}
\end{document}