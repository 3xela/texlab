\documentclass[letterpaper]{article}
\usepackage[letterpaper,margin=1in,footskip=0.25in]{geometry}
\usepackage[utf8]{inputenc}
\usepackage{amsmath}
\usepackage{amsthm}
\usepackage{amssymb, pifont}
\usepackage{mathrsfs}
\usepackage{enumitem}
\usepackage{fancyhdr}
\usepackage{hyperref}

\pagestyle{fancy}
\fancyhf{}
\rhead{MAT 257}
\lhead{Assignment 3}
\rfoot{Page \thepage}

\setlength\parindent{24pt}
\renewcommand\qedsymbol{$\blacksquare$}

\DeclareMathOperator{\R}{\mathbb{R}}
\DeclareMathOperator{\N}{\mathbb{N}}
\DeclareMathOperator{\Z}{\mathbb{Z}}
\DeclareMathOperator{\Q}{\mathbb{Q}}
\DeclareMathOperator{\C}{\mathbb{C}}
\DeclareMathOperator{\ep}{\varepsilon}
\DeclareMathOperator{\identity}{\mathbf{0}}
\DeclareMathOperator{\card}{card}
\newcommand{\suchthat}{;\ifnum\currentgrouptype=16 \middle\fi|;}

\newtheorem{lemma}{Lemma}

\newcommand{\tr}{\mathrm{tr}}
\newcommand{\ra}{\rightarrow}
\newcommand{\lan}{\langle}
\newcommand{\ran}{\rangle}
\newcommand{\norm}[1]{\left\lVert#1\right\rVert}
\newcommand{\inn}[1]{\lan#1\ran}
\newcommand{\ol}{\overline}
\begin{document}
Q4a: \\
$f(x,y,z) = x^y$ can be rewritten in the following way. $f(x,y,z) = x^y = e^{y log x} = e^{\pi_2 log(\pi_1)}$. Note that this is only defined for $x >0$.Thus according to the chain rule this function is differntiable. By the chain rule and the differential of the product function, we get that 
\begin{align*}
   & f^{\prime}(x,y,z) = {e^{ylogx}}^\prime [y {(log \pi_1)}^\prime + (logx) {\pi_2}^\prime]
   \\ & = {e^{ylogx}} [y \frac{1}{\pi(x,y,z)} {\pi}^\prime + log(x)(0,1,0)]
   \\ & = {e^{ylogx}} [\frac{y}{x}(1,0,0) + (0,log(x),0)]
   \\ & = x^y[(\frac{y}{x},log(x),0)]
   \\ & = (yx^{y-1},log(x)x^y,0)
\end{align*}
4b: By Spivak Theorem 2.3(3), it follows that ${f^1}^\prime = (yx^{y-1},log(x)x^y,0)$. It remains to determine $(f^2)^\prime$. Since $f^2 = z$ we can rewrite it as $f^2 = 0\pi_1 + 0\pi_2 + \pi_3$. This is differentiable since it is the sum of differntiable functions. 
Thus we have 
\begin{align*}
    (f^2)^\prime = {\pi_3}^\prime = (0,0,1)
\end{align*}
Applying Spivak theorem 2.3(3) again, we get that $f^\prime=\begin{bmatrix}
    yx^{y-1} & log(x)x^y & 0 \\
            0 & 0 & 1 
    \end{bmatrix}$
\\ Q4c: We first note that we can rewrite $f= (x+y)^z$ as follows. $f= (x+y)^z = e^{zlog(x+y)} = e^{\pi_3(log(\pi_1+\pi_2))}$, for $x+y > 0$. Thus we compute $f^\prime$ using the chain rule and the differentials of linear maps: 
\begin{align*}
    & f^\prime = {e^{z(log(x+y))}}^\prime (log(\pi_1+\pi_2) \pi_3 ^\prime + \pi_3 log^\prime(\pi_1+\pi_2))
    \\ & = {e^{z(log(x+y))}}^\prime (log(x+y)(0,0,1) + z \frac{1}{\pi_1 + \pi_2} (\pi_1^\prime + \pi_2^\prime))
    \\ & = {e^{z(log(x+y))}}^\prime [(0,0,log(x+y))+ (\frac{z}{x+y},\frac{z}{x+y},0)]
    \\ & = (x+y)^z [(\frac{z}{x+y},\frac{z}{x+y}, log(x+y)]
    \\ & = (z(x+y)^{z-1},z(x+y)^{z-1},log(x+y)(x+y)^z)
\end{align*}
\end{document}