\documentclass[letterpaper]{article}
\usepackage[letterpaper,margin=1in,footskip=0.25in]{geometry}
\usepackage[utf8]{inputenc}
\usepackage{amsmath}
\usepackage{amsthm}
\usepackage{amssymb, pifont}
\usepackage{mathrsfs}
\usepackage{enumitem}
\usepackage{fancyhdr}
\usepackage{hyperref}

\pagestyle{fancy}
\fancyhf{}
\rhead{MAT 257}
\lhead{Assignment 17}
\rfoot{Page \thepage}

\setlength\parindent{24pt}
\renewcommand\qedsymbol{$\blacksquare$}

\DeclareMathOperator{\T}{\mathcal{T}}
\DeclareMathOperator{\V}{\mathcal{V}}
\DeclareMathOperator{\U}{\mathcal{U}}
\DeclareMathOperator{\Prt}{\mathbb{P}}
\DeclareMathOperator{\R}{\mathbb{R}}
\DeclareMathOperator{\N}{\mathbb{N}}
\DeclareMathOperator{\Z}{\mathbb{Z}}
\DeclareMathOperator{\Q}{\mathbb{Q}}
\DeclareMathOperator{\C}{\mathbb{C}}
\DeclareMathOperator{\ep}{\varepsilon}
\DeclareMathOperator{\identity}{\mathbf{0}}
\DeclareMathOperator{\card}{card}
\newcommand{\suchthat}{;\ifnum\currentgrouptype=16 \middle\fi|;}

\newtheorem{lemma}{Lemma}

\newcommand{\bd}{\partial}
\newcommand{\tr}{\mathrm{tr}}
\newcommand{\ra}{\rightarrow}
\newcommand{\lan}{\langle}
\newcommand{\ran}{\rangle}
\newcommand{\norm}[1]{\left\lVert#1\right\rVert}
\newcommand{\inn}[1]{\lan#1\ran}
\newcommand{\ol}{\overline}
\begin{document}
\noindent Q5a: Since $M,N$, are manifolds, for each $p\in M,q\in N$ there must exist open $U\ni p, V\ni q$ and a $g:U\to \R^{m-k}$, $f: V\to \R^{n-l}$ with $U \cap M = U \cap g^{-1}(\{0\})$ and $V\cap N = V \cap f^{-1}(\{0\})$ and $Dg(p)$ and $Df(q)$ have full rank. We define $h:U\times V \mapsto \R^{m+n -(k+l)}$ by $h(x,y) = (g(x),f(y))$. We can see that $(U \times  V)\cap (M\times N) = (U\times V) \cap h^{-1}(\{0\})$, and we evaluate the differential of $h$ at $p,q$ as $$Dh(p,q) = \begin{bmatrix}Dg(p) & 0 \\ 0 & Df(q) \end{bmatrix}$$ This will have a rank of $n+m-(k+l)$, and hence we conclude that $M\times N$ is a $k+l$ manifold. 
\newline \\ Q5b: We first claim that $\bd M \times \bd N$ is a $k+l-2$ manifold without boundary. First, note that from discussion in class, we know that $\bd M$, $\bd N$ are $k-1$ and $l-1$ manifolds respectively. Hence by 5a, we have that their cartesian product will be a $(k-1)+(l-1) = k+l-2$ manifold. Next, consider the manifold $M\setminus \bd M$. We have that every point in $M \setminus \bd M$ will have some coordinate chart with domain in the interiour of $\R^{k}_{+}$, and similarly for $N \setminus \bd N$ and $\R^l_+$. Hence we will have that $M\setminus \bd M$ will be a $k$ manifold with boundary, and similarly $N \setminus \bd N$ will be an $l$ manifold with boundary. Note that $M \setminus \bd M$ and $N \setminus \bd N$ have empty boundary, but are still manifolds with boundary. By $5a$, $M \setminus \bd M \times N \setminus \bd N$ is a $k+l$ manifold, and by construction is disjoint from $\bd M \times \bd N$. 
\end{document}