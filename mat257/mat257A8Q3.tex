\documentclass[letterpaper]{article}
\usepackage[letterpaper,margin=1in,footskip=0.25in]{geometry}
\usepackage[utf8]{inputenc}
\usepackage{amsmath}
\usepackage{amsthm}
\usepackage{amssymb, pifont}
\usepackage{mathrsfs}
\usepackage{enumitem}
\usepackage{fancyhdr}
\usepackage{hyperref}

\pagestyle{fancy}
\fancyhf{}
\rhead{MAT 257}
\lhead{Assignment 8}
\rfoot{Page \thepage}

\setlength\parindent{24pt}
\renewcommand\qedsymbol{$\blacksquare$}

\DeclareMathOperator{\Prt}{\mathbb{P}}
\DeclareMathOperator{\R}{\mathbb{R}}
\DeclareMathOperator{\N}{\mathbb{N}}
\DeclareMathOperator{\Z}{\mathbb{Z}}
\DeclareMathOperator{\Q}{\mathbb{Q}}
\DeclareMathOperator{\C}{\mathbb{C}}
\DeclareMathOperator{\ep}{\varepsilon}
\DeclareMathOperator{\identity}{\mathbf{0}}
\DeclareMathOperator{\card}{card}
\newcommand{\suchthat}{;\ifnum\currentgrouptype=16 \middle\fi|;}

\newtheorem{lemma}{Lemma}

\newcommand{\tr}{\mathrm{tr}}
\newcommand{\ra}{\rightarrow}
\newcommand{\lan}{\langle}
\newcommand{\ran}{\rangle}
\newcommand{\norm}[1]{\left\lVert#1\right\rVert}
\newcommand{\inn}[1]{\lan#1\ran}
\newcommand{\ol}{\overline}
\begin{document}
Q3:\\ 
Notice that we can rewrite $F(y)=\int_a^b f(x,y)dx = \int_a^b (\int_c^y D_2f(x,y)dy + f(x,y)dx)$. We compute 
\begin{align*}
     \frac{\partial }{\partial y} F(y) & = \frac{\partial}{\partial y} \int_a^b (\int_c^y D_2f(x,y)dy + f(x,y)dx)
     \\ & = \frac{\partial}{\partial y} \left[\int_a^b \int_c^y D_2f(x,y)dydx + \int_a^b f(x,c)dx \right] \tag{by linearity}
     \\ & = \frac{\partial}{\partial y} \int_c^y \int_a^b D_2f(x,y)dxdy + \frac{\partial}{\partial y}\int_a^b f(x,y)dx \tag{by Fubini's Theorem}
     \\ & = \frac{\partial}{\partial y} \int_c^y \int_a^b D_2f(x,y)dxdy
     \\ & = \int_a^b D_2f(x,y)dx \tag{by FTC}
\end{align*}
\end{document}