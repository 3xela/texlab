\documentclass[letterpaper]{article}
\usepackage[letterpaper,margin=1in,footskip=0.25in]{geometry}
\usepackage[utf8]{inputenc}
\usepackage{amsmath}
\usepackage{amsthm}
\usepackage{amssymb, pifont}
\usepackage{mathrsfs}
\usepackage{enumitem}
\usepackage{fancyhdr}
\usepackage{hyperref}

\pagestyle{fancy}
\fancyhf{}
\rhead{MAT 257}
\lhead{Assignment 15}
\rfoot{Page \thepage}

\setlength\parindent{24pt}
\renewcommand\qedsymbol{$\blacksquare$}

\DeclareMathOperator{\T}{\mathcal{T}}
\DeclareMathOperator{\V}{\mathcal{V}}
\DeclareMathOperator{\U}{\mathcal{U}}
\DeclareMathOperator{\Prt}{\mathbb{P}}
\DeclareMathOperator{\R}{\mathbb{R}}
\DeclareMathOperator{\N}{\mathbb{N}}
\DeclareMathOperator{\Z}{\mathbb{Z}}
\DeclareMathOperator{\Q}{\mathbb{Q}}
\DeclareMathOperator{\C}{\mathbb{C}}
\DeclareMathOperator{\ep}{\varepsilon}
\DeclareMathOperator{\identity}{\mathbf{0}}
\DeclareMathOperator{\card}{card}
\newcommand{\suchthat}{;\ifnum\currentgrouptype=16 \middle\fi|;}

\newtheorem{lemma}{Lemma}

\newcommand{\bd}{\partial}
\newcommand{\tr}{\mathrm{tr}}
\newcommand{\ra}{\rightarrow}
\newcommand{\lan}{\langle}
\newcommand{\ran}{\rangle}
\newcommand{\norm}[1]{\left\lVert#1\right\rVert}
\newcommand{\inn}[1]{\lan#1\ran}
\newcommand{\ol}{\overline}
\begin{document}
\noindent Q3: We know that the angle function $\theta$ is defined as follows: 
$\theta(x,y) = \begin{cases} \arctan(\frac{y}{x}) & x>0,y>0 \\ \pi + \arctan(\frac{y}{x})  & x<0 
\\ 2\pi + \arctan{\frac{y}{x}} & x>0,y<0 \\ \frac{\pi}{2}  & x=0,y>0 \\ \frac{3\pi}{2} & x=0,y<0 \end{cases}$ We see that $\theta(x,y)$ is a continiously differentiable function of $x,y$ on $\R^2\{0\}$. We compute $d\theta$ as
\begin{align*}
    d\theta & = dx \wedge \frac{\bd \theta}{\bd x} + dy \wedge \frac{\bd \theta}{\bd y}
    \\ & = \frac{\frac{-y}{x^2}}{1+(\frac{y}{x})^2}dx + \frac{\frac{1}{x}}{1+(\frac{y}{x})^2}dy
    \\ & = \frac{-y}{x^2+y^2}dx + \frac{x}{x^2+y^2}dy
\end{align*}This is well defined on the domain of $\theta$. 
\end{document}