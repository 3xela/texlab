\documentclass[letterpaper]{article}
\usepackage[letterpaper,margin=1in,footskip=0.25in]{geometry}
\usepackage[utf8]{inputenc}
\usepackage{amsmath}
\usepackage{amsthm}
\usepackage{amssymb, pifont}
\usepackage{mathrsfs}
\usepackage{enumitem}
\usepackage{fancyhdr}
\usepackage{hyperref}

\pagestyle{fancy}
\fancyhf{}
\rhead{MAT 257}
\lhead{Assignment 4}
\rfoot{Page \thepage}

\setlength\parindent{24pt}
\renewcommand\qedsymbol{$\blacksquare$}

\DeclareMathOperator{\R}{\mathbb{R}}
\DeclareMathOperator{\N}{\mathbb{N}}
\DeclareMathOperator{\Z}{\mathbb{Z}}
\DeclareMathOperator{\Q}{\mathbb{Q}}
\DeclareMathOperator{\C}{\mathbb{C}}
\DeclareMathOperator{\ep}{\varepsilon}
\DeclareMathOperator{\identity}{\mathbf{0}}
\DeclareMathOperator{\card}{card}
\newcommand{\suchthat}{;\ifnum\currentgrouptype=16 \middle\fi|;}

\newtheorem{lemma}{Lemma}

\newcommand{\tr}{\mathrm{tr}}
\newcommand{\ra}{\rightarrow}
\newcommand{\lan}{\langle}
\newcommand{\ran}{\rangle}
\newcommand{\norm}[1]{\left\lVert#1\right\rVert}
\newcommand{\inn}[1]{\lan#1\ran}
\newcommand{\ol}{\overline}
\begin{document} 
Q6:\\
Define $h$ as $h(t) = f(tx)$. This is differnetiable since it is the composition of differnetiable functions so from the chain rule we have that $$h^\prime (t) = f^\prime (tx) \cdot x = \sum_{i=1}^n D_i f(tx) \cdot x_i$$
By the fundamental theorem of calculus $$\int_0^1 h^\prime(t) dt = f(x) = \int_0^1 \sum_{i=1}^n D_i f(tx) x_i dt$$ 
From the linearity of the integral we can rewrite it as $$f(x) = \sum_{i=1}^n x_i  \int_0^1 D_i f(tx) dt$$ Thus we choose each $g_i = \int_0^1 D_i f(tx) dt$. and so $$f(x) = \sum_{i=1}^n x_i \cdot g_i $$
\end{document}