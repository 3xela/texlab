\documentclass[letterpaper]{article}
\usepackage[letterpaper,margin=1in,footskip=0.25in]{geometry}
\usepackage[utf8]{inputenc}
\usepackage{amsmath}
\usepackage{amsthm}
\usepackage{amssymb, pifont}
\usepackage{mathrsfs}
\usepackage{enumitem}
\usepackage{fancyhdr}
\usepackage{hyperref}

\pagestyle{fancy}
\fancyhf{}
\rhead{MAT 257}
\lhead{Assignment 6}
\rfoot{Page \thepage}

\setlength\parindent{24pt}
\renewcommand\qedsymbol{$\blacksquare$}

\DeclareMathOperator{\Prt}{\mathbb{P}}
\DeclareMathOperator{\R}{\mathbb{R}}
\DeclareMathOperator{\N}{\mathbb{N}}
\DeclareMathOperator{\Z}{\mathbb{Z}}
\DeclareMathOperator{\Q}{\mathbb{Q}}
\DeclareMathOperator{\C}{\mathbb{C}}
\DeclareMathOperator{\ep}{\varepsilon}
\DeclareMathOperator{\identity}{\mathbf{0}}
\DeclareMathOperator{\card}{card}
\newcommand{\suchthat}{;\ifnum\currentgrouptype=16 \middle\fi|;}

\newtheorem{lemma}{Lemma}

\newcommand{\tr}{\mathrm{tr}}
\newcommand{\ra}{\rightarrow}
\newcommand{\lan}{\langle}
\newcommand{\ran}{\rangle}
\newcommand{\norm}[1]{\left\lVert#1\right\rVert}
\newcommand{\inn}[1]{\lan#1\ran}
\newcommand{\ol}{\overline}
\begin{document}
Q6: \\
It has been shown that $\partial A = [0,1]\setminus A$. We first claim that $\Q \cap [0,1]$ is of measure 0. Given $\varepsilon > 0$ we consider a bijection $f:\Q\cap [0,1] \rightarrow \N$. 
For each $r \in \Q\cap [0,1]$ take the interval $(r-\frac{\varepsilon}{2^{f(r)+1}}, r+ \frac{\varepsilon}{2^{f(r)+1}})$. This will cover $\Q\cap [0,1]$. We compute $\sum_{r\in \Q} \frac{\varepsilon}{2^{f(r)}} < \varepsilon$, so $\Q\cap [0,1] $ is measure 0. Therefore if $\sum_i (b_i-a_i) = l < 1$, then any cover of $\partial A$ has length of at least $1-l$ and thus is not of measure 0 . 
\end{document}