\documentclass[letterpaper]{article}
\usepackage[letterpaper,margin=1in,footskip=0.25in]{geometry}
\usepackage[utf8]{inputenc}
\usepackage{amsmath}
\usepackage{amsthm}
\usepackage{amssymb, pifont}
\usepackage{mathrsfs}
\usepackage{enumitem}
\usepackage{fancyhdr}
\usepackage{hyperref}

\pagestyle{fancy}
\fancyhf{}
\rhead{MAT 257}
\lhead{Assignment 19}
\rfoot{Page \thepage}

\setlength\parindent{24pt}
\renewcommand\qedsymbol{$\blacksquare$}

\DeclareMathOperator{\T}{\mathcal{T}}
\DeclareMathOperator{\V}{\mathcal{V}}
\DeclareMathOperator{\U}{\mathcal{U}}
\DeclareMathOperator{\Prt}{\mathbb{P}}
\DeclareMathOperator{\R}{\mathbb{R}}
\DeclareMathOperator{\N}{\mathbb{N}}
\DeclareMathOperator{\Z}{\mathbb{Z}}
\DeclareMathOperator{\Q}{\mathbb{Q}}
\DeclareMathOperator{\C}{\mathbb{C}}
\DeclareMathOperator{\ep}{\varepsilon}
\DeclareMathOperator{\identity}{\mathbf{0}}
\DeclareMathOperator{\card}{card}
\newcommand{\suchthat}{;\ifnum\currentgrouptype=16 \middle\fi|;}

\newtheorem{lemma}{Lemma}

\newcommand{\bd}{\partial}
\newcommand{\tr}{\mathrm{tr}}
\newcommand{\ra}{\rightarrow}
\newcommand{\lan}{\langle}
\newcommand{\ran}{\rangle}
\newcommand{\norm}[1]{\left\lVert#1\right\rVert}
\newcommand{\inn}[1]{\lan#1\ran}
\newcommand{\ol}{\overline}
\begin{document}
\noindent Q1a: Consider $M=\R$ and $\omega = x$. We have that $M$ is clearly not compact and $\bd M = \emptyset$, and $$\int_{M} d \omega \int_{\R} 1dx = \infty \neq \int_{\bd M} \omega = \int_{\emptyset} x=0$$ 
Now suppose that $M$ is a noncompact manifold and $\omega$ is a differential $k-1$ form which is compactly supported on some subset $S\subset M$. By linearity of integration, it is sufficient to assume that $S$ is connected. Let $c$ be some $k$ orientation preserving chain such that its image is $S$. We will check 2 cases, if $c(I^k)\cap \bd M = \emptyset$ or if $c(I^K) \cap \bd M \neq \emptyset$. For the first case evaluate that 
$$\int_{M} d\omega = \int_{c} d\omega = \int_{I^k} d(c^\ast \omega) = \int_{\bd I^k}c^\ast \omega = \int_{\bd c}\omega$$
Where the equalities follow from Stokes' Theorem on chains. Since $\omega=0$ on the boundary of $c$, we have that $$0= \int_{\bd M}\omega = \int_{M}d\omega$$
Now we check the second case. Suppose that there is a singular orientation preserving $k$ cube with $c_{(k,0)}$ being the only face in $\bd M$, with $\omega=0$ outside of $c(I^k)$. Then by Stokes' Theorem on chains we have that $$\int_{M}d\omega = \int_{c}d\omega = \int_{\bd c}\omega = \int_{\bd M}\omega $$
As desired. 
\newline \\ Q1b: Let $\omega = d\eta$. Let $M$ be a compact oriented manifold with no boundary. Then we can evaluate that $$\int_{M}\omega = \int_{M}d\eta = \int_{\bd M}\eta = \int_{\emptyset}\eta = 0$$ Where the second equality follows from Stokes' Theorem. Use the counterexample from $1a$ as a non-compact manifold where the integral does not vanish. 
\end{document}