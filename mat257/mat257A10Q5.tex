\documentclass[letterpaper]{article}
\usepackage[letterpaper,margin=1in,footskip=0.25in]{geometry}
\usepackage[utf8]{inputenc}
\usepackage{amsmath}
\usepackage{amsthm}
\usepackage{amssymb, pifont}
\usepackage{mathrsfs}
\usepackage{enumitem}
\usepackage{fancyhdr}
\usepackage{hyperref}

\pagestyle{fancy}
\fancyhf{}
\rhead{MAT 257}
\lhead{Assignment 10}
\rfoot{Page \thepage}

\setlength\parindent{24pt}
\renewcommand\qedsymbol{$\blacksquare$}

\DeclareMathOperator{\U}{\mathcal{U}}
\DeclareMathOperator{\Prt}{\mathbb{P}}
\DeclareMathOperator{\R}{\mathbb{R}}
\DeclareMathOperator{\N}{\mathbb{N}}
\DeclareMathOperator{\Z}{\mathbb{Z}}
\DeclareMathOperator{\Q}{\mathbb{Q}}
\DeclareMathOperator{\C}{\mathbb{C}}
\DeclareMathOperator{\ep}{\varepsilon}
\DeclareMathOperator{\identity}{\mathbf{0}}
\DeclareMathOperator{\card}{card}
\newcommand{\suchthat}{;\ifnum\currentgrouptype=16 \middle\fi|;}

\newtheorem{lemma}{Lemma}

\newcommand{\tr}{\mathrm{tr}}
\newcommand{\ra}{\rightarrow}
\newcommand{\lan}{\langle}
\newcommand{\ran}{\rangle}
\newcommand{\norm}[1]{\left\lVert#1\right\rVert}
\newcommand{\inn}[1]{\lan#1\ran}
\newcommand{\ol}{\overline}
\begin{document}
\noindent Q5: \\ First notice that $|\det g^\prime| = r >0$. We can apply the COV Theorem to compute the value of the integral. 
\begin{align*}
    \int_{T(A)} 1  & = \int_{A} 1 |\det g^\prime|
    \\ & = \int_0^{2\pi} \int_{b-a}^{b+a} \int_{-\sqrt{a^2-(r-b)^2}}^{\sqrt{a^2-(r-b)^2}} r \quad  dz dr d\theta \tag{by Fubini's Theorem}
    \\ & = 2 \int_0^{2\pi} \int_{b-a}^{b+a} r\sqrt{a^2-(r-b)^2}\quad dr d\theta
    \\ & = 2\int_0^{2\pi} \int_{-a}^a (u+b)\sqrt{a^2-u^2} \quad du d\theta \tag{substitution u = r-b}
    \\ & = 2\int_0^{2\pi} \int_{-a}^a u\sqrt{a^2-u^2} \quad du d\theta + 2\int_0^{2\pi} \int_{-a}^a b\sqrt{a^2-u^2} \quad du d\theta
    \\ & = 2\int_0^{2\pi} \int_{-a}^a b\sqrt{a^2-u^2} \quad du d\theta \tag{since first integral is of an odd function}
    \\ & = 2b\int_0^{2\pi} \frac{\pi}{2}a^2 d\theta
    \\ & = 2\pi^2 a^2b
\end{align*}
\end{document}