\documentclass[letterpaper]{article}
\usepackage[letterpaper,margin=1in,footskip=0.25in]{geometry}
\usepackage[utf8]{inputenc}
\usepackage{amsmath}
\usepackage{amsthm}
\usepackage{amssymb, pifont}
\usepackage{mathrsfs}
\usepackage{enumitem}
\usepackage{fancyhdr}
\usepackage{hyperref}

\pagestyle{fancy}
\fancyhf{}
\rhead{MAT 257}
\lhead{Assignment 2}
\rfoot{Page \thepage}

\setlength\parindent{24pt}
\renewcommand\qedsymbol{$\blacksquare$}

\DeclareMathOperator{\R}{\mathbb{R}}
\DeclareMathOperator{\N}{\mathbb{N}}
\DeclareMathOperator{\Z}{\mathbb{Z}}
\DeclareMathOperator{\Q}{\mathbb{Q}}
\DeclareMathOperator{\C}{\mathbb{C}}
\DeclareMathOperator{\ep}{\varepsilon}
\DeclareMathOperator{\identity}{\mathbf{0}}
\DeclareMathOperator{\card}{card}
\newcommand{\suchthat}{;\ifnum\currentgrouptype=16 \middle\fi|;}

\newtheorem{lemma}{Lemma}

\newcommand{\tr}{\mathrm{tr}}
\newcommand{\ra}{\rightarrow}
\newcommand{\lan}{\langle}
\newcommand{\ran}{\rangle}
\newcommand{\norm}[1]{\left\lVert#1\right\rVert}
\newcommand{\inn}[1]{\lan#1\ran}
\newcommand{\ol}{\overline}
\begin{document}
Q2a:
\\ First we consider $h(t)$. 
\begin{align*}
    & h(t) = f(tx)
    \\ & = \norm{tx} g(\frac{tx}{\norm{tx}})
    \\ & = |t| \norm{x} g(\frac{tx}{|t|\norm{x}})
    \\ & = t \norm{x}g(\frac{x}{\norm{x}})
    \\ & = tf(x)
\end{align*}Therefore, $h$ is linear and so from single variable calculus we get that $h^\prime= f(x)$
\newline b:
\\ First, if $g=0 $ then clearfly $f$  is differntiable with $Df(x,y)=0$ .If $g(x) \neq 0$ then suppose that $Df(0,0)$ exists. We see that 
\begin{align*}
    &\lim_{h \rightarrow 0 } \frac{\norm{f(h,0) -f(0,0) -Df(0,0)(h,0)}}{\norm{h}}=0
    \\ & \implies Df(0,0) (h,0)= 0
\end{align*}
and 
\begin{align*}
    & \lim_{k \rightarrow 0} \frac{\norm{f(0,k) - f(0,0) -Df(0,0)(0,k)}}{\norm{k}}=0
    \\ & \implies Df(0,0)(0,k) = 0
\end{align*} 
Therefore, $Df(0,0) = 0$.So if this is the differntial it must be the case that 
\begin{align*}
    & \lim_{x \rightarrow 0 } \frac{f(x)- f(0)- Df(0,0)}{\norm{x}}
    \\ & = \lim_{x \rightarrow 0} \frac{\norm{x} g(\frac{x}{\norm{x}})}{\norm{x}}
    \\ & = \lim_{x \rightarrow 0} g(\frac{x}{\norm{x}}) = 0
\end{align*}However by the definition of $g$ this limit does not exist unless $g=0$. We obtain a contradiction and so $Df(0,0)$ does not exist unless $g=0$. 
\end{document}