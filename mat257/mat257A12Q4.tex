\documentclass[letterpaper]{article}
\usepackage[letterpaper,margin=1in,footskip=0.25in]{geometry}
\usepackage[utf8]{inputenc}
\usepackage{amsmath}
\usepackage{amsthm}
\usepackage{amssymb, pifont}
\usepackage{mathrsfs}
\usepackage{enumitem}
\usepackage{fancyhdr}
\usepackage{hyperref}

\pagestyle{fancy}
\fancyhf{}
\rhead{MAT 257}
\lhead{Assignment 12}
\rfoot{Page \thepage}

\setlength\parindent{24pt}
\renewcommand\qedsymbol{$\blacksquare$}

\DeclareMathOperator{\T}{\mathcal{T}}
\DeclareMathOperator{\V}{\mathcal{V}}
\DeclareMathOperator{\U}{\mathcal{U}}
\DeclareMathOperator{\Prt}{\mathbb{P}}
\DeclareMathOperator{\R}{\mathbb{R}}
\DeclareMathOperator{\N}{\mathbb{N}}
\DeclareMathOperator{\Z}{\mathbb{Z}}
\DeclareMathOperator{\Q}{\mathbb{Q}}
\DeclareMathOperator{\C}{\mathbb{C}}
\DeclareMathOperator{\ep}{\varepsilon}
\DeclareMathOperator{\identity}{\mathbf{0}}
\DeclareMathOperator{\card}{card}
\newcommand{\suchthat}{;\ifnum\currentgrouptype=16 \middle\fi|;}

\newtheorem{lemma}{Lemma}

\newcommand{\tr}{\mathrm{tr}}
\newcommand{\ra}{\rightarrow}
\newcommand{\lan}{\langle}
\newcommand{\ran}{\rangle}
\newcommand{\norm}[1]{\left\lVert#1\right\rVert}
\newcommand{\inn}[1]{\lan#1\ran}
\newcommand{\ol}{\overline}
\begin{document}
\noindent Q4: We begin by defining $\underline{n}_s^k = \{ (i_1, \dots i_k): 1\leq i_1 \leq \dots \leq i_k \}$. We claim that $\sigma_I = \sum_{\sigma\in S_k}\varphi_{I}\circ \sigma^*$ is a basis of $S^k(V)$. We will first show that $\sigma_I\in S^k(V)$. Let $u_1,\dots u_k$ be some vectors in $V$. Let $\tau\in S_k$. Then we evaluate that 
\begin{align*}
    \sigma_I \circ \tau^{*}(u_1, \dots , u_k) & = \sum_{\sigma\in S_k} \varphi_I \circ \sigma^{*} ( u_{\tau(1)},  \dots, u_{\tau(k)})
    \\ & = \sum_{\sigma \in S_k} \varphi_I (u_{\sigma(\tau(1))} , \dots u_{\sigma(\tau(k))})
    \\ & = \sum_{\lambda \in S_k} \varphi_I ( u_{\lambda(1)}, \dots , u_{\lambda(k)}) \tag{since for fixed $\tau$, $\sigma\circ\tau$ is $S_k$ for $\sigma\in S_k$}
    \\ & = \sigma_{I}(u_1 , \dots , u_k)
\end{align*} Therefore $\sigma_I$ is in $S^k(V)$. We now claim that $\{\sigma_I: I\in \underline{n}_s^k \}$ forms a basis for $S^k(V)$. We will show this in several steps. Let $v_1\dots v_n$ be a basis for $V$. The first claim we make is that given $I,J\in \underline{n}_s^k,\sigma_I(v_J) = \delta_{IJ}$. We can check indeed that 
$$\sigma_I(v_J) = \sum_{\sigma \in S_k}\varphi_I \circ \sigma^* (v_1 , \dots, v_k)  = \sum_{\sigma \in S_k} \varphi_{i_1}\otimes \dots \otimes \varphi_{i_k}(v_{\sigma(1)}, \dots, v_{\sigma(k)}) = \sum_{\sigma \in S_k} \varphi_{i_k}(v_{\sigma(1)})\cdot \dots \varphi_{i_k}(v_{\sigma(k)}) = \delta_{IJ}$$
Where the last equality holds because the sum will be 0 unless under some permutation, $\sigma^{*}(J)=I$. Since each $\sigma_I$ is a k-tensor, it follows that $S_1=S_2$ if and only if $S_1(v_I)=S_2(v_I)$ for all $I\in \underline{n}_s^k$. We now claim that all the $\sigma_I$ span $S^k(V)$. Let $S\in S^k(V)$. We want to find $a_I$ such that $S=\sum a_I \sigma_I$. Take $a_I=S(v_I)$. Then it is enough to show that $S(v_J)=\sum a_I \sigma_I(v_J)$. $$S(v_J) =\sum a_I \delta_{IJ} = a_J$$
Finally, we claim that $\{\sigma_I \}$ is a linearly independant set. Suppose for some $b_I$, $\sum b_I \omega_I = 0$. Evaluating this on $v_J$ gives us that $$b_J = \sum b_I \sigma_I (v_J) = 0(b_J) = 0$$ Therefore each $b_I$ is 0, and thus this set is linearly independant and spans $S^k(V)$. So it is a basis. By our discussion in lecture 47, $|\underline{n}_s^k| = \begin{pmatrix}
    n+k-1 \\ k
\end{pmatrix}$, and so $dim(S^k)V = | \underline{n}_s^k |$ thus we are done. 
\end{document}