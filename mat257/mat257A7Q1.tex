\documentclass[letterpaper]{article}
\usepackage[letterpaper,margin=1in,footskip=0.25in]{geometry}
\usepackage[utf8]{inputenc}
\usepackage{amsmath}
\usepackage{amsthm}
\usepackage{amssymb, pifont}
\usepackage{mathrsfs}
\usepackage{enumitem}
\usepackage{fancyhdr}
\usepackage{hyperref}

\pagestyle{fancy}
\fancyhf{}
\rhead{MAT 257}
\lhead{Assignment 6}
\rfoot{Page \thepage}

\setlength\parindent{24pt}
\renewcommand\qedsymbol{$\blacksquare$}

\DeclareMathOperator{\Prt}{\mathbb{P}}
\DeclareMathOperator{\R}{\mathbb{R}}
\DeclareMathOperator{\N}{\mathbb{N}}
\DeclareMathOperator{\Z}{\mathbb{Z}}
\DeclareMathOperator{\Q}{\mathbb{Q}}
\DeclareMathOperator{\C}{\mathbb{C}}
\DeclareMathOperator{\ep}{\varepsilon}
\DeclareMathOperator{\identity}{\mathbf{0}}
\DeclareMathOperator{\card}{card}
\newcommand{\suchthat}{;\ifnum\currentgrouptype=16 \middle\fi|;}

\newtheorem{lemma}{Lemma}

\newcommand{\tr}{\mathrm{tr}}
\newcommand{\ra}{\rightarrow}
\newcommand{\lan}{\langle}
\newcommand{\ran}{\rangle}
\newcommand{\norm}[1]{\left\lVert#1\right\rVert}
\newcommand{\inn}[1]{\lan#1\ran}
\newcommand{\ol}{\overline}
\begin{document}
Q1a: \\
Let $P$ be a partition of $A$. We have that 
\begin{align*}
    & m_S(f)+m_S(g) 
    \\ & = \inf_{x\in S} f(x) +  \inf_{x\in S} g(x)
    \\ & \leq \inf_{x\in S} (f+g)(x) \tag*{by discussion in lecture 25}
    \\ & = m_S(f+g)
\end{align*}
 And similiarly, $M_S(f+g) \leq M_S(f) + M_S(g)$. Therefore we have that 
\begin{align*}
    & L(f,P) + L(g,P) 
    \\ & = \sum_{S\in P} \inf_{x\in S} f(x) \cdot vol(S) + \sum_{S\in P} \inf_{x\in S} g(x) \cdot vol(S)
    \\ & \leq \sum_{S\in P} \inf_{x\in S}(f+g) \cdot vol(S) 
    \\ & = L(f+g,P)
\end{align*}
Similiarly we have $U(f+g,P)\leq U(f,P) + U(g,P)$. 
\\ 1b: \\ 
Choose partitions $P_1$ and $P_2$ such that $U(f,P_1)-L(f,P_1) < \frac{\varepsilon}{2}$ and $U(g,P_2)- U(g,P_2) < \frac{\varepsilon}{2}$. Let $P_3$ be a partition which refines $P_1$ and $P_2$, now by 1a we have that 
$$U(f+g,P_3) - L(f+g,P_3) \leq U(f,P_1) + U(g,P_2) - L(f,P_1) - L(g,P_2) < \varepsilon$$ Thus if $f$ and $g$ are integrable so is $f+g$. We also have that 
$$L(f)+ L(g) \leq L(f+g) \leq U(f+g) \leq U(f) + U(g)$$ which implies that $\int_A (f+g) = \int_A f + \int_A g$
\\ 1c: \\ Let $c\in \R$. It suffices to check 3 cases, $c <0,c=0,c>0$. When $c=0$ the statement is trivially true, since $\int_A 0\cdot f = 0 = 0\cdot \int_A f$. If $c>0$,  let $\varepsilon >0$. Choose partition $P$ such that $U(f,P)-L(f,p) < \frac{\varepsilon}{c}$. We compute 
\begin{align*}
& U(cf,P)- L(cf,P)
\\ & = \sum_{S\in P} [M_S(cf) - m_S(cf)] \cdot vol(S)
\\ & = \sum_{s\in P}c [M_S(f)-m_S(f)]\cdot vol(S)
\\ & = c [U(f,P)-L(f,P)]
\\ & < \varepsilon
\end{align*}Now suppose that $c<0$, Let $\varepsilon >0$. Choose a partition $P$ such that $U(cf,P)-L(cf,P)< - \frac{\varepsilon}{c}$. we compute 
\begin{align*}
    & U(cf,P)- L(cf,P)
    \\ & = \sum_{S\in P}[M_S(cf)-m_S(cf)]\cdot vol(S)
    \\ & = \sum_{S\in P}[c\cdot m_S(f)-c \cdot M_s(f)]\cdot vol(S) \tag*{since multiplying by c below 0 swaps sup and inf}
    \\ & = \sum_{S\in P } -c \cdot [M_S(f)-m_S(f)]\cdot vol(S)
    \\ & < \varepsilon
\end{align*} Hence for any constant $c$ we have that if $f$ integrable, so is $cf$. By above we know that 
$$c[U(f,P)-L(f,p)]= U(cf,P)-L(cf,P)< \varepsilon$$ Since this is true for all $\varepsilon>0$, it must be that $\int_A cf = c \int_A f$
\end{document}