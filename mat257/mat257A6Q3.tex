\documentclass[letterpaper]{article}
\usepackage[letterpaper,margin=1in,footskip=0.25in]{geometry}
\usepackage[utf8]{inputenc}
\usepackage{amsmath}
\usepackage{amsthm}
\usepackage{amssymb, pifont}
\usepackage{mathrsfs}
\usepackage{enumitem}
\usepackage{fancyhdr}
\usepackage{hyperref}

\pagestyle{fancy}
\fancyhf{}
\rhead{MAT 257}
\lhead{Assignment 6}
\rfoot{Page \thepage}

\setlength\parindent{24pt}
\renewcommand\qedsymbol{$\blacksquare$}

\DeclareMathOperator{\R}{\mathbb{R}}
\DeclareMathOperator{\N}{\mathbb{N}}
\DeclareMathOperator{\Z}{\mathbb{Z}}
\DeclareMathOperator{\Q}{\mathbb{Q}}
\DeclareMathOperator{\C}{\mathbb{C}}
\DeclareMathOperator{\ep}{\varepsilon}
\DeclareMathOperator{\identity}{\mathbf{0}}
\DeclareMathOperator{\card}{card}
\newcommand{\suchthat}{;\ifnum\currentgrouptype=16 \middle\fi|;}

\newtheorem{lemma}{Lemma}

\newcommand{\tr}{\mathrm{tr}}
\newcommand{\ra}{\rightarrow}
\newcommand{\lan}{\langle}
\newcommand{\ran}{\rangle}
\newcommand{\norm}[1]{\left\lVert#1\right\rVert}
\newcommand{\inn}[1]{\lan#1\ran}
\newcommand{\ol}{\overline}
\begin{document}
Q3:\\ We know that $\frac{\partial (f,g)}{\partial (x,y,z)} = D(f,g)$ and so we have that $$\frac{\partial (f,g)}{\partial (x,y,z)} = \begin{bmatrix} \frac{\partial f}{\partial x} & \frac{\partial f}{\partial y} & \frac{\partial f}{\partial z} \\ \frac{\partial g}{\partial x} & \frac{\partial g}{\partial y} & \frac{\partial g}{\partial z}\end{bmatrix}$$
Since the rank of $D(f,g)(p)=2$ it must be that the dimension of the span of the columns is $2$. Hence one of the column vectors is in the span of the other $2$. Assume $WLOG$ that the first column is as such. So have that columns 2 and 3 must be linearly independant. From linear algebra it must be that $ det \left( \begin{bmatrix} \frac{\partial f}{\partial y} & \frac{\partial f}{\partial z} \\ \frac{\partial g}{\partial y} & \frac{\partial g}{\partial z}\end{bmatrix} \right) \neq 0$ 
By the Implicit Function Theorem, there exists an open neighbourhood $A \ni x_0$ and an open $B \ni (y_0,z_0)$ along with $(h,k):A \rightarrow B$ with $(f,g)(x,h(x),k(x))=0$ for all $x\in A$. If we define $\gamma : A \rightarrow \R^3$ by $\gamma(x) = (x,h(x),k(x))$, this will solve both $f$ and $g$ near $p$.  
\end{document}