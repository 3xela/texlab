\documentclass[letterpaper]{article}
\usepackage[letterpaper,margin=1in,footskip=0.25in]{geometry}
\usepackage[utf8]{inputenc}
\usepackage{amsmath}
\usepackage{amsthm}
\usepackage{amssymb, pifont}
\usepackage{mathrsfs}
\usepackage{enumitem}
\usepackage{fancyhdr}
\usepackage{hyperref}

\pagestyle{fancy}
\fancyhf{}
\rhead{MAT 257}
\lhead{Assignment 3}
\rfoot{Page \thepage}

\setlength\parindent{24pt}
\renewcommand\qedsymbol{$\blacksquare$}

\DeclareMathOperator{\R}{\mathbb{R}}
\DeclareMathOperator{\N}{\mathbb{N}}
\DeclareMathOperator{\Z}{\mathbb{Z}}
\DeclareMathOperator{\Q}{\mathbb{Q}}
\DeclareMathOperator{\C}{\mathbb{C}}
\DeclareMathOperator{\ep}{\varepsilon}
\DeclareMathOperator{\identity}{\mathbf{0}}
\DeclareMathOperator{\card}{card}
\newcommand{\suchthat}{;\ifnum\currentgrouptype=16 \middle\fi|;}

\newtheorem{lemma}{Lemma}

\newcommand{\tr}{\mathrm{tr}}
\newcommand{\ra}{\rightarrow}
\newcommand{\lan}{\langle}
\newcommand{\ran}{\rangle}
\newcommand{\norm}[1]{\left\lVert#1\right\rVert}
\newcommand{\inn}[1]{\lan#1\ran}
\newcommand{\ol}{\overline}
\begin{document}
Q1:\\
Suppose that $f : \R^n \rightarrow \R^m$ is differentiable. Then, we can write $f(a+h) = f(a) + Df(h) + o(h)$, for some linear map $Df(a) \in \mathcal{L}(\R^n,\R^m)$. 
Consider the following expression. 
\begin{align*}
     & \lim_{h \rightarrow 0} f(a+h) -f(a)
     \\ & = \lim_{h \rightarrow 0} \norm{h} \frac{f(a+h)-f(a) +Df(a)h}{\norm{h}} -Df(a)h
     \\ & = \lim_{h \rightarrow 0} \norm{h} \frac{o(h)}{\norm{h}} -Df(a)h \text{ (by assumption)}
     \\ & = \lim_{h\rightarrow 0} \norm{h} \lim_{h \rightarrow 0} \frac{o(h)}{\norm{h}} - \lim_{h\rightarrow 0} Df(a)h \text{ (by properties of limits)}
     \\ & = 0
     \\ & \implies \lim_{h \rightarrow 0} f(a+h) = f(a)
\end{align*}
It follows that $f$ is continous at $a$. $\qed$
\end{document}