\documentclass[letterpaper]{article}
\usepackage[letterpaper,margin=1in,footskip=0.25in]{geometry}
\usepackage[utf8]{inputenc}
\usepackage{amsmath}
\usepackage{amsthm}
\usepackage{amssymb, pifont}
\usepackage{mathrsfs}
\usepackage{enumitem}
\usepackage{fancyhdr}
\usepackage{hyperref}

\pagestyle{fancy}
\fancyhf{}
\rhead{MAT 257}
\lhead{Assignment 1}
\rfoot{Page \thepage}

\setlength\parindent{24pt}
\renewcommand\qedsymbol{$\blacksquare$}

\DeclareMathOperator{\R}{\mathbb{R}}
\DeclareMathOperator{\N}{\mathbb{N}}
\DeclareMathOperator{\Z}{\mathbb{Z}}
\DeclareMathOperator{\Q}{\mathbb{Q}}
\DeclareMathOperator{\C}{\mathbb{C}}
\DeclareMathOperator{\ep}{\varepsilon}
\DeclareMathOperator{\identity}{\mathbf{0}}
\DeclareMathOperator{\card}{card}
\newcommand{\suchthat}{;\ifnum\currentgrouptype=16 \middle\fi|;}

\newtheorem{lemma}{Lemma}

\newcommand{\tr}{\mathrm{tr}}
\newcommand{\ra}{\rightarrow}
\newcommand{\lan}{\langle}
\newcommand{\ran}{\rangle}
\newcommand{\norm}[1]{\left\lVert#1\right\rVert}
\newcommand{\inn}[1]{\lan#1\ran}
\newcommand{\ol}{\overline}
\begin{document}
Q1a: 
\newline " $\implies$ " 
\newline Suppose that $T$ is inner product preserving. Then, 
\begin{equation*}
\inn{Tx,Tx} = \inn{x,x} = \norm{Tx}^2 = \norm{x}^2
\end{equation*}
Since norms are positive, this implies that $\norm{Tx} = \norm{x}$
\newline
"$\impliedby$"
\newline Suppose that $T$ is norm preserving, then by the polarization identity,
\begin{align*}
    \inn{Tx,Ty}  & = \frac{\norm{T(x+y)}^2-\norm{T(x-y)}^2}{4} 
    \\ & = \frac{\norm{x+y}^2-\norm{x-y}^2}{4}
    \\ &= \inn{x,y}
\end{align*}
$\qed$
\newline 1b:
\newline Suppose that $T$ is a norm preserving linear map from $\R^n \rightarrow \R^n$. Consider the case when $\norm{T(x)}=0$. By assumption, it must be that $\norm{x}=0$. By the properties of the norm, this is equvalent to $x=0$. 
Therefore, $T$ is an injective mapping. From the rank-nullity theorem, it follows that the dimension of the range of $T$ is $n$, so $T$ must also be a surjective linear map. Hence $T$ is a bijective mapping. 
Therefore the linear map $T^{-1}$ must exist. By assumption $T$ is norm preserving so $\norm{T^{-1}(T(x))} =\norm{x} = \norm{T(x)}$. By part a, it follows that $T^{-1}$ is also inner product preserving. $\qed$
\end{document}