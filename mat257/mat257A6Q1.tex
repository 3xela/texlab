\documentclass[letterpaper]{article}
\usepackage[letterpaper,margin=1in,footskip=0.25in]{geometry}
\usepackage[utf8]{inputenc}
\usepackage{amsmath}
\usepackage{amsthm}
\usepackage{amssymb, pifont}
\usepackage{mathrsfs}
\usepackage{enumitem}
\usepackage{fancyhdr}
\usepackage{hyperref}

\pagestyle{fancy}
\fancyhf{}
\rhead{MAT 257}
\lhead{Assignment 6}
\rfoot{Page \thepage}

\setlength\parindent{24pt}
\renewcommand\qedsymbol{$\blacksquare$}

\DeclareMathOperator{\R}{\mathbb{R}}
\DeclareMathOperator{\N}{\mathbb{N}}
\DeclareMathOperator{\Z}{\mathbb{Z}}
\DeclareMathOperator{\Q}{\mathbb{Q}}
\DeclareMathOperator{\C}{\mathbb{C}}
\DeclareMathOperator{\ep}{\varepsilon}
\DeclareMathOperator{\identity}{\mathbf{0}}
\DeclareMathOperator{\card}{card}
\newcommand{\suchthat}{;\ifnum\currentgrouptype=16 \middle\fi|;}

\newtheorem{lemma}{Lemma}

\newcommand{\tr}{\mathrm{tr}}
\newcommand{\ra}{\rightarrow}
\newcommand{\lan}{\langle}
\newcommand{\ran}{\rangle}
\newcommand{\norm}[1]{\left\lVert#1\right\rVert}
\newcommand{\inn}[1]{\lan#1\ran}
\newcommand{\ol}{\overline}
\begin{document}
Q1a:\\
Since $f$ is $C^1$, $f(3,-1,2)=0$ and $\frac{\partial f}{\partial y}  = \begin{bmatrix}
    2 & 1 \\ -1 & 1
\end{bmatrix}$ has nonzero determinant, the Implicit Function Theorem asserts that there must exist an open neighbourhood $B$ of $3$ and an open neighbourhood $A$ of $(-1,2)$ with a continous differentiable function $g: B \rightarrow A$ such that $f(x,g_1(x),g_2(x))=0$. 
\\ Q1b: \\
By the Implicit Function Theorem, we can compute $g^\prime$ in the following way. $$g^\prime = - \left[\frac{\partial f}{\partial y}\right] ^{-1} \cdot \frac{\partial f}{\partial x}$$
As given, we know that $\frac{\partial f}{\partial x} = \begin{bmatrix}
    1 \\ 1
\end{bmatrix}$ and $\frac{\partial f}{\partial y} = \begin{bmatrix} 2 & 1 \\ -1 & 1 \end{bmatrix}$. We compute $\frac{\partial f}{\partial y}^{-1} = \begin{bmatrix} \frac{1}{3} & - \frac{1}{3} \\ \frac{1}{3} & \frac{2}{3} \end{bmatrix}$. Therefore 
$$g^\prime(3) = - \begin{bmatrix} \frac{1}{3} & - \frac{1}{3} \\ \frac{1}{3} & \frac{2}{3} \end{bmatrix} \cdot \begin{bmatrix} 1 \\ 1 \end{bmatrix} = \begin{bmatrix} 0 \\ -1 \end{bmatrix}$$
\\ Q1c: \\ To be able to find a function that that gives us $(y_1,y_2)$ in terms of $x$ we would require that $\frac{\partial f}{\partial y}$ at that point be invertible, in order to guarantee the excistence of a function $g(x)= (g_1(x),g_2(x)) = (y_1,y_2)$. For instance, we can not solve for $y_1$ as a function of $x$ and $y_2$ since the matrix $\frac{\partial f}{\partial x,y_2}$ has determinant zero and is not invertible. 
\end{document}