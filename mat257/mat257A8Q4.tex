\documentclass[letterpaper]{article}
\usepackage[letterpaper,margin=1in,footskip=0.25in]{geometry}
\usepackage[utf8]{inputenc}
\usepackage{amsmath}
\usepackage{amsthm}
\usepackage{amssymb, pifont}
\usepackage{mathrsfs}
\usepackage{enumitem}
\usepackage{fancyhdr}
\usepackage{hyperref}

\pagestyle{fancy}
\fancyhf{}
\rhead{MAT 257}
\lhead{Assignment 8}
\rfoot{Page \thepage}

\setlength\parindent{24pt}
\renewcommand\qedsymbol{$\blacksquare$}

\DeclareMathOperator{\Prt}{\mathbb{P}}
\DeclareMathOperator{\R}{\mathbb{R}}
\DeclareMathOperator{\N}{\mathbb{N}}
\DeclareMathOperator{\Z}{\mathbb{Z}}
\DeclareMathOperator{\Q}{\mathbb{Q}}
\DeclareMathOperator{\C}{\mathbb{C}}
\DeclareMathOperator{\ep}{\varepsilon}
\DeclareMathOperator{\identity}{\mathbf{0}}
\DeclareMathOperator{\card}{card}
\newcommand{\suchthat}{;\ifnum\currentgrouptype=16 \middle\fi|;}

\newtheorem{lemma}{Lemma}

\newcommand{\tr}{\mathrm{tr}}
\newcommand{\ra}{\rightarrow}
\newcommand{\lan}{\langle}
\newcommand{\ran}{\rangle}
\newcommand{\norm}[1]{\left\lVert#1\right\rVert}
\newcommand{\inn}[1]{\lan#1\ran}
\newcommand{\ol}{\overline}
\begin{document}
Q4:\\
Given that $f(x,y) = \int_x^x g_1(t,0)dt + \int_0^y g_2(x,t)dt$, we will compute $D_1f$ and $D_2f$. First, $D_1f$
\begin{align*}
    D_1f & = D_1\int_0^x g_1(t,0)dt + D_1 \int_0^y g_2(x,t)dt
    \\ & = g_1(x,0) + D_1 \int_0^y g_2(x,t)dt \tag{by FTC}
    \\ & = g_1(x,0) + \int_0^y D_1 g_2(x,t)dt \tag{by the Leibniz Rule}
    \\ & = g_1(x,0) + \int_0^y D_2 g_1(x,t)dt \tag{by assumption}
    \\ & = g_1(x,0) + g_1(x,y) - g(x,0) \tag{by FTC}
    \\ & = g_1
\end{align*} As desired. We now will compute $D_2f$
\begin{align*}
    D_2 f & = D_2 \int_0^x g_1(t,0)dt + D_2 \int_0^y g_2(x,t)dt 
    \\ & = D_2 \int_0^x g_1(t,0)dt + g_2(x,y) \tag{by FTC}
    \\ & = 0 +  g_2(x,y) \tag{since first integral contstant in y}
    \\ & = g_2(x,y)
\end{align*} As expected.
\end{document}