\documentclass[letterpaper]{article}
\usepackage[letterpaper,margin=1in,footskip=0.25in]{geometry}
\usepackage[utf8]{inputenc}
\usepackage{amsmath}
\usepackage{amsthm}
\usepackage{amssymb, pifont}
\usepackage{mathrsfs}
\usepackage{enumitem}
\usepackage{fancyhdr}
\usepackage{hyperref}

\pagestyle{fancy}
\fancyhf{}
\rhead{MAT 257}
\lhead{Assignment 17}
\rfoot{Page \thepage}

\setlength\parindent{24pt}
\renewcommand\qedsymbol{$\blacksquare$}

\DeclareMathOperator{\T}{\mathcal{T}}
\DeclareMathOperator{\V}{\mathcal{V}}
\DeclareMathOperator{\U}{\mathcal{U}}
\DeclareMathOperator{\Prt}{\mathbb{P}}
\DeclareMathOperator{\R}{\mathbb{R}}
\DeclareMathOperator{\N}{\mathbb{N}}
\DeclareMathOperator{\Z}{\mathbb{Z}}
\DeclareMathOperator{\Q}{\mathbb{Q}}
\DeclareMathOperator{\C}{\mathbb{C}}
\DeclareMathOperator{\ep}{\varepsilon}
\DeclareMathOperator{\identity}{\mathbf{0}}
\DeclareMathOperator{\card}{card}
\newcommand{\suchthat}{;\ifnum\currentgrouptype=16 \middle\fi|;}

\newtheorem{lemma}{Lemma}

\newcommand{\bd}{\partial}
\newcommand{\tr}{\mathrm{tr}}
\newcommand{\ra}{\rightarrow}
\newcommand{\lan}{\langle}
\newcommand{\ran}{\rangle}
\newcommand{\norm}[1]{\left\lVert#1\right\rVert}
\newcommand{\inn}[1]{\lan#1\ran}
\newcommand{\ol}{\overline}
\begin{document}
\noindent Q1a: Define the function $g(x,y)=y-f(x)$. This is clearly a $C^\infty$ function, and $g^{-1}\{0\} =\Gamma_f$. To show that $\Gamma_f$ is an $n$ manifold it is enough to show that $rank(Dg)=n$. Notice that $$Dg = \Big[-\frac{\bd f}{\bd x} | I \Big]$$ This will be an $(n)\times (n+m)$ matrix, and hence will have a rank of $m$, since the identity matrix will span a $m$ dimensional space. Thereforem we have that $\Gamma_f$ is a manifold, if $f$ is smooth. 
\newline \\ Q1b: Consider the graph of the function $f(x) = x^{\frac{1}{3}}$. This will be a smooth manifold, since in every neighborhood, the function locally looks like $\R$, but $x^\frac{1}{3}$ is not $C^1$ since its derivative is not continuous at $0$.  
\end{document}