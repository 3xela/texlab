\documentclass[letterpaper]{article}
\usepackage[letterpaper,margin=1in,footskip=0.25in]{geometry}
\usepackage[utf8]{inputenc}
\usepackage{amsmath}
\usepackage{amsthm}
\usepackage{amssymb, pifont}
\usepackage{mathrsfs}
\usepackage{enumitem}
\usepackage{fancyhdr}
\usepackage{hyperref}

\pagestyle{fancy}
\fancyhf{}
\rhead{MAT 257}
\lhead{Assignment 1}
\rfoot{Page \thepage}

\setlength\parindent{24pt}
\renewcommand\qedsymbol{$\blacksquare$}

\DeclareMathOperator{\R}{\mathbb{R}}
\DeclareMathOperator{\N}{\mathbb{N}}
\DeclareMathOperator{\Z}{\mathbb{Z}}
\DeclareMathOperator{\Q}{\mathbb{Q}}
\DeclareMathOperator{\C}{\mathbb{C}}
\DeclareMathOperator{\ep}{\varepsilon}
\DeclareMathOperator{\identity}{\mathbf{0}}
\DeclareMathOperator{\card}{card}
\newcommand{\suchthat}{;\ifnum\currentgrouptype=16 \middle\fi|;}

\newtheorem{lemma}{Lemma}

\newcommand{\tr}{\mathrm{tr}}
\newcommand{\ra}{\rightarrow}
\newcommand{\lan}{\langle}
\newcommand{\ran}{\rangle}
\newcommand{\norm}[1]{\left\lVert#1\right\rVert}
\newcommand{\inn}[1]{\lan#1\ran}
\newcommand{\ol}{\overline}
\begin{document}
Q3:\newline
We first show that $T$ is a linear mapping. 
\begin{align*}
    & \phi_{\alpha x +y}
    \\ & = T(\alpha x +y)
    \\ & = \inn{\alpha x + y,z}
    \\ & = \alpha \inn{x,z} + \inn{y,z}
    \\ & = \alpha \phi_x + \phi_y
\end{align*}
Therefore $T$ is a linear map from $\R^n$ to $(\R^n)^*$\newline
We now claim that $T$ is injective. Suppose that $\phi_x = \phi_y$. 
\begin{align*}
    & \implies T(x)= T(y)
    \\ & \implies  \inn{x,z} = \inn{y,z} \quad \forall z \in \R^n
    \\ & \implies \inn{x,z} -\inn{y,z} = 0 \quad \forall z \in \R^n
    \\ & \implies \inn{x-y,z} = 0 \quad \forall z\in \R^n
    \\ & \implies   x-y=0
    \\ & \implies   x=y
\end{align*}
Thus $T(y)=T(x) \implies y=x$. We can conclude that $T$ is an injective mapping and so the dimension of the null space of $T$ is 0. The dual space is $\mathcal{L} (\R^n, \R)$ it will have a dimension of $n$. It follows from the rank-nullity theorem
that the image of $T$ is also n dimensional. Equivalently, $T$ is a surjective mapping.Therefore $T$ is a bijection between $\R^n$ and $(\R^n)^*$. So for each $\phi \in (\R^n)^*$ there exists a unique $x\in\R^n$ such that $T(x) = \phi_x = \phi$
\end{document}