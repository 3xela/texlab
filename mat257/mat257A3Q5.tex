\documentclass[letterpaper]{article}
\usepackage[letterpaper,margin=1in,footskip=0.25in]{geometry}
\usepackage[utf8]{inputenc}
\usepackage{amsmath}
\usepackage{amsthm}
\usepackage{amssymb, pifont}
\usepackage{mathrsfs}
\usepackage{enumitem}
\usepackage{fancyhdr}
\usepackage{hyperref}

\pagestyle{fancy}
\fancyhf{}
\rhead{MAT 257}
\lhead{Assignment 3}
\rfoot{Page \thepage}

\setlength\parindent{24pt}
\renewcommand\qedsymbol{$\blacksquare$}

\DeclareMathOperator{\R}{\mathbb{R}}
\DeclareMathOperator{\N}{\mathbb{N}}
\DeclareMathOperator{\Z}{\mathbb{Z}}
\DeclareMathOperator{\Q}{\mathbb{Q}}
\DeclareMathOperator{\C}{\mathbb{C}}
\DeclareMathOperator{\ep}{\varepsilon}
\DeclareMathOperator{\identity}{\mathbf{0}}
\DeclareMathOperator{\card}{card}
\newcommand{\suchthat}{;\ifnum\currentgrouptype=16 \middle\fi|;}

\newtheorem{lemma}{Lemma}

\newcommand{\tr}{\mathrm{tr}}
\newcommand{\ra}{\rightarrow}
\newcommand{\lan}{\langle}
\newcommand{\ran}{\rangle}
\newcommand{\norm}[1]{\left\lVert#1\right\rVert}
\newcommand{\inn}[1]{\lan#1\ran}
\newcommand{\ol}{\overline}
\begin{document}
Q5a:\\
We first begin by finding an upper bound for $\frac{\norm{f(h,k)}}{\norm{h,k}}$
\begin{align*}
   & \frac{\norm{f(h,k)}}{\norm{(h,k)}}
   \\ & = \frac{\norm{f(\sum_{i=1}^n h_i e_i,k)}}{\norm{(h,k)}} \text{ (expressing h as the sum of its componoents)}
   \\ & = \frac{\norm{\sum_{i=1}^n h_i f(e_i,k)}}{\norm{(h,k)}} \text{ (by linearity in the first slot)}
   \\ & \leq \frac{\norm{\sum_{i=1}^n h_i f(e_i,k)}}{\norm{h}} \text{ (since norm of (h,k) is at least norm of h)}
   \\ & \leq \frac{\sum_{i=1}^n \norm{h_i f(e_i,k)}}{\norm{h}} \text{ (by triangle inequality)}
   \\ & = \frac{\sum_{i=1}^n |h_i| \norm{f(e_i,k)}}{\norm{h}} 
   \\ & \leq \frac{\sum_{i=1}^n \norm{h} f(e_i,k)}{\norm{h}} \text{ (since $|x_i| \leq \norm{x}$ )}
   \\ & = \sum_{i=1}^n \norm{f(e_i,k)}
\end{align*}
So we have that $$\frac{\norm{f(h,k)}}{\norm{h,k}} \leq \sum_{i=1}^n f(e_i,k)$$ Applying the limit as $(h,k) \rightarrow 0$ to both sides we get that $\lim_{(h,k)\rightarrow 0 } \frac{\norm{f(h,k)}}{\norm{(h,k)}}=0$. $\qed$
\\ 5b:\\
We want to check if in fact $Df(a,b)(x,y) = f(a,y) + f(x,b)$ is the differential of $f$ at $(a,b)$. We can check using Spivak's definition of differntiablilty. That is if $\lim_{h \rightarrow 0 } \frac{\norm{f(a+h)-f(a)-Dfa(h)}}{\norm{h}} = 0$ then $f$ will be differentiable.
\begin{align*}
   & \lim_{(x,y) \rightarrow 0 } \frac{\norm{f(a+x,b+y) - f(a,b) - f(a,y) - f(x,b)}}{\norm{(x,y)}}
   \\ & = \lim_{(x,y) \rightarrow 0} \frac{\norm{f(a,b) + f(a,y) + f(x,b) + f(x,y) - f(a,b) - f(a,y) - f(x,b)}}{\norm{(x,y)}} \text{ (by bilinearity of f)}
   \\ & = \lim _{(x,y) \rightarrow 0 } \frac{\norm{f(x,y)}}{\norm{(x,y)}}
   \\ & = 0 \text{ (by 5a)}
\end{align*} 
Thus, $f$ is differentiable with $Df(a,b)(x,y) = f(a,y) + f(x,b)$
\\5c: Let $p(x,y) = xy$. This is billinear, by the properties of multiplication of real numbers. According to the result from $5b$, $Dp(a,b)(x,y) = p(a,y)  + p(x,b) = ay + bx$, which is exactly what we proved in class. 
\end{document}