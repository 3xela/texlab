\documentclass[letterpaper]{article}
\usepackage[letterpaper,margin=1in,footskip=0.25in]{geometry}
\usepackage[utf8]{inputenc}
\usepackage{amsmath}
\usepackage{amsthm}
\usepackage{amssymb, pifont}
\usepackage{mathrsfs}
\usepackage{enumitem}
\usepackage{fancyhdr}
\usepackage{hyperref}

\pagestyle{fancy}
\fancyhf{}
\rhead{MAT 257}
\lhead{Assignment 13}
\rfoot{Page \thepage}

\setlength\parindent{24pt}
\renewcommand\qedsymbol{$\blacksquare$}

\DeclareMathOperator{\T}{\mathcal{T}}
\DeclareMathOperator{\V}{\mathcal{V}}
\DeclareMathOperator{\U}{\mathcal{U}}
\DeclareMathOperator{\Prt}{\mathbb{P}}
\DeclareMathOperator{\R}{\mathbb{R}}
\DeclareMathOperator{\N}{\mathbb{N}}
\DeclareMathOperator{\Z}{\mathbb{Z}}
\DeclareMathOperator{\Q}{\mathbb{Q}}
\DeclareMathOperator{\C}{\mathbb{C}}
\DeclareMathOperator{\ep}{\varepsilon}
\DeclareMathOperator{\identity}{\mathbf{0}}
\DeclareMathOperator{\card}{card}
\newcommand{\suchthat}{;\ifnum\currentgrouptype=16 \middle\fi|;}

\newtheorem{lemma}{Lemma}

\newcommand{\tr}{\mathrm{tr}}
\newcommand{\ra}{\rightarrow}
\newcommand{\lan}{\langle}
\newcommand{\ran}{\rangle}
\newcommand{\norm}[1]{\left\lVert#1\right\rVert}
\newcommand{\inn}[1]{\lan#1\ran}
\newcommand{\ol}{\overline}
\begin{document}
\noindent Q4a: For $I\in \underline{n}_{a}^k$ define $I^c\in \underline{n}_{a}^{n-k}$ to be the ascending list of $n-k$ elements such that $I^c \cap I = \emptyset$. It is sufficient to define $\star$ on $\omega_I$ and define it to be linear. Let $\omega_I\in \Lambda^k (\R^n)$; we define $\star \omega_{I} = (-1)^{\sigma(I\cup I^c)} \omega_{I^c}$, with $\star(\alpha \lambda + \eta ) = \alpha \star\lambda + \star \eta$, for all $\alpha\in \R$. 
We can verify that indeed, for some $\omega_I,\omega_J\in \underline{n}_{a}^k$; 
$$ \omega_I \wedge \star \omega_J = \omega_I \wedge (-1)^{\sigma(J \cup J^c)} \omega_{J^c} = \delta_{IJ}\omega_n = \inn{\omega_I,\omega_J}\omega_n$$
Where the second equality holding because when $I=J$, $I \cup J^c = \{1,2\dots n \}$. The $(-1)^{\sigma(J\cup J^c)}$ term takes care of sign swaps occuring when we rearrange each $\varphi_{i_k}, \varphi_{j_k}$ used to construct $\omega_I$ and $\omega_J$. Additionally, take note that if $I\neq J$, then $I\cap J^c\neq \emptyset$ and the following happens. Assume that $I= \{ i_1 , \dots i_k\}$ and $J^c = \{ j_1 , \dots j_{n-k}\} $. At some indices, $i_\alpha = j_\beta$ and so 
\begin{align*} \omega_I \wedge (-1)^{\sigma(J \cup J^c)}\omega_{J^c} & = (-1)^{\sigma(J \cup J^c)} \varphi_{i_1} \wedge \cdots \varphi_{i_{\alpha}} \cdots \wedge \varphi_{i_k} \wedge \varphi_{j_i} \wedge \cdots \varphi_{j_\beta} \cdots \wedge \varphi_{j_{n-k}}
    \\ &  = (-1)^{\sigma(J \cup J^c)+1} \varphi_{i_1} \wedge \cdots \varphi_{{j_\beta}} \cdots \wedge \varphi_{i_k} \wedge \varphi_{j_i} \wedge \cdots \varphi_{i_\alpha} \cdots \wedge \varphi_{j_{n-k}} \tag{swapping the equal $\varphi$}
    \\ & = 0 \tag{since sign changes but the value does not}
\end{align*}
We now claim uniqueness of $\star$. Suppose there is $\star_1$, $\star_2$ which both satisfy $\lambda \wedge \star \eta = \inn{\lambda \eta}\omega_n$. Then we have that for any $\lambda\in \Lambda^k(\R^n)$
$$\lambda \wedge \star_1 \eta -\star_2 \eta = \lambda \wedge \star_1 \eta - \lambda \wedge \star_2 \eta = \inn{\lambda, \eta}\omega_n  - \inn{\lambda, \eta} \omega_n=0$$ Taking $\lambda = \star_1( \star_1 \eta - \star_2 \eta)$ 
$$0 = \star_1 ( \star_1 \eta - \star_2 \eta) \wedge ( \star_1 \eta - \star_2 \eta)  = (-1)^{(n-k)^2} ( \star_1 \eta - \star_2 \eta) \wedge \star_1( \star_1 \eta - \star_2 \eta) =  \inn{( \star_1 \eta - \star_2 \eta) ,( \star_1 \eta - \star_2 \eta) } \omega_n$$
By the properties of the inner product, $( \star_1 \eta - \star_2 \eta) =0$ or equivalently $ \star_1 \eta = \star_2 \eta$. Hence the $\star$ operation is unique. 
\newline \\ Q4b: using the formula for $\star \omega_I$ in Q4a, for $\omega_I\in \Lambda^1 (\R^3),$ we compute the following. $$\star \omega_1 = \omega_2\wedge \omega_3, \star \omega_2 = -\omega_1\wedge \omega_3, \star \omega_3 = \omega_1\wedge \omega_2$$
Similarly, when $n=4$ and $k=2$, using our definition of $\star$, 
$$\star \omega_{12} = \omega_3 \wedge \omega_4, \star \omega_{13} = -\omega_2 \wedge \omega_4, \star \omega_{14} =\omega_2 \wedge \omega_3 , \star \omega_{23} = \omega_1 \wedge \omega_4 , \star \omega_{24} =-\omega_1 \wedge \omega_3 , \star\omega_{34} = \omega_1\wedge \omega_2$$
\newline \\ Q4c: It is sufficient to show that $\star \circ \star $ applied to some basis element of $\Lambda^k (V)$ is scaled by the desired constant. 
Let $I\in \underline{n}_a^k$, $I=\{ i_1, \dots ,i_k \} $. Then we see that 
$$\star \circ \star (\omega_I) = \star (-1)^{\sigma(I \cup I^c)} \omega_{I^c} = (-1)^{\sigma(I^c\cup I)} \cdot(-1)^{\sigma(I \cup I^c)} \omega_I = (-1)^{(k)(n-k)}\omega_I$$
Where the last equality holds since by applying the $\star$ operation twice, we make $k(n-k)$ swaps of the constituent $\omega_i$. 
\end{document}