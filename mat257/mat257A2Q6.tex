\documentclass[letterpaper]{article}
\usepackage[letterpaper,margin=1in,footskip=0.25in]{geometry}
\usepackage[utf8]{inputenc}
\usepackage{amsmath}
\usepackage{amsthm}
\usepackage{amssymb, pifont}
\usepackage{mathrsfs}
\usepackage{enumitem}
\usepackage{fancyhdr}
\usepackage{hyperref}

\pagestyle{fancy}
\fancyhf{}
\rhead{MAT 257}
\lhead{Assignment 2}
\rfoot{Page \thepage}

\setlength\parindent{24pt}
\renewcommand\qedsymbol{$\blacksquare$}

\DeclareMathOperator{\R}{\mathbb{R}}
\DeclareMathOperator{\N}{\mathbb{N}}
\DeclareMathOperator{\Z}{\mathbb{Z}}
\DeclareMathOperator{\Q}{\mathbb{Q}}
\DeclareMathOperator{\C}{\mathbb{C}}
\DeclareMathOperator{\ep}{\varepsilon}
\DeclareMathOperator{\identity}{\mathbf{0}}
\DeclareMathOperator{\card}{card}
\newcommand{\suchthat}{;\ifnum\currentgrouptype=16 \middle\fi|;}

\newtheorem{lemma}{Lemma}

\newcommand{\tr}{\mathrm{tr}}
\newcommand{\ra}{\rightarrow}
\newcommand{\lan}{\langle}
\newcommand{\ran}{\rangle}
\newcommand{\norm}[1]{\left\lVert#1\right\rVert}
\newcommand{\inn}[1]{\lan#1\ran}
\newcommand{\ol}{\overline}
\begin{document}
Q6:
\\
Let $A\subset \R^n$, not closed. Let $x$ be a point such that $x \in \R^n \setminus A$ and $x\notin \text{int}\R^n \setminus A$. Consider $f(y) = \frac{1}{\norm{y-x}}$. We want to show that this function is unbounded and continuous.
We being by showing that $f$ is continuous. Notice that if $g(y) = \norm{x-y}$ and $h(z)= \frac{1}{z}$, then $f = h \circ g$. It is sufficient to show that $g$ is continous and never 0. We will use the following fact to prove that $g$ is continous
\\ Claim: $|\norm{x}-\norm{y}| \leq \norm{x-y}$
\\ pf: since both quantities are positive we can square them
\begin{align*}
    & (\norm{x}-\norm{y})^2 \leq \norm{x-y}^2
    \\ & \iff \norm{x}^2 - 2\norm{x}\norm{y} + \norm{y}^2 \leq \inn{x,x} -2 \inn{x,y} + \inn{y,y}
    \\ & \iff -2 \norm{x} \norm{y} \leq -2 \inn{x,y}
    \\ & \iff \inn{x,y} \leq \norm{x} \norm{y} \text{ which is true by cauchy- shwartz} \qed
\end{align*}
Now we show that $g$ is continous. Let $\epsilon >0$. Let $\delta = \epsilon$. Take $y,z \in A$. Then, 
\begin{align*}
    & \norm{z-y} < \epsilon 
    \\ & \implies \norm{z-x-y+x} < \epsilon
    \\ & \implies \norm{(z-x) - (y-x)} < \epsilon
    \\ & \implies |\norm{z-x}- \norm{y-x}| \leq \norm{(z-x)-(y-x)} < \epsilon \text{ (by claim )}
\end{align*} Therefore $g$ is continous. Now $h$ will be continous since $g$ will never be 0, since $g(y)=0 \iff \norm{y-x} = 0 \iff y=x$, but $x$ is not in $A$. As the composition of two continous functions, $f$ is continous as well. 
 Now we show that $f$ is unbounded. First, note that the point $x$ must be in the boundary of $A$, since it is not in the exteriour of $A$, and is not in $A$ so it can not be in the interiour. 
Therefore, for all $\epsilon >0$, $B_{\epsilon}(x)$ will contain at least one point $z \in A$. Choose $\epsilon > 0$. Suppose that this $f$ is bounded. There must exist some $M$ with $f(y) \leq |M|$ for all $y \in A$. So we see that 
\begin{align*}
    & f(y) \leq |M|
    \\ & \iff \frac{1}{\norm{y-x}} \leq |M|
    \\ & \iff \frac{1}{|M|} \leq \norm{y-x}, \text{ for all } y \in A
\end{align*}
However, we can choose any $\epsilon>0$ and find a $y\in A$ where $\norm{y-x} < \epsilon$. Choosing $\epsilon = \frac{2}{|M|}$ implies that $\frac{1}{|M|} \leq \norm{y-x} < \frac{2}{|M|}$. This is a contradiction, since no such positive number M exists where $\frac{1}{|M|} < \frac{2}{|M|}$. Thus $f$ is not bounded on $A$.$\qed$
\end{document}