\documentclass[letterpaper]{article}
\usepackage[letterpaper,margin=1in,footskip=0.25in]{geometry}
\usepackage[utf8]{inputenc}
\usepackage{amsmath}
\usepackage{amsthm}
\usepackage{amssymb, pifont}
\usepackage{mathrsfs}
\usepackage{enumitem}
\usepackage{fancyhdr}
\usepackage{hyperref}

\pagestyle{fancy}
\fancyhf{}
\rhead{MAT 257}
\lhead{Assignment 2}
\rfoot{Page \thepage}

\setlength\parindent{24pt}
\renewcommand\qedsymbol{$\blacksquare$}

\DeclareMathOperator{\R}{\mathbb{R}}
\DeclareMathOperator{\N}{\mathbb{N}}
\DeclareMathOperator{\Z}{\mathbb{Z}}
\DeclareMathOperator{\Q}{\mathbb{Q}}
\DeclareMathOperator{\C}{\mathbb{C}}
\DeclareMathOperator{\ep}{\varepsilon}
\DeclareMathOperator{\identity}{\mathbf{0}}
\DeclareMathOperator{\card}{card}
\newcommand{\suchthat}{;\ifnum\currentgrouptype=16 \middle\fi|;}

\newtheorem{lemma}{Lemma}

\newcommand{\tr}{\mathrm{tr}}
\newcommand{\ra}{\rightarrow}
\newcommand{\lan}{\langle}
\newcommand{\ran}{\rangle}
\newcommand{\norm}[1]{\left\lVert#1\right\rVert}
\newcommand{\inn}[1]{\lan#1\ran}
\newcommand{\ol}{\overline}
\begin{document}
Q3:\\
We first claim that $f(0)=0$. 
\begin{align*}
    & \norm{f(0)} \leq \norm{0}^2
    \\ & \implies \norm{f(0)} \leq 0
    \\ & \implies \norm{f(0)} = 0
    \\ & \implies f(0) = 0
\end{align*}
We not show that $f \in o(h)$ Now consider $f(h)$ for some $h\in \R^n$. 
\begin{align*}
    & \norm{f(h)} \leq \norm{h}^2
    \\ & \implies \frac{\norm{f(h)}}{\norm{h}} \leq \norm{h}
    \\ & \implies \lim_{h \rightarrow 0} \frac{\norm{f(h)}}{\norm{h}} \leq \lim_{h\rightarrow 0} \norm{h}
    \\ & \implies \lim_{h \rightarrow 0 } \frac{\norm{f(h)}}{\norm{h}} = 0
\end{align*}
Thus we can write $f(h) = f(0) + Lh + o(h)$ for some linear mapping $L$. Since $f\in o(h)$, this implies that $L\in o(h)$. By the lemma from class any linear mapping in $o(h)$ is 0. And so $f$ is differentiable with $Df(0)=0$. 
\end{document}