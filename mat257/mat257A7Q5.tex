\documentclass[letterpaper]{article}
\usepackage[letterpaper,margin=1in,footskip=0.25in]{geometry}
\usepackage[utf8]{inputenc}
\usepackage{amsmath}
\usepackage{amsthm}
\usepackage{amssymb, pifont}
\usepackage{mathrsfs}
\usepackage{enumitem}
\usepackage{fancyhdr}
\usepackage{hyperref}

\pagestyle{fancy}
\fancyhf{}
\rhead{MAT 257}
\lhead{Assignment 6}
\rfoot{Page \thepage}

\setlength\parindent{24pt}
\renewcommand\qedsymbol{$\blacksquare$}

\DeclareMathOperator{\Prt}{\mathbb{P}}
\DeclareMathOperator{\R}{\mathbb{R}}
\DeclareMathOperator{\N}{\mathbb{N}}
\DeclareMathOperator{\Z}{\mathbb{Z}}
\DeclareMathOperator{\Q}{\mathbb{Q}}
\DeclareMathOperator{\C}{\mathbb{C}}
\DeclareMathOperator{\ep}{\varepsilon}
\DeclareMathOperator{\identity}{\mathbf{0}}
\DeclareMathOperator{\card}{card}
\newcommand{\suchthat}{;\ifnum\currentgrouptype=16 \middle\fi|;}

\newtheorem{lemma}{Lemma}

\newcommand{\tr}{\mathrm{tr}}
\newcommand{\ra}{\rightarrow}
\newcommand{\lan}{\langle}
\newcommand{\ran}{\rangle}
\newcommand{\norm}[1]{\left\lVert#1\right\rVert}
\newcommand{\inn}[1]{\lan#1\ran}
\newcommand{\ol}{\overline}
\begin{document}
Q5a:\\
Suppose that $A$ is a unbounded set of content 0. Then for any $\varepsilon>0$ we can find a finite collection of closed rectangles $U_1\dots U_n$ such that $\bigcup_{i=1}^n U_i \supset A$ and $\sum_{i=1}^n vol(U_i) < \varepsilon$
. Since we are dealing with a finite collection of closed rectangles, we can choose a closed rectangle $W$ such that each $U_i \subset W$. This is the same as saying $A \subset \bigcup_{i=1}^n U_i \subset W$. Therefore $A$ is bounded, a contradiction. 
\\ 5b: \\
Consider the set of all integers as a subset of $\R$. This is closed, since $\R \setminus \Z = \bigcup_{i\in \Z} (i, i+1)$ which is the arbitrary union of open sets hence is open. Since $\Z$ is countable, we can rewrite $\Z$ as the countable union of each of its elements. Since a point is of measure 0, Spivak Theorem 3-4 says a countable union of measure 0 sets is of measure 0 as well. Since $\Z$ is unbounded, we have by 5a that it can not be of content 0. Thus $\Z$ viewed as a subset of $\R$ is closed and of measure 0 but not content 0. 
\end{document}