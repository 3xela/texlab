\documentclass[letterpaper]{article}
\usepackage[letterpaper,margin=1in,footskip=0.25in]{geometry}
\usepackage[utf8]{inputenc}
\usepackage{amsmath}
\usepackage{amsthm}
\usepackage{amssymb, pifont}
\usepackage{mathrsfs}
\usepackage{enumitem}
\usepackage{fancyhdr}
\usepackage{hyperref}

\pagestyle{fancy}
\fancyhf{}
\rhead{MAT 257}
\lhead{Assignment 5}
\rfoot{Page \thepage}

\setlength\parindent{24pt}
\renewcommand\qedsymbol{$\blacksquare$}

\DeclareMathOperator{\R}{\mathbb{R}}
\DeclareMathOperator{\N}{\mathbb{N}}
\DeclareMathOperator{\Z}{\mathbb{Z}}
\DeclareMathOperator{\Q}{\mathbb{Q}}
\DeclareMathOperator{\C}{\mathbb{C}}
\DeclareMathOperator{\ep}{\varepsilon}
\DeclareMathOperator{\identity}{\mathbf{0}}
\DeclareMathOperator{\card}{card}
\newcommand{\suchthat}{;\ifnum\currentgrouptype=16 \middle\fi|;}

\newtheorem{lemma}{Lemma}

\newcommand{\tr}{\mathrm{tr}}
\newcommand{\ra}{\rightarrow}
\newcommand{\lan}{\langle}
\newcommand{\ran}{\rangle}
\newcommand{\norm}[1]{\left\lVert#1\right\rVert}
\newcommand{\inn}[1]{\lan#1\ran}
\newcommand{\ol}{\overline}
\begin{document}
Q1: \\
Since $f$ is injective it has an inverse $f^{-1}$ on $f(A)$. Therefore, for all $b\in f(A)$ there exists a unique $a\in A$ where $f(a) = b$. 
We now apply the inverse function theorem. For each $y\in f(A)$, there is an open set $B_y$, and some open set $A_x$ around $f^{-1}(y)=x$ where there exists a $C^1$ inverse $f^{-1}_x: B_y \rightarrow A_x$. From the 
uniqueness of the inverse, we have that $\forall w\in B_y$. $f_x^{-1}(w)=f^{-1}(w)$. This is true for all $w\in F(A)$, so $f^{-1}$ is continous and differentiable. We can write $f(A) = (f^{-1})^{-1} (A)$. Since $f^{-1}$ is continous, and the pre image of open sets under continous maps is open, it follows that $A$ is open. Similarly for any open $B\subset A$, $f(B) = (f^{-1})^{-1}=B$. 
\end{document}