\documentclass[letterpaper]{article}
\usepackage[letterpaper,margin=1in,footskip=0.25in]{geometry}
\usepackage[utf8]{inputenc}
\usepackage{amsmath}
\usepackage{amsthm}
\usepackage{amssymb, pifont}
\usepackage{mathrsfs}
\usepackage{enumitem}
\usepackage{fancyhdr}
\usepackage{hyperref}

\pagestyle{fancy}
\fancyhf{}
\rhead{MAT 257}
\lhead{Assignment 5}
\rfoot{Page \thepage}

\setlength\parindent{24pt}
\renewcommand\qedsymbol{$\blacksquare$}

\DeclareMathOperator{\R}{\mathbb{R}}
\DeclareMathOperator{\N}{\mathbb{N}}
\DeclareMathOperator{\Z}{\mathbb{Z}}
\DeclareMathOperator{\Q}{\mathbb{Q}}
\DeclareMathOperator{\C}{\mathbb{C}}
\DeclareMathOperator{\ep}{\varepsilon}
\DeclareMathOperator{\identity}{\mathbf{0}}
\DeclareMathOperator{\card}{card}
\newcommand{\suchthat}{;\ifnum\currentgrouptype=16 \middle\fi|;}

\newtheorem{lemma}{Lemma}

\newcommand{\tr}{\mathrm{tr}}
\newcommand{\ra}{\rightarrow}
\newcommand{\lan}{\langle}
\newcommand{\ran}{\rangle}
\newcommand{\norm}[1]{\left\lVert#1\right\rVert}
\newcommand{\inn}[1]{\lan#1\ran}
\newcommand{\ol}{\overline}
\begin{document}
Q2a:\\
Suppose not, that is suppose that $f$ is injective. Then we have that $D_1f$ and $D_2f$ are both not identically $0$. $WLOG$ say that $D_1f$ is nonzero on some open set $A$. 
Define $g(x,y) = (f(x,y),y)$. Then by properties of the differential, $$Dg(x,y) = \begin{pmatrix}
 D_1f(x,y)   & D_2f(x,y) \\ 0 & 1
\end{pmatrix}$$ We have that $Det Dg = D_1f(x,y)$ which is nonzero on $A$. Notice that our choice for $g$ makes it injective.  Thus by the inverse function theorem, there exists some open set $A$ around the point $(x_0,y_0)$ and open $B$ around $g(x,y)$ such that $g^{-1}:B \rightarrow A$ exists and is both continous and differentiable. 
Consider distinct points $(\alpha,y_0), (\alpha,y_1)\in B$. Since $g^{-1}$ is injective we have that there is some distinct $(x_0,y_0)$ and $(x_1,y_1)$ where $f(x_0,y_0)= f(x_1,y_1) =\alpha $. We obtain a contradiction and thus $f$ is not injective. 

\end{document}