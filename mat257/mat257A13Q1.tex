\documentclass[letterpaper]{article}
\usepackage[letterpaper,margin=1in,footskip=0.25in]{geometry}
\usepackage[utf8]{inputenc}
\usepackage{amsmath}
\usepackage{amsthm}
\usepackage{amssymb, pifont}
\usepackage{mathrsfs}
\usepackage{enumitem}
\usepackage{fancyhdr}
\usepackage{hyperref}

\pagestyle{fancy}
\fancyhf{}
\rhead{MAT 257}
\lhead{Assignment 13}
\rfoot{Page \thepage}

\setlength\parindent{24pt}
\renewcommand\qedsymbol{$\blacksquare$}

\DeclareMathOperator{\T}{\mathcal{T}}
\DeclareMathOperator{\V}{\mathcal{V}}
\DeclareMathOperator{\U}{\mathcal{U}}
\DeclareMathOperator{\Prt}{\mathbb{P}}
\DeclareMathOperator{\R}{\mathbb{R}}
\DeclareMathOperator{\N}{\mathbb{N}}
\DeclareMathOperator{\Z}{\mathbb{Z}}
\DeclareMathOperator{\Q}{\mathbb{Q}}
\DeclareMathOperator{\C}{\mathbb{C}}
\DeclareMathOperator{\ep}{\varepsilon}
\DeclareMathOperator{\identity}{\mathbf{0}}
\DeclareMathOperator{\card}{card}
\newcommand{\suchthat}{;\ifnum\currentgrouptype=16 \middle\fi|;}

\newtheorem{lemma}{Lemma}

\newcommand{\tr}{\mathrm{tr}}
\newcommand{\ra}{\rightarrow}
\newcommand{\lan}{\langle}
\newcommand{\ran}{\rangle}
\newcommand{\norm}[1]{\left\lVert#1\right\rVert}
\newcommand{\inn}[1]{\lan#1\ran}
\newcommand{\ol}{\overline}
\begin{document}
\noindent Q1: If $v_1,v_2.v_3$ is a basis for $\R^3$, we let $\omega_1,\omega_2,\omega_3$ be a basis for $\Lambda^1 \R^3$ and let $\omega_1 \wedge \omega_2, \omega_2 \wedge \omega_3, \omega_1 \wedge \omega_3$ be a basis for $\Lambda^2 \R^3$.
We can identify $\Lambda^1 \R^3$ with $\R^3$ by viewing $v_i$ as $\omega_i$. Similarly, we can identify $\Lambda^2 \R^3$ with $\R^3$ by $v_1 \to \omega_1 \wedge \omega_2, v_2 \to \omega_2\wedge \omega_3, v_3 \to \omega_1 \wedge \omega_3$. We claim that the usual $\wedge$ product on $\Lambda^1\R^3 \times \Lambda^1\R^3 \to \Lambda^2 \R^3$ can be viewed as a vector product operation on $\R^3\times \R^3$. 
Let $\lambda = \alpha_1 \omega_1 + \alpha_2 \omega_2 + \alpha_3 \omega_3$ and let $\eta  = \beta_1 \omega_1 + \beta_2 \omega_2 + \beta_3 \omega_3$. We evaluate $\lambda \wedge \eta$ in the usual way; 
\begin{align*}
    \lambda \wedge \eta & = (\alpha_1 \omega_1 + \alpha_2 \omega_2 + \alpha_3 \omega_3)\wedge (\beta_1 \omega_1 + \beta_2 \omega_2 + \beta_3 \omega_3)
    \\ & = \alpha_1 \omega_1 \wedge (\beta_1 \omega_1 + \beta_2 \omega_2 + \beta_3 \omega_3) + \alpha_2 \omega_2 \wedge \beta_1 \omega_1 + \beta_2 \omega_2 + \beta_3 \omega_3 + \alpha_3 \omega_3 \wedge \beta_1 \omega_1 + \beta_2 \omega_2 + \beta_3 \omega_3
    \\ & = \alpha_1 \omega_1 \wedge \beta_1 \omega_1 + \alpha_1 \omega_1 \wedge \beta_2 \omega_2 + \alpha_1 \omega_1 \wedge \beta_3 \omega_3 + \alpha_2 \omega_2 \wedge \beta_1 \omega_1 + \alpha_2 \omega_2 \wedge \beta_2 \omega_2 + \\ &  \alpha_2 \omega_2 \wedge \beta_3 \omega_3 + \alpha_3 \omega_3 \wedge \beta_1 \omega_1 + \alpha_3 \omega_3 \wedge \beta_2 \omega_2 + \alpha_3 \omega_3 \wedge \beta_3 \omega_3
    \\ & = \alpha_1 \omega_1 \wedge \beta_2 \omega_2 + \alpha_1 \omega_1 \wedge \beta_3 \omega_3 -\beta_1 \omega_1 \wedge \alpha_2 \omega_2 + \alpha_2 \omega_2 \wedge \beta_3 \omega_3 - \beta_1 \omega_1 \wedge \alpha_3 \omega_3 - \beta_2 \omega_2 \wedge \alpha_3 \omega_3 \tag{by supercommutativity}
    \\ & = (\alpha_1\beta_2-\beta_1\alpha_2)\omega_1\wedge \omega_2 + (\alpha_2\beta_3 - \beta_2 \alpha_3)\omega_2 \wedge \omega_3 + (\alpha_1\beta_3 - \beta_1 \alpha_3) \omega_1 \wedge \omega_3 \tag{by linearity of $\wedge$}
\end{align*} 
Therefore, we have a product operation on vectors in $\R^3$ defined in the following way: $$(\alpha_1,\alpha_2,\alpha_3)\times (\beta_1,\beta_2,\beta_3) \mapsto (\alpha_1\beta_2-\alpha_2\beta_1, \alpha_2\beta_3-\alpha_3\beta_2,\alpha_1\beta_3-\alpha_3\beta_1) $$
Note this product inherits properties of the wedge product. It is bilinear, alternating, supercommutative. 
\end{document}