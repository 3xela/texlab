\documentclass[letterpaper]{article}
\usepackage[letterpaper,margin=1in,footskip=0.25in]{geometry}
\usepackage[utf8]{inputenc}
\usepackage{amsmath}
\usepackage{amsthm}
\usepackage{amssymb, pifont}
\usepackage{mathrsfs}
\usepackage{enumitem}
\usepackage{fancyhdr}
\usepackage{hyperref}

\pagestyle{fancy}
\fancyhf{}
\rhead{MAT 257}
\lhead{Assignment 14}
\rfoot{Page \thepage}

\setlength\parindent{24pt}
\renewcommand\qedsymbol{$\blacksquare$}

\DeclareMathOperator{\T}{\mathcal{T}}
\DeclareMathOperator{\V}{\mathcal{V}}
\DeclareMathOperator{\U}{\mathcal{U}}
\DeclareMathOperator{\Prt}{\mathbb{P}}
\DeclareMathOperator{\R}{\mathbb{R}}
\DeclareMathOperator{\N}{\mathbb{N}}
\DeclareMathOperator{\Z}{\mathbb{Z}}
\DeclareMathOperator{\Q}{\mathbb{Q}}
\DeclareMathOperator{\C}{\mathbb{C}}
\DeclareMathOperator{\ep}{\varepsilon}
\DeclareMathOperator{\identity}{\mathbf{0}}
\DeclareMathOperator{\card}{card}
\newcommand{\suchthat}{;\ifnum\currentgrouptype=16 \middle\fi|;}

\newtheorem{lemma}{Lemma}

\newcommand{\tr}{\mathrm{tr}}
\newcommand{\ra}{\rightarrow}
\newcommand{\lan}{\langle}
\newcommand{\ran}{\rangle}
\newcommand{\norm}[1]{\left\lVert#1\right\rVert}
\newcommand{\inn}[1]{\lan#1\ran}
\newcommand{\ol}{\overline}
\begin{document}
\noindent Q4: Let $(p,v)\in T_p\R^n$. We compute $D_{(p,v)}f$ as 
\begin{align*}
D_{(p,v)}f & = \frac{\partial f}{\partial x_1}(p) \cdot v_1 + \dots + \frac{\partial f}{\partial x_n}(p) \cdot v_n
\\ & = \inn{(\frac{\partial f}{\partial x_1}(p), \dots \frac{\partial f}{\partial x_n}(p)),(v_1,\dots ,v_n)} \tag{definition of inner product}
\\ & = \inn{(p,\frac{\partial f}{\partial x_1}(p), \dots \frac{\partial f}{\partial x_n}(p)),(p,v)} \tag{by definition of inner product on a tangent space}
\\ & = \inn{(grad f)(p),(p,v)} \tag{by definition of grad f}
\end{align*} By the Cauchy-Schwartz inequality, this quantity is maximized when $(p,v)$ is collinear with $grad f(p)$, hence it represents the growth of the function in the direction of $p$. 
\end{document}