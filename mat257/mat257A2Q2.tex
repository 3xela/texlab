\documentclass[letterpaper]{article}
\usepackage[letterpaper,margin=1in,footskip=0.25in]{geometry}
\usepackage[utf8]{inputenc}
\usepackage{amsmath}
\usepackage{amsthm}
\usepackage{amssymb, pifont}
\usepackage{mathrsfs}
\usepackage{enumitem}
\usepackage{fancyhdr}
\usepackage{hyperref}

\pagestyle{fancy}
\fancyhf{}
\rhead{MAT 257}
\lhead{Assignment 2}
\rfoot{Page \thepage}

\setlength\parindent{24pt}
\renewcommand\qedsymbol{$\blacksquare$}

\DeclareMathOperator{\R}{\mathbb{R}}
\DeclareMathOperator{\N}{\mathbb{N}}
\DeclareMathOperator{\Z}{\mathbb{Z}}
\DeclareMathOperator{\Q}{\mathbb{Q}}
\DeclareMathOperator{\C}{\mathbb{C}}
\DeclareMathOperator{\ep}{\varepsilon}
\DeclareMathOperator{\identity}{\mathbf{0}}
\DeclareMathOperator{\card}{card}
\newcommand{\suchthat}{;\ifnum\currentgrouptype=16 \middle\fi|;}

\newtheorem{lemma}{Lemma}

\newcommand{\tr}{\mathrm{tr}}
\newcommand{\ra}{\rightarrow}
\newcommand{\lan}{\langle}
\newcommand{\ran}{\rangle}
\newcommand{\norm}[1]{\left\lVert#1\right\rVert}
\newcommand{\inn}[1]{\lan#1\ran}
\newcommand{\ol}{\overline}
\begin{document}
Q2a: \\
Suppose not. That is, assume that for all $d >0$ and $x \in A^c $ and $y \in A$, $\norm{x-y} < d$.
 Choose $B = B_{d}(x)$.  We notice that $R$ will always contain a point in $A$, namely $y$, since it will be within $d$ of $x$. $x$ is not part of $A$ yet every open ball centered at it will contain at least $y$ and itself. Therefore, $x\in \text{bd} A$. Since $A$ is closed, it must be that $\text{bd}A \subset A$. Therefore, $x\in A$. We obtain a contradiction since we supposed $x \in A^c$. 
\\
 2b:
 \\
 By compactness of $B$, for all open covers $O$, there exists a finite subcover $U_1 \dots U_n$. From 2a, we know for each point,$x$ in $A^c$, there exists some $d_x$ such that $\norm{x-a} \geq d_x \forall a\in A$. For each $x\in U_i \cap B$ we define $d_i$ as follows: $d_i = inf \{d_x : x \in B \cap U_i \}$. This is well defined and exists, since each $U_i$ was chosen to contain at least 1 point in $B$. This set is also bounded from below, since each $d_x >0$.We choose $d= min\{d_1 \dots d_n\}$. From our choice of $d$, we get that for all $x\in A, y\in B , \norm{x-y}\geq d$.
 \\
 2C:
\\
 Consider $A= \{(x,y) \in R^2 :x\in \R,  e^x \leq y \}$ and $B= \{ (x,y)\in \R^2: x\in \R , y \leq -e^x\}$. Both of these sets are closed, yet the distance between them can be made arbitrarily small. Take for example at some point $x_0$. The smallest distance between the two sets will be $\norm{e^{x_0}-(-e^{x_0})} = 2e^{x_0}$, since as $x_0$ approaches $-\infty$, $2e^{x_0}$ approaches $0$.
\end{document}