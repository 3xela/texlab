\documentclass[letterpaper]{article}
\usepackage[letterpaper,margin=1in,footskip=0.25in]{geometry}
\usepackage[utf8]{inputenc}
\usepackage{amsmath}
\usepackage{amsthm}
\usepackage{amssymb, pifont}
\usepackage{mathrsfs}
\usepackage{enumitem}
\usepackage{fancyhdr}
\usepackage{hyperref}

\pagestyle{fancy}
\fancyhf{}
\rhead{MAT 257}
\lhead{Assignment 12}
\rfoot{Page \thepage}

\setlength\parindent{24pt}
\renewcommand\qedsymbol{$\blacksquare$}

\DeclareMathOperator{\T}{\mathcal{T}}
\DeclareMathOperator{\V}{\mathcal{V}}
\DeclareMathOperator{\U}{\mathcal{U}}
\DeclareMathOperator{\Prt}{\mathbb{P}}
\DeclareMathOperator{\R}{\mathbb{R}}
\DeclareMathOperator{\N}{\mathbb{N}}
\DeclareMathOperator{\Z}{\mathbb{Z}}
\DeclareMathOperator{\Q}{\mathbb{Q}}
\DeclareMathOperator{\C}{\mathbb{C}}
\DeclareMathOperator{\ep}{\varepsilon}
\DeclareMathOperator{\identity}{\mathbf{0}}
\DeclareMathOperator{\card}{card}
\newcommand{\suchthat}{;\ifnum\currentgrouptype=16 \middle\fi|;}

\newtheorem{lemma}{Lemma}

\newcommand{\tr}{\mathrm{tr}}
\newcommand{\ra}{\rightarrow}
\newcommand{\lan}{\langle}
\newcommand{\ran}{\rangle}
\newcommand{\norm}[1]{\left\lVert#1\right\rVert}
\newcommand{\inn}[1]{\lan#1\ran}
\newcommand{\ol}{\overline}
\begin{document}
Q4: We define $\underline{n}_s^k = \{ (i_1, \dots i_k): 1\leq i_1 \leq \dots \geq i_k \leq n \}$. Define $\sigma_{I} = \sum_{\sigma\in S_k} \varphi_{I} \circ \sigma^{*}$. We make the claim that $\sigma_I$ is a basis for $S^k(V)$.
We will first show that indeed $\sigma_I\in S^k(V)$. It will definitely be k-linear, since it is the sum of k-linear maps. It is enough to show that it is symmetric on some list of vectors $u_1\dots u_k$. We let $\tau\in S_k$. We evaluate: 
\begin{align*}
    \sigma_I \circ \tau (u_1 , \dots , u_k) & = \sum_{\sigma\in S_k} \varphi_I \circ \sigma^{*} (u_{\tau(1)} , \dots ,  u_{\tau(k)})
    \\ & = \sum_{\sigma \in S_k} \varphi_I (u_{\sigma(\tau(1))} , \dots u_{\sigma(\tau(k))})
    \\ & = \sum_{\lambda \in S_k} \varphi_I \circ \lambda^{*} ( u_1 , \dots , u_k) \tag{since for fixed  $\tau, \sigma\circ \tau$ ranges over $S_k$ }
    \\ & = \sigma_{I}(u_1, \dots u_k)
\end{align*} 
\end{document}