\documentclass[letterpaper]{article}
\usepackage[letterpaper,margin=1in,footskip=0.25in]{geometry}
\usepackage[utf8]{inputenc}
\usepackage{amsmath}
\usepackage{amsthm}
\usepackage{amssymb, pifont}
\usepackage{mathrsfs}
\usepackage{enumitem}
\usepackage{fancyhdr}
\usepackage{hyperref}

\pagestyle{fancy}
\fancyhf{}
\rhead{MAT 257}
\lhead{Assignment 17}
\rfoot{Page \thepage}

\setlength\parindent{24pt}
\renewcommand\qedsymbol{$\blacksquare$}

\DeclareMathOperator{\T}{\mathcal{T}}
\DeclareMathOperator{\V}{\mathcal{V}}
\DeclareMathOperator{\U}{\mathcal{U}}
\DeclareMathOperator{\Prt}{\mathbb{P}}
\DeclareMathOperator{\R}{\mathbb{R}}
\DeclareMathOperator{\N}{\mathbb{N}}
\DeclareMathOperator{\Z}{\mathbb{Z}}
\DeclareMathOperator{\Q}{\mathbb{Q}}
\DeclareMathOperator{\C}{\mathbb{C}}
\DeclareMathOperator{\ep}{\varepsilon}
\DeclareMathOperator{\identity}{\mathbf{0}}
\DeclareMathOperator{\card}{card}
\newcommand{\suchthat}{;\ifnum\currentgrouptype=16 \middle\fi|;}

\newtheorem{lemma}{Lemma}

\newcommand{\bd}{\partial}
\newcommand{\tr}{\mathrm{tr}}
\newcommand{\ra}{\rightarrow}
\newcommand{\lan}{\langle}
\newcommand{\ran}{\rangle}
\newcommand{\norm}[1]{\left\lVert#1\right\rVert}
\newcommand{\inn}[1]{\lan#1\ran}
\newcommand{\ol}{\overline}
\begin{document}
\noindent Q4: Note that when we refer to $\R^{k-1}$, we will be referring to $\R^{k-1}$ as a subset of $\R^k$. First note that since $U = U^\prime \cap \R^k$, and $V = V^\prime \cap \R^k$, it suffices to show that $\phi(U^\prime \cap \R^{k-1}) = V^\prime \cap \R^{k-1}$. First suppose that $x\in \phi(U\prime \cap \R^{k-1})$. Since $\phi$ is a diffeomorphism, there must exist some unique $y\in U^\prime \cap \R^{k-1}$
such that $\phi(y)=x$. We can find some open set $B\cap \R^k$ with $y\in B \cap \R^k$ and $ B\cap \R^k \subset U^\prime \cap \R^k$. Hence since $\phi$ is a diffeomorphism, we have that $\phi(B \cap \R^{k-1}) \subset V^\prime \cap \R^{k-1}$. Hence $x\in V^\prime \cap \R^{k-1}$. Now suppose that $x\in V^\prime \cap \R^{k-1}$. Then we have that $x$ must belong to $V\cap \R^{k}$. There must therefore exist some $y\in U$ such that $\phi(y)=x$. We claim that such a $y$ must belong to $U \cap \R^{k-1}$. Suppose not, then it must not belong to either $U$ or $\R^{k-1}$. It must definitely belong to $U$, so suppose it was not an element of $\R^{k-1}$. If we take a sufficiently small open set $C$ around $y$, such that $C$ is disjoint from $\R^{k-1}$, Then the image of $C$ under $\phi$ must also be disjoint from $\R^{k-1}$. However it contains $y$ and $\phi(y)=x\in V^\prime \cap \R^{k-1}$, a contradiction. We conclude that indeed $\phi(U\cap \R^{k-1}) = V\cap \R^{k-1}$. 
\end{document}