\documentclass[letterpaper]{article}
\usepackage[letterpaper,margin=1in,footskip=0.25in]{geometry}
\usepackage[utf8]{inputenc}
\usepackage{amsmath}
\usepackage{amsthm}
\usepackage{amssymb, pifont}
\usepackage{mathrsfs}
\usepackage{enumitem}
\usepackage{fancyhdr}
\usepackage{hyperref}

\pagestyle{fancy}
\fancyhf{}
\rhead{MAT 257}
\lhead{Assignment 13}
\rfoot{Page \thepage}

\setlength\parindent{24pt}
\renewcommand\qedsymbol{$\blacksquare$}

\DeclareMathOperator{\T}{\mathcal{T}}
\DeclareMathOperator{\V}{\mathcal{V}}
\DeclareMathOperator{\U}{\mathcal{U}}
\DeclareMathOperator{\Prt}{\mathbb{P}}
\DeclareMathOperator{\R}{\mathbb{R}}
\DeclareMathOperator{\N}{\mathbb{N}}
\DeclareMathOperator{\Z}{\mathbb{Z}}
\DeclareMathOperator{\Q}{\mathbb{Q}}
\DeclareMathOperator{\C}{\mathbb{C}}
\DeclareMathOperator{\ep}{\varepsilon}
\DeclareMathOperator{\identity}{\mathbf{0}}
\DeclareMathOperator{\card}{card}
\newcommand{\suchthat}{;\ifnum\currentgrouptype=16 \middle\fi|;}

\newtheorem{lemma}{Lemma}

\newcommand{\tr}{\mathrm{tr}}
\newcommand{\ra}{\rightarrow}
\newcommand{\lan}{\langle}
\newcommand{\ran}{\rangle}
\newcommand{\norm}[1]{\left\lVert#1\right\rVert}
\newcommand{\inn}[1]{\lan#1\ran}
\newcommand{\ol}{\overline}
\begin{document}
\noindent Q2a: We take note that $L_1$ has 2 eigenvalues, $-1$ corresponding to $(1,0)$ and $1$ corresponding to $(0,1)$. The determinant is the product of the eigenvalues, so $det(L_1)=-1<0$. Therefore, this reverses the standard orientation of $\R^2$. 
\newline \\ Q2b: The matrix representing $L_2$ looks like $A= \begin{bmatrix}
    0 & 1 \\ 1 & 0
\end{bmatrix}$ We have that $\det(A) = -1$. Therefore this linear map is orientation reversing. 
\newline \\ Q2c: Let $A$ be the matrix representing the linear transformation $L_3$, then $A = \begin{bmatrix}
    \cos \theta & -\sin \theta \\ \sin \theta & \cos \theta
\end{bmatrix}$ for $\theta = \frac{2\pi}{7}$. This will have a determinant 1, thus $L_3$ is orientation preserving. 
\newline \\ Q2d: The matrix $A$ representing $L_4$ will be represented by the same matrix as in 2c, except $\theta = \frac{12\pi}{7}$. The determinant of $A$ will be 1 as well, thus $L_4$ is orientation preserving. 
\newline \\ Q2e: The matrix $A$ representing $L_5$ takes the form $A = \begin{bmatrix}
    1 & 0 \\ 0 & -1
\end{bmatrix}$ The determinant of $A$ is $-1$, so $L_5$ is orientation reversing. 
\newline \\ Q2f: $L_6$ is represented by the matrix $A = \begin{bmatrix}
    0 & 0 & 1 \\ 1 & 0 & 0 \\ 0 & 1 & 0
\end{bmatrix}$ We see that $\det(A) = 1$. Therefore, $L_6$ is orientation preserving 
\newline \\ Q2g: $L_7$ will have n identical eigenvalues, $-1$, corresponding to each basis vector. Therefore the $\det(L_7) = (-1)^n$. $L_7$ is orientation preserving if $n$ is even, and orientation reversing if $n$ is odd. 
\newline \\ Q2h: Note that $L_8$ can be written as the composition of $n\cdot m$ transpositions, each of which swaps one basis entry with another. Each transposition has a determinant of $-1$, so their composition has a determinant of $(-1)^{n\cdot m}$. Therefore, $L$ preserves orientation if $n\cdot m$ is even, and reverses orientation if $n\cdot m$ is odd. 
\end{document}