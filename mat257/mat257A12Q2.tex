\documentclass[letterpaper]{article}
\usepackage[letterpaper,margin=1in,footskip=0.25in]{geometry}
\usepackage[utf8]{inputenc}
\usepackage{amsmath}
\usepackage{amsthm}
\usepackage{amssymb, pifont}
\usepackage{mathrsfs}
\usepackage{enumitem}
\usepackage{fancyhdr}
\usepackage{hyperref}

\pagestyle{fancy}
\fancyhf{}
\rhead{MAT 257}
\lhead{Assignment 12}
\rfoot{Page \thepage}

\setlength\parindent{24pt}
\renewcommand\qedsymbol{$\blacksquare$}

\DeclareMathOperator{\T}{\mathcal{T}}
\DeclareMathOperator{\V}{\mathcal{V}}
\DeclareMathOperator{\U}{\mathcal{U}}
\DeclareMathOperator{\Prt}{\mathbb{P}}
\DeclareMathOperator{\R}{\mathbb{R}}
\DeclareMathOperator{\N}{\mathbb{N}}
\DeclareMathOperator{\Z}{\mathbb{Z}}
\DeclareMathOperator{\Q}{\mathbb{Q}}
\DeclareMathOperator{\C}{\mathbb{C}}
\DeclareMathOperator{\ep}{\varepsilon}
\DeclareMathOperator{\identity}{\mathbf{0}}
\DeclareMathOperator{\card}{card}
\newcommand{\suchthat}{;\ifnum\currentgrouptype=16 \middle\fi|;}

\newtheorem{lemma}{Lemma}

\newcommand{\tr}{\mathrm{tr}}
\newcommand{\ra}{\rightarrow}
\newcommand{\lan}{\langle}
\newcommand{\ran}{\rangle}
\newcommand{\norm}[1]{\left\lVert#1\right\rVert}
\newcommand{\inn}[1]{\lan#1\ran}
\newcommand{\ol}{\overline}
\begin{document}
\noindent Q2: If $v_1\dots v_n$, and $I\in \underline{n}^n$, then $\det(v_I) = \omega_J(v_I)$ where $J=\{1,2,\dots , n\}$. We know this is true since if $I$ has any repeating elements, $\omega_J(v_I)=0$, and if $I \in \underline{n}_a^n$. Let $\tau\in S_n$ be the unique permutation which satisfies $\tau(J)=I$, 
then $$\omega_J(v_I) = (\sum_{\sigma\in S_n} (-1)^{\sigma} \varphi_{J}\circ \sigma^{*})(v_I) = \sum_{\sigma \in S_n} (-1)^{\sigma} \varphi_I (v_{\sigma(I)} ) = (-1)^{\tau}$$
This lines up with what we know about applying the determinant to the basis of a vector space. Now  consider a collection of $n$ vectors, $u_1,\dots, u_n $, where each $u_i = \sum_{j=1}^n a_{ij} v_j$
We compute $\det(u_1,\dots u_n)$ as follows: 
\begin{align*}
    \det(u_1, \dots , u_n)  & = \det(\sum_{j_1=1}^n a_{1 j_1} v_{j_1},\dots, \sum_{j_n=1}^n a_{n,j_n}v_{j_n})
    \\ & = \sum_{j_1=1}^n a_{1j_1} \det(v_{j_1}, \sum_{j_2=1}^n a_{2 j_2}v_{j_2} , \dots \sum_{j_n=1}^n a_{n,j_n}v_{j_n})
    \\ & = \sum_{j_1=1}^n a_{1j_1} (\sum_{j_2=1}^n a_{2 j_2} \det(v_{j_1},v_{j_2}, \dots u_n))
    \\ & \vdots 
    \\ & = \sum_{j_1 =1}^n a_{1j_1}\sum_{j_2=1}^n a_{2j_2} \dots \sum_{j_n=1}^n a_{nj_n} \det(v_{j_1},\dots v_{j_n})
    \\ & = \sum_{\sigma \in S_n} (-1)^{\sigma} \prod_{j=1}^n a_{i_j \sigma(j)}
    \\ & = \sum_{\sigma \in S_n} (-1)^{\sigma} \varphi_1\otimes \dots \otimes \varphi_n \circ \sigma^{*} ( u_1, \dots u_n)
    \\ & = \omega_I(u_1\dots u_n)
\end{align*}
\end{document}