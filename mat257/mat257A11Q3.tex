\documentclass[letterpaper]{article}
\usepackage[letterpaper,margin=1in,footskip=0.25in]{geometry}
\usepackage[utf8]{inputenc}
\usepackage{amsmath}
\usepackage{amsthm}
\usepackage{amssymb, pifont}
\usepackage{mathrsfs}
\usepackage{enumitem}
\usepackage{fancyhdr}
\usepackage{hyperref}

\pagestyle{fancy}
\fancyhf{}
\rhead{MAT 257}
\lhead{Assignment 11}
\rfoot{Page \thepage}

\setlength\parindent{24pt}
\renewcommand\qedsymbol{$\blacksquare$}

\DeclareMathOperator{\T}{\mathcal{T}}
\DeclareMathOperator{\U}{\mathcal{U}}
\DeclareMathOperator{\Prt}{\mathbb{P}}
\DeclareMathOperator{\R}{\mathbb{R}}
\DeclareMathOperator{\N}{\mathbb{N}}
\DeclareMathOperator{\Z}{\mathbb{Z}}
\DeclareMathOperator{\Q}{\mathbb{Q}}
\DeclareMathOperator{\C}{\mathbb{C}}
\DeclareMathOperator{\ep}{\varepsilon}
\DeclareMathOperator{\identity}{\mathbf{0}}
\DeclareMathOperator{\card}{card}
\newcommand{\suchthat}{;\ifnum\currentgrouptype=16 \middle\fi|;}

\newtheorem{lemma}{Lemma}

\newcommand{\tr}{\mathrm{tr}}
\newcommand{\ra}{\rightarrow}
\newcommand{\lan}{\langle}
\newcommand{\ran}{\rangle}
\newcommand{\norm}[1]{\left\lVert#1\right\rVert}
\newcommand{\inn}[1]{\lan#1\ran}
\newcommand{\ol}{\overline}
\begin{document}
\noindent Q3: To show that $B\in \T^2(\T^k(V))$, we must show that it is 2-linear. Thus, for $T_1,T_2,T_3\in \T^k(V)$ and $\alpha\in \R$, we evaluate $B(T_1 + \alpha T_2,T_3)$ and $B(T_1,T_2 + \alpha T_3)$. We see the following: 
\begin{align*}
    B(T_1 + \alpha T_2,T_3) & = \sum_{i_1,\dots , i_k = 1}^n (T_1 + \alpha T_2)(v_{i_1}\dots v_{i_k})T_3(v_{i_1}\dots v_{i_k})
    \\ & = \sum_{i_1,\dots , i_k = 1}^n [T_1(v_{i_1}\dots v_{i_k}) + \alpha T_2(v_{i_1}\dots v_{i_k})] T_3(v_{i_1}\dots v_{i_k})
    \\ & = \sum_{i_1,\dots ,i_k=1}^n T_1(v_{i_1}\dots v_{i_k}) T_3(v_{i_1}\dots v_{i_k}) + \alpha T_2(v_{i_1}\dots v_{i_k}) T_3(v_{i_1}\dots v_{i_k})
    \\ & = \sum_{i_1,\dots ,i_k=1}^n T_1(v_{i_1}\dots v_{i_k}) T_3(v_{i_1}\dots v_{i_k}) + \alpha \sum_{i_1,\dots ,i_k}^n T_2(v_{i_1}\dots v_{i_k}) T_3 (v_{i_1}\dots v_{i_k})
    \\ & = B(T_1,T_3) + \alpha B(T_2,T_3)
\end{align*}
By almost exactly the same computation, we see that $B(T_1,T_2+\alpha T_3) = B(T_1,T_2) + \alpha B(T_1,T_3)$. B is bilinear and hence belongs to $\T^2(\T^k(V))$
\newline \\ Q3b: We now wish to show that $B$ is an inner product on $\T^k(V)$. We have shown above that $B$ is bilinear. It remains to prove it is symmetric and positive definite. First, observe the following: 
\begin{align*}
    B(T_1,T_2) & = \sum_{i_1,\dots ,i_k = 1}^n T_1(v_{i_1}\dots v_{i_k}) T_2(v_{i_1}\dots v_{i_k})
    \\ & = \sum_{i_1,\dots ,i_k=1}^n T_2 (v_{i_1}\dots v_{i_k}) T_1 (v_{i_1}\dots v_{i_k})
    \\ & = B(T_2,T_1)
\end{align*} Hence $B$ is symmetric. We will now show that for any $T\in \T^k(V)$, $B(T,T)\geq 0$ with equality holding if and only if $T=0$. Observe: 
\begin{align*}
    B(T,T) & = \sum_{i_1,\dots ,i_k =1}^n T(v_{i_1}\dots v_{i_k}) T(v_{i_1}\dots v_{i_k})
    \\ & = \sum_{i_1,\dots ,i_k}^n [T(v_{i_1}\dots v_{i_k})]^2 \geq 0
\end{align*}We note that equality holds if and only iff for each $v_{i_j}$, $T(v_{i_1}\dots v_{i_k})=0$, meaning that on any k-tuple of basis vectors, $T=0$. This is equivalent to saying that $T$ is the 0-mapping. 
\end{document}
