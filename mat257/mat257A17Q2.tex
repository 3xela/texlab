\documentclass[letterpaper]{article}
\usepackage[letterpaper,margin=1in,footskip=0.25in]{geometry}
\usepackage[utf8]{inputenc}
\usepackage{amsmath}
\usepackage{amsthm}
\usepackage{amssymb, pifont}
\usepackage{mathrsfs}
\usepackage{enumitem}
\usepackage{fancyhdr}
\usepackage{hyperref}

\pagestyle{fancy}
\fancyhf{}
\rhead{MAT 257}
\lhead{Assignment 17}
\rfoot{Page \thepage}

\setlength\parindent{24pt}
\renewcommand\qedsymbol{$\blacksquare$}

\DeclareMathOperator{\T}{\mathcal{T}}
\DeclareMathOperator{\V}{\mathcal{V}}
\DeclareMathOperator{\U}{\mathcal{U}}
\DeclareMathOperator{\Prt}{\mathbb{P}}
\DeclareMathOperator{\R}{\mathbb{R}}
\DeclareMathOperator{\N}{\mathbb{N}}
\DeclareMathOperator{\Z}{\mathbb{Z}}
\DeclareMathOperator{\Q}{\mathbb{Q}}
\DeclareMathOperator{\C}{\mathbb{C}}
\DeclareMathOperator{\ep}{\varepsilon}
\DeclareMathOperator{\identity}{\mathbf{0}}
\DeclareMathOperator{\card}{card}
\newcommand{\suchthat}{;\ifnum\currentgrouptype=16 \middle\fi|;}

\newtheorem{lemma}{Lemma}

\newcommand{\bd}{\partial}
\newcommand{\tr}{\mathrm{tr}}
\newcommand{\ra}{\rightarrow}
\newcommand{\lan}{\langle}
\newcommand{\ran}{\rangle}
\newcommand{\norm}[1]{\left\lVert#1\right\rVert}
\newcommand{\inn}[1]{\lan#1\ran}
\newcommand{\ol}{\overline}
\begin{document}
\noindent Q2: Suppose not, that is suppose that in some neighbourhood of a point $p\in M$, we have that $M$ is not measure 0 as a subset of $\R^{22}$. Let $U\ni p$ be a neighbourhood of $p$ such that there exists an open $V\subset R^{22}$, and a diffeomorphism $h:U\to V$ where $h(U\cap M) = V\cap (\R^{12} \times 0_{\R^{10}})$. However, we have that $V\cap (\R^{12} \times 0_{\R^{10}})$ is clearly a measure 0 set. We now claim that the image of a bounded measure 0 set under a diffeomorphism is measure 0. \newline \\ pf: Let $f$ be a diffeomorphism from $A$ to $B$ where $A$ is measure 0. Note that on a bounded set, the mean value theorem implies that for all $x,y \in A$ , we have that $|f(x)-f(y)|\leq M|x-y|$, for some $M\in \R$. Hence we have that $f$ is Lipschitz. Thus, for all $\ep>0$, we can take an open covering of $A$ by squares $\{S_k\}$ such that $\sum_{S_k} vol(S_k) < \ep$. We have that the diameter(maximum distance between points ) of $S_k\cap A $ will be less than or equal to the diameter of $S_k$. 
Therefore, by lipschitz, we have that the diameter of $f(S_k\cap A)$ is less than $M \cdot Diam(S_k\cap A)$. Hence there exists some square $S_k^\prime$ containing $f(S_k\cap A)$ with diameter $M\cdot Diam(S_k)$. We can do this for every square covering $A$, and by continuity $\{S_k^\prime \}$ forms a cover for $B$. We see that $$\sum_{k}S_K^\prime \leq M^n \sum_{k}Diam(S_k)^n \leq nM^n \sum_{k} vol(S_k) \leq nM^n \ep$$
Since $M,n$ are both fixed we have that $f(A)$ must be measure 0$\qed$. 
\newline Therefore applying the inverse of $h$ to our measure 0 set $V\cap (\R^{12} \times 0_{\R^{10}})$ should be measure 0, but by assumption $U\cap M$ is not measure 0. We obtain a contradiction and conclude that every 12 manifold in $\R^{22}$ is measure 0. 
\end{document}