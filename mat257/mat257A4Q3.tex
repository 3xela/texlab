\documentclass[letterpaper]{article}
\usepackage[letterpaper,margin=1in,footskip=0.25in]{geometry}
\usepackage[utf8]{inputenc}
\usepackage{amsmath}
\usepackage{amsthm}
\usepackage{amssymb, pifont}
\usepackage{mathrsfs}
\usepackage{enumitem}
\usepackage{fancyhdr}
\usepackage{hyperref}

\pagestyle{fancy}
\fancyhf{}
\rhead{MAT 257}
\lhead{Assignment 4}
\rfoot{Page \thepage}

\setlength\parindent{24pt}
\renewcommand\qedsymbol{$\blacksquare$}

\DeclareMathOperator{\R}{\mathbb{R}}
\DeclareMathOperator{\N}{\mathbb{N}}
\DeclareMathOperator{\Z}{\mathbb{Z}}
\DeclareMathOperator{\Q}{\mathbb{Q}}
\DeclareMathOperator{\C}{\mathbb{C}}
\DeclareMathOperator{\ep}{\varepsilon}
\DeclareMathOperator{\identity}{\mathbf{0}}
\DeclareMathOperator{\card}{card}
\newcommand{\suchthat}{;\ifnum\currentgrouptype=16 \middle\fi|;}

\newtheorem{lemma}{Lemma}

\newcommand{\tr}{\mathrm{tr}}
\newcommand{\ra}{\rightarrow}
\newcommand{\lan}{\langle}
\newcommand{\ran}{\rangle}
\newcommand{\norm}[1]{\left\lVert#1\right\rVert}
\newcommand{\inn}[1]{\lan#1\ran}
\newcommand{\ol}{\overline}
\begin{document}
Q3a:$$f(x,y) = \int_0^x g_1(t,0)dt + \int_0^y g_2(x,t) dt$$
Applying the fundamental theorem of calculus, we get that $$D_2f(x,y) = \frac{\partial}{\partial y} \int_0^x g_1(t,0)dt + \frac{\partial}{\partial y} \int_0^y g_2(x,t) dt = g_2(x,y) $$
\\ 3b: \\
For $f$ to satisfy $D_2f(x,y) = g_1(x,y)$, define $f$ in the following way. $$f(x,y) = \int_0^x g_1(t,y) dt + \int_0^y g_2(0,t) dt$$ by the FTC we will have that $$D_1f(x,y) = g_1(x,y)$$
\\ 3c: \\
Choose $f_c$ in the following way $$f_c(x,y) = \frac{1}{2}(x^2+y^2)$$ From single variable calculus, we have that $\frac{\partial f_c}{\partial x} = x$ and $\frac{\partial f_c}{\partial y} = y$
\\ 3d:\\
Choose $f_d$ in the following way $$f_d(x,y) = xy$$ From single variable calculus, we have that $\frac{\partial f_d}{\partial x} = y$ and $\frac{\partial f_d}{\partial y} = x$
\end{document}