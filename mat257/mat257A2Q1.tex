\documentclass[letterpaper]{article}
\usepackage[letterpaper,margin=1in,footskip=0.25in]{geometry}
\usepackage[utf8]{inputenc}
\usepackage{amsmath}
\usepackage{amsthm}
\usepackage{amssymb, pifont}
\usepackage{mathrsfs}
\usepackage{enumitem}
\usepackage{fancyhdr}
\usepackage{hyperref}

\pagestyle{fancy}
\fancyhf{}
\rhead{MAT 257}
\lhead{Assignment 2}
\rfoot{Page \thepage}

\setlength\parindent{24pt}
\renewcommand\qedsymbol{$\blacksquare$}

\DeclareMathOperator{\R}{\mathbb{R}}
\DeclareMathOperator{\N}{\mathbb{N}}
\DeclareMathOperator{\Z}{\mathbb{Z}}
\DeclareMathOperator{\Q}{\mathbb{Q}}
\DeclareMathOperator{\C}{\mathbb{C}}
\DeclareMathOperator{\ep}{\varepsilon}
\DeclareMathOperator{\identity}{\mathbf{0}}
\DeclareMathOperator{\card}{card}
\newcommand{\suchthat}{;\ifnum\currentgrouptype=16 \middle\fi|;}

\newtheorem{lemma}{Lemma}

\newcommand{\tr}{\mathrm{tr}}
\newcommand{\ra}{\rightarrow}
\newcommand{\lan}{\langle}
\newcommand{\ran}{\rangle}
\newcommand{\norm}[1]{\left\lVert#1\right\rVert}
\newcommand{\inn}[1]{\lan#1\ran}
\newcommand{\ol}{\overline}
\begin{document}
Q1: 
\\A1: We will determine the interiour, boundary and exteriour of $A_1$. First we consider the exteriour. Since $A_1$ is the closed unit ball, the exteriour must be its compliment in $\R^n$. Thus, ext $A_1 = $ \{ $x \in \R^n$ : $\norm{x} >1$ \}.
Now we consider the boundary of $A_1$.  If we take an open ball of radius $\epsilon$ around any $x$ such that $\norm{x} =1$, this ball will intersect both $A_1$ and $A_1^c$, since it will contain at least one point with a norm greater than 1. It will also contain our chosen point, $x$ which belongs to $A_1$. Now suppose that $x\in bd A_1$. then, consider then open ball with radius $\epsilon$ centered at $x$. This ball will contain points both in $A_1$ and $A_1^c$ so long as $\norm{x}=1$. If we chose a point $y$ with $\norm{y}<1$ then we could take $\epsilon$ to be $1-\norm{y}$. This would be fully contained in the set $A_1$. Thus, bd $A_1=$ \{ $x\in \R^n$ : $ \norm{x} = 1 $\}. 
Since the interiour, boundary and exteriour are disjoint from eachother and have union $\R^n$, the interiour of $A_1$ must be whatever is left over. So int $A_1=$ \{ $x \in \R^n$ : $\norm{x} <1$ \}. 

A2: We will first determine the exteriour of $A_2$. Suppose that $x\in ext A_2$. Then there exists some $\epsilon >0$ with $B_{\epsilon}(x)$ disjoint from $A_2$. $x$ will never have norm 1, since the ball centered at it will always contain itself and so have nonempty intersection with $A_2$. Therefore $\norm{x}>1 \text{ or } \norm{x} <1$. Now suppose that $\norm{x} >1 \text{ or } \norm{x} <1$. Choose $\epsilon = \frac{|\norm{x}-1|}{2}$. This ball by construction will contain no points with norm 1. Thus ext $A_2 = \{ x \in \R^n : \norm{x} >1 \space \text{or} \space \norm{x} <1  \}$. 
Now we will determine the interiour of $A_2$. We suppose that $x\in int A_2$. Then there is some $\epsilon >0$ such that $B_{\epsilon} (x) \subset A_2$. However, every ball about a point with norm 1 will contain some other points with norm$<$1 or norm $>$1, by definition of the ball. Thus $\emptyset \supset int A_2$. Trivially, $\emptyset \subset A_2$. Thus we see that $int A_2 = \emptyset$. Since the interiour, boundary and exteriour are disjoint from eachother and have union of $\R^n$, the boundary of $A_2$ must be whatever is left over i.e. $bd A_2 = A_2$. 
\\
A3: We claim that $bd A_3 = \R^n$.  Clearly, $bd A_3 \subset \R^n$. Now suppose that $x\in \R^n$. Let $\epsilon >0$. Consider the $\epsilon$ cube around $x$, $C= (x_1-\epsilon,x_1+\epsilon) \times \dots (x_n - \epsilon,x_n + \epsilon)$. By the density of rationals in $\R$, for each open interval $(x_i-\epsilon, x_i + \epsilon )$ we can find some rational $q_i\in (x_i-\epsilon, x_i + \epsilon )$, with $(q_1, \dots q_n) \in A_3$. Similarly, we can find an irrational $r_i$ with $r_i\in (x_i-\epsilon, x_i + \epsilon )$, and $(r_1, \dots r_n) \notin A$. Our choice of $x$ was arbitrary and so $bd A_3 = \R^n$. Since the union of the boundary, exteriour and interior is $\R^n$ and they are pairwise disjoint, it follows that $int A_3 = ext A_3 = \emptyset$.
\end{document}