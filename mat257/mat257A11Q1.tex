\documentclass[letterpaper]{article}
\usepackage[letterpaper,margin=1in,footskip=0.25in]{geometry}
\usepackage[utf8]{inputenc}
\usepackage{amsmath}
\usepackage{amsthm}
\usepackage{amssymb, pifont}
\usepackage{mathrsfs}
\usepackage{enumitem}
\usepackage{fancyhdr}
\usepackage{hyperref}

\pagestyle{fancy}
\fancyhf{}
\rhead{MAT 257}
\lhead{Assignment 11}
\rfoot{Page \thepage}

\setlength\parindent{24pt}
\renewcommand\qedsymbol{$\blacksquare$}

\DeclareMathOperator{\U}{\mathcal{U}}
\DeclareMathOperator{\Prt}{\mathbb{P}}
\DeclareMathOperator{\R}{\mathbb{R}}
\DeclareMathOperator{\N}{\mathbb{N}}
\DeclareMathOperator{\Z}{\mathbb{Z}}
\DeclareMathOperator{\Q}{\mathbb{Q}}
\DeclareMathOperator{\C}{\mathbb{C}}
\DeclareMathOperator{\ep}{\varepsilon}
\DeclareMathOperator{\identity}{\mathbf{0}}
\DeclareMathOperator{\card}{card}
\newcommand{\suchthat}{;\ifnum\currentgrouptype=16 \middle\fi|;}

\newtheorem{lemma}{Lemma}

\newcommand{\tr}{\mathrm{tr}}
\newcommand{\ra}{\rightarrow}
\newcommand{\lan}{\langle}
\newcommand{\ran}{\rangle}
\newcommand{\norm}[1]{\left\lVert#1\right\rVert}
\newcommand{\inn}[1]{\lan#1\ran}
\newcommand{\ol}{\overline}
\begin{document}
\noindent Q1a: We claim that $\gamma = (\iota(v_1)\dots \iota(v_n))$ is a basis for $V^{**}$ and that it is the dual basis of $(\phi_1\dots \phi_n)$, the dual basis of $V^*$
We will prove that $\gamma$ is linearly independant and spans $V^{**}$. First suppose that for some scalar $\alpha_1,\dots, \alpha_n$ we have that $$\alpha_1 \iota(v_1)+ \dots \alpha_n \iota(v_n)=0$$
From the definition of $\iota$, we have $$\alpha_1 \phi(v_1) + \dots \alpha_n \phi(v_n)=0 , \forall \phi\in V^*$$ 
Now write $\phi = \beta_1 \phi_q + \dots \beta_n \phi_n$ for scalars $\beta_1,\dots \beta_n$. We see that $$\alpha_1(\beta_1\phi_1(v_1)+\dots + \beta_n\phi_n(v_1))+ \dots \alpha_n(\beta_1\phi_1(v_n)+\dots \beta_n \phi_n(v_n)) = 0$$
From the definition of the dual basis this gives us: $$\alpha_1\beta_1 + \dots + \alpha_n \beta_n = 0$$ Since this is true for every $\beta_i$, we must have that $\alpha_1=\dots = \alpha_n =0$. Hence $\gamma$ is a linearly independant set. We now claim that it spans $V^{**}$.
Now suppose that $\psi \in V^{**}$, and $\psi(\phi_i) = k_i$. Let $\phi= \beta_1 \phi_1 + \dots \beta_n \phi_n$. 
We see that 
\begin{align*}
    \psi(\phi) & = \psi(\beta_1 \phi_1 + \dots + \beta_n \phi_n)
    \\ & = \beta_1 k_1 + \dots + \beta_n k_n
    \\ & = k_1 \phi(v_1) + \dots k_n \phi(v_n)
    \\ & = k_1 \iota_{v_1}(\phi) + \dots + k_n \iota_{v_n}(\phi)
\end{align*} Thus $\gamma$ spans $V^{**}$ and we conclude it is a basis. 
We now want to show that $\gamma$ is dual to $(\phi_1, \dots ,\phi_n)$. Notice that 
$$\iota(v_i)(\phi_j) = \phi_j(v_i) = \delta_{ij}$$
We conclude that $\gamma$ is indeed the dual of $(\phi_1,\dots \phi_n)$
\newline \\ Q1b: First, observe that $\iota(v)(\alpha\phi + \psi) = \alpha\phi + \psi(v) = \alpha\phi(v) + \phi(v) = \alpha \iota(v)(\phi) + \iota(v)(\psi)$, so $\iota$ is linear. 
Note that by 1b, we see that the image of $\iota$ is $n$ dimensional, and the domain is as well $n$ dimensional. Hence by the Rank-Nullity theorem we conclude it is a bijection . Thus it is a linear isomorphism between $V$ and $V^{**}$
\end{document}