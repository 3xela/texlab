\documentclass[letterpaper]{article}
\usepackage[letterpaper,margin=1in,footskip=0.25in]{geometry}
\usepackage[utf8]{inputenc}
\usepackage{amsmath}
\usepackage{amsthm}
\usepackage{amssymb, pifont}
\usepackage{mathrsfs}
\usepackage{enumitem}
\usepackage{fancyhdr}
\usepackage{hyperref}

\pagestyle{fancy}
\fancyhf{}
\rhead{MAT 257}
\lhead{Assignment 16}
\rfoot{Page \thepage}

\setlength\parindent{24pt}
\renewcommand\qedsymbol{$\blacksquare$}

\DeclareMathOperator{\T}{\mathcal{T}}
\DeclareMathOperator{\V}{\mathcal{V}}
\DeclareMathOperator{\U}{\mathcal{U}}
\DeclareMathOperator{\Prt}{\mathbb{P}}
\DeclareMathOperator{\R}{\mathbb{R}}
\DeclareMathOperator{\N}{\mathbb{N}}
\DeclareMathOperator{\Z}{\mathbb{Z}}
\DeclareMathOperator{\Q}{\mathbb{Q}}
\DeclareMathOperator{\C}{\mathbb{C}}
\DeclareMathOperator{\ep}{\varepsilon}
\DeclareMathOperator{\identity}{\mathbf{0}}
\DeclareMathOperator{\card}{card}
\newcommand{\suchthat}{;\ifnum\currentgrouptype=16 \middle\fi|;}

\newtheorem{lemma}{Lemma}

\newcommand{\bd}{\partial}
\newcommand{\tr}{\mathrm{tr}}
\newcommand{\ra}{\rightarrow}
\newcommand{\lan}{\langle}
\newcommand{\ran}{\rangle}
\newcommand{\norm}[1]{\left\lVert#1\right\rVert}
\newcommand{\inn}[1]{\lan#1\ran}
\newcommand{\ol}{\overline}
\begin{document}
\noindent Q1a: It is clear that if $b=\bd c$, then $\bd b = \bd^2c =0$ 
\newline \\ Q1b: Suppose that there is a chain $B$ such that $\bd B = b$. Consider the 1-form $\omega = \frac{-y}{x^2+y^2}dx + \frac{x}{x^2+y^2}dy$. Hence by Stoke's Theorem on Chains, we have that $\int_{\bd B} \omega = \int_{B} d\omega $. Thus we compute $$\int_{\bd B} \omega = \int_{b} \omega = \int_{[0,1]} b^{*}\omega = \int_{[0,1]} 2\pi \sin^2(2\pi t) dt + 2\pi \cos^2(2\pi t) dt \int_{[0,1]} 2\pi dt = 2\pi$$
However we know that $d\omega=0$ and so we have that $\int_{B}d\omega =0$. We obtain a contradiction and conclude that $b$ is not in the image of the boundary operator. We will now verify using Stoke's theorem that $\bd b=0$: 
$$0= \int_{b}d\omega = \int_{\bd b}\omega$$ We see that the integral is 0 and conclude that $\bd b=0$. 
\end{document}