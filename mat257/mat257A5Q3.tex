\documentclass[letterpaper]{article}
\usepackage[letterpaper,margin=1in,footskip=0.25in]{geometry}
\usepackage[utf8]{inputenc}
\usepackage{amsmath}
\usepackage{amsthm}
\usepackage{amssymb, pifont}
\usepackage{mathrsfs}
\usepackage{enumitem}
\usepackage{fancyhdr}
\usepackage{hyperref}

\pagestyle{fancy}
\fancyhf{}
\rhead{MAT 257}
\lhead{Assignment 5}
\rfoot{Page \thepage}

\setlength\parindent{24pt}
\renewcommand\qedsymbol{$\blacksquare$}

\DeclareMathOperator{\R}{\mathbb{R}}
\DeclareMathOperator{\N}{\mathbb{N}}
\DeclareMathOperator{\Z}{\mathbb{Z}}
\DeclareMathOperator{\Q}{\mathbb{Q}}
\DeclareMathOperator{\C}{\mathbb{C}}
\DeclareMathOperator{\ep}{\varepsilon}
\DeclareMathOperator{\identity}{\mathbf{0}}
\DeclareMathOperator{\card}{card}
\newcommand{\suchthat}{;\ifnum\currentgrouptype=16 \middle\fi|;}

\newtheorem{lemma}{Lemma}

\newcommand{\tr}{\mathrm{tr}}
\newcommand{\ra}{\rightarrow}
\newcommand{\lan}{\langle}
\newcommand{\ran}{\rangle}
\newcommand{\norm}[1]{\left\lVert#1\right\rVert}
\newcommand{\inn}[1]{\lan#1\ran}
\newcommand{\ol}{\overline}
\begin{document}
Q3a: Suppose $WLOG$ that $f^\prime (a) > 0$ for all $a\in \R$. Suppose not, that is suppose that $f^\prime (a) >0$ yet $f$ not injective. Then there must exist some distinct $a,b$ and $a < b\in \R$ where $f(a) = f(b)$. Then by the mean value theorem there must exist some $c\in (a,b)$ such that 
$$f^\prime (c) = \frac{f(a)-f(b)}{a-b}$$ By our assumption it must be that $f^\prime(c) = 0$. We obtain a contradiction. 
\\ Q3b: \\
From Spivak Theorem 2-7 we have that $$f^{\prime}(x,y) = \begin{bmatrix} e^x cos(y) &  -e^x sin(y) \\ e^xsin(y) & e^x cos(y)
\end{bmatrix}$$
Computing the determinant, we have $Det (f^{\prime}(x,y)) = e^{2x} cos^2(y) + e^{2x} sin^2(y) = e^{2x}$. This is always nonzero hence $f^\prime$ is invertible for all $(x,y)\in \R^2$. However the function $f$ is not injective since for any choices of $(x,y)$, by the periodicity of $sin$ and $cos$ we will have $$f(x,y) = (e^x cos(y),e^x sin(y)) = (e^x cos(y+2 \pi), e^x sin(y+2 \pi)) = f(x,y+2 \pi)$$
\end{document}