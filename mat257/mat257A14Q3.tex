\documentclass[letterpaper]{article}
\usepackage[letterpaper,margin=1in,footskip=0.25in]{geometry}
\usepackage[utf8]{inputenc}
\usepackage{amsmath}
\usepackage{amsthm}
\usepackage{amssymb, pifont}
\usepackage{mathrsfs}
\usepackage{enumitem}
\usepackage{fancyhdr}
\usepackage{hyperref}

\pagestyle{fancy}
\fancyhf{}
\rhead{MAT 257}
\lhead{Assignment 14}
\rfoot{Page \thepage}

\setlength\parindent{24pt}
\renewcommand\qedsymbol{$\blacksquare$}

\DeclareMathOperator{\T}{\mathcal{T}}
\DeclareMathOperator{\V}{\mathcal{V}}
\DeclareMathOperator{\U}{\mathcal{U}}
\DeclareMathOperator{\Prt}{\mathbb{P}}
\DeclareMathOperator{\R}{\mathbb{R}}
\DeclareMathOperator{\N}{\mathbb{N}}
\DeclareMathOperator{\Z}{\mathbb{Z}}
\DeclareMathOperator{\Q}{\mathbb{Q}}
\DeclareMathOperator{\C}{\mathbb{C}}
\DeclareMathOperator{\ep}{\varepsilon}
\DeclareMathOperator{\identity}{\mathbf{0}}
\DeclareMathOperator{\card}{card}
\newcommand{\suchthat}{;\ifnum\currentgrouptype=16 \middle\fi|;}

\newtheorem{lemma}{Lemma}

\newcommand{\tr}{\mathrm{tr}}
\newcommand{\ra}{\rightarrow}
\newcommand{\lan}{\langle}
\newcommand{\ran}{\rangle}
\newcommand{\norm}[1]{\left\lVert#1\right\rVert}
\newcommand{\inn}[1]{\lan#1\ran}
\newcommand{\ol}{\overline}
\begin{document}
\noindent Q3: If we let that $\gamma(t) = (\gamma_1(t),\gamma_2(t))$, then we note that since $|\gamma(t)|^2=1$, then $\frac{d}{dt}|\gamma(t)|^2=0$. For the two vectors in $T_{\gamma(t)}\R^2$, $(\gamma(t),\gamma(t))$ and $(\gamma(t),\gamma^\prime(t))$ we compute their inner product as 
\begin{align*}
\inn{(\gamma(t),\gamma(t)),(\gamma(t),\gamma^\prime(t))} & = \inn{\gamma(t),\gamma^\prime(t)}
\\ & = \gamma_1(t)\cdot \gamma_1^\prime(t) + \gamma_2(t)\cdot \gamma^\prime(t)
\\ & = \frac{1}{2}\frac{d}{dt}|\gamma(t)|^2
\\ & =0
\end{align*}Therefore, these tangent vectors are perpendicular. 
\end{document}