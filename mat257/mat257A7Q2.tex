\documentclass[letterpaper]{article}
\usepackage[letterpaper,margin=1in,footskip=0.25in]{geometry}
\usepackage[utf8]{inputenc}
\usepackage{amsmath}
\usepackage{amsthm}
\usepackage{amssymb, pifont}
\usepackage{mathrsfs}
\usepackage{enumitem}
\usepackage{fancyhdr}
\usepackage{hyperref}

\pagestyle{fancy}
\fancyhf{}
\rhead{MAT 257}
\lhead{Assignment 6}
\rfoot{Page \thepage}

\setlength\parindent{24pt}
\renewcommand\qedsymbol{$\blacksquare$}

\DeclareMathOperator{\Prt}{\mathbb{P}}
\DeclareMathOperator{\R}{\mathbb{R}}
\DeclareMathOperator{\N}{\mathbb{N}}
\DeclareMathOperator{\Z}{\mathbb{Z}}
\DeclareMathOperator{\Q}{\mathbb{Q}}
\DeclareMathOperator{\C}{\mathbb{C}}
\DeclareMathOperator{\ep}{\varepsilon}
\DeclareMathOperator{\identity}{\mathbf{0}}
\DeclareMathOperator{\card}{card}
\newcommand{\suchthat}{;\ifnum\currentgrouptype=16 \middle\fi|;}

\newtheorem{lemma}{Lemma}

\newcommand{\tr}{\mathrm{tr}}
\newcommand{\ra}{\rightarrow}
\newcommand{\lan}{\langle}
\newcommand{\ran}{\rangle}
\newcommand{\norm}[1]{\left\lVert#1\right\rVert}
\newcommand{\inn}[1]{\lan#1\ran}
\newcommand{\ol}{\overline}
\begin{document}
Q2:\\ $"\implies"$ Suppose that $f$ is integrable, and let $P$ be a partition of $A$. Since $f$ integrable, by Spivak Theorem 3-8, it is continuous except on a set of measure 0. Let $E$ be such set. For any $S\in P$ we claim that $f$ is continous on it except perhaps on a set of measure 0.
 Clearly if $S$ is disjoint from $E$, then $f|_S$ will be continous on $S$ and hence integrable. If $S$ contains any points from $E$, then $S\cap E$ will be of measure 0 since any subset of a set of measure 0 is also measure 0. 
 Hence $f|_S$ will be integrable on S. We now claim that $\int_A f = \sum_{S\in P} \int_S f|_S$. If we take $\chi_S$ to be the characteristic function of $S\in P$, we compute 
\begin{align*}
   & \int_A f
   \\ & = \int_A \sum_{S\in P} \chi_S \cdot f \tag*{rewriting f as sum of its restrictions}
   \\ & = \sum_{S\in P} \int_A \chi_S f \tag*{by question 1}
   \\ & = \sum_{S\in P} \int_S f|_S
\end{align*} As desired. 
\\ $"\impliedby"$ Suppose that $P$ is a partition of $A$, and each $f|_S$ is integrable on $S$ for each $S\in P$. By spivak 3-8, we have that each $f|_S$ is continuous except on some measure 0 set $E_S$. Let $E = \bigcup_{S\in E} E_S$. By spivak theorem 3-4, $E$ will have measure 0. We can now express $f$ in terms of each $f|_S$. We define $\tilde{f}|_S:A\rightarrow \R$ as any function which is equal to $f|_S$ on $S$. 
We can write $f = \sum_{S\in P} \chi_S \tilde{f}|_S$, where $\chi_S$ is the characterstic function of $S$. Thus $f$ will be continous except on $E$, a set of measure 0 and perhaps along the finite union of the boundaries of each $S$.
 From discussion in class, we know that the boundary of a rectangle is of measure 0 so the union over all the boundaries of subrectangles of $P$ will be measure 0. Hence $f$ will be integrable. Using the linearity of the integral shown in question 1, we compute 
 $\int_A f$ as $$\int_A f = \int_A \sum_{S\in P} \chi_S \tilde{f}|_S = \sum_{S\in P}\int_A \chi_S \tilde{f}|_S = \sum_{S\in P } \int_S f|_S$$
\end{document}