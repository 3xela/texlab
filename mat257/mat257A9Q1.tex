\documentclass[letterpaper]{article}
\usepackage[letterpaper,margin=1in,footskip=0.25in]{geometry}
\usepackage[utf8]{inputenc}
\usepackage{amsmath}
\usepackage{amsthm}
\usepackage{amssymb, pifont}
\usepackage{mathrsfs}
\usepackage{enumitem}
\usepackage{fancyhdr}
\usepackage{hyperref}

\pagestyle{fancy}
\fancyhf{}
\rhead{MAT 257}
\lhead{Assignment 9}
\rfoot{Page \thepage}

\setlength\parindent{24pt}
\renewcommand\qedsymbol{$\blacksquare$}

\DeclareMathOperator{\U}{\mathcal{U}}
\DeclareMathOperator{\Prt}{\mathbb{P}}
\DeclareMathOperator{\R}{\mathbb{R}}
\DeclareMathOperator{\N}{\mathbb{N}}
\DeclareMathOperator{\Z}{\mathbb{Z}}
\DeclareMathOperator{\Q}{\mathbb{Q}}
\DeclareMathOperator{\C}{\mathbb{C}}
\DeclareMathOperator{\ep}{\varepsilon}
\DeclareMathOperator{\identity}{\mathbf{0}}
\DeclareMathOperator{\card}{card}
\newcommand{\suchthat}{;\ifnum\currentgrouptype=16 \middle\fi|;}

\newtheorem{lemma}{Lemma}

\newcommand{\tr}{\mathrm{tr}}
\newcommand{\ra}{\rightarrow}
\newcommand{\lan}{\langle}
\newcommand{\ran}{\rangle}
\newcommand{\norm}[1]{\left\lVert#1\right\rVert}
\newcommand{\inn}[1]{\lan#1\ran}
\newcommand{\ol}{\overline}
\begin{document}
\noindent Q1a: \\ We define $f$ in the following way. 
$$ f(x) =   \left\{
    \begin{array}{l}
     \frac{3(-1)^n}{n} \;\text{if}\; x \in [n+\frac{1}{3},n+\frac{2}{3}] \text{for }n\in \N\\
     0 \; \text{if } x<1 \text{ or } x\in (n-\frac{1}{3},n+\frac{1}{3}) \text{ for } n\in \N
    \end{array}
  \right.$$ 
It is readily available that the support of $f$ is as desired, by construction.  It suffices to check that $\int_{n+\frac{1}{3}}^{n+\frac{2}{3}} f = \frac{(-1)^n}{n}$. 
$$ \int_{n+\frac{1}{3}}^{n+\frac{2}{3}} f = \int_{n+\frac{1}{3}}^{n+\frac{2}{3}} \frac{3(-1)^n}{n} dx = (n+ \frac{2}{3} - n - \frac{2}{3}) \cdot \frac{3(-1)^n}{n} = \frac{(-1)^n}{n}$$
\\ 1b: \\
Choose an open cover $\U$ of $\R$ in the following way. Each $A_n = [n+\frac{1}{3},n+\frac{2}{3}]$ has some $U_n\in \U$ with $[n+\frac{1}{3},n+\frac{2}{3}]\subset U_n$, and each $U_n$ disjoint from one another. We cover the rest of $\R$ in any collection of open sets such that these sets will be disjoint from each $A_n$. Find a $PO1$ $\Phi$ subordinate to $\U$. We note that $\sum_{\phi\in \Phi} \int_{R \supset U} \varphi \cdot |f| = \sum_{i=1}^\infty \frac{1}{n}$. 
This is a divergent series, hence $\int_{\R} f$ is not integrable. 
\\ 1c: \\
For each $A_n = [n+\frac{1}{3},n+\frac{2}{3}]$ choose an open cover $U_n$ which contains it disjoint from any other $U_i$. We take $V_n$ to cover sets of the form $(n-\frac{1}{3},n+\frac{1}{3})$ and we say that $V_0 = (-\infty, 1)$. Together, the set of all $U_n$ and $V_n$ cover all of $\R$.  Consider the following rearrangement of the alternating harmonic series. 
We rearrange as $$-1 + \frac{1}{2}  - \frac{1}{3} + \frac{1}{4} \dots = (-1 + \frac{1}{2} + \frac{1}{4}) -\frac{1}{3} \dots$$
Where we rearrange each term to be in groups of 3 of the form $\frac{1}{2k-1}-\frac{1}{2(2k-1)}-\frac{1}{4k}$. From MAT157 this will absolutely converge. Choose a PO1 $\Phi = \{ \phi_i\}$ such that each $\phi_{2i}$ corresponds to the $i'th$ term in our rearrangement of the series subordinate to an open set $U_i$, and each $\phi_{2k+1}$ is subordinate to a $V_i$. 
It follows that $\sum\int \phi_i |f|$ will converge, and $\sum \int \phi_i f = - \frac{1}{2}\log (2)$. We will now find a different rearrangement of our series. Consider the following rearrangement: 
$$-1 + \frac{1}{2} - \frac{1}{3} + \frac{1}{4} \dots = (-1  - \frac{1}{3} - \frac{1}{5}) + (\frac{1}{2} + \frac{1}{4}\dots)$$ Where the negative terms appear consecutively 3 times in a row, and the positive terms appear 7 times in a row. This sequence will absolutely converge, for simliar reason as the first. We also have that its limit will be $-(\log(2) + \log(\frac{3}{7}))$. 
Now consider another PO1, $\Psi = \{\psi_j \}$ where each $\psi_i$ is chosen so that $\psi_i$ will correspond to $A_i$. WLOG we may consider this specific PO1, since it is only defined on the support of the function $f $.We see that $\sum \int \psi_i f = -(\log(2)+ \log(\frac{3}{7}))$
\end{document}