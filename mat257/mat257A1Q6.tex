\documentclass[letterpaper]{article}
\usepackage[letterpaper,margin=1in,footskip=0.25in]{geometry}
\usepackage[utf8]{inputenc}
\usepackage{amsmath}
\usepackage{amsthm}
\usepackage{amssymb, pifont}
\usepackage{mathrsfs}
\usepackage{enumitem}
\usepackage{fancyhdr}
\usepackage{hyperref}

\pagestyle{fancy}
\fancyhf{}
\rhead{MAT 257}
\lhead{Assignment 1}
\rfoot{Page \thepage}

\setlength\parindent{24pt}
\renewcommand\qedsymbol{$\blacksquare$}

\DeclareMathOperator{\R}{\mathbb{R}}
\DeclareMathOperator{\N}{\mathbb{N}}
\DeclareMathOperator{\Z}{\mathbb{Z}}
\DeclareMathOperator{\Q}{\mathbb{Q}}
\DeclareMathOperator{\C}{\mathbb{C}}
\DeclareMathOperator{\ep}{\varepsilon}
\DeclareMathOperator{\identity}{\mathbf{0}}
\DeclareMathOperator{\card}{card}
\newcommand{\suchthat}{;\ifnum\currentgrouptype=16 \middle\fi|;}

\newtheorem{lemma}{Lemma}

\newcommand{\tr}{\mathrm{tr}}
\newcommand{\ra}{\rightarrow}
\newcommand{\lan}{\langle}
\newcommand{\ran}{\rangle}
\newcommand{\norm}[1]{\left\lVert#1\right\rVert}
\newcommand{\inn}[1]{\lan#1\ran}
\newcommand{\ol}{\overline}
\begin{document}
Q6: Suppose not. That is, suppose that there exists an irrational number $q\in[0,1]$ but with $q \notin A$. Since $A$ is closed, it follows that
$\R \setminus A $ is open. Therefore, there exists a $\epsilon > 0$ such that $(q-\epsilon, q+ \epsilon) \subset \R\setminus A$. By the density of the rational numbers in $\R$, there exists some $r\in \Q$ with $r \in (q-\epsilon, q+\epsilon) \subset \R \setminus A$. Note that since $q\in [0,1]$ and $r$ is within $\epsilon$ of $q$, $q\in[0,1]$. However, $q\in \R \setminus A \implies q \notin A$. We obtain a contradiction, since we assume that every rational between 0 and 1 is contained in $A$. $\qed$
\end{document}