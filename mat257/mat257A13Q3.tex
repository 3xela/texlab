\documentclass[letterpaper]{article}
\usepackage[letterpaper,margin=1in,footskip=0.25in]{geometry}
\usepackage[utf8]{inputenc}
\usepackage{amsmath}
\usepackage{amsthm}
\usepackage{amssymb, pifont}
\usepackage{mathrsfs}
\usepackage{enumitem}
\usepackage{fancyhdr}
\usepackage{hyperref}

\pagestyle{fancy}
\fancyhf{}
\rhead{MAT 257}
\lhead{Assignment 13}
\rfoot{Page \thepage}

\setlength\parindent{24pt}
\renewcommand\qedsymbol{$\blacksquare$}

\DeclareMathOperator{\T}{\mathcal{T}}
\DeclareMathOperator{\V}{\mathcal{V}}
\DeclareMathOperator{\U}{\mathcal{U}}
\DeclareMathOperator{\Prt}{\mathbb{P}}
\DeclareMathOperator{\R}{\mathbb{R}}
\DeclareMathOperator{\N}{\mathbb{N}}
\DeclareMathOperator{\Z}{\mathbb{Z}}
\DeclareMathOperator{\Q}{\mathbb{Q}}
\DeclareMathOperator{\C}{\mathbb{C}}
\DeclareMathOperator{\ep}{\varepsilon}
\DeclareMathOperator{\identity}{\mathbf{0}}
\DeclareMathOperator{\card}{card}
\newcommand{\suchthat}{;\ifnum\currentgrouptype=16 \middle\fi|;}

\newtheorem{lemma}{Lemma}

\newcommand{\tr}{\mathrm{tr}}
\newcommand{\ra}{\rightarrow}
\newcommand{\lan}{\langle}
\newcommand{\ran}{\rangle}
\newcommand{\norm}[1]{\left\lVert#1\right\rVert}
\newcommand{\inn}[1]{\lan#1\ran}
\newcommand{\ol}{\overline}
\begin{document}
\noindent Q3: For $\lambda \in \Lambda^{n-k} (V)$, we claim that $\psi_k(\lambda)$ which assigns $\lambda$ to $f_{\lambda}(\eta)$ defined by $f_{\lambda}(\eta)  = \chi(\lambda\wedge \eta)$, $\eta \in \Lambda^{k}(V)$ is our desired choice free linear isomorphism. We will first show linearity of $\psi_k$. For $\alpha\in \R$ and $\lambda_1,\lambda_2\in \Lambda^{n-k}(V)$, we compute 
\begin{align*}
    \psi_{k}(\alpha \lambda_1 + \lambda_2) & =f_{\alpha \lambda_1 + \lambda_2}(\eta )
    \\ & = \chi((\alpha \lambda_1 + \lambda_2) \wedge \eta)
    \\ & = \chi(\alpha\lambda_1 \wedge \eta + \lambda_2 \wedge \eta) \tag{by linearity of $\wedge$}
    \\ & = \alpha \chi(\lambda_1\wedge \eta) + \chi(\lambda_2 \wedge \eta) \tag{by linearity of $\chi$}
    \\ & = \alpha f_{\lambda_1}(\eta) + f_{\lambda_2}(\eta)
    \\ & = \alpha \psi_k (\lambda_1) + \psi_k(\lambda_2)
\end{align*}Thus $\psi_k$ is a linear mapping. It remains to show that it is a bijection between vector spaces. By the rank nullity theorem it is sufficient to show that $\psi_k$ is either injective or surjective. Suppose that $\psi_k(\lambda) = 0$. This is the same as saying $f_{\lambda}(\eta) = 0$ for all $\eta$. Equivalently, $\chi(\lambda \wedge \eta) = 0$ for all $\eta$. Since $\chi$ is a linear isomorphism, it must be that $\lambda \wedge \eta=0$ for all $\eta$. Therefore, we can conclude that $\lambda =0$. Therefore, $\psi_k$ has a trivial null space. Therefore it is injective and it follows that it is a linear isomorphism. 
\end{document}