\documentclass[letterpaper]{article}
\usepackage[letterpaper,margin=1in,footskip=0.25in]{geometry}
\usepackage[utf8]{inputenc}
\usepackage{amsmath}
\usepackage{amsthm}
\usepackage{amssymb, pifont}
\usepackage{mathrsfs}
\usepackage{enumitem}
\usepackage{fancyhdr}
\usepackage{hyperref}

\pagestyle{fancy}
\fancyhf{}
\rhead{MAT 257}
\lhead{Assignment 15}
\rfoot{Page \thepage}

\setlength\parindent{24pt}
\renewcommand\qedsymbol{$\blacksquare$}

\DeclareMathOperator{\T}{\mathcal{T}}
\DeclareMathOperator{\V}{\mathcal{V}}
\DeclareMathOperator{\U}{\mathcal{U}}
\DeclareMathOperator{\Prt}{\mathbb{P}}
\DeclareMathOperator{\R}{\mathbb{R}}
\DeclareMathOperator{\N}{\mathbb{N}}
\DeclareMathOperator{\Z}{\mathbb{Z}}
\DeclareMathOperator{\Q}{\mathbb{Q}}
\DeclareMathOperator{\C}{\mathbb{C}}
\DeclareMathOperator{\ep}{\varepsilon}
\DeclareMathOperator{\identity}{\mathbf{0}}
\DeclareMathOperator{\card}{card}
\newcommand{\suchthat}{;\ifnum\currentgrouptype=16 \middle\fi|;}

\newtheorem{lemma}{Lemma}

\newcommand{\bd}{\partial}
\newcommand{\tr}{\mathrm{tr}}
\newcommand{\ra}{\rightarrow}
\newcommand{\lan}{\langle}
\newcommand{\ran}{\rangle}
\newcommand{\norm}[1]{\left\lVert#1\right\rVert}
\newcommand{\inn}[1]{\lan#1\ran}
\newcommand{\ol}{\overline}
\begin{document}
\noindent Q1a: Using the definition of the exteriour derivative on 0-forms, we compute that 
$$df = \sum_{i=1}^3 \frac{\partial f}{\partial x_i} dx_i = \frac{\partial f}{\partial x}dx+ \frac{\partial f}{\partial y}dy + \frac{\partial f}{\partial z}dz = \omega_{\text{grad f}}^1$$
We now will evaluate $d(\omega_{F}^1)$ using the definition of the exteriour derivative: 
\begin{align*} 
d(\omega_F^1)  &= \sum_{i=1}^3 dx_i \wedge \frac{\partial \omega_F^1}{\partial x_i}
\\ & = dx \wedge (\frac{\partial F_1}{\partial x}dx + \frac{\partial F_2}{\partial x}dy + \frac{\partial F_3}{\partial x}dz) + dy \wedge (\frac{\partial F_1}{\partial y}dx + \frac{\partial F_2}{\partial y}dy + \frac{\partial F_3}{\partial y}dz) + dz \wedge (\frac{\partial F_1}{\partial z}dx + \frac{\partial F_2}{\partial z}dy + \frac{\partial F_3}{\partial z}dz)
\\ & = \frac{\partial F_2}{\partial x} dx \wedge dy + \frac{\partial F_3}{\partial x} dx\wedge dz + \frac{\partial F_1}{\partial y}dy \wedge dx + \frac{\partial F_3}{\partial y}dy \wedge dz + \frac{\partial F_1}{\partial z}dz \wedge dx + \frac{\partial F_2}{\partial z}dz \wedge dy
\\ & = (\frac{\partial F_3}{\partial y} - \frac{\partial F_2}{\partial z})dy \wedge dz + (\frac{\partial F_1}{\partial z} - \frac{\partial F_3}{\partial x})dz \wedge dx + (\frac{\partial F_2}{\partial x} - \frac{\partial F_1}{\partial y})dx \wedge dy
\\ & = \omega_{\text{curl F}}^2
\end{align*}
Finally we compute $d(\omega_F^2)$. Using the definition of $d$ we see that 
\begin{align*}
d(\omega_F^2) & = \sum_{i=1}^3 dx_i \wedge \frac{\bd \omega_F^2}{\bd x_i} 
\\ & =  dx \wedge (\frac{\bd F_1}{\bd x} dy\wedge dz + \frac{\bd F_2}{\bd x} dz \wedge dx + \frac{\bd F_3}{\bd x} dx \wedge dy) + dy\wedge (\frac{\bd F_1}{\bd y}dy \wedge dz + \frac{\bd F_2}{\bd y}dx \wedge dz + \frac{\bd F_3}{\bd y}dx \wedge dy)
\\ & + dz\wedge(\frac{\bd F_1}{\bd z}dy \wedge dz + \frac{\bd F_2}{\bd z}dz \wedge dx + \frac{\bd F_3}{\bd z}dx \wedge dy )
\\ & = \frac{\bd F_1}{\bd x}dx \wedge dy \wedge dz + \frac{\bd F_2}{\bd y} dx \wedge dy \wedge dz + \frac{\bd F_3}{\bd z}dx \wedge dy \wedge dz\wedge
\\ & = (\frac{\bd F_1}{\bd x} + \frac{\bd F_2}{\bd y} + \frac{\bd F_3}{\bd z})dx\wedge dy \wedge dz
\\ & = (\text{div F})dx \wedge dy \wedge dz
\end{align*}
As desired. 
\newline \\ Q1b: We will now show that gradient and curl fields are closed. Using the identity that $d^2=0$, we see that 
$$0 = d^2 f = d(\omega_{\text{grad f}}^1) = \omega_{\text{grad curl f}}^2$$
Since $dy\wedge dz,dz\wedge dx,dx\wedge dy$ form a basis, by linear independence the coefficient functions must be 0 and we conclude the curl of a gradient field is 0.
Now by a similar computation, we see that 
$$0 = d^2F =d(\omega_{\text{curl F}}^2) = (\text{div curl F})dx\wedge dy \wedge dz$$
Hence this is a 0 3-form and we conclude that the divergence of a curl field is 0. 
\end{document}