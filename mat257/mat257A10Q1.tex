\documentclass[letterpaper]{article}
\usepackage[letterpaper,margin=1in,footskip=0.25in]{geometry}
\usepackage[utf8]{inputenc}
\usepackage{amsmath}
\usepackage{amsthm}
\usepackage{amssymb, pifont}
\usepackage{mathrsfs}
\usepackage{enumitem}
\usepackage{fancyhdr}
\usepackage{hyperref}

\pagestyle{fancy}
\fancyhf{}
\rhead{MAT 257}
\lhead{Assignment 10}
\rfoot{Page \thepage}

\setlength\parindent{24pt}
\renewcommand\qedsymbol{$\blacksquare$}

\DeclareMathOperator{\U}{\mathcal{U}}
\DeclareMathOperator{\Prt}{\mathbb{P}}
\DeclareMathOperator{\R}{\mathbb{R}}
\DeclareMathOperator{\N}{\mathbb{N}}
\DeclareMathOperator{\Z}{\mathbb{Z}}
\DeclareMathOperator{\Q}{\mathbb{Q}}
\DeclareMathOperator{\C}{\mathbb{C}}
\DeclareMathOperator{\ep}{\varepsilon}
\DeclareMathOperator{\identity}{\mathbf{0}}
\DeclareMathOperator{\card}{card}
\newcommand{\suchthat}{;\ifnum\currentgrouptype=16 \middle\fi|;}

\newtheorem{lemma}{Lemma}

\newcommand{\tr}{\mathrm{tr}}
\newcommand{\ra}{\rightarrow}
\newcommand{\lan}{\langle}
\newcommand{\ran}{\rangle}
\newcommand{\norm}[1]{\left\lVert#1\right\rVert}
\newcommand{\inn}[1]{\lan#1\ran}
\newcommand{\ol}{\overline}
\begin{document}
\noindent Q1:\\
First letting $A = (0,a)\times (0,\frac{\pi}{2}) \times (0, 2\pi)$ we see that $g(A)=V\setminus C$, for some content 0 set $C$. Thus by COV $\int_{g(A)} z = \int_A z\circ g \cdot |det g^\prime|$. We see that $z\circ g = r sin\phi$
Computing $g^\prime $ get $$ g^\prime =  \begin{bmatrix}  \cos\theta\cos\phi & -r \sin\phi\cos\theta & -r\cos\phi \sin\theta \\ \cos\phi\sin\theta & -r \sin\phi \sin\theta & r\cos\phi \cos\theta  \\ \sin\phi & r\cos\phi & 0 \end{bmatrix}$$
We have that $|\det g^\prime| = r^2 \cos\phi$. This will be nonzero on the domain of $g$, so we can apply COV. 
We evaluate: 
\begin{align*}
     \int_{g(A)} z &= \int_A z\circ g |\det g^\prime| \tag{by COV}
    \\ & = \int_A r^3 sin\phi cos\phi
    \\ & = \int_0^a \int_0^{2\pi} \int_0^{\frac{\pi}{2}} r^3 \sin\phi \cos\phi \quad d\phi d\theta dr \tag{by Fubini's Theorem}
    \\ & = \int_0^a \int_0^{2\pi} r^3 \frac{\sin^2 \phi}{2}\bigg|_0^\frac{\pi}{2} \quad d\theta dr 
    \\ & = \int_0^a \int_0^{2\pi} \frac{r^3}{2} \quad d\theta dr 
    \\ & = \int_0^a \pi r^3 dr
    \\ & = \frac{\pi a^4}{4}
\end{align*}
\end{document}
