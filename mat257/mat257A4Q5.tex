\documentclass[letterpaper]{article}
\usepackage[letterpaper,margin=1in,footskip=0.25in]{geometry}
\usepackage[utf8]{inputenc}
\usepackage{amsmath}
\usepackage{amsthm}
\usepackage{amssymb, pifont}
\usepackage{mathrsfs}
\usepackage{enumitem}
\usepackage{fancyhdr}
\usepackage{hyperref}

\pagestyle{fancy}
\fancyhf{}
\rhead{MAT 257}
\lhead{Assignment 4}
\rfoot{Page \thepage}

\setlength\parindent{24pt}
\renewcommand\qedsymbol{$\blacksquare$}

\DeclareMathOperator{\R}{\mathbb{R}}
\DeclareMathOperator{\N}{\mathbb{N}}
\DeclareMathOperator{\Z}{\mathbb{Z}}
\DeclareMathOperator{\Q}{\mathbb{Q}}
\DeclareMathOperator{\C}{\mathbb{C}}
\DeclareMathOperator{\ep}{\varepsilon}
\DeclareMathOperator{\identity}{\mathbf{0}}
\DeclareMathOperator{\card}{card}
\newcommand{\suchthat}{;\ifnum\currentgrouptype=16 \middle\fi|;}

\newtheorem{lemma}{Lemma}

\newcommand{\tr}{\mathrm{tr}}
\newcommand{\ra}{\rightarrow}
\newcommand{\lan}{\langle}
\newcommand{\ran}{\rangle}
\newcommand{\norm}[1]{\left\lVert#1\right\rVert}
\newcommand{\inn}[1]{\lan#1\ran}
\newcommand{\ol}{\overline}
\begin{document} 
Q5:\\ Suppose that $f$ is homogenous of degree $m$. Then we have that $f(tx) = t^m f(x)$. Taking the derivative with respect to $t$ on both sides we see that
\begin{align*}
   & \frac{\partial f(tx)}{\partial t}
   \\ & = f^\prime (tx) \cdot \frac{\partial tx}{t} \text{ (from chain rule)}
   \\ & = f^\prime (tx) \cdot x
   \\ & = \sum_{i=1}^n D_i f(tx)\cdot x_i
\end{align*} From simple differentiation in 1 dimension, $\frac{\partial (t^m f(x))}{\partial t} = m t^{m-1}f(x)$. Thus we get that $$mt^{m-1}f(x) = \sum_{i=1}^n D_i f(tx) x_i$$ Choosing $t=1$ gives the desired result. 
\end{document}