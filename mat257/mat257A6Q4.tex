\documentclass[letterpaper]{article}
\usepackage[letterpaper,margin=1in,footskip=0.25in]{geometry}
\usepackage[utf8]{inputenc}
\usepackage{amsmath}
\usepackage{amsthm}
\usepackage{amssymb, pifont}
\usepackage{mathrsfs}
\usepackage{enumitem}
\usepackage{fancyhdr}
\usepackage{hyperref}

\pagestyle{fancy}
\fancyhf{}
\rhead{MAT 257}
\lhead{Assignment 6}
\rfoot{Page \thepage}

\setlength\parindent{24pt}
\renewcommand\qedsymbol{$\blacksquare$}

\DeclareMathOperator{\R}{\mathbb{R}}
\DeclareMathOperator{\N}{\mathbb{N}}
\DeclareMathOperator{\Z}{\mathbb{Z}}
\DeclareMathOperator{\Q}{\mathbb{Q}}
\DeclareMathOperator{\C}{\mathbb{C}}
\DeclareMathOperator{\ep}{\varepsilon}
\DeclareMathOperator{\identity}{\mathbf{0}}
\DeclareMathOperator{\card}{card}
\newcommand{\suchthat}{;\ifnum\currentgrouptype=16 \middle\fi|;}

\newtheorem{lemma}{Lemma}

\newcommand{\tr}{\mathrm{tr}}
\newcommand{\ra}{\rightarrow}
\newcommand{\lan}{\langle}
\newcommand{\ran}{\rangle}
\newcommand{\norm}[1]{\left\lVert#1\right\rVert}
\newcommand{\inn}[1]{\lan#1\ran}
\newcommand{\ol}{\overline}
\begin{document}
Q4:\\ Since $Df$ has rank n, assume $WLOG$ that the last n columns of $Df$ are linearly independant. Define $F(x,y)=(x,f(x,y))$. This will be a function from $\R^{k+n}$ to $\R^{k+n}$. We have that $DF = \begin{bmatrix}  I & 0 \\ \frac{\partial f}{\partial (x_1 \dots x_k)} & \frac{\partial f}{\partial (x_{k+1}\dots x_{k+n})}\end{bmatrix}$. 
At the point $a$, $det DF = \frac{\partial f}{\partial (x_{k+1}\dots x_{k+n})} \neq 0$ since we assume that the columns of $\frac{\partial f}{\partial (x_{k+1} \dots x_{k+n})}$ are linearly independant. Since the differential of $F$ is invertible, we can apply the Inverse function theorem. So there must be an open neighbourhood $U \times V \ni a$ and an open neighbourhood $W_1 \times W_2 \ni (a_1, \dots a_k, 0\dots 0)$ along with a $G: W_1 \times W_2: \rightarrow U \times V$. 
We now show that for all $c\in W_2$, there is some $x\in U \times V$ with $f(x)=c$. First, note that $G(x,y) = (x,h(x,y))$ for some $h$. We see that $$(a_1,\dots a_k,c) = F(G(a_1,\dots a_k,c)) = F(a_1,\dots a_k, h(a_1,\dots a_k,c)) = (a_1,\dots a_k, f(a_1,\dots a_k,h(a_1,\dots a_k,c)))$$. We see that $c= f(a_1,\dots a_k,h(a_1,\dots a_k,c))$ as desired. 
\end{document}