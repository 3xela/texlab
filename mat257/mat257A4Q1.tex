\documentclass[letterpaper]{article}
\usepackage[letterpaper,margin=1in,footskip=0.25in]{geometry}
\usepackage[utf8]{inputenc}
\usepackage{amsmath}
\usepackage{amsthm}
\usepackage{amssymb, pifont}
\usepackage{mathrsfs}
\usepackage{enumitem}
\usepackage{fancyhdr}
\usepackage{hyperref}

\pagestyle{fancy}
\fancyhf{}
\rhead{MAT 257}
\lhead{Assignment 4}
\rfoot{Page \thepage}

\setlength\parindent{24pt}
\renewcommand\qedsymbol{$\blacksquare$}

\DeclareMathOperator{\R}{\mathbb{R}}
\DeclareMathOperator{\N}{\mathbb{N}}
\DeclareMathOperator{\Z}{\mathbb{Z}}
\DeclareMathOperator{\Q}{\mathbb{Q}}
\DeclareMathOperator{\C}{\mathbb{C}}
\DeclareMathOperator{\ep}{\varepsilon}
\DeclareMathOperator{\identity}{\mathbf{0}}
\DeclareMathOperator{\card}{card}
\newcommand{\suchthat}{;\ifnum\currentgrouptype=16 \middle\fi|;}

\newtheorem{lemma}{Lemma}

\newcommand{\tr}{\mathrm{tr}}
\newcommand{\ra}{\rightarrow}
\newcommand{\lan}{\langle}
\newcommand{\ran}{\rangle}
\newcommand{\norm}[1]{\left\lVert#1\right\rVert}
\newcommand{\inn}[1]{\lan#1\ran}
\newcommand{\ol}{\overline}
\begin{document}
Q1a:
\\ $$f(x,y) = \int_a^{x+y} g$$. By the Fundamental Theorem of Calculus, $$\frac{\partial f}{\partial x} = g(x+y)$$ and similarly $$\frac{\partial f}{\partial y} = g(x+y)$$
\\ 1b: \\ $$f(x,y) = \int_y^x g$$ By properties of integration, we can rewrite $f$ in the following way, for some $a\in (x,y)$ 
$$f(x,y) = \int_y^x g = \int_y^a g + \int_a^x g = - \int_a^y g + \int_a^x g$$
Applying FTC, we see that $$\frac{\partial f}{\partial x} = g(x)$$ and $$\frac{\partial f}{\partial y} = -g(y)$$
\\ 1c: \\
$$f(x,y) = \int_a^{xy} g$$ By applying both FTC and chain rule, we get $$\frac{\partial f}{\partial x} = g(xy)y$$ and $$\frac{\partial f}{\partial y} = g(xy)x$$
\\ 1d: \\
$$f(x,y) = \int_a^{\int_b^y g} g$$ By the FTC and chain rule, $$\frac{\partial f}{\partial x} =0$$ and $$\frac{\partial f}{\partial y} = g(\int_b^y g) \cdot g(y)$$
\end{document}