\documentclass[letterpaper]{article}
\usepackage[letterpaper,margin=1in,footskip=0.25in]{geometry}
\usepackage[utf8]{inputenc}
\usepackage{amsmath}
\usepackage{amsthm}
\usepackage{amssymb, pifont}
\usepackage{mathrsfs}
\usepackage{enumitem}
\usepackage{fancyhdr}
\usepackage{hyperref}

\pagestyle{fancy}
\fancyhf{}
\rhead{MAT 257}
\lhead{Assignment 1}
\rfoot{Page \thepage}

\setlength\parindent{24pt}
\renewcommand\qedsymbol{$\blacksquare$}

\DeclareMathOperator{\R}{\mathbb{R}}
\DeclareMathOperator{\N}{\mathbb{N}}
\DeclareMathOperator{\Z}{\mathbb{Z}}
\DeclareMathOperator{\Q}{\mathbb{Q}}
\DeclareMathOperator{\C}{\mathbb{C}}
\DeclareMathOperator{\ep}{\varepsilon}
\DeclareMathOperator{\identity}{\mathbf{0}}
\DeclareMathOperator{\card}{card}
\newcommand{\suchthat}{;\ifnum\currentgrouptype=16 \middle\fi|;}

\newtheorem{lemma}{Lemma}

\newcommand{\tr}{\mathrm{tr}}
\newcommand{\ra}{\rightarrow}
\newcommand{\lan}{\langle}
\newcommand{\ran}{\rangle}
\newcommand{\norm}[1]{\left\lVert#1\right\rVert}
\newcommand{\inn}[1]{\lan#1\ran}
\newcommand{\ol}{\overline}
\begin{document}
Q5: \newline
"$\subset $"
\newline Suppose that point $x\in Bd A$. By definition, $x \in \ol A \setminus \text{int} A$. We want to show that $\ol A \subset [0,1]$. It suffices to show that $\text{ext}[0,1] \subset \text{ext}A$. Suppose that $y\in \text{ext} [0,1]$. Clearly, it must be that $y<0$ or $y>1$. Letting $\epsilon  = min(|y-1|,|y|)$ taking $R= (y-\epsilon, y+\epsilon)$will ensure that the rectangle is disjoint from $A$. Hence, $y\in \text{ext}A$ and so $\text{ext}[0,1] \subset \text{ext} A $ . Equivalently, $\ol A \subset [0,1]$
 Since $A$ is the union of open intervals, it is an open set as well, so $\text{int}A = A$. 
 It follows that $\ol A \setminus \text{int}A \subset [0,1] \setminus A$, and so $x \in [0,1] \setminus A$. Therfore $\text{Bd} A \subset [0,1] \setminus A$
\newline "$\supset$"
\newline Suppose that $x \in [0,1]\setminus A$. Consider the open set $U = (x-\epsilon, x+ \epsilon)$ for $\epsilon > 0$ . By the density of the rational numbers in $\R$, there exists some rational number $r \in (x-\epsilon, x+ \epsilon)$. By the definition of $A$, there must exist an $i$ such that $r \in (a_i,b_i)\subset A$. From our choice of $x$ we see that $U \cap \R\setminus A \neq \emptyset$ and $U \cap A \neq \emptyset$. 
This is exactly what it means to be in the boundary of $A$. Thus $\text{Bd} A \supset [0,1]\setminus A \qed$ 
\end{document}