\documentclass[letterpaper]{article}
\usepackage[letterpaper,margin=1in,footskip=0.25in]{geometry}
\usepackage[utf8]{inputenc}
\usepackage{amsmath}
\usepackage{amsthm}
\usepackage{amssymb, pifont}
\usepackage{mathrsfs}
\usepackage{enumitem}
\usepackage{fancyhdr}
\usepackage{hyperref}

\pagestyle{fancy}
\fancyhf{}
\rhead{MAT 257}
\lhead{Assignment 11}
\rfoot{Page \thepage}

\setlength\parindent{24pt}
\renewcommand\qedsymbol{$\blacksquare$}

\DeclareMathOperator{\U}{\mathcal{U}}
\DeclareMathOperator{\Prt}{\mathbb{P}}
\DeclareMathOperator{\R}{\mathbb{R}}
\DeclareMathOperator{\N}{\mathbb{N}}
\DeclareMathOperator{\Z}{\mathbb{Z}}
\DeclareMathOperator{\Q}{\mathbb{Q}}
\DeclareMathOperator{\C}{\mathbb{C}}
\DeclareMathOperator{\ep}{\varepsilon}
\DeclareMathOperator{\identity}{\mathbf{0}}
\DeclareMathOperator{\card}{card}
\newcommand{\suchthat}{;\ifnum\currentgrouptype=16 \middle\fi|;}

\newtheorem{lemma}{Lemma}

\newcommand{\tr}{\mathrm{tr}}
\newcommand{\ra}{\rightarrow}
\newcommand{\lan}{\langle}
\newcommand{\ran}{\rangle}
\newcommand{\norm}[1]{\left\lVert#1\right\rVert}
\newcommand{\inn}[1]{\lan#1\ran}
\newcommand{\ol}{\overline}
\begin{document}
\noindent Q2: Since $V$ a 3 dimensional vector space we have from basic linear algebra that $V^*$ will also be 3 dimensional.
It suffices to check that $\phi_{-1},\phi_0,\phi_1$ either span $V^*$ or are linearly independant. We will show linear independance. We will denote $p\in V$ as $p(x)=ax^2+bx+c$. Suppose that for some scalars $\alpha_1,\alpha_2,\alpha_3$,
$$\alpha_1\phi_{-1}(p) + \alpha_2 \phi_{0}(p) + \alpha_3 \phi_{1}(p)=0, \forall p\in V$$
Then we have that $$\alpha_1(a-b+c)+\alpha_2(c)+\alpha_3(a+b+c)=0$$ Re writing this expression get that 
$$a(\alpha_1+ \alpha_4) + b(\alpha_3-\alpha_3) +c(\alpha_2+\alpha_3)$$ Since this is true for all polynomials, we, taking $b=1=a,c=0$ we see that $\alpha_3=0$, taking $a=c=0,b=1$ gives us that $\alpha_1=\alpha_3=0$.
Finally, if we take $a=1$, we see that $\alpha_2 = -\alpha_1=0$. Thus we conclude this is a linearly independant list, and so it is a basis of $V^*$. 
We now will find a basis $\beta=(p_{-1},p_0,p_1)$ of $V$ so that $\beta^*=\gamma$. In other words, for each $\phi_i$, $\phi_i(p_j)=\delta_{ij}$. First consider $p_{-1}$. We require that $p_{-1}(-1)=1,p_{-1}(0)=p_{-1}(1)=0$. Choosing $p_{-1}(x) = \frac{1}{2}x^2  - \frac{1}{2}x$ will satisfy these properties. 
Setting $p_0(x)=-x^2+1$, we see that $p_0(-1)=p_0(1)=0$ and $p_0(0)=1$. Finally, setting $p_1(x) = \frac{1}{2}x^2 + \frac{1}{2}$ will give us the desired properties. Hence the basis $\beta = (\frac{1}{2}x^2-\frac{1}{2}x,-x^2+1,\frac{1}{2}x^2+\frac{1}{2}x)$ satisifes $\beta^* = \gamma$. 
\end{document}