\documentclass[letterpaper]{article}
\usepackage[letterpaper,margin=1in,footskip=0.25in]{geometry}
\usepackage[utf8]{inputenc}
\usepackage{amsmath}
\usepackage{amsthm}
\usepackage{amssymb, pifont}
\usepackage{mathrsfs}
\usepackage{enumitem}
\usepackage{fancyhdr}
\usepackage{hyperref}

\pagestyle{fancy}
\fancyhf{}
\rhead{MAT 257}
\lhead{Assignment 19}
\rfoot{Page \thepage}

\setlength\parindent{24pt}
\renewcommand\qedsymbol{$\blacksquare$}

\DeclareMathOperator{\s}{\mathcal{S}}
\DeclareMathOperator{\T}{\mathcal{T}}
\DeclareMathOperator{\V}{\mathcal{V}}
\DeclareMathOperator{\U}{\mathcal{U}}
\DeclareMathOperator{\Prt}{\mathbb{P}}
\DeclareMathOperator{\R}{\mathbb{R}}
\DeclareMathOperator{\N}{\mathbb{N}}
\DeclareMathOperator{\Z}{\mathbb{Z}}
\DeclareMathOperator{\Q}{\mathbb{Q}}
\DeclareMathOperator{\C}{\mathbb{C}}
\DeclareMathOperator{\ep}{\varepsilon}
\DeclareMathOperator{\identity}{\mathbf{0}}
\DeclareMathOperator{\card}{card}
\newcommand{\suchthat}{;\ifnum\currentgrouptype=16 \middle\fi|;}

\newtheorem{lemma}{Lemma}

\newcommand{\bd}{\partial}
\newcommand{\tr}{\mathrm{tr}}
\newcommand{\ra}{\rightarrow}
\newcommand{\lan}{\langle}
\newcommand{\ran}{\rangle}
\newcommand{\norm}[1]{\left\lVert#1\right\rVert}
\newcommand{\inn}[1]{\lan#1\ran}
\newcommand{\ol}{\overline}
\begin{document}
\noindent Q3a: By Stoke's Theorem, we know that $$\int_{\s_r^2} \omega = \int_{D_r^3}d\omega$$ Hence we compute that $$\int_{ D_d^3 \setminus in tD_c^3} d\omega  = \int_{\bd D_d^3}\omega - \int_{\bd int D_c^3}\omega = (a+\frac{b}{d}) - (a+\frac{b}{c}) = \frac{b}{d}- \frac{b}{c}$$
\newline \\ Q2b: Suppose that $\omega$ is closed. Then by Stokes' Theorem we see that 
$$a+ \frac{b}{r} = \int_{s^2_r} \omega = \int_{D_r^3} d\omega = \int_{D_r^3} 0 = 0$$
And we conclude that $a=b=0$. 
\newline \\ Q2c: Suppose that $d\eta =\omega$. Then by Stokes' Theorem we see that $$a+\frac{b}{r} = \int_{S_r^2} d\eta  = \int_{D_r^3} d^2\eta = \int_{D_r^3}0 =0$$
We conclude that $a=b=0$. 
\end{document}