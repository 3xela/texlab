\documentclass[letterpaper]{article}
\usepackage[letterpaper,margin=1in,footskip=0.25in]{geometry}
\usepackage[utf8]{inputenc}
\usepackage{amsmath}
\usepackage{amsthm}
\usepackage{amssymb, pifont}
\usepackage{mathrsfs}
\usepackage{enumitem}
\usepackage{fancyhdr}
\usepackage{hyperref}

\pagestyle{fancy}
\fancyhf{}
\rhead{MAT 257}
\lhead{Assignment 4}
\rfoot{Page \thepage}

\setlength\parindent{24pt}
\renewcommand\qedsymbol{$\blacksquare$}

\DeclareMathOperator{\R}{\mathbb{R}}
\DeclareMathOperator{\N}{\mathbb{N}}
\DeclareMathOperator{\Z}{\mathbb{Z}}
\DeclareMathOperator{\Q}{\mathbb{Q}}
\DeclareMathOperator{\C}{\mathbb{C}}
\DeclareMathOperator{\ep}{\varepsilon}
\DeclareMathOperator{\identity}{\mathbf{0}}
\DeclareMathOperator{\card}{card}
\newcommand{\suchthat}{;\ifnum\currentgrouptype=16 \middle\fi|;}

\newtheorem{lemma}{Lemma}

\newcommand{\tr}{\mathrm{tr}}
\newcommand{\ra}{\rightarrow}
\newcommand{\lan}{\langle}
\newcommand{\ran}{\rangle}
\newcommand{\norm}[1]{\left\lVert#1\right\rVert}
\newcommand{\inn}[1]{\lan#1\ran}
\newcommand{\ol}{\overline}
\begin{document} 
4a:\\ From the definition of $D_{e_i}f(a)$ we have that 
\begin{align*}
   & D_{e_{i}}f(a) = \lim_{h \rightarrow 0 } \frac{f(a+he_i)-f(a)}{h}
   \\ & = \lim_{h \rightarrow 0} \frac{f(a_1, \dots a_i + h , a_{i+1} \dots a_n ) - f(a)}{h}
   \\ & = D_i f(a) \text{ by the definition of partial derivative}
\end{align*}
\\ 4b: \\ Define the function $h$  as $h(z,t) = f(a+ztx)$ for some points $a$ and $x$. We have that 
\begin{align*}
    & \frac{\partial h}{\partial z}(0,t) = \lim_{s \rightarrow 0} \frac{h(s,t)-h(0,t)}{s}
    \\ & = \lim_{s \rightarrow 0 } \frac{f(a+stx)-f(a)}{s}
    \\ & = D_{tx}f(a)
\end{align*}
Now define $g(k) = f(a+kx)$. Letting $k(t) = zt$ for some $z\in \R$, we have $g(zt) = f(a+ztx)$. Clearly $g=h$ and so $\frac{\partial g}{\partial z} = \frac{\partial h}{\partial z}$. So by the chain rule we have that 
\begin{align*}
    & \frac{\partial g}{\partial z} (0,t)
    \\ &  = \frac{\partial g}{\partial k} \cdot \frac{\partial k}{\partial z}
    \\ &  = \lim_{m \rightarrow 0 } \frac{g(m)-g(0)}{m} \cdot \frac{\partial (z\cdot t)}{\partial z}
    \\ & = \lim_{m\rightarrow 0 } \frac{f(a+mx)-f(a)}{m} \cdot t
    \\ & = D_xf(a) t
\end{align*}Therefore we have that $D_{tx}f(a) = D_xf(a)\cdot t$
\\ 4c: \\ Let $z= a+kx$ for fixed $a,x\in \R^n$ and for $k\in \R$. The composition of differntiable functions is differntiable, so from chain rule we have that 
\begin{align*}
    & \frac{\partial f}{ \partial k} 
    \\ & = f ^\prime (z) \cdot \frac{\partial z}{\partial k}
    \\ & = Df(z)\cdot x
\end{align*} Evaluating at $k=0$ we have that $\frac{\partial f}{\partial k} = Df(a)\cdot x$. We now want to compute $\frac{\partial f}{ \partial k}$  at $k=0$ with limits. 
\begin{align*}
   & \frac{\partial f }{\partial k} = \lim_{h \rightarrow 0 } \frac{f(a+hx) - f(a)}{h}
   \\ & = D_x f(a) \text{ (by definition)}
\end{align*}So we have that $D_x f(a) = Df(a) \cdot x$. By linearity, $$D_{x+y}f(a) = D f(a)\cdot (x+y) = D f(a)\cdot x + D f(a)\cdot y = D_x f(a) + D_y f(a)$$
\end{document}