\documentclass[letterpaper]{article}
\usepackage[letterpaper,margin=1in,footskip=0.25in]{geometry}
\usepackage[utf8]{inputenc}
\usepackage{amsmath}
\usepackage{amsthm}
\usepackage{amssymb, pifont}
\usepackage{mathrsfs}
\usepackage{enumitem}
\usepackage{fancyhdr}
\usepackage{hyperref}

\pagestyle{fancy}
\fancyhf{}
\rhead{MAT 257}
\lhead{Assignment 18}
\rfoot{Page \thepage}

\setlength\parindent{24pt}
\renewcommand\qedsymbol{$\blacksquare$}

\DeclareMathOperator{\T}{\mathcal{T}}
\DeclareMathOperator{\V}{\mathcal{V}}
\DeclareMathOperator{\U}{\mathcal{U}}
\DeclareMathOperator{\Prt}{\mathbb{P}}
\DeclareMathOperator{\R}{\mathbb{R}}
\DeclareMathOperator{\N}{\mathbb{N}}
\DeclareMathOperator{\Z}{\mathbb{Z}}
\DeclareMathOperator{\Q}{\mathbb{Q}}
\DeclareMathOperator{\C}{\mathbb{C}}
\DeclareMathOperator{\ep}{\varepsilon}
\DeclareMathOperator{\identity}{\mathbf{0}}
\DeclareMathOperator{\card}{card}
\newcommand{\suchthat}{;\ifnum\currentgrouptype=16 \middle\fi|;}

\newtheorem{lemma}{Lemma}

\newcommand{\bd}{\partial}
\newcommand{\tr}{\mathrm{tr}}
\newcommand{\ra}{\rightarrow}
\newcommand{\lan}{\langle}
\newcommand{\ran}{\rangle}
\newcommand{\norm}[1]{\left\lVert#1\right\rVert}
\newcommand{\inn}[1]{\lan#1\ran}
\newcommand{\ol}{\overline}
\begin{document}
\noindent Q5: $\implies$
\newline \\ Suppose that an $n-1$ manifold $M$ is orientable. Let $\mu_a$ be an orientation of $T_a\R^{n-1}$. Then we know that for all coordinate patches $f$, $f_\ast \mu_a = f_\ast \mu_b$. Now let $a\in W \subset \R^n$ corresponding to some $f:W\to M$. Consider the matrix given by $$A = \begin{bmatrix} Df(a)\cdot e_1 \\ Df(a)\cdot e_2 \\ \vdots \\ Df(a)\cdot e_{n-1} \end{bmatrix}$$
This matrix is rank $n-1$ for all points $a$, and so by the fundamental theorem of linear maps it must have a kernel of dimension 1.  Define $n(f(a))$ to be the nontrivial vector in the kernel such that $$\det \begin{pmatrix} n(f(a)) \\ Df(a)\cdot e_1 \\ \vdots \\ Df(a)\cdot e_{n-1} \end{pmatrix}>0$$ Note that we can always make the determinant positive, by appropriately scaling $n(f(a))$ and that $\inn{n(a),Df(a)\cdot e_{i}}=0$ for all $i$. We now claim that the function $n(f(a))$ is a smooth vector field on $M$ which vanishes nowhere. We first note that to obtain $n(f(a))$, we must compute the nullspace of a matrix of smooth entries. Computing null spaces amounts to arithemetic which is smooth, so our function is a composition of smooth functions and hence is smooth. Note that as we change between different coordinate patches, we can precompose with a smooth transition map to ensure smoothness. Note that by construction, the normal vector will not vanish. This function is our desired $\nu$. 
\newline \\ $\impliedby$
\newline \\ Suppose that there is a consistent non-zero normal field $\nu$ to $M$ in $\R^n$. Let $p\in M$, define $\eta_p$ on $M$ by $$\eta_p (v_1, \dots v_{k-1}) = \omega(v_1 \dots v_{k-1}, \nu(p))$$ Where $\omega$ is a volume form on $\R^n$. We can see that our choice of $\eta_p$ does not vanish on $M$, since we have that $n(p)$ orthorgonal to all $v_i\in T_pM$. Hence we have a choice of a top form which does not vanish on $M$, so $M$ is orientable. 
\end{document}