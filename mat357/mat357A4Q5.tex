\documentclass[letterpaper]{article}
\usepackage[letterpaper,margin=1in,footskip=0.25in]{geometry}
\usepackage[utf8]{inputenc}
\usepackage{amsmath}
\usepackage{amsthm}
\usepackage{amssymb, pifont}
\usepackage{mathrsfs}
\usepackage{enumitem}
\usepackage{fancyhdr}
\usepackage{hyperref}

\pagestyle{fancy}
\fancyhf{}
\rhead{MAT 357}
\lhead{Assignment 4(done with Payam Fakoorziba)}
\rfoot{Page \thepage}

\setlength\parindent{24pt}
\renewcommand\qedsymbol{$\blacksquare$}

\DeclareMathOperator{\T}{\mathcal{T}}
\DeclareMathOperator{\V}{\mathcal{V}}
\DeclareMathOperator{\U}{\mathcal{U}}
\DeclareMathOperator{\Prt}{\mathbb{P}}
\DeclareMathOperator{\R}{\mathbb{R}}
\DeclareMathOperator{\N}{\mathbb{N}}
\DeclareMathOperator{\Z}{\mathbb{Z}}
\DeclareMathOperator{\Q}{\mathbb{Q}}
\DeclareMathOperator{\C}{\mathbb{C}}
\DeclareMathOperator{\ep}{\varepsilon}
\DeclareMathOperator{\identity}{\mathbf{0}}
\DeclareMathOperator{\card}{card}
\newcommand{\suchthat}{;\ifnum\currentgrouptype=16 \middle\fi|;}

\newtheorem{lemma}{Lemma}

\newcommand{\tr}{\mathrm{tr}}
\newcommand{\ra}{\rightarrow}
\newcommand{\lan}{\langle}
\newcommand{\ran}{\rangle}
\newcommand{\norm}[1]{\left\lVert#1\right\rVert}
\newcommand{\inn}[1]{\lan#1\ran}
\newcommand{\ol}{\overline}
\begin{document}
\noindent Q5: We first claim that for all $k\geq 0$, $f_n \to 0$. Let $\varepsilon>0$, then for any $x$, take $N > \frac{|x^k|-\varepsilon|x^2|}{\varepsilon}$. Then, if we take $n\geq N$, we see that $|f_n(x)|<\varepsilon$. Now notice if $f_n \rightrightarrows f$, then $f_n \rightarrow f$ as well. By uniqueness of limits, if sequence is uniformly convergent it must converge to 0. We claim that only for $k=0,1$ the sequence $f_n(x) = \frac{x^k}{x^2+1}$ converges uniformly. 
For $k=1$, take $N=\frac{1}{\varepsilon}$. Then if $n\geq N$, then we have that $|\frac{1}{x^2+n}|<\varepsilon$. For $k=1$, Take $N> \max\{ \frac{\varepsilon}{x} -x^2\}$. Then if $n>N$, $n>\frac{x}{\varepsilon}-x^2$, we have that $\frac{x}{x^2+n}<\varepsilon$. However, if $k>2$, we would need $\frac{x^k}{\varepsilon}-x^2<n$ to hold for all x, for any given $n$ sufficiently large. We see from properties of polynomials that this will not happen. We now claim that on any bounded subset of $\R$, for any $k$, $f_n$ converges. Without loss of generality, assume that $0<x<M$. Then $f_n(x) = \frac{x^k}{x^2+n}<\frac{M^k}{n}$. If we take $n$ sufficiently large we can make this as small as we desire. Hence $f_n$ uniformly converges to 0. 
\end{document}