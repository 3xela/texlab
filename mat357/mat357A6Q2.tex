\documentclass[letterpaper]{article}
\usepackage[letterpaper,margin=1in,footskip=0.25in]{geometry}
\usepackage[utf8]{inputenc}
\usepackage{amsmath}
\usepackage{amsthm}
\usepackage{amssymb, pifont}
\usepackage{mathrsfs}
\usepackage{enumitem}
\usepackage{fancyhdr}
\usepackage{hyperref}

\pagestyle{fancy}
\fancyhf{}
\rhead{MAT 357}
\lhead{Assignment 6}
\rfoot{Page \thepage}

\setlength\parindent{24pt}
\renewcommand\qedsymbol{$\blacksquare$}

\DeclareMathOperator{\F}{\mathbb{F}}
\DeclareMathOperator{\T}{\mathcal{T}}
\DeclareMathOperator{\V}{\mathcal{V}}
\DeclareMathOperator{\U}{\mathcal{U}}
\DeclareMathOperator{\Prt}{\mathbb{P}}
\DeclareMathOperator{\R}{\mathbb{R}}
\DeclareMathOperator{\N}{\mathbb{N}}
\DeclareMathOperator{\Z}{\mathbb{Z}}
\DeclareMathOperator{\Q}{\mathbb{Q}}
\DeclareMathOperator{\C}{\mathbb{C}}
\DeclareMathOperator{\ep}{\varepsilon}
\DeclareMathOperator{\identity}{\mathbf{0}}
\DeclareMathOperator{\card}{card}
\newcommand{\suchthat}{;\ifnum\currentgrouptype=16 \middle\fi|;}

\newtheorem{lemma}{Lemma}

\newcommand{\tr}{\mathrm{tr}}
\newcommand{\ra}{\rightarrow}
\newcommand{\lan}{\langle}
\newcommand{\ran}{\rangle}
\newcommand{\norm}[1]{\left\lVert#1\right\rVert}
\newcommand{\inn}[1]{\lan#1\ran}
\newcommand{\ol}{\overline}
\begin{document}
\noindent Q2: This is not true. Consider the function algebra generated by $\mathcal{A} = \{1,x^2, \dots x^n\}$, where we allow finite sums, products and scaling with real numbers between elements of $\mathcal{A}$ on $[0,1]$. Note that this is indeed a function algebra, since scaling a polynomial with no linear term will yield another polynomial with no linear term, and similarly for addition and multiplication. 
This will split points since $x^2$ is injective on $[0,1]$. It also vanishes nowhere since the constant function is a part of this algebra. Hence by the Stone Weierstrass Theorem, for each $\ep>0$ there is a $g= b_0 + b_2x^2 + \dots b_nx^n$ with $|f-g|<\ep$. 
Then observe: 
\begin{align*}
    \int_{0}^1 f^2 dx &= |\int_0^1 fg dx + \int_0^1 f(f-g)dx|
    \\ & \leq |\int_0^1fg dx| +\int_{0}^1 |f||g-f|dx
    \\ & = |\int_0^1 f(x)(b_0 + b_2x^2 + \dots + b_n x^n dx| + \int_{0}^1 |f||f-g| dx
    \\ & \leq 0 + sup|f|\ep
\end{align*} Since $\ep$ can be made arbitrarily small, we have that $\int_0^1f^2 dx=0$. Thus it is not possible that $\int_0^1 x\cdot f dx =1$. 
\end{document}