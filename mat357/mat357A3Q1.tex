\documentclass[letterpaper]{article}
\usepackage[letterpaper,margin=1in,footskip=0.25in]{geometry}
\usepackage[utf8]{inputenc}
\usepackage{amsmath}
\usepackage{amsthm}
\usepackage{amssymb, pifont}
\usepackage{mathrsfs}
\usepackage{enumitem}
\usepackage{fancyhdr}
\usepackage{hyperref}

\pagestyle{fancy}
\fancyhf{}
\rhead{MAT 357}
\lhead{Assignment 3 done with Payam Fakoorziba}
\rfoot{Page \thepage}

\setlength\parindent{24pt}
\renewcommand\qedsymbol{$\blacksquare$}

\DeclareMathOperator{\U}{\mathcal{U}}
\DeclareMathOperator{\Prt}{\mathbb{P}}
\DeclareMathOperator{\R}{\mathbb{R}}
\DeclareMathOperator{\N}{\mathbb{N}}
\DeclareMathOperator{\Z}{\mathbb{Z}}
\DeclareMathOperator{\Q}{\mathbb{Q}}
\DeclareMathOperator{\C}{\mathbb{C}}
\DeclareMathOperator{\ep}{\varepsilon}
\DeclareMathOperator{\identity}{\mathbf{0}}
\DeclareMathOperator{\card}{card}
\newcommand{\suchthat}{;\ifnum\currentgrouptype=16 \middle\fi|;}

\newtheorem{lemma}{Lemma}

\newcommand{\tr}{\mathrm{tr}}
\newcommand{\ra}{\rightarrow}
\newcommand{\lan}{\langle}
\newcommand{\ran}{\rangle}
\newcommand{\norm}[1]{\left\lVert#1\right\rVert}
\newcommand{\inn}[1]{\lan#1\ran}
\newcommand{\ol}{\overline}
\begin{document}
\noindent Q1a: We define $F(x):M\rightarrow M\times \R$ by $F(x) = (x,f(x))$. This is the composition of continuous functions, and hence is continuous. The image of $F$ is the graph of $f$, and is connected, since the image of a connected set under a continous mapping is connected.
\newline \\ Q1b: Consider the set $\{(x,\sin{\frac{1}{x}}) : x\in(0,1)\} \cup \{(0,0) \}$. This is a connected graph of the function $f(x) = \begin{cases}
    \sin(\frac{1}{x}), x\neq 0  \\ 0 , x=0
\end{cases}  $ We have that this graph is connected, yet $f$ is not continuous at $0$. 
\newline \\ Q1c: Let $a,b\in M$. Let $\gamma:[0,1]\to M$ be a path between $a$ and $b$. We define $F$ in the same way as in Q1a. Then for any $(a,f(a))$ and $(b,f(b))$ in the graph of $f$, we have that $F\circ \gamma$ will be a path. This will be continous, as it is the composition of continuous mappings. Thus the graph of $f$ is path connected. 
\newline \\ Q1d: Suppose that $\Gamma(f)$ the graph of $f$ is path connected. Suppose that $M$ has 2 points, $a,b$ and there does not exist some continuous path between them. Then there must exist some continuous $\gamma:[0,1]\to \Gamma(f)$ such that $\gamma(0)=(a,f(a))$ and $\gamma(1) = (b,f(b))$. However, if $\pi_M$ is defined to be the projection map onto $M$, then we can create a path from $a$ to $b$ in $M$ by composing $\pi_M \circ \gamma$. This contradicts that $M$ is not path connected. Now suppose that $\Gamma(f)$ path connected, but $f$ is not continuous.
If $\pi_{\R}$ is the projection onto the real space, we take note that $\pi_{\R} \circ \gamma$ is a continuous map with $\pi_{\R} \circ \gamma(0)=f(b)$ and $\pi_{\R} \circ \gamma(1)=f(b)$. Since $a,b$ were chosen arbitrarily, $f$ is continuous on all of $M$. A contradiction. 
\end{document}