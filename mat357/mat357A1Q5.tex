\documentclass[letterpaper]{article}
\usepackage[letterpaper,margin=1in,footskip=0.25in]{geometry}
\usepackage[utf8]{inputenc}
\usepackage{amsmath}
\usepackage{amsthm}
\usepackage{amssymb, pifont}
\usepackage{mathrsfs}
\usepackage{enumitem}
\usepackage{fancyhdr}
\usepackage{hyperref}

\pagestyle{fancy}
\fancyhf{}
\rhead{MAT 357}
\lhead{Assignment 1(done with Payam Fakoorziba)}
\rfoot{Page \thepage}

\setlength\parindent{24pt}
\renewcommand\qedsymbol{$\blacksquare$}

\DeclareMathOperator{\U}{\mathcal{U}}
\DeclareMathOperator{\Prt}{\mathbb{P}}
\DeclareMathOperator{\R}{\mathbb{R}}
\DeclareMathOperator{\N}{\mathbb{N}}
\DeclareMathOperator{\Z}{\mathbb{Z}}
\DeclareMathOperator{\Q}{\mathbb{Q}}
\DeclareMathOperator{\C}{\mathbb{C}}
\DeclareMathOperator{\ep}{\varepsilon}
\DeclareMathOperator{\identity}{\mathbf{0}}
\DeclareMathOperator{\card}{card}
\newcommand{\suchthat}{;\ifnum\currentgrouptype=16 \middle\fi|;}

\newtheorem{lemma}{Lemma}

\newcommand{\tr}{\mathrm{tr}}
\newcommand{\ra}{\rightarrow}
\newcommand{\lan}{\langle}
\newcommand{\ran}{\rangle}
\newcommand{\norm}[1]{\left\lVert#1\right\rVert}
\newcommand{\inn}[1]{\lan#1\ran}
\newcommand{\ol}{\overline}
\begin{document}
\noindent Q5: Let $f:C_{max}\rightarrow C_{int}$ by the identity function. It is clearly a bijection, since the sets $C_{max}$ and $C_{int}$ are identical as sets, only differing by the metric on them. 
We see $f$ is also linear, since if $g,h\in C_{max}$,$\lambda\in \R$ we see $f(\lambda g + h) = \lambda g + h = \lambda f(g)+f(h)$. It remains to show that it is continous. Let $\varepsilon > 0$, and take any $\delta <\varepsilon$. 
We see that $$max(|f(x)-g(x)|)<\delta \implies \int_0^1 |f(x)-g(x)| \leq \int_0^1 \delta  = \delta < \varepsilon$$
Hence the identity mapping is continuous. We now claim that the inverse map $f^{-1}:C_{int}\rightarrow C_{max}$ is not continuous. 
Take $f=0,g(x) = \frac{1}{(x+1)^n}$ for some $n\in\N$. Notice that $f$ and $g$ are both continuous on $[0,1]$. Take $\varepsilon = 1$, and $\delta >0$ We notice that $max(|f(x)-g(x)|)=1$ regardless of our choice of n, with the maximum occuring at $x=0$. 
Note as well that 
\begin{align*}
    \int_0^1 |0-\frac{1}{(x+1)^n}|dx & = \int_0^1 \frac{1}{(x+1)^n}dx
    \\ & = \frac{2^n-2}{(n-1)2^n}
    \\ & = \frac{(1-2^{n-1})}{(n-1)}
\end{align*}
We can make the quanitity $\frac{(1-2^{n-1})}{(n-1)}<\delta$ for sufficiently large n. Hence $f^-1$ is not continuous. 
\end{document}