\documentclass[letterpaper]{article}
\usepackage[letterpaper,margin=1in,footskip=0.25in]{geometry}
\usepackage[utf8]{inputenc}
\usepackage{amsmath}
\usepackage{amsthm}
\usepackage{amssymb, pifont}
\usepackage{mathrsfs}
\usepackage{enumitem}
\usepackage{fancyhdr}
\usepackage{hyperref}

\pagestyle{fancy}
\fancyhf{}
\rhead{MAT 357}
\lhead{Assignment 2(done with Payam Fakooirziba)}
\rfoot{Page \thepage}

\setlength\parindent{24pt}
\renewcommand\qedsymbol{$\blacksquare$}

\DeclareMathOperator{\U}{\mathcal{U}}
\DeclareMathOperator{\Prt}{\mathbb{P}}
\DeclareMathOperator{\R}{\mathbb{R}}
\DeclareMathOperator{\N}{\mathbb{N}}
\DeclareMathOperator{\Z}{\mathbb{Z}}
\DeclareMathOperator{\Q}{\mathbb{Q}}
\DeclareMathOperator{\C}{\mathbb{C}}
\DeclareMathOperator{\ep}{\varepsilon}
\DeclareMathOperator{\identity}{\mathbf{0}}
\DeclareMathOperator{\card}{card}
\newcommand{\suchthat}{;\ifnum\currentgrouptype=16 \middle\fi|;}

\newtheorem{lemma}{Lemma}

\newcommand{\tr}{\mathrm{tr}}
\newcommand{\ra}{\rightarrow}
\newcommand{\lan}{\langle}
\newcommand{\ran}{\rangle}
\newcommand{\norm}[1]{\left\lVert#1\right\rVert}
\newcommand{\inn}[1]{\lan#1\ran}
\newcommand{\ol}{\overline}
\begin{document}
\noindent Q5: We will reason contrapositively. Assume that $M$ is not a compact metric space. Thus there exists a sequence $(p_n)$ with no convergent subsequence. WLOG assume that each $p_i$ is distinct.
For each $n$, find a $r_n>0$ such that the neighbourhoods $M_{r_n}(p_n)$ are disjoint from one another and no sequence $q_n\in M_{r_n}(p_n)$ converges. We define $f_n(x)$ as: 
$$f_n(x) = \frac{r_n-d(x,p_n)}{\frac{r_n}{n} + d(x,p_n)}$$
We set $f(x)=f_n(x)$ if $x\in M_{r_n}(p_n)$ and $f(x)=0$ else. Observe that $f$ is a composition of continuous maps, hence is continuous. We see that $f(p_n) = \frac{r_n-d(p_n,p_n)}{\frac{r_n}{n}-d(p_n,p_n)} = \frac{r_n}{\frac{r_n}{n}} = n$. We see that $f(p_n)$ is unbounded. Thus we have proved the claim.  
\end{document}