\documentclass[letterpaper]{article}
\usepackage[letterpaper,margin=1in,footskip=0.25in]{geometry}
\usepackage[utf8]{inputenc}
\usepackage{amsmath}
\usepackage{amsthm}
\usepackage{amssymb, pifont}
\usepackage{mathrsfs}
\usepackage{enumitem}
\usepackage{fancyhdr}
\usepackage{hyperref}

\pagestyle{fancy}
\fancyhf{}
\rhead{MAT 357}
\lhead{Assignment 5(done with payam fakoorziba)}
\rfoot{Page \thepage}

\setlength\parindent{24pt}
\renewcommand\qedsymbol{$\blacksquare$}

\DeclareMathOperator{\F}{\mathbb{F}}
\DeclareMathOperator{\T}{\mathcal{T}}
\DeclareMathOperator{\V}{\mathcal{V}}
\DeclareMathOperator{\U}{\mathcal{U}}
\DeclareMathOperator{\Prt}{\mathbb{P}}
\DeclareMathOperator{\R}{\mathbb{R}}
\DeclareMathOperator{\N}{\mathbb{N}}
\DeclareMathOperator{\Z}{\mathbb{Z}}
\DeclareMathOperator{\Q}{\mathbb{Q}}
\DeclareMathOperator{\C}{\mathbb{C}}
\DeclareMathOperator{\ep}{\varepsilon}
\DeclareMathOperator{\identity}{\mathbf{0}}
\DeclareMathOperator{\card}{card}
\newcommand{\suchthat}{;\ifnum\currentgrouptype=16 \middle\fi|;}

\newtheorem{lemma}{Lemma}

\newcommand{\tr}{\mathrm{tr}}
\newcommand{\ra}{\rightarrow}
\newcommand{\lan}{\langle}
\newcommand{\ran}{\rangle}
\newcommand{\norm}[1]{\left\lVert#1\right\rVert}
\newcommand{\inn}[1]{\lan#1\ran}
\newcommand{\ol}{\overline}
\begin{document}
\noindent Q5: Let $A$ be the set of all polynomials $p(x)$ on $[a,b]$ such that $p^\prime(a) =0$. We claim that $A$ is a function algebra which vanishes nowhere and separates points. 
First note that this is indeed a function algebra, since it is closed under addition and scaling by the properties of differentiation. Similarly, by the product rule it is closed under multiplication. We now claim that $A$ vanishes nowhere. Let $p\in [a,b]$. We will construct an $f\in A$ which does not vanish at $p$. Let $p \in [a,b]$. Define $f(x) = (x-a)^2+c$ with constant $c$ chosen so that $c\neq - (p-a)^2$. We see that $f^\prime(a) = 0$ and $f$ does not vanish at $p$, hence this function algebra is nowhere vanishing. We will now show that $A$ separates points. Let $p_1,p_2 \in [a,b]$ be distinct points. We define $f(x) = (x-a)^2-(p_1-a)^2$. We see that $f$ belongs to $A$ and $f(p_1)=0 \neq f(p_2)$. We have that $A$ is a function algebra, which vanishes nowhere and separates points. Therefore, by the Stone-Weierstrass Theorem for each $\frac{\varepsilon}{2}>0$ there exists some $q(x)$ where $q^\prime(a)=0$ and $|q(x)-f(x)|<\frac{\varepsilon}{2}$. Let $\varepsilon_0 = q(a)-f(a)$. Now let $p(x) = q(x)-\varepsilon_0$. We have that $p^\prime(a)=0,p(a)=f(a)$ and $|p(x)-f(x)|<\varepsilon$.
\end{document}