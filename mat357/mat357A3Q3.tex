\documentclass[letterpaper]{article}
\usepackage[letterpaper,margin=1in,footskip=0.25in]{geometry}
\usepackage[utf8]{inputenc}
\usepackage{amsmath}
\usepackage{amsthm}
\usepackage{amssymb, pifont}
\usepackage{mathrsfs}
\usepackage{enumitem}
\usepackage{fancyhdr}
\usepackage{hyperref}

\pagestyle{fancy}
\fancyhf{}
\rhead{MAT 357}
\lhead{Assignment 3 done with Payam Fakoorziba}
\rfoot{Page \thepage}

\setlength\parindent{24pt}
\renewcommand\qedsymbol{$\blacksquare$}

\DeclareMathOperator{\V}{\mathcal{V}}
\DeclareMathOperator{\U}{\mathcal{U}}
\DeclareMathOperator{\Prt}{\mathbb{P}}
\DeclareMathOperator{\R}{\mathbb{R}}
\DeclareMathOperator{\N}{\mathbb{N}}
\DeclareMathOperator{\Z}{\mathbb{Z}}
\DeclareMathOperator{\Q}{\mathbb{Q}}
\DeclareMathOperator{\C}{\mathbb{C}}
\DeclareMathOperator{\ep}{\varepsilon}
\DeclareMathOperator{\identity}{\mathbf{0}}
\DeclareMathOperator{\card}{card}
\newcommand{\suchthat}{;\ifnum\currentgrouptype=16 \middle\fi|;}

\newtheorem{lemma}{Lemma}

\newcommand{\tr}{\mathrm{tr}}
\newcommand{\ra}{\rightarrow}
\newcommand{\lan}{\langle}
\newcommand{\ran}{\rangle}
\newcommand{\norm}[1]{\left\lVert#1\right\rVert}
\newcommand{\inn}[1]{\lan#1\ran}
\newcommand{\ol}{\overline}
\begin{document}
\noindent Q3: For each $x\in C$, we can rewrite it as a base 3 decimal expansion whose entries are either 0 or 2. Denote this by $x=(x_n)$. 
Consider the mapping $f: C \to [0,1]$ defined as $$f((x_n)_{base 3}) = (g(x_1),g(x_2) \dots)_{base 2} $$ Where $$g(x_i) =  \begin{cases} 
    1 & x_i = 2 \\
    0 & x_i = 0
 \end{cases} $$ 
We claim that $f$ is a continuous surjective map. Suppose that $y\in[0,1]$. Then it has some decimal expansion of the form $y=(a_n)$.
In base 2 this will be a string of 0's and 1's. If we take $x\in C$ to have a 2 where ever there is a 1 in the decimal expansion of $y$, with the 0's remaining unchanged, we have that $f(x)=y$.
We now claim that such an $f$ is continuous. Notice that $f$ can be represented as the composition of several continuous maps. First, we swap every 2 to a 1, then we convert from base 3 to base 2. 
Converting between number bases amounts to addition and multiplication, and thus is continuous. It remains to show that swapping 2's to 1's in a trinary number is a continuous function. Let $h$ be the function  on the Cantor set which changes a 2 to a 1. Since each element in the Cantor set has a trinary expansion with either a 2 or a 0, then 
$h(x)=\frac{x}{2}$. Thus $h$ is division by a nonzero and hence is continuous. 
\end{document}