\documentclass[letterpaper]{article}
\usepackage[letterpaper,margin=1in,footskip=0.25in]{geometry}
\usepackage[utf8]{inputenc}
\usepackage{amsmath}
\usepackage{amsthm}
\usepackage{amssymb, pifont}
\usepackage{mathrsfs}
\usepackage{enumitem}
\usepackage{fancyhdr}
\usepackage{hyperref}

\pagestyle{fancy}
\fancyhf{}
\rhead{MAT 357}
\lhead{Assignment 1(done with Payam Fakoorziba)}
\rfoot{Page \thepage}

\setlength\parindent{24pt}
\renewcommand\qedsymbol{$\blacksquare$}

\DeclareMathOperator{\U}{\mathcal{U}}
\DeclareMathOperator{\Prt}{\mathbb{P}}
\DeclareMathOperator{\R}{\mathbb{R}}
\DeclareMathOperator{\N}{\mathbb{N}}
\DeclareMathOperator{\Z}{\mathbb{Z}}
\DeclareMathOperator{\Q}{\mathbb{Q}}
\DeclareMathOperator{\C}{\mathbb{C}}
\DeclareMathOperator{\ep}{\varepsilon}
\DeclareMathOperator{\identity}{\mathbf{0}}
\DeclareMathOperator{\card}{card}
\newcommand{\suchthat}{;\ifnum\currentgrouptype=16 \middle\fi|;}

\newtheorem{lemma}{Lemma}

\newcommand{\tr}{\mathrm{tr}}
\newcommand{\ra}{\rightarrow}
\newcommand{\lan}{\langle}
\newcommand{\ran}{\rangle}
\newcommand{\norm}[1]{\left\lVert#1\right\rVert}
\newcommand{\inn}[1]{\lan#1\ran}
\newcommand{\ol}{\overline}
\begin{document}
\noindent Q3a:
Suppose that $V$ is a vector space whose inner product induces a norm. Let $x,y\in V$, consider the following chain of equalities: 
\begin{align*}
 \norm{x+y}^2 + \norm{x-y}^2 &= \inn{x+y,x+y} + \inn{x-y,x-y} \tag{by the defintion of the norm}
\\ & =  \inn{x,x} +\inn{x,y} + \inn{y,x} + \inn{y,y} + \inn{x,x} - \inn{x,y} -\inn{y,x} + \inn{y,y} \tag{by linearity}
\\ & = \inn{x,x} +\inn{x,x} + \inn{y,y} +\inn{y,y}
\\ & = 2\norm{x}^2 + 2\norm{y}^2 \tag{by definition of the norm}
\end{align*}
We obtain the desired equality, thus we are done. 
\newline \\ Q3b:
Suppose that $\norm{\cdot}:V\rightarrow \R$ is a norm on $V$ obeying the equality in $3a$. We claim that $\inn{x,y}:= \norm{\frac{x+y}{2}}^2 - \norm{\frac{x-y}{2}}^2$ is an inner product on $V$ uniquely determined by $\norm{\cdot}$. We first show that $\inn{x,y}$ satisfies all the properties of an inner product. 
First, note that $\inn{x,x} = \norm{\frac{x+x}{2}}^2 = \norm{x}^2\geq 0$, with equality holding iff $x=0$, by the properties of the norm. 
Next, we get that $\inn{x,y} = \norm{\frac{x+y}{2}}^2 - \norm{\frac{x-y}{2}}^2 = \norm{\frac{y+x}{2}}^2 - \norm{\frac{y-x}{2}}^2 = \inn{y,x}$. 
We skip the proof of bilinearity. We now claim uniqueness. Suppose that $\norm{\cdot}$ is a norm satisfying the parallelogram inequality, arising from two distinct inner products, $\inn{\cdot,\cdot}_1$ and $\inn{\cdot,\cdot}_2$, 
where $\inn{\cdot,\cdot}_1$ is defined as previously, and $\inn{\cdot,\cdot}_2= \norm{x}^2$
Then we have that for any $x,y\in V$ 
\begin{align*} 
\inn{x,y}_1 & =\norm{\frac{x+y}{2}}^2 -\norm{\frac{x-y}{2}}^2
\\ & =\frac{1}{4}[\norm{x+y}^2 -\norm{x-y}^2]
\\ & = \frac{1}{4}[\inn{x+y,x+y}_2-\inn{x-y,x-y}_2]
\\ & = \frac{1}{4}[\inn{x+x}_2 + 2\inn{x,y}_2 + \inn{y,y}_2 -\inn{x,x}_2 +\inn{x,y}_2-\inn{y,y}_2]
\\ & = \frac{1}{4}4\inn{x,y}_2
\\ & = \inn{x,y}_2
\end{align*}
Thus we have $\inn{x,y}_1=\inn{x,y}_2$ and we conclude that any norm obeying the parallelogram equality arises from a unique inner product.

\end{document}