\documentclass[letterpaper]{article}
\usepackage[letterpaper,margin=1in,footskip=0.25in]{geometry}
\usepackage[utf8]{inputenc}
\usepackage{amsmath}
\usepackage{amsthm}
\usepackage{amssymb, pifont}
\usepackage{mathrsfs}
\usepackage{enumitem}
\usepackage{fancyhdr}
\usepackage{hyperref}

\pagestyle{fancy}
\fancyhf{}
\rhead{MAT 357}
\lhead{Assignment 2(done with Payam Fakooirziba)}
\rfoot{Page \thepage}

\setlength\parindent{24pt}
\renewcommand\qedsymbol{$\blacksquare$}

\DeclareMathOperator{\U}{\mathcal{U}}
\DeclareMathOperator{\Prt}{\mathbb{P}}
\DeclareMathOperator{\R}{\mathbb{R}}
\DeclareMathOperator{\N}{\mathbb{N}}
\DeclareMathOperator{\Z}{\mathbb{Z}}
\DeclareMathOperator{\Q}{\mathbb{Q}}
\DeclareMathOperator{\C}{\mathbb{C}}
\DeclareMathOperator{\ep}{\varepsilon}
\DeclareMathOperator{\identity}{\mathbf{0}}
\DeclareMathOperator{\card}{card}
\newcommand{\suchthat}{;\ifnum\currentgrouptype=16 \middle\fi|;}

\newtheorem{lemma}{Lemma}

\newcommand{\tr}{\mathrm{tr}}
\newcommand{\ra}{\rightarrow}
\newcommand{\lan}{\langle}
\newcommand{\ran}{\rangle}
\newcommand{\norm}[1]{\left\lVert#1\right\rVert}
\newcommand{\inn}[1]{\lan#1\ran}
\newcommand{\ol}{\overline}
\begin{document}
\noindent Q4: We will show that the toplogists sin circle is path connected, yet not locally path connected. First suppose that we have two points $x,y$ belonging to the circular subset of the toplogists sine circle. 
We simply take a path along the subset of the circle between these two points. If the points $x$ belongs to the circular portion of our set, and $y$ belongs to the sine curve, we choose a path along $y\mapsto \sin(\frac{1}{y}) $ which is contiuous on $(0,1)$. Since the sine curve has a common point with the circle segment, namely $(0,0)$, choose a path from $x$ to $0,0$ along the circle. 
This path makes sense, since $(0,1)$ is path connected, and $\sin(\frac{1}{x})$ is continuous on this set, hence the image of it is path connected as well, since we can compose and contious path with $\sin(\frac{1}{x})$ and obtain a new path
Next case is when our points $x,y$ belong to the sine curve. We simply travel along the continuous curve from $x$ until we reach $(0,0)$, and travel from $y$ to $0,0$ in the same fashion. We now claim that in some neighbourhood, the topoligists sin curve is not path connected. 
Suppose that it is.Let $B=M_{\frac{1}{2}}(0,1)$ Then there exists a continuous path from $(\frac{2}{5\pi},1)$ to $(\frac{2}{9\pi},1)$ along the $\sin$ curve, which is contained in $B$ intersected with the sin curve. Take the path $\gamma:[\frac{2}{5\pi},\frac{2}{9\pi}] \rightarrow \R^2$ with $\gamma: x\mapsto (x,\sin(\frac{1}{x}))$. This is a path between the points, but it escapes the set $B$ at the point $\frac{2}{7\pi}$, $\gamma(\frac{2}{7\pi}) = (\frac{2}{7\pi},-1)$. Hence there is some neighbourhood in which the topologists sine circle is not path connected. So we conclude it is not locally path connected. 
\end{document}