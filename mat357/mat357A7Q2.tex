\documentclass[letterpaper]{article}
\usepackage[letterpaper,margin=1in,footskip=0.25in]{geometry}
\usepackage[utf8]{inputenc}
\usepackage{amsmath}
\usepackage{amsthm}
\usepackage{amssymb, pifont}
\usepackage{mathrsfs}
\usepackage{enumitem}
\usepackage{fancyhdr}
\usepackage{hyperref}

\pagestyle{fancy}
\fancyhf{}
\rhead{MAT 357}
\lhead{Assignment 7}
\rfoot{Page \thepage}

\setlength\parindent{24pt}
\renewcommand\qedsymbol{$\blacksquare$}

\DeclareMathOperator{\F}{\mathbb{F}}
\DeclareMathOperator{\T}{\mathcal{T}}
\DeclareMathOperator{\V}{\mathcal{V}}
\DeclareMathOperator{\U}{\mathcal{U}}
\DeclareMathOperator{\Prt}{\mathbb{P}}
\DeclareMathOperator{\R}{\mathbb{R}}
\DeclareMathOperator{\N}{\mathbb{N}}
\DeclareMathOperator{\Z}{\mathbb{Z}}
\DeclareMathOperator{\Q}{\mathbb{Q}}
\DeclareMathOperator{\C}{\mathbb{C}}
\DeclareMathOperator{\ep}{\varepsilon}
\DeclareMathOperator{\identity}{\mathbf{0}}
\DeclareMathOperator{\card}{card}
\newcommand{\suchthat}{;\ifnum\currentgrouptype=16 \middle\fi|;}

\newtheorem{lemma}{Lemma}

\newcommand{\tr}{\mathrm{tr}}
\newcommand{\ra}{\rightarrow}
\newcommand{\lan}{\langle}
\newcommand{\ran}{\rangle}
\newcommand{\norm}[1]{\left\lVert#1\right\rVert}
\newcommand{\inn}[1]{\lan#1\ran}
\newcommand{\ol}{\overline}
\begin{document}
\noindent Q2: We will verify that $\omega$ satisfies the 3 conditions of being a measure. First note that the cardinality of $\emptyset$ is 0 by definition, so $\omega(\emptyset)=0$. Next, let $A\subset B$. There exists an injection $\iota:A\to B$ via inclusion, which by definition means that $\omega(A)\leq \omega(B)$. Finally let $\{E_i\}$ be a collection of measurable sets. We have that $$\omega(\bigcup_{k\geq 1} E_k) = \# \bigcup_{k\geq 1} E_k \leq \sum_{k\geq 1} \# E_k = \sum_{k\geq 1} \omega(E_k)$$
Where the inequality follows from cardinality of sets being at most the sum of the cardinalities, since the sets may not necessaily be disjoint. We now show that any $E\subset M$ is measurable. We see that for any $E\subset M$, and an arbitrary test set $X$,
$$\omega(X) = \#X = \#X \cap(E\cup E^c) = \# X\cap E + \# X\cap E^c = \omega(X\cap E) + \omega(X\cap E^c)$$
Where the third equality follows from the fact that $E$ is disjoint with $E^c$.  
\end{document}