\documentclass[letterpaper]{article}
\usepackage[letterpaper,margin=1in,footskip=0.25in]{geometry}
\usepackage[utf8]{inputenc}
\usepackage{amsmath}
\usepackage{amsthm}
\usepackage{amssymb, pifont}
\usepackage{mathrsfs}
\usepackage{enumitem}
\usepackage{fancyhdr}
\usepackage{hyperref}

\pagestyle{fancy}
\fancyhf{}
\rhead{MAT 357}
\lhead{Assignment 1(done with Payam Fakoorziba)}
\rfoot{Page \thepage}

\setlength\parindent{24pt}
\renewcommand\qedsymbol{$\blacksquare$}

\DeclareMathOperator{\U}{\mathcal{U}}
\DeclareMathOperator{\Prt}{\mathbb{P}}
\DeclareMathOperator{\R}{\mathbb{R}}
\DeclareMathOperator{\N}{\mathbb{N}}
\DeclareMathOperator{\Z}{\mathbb{Z}}
\DeclareMathOperator{\Q}{\mathbb{Q}}
\DeclareMathOperator{\C}{\mathbb{C}}
\DeclareMathOperator{\ep}{\varepsilon}
\DeclareMathOperator{\identity}{\mathbf{0}}
\DeclareMathOperator{\card}{card}
\newcommand{\suchthat}{;\ifnum\currentgrouptype=16 \middle\fi|;}

\newtheorem{lemma}{Lemma}

\newcommand{\tr}{\mathrm{tr}}
\newcommand{\ra}{\rightarrow}
\newcommand{\lan}{\langle}
\newcommand{\ran}{\rangle}
\newcommand{\norm}[1]{\left\lVert#1\right\rVert}
\newcommand{\inn}[1]{\lan#1\ran}
\newcommand{\ol}{\overline}
\begin{document}
\noindent Q4a:
If $f$ is open, then it is not necessarily continuous. Consider the following function, $\sigma(x):\R \rightarrow \{-1,0,1\}$ which maps $x>0$ to 1, $x<0$ to -1, and 0 to 0, with the codomain equipped with the discrete topology. 
This is an open map, since the image of any open set will be either -1,0,1 or some union of them. However, $f^{pre}(\{ 0 \}) = 0$ which is not open in $\R$. 
\newline \\ 4b
We claim any homeomorphism, $f: M\rightarrow N$, is an open map. Let U be some open set in $M$. Then, ${f^{-1}}^{pre}(U)$ is open. Since $f$ is a bijection, ${f^{-1}}^{pre}(U)= f(U)$. Thus $f$ is an open map. 
\newline \\ 4c: 
We claim an open, continuous bijection is a homeomorphism. Let U be an open set in N. Then $f^{pre}(U)$ is open, and by bijecivity, $f(f^{pre}(U))=U$, and so $f^{pre}(U)=f^{-1}(U)$ is open. Thus $f^-1$ is a homeomorphism
\newline \\ 4d: Consider the function $f(x)= x^3-4x$. This is clearly surjective and continuous. Consider the open set $U=(0,1.5)$. We evaluate $f(U)= [-3.079,0)$, which is not an open set in $\R$
\newline \\ 4e: We claim that and $f:\R \rightarrow \R$, which is a continuous, surjective and open, is a homeomorphism. It suffices to show injectivity, then we can apply the result from c and conclude it is a homeomorphism. Suppose that $f$ is not injective. Then for some $x,y\in \R$, $x<y$, we have $f(x)=f(y)$.
By continuity, $f$ attains a maximum and minimum on the set $[x,y]$. Let $a$ correspond to the point at which $f$ attains a minimum, $b$ be the point where $f$ attains its maximum. First consider the case where $a,b\in \{x,y\}$. This would imply that $f$ is constant on $[x,y]$, violating openness, since the image of any open set contained in $[x,y]$ will be a point, which is closed. 
Suppose that $a\in \{x,y\}$, and $b\in (x,y)$. Then the image of $(x,y)$ under $f$ should be open, but since it attains its minimum on the interiour, $f((x,y)) = [f(a),f(b))$. Similarly, if $b\in \{x,y\}$ and $a\in(x,y)$ then $f((x,y)) = (f(a),f(b)]$. Either case contradicts openness. Now consider the case $a,b\in(x,y)$. We see that $f((x,y))=[f(a),f(b)]$. We contradict openness again. 
Hence $f$ must be injective, and we conclude it is a homeomorphism.
\newline \\ 4f: Consider the map $f:S^1\rightarrow S^1$ which maps $z\mapsto z^2$ in the complex plane. This is a continuous, open surjection, which doubles the argument of any $x\in S^1$. We notice that it is not injective since $f((1,0)) = f((-1,0))=0$ 

\end{document}