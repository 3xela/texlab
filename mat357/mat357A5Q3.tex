\documentclass[letterpaper]{article}
\usepackage[letterpaper,margin=1in,footskip=0.25in]{geometry}
\usepackage[utf8]{inputenc}
\usepackage{amsmath}
\usepackage{amsthm}
\usepackage{amssymb, pifont}
\usepackage{mathrsfs}
\usepackage{enumitem}
\usepackage{fancyhdr}
\usepackage{hyperref}

\pagestyle{fancy}
\fancyhf{}
\rhead{MAT 357}
\lhead{Assignment 5 (done with Walter Merjo/Payam Fakoorziba)}
\rfoot{Page \thepage}

\setlength\parindent{24pt}
\renewcommand\qedsymbol{$\blacksquare$}

\DeclareMathOperator{\F}{\mathbb{F}}
\DeclareMathOperator{\T}{\mathcal{T}}
\DeclareMathOperator{\V}{\mathcal{V}}
\DeclareMathOperator{\U}{\mathcal{U}}
\DeclareMathOperator{\Prt}{\mathbb{P}}
\DeclareMathOperator{\R}{\mathbb{R}}
\DeclareMathOperator{\N}{\mathbb{N}}
\DeclareMathOperator{\Z}{\mathbb{Z}}
\DeclareMathOperator{\Q}{\mathbb{Q}}
\DeclareMathOperator{\C}{\mathbb{C}}
\DeclareMathOperator{\ep}{\varepsilon}
\DeclareMathOperator{\identity}{\mathbf{0}}
\DeclareMathOperator{\card}{card}
\newcommand{\suchthat}{;\ifnum\currentgrouptype=16 \middle\fi|;}

\newtheorem{lemma}{Lemma}

\newcommand{\tr}{\mathrm{tr}}
\newcommand{\ra}{\rightarrow}
\newcommand{\lan}{\langle}
\newcommand{\ran}{\rangle}
\newcommand{\norm}[1]{\left\lVert#1\right\rVert}
\newcommand{\inn}[1]{\lan#1\ran}
\newcommand{\ol}{\overline}
\begin{document}
\noindent Q3a: Consider the function $f(x) = \frac{1}{2}x^2$ on $(-1,1)$. We claim that this is a weak contraction, yet not a contraction. Since $x,y\in (-1,1)$, note that for distinct $x,y$
$$\frac{1}{2}|(x+y)|<1 \iff \frac{1}{2}|x+y|\cdot|x-y|<|x-y| \iff \frac{1}{2}|x^2-y^2| < |x-y|$$
Therefore $d(fx,fy)<d(x,y)$. We claim that this function is not a contraction. Let $0<k<1$. Choosing $x,y$ such that $\frac{1}{2}|x+y|>k$ we have that $$\frac{1}{2}|x+y|> k \iff \frac{1}{2}|x-y|\cdot|x+y|>k|x-y|$$
And so $d(fx,fy)>k\cdot d(x,y)$
\newline \\ Q3b: Use the same $f(x)$ as defined in $3a$, except we define it on $[-1,1]$. Using the exact same proof as in $3a$, we have that this function is a weak contraction yet not a contraction. 
\newline \\ Q3c: Define $g: M\to M$ by $g(x) = d(fx,x)$. This is a continuous map and so $g(M)=[a,b]$, for some $0<a<b$. We claim that $a=0$. Suppose not, that it assume that $g$ attains its minimum at some $y\in M$, and $g(y)>0$. Consider however $g(fy)$. We have that $g(fy) = d(f^2y,fy)<d(fy,y)=g(y)$. This is a contradiction and hence $g$ attains a minimum of $0$ at some point $x_0$. Therefore, $0=x_0=d(fx_0,x_0)$ and so $fx_0=x_0$. We now claim uniqueness. Suppose that $x,y$ are two distinct fixed points. We therefore have that $d(x,y)=d(fx,fy)<d(x,y)$. This is a contradiction. Hence the fixed point of $f$ is unique. 
\end{document}
