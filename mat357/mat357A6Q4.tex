\documentclass[letterpaper]{article}
\usepackage[letterpaper,margin=1in,footskip=0.25in]{geometry}
\usepackage[utf8]{inputenc}
\usepackage{amsmath}
\usepackage{amsthm}
\usepackage{amssymb, pifont}
\usepackage{mathrsfs}
\usepackage{enumitem}
\usepackage{fancyhdr}
\usepackage{hyperref}

\pagestyle{fancy}
\fancyhf{}
\rhead{MAT 357}
\lhead{Assignment 6}
\rfoot{Page \thepage}

\setlength\parindent{24pt}
\renewcommand\qedsymbol{$\blacksquare$}

\DeclareMathOperator{\F}{\mathbb{F}}
\DeclareMathOperator{\T}{\mathcal{T}}
\DeclareMathOperator{\V}{\mathcal{V}}
\DeclareMathOperator{\U}{\mathcal{U}}
\DeclareMathOperator{\Prt}{\mathbb{P}}
\DeclareMathOperator{\R}{\mathbb{R}}
\DeclareMathOperator{\N}{\mathbb{N}}
\DeclareMathOperator{\Z}{\mathbb{Z}}
\DeclareMathOperator{\Q}{\mathbb{Q}}
\DeclareMathOperator{\C}{\mathbb{C}}
\DeclareMathOperator{\ep}{\varepsilon}
\DeclareMathOperator{\identity}{\mathbf{0}}
\DeclareMathOperator{\card}{card}
\newcommand{\suchthat}{;\ifnum\currentgrouptype=16 \middle\fi|;}

\newtheorem{lemma}{Lemma}

\newcommand{\tr}{\mathrm{tr}}
\newcommand{\ra}{\rightarrow}
\newcommand{\lan}{\langle}
\newcommand{\ran}{\rangle}
\newcommand{\norm}[1]{\left\lVert#1\right\rVert}
\newcommand{\inn}[1]{\lan#1\ran}
\newcommand{\ol}{\overline}
\begin{document}
\noindent Q4ai: Note that the interval $(-\ep,\ep)$ covers the empty set, and by the epsilon principle it has a Jordan content of 0. 
\newline ii: This is true since every open cover of $B$ will also be an open cover of $A$. Hence the Jordan content of $B$ will be at least the Jordan content of $A$
\newline iii: We see given $\ep >0$ we can cover each $A_n$ with a covering $\{ I_{k,n}: k\in \N\}$, we have that $$\sum_k |I_{k,n}| \leq J^{*}A_n + \frac{\ep}{2^n}$$. We have that $\{I_{k,n}: k,n\in \N\}$ is a covering of $A$, and $$\sum_{n=1}^N \sum_{k=1} |I_{k,n}| \leq \sum_{n=1}^N J^{*}A_n + \frac{\varepsilon}{2^n} < \sum_{n=1}^N m^{*}A + \varepsilon$$
Therefore the infimum of total lengths of finite coverings of $A$ is less than sum of lengths of coverings for each $A_n$
\newline \\ Q4b: Consider the set $A = (0,1)\cap \Q$. We can write $A$ as countable union of singletons $q\in A$. We have that $J^{*}A_i=0$, since singletons can be covered with an arbitrarily small interval, yet $J^{*}A=1$. 
\newline \\ Q4c: It is clear that $m^{*}A\leq J^{*}A$ since every finite cover of a set is also a countable cover, hence we are intaking the infimum over a larger set, so it can be less. If $A$ is compact, then we have that every countable cover will have a finite subcover, so the infimum of the volumes of intervals which cover the set $A$ will be equal. The converse is not true however, since if we take $A=(0,1)$ we have that $J^{*}A = m^{*}A$ yet $A$ is not compact. 
\newline \\ Q4d: It is sufficient to show that the Jordan measure of an open and closed interval are equal. Note that for $\ep>0$, $[a,b]\subset (a-\ep,b+\ep)$ and $(a,b)\subset (a-\ep,b+\ep)$ and hence they have the same measure. Therefore it is equivalent to cover a set with either open or closed intervals. 
\end{document}