\documentclass[letterpaper]{article}
\usepackage[letterpaper,margin=1in,footskip=0.25in]{geometry}
\usepackage[utf8]{inputenc}
\usepackage{amsmath}
\usepackage{amsthm}
\usepackage{amssymb, pifont}
\usepackage{mathrsfs}
\usepackage{enumitem}
\usepackage{fancyhdr}
\usepackage{hyperref}

\pagestyle{fancy}
\fancyhf{}
\rhead{MAT 357}
\lhead{Assignment 5(done with Payam Fakoorziba)}
\rfoot{Page \thepage}

\setlength\parindent{24pt}
\renewcommand\qedsymbol{$\blacksquare$}

\DeclareMathOperator{\F}{\mathbb{F}}
\DeclareMathOperator{\T}{\mathcal{T}}
\DeclareMathOperator{\V}{\mathcal{V}}
\DeclareMathOperator{\U}{\mathcal{U}}
\DeclareMathOperator{\Prt}{\mathbb{P}}
\DeclareMathOperator{\R}{\mathbb{R}}
\DeclareMathOperator{\N}{\mathbb{N}}
\DeclareMathOperator{\Z}{\mathbb{Z}}
\DeclareMathOperator{\Q}{\mathbb{Q}}
\DeclareMathOperator{\C}{\mathbb{C}}
\DeclareMathOperator{\ep}{\varepsilon}
\DeclareMathOperator{\identity}{\mathbf{0}}
\DeclareMathOperator{\card}{card}
\newcommand{\suchthat}{;\ifnum\currentgrouptype=16 \middle\fi|;}

\newtheorem{lemma}{Lemma}

\newcommand{\tr}{\mathrm{tr}}
\newcommand{\ra}{\rightarrow}
\newcommand{\lan}{\langle}
\newcommand{\ran}{\rangle}
\newcommand{\norm}[1]{\left\lVert#1\right\rVert}
\newcommand{\inn}[1]{\lan#1\ran}
\newcommand{\ol}{\overline}
\begin{document}
\noindent Q1a: We first claim that for an open covering of $[a,b]$ by open balls, each ball will intersect at least one other. Suppose not, that is assume that $(x,y)$ belongs to the open cover of $[a,b]$, and is disjoint from every other open cover. Then the point $y$ belongs to some open set $B$ belonging to the cover. By openness of  $B$, there is some open ball $U_y$ contained in $B$ and containing $y$. This ball must intersect $(x,y)$ which contradicts our assumption. We now prove the main result. Let $\varepsilon $ be given. By equicontinuity of $f_n$, for some $\delta>0$ we have an open covering of $[a,b]$ of the form $\mathcal{O} = \{B_{\frac{\delta}{2}}(x) =(x-\frac{\delta}{2},x+\frac{\delta}{2}): x\in [a,b]\}$. By compactness of $[a,b]$, there exists finitely many, and strictly increasing $x_1, \dots , x_k$ where $B_{\frac{\delta}{2}}(x_1), \dots , B_{\frac{\delta}{2}}(x_k)$ cover $[a,b]$. So $p$ must belong to some $B_{\frac{\delta}{2}}(x_\alpha)$. By equicontinuinty of $f_n$, and by the boundedness at $f_n(p)$, we have that $f_n(x)$ will be bounded by $M+\varepsilon$ on $B_{\frac{\delta}{2}}(x_\alpha)$. Now take $x_0,x_1\in B_{\frac{\delta}{2}}(x_\alpha)$ with $x_0\in B_{\frac{\delta}{2}}(x_{\alpha-1})$ and $x_1\in B_{\frac{\delta}{2}}(x_{\alpha+1})$. Repeating the same process as above we have that $|f_n(x)| \leq M+2\varepsilon$ on $B_{\frac{\delta}{2}}(x_{\alpha-1})\cup B_{\frac{\delta}{2}}(x_\alpha)\cup B_{\frac{\delta}{2}}(x_{\alpha+1})$. We repeat this process up to $k$ times we have that $|f_n(x)|\leq M+k\varepsilon$. Therefore, $f_n$ is uniformly bounded. 
\newline \\ Q1b: Since being bounded at a point propagates to boundedness on entire domain, we can reformulate the Arzela-Ascoli in the following way. Let $(f_n)\subset C^o([a,b],\R)$ be an equicontinuous sequence of function. Suppose that for some $p\in [a,b]$, $|f_n(p)|<M$ for some $M$, then $(f_n)$ admits a uniformly convergent subsequence. 
\newline \\ Q1c: This is true for $(a,b)$. Since we can extend any continuous function on $[a,b]$ by $\tilde{f}(x) = f(x)$ for $x\in [a,b]$, and $\tilde{f}(a) = \lim_{x\rightarrow a_{-}}f(x)$ and $\tilde{f}(b) = \lim_{x\to b_{+}} f(x)$. The fact from 1a is true for $\tilde{f}$ so it will also be true for $f$. The result from $1a$ does not hold on either $\Q,\R,\N$ since they are not compact, or can not be extended to a compact set. Consider the family $(fn)$ where $f_n(x) = x$. We have that on $\Q,\N,\R$ that $f_n(1)$ is bounded for all $n$, namely by $1$ and $(f_n)$ is equicontinuous.  However, these functions are unbounded on $\Q,\N,\R$. 
\end{document}