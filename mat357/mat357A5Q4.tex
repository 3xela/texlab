\documentclass[letterpaper]{article}
\usepackage[letterpaper,margin=1in,footskip=0.25in]{geometry}
\usepackage[utf8]{inputenc}
\usepackage{amsmath}
\usepackage{amsthm}
\usepackage{amssymb, pifont}
\usepackage{mathrsfs}
\usepackage{enumitem}
\usepackage{fancyhdr}
\usepackage{hyperref}

\pagestyle{fancy}
\fancyhf{}
\rhead{MAT 357}
\lhead{Assignment 5(done with payam fakoorziba)}
\rfoot{Page \thepage}

\setlength\parindent{24pt}
\renewcommand\qedsymbol{$\blacksquare$}

\DeclareMathOperator{\F}{\mathbb{F}}
\DeclareMathOperator{\T}{\mathcal{T}}
\DeclareMathOperator{\V}{\mathcal{V}}
\DeclareMathOperator{\U}{\mathcal{U}}
\DeclareMathOperator{\Prt}{\mathbb{P}}
\DeclareMathOperator{\R}{\mathbb{R}}
\DeclareMathOperator{\N}{\mathbb{N}}
\DeclareMathOperator{\Z}{\mathbb{Z}}
\DeclareMathOperator{\Q}{\mathbb{Q}}
\DeclareMathOperator{\C}{\mathbb{C}}
\DeclareMathOperator{\ep}{\varepsilon}
\DeclareMathOperator{\identity}{\mathbf{0}}
\DeclareMathOperator{\card}{card}
\newcommand{\suchthat}{;\ifnum\currentgrouptype=16 \middle\fi|;}

\newtheorem{lemma}{Lemma}

\newcommand{\tr}{\mathrm{tr}}
\newcommand{\ra}{\rightarrow}
\newcommand{\lan}{\langle}
\newcommand{\ran}{\rangle}
\newcommand{\norm}[1]{\left\lVert#1\right\rVert}
\newcommand{\inn}[1]{\lan#1\ran}
\newcommand{\ol}{\overline}
\begin{document}
\noindent Q4a: By the Weierstrass Approximation Theorem, for each $\varepsilon>0$, there exists some polynomial $p(x)$ where $|p(x)-f(x)|<\varepsilon$. We first claim that for such an $\varepsilon$ and $p(x)$, $$\lim_{n\to \infty} \int_{0}^1 x^n p(x) dx = \lim_{n \to \infty}\int_0^1 x^n f(x) dx$$
Let $\varepsilon>0$ be given, let $p(x)$ be such that $|p(x)-f(x)|<\varepsilon$, then we evaluate that
$$\lim_{x \to \infty}|\int_{0}^1 x^nf(x) -x^nf(x) dx| \leq \lim_{n\to \infty} \int_0^1 x^n|p(x)-f(x)|dx < \lim_{n\to \infty} \int_0^1 x^n \varepsilon dx = \varepsilon \lim_{n\to \infty}\Big[\frac{x^{n+1}}{n+1}\Big] \Big|_{x=0}^{x=1} =0$$
Thus these limits are equal. To find $\lim_{x \to \infty}\int_{0}^1 x^nf(x)$ it suffices to compute $\lim_{x \to \infty}\int_{0}^1 x^np(x)$. Let $p(x) = \sum_{k=0}^m a_k x^k$
We compute:
\begin{align*}
    \lim_{x \to \infty}\int_{0}^1 x^n p(x) dx & = \lim_{x \to \infty}\int_{0}^1 x^n \sum_{k=0}^m a_k x^k dx
    \\ & = \lim_{x \to \infty}\sum_{k=0}^m a_k \int_{0}^1 x^{n+k} dx
    \\ & = \lim_{x \to \infty} \sum_{k=0}^n a_k \Big[\frac{x^{n+k+1}}{n+k+1} \Big] \Big|_{x=0}^{x=1}
    \\ & = \lim_{x \to \infty} \sum_{k=0}^m a_k \frac{1}{n+k+1}
    \\ & = 0
\end{align*} Therefore $\lim_{n\to \infty } \int_{0}^1 x^n f(x)dx = 0$
\newline \\ Q4b: By Q5, for each $\varepsilon$ there exists a polynomial $p(x)$ such that $|p(x)-f(x)|< \varepsilon$ and $p(1)=f(1)$. We claim that $$\lim_{n \to \infty} n \int_{0}^1 x^n p(x) dx = \lim_{n \to \infty} n \int_{0}^1 x^n f(x) dx $$
By a similar computation to 4a, we see that 
$$\lim_{n \to \infty } n \int_{0}^1 x^n|p(x)-f(x)| dx < \lim_{n \to \infty } n \int_{0}^1 x^n \varepsilon  dx = \varepsilon \Big[\frac{x^{n+1}}{n+1} \Big] \Big|_{x=0}^{x=1} = \varepsilon$$
The limits can be made within epsilon of eachother, hence they are equal. Let $p(x) = \sum_{k=0}^m a_k x^m$ .We now will evaluate $\lim_{n \to \infty } n \int_{0}^1 x^n p(x) dx$ as : 
\begin{align*}
    \lim_{n \to \infty } n \int_{0}^1x^n p(x) dx  & = \lim_{n \to \infty } n \int_{0}^1 x^n \sum_{k=0}^m a_k x^k dx 
    \\ & = \lim_{n \to \infty } n \sum_{k=0}^m a_k \int_{0}^1 x^{n+k}
    \\ & = \lim_{n \to \infty } n \sum_{k=0}^m a_k \Big[\frac{x^{n+k+1}}{n+k+1} \Big] \Big|_{x=0}^{x=1}
    \\ & = \lim_{n \to \infty } \sum_{k=0}^m a_k \frac{n}{n+k+1}
    \\ & = \sum_{k=0}^m a_k
    \\ & = p(1)
\end{align*}
Therefore this limit converges to $p(1)=f(1)$.
\end{document}