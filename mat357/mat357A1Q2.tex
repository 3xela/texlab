\documentclass[letterpaper]{article}
\usepackage[letterpaper,margin=1in,footskip=0.25in]{geometry}
\usepackage[utf8]{inputenc}
\usepackage{amsmath}
\usepackage{amsthm}
\usepackage{amssymb, pifont}
\usepackage{mathrsfs}
\usepackage{enumitem}
\usepackage{fancyhdr}
\usepackage{hyperref}

\pagestyle{fancy}
\fancyhf{}
\rhead{MAT 357}
\lhead{Assignment 1(done with Payam Fakoorziba)}
\rfoot{Page \thepage}

\setlength\parindent{24pt}
\renewcommand\qedsymbol{$\blacksquare$}

\DeclareMathOperator{\U}{\mathcal{U}}
\DeclareMathOperator{\Prt}{\mathbb{P}}
\DeclareMathOperator{\R}{\mathbb{R}}
\DeclareMathOperator{\N}{\mathbb{N}}
\DeclareMathOperator{\Z}{\mathbb{Z}}
\DeclareMathOperator{\Q}{\mathbb{Q}}
\DeclareMathOperator{\C}{\mathbb{C}}
\DeclareMathOperator{\ep}{\varepsilon}
\DeclareMathOperator{\identity}{\mathbf{0}}
\DeclareMathOperator{\card}{card}
\newcommand{\suchthat}{;\ifnum\currentgrouptype=16 \middle\fi|;}

\newtheorem{lemma}{Lemma}

\newcommand{\tr}{\mathrm{tr}}
\newcommand{\ra}{\rightarrow}
\newcommand{\lan}{\langle}
\newcommand{\ran}{\rangle}
\newcommand{\norm}[1]{\left\lVert#1\right\rVert}
\newcommand{\inn}[1]{\lan#1\ran}
\newcommand{\ol}{\overline}
\begin{document}
\noindent Q2a: Suppose that $f$ is uniformly continuous. Then we have that for each $\varepsilon>0$ there exists some $\delta>0$ such that $|x-t|<\delta \implies |f(x)-f(t)|< \varepsilon$. 
Now if we fix $x$ in the domain of $f$, we have continuity of $f$ at $x$. This is true for any $x$ in the domain of $f$ so $f$ is continuous.
We now claim that $f(x): (0,1) \rightarrow \R$ defined by $x\mapsto \sin(\frac{1}{x})$ is continuous yet not uniformly continuous. It is easy to see that it is continuous, as it is the composition of two continuous maps. 
Choose $\varepsilon=1$. Then for every $\delta>0$, we can find $x,t\in (0,\delta)$ where $f(x)=1$ and $f(t)=-1$ in the following way. Choose x so that $\frac{1}{x}> \frac{1}{\delta}$, and $x$ is of the form $\frac{1}{x}=\frac{\pi}{2} + 2k\pi$ for sufficiently large $k$. 
Similarly, choose $\frac{1}{t} = \frac{\pi}{2} + (2k+1)\pi$ for sufficiently large $k$. We have that $|f(x)-f(t)|=2>\varepsilon$
\newline  \\ Q2b: We claim $f(x)=2x$ is uniformly continuous on $\R$. Let $\varepsilon>0$ be given. Choose $\delta = \frac{\varepsilon}{2}$. 
For any $x,y\in \R$ we compute that $$|x-y|< \frac{\varepsilon}{2}\implies |2x-2y|< \varepsilon \implies |f(x)-f(y)|< \varepsilon$$
Hence $f$ is uniformly continuous. 
\newline \\ Q2c: We claim $f(x)=x^2$ is not uniformly continuous on $\R$. Choosing $\varepsilon = 1$, and choose $x=y+ \frac{\delta}{2}$. We have $|x-y|=|\frac{\delta}{2}|<\delta$. We see that for sufficiently large $y$, $|f(x)-f(y)| = |\frac{\delta^2}{4}+ \delta y|>1$. Hence $f$ will not be uniformly continuous. 
\end{document}