\documentclass[letterpaper]{article}
\usepackage[letterpaper,margin=1in,footskip=0.25in]{geometry}
\usepackage[utf8]{inputenc}
\usepackage{amsmath}
\usepackage{amsthm}
\usepackage{amssymb, pifont}
\usepackage{mathrsfs}
\usepackage{enumitem}
\usepackage{fancyhdr}
\usepackage{hyperref}

\pagestyle{fancy}
\fancyhf{}
\rhead{MAT 357}
\lhead{Assignment 1(done with Payam Fakoorziba)}
\rfoot{Page \thepage}

\setlength\parindent{24pt}
\renewcommand\qedsymbol{$\blacksquare$}

\DeclareMathOperator{\U}{\mathcal{U}}
\DeclareMathOperator{\Prt}{\mathbb{P}}
\DeclareMathOperator{\R}{\mathbb{R}}
\DeclareMathOperator{\N}{\mathbb{N}}
\DeclareMathOperator{\Z}{\mathbb{Z}}
\DeclareMathOperator{\Q}{\mathbb{Q}}
\DeclareMathOperator{\C}{\mathbb{C}}
\DeclareMathOperator{\ep}{\varepsilon}
\DeclareMathOperator{\identity}{\mathbf{0}}
\DeclareMathOperator{\card}{card}
\newcommand{\suchthat}{;\ifnum\currentgrouptype=16 \middle\fi|;}

\newtheorem{lemma}{Lemma}

\newcommand{\tr}{\mathrm{tr}}
\newcommand{\ra}{\rightarrow}
\newcommand{\lan}{\langle}
\newcommand{\ran}{\rangle}
\newcommand{\norm}[1]{\left\lVert#1\right\rVert}
\newcommand{\inn}[1]{\lan#1\ran}
\newcommand{\ol}{\overline}
\begin{document}
\noindent Q1a:
Assume $f:A\rightarrow A$ has some fixed point $b$. Then we have that $(b,f(b)) = (b,b)$ will be in the graph of $f$. By definition, this will also belong to the diagonal. 
Conversely, assume that the graph of $f$ intersects the diagonal of $A \times A$. Then for some point $b\in A$ we have that $(b,f(b))= (b,b)$. 
\newline \\    Q1b:
We assume away the case in which $f(0)=0$ and $f(1)=1$. Define $g(x):= f(x)-x$, this will be continuous. We have that by our assumption, $g(0)>0$ and $g(1)<1$. Thus by the intermediate value theorem there is some $x_0$ where $g(x_0)=0$ or equivalently, $f(x_0)=x_0$. 
\newline \\  Q1c:  No, consider the function $f(x)=x^2$. If we solve for $x^2=x$ we see that $x=1$ and $x=0$ are both fixed points, yet they are not elements of the domain $(0,1)$.
\newline \\ Q1d: No, consider the function $\chi_{\Q}$ on $(0,1)$, defined to be $1$ on every rational, and 0 else. This function will fix no points on $(0,1)$. 
\end{document}