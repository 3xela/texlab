\documentclass[12pt, a4paper]{article}
\usepackage[lmargin =0.5 in, 
rmargin=0.5in, 
tmargin=1in,
bmargin=0.5in]{geometry}
\geometry{letterpaper}
\usepackage{tikz-cd}
\usepackage{amsmath}
\usepackage{amssymb}
\usepackage{blindtext}
\usepackage{titlesec}
\usepackage{enumitem}
\usepackage{fancyhdr}
\usepackage{amsthm}
\usepackage{graphicx}
\usepackage{cool}
\usepackage{thmtools}
\usepackage{hyperref}
\graphicspath{ }					%path to an image

%-------- sexy font ------------%
%\usepackage{libertine}
%\usepackage{libertinust1math}

%\usepackage{mlmodern}				% very nice and classic
%\usepackage[utopia]{mathdesign}
%\usepackage[T1]{fontenc}


\usepackage{mlmodern}
\usepackage{eulervm}
%\usepackage{tgtermes} 				%times new roman
%-------- sexy font ------------%


% Problem Styles
%====================================================================%


\newtheorem{problem}{Problem}


\theoremstyle{definition}
\newtheorem{thm}{Theorem}
\newtheorem{lemma}{Lemma}
\newtheorem{prop}{Proposition}
\newtheorem{cor}{Corollary}
\newtheorem{fact}{Fact}
\newtheorem{defn}{Definition}
\newtheorem{example}{Example}
\newtheorem{question}{Question}

\newtheorem{manualprobleminner}{Problem}

\newenvironment{manualproblem}[1]{%
	\renewcommand\themanualprobleminner{#1}%
	\manualprobleminner
}{\endmanualprobleminner}

\newcommand{\penum}{ \begin{enumerate}[label=\bf(\alph*), leftmargin=0pt]}
	\newcommand{\epenum}{ \end{enumerate} }

% Math fonts shortcuts
%====================================================================%

\newcommand{\ring}{\mathcal{R}}
\newcommand{\N}{\mathbb{N}}                           % Natural numbers
\newcommand{\Z}{\mathbb{Z}}                           % Integers
\newcommand{\R}{\mathbb{R}}                           % Real numbers
\newcommand{\C}{\mathbb{C}}                           % Complex numbers
\newcommand{\F}{\mathbb{F}}                           % Arbitrary field
\newcommand{\Q}{\mathbb{Q}}                           % Arbitrary field
\newcommand{\PP}{\mathcal{P}}                         % Partition
\newcommand{\M}{\mathcal{M}}                         % Mathcal M
\newcommand{\eL}{\mathcal{L}}                         % Mathcal L
\newcommand{\T}{\mathbb{T}}                         % Mathcal T
\newcommand{\U}{\mathcal{U}}                         % Mathcal U\\
\newcommand{\V}{\mathcal{V}}                         % Mathcal V

% symbol shortcuts
%====================================================================%

\newcommand{\bd}{\partial}
\newcommand{\grad}{\nabla}
\newcommand{\lam}{\lambda}
\newcommand{\imp}{\implies}
\newcommand{\all}{\forall}
\newcommand{\exs}{\exists}
\newcommand{\delt}{\delta}
\newcommand{\ep}{\varepsilon}
\newcommand{\ra}{\rightarrow}
\newcommand{\vph}{\varphi}

\newcommand{\ol}{\overline}
\newcommand{\f}{\frac}
\newcommand{\lf}{\lfrac}
\newcommand{\df}{\dfrac}

% bracketting shortcuts
%====================================================================%
\newcommand{\abs}[1]{\left| #1 \right|}
\newcommand{\babs}[1]{\Big|#1\Big|}
\newcommand{\bound}{\Big|}
\newcommand{\BB}[1]{\left(#1\right)}
\newcommand{\dd}{\mathrm{d}}
\newcommand{\artanh}{\mathrm{artanh}}
\newcommand{\Med}{\mathrm{Med}}
\newcommand{\Cov}{\mathrm{Cov}}
\newcommand{\Corr}{\mathrm{Corr}}
\newcommand{\tr}{\mathrm{tr}}
\newcommand{\Range}[1]{\mathrm{range}(#1)}
\newcommand{\Null}[1]{\mathrm{null}(#1)}
\newcommand{\lan}{\langle}
\newcommand{\ran}{\rangle}
\newcommand{\norm}[1]{\left\lVert#1\right\rVert}
\newcommand{\inn}[1]{\lan#1\ran}
\newcommand{\op}[1]{\operatorname{#1}}
\newcommand{\bmat}[1]{\begin{bmatrix}#1\end{bmatrix}}
\newcommand{\pmat}[1]{\begin{pmatrix}#1\end{pmatrix}}
\newcommand{\vmat}[1]{\begin{vmatrix}#1\end{vmatrix}}

\newcommand{\amogus}{{\bigcap}\kern-0.8em\raisebox{0.3ex}{$\subset$}}
\newcommand{\Note}{\textbf{Note: }}
\newcommand{\Aside}{{\bf Aside: }}
%restriction
%\newcommand{\op}[1]{\operatorname{#1}}
%\newcommand{\done}{$$\mathcal{QED}$$}

%====================================================================%


\setlength{\parindent}{0pt}      	% No paragraph indentations
\pagestyle{fancy}
\fancyhf{}							% fancy header

\setcounter{secnumdepth}{0}			% sections are numbered but numbers do not appear
\setcounter{tocdepth}{2} 			% no subsubsections in toc

%template
%====================================================================%
%\begin{manualproblem}{1}
%Spivak.
%\end{manualproblem}

%\begin{proof}[Solution]
%\end{proof}

%----------- or -----------%

%\begin{problem} 		
%\end{problem}	

%\penum
%	\item
%\epenum
%====================================================================%


\newcommand{\Course}{MAT351}
\newcommand{\hwNumber}{10}

%preamble

\title{}
\author{A.N.}
\date{\today}
\lhead{\Course A\hwNumber}
\rhead{\thepage}
%\cfoot{\thepage}


%====================================================================%
\begin{document}



\begin{problem}
\end{problem}
We will solve for a radial solution of this PDE. If $u(r,\theta) = R(r)$. Taking the polar coordinate laplacian tells us that $R$ must satisfy 
$$R'' + \frac{1}{r}R  = 1,$$
with $R(1) = R(2) = 0$. This ODE wil be solved by $R(r) = A + B \log r + \frac{r^2}{4}$. To satisfy the boundary values, we must have that $0 = R(1) = A + \frac{1}{2}$ and $ 0 = R(2) = A + B \log 2 + 1$. Therefore $A = - \frac{1}{2}$, $B = -\frac{1}{2\log 2}$. 
\newpage
\begin{problem}
\end{problem}
\penum
\item Setting $v(x,y) = u(y,x)$ we have that $v$ is harmonic since 
$$\Delta v(x,y) = \Delta u(y,x) =u_{yy} + u_{xx} = 0.$$
Therefore $v$ must satisfy:
$$\begin{cases}
	v(0,y) = u(y,0) = 0 & y\in[0,1].
	\\v(x,0) = u(0,x) = 0 & x\in [0,1].
	\\ v(1,y) = u(y,1) = g(y) & y\in [0,1].
	\\ v(x,y) = u(1,x) = f(y) & x\in[0,1].	
\end{cases}$$
\item 
If $u$ is a solution of (squareLE) with $f = -h$ and $g = h$, then by linearity we have that $-u$ will solve with boundary conditions $f = h$ and $g = -h$. By uniqueness, this solution will be equal to $u(y,x)$ by part $a$. We must have that $u(x,y) = -u(y,x)$ and so $u(x,x) = -u(x,x)$ . Therefore $u(x,x) = 0$. 
\item
We wish to solve 
$$\begin{cases}
	\Delta u &= 0
	\\ u(0,y) &= 0
	\\ u(1,y) &= 0
	\\ u(x,1)& = -h(x)
	\\ u(1,y) &= h(y)
\end{cases} $$
By part $b$ this will be $0$ on the diagonal, and when restricted to the triangle will satisfy the desired boundary conditions and hence be unique. We first set the condition $u(x,1 ) = -h(x)$ to $0$ and solve by separation of variables. We get that 
$$u(x,y) = X(x)Y(y), \Delta u = 0 \implies \frac{X''}{X} = -\frac{Y''}{Y} = \lambda^2.$$
Therefore $X(x ) = A \sinh \lambda x + B \cosh \lambda x$, $Y(y) = C\sin \lambda y + D \cos \lambda x $. Boundary conditions on $Y$ are $Y(0) = Y(1) = 0$. Therefore $D = 0, \lambda = n\pi$ for $n\in \N$. We also have that $X(0) = 0$ so $B =0$. Therefore we can write
$$u(x,y) = \sum_{n\geq 1}^\infty  A_n \sinh n\pi y \sin n\pi x. $$
Since $u(1,y) = h(y)$, we have that 
$$h(y ) = \sum_{n\geq 1}^\infty A_n \sinh n\pi \sin n \pi x \implies A_n = \frac{2}{n\pi \sinh n\pi }\int_{0}^1 h(y) \sin n\pi x dx.$$
Now to solve with $u(1,y) = 0$, $u(x,1) = -h(x)$. We separate and get that 
$$\frac{X''}{X}  = - \frac{Y''}{Y} = -\lambda^2.$$
So we have that $X(x) = A\sin \lambda x + B \cos \lambda x, Y(y) = C \sinh \lambda y + D \cosh \lambda y$. The same process as above except interchanging the roles of $X,Y$ gives us that 
$$u(x,y) = \sum_{n\geq 1}^\infty A_n \sinh n\pi x \sin \pi y, \quad A_n = -\frac{2}{n\pi \sinh n\pi} \int_0^1 h(y) \sin n\pi x dx.$$
Therefore the solution on the triangle will be given by:
$$u = \sum_{n\geq 1}^\infty A_n \left[\sinh n\pi x \sin n\pi y - \sinh n\pi y \sin n\pi x\right]$$
\epenum
\newpage
\begin{problem}
\end{problem}
\penum
\item 
Without loss of generality translate so that $x=0$. We have that 
$$u(0) = \frac{1}{2\pi} \int_{0}^{2\pi} u(r \cos \theta, r \sin \theta)d\theta.$$
Multiplying by $s$ an integrating from $0$ to $r$, we have that 4
$$\int_0^r s u(0)dr =\frac{1}{2\pi} \int_0^r \int_0^{2\pi} s u(s \cos \theta ,  s  \sin \theta) ds d \theta.$$
The righthand side is the integral over the ball in spherical coordinates, so
$$\frac{r^2}{2}u(0) = \frac{1}{2\pi} \int_{B_r(0)} u(x) \implies u(0) = \frac{1}{\pi r^2} \int_{B_r(0)}u(x) dV.$$
Therefore if $u$ satisfies the mean value property on circles it satisfies it on balls as well. 
\item If $u(x) =0$ and $u \geq 0$ and satisfies mean value property, then for any $r$ so that $B_r(x) \subset \Omega$ We have that 
$$0 = u(x) = \frac{1}{\pi r^2} \int_{B_r(x)}u(x) dv.$$
Since $u$ is nonnegative it must be in fact $0$ on $B_r(x)$. Let $L = \{x: u(x) = 0 \}$. By the previous argument we have that $L$ is an open set. It is also nonempty by assumption. If $x\in \bd L$, take a ball $B_\ep(y)$ for some $y$ inside $L$ so that $x\in \bd B_\ep(y)$. By the mean value theorem for circles, we have that $u$ is $0$ on $\bd \bd B_\ep(y) $. So $x\in L$. The set $L$ is open and closed, and nonempty therefore it must be all of $\Omega$ since $\Omega$ is connected. 
\item Let $v$ be the harmonic function so that $v|_{\bd D} = u$. Suppose that $\Delta u \neq 0$. At some point $x$ we must have that $\Delta u(y) >0$ for $y\in B_\ep(x)$ for sufficiently small $\ep$. Then by linearity of integration we have that $u-v$ has mean value property. Furthermore, $u(0)-v(0)=0$, and $u-v$ is $0$ on $\bd D$. Now suppose that $u-v$ attains a minimum, $m$ on $D$. Then $u-v-m$ also satisfies the mean value property, vanishes at a point so it must be 0 on $D$ by part $b)$. Thus $u-v = m$. Since $u(0)-v(0)= 0$ we have that $m= 0$ and hence $u = v$. 
\epenum
\newpage
\begin{problem}
\end{problem}
\penum
\item Changing to spherical coordinates, we have that $v(r,\theta) = u(1/r, \theta)$. It follows from the chain rule that 
	$$\Delta v = v_{rr} + \frac{1}{r} v_r + \frac{1}{r^2}v_{\theta \theta} = (u(1/r, \theta))_{rr} + \frac{1}{r}(u(1/r), \theta)_{r}  + \frac{1}{r^2}(u(1/r, \theta))_{\theta \theta} = \frac{1}{r^4}u_{rr} + \frac{1}{r^5}u_r + \frac{1}{r^6}u_{\theta \theta}. $$
Therefore we have that $\Delta v = \frac{1}{r^4}\Delta u$. 
\item If we have that $\Delta u =0$, by above then $\Delta v = 0$. 
\item We can write 
$$u(r, \theta) = \frac{1}{2\pi}\int_0^{2\pi} h(\phi ) \frac{1- r^2}{1+r^2- 2r \cos \theta - \phi}.$$
Since $v = u(1/r)$, 
$$v(r, \theta) = \frac{1}{2\pi}\int_0^{2\pi} h(\phi ) \frac{1- (1/r)^2}{1+(1/r)^2 - (2/r) \cos \theta - \phi}d\phi.$$
Where $h(\phi) = u|_{\bd D}$. 
\item Let $v$ be the harmonic function on $D$ so that $u|\bd D = g$. Define $v$ so that $v(r, \theta) = u(1/r, \theta)$. This will be harmonic by $b)$. We have shown on past problem sets that the bounded solutions to Laplace's equation on $\R^2 \setminus D$ are unique so there is only one such $v$.  
Thus if $g = u |_{\bd D}$ by $c$ we can write 
$$v(r, \theta) = \frac{1}{2\pi}\int_0^{2\pi} g(\phi ) \frac{1- (1/r)^2}{1+(1/r)^2 - (2/r) \cos \theta - \phi}d\phi.$$
Taking the limit as $r \to \infty$, we have that 
$$\lim_{r \to \infty} v(r , \theta) = \int_0^{2\pi} g(\phi) d\phi$$
since $$\lim_{r \to \infty} \frac{1- (1/r)^2}{1+(1/r)^2 - (2/r) \cos \theta - \phi} = 1.$$
\epenum


\end{document}
