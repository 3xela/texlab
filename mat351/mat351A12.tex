\documentclass[12pt, a4paper]{article}
\usepackage[lmargin =0.5 in, 
rmargin=0.5in, 
tmargin=1in,
bmargin=0.5in]{geometry}
\geometry{letterpaper}
\usepackage{tikz-cd}
\usepackage{amsmath}
\usepackage{amssymb}
\usepackage{blindtext}
\usepackage{titlesec}
\usepackage{enumitem}
\usepackage{fancyhdr}
\usepackage{amsthm}
\usepackage{graphicx}
\usepackage{cool}
\usepackage{thmtools}
\usepackage{hyperref}
\graphicspath{ }					%path to an image

%-------- sexy font ------------%
%\usepackage{libertine}
%\usepackage{libertinust1math}

%\usepackage{mlmodern}				% very nice and classic
%\usepackage[utopia]{mathdesign}
%\usepackage[T1]{fontenc}


\usepackage{mlmodern}
\usepackage{eulervm}
%\usepackage{tgtermes} 				%times new roman
%-------- sexy font ------------%


% Problem Styles
%====================================================================%


\newtheorem{problem}{Problem}


\theoremstyle{definition}
\newtheorem{thm}{Theorem}
\newtheorem{lemma}{Lemma}
\newtheorem{prop}{Proposition}
\newtheorem{cor}{Corollary}
\newtheorem{fact}{Fact}
\newtheorem{defn}{Definition}
\newtheorem{example}{Example}
\newtheorem{question}{Question}

\newtheorem{manualprobleminner}{Problem}

\newenvironment{manualproblem}[1]{%
	\renewcommand\themanualprobleminner{#1}%
	\manualprobleminner
}{\endmanualprobleminner}

\newcommand{\penum}{ \begin{enumerate}[label=\bf(\alph*), leftmargin=0pt]}
	\newcommand{\epenum}{ \end{enumerate} }

% Math fonts shortcuts
%====================================================================%

\newcommand{\ring}{\mathcal{R}}
\newcommand{\N}{\mathbb{N}}                           % Natural numbers
\newcommand{\Z}{\mathbb{Z}}                           % Integers
\newcommand{\R}{\mathbb{R}}                           % Real numbers
\newcommand{\C}{\mathbb{C}}                           % Complex numbers
\newcommand{\F}{\mathbb{F}}                           % Arbitrary field
\newcommand{\Q}{\mathbb{Q}}                           % Arbitrary field
\newcommand{\PP}{\mathcal{P}}                         % Partition
\newcommand{\M}{\mathcal{M}}                         % Mathcal M
\newcommand{\eL}{\mathcal{L}}                         % Mathcal L
\newcommand{\T}{\mathbb{T}}                         % Mathcal T
\newcommand{\U}{\mathcal{U}}                         % Mathcal U\\
\newcommand{\V}{\mathcal{V}}                         % Mathcal V

% symbol shortcuts
%====================================================================%

\newcommand{\bd}{\partial}
\newcommand{\grad}{\nabla}
\newcommand{\lam}{\lambda}
\newcommand{\imp}{\implies}
\newcommand{\all}{\forall}
\newcommand{\exs}{\exists}
\newcommand{\delt}{\delta}
\newcommand{\ep}{\varepsilon}
\newcommand{\ra}{\rightarrow}
\newcommand{\vph}{\varphi}

\newcommand{\ol}{\overline}
\newcommand{\f}{\frac}
\newcommand{\lf}{\lfrac}
\newcommand{\df}{\dfrac}

% bracketting shortcuts
%====================================================================%
\newcommand{\abs}[1]{\left| #1 \right|}
\newcommand{\babs}[1]{\Big|#1\Big|}
\newcommand{\bound}{\Big|}
\newcommand{\BB}[1]{\left(#1\right)}
\newcommand{\dd}{\mathrm{d}}
\newcommand{\artanh}{\mathrm{artanh}}
\newcommand{\Med}{\mathrm{Med}}
\newcommand{\Cov}{\mathrm{Cov}}
\newcommand{\Corr}{\mathrm{Corr}}
\newcommand{\tr}{\mathrm{tr}}
\newcommand{\Range}[1]{\mathrm{range}(#1)}
\newcommand{\Null}[1]{\mathrm{null}(#1)}
\newcommand{\lan}{\langle}
\newcommand{\ran}{\rangle}
\newcommand{\norm}[1]{\left\lVert#1\right\rVert}
\newcommand{\inn}[1]{\lan#1\ran}
\newcommand{\op}[1]{\operatorname{#1}}
\newcommand{\bmat}[1]{\begin{bmatrix}#1\end{bmatrix}}
\newcommand{\pmat}[1]{\begin{pmatrix}#1\end{pmatrix}}
\newcommand{\vmat}[1]{\begin{vmatrix}#1\end{vmatrix}}

\newcommand{\amogus}{{\bigcap}\kern-0.8em\raisebox{0.3ex}{$\subset$}}
\newcommand{\Note}{\textbf{Note: }}
\newcommand{\Aside}{{\bf Aside: }}
%restriction
%\newcommand{\op}[1]{\operatorname{#1}}
%\newcommand{\done}{$$\mathcal{QED}$$}

%====================================================================%


\setlength{\parindent}{0pt}      	% No paragraph indentations
\pagestyle{fancy}
\fancyhf{}							% fancy header

\setcounter{secnumdepth}{0}			% sections are numbered but numbers do not appear
\setcounter{tocdepth}{2} 			% no subsubsections in toc

%template
%====================================================================%
%\begin{manualproblem}{1}
%Spivak.
%\end{manualproblem}

%\begin{proof}[Solution]
%\end{proof}

%----------- or -----------%

%\begin{problem} 		
%\end{problem}	

%\penum
%	\item
%\epenum
%====================================================================%


\newcommand{\Course}{MAT351}
\newcommand{\hwNumber}{12}

%preamble

\title{}
\author{A.N.}
\date{\today}
\lhead{\Course A\hwNumber}
\rhead{\thepage}
%\cfoot{\thepage}


%====================================================================%
\begin{document}



\begin{problem}
\end{problem}
\penum
\item By Kirchoff's formula, the solution will be given as
	$$u(x,t) = \frac{1}{4\pi c^2t} \int_{|y-x| = ct}g(y) dS_y + \partial_t \left[\frac{1}{4\pi c^2 t} \int_{|y-x| = ct} f(y) dy \right].$$
If $A$ is an orthogonal transformation, then 
\begin{align*}
	u(Ax, t) & = \frac{1}{4\pi c^2t} \int_{|y-Ax| = ct}g(y) dS_y + \partial_t \left[\frac{1}{4\pi c^2 t} \int_{|y-Ax| = ct} f(y) dy \right]
	\\ & = \frac{1}{4\pi c^2t} \int_{|A^{-1}y-x| = ct}g(y) dS_y + \partial_t \left[\frac{1}{4\pi c^2 t} \int_{|A^{-1}y-x| = ct} f(y) dy \right] \tag{Since $A$ preserves lengths}
	\\ & =  \frac{1}{4\pi c^2t} \int_{|y-x| = ct}g(A^{-1}y) dS_y + \partial_t \left[\frac{1}{4\pi c^2 t} \int_{|y-x| = ct} f(A^{-1}y) dy \right] \tag{Change of variables}
	\\ & =  \frac{1}{4\pi c^2t} \int_{|y-x| = ct}g(y) dS_y + \partial_t \left[\frac{1}{4\pi c^2 t} \int_{|y-x| = ct} f(y) dy \right] \tag{Since $f,g$ are radial}
	\\ & = u(x,t).
\end{align*}
\item Suppose $u$ is a solution of the wave equation for radially symmetric initial conditions. Then $$v(x,t) = u(Ax,t)$$ for $A$ orthogonal will solve the wave equation with the same initial conditions. 
By uniqueness, we have that $v = u$, and so $u(x,t) = u(Ax,t)$. 
\epenum
\newpage
\begin{problem}
\end{problem}
We can define the operator $\mathcal{S}$ as
$$S_t g(x) = \frac{1}{4\pi c^2 t} \int_{|y-x| = ct} g(y) dS_y.$$
Therefore the homogenous wave equation is solved by
$$u(x,t) = \partial_t \mathcal{S}_t \phi(x) + \mathcal{S}_t \psi(x). $$
From this it follows by duhamels principle that for the inhomogenous wave equation, that
$$u(x,t) = \int_0^t S_{t-s} f(x,s) ds = \int_0^t \frac{1}{4\pi c^2(t-s)^2} \int_{|y-x| = c(t-s)} f(y,s)dS_y ds. $$
\newpage
\begin{problem}
\end{problem}
\penum 
\item Suppose that $\dot{E}(t) \leq a(t) E(t)$. Dividing both sides by $E(t)$ and integrating from $t_0$ to $t$, we see
$$\int_{t_0}^t \frac{\dot{E}(s)}{E(s)} ds \leq \int_{t_0}^t a(s) ds. $$
Evaluating the lefthand side integral, we get: 
$$\log E(t) - \log E(0) \leq \int_{t_0}^t a(s) ds.$$
Rearranging and exponentiating we get
$$E(t) \leq e^{\int_{t_0}^t a(s) ds} E(t_0)$$
as desired. 
\item Define
	$$h(t) = \exp \left(-\int_0^t a(s)ds \right)\left(\int_0^t a(s) E(s) ds + c(t) \right).$$
We compute its derivative. 
\begin{align*}
	\frac{d}{dt}h(t)  & = \exp \left(-\int_0^t a(s)ds \right) \cdot -a(t) c(t) +\exp \left(-\int_0^t a(s)ds \right) c^\prime(t) \\ & + \exp \left(-\int_0^t a(s)ds \right) \cdot -a(t) \int_0^t E(s)a(s) ds + \exp \left(-\int_0^t a(s)ds \right)E(t)a(t)
	\\ & = \exp \left(-\int_0^t a(s)ds \right) \left[-a(t)c(t) - a(t) \int_0^t E(s)a(s) ds + E(t)a(t) \right] + c^\prime(t) \exp \left(-\int_0^t a(s)ds \right)
	\\ & \leq \exp \left(-\int_0^t a(s)ds \right) c^\prime(t) \tag{Multiplying assumed inequality by $a(t)$}.
\end{align*} 
Therefore we have 
$$h^\prime(s) \leq c^\prime(s) \exp \left(-\int_0^s a(r)dr \right) .$$
Integrating from $0$ to $t$ we get:
	$$h(t)  = h(0) + \int_0^t c^\prime(s) \exp\left(\int_s^0 a(r) dr\right)ds $$
By the definition of $h(t)$, we have that 
$$ \exp \left(-\int_0^t a(s)ds \right)\left(\int_0^t a(s) E(s) ds + c(t) \right) \leq  h(0) + \int_0^t c^\prime(s) \exp\left(\int_t^0 a(r) dr\right)ds.$$
We multiply by $\exp \left( \int_0^t a(s)ds\right)$ and apply the inequality on $E(t)$ and get:
$$E(t) \leq h(0)\exp \left( \int_0^t a(s)ds\right) + \exp \left( \int_0^t a(s)ds\right)\int_0^tc^\prime(s) \exp\left(\int_s^0 a(r) dr \right)ds.$$
Finally we notice that $h(0) = c$, and we can bring in the constant on the second summand and use the properties of $\exp$ to get that:
$$E(t) \leq c(0) \exp \left( \int_0^t a(s) ds \right) + \int_0^t c^\prime(s) \exp\left(\int_s^t a(r)dr \right)ds.$$
\epenum
\newpage
\begin{problem}
\end{problem}
Recall by Kirchoff's formula we have
$$t|u(x,t)| = \left|\frac{1}{4\pi } \int_{S} h(y) dS_y + t \partial_t\frac{1}{4\pi t} \int_{S} f(y) dS_y \right| \leq \left|\frac{1}{4\pi} \int_{S} h(y) dS_y \right| + t\left| \partial_t\frac{1}{4\pi } \int_{S} f(y) dS_y \right| $$
Where $S = \{y: |x-y| = t\}$. Since $h$ has compact support, we have 
$$ \left|\frac{1}{4\pi t} \int_{S} h(y) dS_y \right| \leq \frac{\norm{h}_{L^1}}{4\pi }.$$
For the second summand, we have that:
$$t \left| \partial_t\frac{1}{4\pi t} \int_{S} f(y) dS_y \right|  =t \left|-\frac{1}{4\pi t^2} \int_{S} f(y) dS_y + \frac{1}{4\pi t} \partial_t \int_S f(y) dy \right| \leq  \left|-\frac{1}{4\pi t} \int_{S} f(y) dS_y\right| +\left| \frac{1}{4\pi } \partial_t \int_S f(y) dy \right|.$$
We can bound the first term by performing a change of variables: 
$$\frac{1}{4\pi t} \left|\int_{S} f(y) dS_y\right| = \frac{1}{4\pi t}\left|t \int_{S/t}f(y) dS_y \right| \leq \frac{1}{4\pi} \sup f \cdot |B_1|. $$
The second summand is bounded in a similar way: 
$$\frac{1}{4\pi} \left| \partial_t \int_S f(y) dS_y \right| = \frac{1}{4\pi} \left|\partial_t \int_{B_1(x)} tf(y) dS_y \right|\leq \frac{1}{4\pi} \sup |f| \cdot |B_1(x)|.  $$
Therefore $t |u(x,t)| \leq C = \frac{1}{4\pi}\left(\norm{h}_{L^1} + 2\sup |f|\cdot |B_1| \right)$. 
\newpage
\begin{problem}
\end{problem}
\penum 
\item First in the case of linear klein gordon equation, we multiply the PDE by $u_t$ to get:
$$u_t u_{tt} - \Delta u u_t + \mu u u_t = 0.$$
Notice that we can write this as
$$\left(\frac{1}{2} u^2_t + \frac{1}{2} |\grad u|^2 + \frac{\mu}{2} u^2 \right)_t - \grad \cdot ( u_t \grad u) = 0.$$
Furthermore the second summand goes to $0$ as $|x| \to \infty$, so we can integrate over $\R^n$ to get that
$$ \frac{d}{dt} \int_{\R^n} \frac{1}{2} u^2_t + \frac{1}{2} |\grad u|^2 + \frac{\mu}{2} u^2  dx = 0.$$
Therefore our conserved quantity is 
$$ \int_{\R^n} \frac{1}{2} u^2_t + \frac{1}{2} |\grad u|^2 + \frac{\mu}{2} u^2  dx = 0.$$
Now consider for when $ f= |u|^{p-1} u$. Define $E(t) =  \int_{\R^n} \frac{1}{2} u^2_t + \frac{1}{2} |\grad u|^2 + \frac{\mu}{2} u^2  dx$, and set $\frac{d}{dt}E(t) = 0:$
We compute that
$$0 = \frac{d}{dt} E(t) =- \int_{\R^n} -|u|^{p-1} u u_t dx =   -\int_{\R^n} \frac{d}{dt} \left(\frac{1}{p}|u|^p u\right) dx,$$
Assuming $u$ decays sufficiently fast. 
Therefore the conserved quantity is 
$$E = \int_{\R^n} \frac{1}{2}u_t^2 +  \frac{1}{2} |\grad u|^2 + \frac{\mu}{2} u^2 + \frac{1}{p} |u|^{p-1}p dx. $$
\item Let $u_1,u_2$ be given with corresponding initial conditions $f,g$ solving inhomogenous Klein-Gordon Equation. Define $w = u_1-u_2$. 
Then the energy of $w$ is given as:
$$E(t) = \int_{\R^n}\frac{1}{2} w_t^2 + \frac{1}{2} |\grad w|^2 + \frac{\mu}{2} w^2  dx.$$
Since $w$ solves homogenous Klein-Gordon equation, we must have that this is conserved. Taking the derivative, we get that 
$$0 = \frac{d}{dt} E(t) = \int_{\R^n}w_t w_{tt}+ \grad w \cdot \grad w_t +\mu  w w_t dx.$$
Integrating by parts, we get that
$$0 = \int_{\R^n} \Delta w_t-  w_{ttt} - w_{tt} dx .$$
Thus this satisfies an inhomogenous wave equation, with $f = w_{t}$. By $Q2$, we have that $$w(x,t) = \int_0^t \frac{1}{4\pi (t-s)^2} \int_{|y-x| = t}\frac{w_t(y, t - |y-x|)}{|y-x|} dS_y.$$
Since $w$ is bounded, this must be identically $0$. Therefore solutions to Klein-gordon are Unique.
\epenum

\end{document}
