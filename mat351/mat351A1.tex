\documentclass[12pt, a4paper]{article}
\usepackage[lmargin =0.5 in, 
rmargin=0.5in, 
tmargin=1in,
bmargin=0.5in]{geometry}
\geometry{letterpaper}
\usepackage{tikz-cd}
\usepackage{amsmath}
\usepackage{amssymb}
\usepackage{blindtext}
\usepackage{titlesec}
\usepackage{enumitem}
\usepackage{fancyhdr}
\usepackage{amsthm}
\usepackage{graphicx}
\usepackage{cool}
\usepackage{thmtools}
\usepackage{hyperref}
\graphicspath{ }					%path to an image

%-------- sexy font ------------%
%\usepackage{libertine}
%\usepackage{libertinust1math}

%\usepackage{mlmodern}				% very nice and classic
%\usepackage[utopia]{mathdesign}
%\usepackage[T1]{fontenc}


\usepackage{mlmodern}
\usepackage{eulervm}
%\usepackage{tgtermes} 				%times new roman
%-------- sexy font ------------%


% Problem Styles
%====================================================================%


\newtheorem{problem}{Problem}


\theoremstyle{definition}
\newtheorem{thm}{Theorem}
\newtheorem{lemma}{Lemma}
\newtheorem{prop}{Proposition}
\newtheorem{cor}{Corollary}
\newtheorem{fact}{Fact}
\newtheorem{defn}{Definition}
\newtheorem{example}{Example}
\newtheorem{question}{Question}

\newtheorem{manualprobleminner}{Problem}

\newenvironment{manualproblem}[1]{%
	\renewcommand\themanualprobleminner{#1}%
	\manualprobleminner
}{\endmanualprobleminner}

\newcommand{\penum}{ \begin{enumerate}[label=\bf(\alph*), leftmargin=0pt]}
	\newcommand{\epenum}{ \end{enumerate} }

% Math fonts shortcuts
%====================================================================%

\newcommand{\ring}{\mathcal{R}}
\newcommand{\N}{\mathbb{N}}                           % Natural numbers
\newcommand{\Z}{\mathbb{Z}}                           % Integers
\newcommand{\R}{\mathbb{R}}                           % Real numbers
\newcommand{\C}{\mathbb{C}}                           % Complex numbers
\newcommand{\F}{\mathbb{F}}                           % Arbitrary field
\newcommand{\Q}{\mathbb{Q}}                           % Arbitrary field
\newcommand{\PP}{\mathcal{P}}                         % Partition
\newcommand{\M}{\mathcal{M}}                         % Mathcal M
\newcommand{\eL}{\mathcal{L}}                         % Mathcal L
\newcommand{\T}{\mathcal{T}}                         % Mathcal T
\newcommand{\U}{\mathcal{U}}                         % Mathcal U\\
\newcommand{\V}{\mathcal{V}}                         % Mathcal V

% symbol shortcuts
%====================================================================%

\newcommand{\bd}{\partial}
\newcommand{\grad}{\nabla}
\newcommand{\lam}{\lambda}
\newcommand{\imp}{\implies}
\newcommand{\all}{\forall}
\newcommand{\exs}{\exists}
\newcommand{\delt}{\delta}
\newcommand{\ep}{\varepsilon}
\newcommand{\ra}{\rightarrow}
\newcommand{\vph}{\varphi}

\newcommand{\ol}{\overline}
\newcommand{\f}{\frac}
\newcommand{\lf}{\lfrac}
\newcommand{\df}{\dfrac}

% bracketting shortcuts
%====================================================================%
\newcommand{\abs}[1]{\left| #1 \right|}
\newcommand{\babs}[1]{\Big|#1\Big|}
\newcommand{\bound}{\Big|}
\newcommand{\BB}[1]{\left(#1\right)}
\newcommand{\dd}{\mathrm{d}}
\newcommand{\artanh}{\mathrm{artanh}}
\newcommand{\Med}{\mathrm{Med}}
\newcommand{\Cov}{\mathrm{Cov}}
\newcommand{\Corr}{\mathrm{Corr}}
\newcommand{\tr}{\mathrm{tr}}
\newcommand{\Range}[1]{\mathrm{range}(#1)}
\newcommand{\Null}[1]{\mathrm{null}(#1)}
\newcommand{\lan}{\langle}
\newcommand{\ran}{\rangle}
\newcommand{\norm}[1]{\left\lVert#1\right\rVert}
\newcommand{\inn}[1]{\lan#1\ran}
\newcommand{\op}[1]{\operatorname{#1}}
\newcommand{\bmat}[1]{\begin{bmatrix}#1\end{bmatrix}}
\newcommand{\pmat}[1]{\begin{pmatrix}#1\end{pmatrix}}
\newcommand{\vmat}[1]{\begin{vmatrix}#1\end{vmatrix}}

\newcommand{\amogus}{{\bigcap}\kern-0.8em\raisebox{0.3ex}{$\subset$}}
\newcommand{\Note}{\textbf{Note: }}
\newcommand{\Aside}{{\bf Aside: }}
%restriction
%\newcommand{\op}[1]{\operatorname{#1}}
%\newcommand{\done}{$$\mathcal{QED}$$}

%====================================================================%


\setlength{\parindent}{0pt}      	% No paragraph indentations
\pagestyle{fancy}
\fancyhf{}							% fancy header

\setcounter{secnumdepth}{0}			% sections are numbered but numbers do not appear
\setcounter{tocdepth}{2} 			% no subsubsections in toc

%template
%====================================================================%
%\begin{manualproblem}{1}
%Spivak.
%\end{manualproblem}

%\begin{proof}[Solution]
%\end{proof}

%----------- or -----------%

%\begin{problem} 		
%\end{problem}	

%\penum
%	\item
%\epenum
%====================================================================%


\newcommand{\Course}{351}
\newcommand{\hwNumber}{1}

%preamble

\title{}
\author{A.N.}
\date{\today}
\lhead{\Course A\hwNumber}
\rhead{\thepage}
%\cfoot{\thepage}


%====================================================================%
\begin{document}



\begin{problem}
\end{problem}
\penum
\item This is linear and homogenous. Consider the operator $\mathcal{L}= \partial_t + t^2 \partial_x $. This is a linear operator, and we have that $\mathcal{L}u=0$ is our desired PDE. 
\item This is a nonlinear PDE, since we have a term with $uu_x$. It is not fully nonlinear since it is linear in $u_{xxx}$. 
\item This is inhomogenous linear PDE, since it is of the form $Lu=g$, where $L = \partial_t^2 + \partial_x, g=t^2$.  
\item This is a totally nonlinear PDE. It is nonlinear in each of the partial derivatives. 
\epenum
 \newpage 
\begin{problem}
\end{problem}
\penum
\item Integrating $u_x$ by $x$, we get that $$\int u_x dx = \sin(xy) + \frac{A}{2} x^2y + F(y) \quad (1),$$ for some arbitrary function $F(y)$.
	Similarly, for $u_y$ we compute that $$ \int u_ydy = sin(xy) + 3x^2y + arctan(y) + G(x). \quad (2)$$
Thus we must have that $A=6$ for this PDE to have a solution. 
\item Given that $u(0,0) = B$, using our expressions from $a)$ we get that $$B = u(0,0) = F(0) = G(0).$$
By differentiating expression $2$ with respect to $x$, we get that $G_x=0$. So $G=B$. Therefore we have that $u(x,y) = sin(xy)+3x^2y+\arctan(y)+B$ is the solution to the PDE. 
\item We claim that a necessary condition on $f$ is ${f_1}_y ={f_2}_x$. Suppose that $u$ solves $\grad u = f$. 
Then on any closed curve $\gamma$ with interiour $D$, we have that $$0 = \int_\gamma \grad u = \int_\gamma f = \int_D (f_{1y} - f_{2x}) dy dx $$
Since this holds for all $\gamma, D$, we have that $f_{1y} = f_{2x}$. 
\epenum
\newpage 
\begin{problem}
\end{problem}
Maxwells Equations tells us that $\frac{1}{c} E_t = \grad \times B$. Applying $\partial_t$ we get that 
\begin{align*}\frac{1}{c} E_{tt} & = \partial_t(\grad \times B)
	\\& = \grad \times B_t \tag{since derivatives commute}
	\\& = - \frac{1}{c} \grad \times ( \grad \times E) \tag{Maxwells Equations}
	\\& =\frac{1}{c} (E_{xx},E_{yy}, E_{zz})\tag{Vector Calc identity for $\grad \cdot E=0$} .
\end{align*}
Similarly for $B$ we compute that: 
\begin{align*}
	\frac{1}{c} B_{tt}& = - \partial_t(\grad \times E)
	\\ & = - \grad \times E_t \tag{since deriavtives commute}
	\\ & = -\frac{1}{c}  \grad \times (\grad \times B) \tag{Maxwells Equations}
	\\ & = \frac{1}{c} (B_{xx}, B_{yy}, B_{zz}) \tag{Vector calc identity for $\grad \cdot B=0$}
\end{align*}
Therefore every component of $E,B$ satisfy the wave equation. 
 \newpage 
\begin{problem}
\end{problem}
We wish to show that $\int_{\R^n} \grad \cdot F(x) dx = 0$. Let $B(r)$ be the ball of radius $r$.  We have by the divergence theorem, that
\begin{align*}
	\left| \int_{\R^n} \grad \cdot F(x) dx \right| & = \lim_{r \to \infty} \left|\int_{B(r)} \left( F(x) \cdot n(x) \right) dx  \right|\tag{Divergence theorem on $\R^n$}
	\\& \leq \lim_{r\to \infty} \int_{B(r)} |F(x)|\cdot |n(x)| dx \tag{Integral ineqality + Cauchy-Schwartz}
	\\ & \leq \lim_{r\to \infty} \int_{B(r)} |F(x)| dx \tag{since $|n(x)|=1$}
	\\ & \leq \lim_{r\to \infty} \int_{B(r)} C |x|^{-n}dx \tag{upper bound on $|F(x)|$}
	\\ & \leq \lim_{r \to \infty} C |r|^{-n} \cdot \sigma(S^{n-1}) |r|^{n-1}
	\tag{since $ \int_{B(r)}1 dx = \sigma(S^{n-1})r^{n-1}$, where $\sigma(S^{n-1}) =\int_{S^{n-1}} 1 dx $ }
	\\ & = \lim_{r \to \infty} \frac{C \sigma(S^{n-1})}{r}
	\\ & = 0.
\end{align*}
We conclude that $\int_{\R^n} \grad \cdot F(x) dx = 0$. 
\newpage 
\begin{problem}
\end{problem}
\penum

\item We claim that $\int_D f = \int_{\bd D}g$ is necessary. Observe that if $u$ is a solution to this PDE, then by the divergence theorem we have that 
	$$\int_D f = \int_D \Delta u = \int_D \grad \cdot \grad u = \int_{\bd D} \grad u \cdot n(x) = \int_{\bd D} g . $$
Thus $\int_D f = \int_{\bd D} g$ is a necessary condition. 
\item Suppose that $u$ solves this PDE. Then for any constant $c$ we also have that
	$$ \Delta(u+c)=\Delta u = f, \frac{\partial(u+c)}{\partial n} = \frac{\partial u}{\partial n} + \frac{\partial c}{\partial n} = g$$

\epenum

\end{document}
