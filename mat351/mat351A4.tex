\documentclass[12pt, a4paper]{article}
\usepackage[lmargin =0.5 in, 
rmargin=0.5in, 
tmargin=1in,
bmargin=0.5in]{geometry}
\geometry{letterpaper}
\usepackage{tikz-cd}
\usepackage{amsmath}
\usepackage{amssymb}
\usepackage{blindtext}
\usepackage{titlesec}
\usepackage{enumitem}
\usepackage{fancyhdr}
\usepackage{amsthm}
\usepackage{graphicx}
\usepackage{cool}
\usepackage{thmtools}
\usepackage{hyperref}
\graphicspath{ }					%path to an image

%-------- sexy font ------------%
%\usepackage{libertine}
%\usepackage{libertinust1math}

%\usepackage{mlmodern}				% very nice and classic
%\usepackage[utopia]{mathdesign}
%\usepackage[T1]{fontenc}


\usepackage{mlmodern}
\usepackage{eulervm}
%\usepackage{tgtermes} 				%times new roman
%-------- sexy font ------------%


% Problem Styles
%====================================================================%


\newtheorem{problem}{Problem}


\theoremstyle{definition}
\newtheorem{thm}{Theorem}
\newtheorem{lemma}{Lemma}
\newtheorem{prop}{Proposition}
\newtheorem{cor}{Corollary}
\newtheorem{fact}{Fact}
\newtheorem{defn}{Definition}
\newtheorem{example}{Example}
\newtheorem{question}{Question}

\newtheorem{manualprobleminner}{Problem}

\newenvironment{manualproblem}[1]{%
	\renewcommand\themanualprobleminner{#1}%
	\manualprobleminner
}{\endmanualprobleminner}

\newcommand{\penum}{ \begin{enumerate}[label=\bf(\alph*), leftmargin=0pt]}
	\newcommand{\epenum}{ \end{enumerate} }

% Math fonts shortcuts
%====================================================================%

\newcommand{\ring}{\mathcal{R}}
\newcommand{\N}{\mathbb{N}}                           % Natural numbers
\newcommand{\Z}{\mathbb{Z}}                           % Integers
\newcommand{\R}{\mathbb{R}}                           % Real numbers
\newcommand{\C}{\mathbb{C}}                           % Complex numbers
\newcommand{\F}{\mathbb{F}}                           % Arbitrary field
\newcommand{\Q}{\mathbb{Q}}                           % Arbitrary field
\newcommand{\PP}{\mathcal{P}}                         % Partition
\newcommand{\M}{\mathcal{M}}                         % Mathcal M
\newcommand{\eL}{\mathcal{L}}                         % Mathcal L
\newcommand{\T}{\mathbb{T}}                         % Mathcal T
\newcommand{\U}{\mathcal{U}}                         % Mathcal U\\
\newcommand{\V}{\mathcal{V}}                         % Mathcal V

% symbol shortcuts
%====================================================================%

\newcommand{\bd}{\partial}
\newcommand{\grad}{\nabla}
\newcommand{\lam}{\lambda}
\newcommand{\imp}{\implies}
\newcommand{\all}{\forall}
\newcommand{\exs}{\exists}
\newcommand{\delt}{\delta}
\newcommand{\ep}{\varepsilon}
\newcommand{\ra}{\rightarrow}
\newcommand{\vph}{\varphi}

\newcommand{\ol}{\overline}
\newcommand{\f}{\frac}
\newcommand{\lf}{\lfrac}
\newcommand{\df}{\dfrac}

% bracketting shortcuts
%====================================================================%
\newcommand{\abs}[1]{\left| #1 \right|}
\newcommand{\babs}[1]{\Big|#1\Big|}
\newcommand{\bound}{\Big|}
\newcommand{\BB}[1]{\left(#1\right)}
\newcommand{\dd}{\mathrm{d}}
\newcommand{\artanh}{\mathrm{artanh}}
\newcommand{\Med}{\mathrm{Med}}
\newcommand{\Cov}{\mathrm{Cov}}
\newcommand{\Corr}{\mathrm{Corr}}
\newcommand{\tr}{\mathrm{tr}}
\newcommand{\Range}[1]{\mathrm{range}(#1)}
\newcommand{\Null}[1]{\mathrm{null}(#1)}
\newcommand{\lan}{\langle}
\newcommand{\ran}{\rangle}
\newcommand{\norm}[1]{\left\lVert#1\right\rVert}
\newcommand{\inn}[1]{\lan#1\ran}
\newcommand{\op}[1]{\operatorname{#1}}
\newcommand{\bmat}[1]{\begin{bmatrix}#1\end{bmatrix}}
\newcommand{\pmat}[1]{\begin{pmatrix}#1\end{pmatrix}}
\newcommand{\vmat}[1]{\begin{vmatrix}#1\end{vmatrix}}

\newcommand{\amogus}{{\bigcap}\kern-0.8em\raisebox{0.3ex}{$\subset$}}
\newcommand{\Note}{\textbf{Note: }}
\newcommand{\Aside}{{\bf Aside: }}
%restriction
%\newcommand{\op}[1]{\operatorname{#1}}
%\newcommand{\done}{$$\mathcal{QED}$$}

%====================================================================%


\setlength{\parindent}{0pt}      	% No paragraph indentations
\pagestyle{fancy}
\fancyhf{}							% fancy header

\setcounter{secnumdepth}{0}			% sections are numbered but numbers do not appear
\setcounter{tocdepth}{2} 			% no subsubsections in toc

%template
%====================================================================%
%\begin{manualproblem}{1}
%Spivak.
%\end{manualproblem}

%\begin{proof}[Solution]
%\end{proof}

%----------- or -----------%

%\begin{problem} 		
%\end{problem}	

%\penum
%	\item
%\epenum
%====================================================================%


\newcommand{\Course}{351}
\newcommand{\hwNumber}{4}

%preamble

\title{}
\author{A.N.}
\date{\today}
\lhead{\Course A\hwNumber}
\rhead{\thepage}
%\cfoot{\thepage}


%====================================================================%
\begin{document}



\begin{problem}
\end{problem}
\penum
\item We claim that $P(t)$ is conserved. We compute that: 
\begin{align*}
	\dot{P} (t) &= \int \frac{d}{dt} u_t u_x dx
	\\& = \int u_{tt} u_x + u_{tx} u_t dx
	\\ & = \int c^2 u_{xx} u_x + u_{tx} u_x dx
\end{align*}
Computing each summand seperately, we have that 
$$\int u_{tx} u_t dx = u_t \cdot u_t|_\R - \int u_{tx} u_t dx = - \int u_{tx} u_t dx \implies \int u_{tx} u_t dx = 0,$$ assuming that $u_t$ vanishes at $\infty$. Similarly if we assume that $u_{x}$ vanishes at $\infty$. we have that 
$$\int u_{xx} u_x dx = -\int u_xx u_x dx \implies \int u_xx u_x dx = 0. $$
\item We compute: 
$$e_t = u_t \cdot u_{tt} + c^2 u_{x} \cdot u_{xt}= c^2 \left(u_{tx}\cdot u_x + \frac{1}{c^2} u_t \cdot u_{tt} 	\right) = c^2 p_x.$$
Similarly, we compute that 
$$p_t = u_{tt} \cdot u_t + u_t \cdot u_{xt} = e_x.$$
Thus we see that $$e_{tt} = c^2 p_{xt}, e_{xx} = p_{xt} \implies e_{tt} - c^2 e_{xx} = 0.$$
For $p$ we get $$p_{tt} = e_{xt}, c^2p_{xx} = e_{xt} \implies p_{xx} - \frac{1}{c^2} p_{tt}.$$
\item Since we have that $E, P$ are both conserved. Therefore they are independant of the choice of $t$. So we can write $$E(t) = E(0) = \int \frac{1}{2} \phi_0^2(x) + c^2\phi_1^2(x)dx,$$
and $$P(t) = \int \phi_0(x) \phi_1(x) dx$$
\epenum
 \newpage 
\begin{problem}
\end{problem}
\penum
\item As a reminder we write $$E(t) = \int \frac{1}{2} u_t^2 + c^2 u_x^2 dx.$$
Taking the time derivative, we see
$$\frac{d}{dt} E(t)  = \int u_t u_{tt} + c^2 u_x u_{xt} dx = \int u_t( u_{tt} - c^2 u_{xx}) dx = -r \int u_t^2 dx.$$
Where we use integration by parts, and the PDE condition. Note that $E(0)$ is positive, and the derivative is decreasing strictly. Therefore as $t\to \infty$ then $E(t) \to 0$.  
\item Yes solutions are unique. Suppose $u_1, u_2$ are both solutions with the same initial datum $u^i(x,0) = \phi_0(x)$, $u_t^i(x,0) = \phi_1(x)$. Write $v = u_1 - u_2$. Note that $v$ has 0 boundary conditions. Note that $E(0)=0$, but is also always at least $0$. Since $E$ is decreasing we have that $E(t) = 0$ for all time. Therefore $v \equiv 0$ and $u_1 = u_2$. 
\item  If $r<0$ then $$\dot{E}(t) = -r \int u_t^2 dx >0. $$
Thus the energy must go off to infinity as time goes to infinity. Note that we can write 
$$\dot{E}(t) = -2rE(t) - r\int c^2 u_{xx} dx \implies \dot{E}(t) \leq -3rE(t). $$
By Gronwalls inequality, we have that $E(t) = E(0) e^{-3rt}$. Therefore if we take two initial value problems as we did in $b)$,  and apply energy to their difference we have that $E(0) = 0$. By gronwalls we have $E(t) = 0$ for all $t$. Thus $v \equiv 0$ and so uniqueness also holds. 
\epenum
 \newpage 
\begin{problem}
\end{problem}
\penum
\item If $L$ lorentz, we have $$L^{-1} = \Gamma L^T \Gamma.$$
Since $\Gamma^{-1} = \Gamma$ taking the inverse of both sides we see that $$L = \Gamma {L^{-1}}^T \Gamma.$$
Therefore $L^{-1}$ is lorentz. 
Now suppose that $L,M$ are both lorzentz. Then, 
$$(LM)^{-1} = M^{-1} L^{-1} = \Gamma M^T \Gamma \Gamma L^T \Gamma = \Gamma(LM)^T \Gamma.$$
\item Note that $m(w) = w^T \Gamma w$. First suppose that $L$ is lorentz. Then, 
$$m(Lw) = (Lw)^T \Gamma Lw = w^T \Gamma \Gamma L^T  \Gamma L w = w^T \Gamma L^{-1} L L w = w \Gamma w.$$
Suppose the converse. Then we have that $$w^TL^T \Gamma L w = w^T \Gamma w.$$
Since this is true for all $w$, we have that $L^T\Gamma L = \Gamma$, and so $L$ is invertible. Thus we have that $$L^T \Gamma  = \Gamma L^{-1} \implies L^{-1} = \Gamma L^T \Gamma. $$
\item We write the following vector $v = \bmat{\partial_x & \partial_y & \partial_z & \partial_t}$.
We have that $m(v)u = 0$. Therefore $m(Lv)u =0$ or 
$$(v L^T \Gamma L v )u = u(L(x,y,z,t))$$
\item We write the matrix of $L$ as follows: 
$$L = \bmat{\gamma & 0 & 0 & -\gamma v\\ 0 & 1 & 0 & 0\\ 0 & 0 & 1 & 0 \\ -\gamma v & 0 & 0 & \gamma}.$$
We compute the inverse as 
$$L^{-1} = \bmat{-\frac{-1}{\gamma (v+1)(v-1)} & 0 & 0 & \frac{v}{\gamma - \gamma v^2}\\ 0 & 1 & 0 & 0\\ 0 & 0 & 1 & 0 \\ \frac{v}{\gamma - \gamma v^2} & 0 & 0 & \frac{1}{\gamma - \gamma v^2}}.$$
Using the fact that $L^{-1} = \Gamma L^{T} \Gamma$, we have that $$\bmat{-\frac{-1}{\gamma (v+1)(v-1)} & 0 & 0 & \frac{v}{\gamma - \gamma v^2}\\ 0 & 1 & 0 & 0\\ 0 & 0 & 1 & 0 \\ \frac{v}{\gamma - \gamma v^2} & 0 & 0 & \frac{1}{\gamma - \gamma v^2}} = \bmat{\gamma & 0 & 0 & \gamma v\\ 0 & 1 & 0 & 0\\ 0 & 0 & 1 & 0 \\ \gamma v & 0 & 0 & \gamma}.$$
For this to hold i.e. $L$ be lorenztiam we must have that $v \in(-1,1), \gamma>1$ and 
$$\gamma^2 = \frac{1}{1-v^2}	$$
\epenum
 \newpage 
\begin{problem}
\end{problem}
\penum
\item Let $w = v-u$. Note that $w\geq 0$ on the boundary. If at some point $a,b$ we had that $u-v(a,b)>0$, then $w(a,b)<0$. This can not happen by the minimum principle. 
\item We first claim that if $u$ solves $u_{t} = u_{xx} + f$, then $\min_{int} u \geq \min_{bd} u + l^2 \min_{bd} f$. Consider the function $v = u + cx^2 - l^2c$, where we choose $ c= \frac{\min f}{2}$. We have that $v_t= u_t$ and $v_{xx} = u_{xx} + \frac{\min f}{2}$. We have that $v_t \geq v_{xx}$, Therefore $v$ attains its min when $x=0$ or $x=l$ or $t=0$ by the minimum principle.
Therefore we have $\min_{int} u \geq \min_{bd} u + l^2 \min_{bd} f$. We now apply this to $w = v-u$. 
We must have that $\min_{int } w \geq \min_{bd} w + l^2 \min_{bd}l^2(g-f)$. The lefthand side ofe the equality must always be positive since the righthand side is always positive. Therefore $v\geq u$. 
\item Take $u(t,x) = (1-e^{-t})\sin x$. We see that $u_t - u_{xx} = \sin x$. As well, we have $u(0,x) =\sin x $, $u(t, 0) = 0 = u(t, \pi)$. Since we have that $v \geq u$ on the boundary, then by part $b$ it must also be true on the interiour $i.e.$ , $v(x,t) \geq (1-e^{-t})\sin x$.  
\epenum
 \newpage 
\begin{problem}
\end{problem}
 \penum 
 \item We define the function $v(x,t) = u(x,t) - \ep\left(t + \frac{1}{2} |x-y|^2\right)$ for arbitrary $y$. Set $\rho = |x-y|$. Define 
$$\Omega = \left\{x,t : |x-y|< \rho, 0<t<T \right\}.$$
By the proof of the weak max principle, have that $v$ must satisfy the weak max principle, so therefore $$v(x,t) \leq \max\left\{ f(x) - \frac{\ep}{2} \rho^2 , u(y-\rho, t) - \ep t +\frac{\ep}{2} \rho^2 , u(y+\rho, t) - \frac{\ep}{2} + \frac{\ep}{2} \rho^2 \right\}.$$
If we take $x=y$, then for all $y,\rho$, 
$$v(y,t)\leq \sup_{x\in \R} f(x), $$ since we can range $x $ over all $\R$. Since this is true for all $\ep$, we have that $u(y,t) \leq \sup_{x\in \R} f(x)$. 
Since this system is linear if we have two solutions $u_1, u_2$ for the initial value $f(x)$, their difference will be a solution to the same equation with initial value $0$. Since the solution is bounded above by $0$ is must be identically $0$. 
\item We have solved the heat equation on the real line. We have that $$u(x,t) = \frac{1}{2 \sqrt{k t}} \int_{\R} e^{\frac{-(x-y)^2}{4kt}} f(y) dy. $$
Therefore, $$|u(x,t)| = \Big| \frac{1}{2 \sqrt{k t}} \int_{\R} e^{\frac{-(x-y)^2}{4kt}} f(y) dy\Big| \leq C t^{-1/2} \int_\R e^{\frac{-(x-y)^2}{4kt}}dy \cdot \int_\R |f(y)|dy  \leq  Dt^{-1/2},$$
Where we use the fact that $\int e^{\frac{-(x-y)^2}{4kt}} dy $ is a constant independant of $t$, and $f\in L^1$. 
 \epenum
\end{document}
