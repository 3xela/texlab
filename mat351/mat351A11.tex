\documentclass[12pt, a4paper]{article}
\usepackage[lmargin =0.5 in, 
rmargin=0.5in, 
tmargin=1in,
bmargin=0.5in]{geometry}
\geometry{letterpaper}
\usepackage{tikz-cd}
\usepackage{amsmath}
\usepackage{amssymb}
\usepackage{blindtext}
\usepackage{titlesec}
\usepackage{enumitem}
\usepackage{fancyhdr}
\usepackage{amsthm}
\usepackage{graphicx}
\usepackage{cool}
\usepackage{thmtools}
\usepackage{hyperref}
\graphicspath{ }					%path to an image

%-------- sexy font ------------%
%\usepackage{libertine}
%\usepackage{libertinust1math}

%\usepackage{mlmodern}				% very nice and classic
%\usepackage[utopia]{mathdesign}
%\usepackage[T1]{fontenc}


\usepackage{mlmodern}
\usepackage{eulervm}
%\usepackage{tgtermes} 				%times new roman
%-------- sexy font ------------%


% Problem Styles
%====================================================================%


\newtheorem{problem}{Problem}


\theoremstyle{definition}
\newtheorem{thm}{Theorem}
\newtheorem{lemma}{Lemma}
\newtheorem{prop}{Proposition}
\newtheorem{cor}{Corollary}
\newtheorem{fact}{Fact}
\newtheorem{defn}{Definition}
\newtheorem{example}{Example}
\newtheorem{question}{Question}

\newtheorem{manualprobleminner}{Problem}

\newenvironment{manualproblem}[1]{%
	\renewcommand\themanualprobleminner{#1}%
	\manualprobleminner
}{\endmanualprobleminner}

\newcommand{\penum}{ \begin{enumerate}[label=\bf(\alph*), leftmargin=0pt]}
	\newcommand{\epenum}{ \end{enumerate} }

% Math fonts shortcuts
%====================================================================%

\newcommand{\ring}{\mathcal{R}}
\newcommand{\N}{\mathbb{N}}                           % Natural numbers
\newcommand{\Z}{\mathbb{Z}}                           % Integers
\newcommand{\R}{\mathbb{R}}                           % Real numbers
\newcommand{\C}{\mathbb{C}}                           % Complex numbers
\newcommand{\F}{\mathbb{F}}                           % Arbitrary field
\newcommand{\Q}{\mathbb{Q}}                           % Arbitrary field
\newcommand{\PP}{\mathcal{P}}                         % Partition
\newcommand{\M}{\mathcal{M}}                         % Mathcal M
\newcommand{\eL}{\mathcal{L}}                         % Mathcal L
\newcommand{\T}{\mathbb{T}}                         % Mathcal T
\newcommand{\U}{\mathcal{U}}                         % Mathcal U\\
\newcommand{\V}{\mathcal{V}}                         % Mathcal V

% symbol shortcuts
%====================================================================%

\newcommand{\bd}{\partial}
\newcommand{\grad}{\nabla}
\newcommand{\lam}{\lambda}
\newcommand{\imp}{\implies}
\newcommand{\all}{\forall}
\newcommand{\exs}{\exists}
\newcommand{\delt}{\delta}
\newcommand{\ep}{\varepsilon}
\newcommand{\ra}{\rightarrow}
\newcommand{\vph}{\varphi}

\newcommand{\ol}{\overline}
\newcommand{\f}{\frac}
\newcommand{\lf}{\lfrac}
\newcommand{\df}{\dfrac}

% bracketting shortcuts
%====================================================================%
\newcommand{\abs}[1]{\left| #1 \right|}
\newcommand{\babs}[1]{\Big|#1\Big|}
\newcommand{\bound}{\Big|}
\newcommand{\BB}[1]{\left(#1\right)}
\newcommand{\dd}{\mathrm{d}}
\newcommand{\artanh}{\mathrm{artanh}}
\newcommand{\Med}{\mathrm{Med}}
\newcommand{\Cov}{\mathrm{Cov}}
\newcommand{\Corr}{\mathrm{Corr}}
\newcommand{\tr}{\mathrm{tr}}
\newcommand{\Range}[1]{\mathrm{range}(#1)}
\newcommand{\Null}[1]{\mathrm{null}(#1)}
\newcommand{\lan}{\langle}
\newcommand{\ran}{\rangle}
\newcommand{\norm}[1]{\left\lVert#1\right\rVert}
\newcommand{\inn}[1]{\lan#1\ran}
\newcommand{\op}[1]{\operatorname{#1}}
\newcommand{\bmat}[1]{\begin{bmatrix}#1\end{bmatrix}}
\newcommand{\pmat}[1]{\begin{pmatrix}#1\end{pmatrix}}
\newcommand{\vmat}[1]{\begin{vmatrix}#1\end{vmatrix}}

\newcommand{\amogus}{{\bigcap}\kern-0.8em\raisebox{0.3ex}{$\subset$}}
\newcommand{\Note}{\textbf{Note: }}
\newcommand{\Aside}{{\bf Aside: }}
%restriction
%\newcommand{\op}[1]{\operatorname{#1}}
%\newcommand{\done}{$$\mathcal{QED}$$}

%====================================================================%


\setlength{\parindent}{0pt}      	% No paragraph indentations
\pagestyle{fancy}
\fancyhf{}							% fancy header

\setcounter{secnumdepth}{0}			% sections are numbered but numbers do not appear
\setcounter{tocdepth}{2} 			% no subsubsections in toc

%template
%====================================================================%
%\begin{manualproblem}{1}
%Spivak.
%\end{manualproblem}

%\begin{proof}[Solution]
%\end{proof}

%----------- or -----------%

%\begin{problem} 		
%\end{problem}	

%\penum
%	\item
%\epenum
%====================================================================%


\newcommand{\Course}{MAT351}
\newcommand{\hwNumber}{11}

%preamble
\title{} \author{A.N.}
\date{\today}
\lhead{\Course A\hwNumber}
\rhead{\thepage}
%\cfoot{\thepage}


%====================================================================%
\begin{document}



\begin{problem}
\end{problem}
Assume $G_1,G_2$ are both greens functions on domain $D$. Then at any point $y$, we have that the follwing must be harmonic: 
$$H_i(x,y) =G_i(x,y) + \frac{1}{4\pi |x-y|}. $$
Since harmonic functions form a vector space it follows that their difference is harmonic as well:
$$H_1 - H_2 = G_1 - G_2. $$
By Green's First identity for Harmonic functions, we have that 
$$\int_{\bd D} (G_1 - G_2)(x,y) \frac{\partial ( G_1 - G_2)}{\partial n}(x,y) dx = \int_D |\grad(G_1 - G_2)(x,y)|^2dx.$$
Since $G_1,G_2$ vanish on the boundary, the above is equal to 0 for all $y$. It follows that $G_1(x,y)$ and $G_2(x,y)$ differ by a constant. Since they are both $0$ on the boundary that constant must be 0. 
\newpage
\begin{problem}
\end{problem}
First note that we can decompose $u(x)$ in the following way. We write $u(x) = u_1(x) + u_2(x)$ where 
$$\begin{cases} \Delta u_2& = f \text{ in D}
	\\ u_2& = 0 \text{ on $\bd D$}
\end{cases} \quad,
\begin{cases}
	\Delta u_1 &= 0 \text{ in D}
	\\ u_1& = h \text{ on $\bd D$} 
\end{cases}.$$
It was proven in Strauss that 
$$u_1(x) = \int_{\bd D} h(y) \frac{\partial G}{\partial n} (x,y) dy.$$
It remains to show that 
$$u_2(x) = \int_D f(y) G(x,y) dy.$$
Consider the harmonic function $H(x,y) = G(x,y) + \frac{1}{4\pi |x-y|}$. By Greens second identity, 
$$\int_D u_2 \Delta H - \Delta u_2 dy = \int_{\bd D} u_2 \frac{\partial H}{\partial n} - H \frac{\partial u_2}{\partial n} dS(y).$$
By the Harmonicity of $H$ and boundary values of $u_2$ this reduces to 
$$\int_D f(y) H(x,y) dy = \int_{\bd D} \frac{1}{4\pi |y-x|}\frac{\partial u_2}{\partial n}(y) dS_y.$$
We can rewrite this as: 
$$\int_D f(y)G (x,y) dy = \int_{\bd D} \frac{1}{4\pi |y-x|} \frac{\partial u_2}{\partial n} (y) dS_y - \int_{D} \Delta u_2(y) \frac{1}{4\pi |y-x|} dy.$$
For $x\in int D$ take $\ep>0$ and define $D_\ep = D \setminus B_\ep(x)$. Split up the righthandside in the following way: 
$$\int_{\bd D } \frac{1}{4\pi |x-y|} \frac{\partial u_2}{\partial n} (y) dy - \left(\int_{D_\ep} \Delta u_2(y) \frac{1}{4\pi |y-x|} dy + \int_{B_\ep(x)}\Delta u_2(y) \frac{4\pi |y-x|} d(y) \right).$$
Since the the function is smooth on $D_\ep$  and harmonic we apply Greens second identity to get: 
$$  \left( \int_{\partial B_\ep (x) } \frac{1}{4\pi|x-y|} \frac{\partial u_2}{\partial n}(y)  + \int_{\bd B_\ep(x) }\grad \frac{1}{4\pi |x-y|} \grad u_2  \right) - \int_{B_\ep(x)} \frac{f(x)}{4\pi |x-y|} dy.$$
Changing to Spherical coordinates, we get
$$\int_{\bd B_\ep(x)} \frac{1}{4\pi r} \frac{\partial u_2}{\partial r} dS_r + \int_{\bd B_\ep(x)} \grad\frac{1}{4\pi r}  \cdot \grad {u_2} dV - \int_{B_\ep(x)} \frac{f(r)}{4\pi r} dV.$$
This is equal to 
$$ \ep^2 \ol{u} + \ol{u} + \ep^2\ol{f}.$$
Where the overline indicates we are averaging over the ball of radius $\ep$. Note that as we take $\ep \to 0$, the $\ep$ terms go to $0$, and since $u $ is continuous, we have that $\ol{u} = u_2(x)$.
Therefore, $$\int_{D} f(y) G(x,y) dy = u_2(x),$$
and so $$u(x) = \int_{\bd D} h(y) G(x,y) dS_y + \int_{D} f(y) G(x,y) dy. $$
\newpage
\begin{problem}
\end{problem}
\penum
\item Let $v$ be a smooth function so that $v|_{\bd U} = 0$. Then for any $u\in \mathcal{A}$, $u+ \ep v\in \mathcal{A}$. Then
	$$I[u + \ep v] = \frac{1}{2} \int_U \left|\grad u  +\ep \grad v  \right|^2 dx-  \frac{1}{4} \int_U (u + \ep v)^4 dx.$$
	Expanding this out we get:
	$$I[u+ \ep v] = \frac{1}{2} \int_U |\grad u|^2 + 2 \ep \grad u \cdot \grad v + \ep^2 |\grad v^2| dx -  \frac{1}{4} \int_U (u + \ep v)^4 dx.$$
A minimizer will satisfy $\frac{d}{d \ep }\Big|_{\ep = 0} I[u + \ep v]$. Applying this we get that
$$\frac{d}{d\ep} \Big|_{\ep = 0} I[u + \ep v] = \int_U \grad u \cdot \grad v - \int_U  u^3 v dx = 0.$$
Applying Greens First identity to the first summand, we get that 
$$0 = \int_U \grad u \cdot \grad v - \int_U u^3 v dx = \int_U  -v \Delta udx - \int_U v u^3 dx = \int_U v \left( u^3+  \Delta u \right) dx,$$
Since $v$ vanishes near the boundary of $U$. Furthermore, for any $U^\prime \subset \subset V \subset \subset U$, we can take $v$ so that $v \equiv 1$ on $U^\prime$ and $v \equiv 0$ on $V^c$. 
This means that 
$$0= \int_{U^\prime} (u^3+ \Delta u)dx $$
For any $U^\prime \subset \subset U$. It follows that $\Delta u = -u^3$. 
\item Taking $v \in C^\infty_c([0,T] \times U)$, we have
$$\eL [u + \ep v] = \frac{1}{2} \int_0^T \int_U (u + \ep v)_t^2 - |\grad u + \ep \grad v|^2 dx dt.$$
Taking the functional derivative, we compute:
\begin{align*}
	\frac{d}{d\ep} \Big|_{\ep = 0} \eL [u + \ep v]  & =\int_0^T\int_U u_t v_t - \grad u \cdot \grad v dx dt
\\ & = \int_U \int_0^T u_t v_T dt dx- \int_0^T \int_U \grad u \cdot \grad v dx dt
	\\ & = \int_U \int_0^T  - v u_{tt} dt dx + \int_0^T \int_U v \Delta u dx dt
	\\ & = \int_0^T \int_U v \left( \Delta u - u_{tt}\right) dx dt. 
\end{align*}
Once again, we can take $v$ so that it is compactly supported and identically equal to $1$ on a comapct set and get that
$$\int_0^{T^\prime}\int_{U^\prime}\Delta u - u_{tt} dx dt = 0.$$
Therefore the energy minimizing functions will satisfy $\Delta u = u_{tt}$ i.e. solutions to the $n-$dimensional wave equation. 
\epenum
\newpage
\begin{problem}
\end{problem}
\penum
\item Let $w\in C^2(\Omega)$ satisfying the same boundary conditions as $u$, say $u|_{\bd \Omega} = w|_{\bd \Omega} = h$. Let $v = u-w$. Then
\begin{align*} 
	J[w] & = \frac{1}{2} \int_\Omega |\grad w|^2 dx- \int_\Omega fwdx\tag{By Definition}
	\\ & = \frac{1}{2} \int_\Omega |\grad u - \grad v|^2 dx - \int_\Omega f(u-v) dx \tag{Definition of $v$}
	\\ & =\frac{1}{2} \int_\Omega |\grad u|^2 - 2 \grad u \cdot \grad v + |\grad v|^2 dx - \int_\Omega ufdx + \int_{\Omega} vfdx \tag{Expanding out}
	\\ & = \left( \frac{1}{2} \int_\Omega |\grad u|^2dx - \int_{\Omega} ufdx \right) + \frac{1}{2} \int_{\Omega} |\grad v|^2 - 2 \grad u \cdot \grad vdx  + \int_{\Omega} vfdx \tag{Rearranging Summands}
	\\ & = J[u] + \left( \frac{1}{2} \int_\Omega |\grad v|^2 - 2 \grad u \cdot \grad vdx  - \int_\Omega v\Delta u dx \right)
	\\ & = J[u] + \frac{1}{2} \int_{\Omega} |\grad v|^2 dx  - \int_{\bd \Omega} v \frac{\partial u}{\partial n} dx \tag{Greens First Identity}
	\\ & = J[u] + \frac{1}{2}\int_\Omega |\grad v|^2 dx \tag{Since $v = 0$ on $\partial \Omega$}
\end{align*}
Since$ |\grad v|^2 \geq 0$ we have that $J[w] \geq J[u]$. Equality holds if and only of $|\grad v|^2 = 0$ i.e. if $w = v$. 
\item Consider the following Dirichlet Problem : 
	$$\begin{cases}
		\Delta u & = -1 \text{ in D}
		\\	u &= 0 \text{ on $\bd D$}
	\end{cases}.$$
	The solution to this will be the minimizer of the functional $$J[u] = \frac{1}{2} \int_{D} |\grad u|^2 dx dy - \int_{D} u dxdy$$
over the class of $C^2$ functions on the disk with boundary value 0 i.e. for all $w$ with $w|_{\bd D } = 0$
$$J[w] \geq \frac{1}{2}\int_{D} |\grad u|^2 - \int_{D} u.  $$
Where $u$ solves the above dirichlet problem. This will be solved in polar coordinates by $u(r,\theta) = \frac{1}{4} (1 - r^2).$ We compute the integral in polar coordinates as: 
\begin{align*}
	\frac{1}{2} \int_D |\grad u|^2 - \int_D u& = \frac{1}{2} \int_0^{2\pi} \int_0^1 \frac{r^2}{4}r dr d\theta  - \int_0^{2\pi} \int_0^2 \frac{1}{4} (1-r^2) r dr \theta \tag{By definition of Grad in polar coordiantes}
	\\ & = \frac{\pi}{16} - \frac{\pi}{8}
	\\ & = -\frac{\pi}{16}
\end{align*}
Therefore by part $a$ any function $u$ satistfying $u|_{\bd D} = 0$ must satisfy 
$$J[u] \geq - \frac{\pi}{16}.$$
\epenum
\newpage
\begin{problem}
\end{problem}
\penum 
\item We write $(x,y,z,t) = (x,t) $. Consider the change of coordinates $(x,t) \mapsto \lambda (x,t)$. Then: 
	$$\Delta u(\lambda x,  \lambda t) = \lambda^2 \Delta u( x,  t) = \lambda^2 u_{tt}(x,t) = (u(\lambda x, \lambda t))_{tt}. $$
It follows that as a function of $\lambda$,
$$\square (u(\lambda x, \lambda t)) = 0.$$
Taking the derivative with respect to lambda, we compute:
$$0 = \frac{d}{d\lambda} \square u(\lambda x, \lambda t) = \square \left(\frac{d}{d\lambda} u(\lambda, \lambda t) \right) = \square \left( x u_x  +  y u_y + z u_z + t u_t   \right)(\lambda x, \lambda t). $$
At $\lambda =1$ we have that 
$$\square\left( (x\partial_x + y \partial_y + z \partial_z + t \partial_t)u\right) = 0$$
So this function also satisfies the wave equation.
\item It has been shown that any Lorentz transformation preserves the solutions to the wave equation. Consider the transformation parametrized as:
	$$L(x,y,z,t) = (\gamma(x - vt), y,z,\gamma(t - vx)).$$
Therefore we have that $$\square u(L(x,t)) =u(\gamma(x- vt) , y,z,\gamma(t-vx)) =  0.$$
Taking the derivative with respect to $v$, we have that 
$$\frac{d}{dv} \square u(L(x,t)) = \square \frac{d}{dv} u(L(x,t)) =\square (-\gamma t u_x -\gamma x u_t ) = 0.$$
Recall that when $v = 0$, $\gamma =1$ so at $v = 0$ we have that 
$$\square((t \partial_x + x \partial_ t)u) = 0$$
So this solve the wave equation. 
\item Applying $\square$ to $xu_y - yu_x$ we see:
\begin{align*}
	\square( xu_y - yu_x) & =  \square (xu_y) - \square(yu_x)
	\\ & = \left(u_{yx} + u_{yx}+ xu_{yxx} + xu_{yyy} + xu_{yzz}  - xu_{ytt} \right) - \left(yu_{xxx} + u_{xy} + u_{xy} + y_{xyy}   + yu_{zzx} - yu_{xtt} \right).
	\\ & = x (u_{xx} + u_{yy} + u_{zz} - u_{tt})_y + y(u_{xx} + u_{yy} + u_{zz} - u_{tt} )_x
	\\ & = 0 \tag{since $\square u = 0$}
\end{align*}
\item This is not invariant under scaling. Observe that if $(x,t) \mapsto (\lambda x, \lambda t)$ and $u$ is a solution,
	$$u_{tt} (\lambda x, \lambda t) - \Delta u(\lambda x,\lambda t) + \mu u(\lambda x, \lambda t) = \lambda^2 u_{tt} - \lambda^2 \Delta u + \mu u.$$
This will not automatically be 0 when $\lambda \neq \pm 1$.
Suppose now that $u$ solves the Klein-Gordon Equation. Applying $\square + \mu$ to $xu_y - yu_x$ we see:
\begin{align*}
	(\square + \mu I ) \left( xu_y - yu_x\right) & =  \square( xu_y - yu_x) + \mu (xu_y  - yu_x)
	\\ & = ( xu_{tt} - x\Delta u + x\mu u)_y + (yu_{tt} - y \Delta u+ y \mu u)_{x}
\end{align*}
This will be 0. 
\epenum

\end{document}
