\documentclass[12pt, a4paper]{article}
\usepackage[lmargin =0.5 in, 
rmargin=0.5in, 
tmargin=1in,
bmargin=0.5in]{geometry}
\geometry{letterpaper}
\usepackage{tikz-cd}
\usepackage{amsmath}
\usepackage{amssymb}
\usepackage{blindtext}
\usepackage{titlesec}
\usepackage{enumitem}
\usepackage{fancyhdr}
\usepackage{amsthm}
\usepackage{graphicx}
\usepackage{cool}
\usepackage{thmtools}
\usepackage{hyperref}
\graphicspath{ }					%path to an image

%-------- sexy font ------------%
%\usepackage{libertine}
%\usepackage{libertinust1math}

%\usepackage{mlmodern}				% very nice and classic
%\usepackage[utopia]{mathdesign}
%\usepackage[T1]{fontenc}


\usepackage{mlmodern}
\usepackage{eulervm}
%\usepackage{tgtermes} 				%times new roman
%-------- sexy font ------------%


% Problem Styles
%====================================================================%


\newtheorem{problem}{Problem}


\theoremstyle{definition}
\newtheorem{thm}{Theorem}
\newtheorem{lemma}{Lemma}
\newtheorem{prop}{Proposition}
\newtheorem{cor}{Corollary}
\newtheorem{fact}{Fact}
\newtheorem{defn}{Definition}
\newtheorem{example}{Example}
\newtheorem{question}{Question}

\newtheorem{manualprobleminner}{Problem}

\newenvironment{manualproblem}[1]{%
	\renewcommand\themanualprobleminner{#1}%
	\manualprobleminner
}{\endmanualprobleminner}

\newcommand{\penum}{ \begin{enumerate}[label=\bf(\alph*), leftmargin=0pt]}
	\newcommand{\epenum}{ \end{enumerate} }

% Math fonts shortcuts
%====================================================================%

\newcommand{\ring}{\mathcal{R}}
\newcommand{\N}{\mathbb{N}}                           % Natural numbers
\newcommand{\Z}{\mathbb{Z}}                           % Integers
\newcommand{\R}{\mathbb{R}}                           % Real numbers
\newcommand{\C}{\mathbb{C}}                           % Complex numbers
\newcommand{\F}{\mathbb{F}}                           % Arbitrary field
\newcommand{\Q}{\mathbb{Q}}                           % Arbitrary field
\newcommand{\PP}{\mathcal{P}}                         % Partition
\newcommand{\M}{\mathcal{M}}                         % Mathcal M
\newcommand{\eL}{\mathcal{L}}                         % Mathcal L
\newcommand{\T}{\mathbb{T}}                         % Mathcal T
\newcommand{\U}{\mathcal{U}}                         % Mathcal U\\
\newcommand{\V}{\mathcal{V}}                         % Mathcal V

% symbol shortcuts
%====================================================================%

\newcommand{\bd}{\partial}
\newcommand{\grad}{\nabla}
\newcommand{\lam}{\lambda}
\newcommand{\imp}{\implies}
\newcommand{\all}{\forall}
\newcommand{\exs}{\exists}
\newcommand{\delt}{\delta}
\newcommand{\ep}{\varepsilon}
\newcommand{\ra}{\rightarrow}
\newcommand{\vph}{\varphi}

\newcommand{\ol}{\overline}
\newcommand{\f}{\frac}
\newcommand{\lf}{\lfrac}
\newcommand{\df}{\dfrac}

% bracketting shortcuts
%====================================================================%
\newcommand{\abs}[1]{\left| #1 \right|}
\newcommand{\babs}[1]{\Big|#1\Big|}
\newcommand{\bound}{\Big|}
\newcommand{\BB}[1]{\left(#1\right)}
\newcommand{\dd}{\mathrm{d}}
\newcommand{\artanh}{\mathrm{artanh}}
\newcommand{\Med}{\mathrm{Med}}
\newcommand{\Cov}{\mathrm{Cov}}
\newcommand{\Corr}{\mathrm{Corr}}
\newcommand{\tr}{\mathrm{tr}}
\newcommand{\Range}[1]{\mathrm{range}(#1)}
\newcommand{\Null}[1]{\mathrm{null}(#1)}
\newcommand{\lan}{\langle}
\newcommand{\ran}{\rangle}
\newcommand{\norm}[1]{\left\lVert#1\right\rVert}
\newcommand{\inn}[1]{\lan#1\ran}
\newcommand{\op}[1]{\operatorname{#1}}
\newcommand{\bmat}[1]{\begin{bmatrix}#1\end{bmatrix}}
\newcommand{\pmat}[1]{\begin{pmatrix}#1\end{pmatrix}}
\newcommand{\vmat}[1]{\begin{vmatrix}#1\end{vmatrix}}

\newcommand{\amogus}{{\bigcap}\kern-0.8em\raisebox{0.3ex}{$\subset$}}
\newcommand{\Note}{\textbf{Note: }}
\newcommand{\Aside}{{\bf Aside: }}
%restriction
%\newcommand{\op}[1]{\operatorname{#1}}
%\newcommand{\done}{$$\mathcal{QED}$$}

%====================================================================%


\setlength{\parindent}{0pt}      	% No paragraph indentations
\pagestyle{fancy}
\fancyhf{}							% fancy header

\setcounter{secnumdepth}{0}			% sections are numbered but numbers do not appear
\setcounter{tocdepth}{2} 			% no subsubsections in toc

%template
%====================================================================%
%\begin{manualproblem}{1}
%Spivak.
%\end{manualproblem}

%\begin{proof}[Solution]
%\end{proof}

%----------- or -----------%

%\begin{problem} 		
%\end{problem}	

%\penum
%	\item
%\epenum
%====================================================================%


\newcommand{\Course}{MAT351}
\newcommand{\hwNumber}{8}

%preamble

\title{}
\author{A.N.}
\date{\today}
\lhead{\Course A\hwNumber}
\rhead{\thepage}
%\cfoot{\thepage}


%====================================================================%
\begin{document}



\begin{problem}
\end{problem}
We perform a separation of variables, writing $u(x,t) = X(x)T(t)$. The PDE gives us that
$$T^\prime (t)X(x) = i X^{\prime\prime}(x)T(t) \implies \frac{T^\prime(t)}{T(t)} = i\frac{X^{\prime \prime}(x)}{X(x)} = -i\lambda^2.$$
This means that $X(x) = A\sin \lambda x + B \cos \lambda x$. The conditions $X(\pi) = X(-\pi) = 0$ implies that $B=0$, and $\lambda = n$. Therefore $T(t) = Ce^{-in^2 t}$. 
We can thus write $$u(x,t)= Ce^{-in^2t}\sin nx = Ce^{-in^2}\left( e^{-inx} + e^{inx} \right).$$
Therefore we can write the fourier series of $u$ as $$u(x,t)  = \sum_{-\infty}^\infty \phi_n e^{-in^2 t}e^{inx}.$$
The initial conditions give us that $$\phi(x) = u(x,0) = \sum_{-\infty}^\infty\phi_m e^{imx}.$$
Multiplying by $e^{-in x}$ and integrating, we get $$\int_0^{2\pi} \phi(x)e^{-inx} = \sum_{-\infty}^\infty \phi_m \int_0^{2\pi}e^{i(m-n)x}dx.$$
Note that for $n\neq m$, $$\int_0^{2\pi} e^{i(n-m)x} dx= \frac{-i}{n-m} e^{i(n-m)x}\Big|_0^{2\pi}= 0.$$
When $n=m$ we integrate $1$ over $[0,2\pi]$ so we get $2\pi$. Therefore $$2\pi \phi_n = \int_0^{2pi}\phi(x) e^{-inx}dx.$$
\newpage
\begin{problem}
\end{problem}
The complex fourier coefficients, $C_n^h$ of $h(x)$ are given as:
\begin{align*}
	C_n^h &= \frac{1}{2\pi}\int_0^{2\pi} e^{-inx}h(x) dx
	\\ & = \frac{1}{2\pi}\int_0^{2\pi}e^{-inx}  \int_0^{2\pi}f(x-y)g(y)dy dx
	\\ & = \frac{1}{2\pi}\int_0^{2\pi} \int_0^{2\pi} e^{-in(x +y-y)} f(x-y)g(y) dy dx \tag{Fubini's Theorem}
	\\ & = C_n^f \int_0^{2\pi} e^{-iny}g(y) dy 
	\\ & =2\pi C_n^f C_n^g
\end{align*}
\newpage
\begin{problem}
\end{problem}
Recall that $L^2$ is a Hilbert Space. The inner product obeys the polarization identity;
$$\int_0^{2\pi} f(x)\ol{g(x)}dx = \inn{f,g} = \frac{1}{4} \left[\norm{f+g}^2 - \norm{f-g}^2 + i\norm{f+ig}^2 - i \norm{f-ig}^2\right].$$
We apply Parseval's Identity to the right-hand side: 
\begin{align*}
	\int_0^{2\pi} f(x)\ol{g(x)}dx & = \frac{\pi}{2} \sum_{-\infty}^\infty |f_n+g_n|^2 - |f_n-g_n|^2 + i|f_n+ig_n|^2 - i|f_n-ig_n|^2
	\\ & = \frac{\pi}{2} \sum_{-\infty}^\infty |f_n|^2 + f_n\ol{g_n} + \ol{f_n}g_n + |g_n|^2- (|f_n|^2 - f_n\ol{g_n} - \ol{f_n}g_n +|g_n|^2) 
	 \\& + i(|f_n|^2 - i f_n\ol{g_n} + i\ol{f_n}g_n +|g_n|^2) - i(|f_n|^2 + i f_n\ol{g_n} - i\ol{f_n} g_n +|g_n|^2 )
	 \\ & = \frac{\pi}{2 } \sum_{-\infty}^\infty 4f_n \ol{g_n}
	 \\ & = 2\pi \sum_{-\infty}^\infty f_n \ol{g_n}
\end{align*}
\newpage
\begin{problem}
\end{problem}
By Parseval's Identity, we compute: 
\begin{align*}
	\norm{f-\ol{f}}^2 &= 2\pi \sum_{-\infty}^\infty |f_n - \ol{f}_n|^2 
	\\ & = 2\pi \sum_{-\infty}^\infty \Big|  \int_0^{2\pi } e^{-inx}f(x) dx - \int_0^{2\pi} e^{-inx}\ol{f}dx\Big|^2
	\\ & = 2\pi \sum_{-\infty}^\infty \Big|\frac{1}{-in} \int_0^{2\pi} e^{-inx}f^\prime(x)dx \Big|^2 \tag{integrating by parts, also constant term is 0}
	\\ & = 2\pi \sum_{-\infty}^\infty \frac{1}{n^2} \Big| \in_0^{2\pi} e^{-inx}f^\prime(x) dx \Big|
	\\ & = 2\pi \sum_{-\infty}^\infty \frac{1}{n^2} |f_n^\prime|^2
	\\ & \leq 2\pi \sum_{-\infty}^\infty |f_n^\prime|^2 \tag{since $\frac{|f^\prime_n}{n}\leq |f_n^\prime|$}
	\\ & = \norm{f_n^\prime}^2
\end{align*}
\newpage
\begin{problem}
\end{problem}
Let $u$ be subharmonic. A point $x_0$ will occur if $\Delta u(x_0) \leq 0$. Since $u$ is subharmonic we must have that $\Delta u (x_0)= 0$. Define $v = u + \ep|x|^2$. We have that $\Delta v = \Delta u + 2\ep n \geq 2\ep n>0$. Our function $v$ can not have a maximum in the interior. 
So $$\max_{x\in \ol{D}} u(x) \leq\max_{x\in \ol{D}} v(x) \leq \max_{x\in \bd D} v(x).$$
The max of $v$ is attained at some point $x_0$ in the boundary, so we have that $u(x) \leq u(x_0) + \ep |x_0|^2$. Since
$\ep$ was arbitrary, we have that $u(x) \leq u(x_0)$ for $x_0\in \bd D$. 
Note that the minimum may be attained in the interiour. Consider $f(x) = x(x-1)$ on $(0,1)$. We have that $\Delta f = 2 \geq 0$, and the maximum is attained exactly on $\bd D = \{0,1\}$. The minimum is attained on the interiour at $x= \frac{1}{2}$. 

\end{document}