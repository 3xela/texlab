\documentclass[12pt, a4paper]{article}
\usepackage[lmargin =0.5 in, 
rmargin=0.5in, 
tmargin=1in,
bmargin=0.5in]{geometry}
\geometry{letterpaper}
\usepackage{tikz-cd}
\usepackage{amsmath}
\usepackage{amssymb}
\usepackage{blindtext}
\usepackage{titlesec}
\usepackage{enumitem}
\usepackage{fancyhdr}
\usepackage{amsthm}
\usepackage{graphicx}
\usepackage{cool}
\usepackage{thmtools}
\usepackage{hyperref}
\graphicspath{ }					%path to an image

%-------- sexy font ------------%
%\usepackage{libertine}
%\usepackage{libertinust1math}

%\usepackage{mlmodern}				% very nice and classic
%\usepackage[utopia]{mathdesign}
%\usepackage[T1]{fontenc}


\usepackage{mlmodern}
\usepackage{eulervm}
%\usepackage{tgtermes} 				%times new roman
%-------- sexy font ------------%


% Problem Styles
%====================================================================%


\newtheorem{problem}{Problem}


\theoremstyle{definition}
\newtheorem{thm}{Theorem}
\newtheorem{lemma}{Lemma}
\newtheorem{prop}{Proposition}
\newtheorem{cor}{Corollary}
\newtheorem{fact}{Fact}
\newtheorem{defn}{Definition}
\newtheorem{example}{Example}
\newtheorem{question}{Question}

\newtheorem{manualprobleminner}{Problem}

\newenvironment{manualproblem}[1]{%
	\renewcommand\themanualprobleminner{#1}%
	\manualprobleminner
}{\endmanualprobleminner}

\newcommand{\penum}{ \begin{enumerate}[label=\bf(\alph*), leftmargin=0pt]}
	\newcommand{\epenum}{ \end{enumerate} }

% Math fonts shortcuts
%====================================================================%

\newcommand{\ring}{\mathcal{R}}
\newcommand{\N}{\mathbb{N}}                           % Natural numbers
\newcommand{\Z}{\mathbb{Z}}                           % Integers
\newcommand{\R}{\mathbb{R}}                           % Real numbers
\newcommand{\C}{\mathbb{C}}                           % Complex numbers
\newcommand{\F}{\mathbb{F}}                           % Arbitrary field
\newcommand{\Q}{\mathbb{Q}}                           % Arbitrary field
\newcommand{\PP}{\mathcal{P}}                         % Partition
\newcommand{\M}{\mathcal{M}}                         % Mathcal M
\newcommand{\eL}{\mathcal{L}}                         % Mathcal L
\newcommand{\T}{\mathbb{T}}                         % Mathcal T
\newcommand{\U}{\mathcal{U}}                         % Mathcal U\\
\newcommand{\V}{\mathcal{V}}                         % Mathcal V

% symbol shortcuts
%====================================================================%

\newcommand{\bd}{\partial}
\newcommand{\grad}{\nabla}
\newcommand{\lam}{\lambda}
\newcommand{\imp}{\implies}
\newcommand{\all}{\forall}
\newcommand{\exs}{\exists}
\newcommand{\delt}{\delta}
\newcommand{\ep}{\varepsilon}
\newcommand{\ra}{\rightarrow}
\newcommand{\vph}{\varphi}

\newcommand{\ol}{\overline}
\newcommand{\f}{\frac}
\newcommand{\lf}{\lfrac}
\newcommand{\df}{\dfrac}

% bracketting shortcuts
%====================================================================%
\newcommand{\abs}[1]{\left| #1 \right|}
\newcommand{\babs}[1]{\Big|#1\Big|}
\newcommand{\bound}{\Big|}
\newcommand{\BB}[1]{\left(#1\right)}
\newcommand{\dd}{\mathrm{d}}
\newcommand{\artanh}{\mathrm{artanh}}
\newcommand{\Med}{\mathrm{Med}}
\newcommand{\Cov}{\mathrm{Cov}}
\newcommand{\Corr}{\mathrm{Corr}}
\newcommand{\tr}{\mathrm{tr}}
\newcommand{\Range}[1]{\mathrm{range}(#1)}
\newcommand{\Null}[1]{\mathrm{null}(#1)}
\newcommand{\lan}{\left\langle}
\newcommand{\ran}{\right\rangle}
\newcommand{\norm}[1]{\left\lVert#1\right\rVert}
\newcommand{\inn}[1]{\lan#1\ran}
\newcommand{\op}[1]{\operatorname{#1}}
\newcommand{\bmat}[1]{\begin{bmatrix}#1\end{bmatrix}}
\newcommand{\pmat}[1]{\begin{pmatrix}#1\end{pmatrix}}
\newcommand{\vmat}[1]{\begin{vmatrix}#1\end{vmatrix}}

\newcommand{\amogus}{{\bigcap}\kern-0.8em\raisebox{0.3ex}{$\subset$}}
\newcommand{\Note}{\textbf{Note: }}
\newcommand{\Aside}{{\bf Aside: }}
%restriction
%\newcommand{\op}[1]{\operatorname{#1}}
%\newcommand{\done}{$$\mathcal{QED}$$}

%====================================================================%


\setlength{\parindent}{0pt}      	% No paragraph indentations
\pagestyle{fancy}
\fancyhf{}							% fancy header

\setcounter{secnumdepth}{0}			% sections are numbered but numbers do not appear
\setcounter{tocdepth}{2} 			% no subsubsections in toc

%template
%====================================================================%
%\begin{manualproblem}{1}
%Spivak.
%\end{manualproblem}

%\begin{proof}[Solution]
%\end{proof}

%----------- or -----------%

%\begin{problem} 		
%\end{problem}	

%\penum
%	\item
%\epenum
%====================================================================%


\newcommand{\Course}{MAT351}
\newcommand{\hwNumber}{15}

%preamble

\title{}
\author{A.N.}
\date{\today}
\lhead{\Course A\hwNumber}
\rhead{\thepage}
%\cfoot{\thepage}


%====================================================================%
\begin{document}



\begin{problem}
% problem number 1
\end{problem}
\begin{enumerate}[label = \textbf{\roman*)}, leftmargin = 0pt]
	\item $f = \chi_{[-1,1]}$.
We compute the fourier transform of $f$ as:
\begin{align*}
	\hat{f}(\xi) & = \int_{\R} e^{-2\pi i x \cdot \xi} \chi_{[-1,1]} dx
	\\ & = \int_{[-1,1]} e^{-2\pi i x \cdot \xi} dx
	\\ & = \int_{[-1,1]} \cos \left( 2\pi \xi x \right) - i \sin \left( 2\pi \xi x \right) dx
	\\ & = \int_{[-1,1]} \cos \left( 2\pi \xi x \right) dx \tag{ integral of $\sin$ on symmetric domain is 0}
	\\ & = 2\int_{[0,1]} \cos \left( 2\pi \xi x \right) dx
	\\ & =  \frac{ 2 }{ 2\pi \xi }\sin \left( 2\pi \xi x \right)\Big|_{0}^1
	\\ & = \frac{ \sin \left( 2\pi \xi \right) }{ \pi \xi }
\end{align*}
\item $g = \left( 1 -|x| \right) \chi_{[-1,1]}$. 
We first make the following preliminary computation: 
		$$ \frac{ d }{ dx } \left( \frac{ x }{ -2\pi i \xi } e^{ -2\pi i x \cdot \xi} - \frac{ 1 }{ \left( 2\pi i \xi \right)^2 } e^{-2\pi i x \cdot \xi}\right)  = x e^{-2\pi i x \cdot \xi}.$$ 
	The fourier transform is :
	\begin{align*}
		\hat{g} (\xi ) & = \int_{\R} e^{-2\pi i x \cdot \xi } g(x) dx
		\\ & = \int_{[-1,1]} e^{-2\pi i x \cdot \xi} \left( 1-|x| \right) dx
		\\ & = \frac{ \sin \left( 2\pi \xi \right) }{ \pi \xi } - \int_{[-1,1]} e^{-2\pi i x \cdot \xi} |x| dx
		\\ & = \frac{ \sin 2\pi \xi }{ \pi \xi }- \left( \int_{-1}^0 -x e^{-2 \pi i x \cdot \xi} dx + \int_0^1 xe^{-2\pi i x \cdot \xi} dx\right) 
		\\ & = \frac{ \sin 2\pi \xi }{ \pi \xi } +  \int_{-1}^0 \frac{ d }{ dx } \left( \frac{ x }{ -2\pi i \xi } e^{ -2\pi i x \cdot \xi} - \frac{ 1 }{ \left( 2\pi i \xi \right)^2 } e^{-2\pi i x \cdot \xi}\right) dx
		\\ &- \int_0^1  \frac{ d }{ dx } \left( \frac{ x }{ -2\pi i \xi } e^{ -2\pi i x \cdot \xi} - \frac{ 1 }{ \left( 2\pi i \xi \right)^2 } e^{-2\pi i x \cdot \xi}\right)    
		\\ & = \frac{ \sin 2\pi \xi }{ \pi \xi } + \frac{ 2i  }{ -2\pi i \xi } Im \left( e^{-2\pi ix \cdot \xi} \right) + \frac{ (2i)^2}{ \left( 2\pi i \xi \right)^2 } Im \left( e^{\pi i x \cdot \xi} \right)^2
		\\ & = \frac{ \sin^2 \left( \pi \xi \right) }{ \left( \pi \xi \right)^2 }
	\end{align*}
\end{enumerate}
\newpage
\begin{problem}
% problem number 2
\end{problem}
It is know that $f\in L^1 \left( \R \right)$. We claim that $\hat{f}\in L^1$. First observe: 
$$ \int_{|\xi| \leq 1} |\hat{f}| d\xi = \int_{|\xi|\leq 1} \left| \int_\R e^{-2\pi i x \cdot \xi} f(x) dx\right| d \xi  \leq \int_{|\xi| \leq 1} \int_\R \left| f(x) \right|dx d\xi= \int_{|\xi|\leq 1} \norm{f}_{L^1}\leq 2 \norm{f_{L^1}}.$$ 
Now observe:
$$ \int_{\R} | \hat{f}(\xi) |d\xi = \int_{|\xi| \geq 1} | \hat{f} (\xi) |d \xi + \int_{|\xi| \leq 1} |\hat{f}(\xi)| d\xi \leq \int_{|\xi| \geq 1} C|\xi|^{-1-\alpha} d\xi + 2 \norm{f}_{L^1} <\infty.$$ 
Where we use the fact that $|x|^r$ is integrable if and only if $r<-1$. Thus $\hat{f}\in L^1$ and so we have access to the Fourier Inversion Theorem. we have that
\begin{align*}
	\left| f(x+h) - f(x)  \right| & = \left| \int_\R e^{2\pi i (x+h) \xi}\hat{f}(\xi) - e^{2\pi i x \xi} \hat{f}(\xi) d\xi \right|
	\\ & = \left| \int_\R e^{2\pi i x \cdot \xi} \hat{f}(\xi) \left( e^{2\pi i h \xi} - 1 \right)d \xi \right|
	\\ & \leq \left| \int_{|\xi| \geq 1 /h} e^{2\pi i x \cdot \xi} \hat{f}(\xi) \left( e^{2\pi i h \xi} - 1 \right)d \xi  \right| + \left| \int_{|\xi| \leq 1/h} e^{2\pi i x \cdot \xi} \hat{f}(\xi) \left( e^{2\pi i h \xi} - 1 \right)d \xi \right|
	\\ &  \leq 2 \int_{|\xi| \geq 1/h}|\hat{f} (\xi)| d\xi + 2\int_{|\xi| < 1/h} |\hat{f} (\xi)|d\xi 
	\\ & \leq M \int_{|\xi | > 1/h} |\xi|^{-1 - \alpha} d\xi + 2 \int_{|\xi^\prime| \geq 1/h} |\hat{f} (\xi^\prime) |\left(\frac{ 1 }{\xi^\prime}\right)^2 d\xi^\prime \tag{ apply bound on first term, set $\xi = \frac{ 1 }{ \xi^\prime }$ in second }
	\\ &  \leq M|h|^{\alpha} + 2 \int_{|\xi^\prime| \geq 1/h} |\xi^\prime |^{-3-\alpha}d \xi
	\\ & \leq M|h|^\alpha + K |h|^\alpha \tag{integrating second term}
\end{align*}
\newpage
\begin{problem}
% problem number 3
\end{problem}
Let $f, g \in \mathcal{S} (\R)$. We claim that $f \ast g \in \mathcal{S}(\R)$. First note that since $g \in \mathcal{S}$, we have that $|x|^2 |g(x)| < C$ and $|g(x)| <D$ so $(1 +|x|^2)|g(x)| < A$.
From this it follows that $|g(x)| < \frac{ A }{ 1+|x|^2 }$ and so $g\in L^1$. Therefore:
\begin{align*}
	\sup_{x} |x|^\alpha D^\beta |f\ast g(x)| & \leq \sup_x |x|^\alpha D^\beta \int_\R| f(x-y) g(y)| dy
	\\ & = \sup_x \int_\R |x|^\alpha D^\beta| f(x-y) g(y)| dy
	\\ & \leq \int_\R \sup_x |x|^\alpha D^\beta |f(x-y) g(y)| dy \tag{interchange derivative and integral}
	\\ & \leq \int_\R C_{f,\alpha,\beta} |g(y)| dy \tag{since $f \in \mathcal{S}$}
	\\ & <\infty \tag{since $g\in L^1$}
\end{align*}
\newpage
\begin{problem}
% problem number 4
\end{problem}
The following chain of inequalities is true:
\begin{align*}
	\norm{u}_{L^\infty} &\leq \norm{\hat{u}}_{L^1} \tag{shown in class}
	\\ & \leq C \left( \int \left( 1+| \xi|^4\right)|\hat{u}(\xi)|^2 d\xi \right)^{1/2} \tag{assumption}
	\\ & \leq C \left( \norm{\hat{u}}_{L^2} + \norm{|\xi|^2 \hat{u}}_{L^2} \right) \tag{triangle inequality for $L^2$ norm}
	\\ &\leq C \left( \norm{u}_{L^2} + K\norm{\widehat{\Delta u}} \right) \tag{Plancharel + Fourier transforming polynomials into derivatives}
	\\ & \leq C \left( \norm{u}_{L^2} + \norm{\Delta u}_{L^2} \right) \tag{Plancharel, enlarging C if necessary}
\end{align*}
\newpage
\begin{problem}
% problem number 5
\end{problem}
\penum
\item Given the PDE
$$ \begin{cases}
	iu_t + \Delta u = 0
	\\ u(t=0)  = g
\end{cases} .$$ 
We Fourier transform in $x$ to get:
$$ \begin{cases}
	\hat{u}_t  = i |\xi|^2 \hat{u}
	\\ \hat{u}(t = 0)  = \hat{g}
\end{cases} $$ 
This is a linear ODE in $t$, which we know is solved by 
$$ \hat{u}(t, \xi) = e^{i |\xi|^2 t}\hat{g}(\xi) $$ 
\item Since $\widehat{e^{i\pi |x|^2}} = e^{i \pi |\xi|^2}$, by dilation we have:
	$$ \left( e^{i|x|^2 t} \right)^{\wedge} = \left( e^{i\pi \left| \frac{ \sqrt{t} }{ \sqrt{\pi} }x \right|^2 } \right)^\wedge = \left( \frac{ \pi }{ t } \right)^{n/2}e^{i t|\xi|^2 }.$$ 
Since the Fourier transform, and hence the inverse takes products to convolutions, we have:
$$ u(x,t) = \left( \hat{u}(\xi,t) \right)^\vee  = \left( e^{it|\xi|^2} \hat{g}(\xi )\right)^\vee = \left( \frac{ \pi }{ t } \right)^{n/2} e^{it|x|^2} \ast g(x).$$ 
\item 
\item 
\epenum
\newpage


\end{document}
