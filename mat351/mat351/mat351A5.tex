\documentclass[12pt, a4paper]{article}
\usepackage[lmargin =0.5 in, 
rmargin=0.5in, 
tmargin=1in,
bmargin=0.5in]{geometry}
\geometry{letterpaper}
\usepackage{tikz-cd}
\usepackage{amsmath}
\usepackage{amssymb}
\usepackage{blindtext}
\usepackage{titlesec}
\usepackage{enumitem}
\usepackage{fancyhdr}
\usepackage{amsthm}
\usepackage{graphicx}
\usepackage{cool}
\usepackage{thmtools}
\usepackage{hyperref}
\graphicspath{ }					%path to an image

%-------- sexy font ------------%
%\usepackage{libertine}
%\usepackage{libertinust1math}

%\usepackage{mlmodern}				% very nice and classic
%\usepackage[utopia]{mathdesign}
%\usepackage[T1]{fontenc}


\usepackage{mlmodern}
\usepackage{eulervm}
%\usepackage{tgtermes} 				%times new roman
%-------- sexy font ------------%


% Problem Styles
%====================================================================%


\newtheorem{problem}{Problem}


\theoremstyle{definition}
\newtheorem{thm}{Theorem}
\newtheorem{lemma}{Lemma}
\newtheorem{prop}{Proposition}
\newtheorem{cor}{Corollary}
\newtheorem{fact}{Fact}
\newtheorem{defn}{Definition}
\newtheorem{example}{Example}
\newtheorem{question}{Question}

\newtheorem{manualprobleminner}{Problem}

\newenvironment{manualproblem}[1]{%
	\renewcommand\themanualprobleminner{#1}%
	\manualprobleminner
}{\endmanualprobleminner}

\newcommand{\penum}{ \begin{enumerate}[label=\bf(\alph*), leftmargin=0pt]}
	\newcommand{\epenum}{ \end{enumerate} }

% Math fonts shortcuts
%====================================================================%

\newcommand{\ring}{\mathcal{R}}
\newcommand{\N}{\mathbb{N}}                           % Natural numbers
\newcommand{\Z}{\mathbb{Z}}                           % Integers
\newcommand{\R}{\mathbb{R}}                           % Real numbers
\newcommand{\C}{\mathbb{C}}                           % Complex numbers
\newcommand{\F}{\mathbb{F}}                           % Arbitrary field
\newcommand{\Q}{\mathbb{Q}}                           % Arbitrary field
\newcommand{\PP}{\mathcal{P}}                         % Partition
\newcommand{\M}{\mathcal{M}}                         % Mathcal M
\newcommand{\eL}{\mathcal{L}}                         % Mathcal L
\newcommand{\T}{\mathbb{T}}                         % Mathcal T
\newcommand{\U}{\mathcal{U}}                         % Mathcal U\\
\newcommand{\V}{\mathcal{V}}                         % Mathcal V

% symbol shortcuts
%====================================================================%

\newcommand{\bd}{\partial}
\newcommand{\grad}{\nabla}
\newcommand{\lam}{\lambda}
\newcommand{\imp}{\implies}
\newcommand{\all}{\forall}
\newcommand{\exs}{\exists}
\newcommand{\delt}{\delta}
\newcommand{\ep}{\varepsilon}
\newcommand{\ra}{\rightarrow}
\newcommand{\vph}{\varphi}

\newcommand{\ol}{\overline}
\newcommand{\f}{\frac}
\newcommand{\lf}{\lfrac}
\newcommand{\df}{\dfrac}

% bracketting shortcuts
%====================================================================%
\newcommand{\abs}[1]{\left| #1 \right|}
\newcommand{\babs}[1]{\Big|#1\Big|}
\newcommand{\bound}{\Big|}
\newcommand{\BB}[1]{\left(#1\right)}
\newcommand{\dd}{\mathrm{d}}
\newcommand{\artanh}{\mathrm{artanh}}
\newcommand{\Med}{\mathrm{Med}}
\newcommand{\Cov}{\mathrm{Cov}}
\newcommand{\Corr}{\mathrm{Corr}}
\newcommand{\tr}{\mathrm{tr}}
\newcommand{\Range}[1]{\mathrm{range}(#1)}
\newcommand{\Null}[1]{\mathrm{null}(#1)}
\newcommand{\lan}{\langle}
\newcommand{\ran}{\rangle}
\newcommand{\norm}[1]{\left\lVert#1\right\rVert}
\newcommand{\inn}[1]{\lan#1\ran}
\newcommand{\op}[1]{\operatorname{#1}}
\newcommand{\bmat}[1]{\begin{bmatrix}#1\end{bmatrix}}
\newcommand{\pmat}[1]{\begin{pmatrix}#1\end{pmatrix}}
\newcommand{\vmat}[1]{\begin{vmatrix}#1\end{vmatrix}}

\newcommand{\amogus}{{\bigcap}\kern-0.8em\raisebox{0.3ex}{$\subset$}}
\newcommand{\Note}{\textbf{Note: }}
\newcommand{\Aside}{{\bf Aside: }}
%restriction
%\newcommand{\op}[1]{\operatorname{#1}}
%\newcommand{\done}{$$\mathcal{QED}$$}

%====================================================================%


\setlength{\parindent}{0pt}      	% No paragraph indentations
\pagestyle{fancy}
\fancyhf{}							% fancy header

\setcounter{secnumdepth}{0}			% sections are numbered but numbers do not appear
\setcounter{tocdepth}{2} 			% no subsubsections in toc

%template
%====================================================================%
%\begin{manualproblem}{1}
%Spivak.
%\end{manualproblem}

%\begin{proof}[Solution]
%\end{proof}

%----------- or -----------%

%\begin{problem} 		
%\end{problem}	

%\penum
%	\item
%\epenum
%====================================================================%


\newcommand{\Course}{351}
\newcommand{\hwNumber}{5}

%preamble

\title{}
\author{A.N.}
\date{\today}
\lhead{\Course A\hwNumber}
\rhead{\thepage}
%\cfoot{\thepage}


%====================================================================%
\begin{document}
\begin{problem}
\end{problem}
\penum
\item Consider the heat equation on the real line, with initial datum $\phi(x)$ even. The solution $u(x,t)$ is given by: 
$$u(x,t ) = \frac{1}{\sqrt{4\pi kt }} \int_{-\infty}^\infty e^{-\frac{|x-y|^2}{4kt }}\phi(y) dy.$$
We compute: 
\begin{align*}
	u(-x,t) &=\frac{1}{\sqrt{4\pi kt }} \int_{-\infty}^\infty e^{-\frac{|-x-y|^2}{4kt }}\phi(y) dy
	\\ & = \frac{1}{\sqrt{4\pi kt }} \int_{-\infty}^\infty e^{-\frac{|x+y|^2}{4kt }}\phi(y) dy
	\\ & = -\frac{1}{\sqrt{4\pi kt }} \int_{\infty}^{-\infty} e^{-\frac{|x-y^\prime|^2}{4kt }}\phi(-y^\prime) dy^\prime \tag*{sub $y=-y^\prime$}
	\\ & = \frac{1}{\sqrt{4\pi kt }} \int_{-\infty}^{\infty} e^{-\frac{|x-y^\prime|^2}{4kt }}\phi(-y^\prime) dy^\prime \tag*{(*) flipping bounds}
	\\ & = \frac{1}{\sqrt{4\pi kt }} \int_{-\infty}^{\infty} e^{-\frac{|x-y^\prime|^2}{4kt }}\phi(y^\prime) dy^\prime \tag*{$\phi$ even}
	\\ & = u(x,t)
\end{align*}
Therefore $u$ is even. If our initial datum $\phi$ is odd, then the calculation above is identical until $(*)$, at which point we have: 
$$u(-x,t) =\frac{1}{\sqrt{4\pi kt }} \int_{-\infty}^{\infty} e^{-\frac{|x-y^\prime|^2}{4kt }}\phi(-y^\prime) dy^\prime =- \frac{1}{\sqrt{4\pi kt }} \int_{-\infty}^{\infty} e^{-\frac{|x-y^\prime|^2}{4kt }}\phi(y^\prime) dy^\prime = -u(x,t).$$
Thus the solutions to the heat equation with even or odd initial datum is also even or odd.
\item We now prove the analogous result for the wave equation. Let $\phi(x), \psi(x)$, be even initial conditions. Let $u(x,t)$ be the unique solution. By D'Alemberts formula, we have: 
\begin{align*}
	u(-x,t) & = \frac{1}{2} \left[\phi(-x+ct) + \phi(-x-ct) \right] + \frac{1}{2c} \int_{-x-ct}^{-x+ct}\psi(y) dy
	\\ & = \frac{1}{2} \left[\phi(x-ct)  + \phi(x+ct)\right] - \frac{1}{2c} \int_{x+ct}^{x-ct} \psi(-y^\prime) dy^\prime \tag{$\phi$ even, substitute $y= -y^\prime$}
	\\ & = \frac{1}{2} \left[\phi(x-ct)  + \phi(x+ct)\right] + \frac{1}{2c}\int_{x-ct}^{x+ct}  \psi(y^\prime) dy^\prime \tag{flip bounds, $\psi$ even.}
	\\ & = u(x,t)
\end{align*}
Similarly with initial data $\phi, \psi$ odd, we get instead: 
$$u(-x,t) = -\frac{1}{2} \left[\phi(x+ct) + \phi(x-ct) \right]  \frac{1}{2c}\int_{x-ct}^{x+ct}\phi(-y^\prime) dy^\prime = -u(x,t).$$
\epenum
 \newpage 
\begin{problem}
\end{problem}
\penum
\item We compute the time derivative of the energy functional: 
\begin{align*}
	\dot{E}(t) & =2 \int_{U} v\cdot v_t dx
	\\ & =2 \int_{u} v \cdot ( -\grad p - (v\cdot \grad) v)dx
	\\ & = -2 \int_{U} v \cdot \grad p dx - 2 \int_{u} v \cdot (v \cdot \grad)v dx
	\\ & = -2 \int_U \grad \cdot (p v) dx - 2\int_U \grad \cdot ( X v) dx \tag{vector calc identities, for some  matrix X}
	\\ & = -2 \int_{\bd U} pv\cdot \nu da - 2\int_{\bd U}Xv \cdot \nu  da
	\\ & = 0 \tag{since v=0 on $\partial U$ and is perpendicular along it. }
\end{align*}
Such a matrix $X$ must exist since we can write a system of $ODE's$ where $\grad \cdot Xv = v \cdot (v \cdot \grad)v  $ has a unique solution. 
Therefore energy is conserved. 
\item For the NS equation, we compute the time derivative of the energy functional as: 
\begin{align*}
	\dot{E}(t)  &  =2 \int_{U} v \cdot ( -\grad p - (v\cdot \grad) v + \mu \Delta v)dx
	\\& = 2 \mu \int_{U} v \cdot \Delta v dx \tag{by a)}
	\\ & = 2 \mu \int_U v \cdot (\grad \times (\grad \times v))dx \tag{vector laplace definition and $\grad \cdot v=0$.}
	\\ & = -2\mu \int_U \grad \cdot(( \grad \times \grad \times v) \times v) dx -2\mu  \int_U (\grad \times v)\cdot (\grad \times v)dx 
	\tag{vector calv identity}
	\\ & = -2\mu\int_{\bd U} ( \grad \times \grad \times v) -2\mu \int_U \norm{\grad \times v}^2dx \tag{divergence theorem}
	\\ & = -2\mu \int_U \norm{\grad \times v}^2dx \tag{since $v$ vanishes on boundary}
\end{align*}
Thus the energy decreases for all time $t$. We can write 
$$E(t) = E(0) -2\mu \int_0^t\int_U \norm{\grad \times v}^2dx dt $$
\epenum
 \newpage 
\begin{problem}
\end{problem}
\penum 
\item Apply the curl operator to both sides of the PDE, use vector calculus identities to get that:
$$\grad\times(v_t - \mu \Delta v + \grad p) = \grad \times \grad f \implies \grad \times v_t - \mu \grad \times \Delta v \implies \omega_t = \mu \Delta \omega.$$
Thus $\omega$'s compononents satisfy the heat equation. 
\item We compute that $$\omega_i(x,t) = \frac{1}{\sqrt{4\pi t \mu}} \int_{-\infty}^\infty e^{- \frac{(x-y)^2}{4 t  \mu}} (\grad \times v_0(y))_i dy $$
\epenum
 \newpage 
\begin{problem}
\end{problem}
Define the function $u = v-h$ so that $u$ satisfies: 
$$\begin{cases}
	u_t - u_{xx} = (f-h_t)\\
	u(t,0) = 0\\
	u(x,0) = \phi(x)
\end{cases}$$ on the half line. Let $\Phi(x)$ be the odd extension of $\phi(x)$. 
We now wish to solve the PDE:
$$\begin{cases}
	u_{t}- u_{xx} = (f-h_t)\\
	u(x,0) = \Phi(x)
\end{cases}$$
We know that this will be solved by $$u(x,t) = \int_{-\infty}^\infty S(x-y,t)\Phi(y)dy + \int_0^t \int_{-\infty}^\infty S(x-y,t-s) (f-h-t)(y,s) dy ds,$$
Where $S(x,t) = \frac{1}{\sqrt{4\pi t}} e^{- \frac{(x-y)^2}{4t}}$. We now set $v = u+h(t)$, and restrict to the half line to get the desired solution. 
 \newpage 
\begin{problem}
\end{problem}
\penum
\item Define the energy functional $E(t) = \frac{1}{2}\int_{\R}|u_t|^2 dx + K(t)$. We wish to determine $K(t)$ so that $\dot{E}(t)= 0$. We see that: 
\begin{align*}
	-\dot{K}(t) &= \int u_t  u_{tt}
	\\ & = \int u_t (c^2 u_{xx} - m^2 u)dx
	\\ & = c^2 \int u_t u_{xx} dx - m^2 \int u_t  u dx
	\\ & = c^2 u_x u_t \Big|_{-\infty}^\infty - c^2 \int u_{tx}u_x dx - m^2 \int u_t  u dx\tag{integrating by parts}
	\\ & = -c^2 \int u_{tx}u_x dx - m^2 \int u_t u
\end{align*}
Therefore $K(t) = \frac{c^2}{2} \int u_x^2 + \frac{m^2}{2} \int u^2 dx$. By construction this choice of $E(t)$ will be conserved for solutions of the klein gordon equation. 
\item Suppose that $u_1, u_2$ are two solutions with identical initial conditions, $\phi(x), \psi(x)$. Using conservation of energy on $u_1- u_2$, we see: 
$$E(0) = \frac{1}{2}\int |u_1(x,0)- u_2(x,0)|^2 dx + \frac{c^2}{2}\int |[u_1(x,0)-u_2(x,0)]_t |^2dx + \frac{m^2}{2} \int [u_1(x,0) - u_2(x,0)]^2dx = 0 = E(t).$$
Since $E(t)=0$ for all $t$, we have that $u_1 = u_2$. 
\epenum


\end{document}
