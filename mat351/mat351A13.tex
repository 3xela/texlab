\documentclass[12pt, a4paper]{article}
\usepackage[lmargin =0.5 in, 
rmargin=0.5in, 
tmargin=1in,
bmargin=0.5in]{geometry}
\geometry{letterpaper}
\usepackage{tikz-cd}
\usepackage{amsmath}
\usepackage{amssymb}
\usepackage{blindtext}
\usepackage{titlesec}
\usepackage{enumitem}
\usepackage{fancyhdr}
\usepackage{amsthm}
\usepackage{graphicx}
\usepackage{cool}
\usepackage{thmtools}
\usepackage{hyperref}
\graphicspath{ }					%path to an image

%-------- sexy font ------------%
%\usepackage{libertine}
%\usepackage{libertinust1math}

%\usepackage{mlmodern}				% very nice and classic
%\usepackage[utopia]{mathdesign}
%\usepackage[T1]{fontenc}


\usepackage{mlmodern}
\usepackage{eulervm}
%\usepackage{tgtermes} 				%times new roman
%-------- sexy font ------------%


% Problem Styles
%====================================================================%


\newtheorem{problem}{Problem}


\theoremstyle{definition}
\newtheorem{thm}{Theorem}
\newtheorem{lemma}{Lemma}
\newtheorem{prop}{Proposition}
\newtheorem{cor}{Corollary}
\newtheorem{fact}{Fact}
\newtheorem{defn}{Definition}
\newtheorem{example}{Example}
\newtheorem{question}{Question}

\newtheorem{manualprobleminner}{Problem}

\newenvironment{manualproblem}[1]{%
	\renewcommand\themanualprobleminner{#1}%
	\manualprobleminner
}{\endmanualprobleminner}

\newcommand{\penum}{ \begin{enumerate}[label=\bf(\alph*), leftmargin=0pt]}
	\newcommand{\epenum}{ \end{enumerate} }

% Math fonts shortcuts
%====================================================================%

\newcommand{\ring}{\mathcal{R}}
\newcommand{\N}{\mathbb{N}}                           % Natural numbers
\newcommand{\Z}{\mathbb{Z}}                           % Integers
\newcommand{\R}{\mathbb{R}}                           % Real numbers
\newcommand{\C}{\mathbb{C}}                           % Complex numbers
\newcommand{\F}{\mathbb{F}}                           % Arbitrary field
\newcommand{\Q}{\mathbb{Q}}                           % Arbitrary field
\newcommand{\PP}{\mathcal{P}}                         % Partition
\newcommand{\M}{\mathcal{M}}                         % Mathcal M
\newcommand{\eL}{\mathcal{L}}                         % Mathcal L
\newcommand{\T}{\mathbb{T}}                         % Mathcal T
\newcommand{\U}{\mathcal{U}}                         % Mathcal U\\
\newcommand{\V}{\mathcal{V}}                         % Mathcal V

% symbol shortcuts
%====================================================================%

\newcommand{\bd}{\partial}
\newcommand{\grad}{\nabla}
\newcommand{\lam}{\lambda}
\newcommand{\imp}{\implies}
\newcommand{\all}{\forall}
\newcommand{\exs}{\exists}
\newcommand{\delt}{\delta}
\newcommand{\ep}{\varepsilon}
\newcommand{\ra}{\rightarrow}
\newcommand{\vph}{\varphi}

\newcommand{\ol}{\overline}
\newcommand{\f}{\frac}
\newcommand{\lf}{\lfrac}
\newcommand{\df}{\dfrac}

% bracketting shortcuts
%====================================================================%
\newcommand{\abs}[1]{\left| #1 \right|}
\newcommand{\babs}[1]{\Big|#1\Big|}
\newcommand{\bound}{\Big|}
\newcommand{\BB}[1]{\left(#1\right)}
\newcommand{\dd}{\mathrm{d}}
\newcommand{\artanh}{\mathrm{artanh}}
\newcommand{\Med}{\mathrm{Med}}
\newcommand{\Cov}{\mathrm{Cov}}
\newcommand{\Corr}{\mathrm{Corr}}
\newcommand{\tr}{\mathrm{tr}}
\newcommand{\Range}[1]{\mathrm{range}(#1)}
\newcommand{\Null}[1]{\mathrm{null}(#1)}
\newcommand{\lan}{\left\langle}
\newcommand{\ran}{\right\rangle}
\newcommand{\norm}[1]{\left\lVert#1\right\rVert}
\newcommand{\inn}[1]{\lan#1\ran}
\newcommand{\op}[1]{\operatorname{#1}}
\newcommand{\bmat}[1]{\begin{bmatrix}#1\end{bmatrix}}
\newcommand{\pmat}[1]{\begin{pmatrix}#1\end{pmatrix}}
\newcommand{\vmat}[1]{\begin{vmatrix}#1\end{vmatrix}}

\newcommand{\amogus}{{\bigcap}\kern-0.8em\raisebox{0.3ex}{$\subset$}}
\newcommand{\Note}{\textbf{Note: }}
\newcommand{\Aside}{{\bf Aside: }}
%restriction
%\newcommand{\op}[1]{\operatorname{#1}}
%\newcommand{\done}{$$\mathcal{QED}$$}

%====================================================================%


\setlength{\parindent}{0pt}      	% No paragraph indentations
\pagestyle{fancy}
\fancyhf{}							% fancy header

\setcounter{secnumdepth}{0}			% sections are numbered but numbers do not appear
\setcounter{tocdepth}{2} 			% no subsubsections in toc

%template
%====================================================================%
%\begin{manualproblem}{1}
%Spivak.
%\end{manualproblem}

%\begin{proof}[Solution]
%\end{proof}

%----------- or -----------%

%\begin{problem} 		
%\end{problem}	

%\penum
%	\item
%\epenum
%====================================================================%


\newcommand{\Course}{MAT351}
\newcommand{\hwNumber}{13}

%preamble

\title{}
\author{A.N.}
\date{\today}
\lhead{\Course A\hwNumber}
\rhead{\thepage}
%\cfoot{\thepage}


%====================================================================%
\begin{document}



\begin{problem}
% problem number 1
\end{problem}
\penum 
\item Solutions to the heat equation in $3-$d space are given as:
$$ v(x,t)  = \frac{ 1 }{( 4\pi t)^{n/2} }\int_{\R^n} e^{- \frac{ |x-y|^2 }{ 4t } } v_0(y) dt .$$
Taking absolutely value and supremums of both sides, we see that: 
$$ \sup_{x\in \R^n} |v(x,t)| \leq \sup_{x\in \R^n}\left| \frac{ 1 }{ (4\pi t)^{n/2} } \right|\int_{\R^n} e^{- \frac{ |x-y|^2 }{ 4t } } \left| v_0(y) \right|dy \leq   \sup_{x\in \R^n}\left| \frac{ 1 }{ (4\pi t)^{n/2} } \right|\int_{\R^n} e^{- \frac{ |x-y|^2 }{ 4t } }dy \cdot \int_{\R^n} |v_0(y)|dy \leq B. $$ 
Similarly, we have: 
$$ \sup_{x\in \R^n}|v(x,t)|  = \frac{ 1 }{ t^{n/2} }\sup_{x\in \R^n} \frac{ 1 }{ (4\pi)^{n/2}} \left| \int_{\R^n} e^{- \frac{ |x-y|^2 }{ 4t }} v_0(y) dy \right|\leq \frac{ 1 }{ t^{n/2} } \sup_{x\in \R^n} |v_0(x)| \frac{ 1 }{(4\pi)^{n/2} }\int_{\R^n} e^{-y^2}{4t} dy\leq \frac{ B }{ t^{n/2} }.$$
Notice that for $t<1$, the first bound is stronger, and that for $t\geq 1$ the second bound is stronger. Consider $C_n = 2^n$. Then we have for $t<1$:
$$\sup_{x\in \R^n} |v(x,t)| \leq B \leq 2^n \frac{ B}{t^{n/2}},$$
and for $t \geq 1$, 
$$ \sup_{x \in \R^n} |v(x,t)| \leq \frac{ B }{ t^{n/2} }\leq 2^n \frac{ B }{ (1+t)^{n/2} }. $$ 
\item Recall that the solution to the inhomogenous heat equation will be given by: 
	$$ u(x,t) = \frac{ 1 }{ (4\pi t)^{n/2}  } \int_{\R^n} e^{- \frac{ |x-y|^2 }{ 4t }} u_0(y) dy + \int_0^t \frac{ 1 }{ (4\pi(t-s))^{n/2} }  \int_{\R^n} e^{- \frac{ |x-y| }{ 4(t-s) }} F(y,s) dy dx.$$
Therefore the $L^2$ norm (across x) of $u(x,t)$ is controlled by: 
\begin{align*}
	\norm{u(\cdot,t)}_{L^2} & \leq \norm{ \frac{ 1 }{ (4\pi t)^{n/2} } \int_{\R^n} e^{ - \frac{ |x-y|^2  }{ 4t }} u_0(y) dy}_{L^2} + \norm{ \int_0^t \frac{ 1 }{ (4\pi(t-s))^{n/2} }  \int_{\R^n} e^{- \frac{ |x-y| }{ 4(t-s) }} F(y,s) dy dx.}
	\\ & \leq \norm{u_0}_{L^2}  + \int_0^t \norm{ \frac{ 1 }{ 4\pi(t-s) } \int_{\R^n} e^{- \frac{ |x-y|^2 }{ 4(t-s) }} F(y,s) dy}_{L^2}ds  \tag{By Cauchy-Schwartz, exponential has norm 1 }
	\\ & \leq \norm{u_0}_{L^2} + \int_0^t \norm{F(\cdot, s)}ds \tag{Cauchy-Schwartz on second summand}
\end{align*}
\item For $F = v^3(x,t)$, we have that:
	$$ \norm{u(\cdot , t)}\leq \norm{u_0}_{L^2} + \int_{0}^t \norm{v^3(\cdot,s)}_{L^2}ds.$$
	Provided that the righthand side is finite we have that $u_0\in L^2$ so we simply need to bound $\int_0^t \norm{v^3(\cdot, s) }_{L^2} ds$ with a constant. 
We have: 
\begin{align*}
	\int_0^t \norm{v^3(\cdot, s) }_{L^2} ds &= \int_0^t \left( \int_{\R^n} \left| v(y,s) \right|^6dy \right)^{1/2}ds
	\\ & \leq \int_0^t B^3 C^3 \left(\frac{1}{ (1+s)^{3n} } \right)^{1/2}ds\tag{by bound on $v$}
\end{align*}
\epenum
\newpage
\begin{problem}
% problem number 2
\end{problem}
\penum
\item We first compute $\Delta w$, $w_{tt}$:
	$$ \Delta w = \Delta e^{Dt} u = e^{Dt} \Delta u. $$ 
	$$ w_{tt} = \left( De^{Dt} u + e^{Dt} u_t \right)_t = D^2e^{Dt} u +2 De^{Dt} u_t + e^{Dt} u_{tt}. $$ 
	To see what wave-liek equation $w$ satifies, we compute 
	$$w_{tt} - \Delta w = e^{Dt} \left( D^2u + 2Du_t + u_{tt} - \Delta u \right) = e^{Dt} \left( D^2u+ 2Du_t - A^2 u_{t} \right).$$
	We see that if we take $D = \frac{ A^2 }{ 2 }$, then $w$ satisfies
	$$ w_{tt} - \Delta w = D^2 w. $$ 
\item We verify that $v$ satisfies the wave equation for some choice of $B$:
	$$ \Delta v = e^{Bz} \Delta w + B^2 e^{Bz} w. $$
	$$ v_{tt} = e^{Bz} w_{tt}. $$ 
Thus
$$ v_{tt} - \Delta v = e^{Bz} \left( w_{tt} - \Delta w - B^2 w \right) $$ 
Taking $B = \pm D$ suffices.
\item Note that $v(0,x) = e^{Dz} w(x,y,0) = e^{Dz} u(x,y,0) = 0$ as well as $v_t(0,x) =D  h$
	By Kirchoffs formula, we have:
	$$v(x,t) = \frac{ 1 }{ 4\pi t }\int_{|x-y| = t} Dh dS_y.$$
But $v(x,t) = e^{Dt + Dz} u(x,y,t)$, so
$$u(x,y,t) = \frac{ e^{ \frac{ -A^2 z - A^2t }{ 2 }} }{ 4\pi t } \int_{|x-y|= t} \frac{ -A^2 }{ 2 }h dS_y.$$
\epenum
\newpage
\begin{problem}
% problem number 3
\end{problem}
\penum 
\item Suppose that $Q(x,t) = t^{-\alpha} v(t^{-\beta} x)$ satisfies the 3d heat equation for some $v$.  Then we must have:
	$$ \Delta Q = t^{- \alpha} \Delta v(t^{-\beta} x) \cdot t^{-2\beta} = -\alpha t^{-\alpha-1} v(t^{-\beta} x) + t^{-\alpha} \grad v(t^{-\beta} x ) \cdot -\beta t^{-\beta - 1} x = Q_t. $$ 
\item Choose $\beta = \frac{ 1 }{ 2 } $, and change to radial coordinates to get :
	$$  \Delta_r v(t^{-1/2} r) = -\alpha t^{-1} v(t^{-1/2} r) - \frac{ r }{ 2 } t^{- 1/2- 1} \frac{ \partial v }{ \partial r }(t^{-1/2} r). $$ 
We can further simplify as
$$ t^{-1} \frac{ \partial^2 v }{ \partial r^2 } + \frac{ 2t^{-1/2} }{ r } \frac{ \partial v }{ \partial v } =  -\alpha t^{-1} v(t^{-1/2} r) - \frac{ r }{ 2 } t^{- 1/2- 1} \frac{ \partial v }{ \partial r }(t^{-1/2} r). $$ 
\item 
\item Given $Q(x,t) = \frac{ 1 }{ (2\pi t)^{3/2} }e^{- \frac{ |x|^2 }{ 4t }}$, we wish to show that $$ \lim_{t\to 0} Q_t \ast \phi(x) = \phi.$$ 
We claim that 
$$ \lim_{t\to 0} \int_{\R^3} \left| Q_t\ast \phi(y) - \phi(x) \right| dy =0. $$ 
The idea is that as $t\to 0$ then $Q_t$ behaves like a delta function. We break up the integral into a large region where $Q$ is sufficiently small, and a small region where $Q$ is large, but $\phi(y) $ is close to $\phi(x)$. Then this integral will vanish as we take $t\to 0$. 
\epenum
\newpage
\begin{problem}
% problem number 4
\end{problem}
\penum 
\item On each component of $P$, by integrating by parts we have
	$$ P_i = -i \int \ol{u} \partial_i u dx= i \int \ol{\partial_i u}u dx = \ol{P_i}. $$ 
Therefore $P_i$ and so $P$ is real. 
We take the time derivative of $P_i$, and once again by intgration by parts we have: 
$$ \frac{ d }{ dt }P_i = -i\int \ol{u}_t \cdot \partial_i u dx - i\int \ol{u} \cdot  \partial_i u_t = -i \int \ol{u}_t \cdot \partial_i u dx + i \int \partial_i \ol{u} u_t dx = 2i Im \left( \int \ol{u_t} \cdot \partial_i u dx \right) .$$
Since $P$ is real valued, it can only have a complex derivative if the derivative is identically 0. 
\item 
\item 
\item 
\epenum
\newpage


\end{document}
