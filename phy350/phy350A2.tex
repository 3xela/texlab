\documentclass[12pt, a4paper]{article}
\usepackage[lmargin =0.5 in, 
rmargin=0.5in, 
tmargin=1in,
bmargin=0.5in]{geometry}
\geometry{letterpaper}
\usepackage{tikz-cd}
\usepackage{amsmath}
\usepackage{amssymb}
\usepackage{blindtext}
\usepackage{titlesec}
\usepackage{enumitem}
\usepackage{fancyhdr}
\usepackage{amsthm}
\usepackage{graphicx}
\usepackage{cool}
\usepackage{thmtools}
\usepackage{hyperref}
\graphicspath{ }					%path to an image

%-------- sexy font ------------%
%\usepackage{libertine}
%\usepackage{libertinust1math}

%\usepackage{mlmodern}				% very nice and classic
%\usepackage[utopia]{mathdesign}
%\usepackage[T1]{fontenc}


\usepackage{mlmodern}
\usepackage{eulervm}
%\usepackage{tgtermes} 				%times new roman
%-------- sexy font ------------%


% Problem Styles
%====================================================================%


\newtheorem{problem}{Problem}


\theoremstyle{definition}
\newtheorem{thm}{Theorem}
\newtheorem{lemma}{Lemma}
\newtheorem{prop}{Proposition}
\newtheorem{cor}{Corollary}
\newtheorem{fact}{Fact}
\newtheorem{defn}{Definition}
\newtheorem{example}{Example}
\newtheorem{question}{Question}

\newtheorem{manualprobleminner}{Problem}

\newenvironment{manualproblem}[1]{%
	\renewcommand\themanualprobleminner{#1}%
	\manualprobleminner
}{\endmanualprobleminner}

\newcommand{\penum}{ \begin{enumerate}[label=\bf(\alph*), leftmargin=0pt]}
	\newcommand{\epenum}{ \end{enumerate} }

% Math fonts shortcuts
%====================================================================%

\newcommand{\ring}{\mathcal{R}}
\newcommand{\N}{\mathbb{N}}                           % Natural numbers
\newcommand{\Z}{\mathbb{Z}}                           % Integers
\newcommand{\R}{\mathbb{R}}                           % Real numbers
\newcommand{\C}{\mathbb{C}}                           % Complex numbers
\newcommand{\F}{\mathbb{F}}                           % Arbitrary field
\newcommand{\Q}{\mathbb{Q}}                           % Arbitrary field
\newcommand{\PP}{\mathcal{P}}                         % Partition
\newcommand{\M}{\mathcal{M}}                         % Mathcal M
\newcommand{\eL}{\mathcal{L}}                         % Mathcal L
\newcommand{\T}{\mathbb{T}}                         % Mathcal T
\newcommand{\U}{\mathcal{U}}                         % Mathcal U\\
\newcommand{\V}{\mathcal{V}}                         % Mathcal V

% symbol shortcuts
%====================================================================%

\newcommand{\bd}{\partial}
\newcommand{\grad}{\nabla}
\newcommand{\lam}{\lambda}
\newcommand{\imp}{\implies}
\newcommand{\all}{\forall}
\newcommand{\exs}{\exists}
\newcommand{\delt}{\delta}
\newcommand{\ep}{\varepsilon}
\newcommand{\ra}{\rightarrow}
\newcommand{\vph}{\varphi}

\newcommand{\ol}{\overline}
\newcommand{\f}{\frac}
\newcommand{\lf}{\lfrac}
\newcommand{\df}{\dfrac}

% bracketting shortcuts
%====================================================================%
\newcommand{\abs}[1]{\left| #1 \right|}
\newcommand{\babs}[1]{\Big|#1\Big|}
\newcommand{\bound}{\Big|}
\newcommand{\BB}[1]{\left(#1\right)}
\newcommand{\dd}{\mathrm{d}}
\newcommand{\artanh}{\mathrm{artanh}}
\newcommand{\Med}{\mathrm{Med}}
\newcommand{\Cov}{\mathrm{Cov}}
\newcommand{\Corr}{\mathrm{Corr}}
\newcommand{\tr}{\mathrm{tr}}
\newcommand{\Range}[1]{\mathrm{range}(#1)}
\newcommand{\Null}[1]{\mathrm{null}(#1)}
\newcommand{\lan}{\langle}
\newcommand{\ran}{\rangle}
\newcommand{\norm}[1]{\left\lVert#1\right\rVert}
\newcommand{\inn}[1]{\lan#1\ran}
\newcommand{\op}[1]{\operatorname{#1}}
\newcommand{\bmat}[1]{\begin{bmatrix}#1\end{bmatrix}}
\newcommand{\pmat}[1]{\begin{pmatrix}#1\end{pmatrix}}
\newcommand{\vmat}[1]{\begin{vmatrix}#1\end{vmatrix}}

\newcommand{\amogus}{{\bigcap}\kern-0.8em\raisebox{0.3ex}{$\subset$}}
\newcommand{\Note}{\textbf{Note: }}
\newcommand{\Aside}{{\bf Aside: }}
%restriction
%\newcommand{\op}[1]{\operatorname{#1}}
%\newcommand{\done}{$$\mathcal{QED}$$}

%====================================================================%


\setlength{\parindent}{0pt}      	% No paragraph indentations
\pagestyle{fancy}
\fancyhf{}							% fancy header

\setcounter{secnumdepth}{0}			% sections are numbered but numbers do not appear
\setcounter{tocdepth}{2} 			% no subsubsections in toc

%template
%====================================================================%
%\begin{manualproblem}{1}
%Spivak.
%\end{manualproblem}

%\begin{proof}[Solution]
%\end{proof}

%----------- or -----------%

%\begin{problem} 		
%\end{problem}	

%\penum
%	\item
%\epenum
%====================================================================%


\newcommand{\Course}{PHY350}
\newcommand{\hwNumber}{2}

%preamble

\title{__TITLE__}
\author{A.N.}
\date{\today}
\lhead{\Course A\hwNumber}
\rhead{\thepage}
%\cfoot{\thepage}


%====================================================================%
\begin{document}
\begin{problem}
	Griffiths 3.44
\end{problem}
First note that $\rho=0$  everywhere except for $(0,0,z)$ for $z\in [-a,a]$. The formula for potential is given by $$V(r) = \frac{1}{4\pi \ep_0} \sum_{n=0}^\infty \frac{1}{r^{n+1}} \int (r^\prime)^n P_n \cos \theta \rho(r^\prime) d\tau^\prime. $$
Substituting $r^\prime = z\cos \theta$ , and evaluating when $\theta = 0$ and $\theta = \frac{\pi}{2}$, we see: 
\begin{align*}
	V(r) & = \frac{1}{4\pi \ep_0} \sum_{n=0}^\infty\frac{1}{r^{n+1}} \frac{Q}{2a}  \left[ \int_0^a(r^\prime)^n P_n \cos \alpha \Big|_{\theta = 0} dr^\prime + \int_0^a (r^\prime)^n P_n \cos \alpha \Big|_{\theta = \pi}dr^\prime \right]
	\\ & = \frac{1}{4\pi \ep_0} \sum_{n=0}^\infty \frac{1}{r^{n+1}} \frac{Q}{2a} \left[ \int_0^a z^n P_n \cos \theta dz + \int_0^a z^n  P_n \cos( \theta - \pi) dz \right] \tag{making the substitution}
	\\ & = \frac{1}{4\pi \ep_0} \sum_{n=0}^\infty \frac{1}{r^{n+1}} \frac{Q}{2a} \left[ P_n \cos \theta + P_n \cos( \theta -\pi) \right] \int_{0}^a z^n dz
	\\ & = \frac{Q}{4\pi \ep_0} \sum_{n=0}^\infty \frac{a^n}{r^{n+1}} P_n \cos \theta \tag{Since when n is odd $P_n \cos \theta + P_n \cos(\theta - \pi) = 0$}
\end{align*}
As Desired.
\newpage 
\begin{problem}
	Griffiths 3.47
\end{problem}
\penum
\item We compute the average electric field as follows: 
	$$E_{avg} = \frac{1}{4/3 \pi R^3} \int E(\mathfrak{r}) d\tau^\prime = \frac{1}{4/3\pi R^3} \int \frac{-q}{4\pi \ep_0} \frac{ \hat{\mathfrak{r}} }{\mathfrak{r}^2} d\tau^\prime = \frac{1}{4/3\pi R^3} \cdot \frac{-q}{4\pi \ep_0} \int \frac{1}{\mathfrak{r}^2} \hat{\mathfrak{r}} d\tau^\prime.$$
	Using formula 2.15 we compute the field of a sphere at as: 
	$$E(r) = \frac{1}{4\pi \ep_0} \int \frac{\hat{\mathfrak{r}}}{\mathfrak{r}^2} \rho(r^\prime) d\tau^\prime = \frac{1}{4\pi \ep_0} \frac{1}{4/3 \pi R^3} \int \frac{-q}{\mathfrak{r}^2} \hat{\mathfrak{r}} d\tau^\prime.$$
\item Using Gauss' Law, we compute: 
	$$\oint E \cdot da = \frac{q_{enc}}{\ep_0}\implies |E|\cdot 4\pi r^2= \frac{-q \cdot(4/3)\pi r^3}{\ep_0 \cdot 4/3 \pi R^3} \implies E = \frac{-qr}{4\pi \ep_0} \hat{r}= \frac{-P}{4\pi \ep_0}.$$
\item We can write an arbitrary charge distribution as $P = \sum q_i r_i^\prime$. Since integration is linear by superposition the formula above should hold as well. 
\item If we place a charge $q$ outside of the sphere, the average E will be : 
	$$E_{avg} = \frac{1}{4/3\pi R^3} \int E da = \frac{1}{4/3 \pi R^3} \frac{-q}{r^2} \cdot \frac{4/3\pi R^3}{4\pi \ep_0} =\frac{q}{4\pi \ep_0 r^2} \hat{r}. $$
	We can use superposition to extend to any distribution of charges. 
\epenum
\newpage
\begin{problem}
	Griffiths 3.56
\end{problem}
First note that $F_{dip}(r, \theta) = \frac{q \bf{p}}{4\pi \ep_0 r^3}(2\cos \theta \hat{r} + sin \theta \hat{\theta})$. We assume that $\phi = 0$, and $r$ is fixed. In spherical coordinates, we have that acceleration can be written as: 
$$(\ddot{r} - r\dot{\theta}^2 - r\dot{\phi} \sin^2\theta)\hat{r}+ (r \ddot{\theta} + 2\dot{r} \dot{\phi} - r \dot{\phi}^2 \sin\theta \cos \theta ) \hat{\theta}+ (r \ddot{\phi} + 2\dot{r} \dot{\phi} \cos \theta )\hat{\phi}.$$
Equating this to Force, and approximating $\sin \theta $ as $\theta$, we get that the last term vanishes, and $$\ddot{\theta} = \frac{q\cdot p}{4\pi \ep_0 r^4 m} \theta.$$
This is solved by a $\theta$ which is periodic in $t$. 

Since $r$ is fixed and $\phi$ is 0, we have that the particle traces out an arc. 
\newpage
\begin{problem}
	Griffiths 4.2
\end{problem}
We first compute the electric field due to the electron cloud. At a radius of $r$, the enclosed charge is: 
$$Q_{enc} = \int_0^r \rho d\tau = \frac{4 q}{3a^3}\int_0^r (r^\prime)^2 e^{-\frac{2r^\prime}{a}} dr^\prime = q\left(1 - e^{-\frac{2r}{a}}(1 + \frac{2r}{a} + \frac{2r^2}{a^2}) \right).$$
Therefore by Gauss' Law, we have that 
$$E_{in} =\frac{1}{4\pi \ep_0r^2} q\left( 1 - e^{-\frac{2a}{r}} (1+ \frac{2r}{a} + \frac{2r^2}{a}) \right).$$
If we apply some small electric field $E_{ext}$, the nucleus will get shifted by some small quantity $d$ so that $E_{ext} = E_{in}$. We taylor expand $e^{-\frac{2r}{a}}$ as: 
$$e^{-\frac{2r}{a}}= 1 - \frac{2d}{a} + \frac{2d^2}{a^2} - \frac{4d^3}{3a^3}+ \dots $$
And so
$$1 - e^{-\frac{2a}{d}} \left(1 + \frac{2d}{a} + \frac{2d^2}{a^2} \right) = 1 - 1 -\frac{2d}{a} -\frac{2d^2}{a^2}+\frac{2d}{a} + \frac{4d^2}{a^2} - \frac{2d^2}{a^2}  + \frac{4d^3}{3a^3}+ \dots  = \frac{4}{3} \frac{d^3}{a^3} +O \left(\frac{d^5}{a^5} \right).$$
Thus at $d$ we have $E_{in} = E_{out}$ and so 
$$E = \frac{1}{4\pi \ep_0} \cdot \frac{q}{d^2} \cdot \frac{4d^3}{3a^3} = \frac{qd}{3a^3 \pi \ep_0}= \frac{P}{3a^3 \pi \ep_0}.$$
Therefore $\alpha = 3a^3 \pi \ep_0$. 
\newpage
\begin{problem}
	Griffiths 4.4
\end{problem}
We have that the electric field of $q$ at distance $r$ is 
$$|E| = \frac{q}{4\pi \ep_0 r^2}.$$
We can place the charge on the $x$ axis so that $\vec{p} = \alpha E = \frac{\alpha q}{4\pi \ep_0 r^2}\hat{x}.$
We know that $E_{dip}(r) = \frac{1}{4\pi \ep_0 r^3} \left[3(p \cdot \hat{r} )- p \right]$. Since $\hat{p}= \hat{x}$, we have that $$3 p\cdot \hat{r} - p = \frac{3\alpha q}{4\pi \ep_0 r^3}\hat{x} - \frac{\alpha q}{4\pi \ep_0r^3}\hat{x} = \frac{2\alpha q}{4\pi \ep_0 r^3} \hat{x}.$$
Therefore $F = qE = \frac{2\alpha q^2}{16\pi^2 \ep_0^2 r^3} \hat{x}$. 
\newpage
\begin{problem}
	Griffiths 4.6
\end{problem}
We first place an dipole at $-z$. The boundary condtions are the same, so we can instead compute the torque this dipole exerts on $p$. 
We have that $E_{dip} (r, \theta)= \frac{p}{4\pi \ep_0 r^3} ( 2\cos \theta \hat{r} + \sin \theta \hat{\theta})$. 
At a distance of $2z$ this becomes: 
$$E_{dip} = \frac{p}{4\pi \ep_0 (2z)^3}(2 \cos \theta \hat{r} + \sin \theta \hat{\theta}). $$
We can also write the dipole as $p = p \cos \theta \hat{r} + p \sin \theta \hat{\theta}$. 
We compute the torque as $$N = p \times E = \frac{p^2}{4\pi \ep_0 (2z)^3}\left[(\cos \theta \hat{r} + \sin \theta \hat{\theta}) \times (2 \cos \theta \hat{r} + \sin \theta \hat{\theta}) \right] = \frac{-p^2\sin \theta \cos \theta}{4\pi \ep_0 (2z^3)}\hat{\phi}.$$
If we allow the dipole to rotate, it will come to a rest when the torque is at a minimum. so at $\theta = \frac{\pi}{4}, \frac{5\pi}{4}$. 
\end{document}
