\documentclass[12pt, a4paper]{article}
\usepackage[lmargin =0.5 in, 
rmargin=0.5in, 
tmargin=1in,
bmargin=0.5in]{geometry}
\geometry{letterpaper}
\usepackage{tikz-cd}
\usepackage{amsmath}
\usepackage{amssymb}
\usepackage{blindtext}
\usepackage{titlesec}
\usepackage{enumitem}
\usepackage{fancyhdr}
\usepackage{amsthm}
\usepackage{graphicx}
\usepackage{cool}
\usepackage{thmtools}
\usepackage{hyperref}
\graphicspath{ }					%path to an image

%-------- sexy font ------------%
%\usepackage{libertine}
%\usepackage{libertinust1math}

%\usepackage{mlmodern}				% very nice and classic
%\usepackage[utopia]{mathdesign}
%\usepackage[T1]{fontenc}


\usepackage{mlmodern}
\usepackage{eulervm}
%\usepackage{tgtermes} 				%times new roman
%-------- sexy font ------------%


% Problem Styles
%====================================================================%


\newtheorem{problem}{Problem}


\theoremstyle{definition}
\newtheorem{thm}{Theorem}
\newtheorem{lemma}{Lemma}
\newtheorem{prop}{Proposition}
\newtheorem{cor}{Corollary}
\newtheorem{fact}{Fact}
\newtheorem{defn}{Definition}
\newtheorem{example}{Example}
\newtheorem{question}{Question}

\newtheorem{manualprobleminner}{Problem}

\newenvironment{manualproblem}[1]{%
	\renewcommand\themanualprobleminner{#1}%
	\manualprobleminner
}{\endmanualprobleminner}

\newcommand{\penum}{ \begin{enumerate}[label=\bf(\alph*), leftmargin=0pt]}
	\newcommand{\epenum}{ \end{enumerate} }

% Math fonts shortcuts
%====================================================================%

\newcommand{\ring}{\mathcal{R}}
\newcommand{\N}{\mathbb{N}}                           % Natural numbers
\newcommand{\Z}{\mathbb{Z}}                           % Integers
\newcommand{\R}{\mathbb{R}}                           % Real numbers
\newcommand{\C}{\mathbb{C}}                           % Complex numbers
\newcommand{\F}{\mathbb{F}}                           % Arbitrary field
\newcommand{\Q}{\mathbb{Q}}                           % Arbitrary field
\newcommand{\PP}{\mathcal{P}}                         % Partition
\newcommand{\M}{\mathcal{M}}                         % Mathcal M
\newcommand{\eL}{\mathcal{L}}                         % Mathcal L
\newcommand{\T}{\mathcal{T}}                         % Mathcal T
\newcommand{\U}{\mathcal{U}}                         % Mathcal U\\
\newcommand{\V}{\mathcal{V}}                         % Mathcal V

% symbol shortcuts
%====================================================================%

\newcommand{\bd}{\partial}
\newcommand{\grad}{\nabla}
\newcommand{\lam}{\lambda}
\newcommand{\imp}{\implies}
\newcommand{\all}{\forall}
\newcommand{\exs}{\exists}
\newcommand{\delt}{\delta}
\newcommand{\ep}{\varepsilon}
\newcommand{\ra}{\rightarrow}
\newcommand{\vph}{\varphi}

\newcommand{\ol}{\overline}
\newcommand{\f}{\frac}
\newcommand{\lf}{\lfrac}
\newcommand{\df}{\dfrac}

% bracketting shortcuts
%====================================================================%
\newcommand{\abs}[1]{\left| #1 \right|}
\newcommand{\babs}[1]{\Big|#1\Big|}
\newcommand{\bound}{\Big|}
\newcommand{\BB}[1]{\left(#1\right)}
\newcommand{\dd}{\mathrm{d}}
\newcommand{\artanh}{\mathrm{artanh}}
\newcommand{\Med}{\mathrm{Med}}
\newcommand{\Cov}{\mathrm{Cov}}
\newcommand{\Corr}{\mathrm{Corr}}
\newcommand{\tr}{\mathrm{tr}}
\newcommand{\Range}[1]{\mathrm{range}(#1)}
\newcommand{\Null}[1]{\mathrm{null}(#1)}
\newcommand{\lan}{\langle}
\newcommand{\ran}{\rangle}
\newcommand{\norm}[1]{\left\lVert#1\right\rVert}
\newcommand{\inn}[1]{\lan#1\ran}
\newcommand{\op}[1]{\operatorname{#1}}
\newcommand{\bmat}[1]{\begin{bmatrix}#1\end{bmatrix}}
\newcommand{\pmat}[1]{\begin{pmatrix}#1\end{pmatrix}}
\newcommand{\vmat}[1]{\begin{vmatrix}#1\end{vmatrix}}

\newcommand{\amogus}{{\bigcap}\kern-0.8em\raisebox{0.3ex}{$\subset$}}
\newcommand{\Note}{\textbf{Note: }}
\newcommand{\Aside}{{\bf Aside: }}
%restriction
%\newcommand{\op}[1]{\operatorname{#1}}
%\newcommand{\done}{$$\mathcal{QED}$$}

%====================================================================%


\setlength{\parindent}{0pt}      	% No paragraph indentations
\pagestyle{fancy}
\fancyhf{}							% fancy header

\setcounter{secnumdepth}{0}			% sections are numbered but numbers do not appear
\setcounter{tocdepth}{2} 			% no subsubsections in toc

%template
%====================================================================%
%\begin{manualproblem}{1}
%Spivak.
%\end{manualproblem}

%\begin{proof}[Solution]
%\end{proof}

%----------- or -----------%

%\begin{problem} 		
%\end{problem}	

%\penum
%	\item
%\epenum
%====================================================================%


\newcommand{\Course}{PHY350}
\newcommand{\hwNumber}{1}

%preamble

\title{__TITLE__}
\author{A.N.}
\date{\today}
\lhead{\Course A\hwNumber}
\rhead{\thepage}
%\cfoot{\thepage}


%====================================================================%
\begin{document}

\begin{problem}
	Griffiths 3.9
\end{problem}
Since the sphere has potential $0$, we only need to add one point charge at the center that will make the sphere have a potential of $V_0$. A point charge placed at the center of the sphere will create a potential of $V(r) = \frac{1}{4\pi\ep_0} \cdot \frac{q}{r}$, so for 
our sphere to have a potential of $V_0$ we can take $q^{\prime \prime} = 4\pi \ep_0 V_0 R$. 
In a neutral sphere, where $q^\prime + q^{\prime \prime}=0 $, so the force acting on $q$ will be 
\begin{align*}
	F(a)& = \frac{1}{4\pi \ep_0} q \left[ \frac{q^{\prime \prime}}{a^2} + \frac{q^\prime}{(a-b)^2} \right]
	\\ & = \frac{qq^\prime}{4\pi \ep_0} \left[ -\frac{1}{a^2} + \frac{1}{(a-b)^2}  \right]
	\\ & = \frac{qq^\prime}{4\pi \ep_0} \left[ \frac{b(b-2a)}{a^2(a-b)^2} \right]
	\\ & = \frac{q^2}{4\pi \ep_0} \left(- \frac{R}{a} \right)^3  \left[ \frac{R^2-2a^2}{(a^2-R^2)^2} \right]
\end{align*}
\newpage
\begin{problem}
	Griffiths 3.12
\end{problem}
Recall that the potential function of two wires with charge per length $\lambda, -\lambda$ at positions $a,-a$ is:
$$V_{total} (x,y,z) =  \frac{\lambda}{4\pi \ep_0}  \log  \left( \frac{z^2 + (y+a)^2}{z^2+(y-a)^2}  \right). $$
We wish to solve for the parameter $a$ so that there will be an equipotential of $V_0$ and $-V_0$ at
$(0,R+d,0)$ and $ (0,R-d,0)$. 
We see that
$$e^{\frac{V_04\pi \ep_0}{\lambda} } = \frac{(R+d+a)^2 }{(R+d-a)^2}, e^{\frac{-V_0 4\pi\ep_0}{\lambda }}  = \frac{(R-d+a)^2}{(R-d-a)^2} \implies \frac{(R+d+a)^2 }{(R+d-a)^2} = \frac{(R-d+a)^2}{(R-d-a)^2} \implies a=\sqrt{d^2-R^2}.   $$
We must therefore place the wires at $a = \pm \sqrt{d^2-R^2}$. We now solve for $\lambda$. 
We have that $$\lambda = 4V_0\pi\ep_0 log \left( \frac{(R+d + \sqrt{d^2-R^2})^2 }{(R-d-\sqrt{d^2-R^2})^2}  \right)$$
Thus we are done by uniqueness. 
\newpage
\begin{problem}
	Griffiths 3.13
\end{problem}
We wish to solve for the potential with the given boundary conditions: 
$$\begin{cases}
	1)V = 0 &  y=0\\
	2) V = 0 & y=a\\
	3) V = V_0 &  x=0, y\in [0, \frac{a}{2}]\\
	4) V = -V_0 & x=0 , y\in (\frac{a}{2}, a ]\\
	5) V \to 0 & x\to \infty
\end{cases}$$
We write the potential $V(x,y,z) = X(x)Y(y)$, since this function must be independant of $z$. 
By a similar reasoning as in Griffiths, we must have that $$X(x) = Ae^{kx} + Be^{-kx}, Y(y) = C\sin(kx) + D\cos(ky).$$
Condition $5$ implies that $A = 0$, and condition $1$ implies that $D =0$. Thus we can write $$V(x,y,z) = e^{-kx} C\sin(ky) = e^{-\frac{n\pi x}{a}} \sin \left(\frac{n\pi y}{a} \right).$$
Writing this as an infinite sum, we have $$V(x,y,z) = \sum_{n=0}^\infty A_n e^{-\frac{n\pi x}{a}} \sin \left(\frac{n\pi y}{a} \right).$$
We now compute the coefficients on the parts with potential $V_0, -V_0$ respectively. 
\begin{align*}
	C_n & = \frac{2}{a} \int_{0}^{\frac{a}{2}} V_0 \sin \left(\frac{n\pi y}{a}\right) dy
	\\ & = - \frac{2}{a} \cdot \frac{aV_0}{n\pi } \left[ \cos \left( \frac{n\pi y}{a} \right) \right] \Bigg|_{0}^{\frac{a}{2}}
	\\ & = \begin{cases}
		0 & n \equiv 0 \mod(4)\\
		\frac{2V_0}{n\pi} & n \equiv 1,3 \mod(4)\\
		\frac{4V_0}{n\pi} & n\equiv 2 \mod(4)
	\end{cases}
\end{align*}
Similarly,
\begin{align*}
	C_n^\prime & = \frac{2}{a} \int_{\frac{a}{2}}^a -V_0 \sin \left( \frac{n\pi y}{a}\right) dy
	\\ & = \begin{cases}
		0 & n \equiv 0 \mod(4)\\
		-\frac{2V_0}{n\pi} & n\equiv 1,3 \mod(4)\\
		\frac{4V_0}{n\pi} & n\equiv 2 \mod(4)
	\end{cases}
\end{align*}
Summing these together, we get that 
$$A_n = \begin{cases} 0 & n\equiv 0,1,3 \mod(4) \\ \frac{8V_0}{\pi n} & n \equiv 2 \mod(4)  \end{cases}.$$
Taking these coefficients gives us the desired potential function. 
\newpage
\begin{problem}
Griffiths 3.20
\end{problem}
First note that on the boundary, the potential function must satisfy $$V_0(R,\theta) = \sum_{l=0}^\infty A_l r^l P_l( \cos \theta) = \sum_{l}^\infty \frac{B_l}{r^{l+1}} P_l(\cos \theta). $$
We know from Griffiths, that $$A_l = \frac{2l+1}{2} \int_0^\pi V_0(\theta) \sin \theta P_l(\cos\theta) d\theta .$$
We now compute the electric field on the inside and outside, at $r=R$ and use the fact that $E_{out} - E_{in} = \frac{1}{\ep_0} \sigma$. 
We see that $$E_{out}(R) = - \grad V|_{r=R} = \sum_{l=0}^\infty \frac{B_l (l+1)}{R^{l+2}} P_l(\cos \theta),$$
and 
$$E_{in}(R) = -\sum_{l=0}^\infty l r^{l-1} A_l P_l(\cos \theta).$$
Computing their difference yields: 
\begin{align*} \frac{1}{\ep_0} \sigma(\theta)& =  E_{out} - E_{in}
	\\ &= \sum_{l=0}^\infty \left(B_l \cdot\frac{l+1}{R^{l+2}}  + l R^{l-1} A_l \right) P_l(\cos \theta)
	\\ &  = \sum_{l=0}^\infty \frac{A_l}{R}(2l+1) P_l(\cos \theta)
	\\ & = \frac{1}{2R}  \sum_{l=0}^\infty (2l+1)^2 \left[\int_0^\pi V_0(\theta) P_l(\cos \theta) \sin \theta d\theta \right] P_l(\cos \theta). 
\end{align*}
\newpage
\begin{problem}
	Griffiths 3.25
\end{problem}
The solution to the Laplace equation on cylidrical coordinates is 
$$V(s,\phi) = A_0+  B_0 \log(s) + \sum_{k=1}^\infty \left( A_k s^k + B_k s^{-k} \right) \left(C_k \cos k \phi + D_k \sin k\phi \right), $$
with the conditions $V(R, \phi) = 0,V(s, \phi) \to -E_0s \cos \phi $ as $s\to \infty$. 
We must have that $A_k=B_k =0$ for all $k \neq 1$, since the potential must converge to the electric field. So we can write $$V(s, \phi) = a_1s \cos \phi + \frac{a_2}{s} \cos \phi . $$
The condition $V(R, \phi) = 0$ implies that 
$$0 = E_0a_1 R + \frac{a_2}{R} = 0 \implies a_2 = -a_1 R^2$$
Using the limit at infinity condition, we compute that 
$$\lim_{s\to \infty} \frac{\partial V}{\partial s}(s,0) =  a_1 + \frac{a_1}{s^2} = -E_0 \implies a_1 =-E_0.  $$
Therefore the potential function is $V(s, \phi ) = -E_0 \left(s - \frac{R^2}{s} \right) \cos \phi $. We now compute the induced surface charge. 
We have that the potenial must vanish on the inside, since the boundary has $0$ potential, and there is no enclosed charge. Therefore using the formula for surface charge, we compute that 
$$ \frac{1}{\ep_0} \sigma(\theta) =\left(\frac{\partial V_{out}}{\partial s }  \right)\Big|_{s=R} = -E_0 -E_0\frac{R^2}{R^2} \cos \phi = -2E_0 \cos \phi . $$
\newpage
\begin{problem}
Griffiths 3.41
\end{problem}
Using the result of Griffiths 3.9, we can write the force as 
$$F(a) = \frac{q}{4\pi \ep_0} \left[\frac{q^{\prime \prime}}{a^2} + \frac{q^\prime}{(a-b)^2} \right]. $$
We impose the condition that $q^{\prime \prime} + q^\prime = q$, since we want the charge of the sphere to be $q$. 
Using this we can compute that 
\begin{align*}
	F(a) & = \frac{q}{4\pi \ep_0} \left[\frac{q-q^\prime}{a^2} + \frac{q^\prime}{(a-b^2}  \right]
	\\ & = \frac{q}{4\pi \ep_0} \left[ \frac{q}{a^2} - \frac{q^\prime}{a^2} + \frac{q^\prime}{(a-b)^2}  \right]
	\\ & = \frac{q^2}{4\pi \ep_0} \cdot \frac{1}{a^2} + \frac{qq^\prime}{4\pi \ep_0} \cdot \frac{1}{a^2}  \left[\frac{2a-b}{(a-b)^2}  \right]
	\\ & = \frac{q^2}{4\pi \ep_0 a^3} \left[a - \frac{R^3(2a^2- R^2)}{(a^2-R^2)^2} \right]
	\tag{using $q^\prime = q$}
\end{align*}
The potential will be $0$ exactly when $a - \left[ \frac{R^3(2a^2-R^2)}{(a^2-R^2)^2}  \right] = 0$, or when $a(a^2-R^2)^2 = R^3(2a^2-R^2)$. Using wolfram alpha, this has a solution exactly when $a = \varphi R$, where $\varphi$ is the golden ratio. Approximately, we have that $a \approx  5.66\AA $. 
The work to take a particle from $\infty$ to $r$ is computed as the following integral: 
$$W = \frac{q^2}{4\pi \ep_0}  \int_\infty^r \frac{1}{a^3} \left[a - \frac{R^3(2a^2-R^2)}{(a^2-R^2)^2} \right] = \frac{q^2}{8\pi \ep_0R} da,$$
According to wolfram alpha. 
We can explicitly compute the work as $$W = \frac{q^2}{8\pi \ep_0R}= \frac{(1.6012\times 10^{-19})^2}{8\pi 3.5\AA * 8.85\times10^{-12}} = 2.03\times 10^{-10} J$$
\end{document}
